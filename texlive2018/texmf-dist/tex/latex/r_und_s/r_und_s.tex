%
% Herein I define some safety advices which I had to include with my
% lab protocols in organic chemistry.
% The latest version will be made available at
% http://projects.thiemo.net/RundS/
%
%
% Copyright (c) 1997-2005 Thiemo Nordenholz <nz@thiemo.net>
% All rights reserved.
% $Id: r_und_s.tex,v 1.4 2005/12/01 12:21:21 thiemo Exp $
%
% Redistribution and use in source and binary forms, with or without
% modification, are permitted provided that the following conditions
% are met:
% 1. Redistributions of source code must retain the above copyright
%    notice, this list of conditions and the following disclaimer.
% 2. Redistributions in binary form must reproduce the above copyright
%    notice, this list of conditions and the following disclaimer in the
%    documentation and/or other materials provided with the distribution.
%
% THIS SOFTWARE IS PROVIDED BY THE AUTHOR AND CONTRIBUTORS ``AS IS'' AND
% ANY EXPRESS OR IMPLIED WARRANTIES, INCLUDING, BUT NOT LIMITED TO, THE
% IMPLIED WARRANTIES OF MERCHANTABILITY AND FITNESS FOR A PARTICULAR PURPOSE
% ARE DISCLAIMED.  IN NO EVENT SHALL THE AUTHOR OR CONTRIBUTORS BE LIABLE
% FOR ANY DIRECT, INDIRECT, INCIDENTAL, SPECIAL, EXEMPLARY, OR CONSEQUENTIAL
% DAMAGES (INCLUDING, BUT NOT LIMITED TO, PROCUREMENT OF SUBSTITUTE GOODS
% OR SERVICES; LOSS OF USE, DATA, OR PROFITS; OR BUSINESS INTERRUPTION)
% HOWEVER CAUSED AND ON ANY THEORY OF LIABILITY, WHETHER IN CONTRACT, STRICT
% LIABILITY, OR TORT (INCLUDING NEGLIGENCE OR OTHERWISE) ARISING IN ANY WAY
% OUT OF THE USE OF THIS SOFTWARE, EVEN IF ADVISED OF THE POSSIBILITY OF
% SUCH DAMAGE.
%
%

\message{Package 'r_und_s', 01.12.2005 (v1.3i), nz@thiemo.net}
% translated into TeX by eike.kunst@tu-clausthal.de (13.05.1997)

% numbers are coded o, e, z, d, v, f, s, i, a, n
% combinations are prefixed with an extra-'c'.
\catcode`\"=13 \def"#1{\if#1s{\ss}\else\accent'177 #1\fi}
% define some 'r-saetze'
% R1-9
\def\cre{R1 In trockenem Zustand explosionsgef"ahrlich.}
\def\crz{R2 Durch Schlag, Reibung, Feuer oder andere Z"undquellen 
explosionsgef"ahrlich.}
\def\crd{R3 Durch Schlag, Reibung, Feuer oder andere Z"undquellen
besonders explosionsgef"ahrlich.}
\def\crv{R4 Bildet hochempfindliche explosionsgef"ahrliche
Metallverbindungen.}
\def\crf{R5 Beim Erw"armen explosionsf"ahig.}
\def\crs{R6 Mit und ohne Luft explosionsf"ahig.}
\def\cri{R7 Kann Brand verursachen.}
\def\cra{R8 Feuergefahr bei Ber"uhrung mit brennbaren Stoffen.}
\def\crn{R9 Explosionsgefahr bei Mischung mit brennbaren Stoffen.}
% R10-19
\def\creo{R10 Entz"undlich.}
\def\cree{R11 Leichtentz"undlich.}
\def\crez{R12 Hochentz"undlich.}
\def\cred{R13 Hochentz"undliches Fl"ussiggas.}
\def\crev{R14 Reagiert heftig mit Wasser.}
\def\cref{R15 Reagiert mit Wasser unter Bildung leicht
entz"undlicher Gase.}
\def\cres{R16 Explosionsgef"ahrlich in Mischung mit
brandf"ordernden Stoffen}
\def\crei{R17 Selbstentz"undlich an der Luft.}
\def\crea{R18 Bei Gebrauch Bildung
explosionsf"ahiger/Leichtentz"undlicher Dampf-Luftgemische m"oglich.}
\def\cren{R19 Kann explosionsf"ahige Peroxide bilden.}
% R20-29
\def\crzo{R20 Gesundheitssch"adlich beim Einatmen.}
\def\crze{R21 Gesundheitssch"adlich bei Ber"uhrung mit der Haut.}
\def\crzz{R22 Gesundheitssch"adlich beim Verschlucken.}
\def\crzd{R23 Giftig beim Einatmen.}
\def\crzv{R24 Giftig bei Ber"uhrung mit der Haut.}
\def\crzf{R25 Giftig beim Verschlucken.}
\def\crzs{R26 Sehr giftig beim Einatmen.}
\def\crzi{R27 Sehr giftig bei Ber"uhrung mit der Haut.}
\def\crza{R28 Sehr giftig beim Verschlucken.}
\def\crzn{R29 Entwickelt mit Wasser giftige Gase.}
% R30-39
\def\crdo{R30 Kann bei Gebrauch leicht entz"undlich wirken.}
\def\crde{R31 Entwickelt bei Ber"uhrung mit S"aure giftige Gase.}
\def\crdz{R32 Entwickelt bei Ber"uhrung mit S"aure sehr giftige
Gase.}
\def\crdd{R33 Gefahr kumulativer Wirkungen.}
\def\crdv{R34 Verursacht Ver"atzungen}
\def\crdf{R35 Verursacht schwere Ver"atzungen.}
\def\crds{R36 Reizt die Augen.}
\def\crdi{R37 Reizt die Atmungsorgane.}
\def\crda{R38 Reizt die Haut.}
\def\crdn{R39 Ernste Gefahr irreversiblen Schadens.}
% R40-48
\def\crvo{R40 Irreversibler Schaden m"oglich.}
\def\crve{R41 Gefahr ernster Augensch"aden.}
\def\crvz{R42 Sensibilisierung durch Einatmen m"oglich.}
\def\crvd{R43 Sensibilisierung durch Hautkontakt m"oglich.}
\def\crvv{R44 Explosionsgefahr beim Erhitzen unter Einschlu"s.}
\def\crvf{R45 Kann Krebs erzeugen.}
\def\crvs{R46 Kann vererbbare Sch"aden verursachen.}
\def\crvi{R47 Kann Mi"sbildungen verursachen.}
\def\crva{R48 Gefahr ernster Gesundheitssch"aden bei l"angerer
Exposition.}
\def\crvn{R49 Kann Krebs erzeugen beim Einatmen.}
\def\crfo{R50 Sehr giftig f"ur Waaserorganismen.}
\def\crfe{R51 Giftig f"ur Wasserorganismen.}
\def\crfz{R52 Sch"adlich f"ur Wasserorganismen.}
\def\crfd{R53 Kann in Gew"assern l"angerfristig sch"adliche
Wirkungen haben.}
\def\crcrfv{R54 Giftig f"ur Pflanzen.}
\def\crff{R55 Giftig f"ur Tiere.}
\def\crfs{R56 Giftig f"ur Bodenorganismen.}
\def\crfi{R57 Giftig f"ur Bienen.}
\def\crfa{R58 Kann l"angerfristig sch"adliche Wirkungen auf die
Umwelt haben.}
\def\crfn{R59 Gef"ahrlich f"ur die Ozonschicht.}
\def\crso{R60 Kann die Fortpflanzungsf"ahigkeit beeintr"achtigen.}
\def\crse{R61 Kann das Kind im Mutterleib sch"adigen.}
\def\crsz{R62 Kann m"oglicherweise die Fortpflanzungsf"ahigkeit
beeintr"achtigen}
\def\crsd{R63 Kann das Kind im Mutterleib m"oglicherweise sch"adigen.}
\def\crsv{R64 Kann S"auglinge "uber die Muttermilch sch"adigen.}

% Kombinationen der R-Saetze
\def\crcevef{R14/15 Reagiert heftig mit Wasser unter Bildung leicht
entz"undlicher Gase.}
\def\crcefzn{R15/29 Reagiert mit Wasser unter Bildung giftiger und
leicht entz"undlicher Gase.}
\def\crczoze{R20/21 Gesundheitssch"adlich beim Einatmen und bei
Ber"uhrung mit der Haut.}
\def\crczozz{R20/22 Gesundheitssch"adlich beim Einatmen und Verschlucken.}
\def\crczozezz{R20/21/22 Gesundheitssch"adlich beim Einatmen,
Verschlucken und Ber"uhrung mit der Haut.}
\def\crczdzv{R23/24 Giftig beim Einatmen und bei Ber"uhrung mit der
Haut.}
\def\crczvzf{R24/25 Giftig bei Ber"uhrung mit der Haut und beim
Verschlucken.}
\def\crczdzf{R23/25 Giftig beim Einatmen und Verschlucken.}
\def\crczdzvzf{R23/24/25 Giftig beim Einatmen, Verschlucken und
Ber"uhrung mit der Haut.}
\def\crczszi{R26/27 Sehr giftig beim Einatmen und bei Ber"uhrung
mit der Haut.}
\def\crcziza{R27/28 Sehr giftig bei Ber"uhrung mit der Haut und beim
Verschlucken.}
\def\crczsza{R26/28 Sehr giftig beim Einatmen und Verschlucken.}
\def\crczsziza{R26/27/28 Sehr giftig beim Einatmen, Verschlucken
und Ber"uhrung mit der Haut.}
\def\crcdsdi{R36/37 Reizt die Augen und die Atmungsorgane.}
\def\crcdida{R37/38 Reizt die Atmungsorgane und die Haut.}
\def\crcdsda{R36/38 Reizt die Augen und die Haut.}
\def\crcdsdida{R36/37/38 Reizt die Augen, Atmungsorgane und die Haut.}
\def\crcvzvd{R42/43 Sensibilisierung durch Einatmen und Hautkontakt
m"oglich.}
\def\crcvazozz{R48/20/22 Gesundheitssch"adlich: Gefahr ernster
Gesundheitssch"aden bei l"angerer Exposition durch Einatmen, Ber"uhrung mit
der Haut und durch Verschlucken.}
\def\crcfofd{R50/53 Sehr giftig f"ur Wasserorganismen, kann in
Gew"assern l"angerfristig sch"adliche Wirkungen haben.}
\def\crcfefd{R51/53  Giftig f"ur Wasserorganismen, kann in
Gew"assern l"angerfristig sch"adliche Wirkungen haben.}
\def\crcfzfd{R52/53 Sch"adlich f"ur Wasserorganismen, kann in
Gew"assern l"angerfristig sch"adliche Wirkungen haben.}

% also some 's-saetze'
% S1-9
\def\cse{S1 Unter Verschlu"s aufbewahren.}
\def\csz{S2 Darf nicht in die H"ande von Kindern gelangen.}
\def\csd{S3 K"uhl aufbewahren.}
\def\csv{S4 Von Wohnpl"atzen fernhalten.}
\def\csf#1{S5 Unter #1 aufbewahren.}
\def\css#1{S6 Unter #1 aufbewahren.}
\def\csi{S7 Beh"alter dicht geschlossen halten.}
\def\csa{S8 Beh"alter trocken halten.}
\def\csn{S9 Beh"alter an einem gut gel"ufteten Ort aufbewahren.}
% S10-19
\def\csez{S12 Beh"alter nicht gasdicht verschlie"sen.}
\def\csed{S13 Von Nahrungsmitteln, Getr"anken und Futtermitteln
fernhalten.}
\def\csev#1{S14 Von #1 fernhalten.}
\def\csef{S15 Vor Hitze sch"utzen.}
\def\cses{S16 Von Z"undquellen fernhalten.}
\def\csei{S17 Von brennbaren Stoffen fernhalten.}
\def\csea{S18 Beh"alter mit Vorsicht "offnen und handhaben.}
% S20-29
\def\cszo{S20 Bei der Arbeit nicht essen und trinken.}
\def\csze{S21 Bei der Arbeit nicht rauchen.}
\def\cszz{S22 Staub nicht einatmen.}
\def\cszd{S23 Gas/Rauch/Dampf/Aerosol nicht einatmen.}
\def\cszv{S24 Ber"uhrung mit der Haut vermeiden.}
\def\cszf{S25 Ber"uhrung mit den Augen vermeiden.}
\def\cszs{S26 Bei Ber"uhrung mit den Augen gr"undlich mit Wasser
absp"ulen und Arzt konsultieren.}
\def\cszi{S27 Beschmutzte, getr"ankte Kleidung sofort ausziehen.}
\def\csza#1{S28 Bei Ber"uhrung mit der Haut sofort abwaschen mit viel
#1 .}
\def\cszn{S29 Nicht in die Kanalisation gelangen lassen.}
% S30-39
\def\csdo{S30 Niemals Wasser hinzugie"sen.}
\def\csdd{S33 Ma"snahmen gegen elektrostatische Aufladungen
treffen.}
\def\csdv{S34 Schlag und Reibung vermeiden.}
\def\csdf{S35 Abf"alle und Beh"alter m"ussen in gesicherter Weise
beseitigt werden.}
\def\csds{S36 Bei der Arbeit geeignete Schutzkleidung tragen.}
\def\csdi{S37 Geeignete Schutzhandschuhe tragen.}
\def\csda{S38 Bei unzureichender Bel"uftung Atemschutzger"at
anlegen.}
\def\csdn{S39 Schutzbrille/Gesichtsschutz tragen.}
% S40-49
\def\csvo#1{S40 Fu"sboden und verunreinigte Gegenst"ande mit #1
reinigen.}
\def\csve{S41 Explosions- und Brandgase nicht einatmen.}
\def\csvz{S42 Beim R"auchern/Verspr"uhen geeignetes
Atemschutzger"at anlegen.}
\def\csvd#1{S43 Zum L"oschen #1 verwenden.}
\def\csvv{S44 Bei Unwohlsein "arztlichen Rat einholen (wenn
m"oglich, Etikett vorzeigen).}
\def\csvf{S45 Bei Unfall oder Unwohlsein sofort Arzt zuziehen (wenn
m"oglich, Etikett vorzeigen).}
\def\csvs{S46 Bei Verschlucken sofort "arztlichen Rat einholen und
Verpackung oder Etikett vorzeigen.}
\def\csvi#1{S47 Nicht bei Temperaturen "uber #1$^\mathrm{o}$C
aufbewahren.}
\def\csva#1{S48 Feucht halten mit #1 .}
\def\csvn{S49 Nur im Originlbeh"alter aufbewahren.}
% S50-59
\def\csfo#1{S50 Nicht mischen mit #1 .}
\def\csfe{S51 Nur in gut gel"ufteten Bereichen verwenden.}
\def\csfz{S52 Nicht gro"sfl"achig f"ur Wohn- und Aufenthaltsr"aume
zu verwenden.}
\def\csfd{S53 Exposition vermeiden - vor Gebrauch besondere
Anweisung einholen.}
\def\csfv{S54 Vor Ableitung in Kl"aranlagen Einwilligung der zust"andigen
Beh"orden einholen.}
\def\csff{S55 Vor Ableitung in die Kanalisation oder in Gew"asser nach dem
Stand der Technik behandeln.}
\def\csfs{S56 Diesen Stoff und seinen Beh"alter auf entsprechend
genehmigter Sonderm"ulldeponie entsorgen.}
\def\csfi{S57 Zur Vermeidung einer Kontamination der Umwelt
geeigneten Beh"alter verwenden.}
\def\csfa{S58 Als gef"ahrlichen Abfall entsorgen.}
\def\csfn{S59 Informationen zur Wiederverwendung / Wiederverwertung
beim Hersteller / Lieferanten erfragen.}
\def\csso{S60 Dieser Stoff und sein Beh"alter sind als
gef"ahrlicher Abfall zu entsorgen.}
\def\csse{S61 Freisetzung in die Umwelt vermeiden. Besondere
Anweisungen einholen / Sicherheitsdatenblatt zu Rate ziehen.}
\def\cssz{S62 Bei Verschlucken kein Erbrechen herbeif"uhren. Sofort
"arztlichen Rat einholen und Verpackung oder dieses Etikett vorzeigen.}

% Kombinationen der S-Saetze.
\def\cscez{S1/2 Unter Verschlu"s und f"ur Kinder unzug"anglich
aufbewahren.}
\def\cscdsn{S3/7/9 Beh"alter dicht geschlossen halten und an einem
k"uhlen, gut gel"ufteten Ort aufbewahren.}
\def\cscdn{S3/9 Beh"alter an einem k"uhlen, gut gel"ufteten ort
aufbewahren.}
\def\cscdev#1{S3/14 An einem k"uhlen Ort entfernt von #1
aufbewahren.}
\def\cscdnev#1{S3/9/14 An einem k"uhlen, gut gel"ufteten Ort,
entfernt von #1 aufbewahren.}
\def\cscdnvn{S3/9/49 Nur im Originalbeh"alter an einem k"uhlen, gut
gel"ufteten Ort aufbewahren.}
\def\cscdnevvn#1{S3/9/14/49 Nur im Originalbeh"alter an einem
k"uhlen, gut gel"ufteten Ort, entfernt von #1 aufbewahren.}
\def\cscia{S7/8 Beh"alter trocken und dicht geschlossen halten.}
\def\cscin{S7/9 Beh"alter dicht geschlossen an einem gut gel"ufteten
Ort aufbewahren.}
\def\csczoze{S20/21 Bei der Arbeit nicht essen, trinken, rauchen.}
\def\csczvzf{S24/25 Ber"uhrung mit den Augen und der Haut
vermeiden.}
\def\cscdsdi{S36/37 Bei der Arbeit geeignete Schutzhandschuhe und
Schutzkleidung tragen.}
\def\cscdsdn{S36/39 Bei der Arbeit geeignete Schutzkleidung und
Schutzbrille/Gesichtsschutz tragen.}
\def\cscdidn{S37/39 Bei der Arbeit geeignete Schutzhandschuhe und
Schutzbrille/Gesichtsschutz tragen.}
\def\cscdsdidn{S37/38/39 Bei der Arbeit geeignete Schutzkleidung,
Schutzhandschuhe und Schutzbrille/Gesichtsschutz tragen.}
\def\cscvivn#1{S47/49 Nur im Originalbeh"alter bei einer
Temperatur von nicht "uber #1$^\mathrm{o}$C aufbewahren.}

% -- Pooh! got it :)
