%% $Id: pst-spirograph.tex 80 2014-08-23 05:50:14Z herbert $
%%
%% This is file `pst-spirograph.tex',
%%
%% IMPORTANT NOTICE:
%%
%% Package `pst-spirograph.tex'
%%
%% COPYRIGHT 2015 by 
%%     Manuel Luque <manuel.luque27@gmail.com>
%%     Herbert Voss <hvoss@tug.org>
%%
%% This program can be redistributed and/or modified under the terms
%% of the LaTeX Project Public License Distributed from CTAN archives
%% in directory CTAN:/macros/latex/base/lppl.txt.
%%
%% DESCRIPTION:
%%   `pst-spirograph' is a PSTricks package to show spirograph curves
%%
%%
\csname PSTSPIROGRAPHLoaded\endcsname
\let\PSTSPIROGRAPHLoaded\endinput
% Requires some packages
\ifx\PSTricksLoaded\endinput\else \input pstricks \fi
\ifx\PSTXKeyLoaded\endinput\else  \input pst-xkey \fi
\def\fileversion{0.41}
\def\filedate{2014/08/23}
\message{`PSTSPIROGRAPH' v\fileversion, \filedate\ (ml,hv)}

\edef\PstAtCode{\the\catcode`\@} 
\catcode`\@=11\relax 
\pst@addfams{pst-spirograph}
\pstheader{pst-spirograph.pro}
\define@key[psset]{pst-spirograph}{Z1}[20]{\def\psk@ZA{#1 }}
\psset[pst-spirograph]{Z1=20}
\define@key[psset]{pst-spirograph}{Z2}[10]{\def\psk@ZB{#1 }}
\psset[pst-spirograph]{Z2=10}
\define@key[psset]{pst-spirograph}{m}[0.5]{\def\psk@m{#1 }}
\psset[pst-spirograph]{m=0.5}
\define@key[psset]{pst-spirograph}{ap}[20]{\def\psk@ap{#1 }}
\psset[pst-spirograph]{ap=20}
\define@key[psset]{pst-spirograph}{polarangle}[0]{\def\psk@polarangle{#1 }}
\psset[pst-spirograph]{polarangle=0}
\define@key[psset]{pst-spirograph}{holenumber}[1]{\def\psk@holenumber{#1 }}
\psset[pst-spirograph]{holenumber=1}
\define@key[psset]{pst-spirograph}{thetamax}[360]{\def\psk@thetamax{#1 }}
\psset[pst-spirograph]{thetamax=360}
%
\define@key[psset]{pst-spirograph}{color1}[{[rgb]{0.625 0.75 1}}]{\pst@getcolor{#1}\pscolora}
\psset[pst-spirograph]{color1={[rgb]{0.625 0.75 1}}}
\define@key[psset]{pst-spirograph}{color2}[{[rgb]{0.75 1 0.75}}]{\pst@getcolor{#1}\pscolorb}
\psset[pst-spirograph]{color2={[rgb]{0.75 1 0.75}}}
\define@key[psset]{pst-spirograph}{circlescolor}[red]{\pst@getcolor{#1}\pscolorc}
\psset[pst-spirograph]{circlescolor=red}
\define@key[psset]{pst-spirograph}{curvecolor}[red]{\pst@getcolor{#1}\pscolord}
\psset[pst-spirograph]{curvecolor=red}
%
\newdimen\pscurvewidth
\define@key[psset]{pst-spirograph}{curvewidth}[1pt]{\pssetlength\pscurvewidth{#1}}
\psset[pst-spirograph]{curvewidth=1pt}
%% === Option pour dessiner le type d'engrenage ---------------------
\define@boolkey[psset]{pst-spirograph}[Pst@spirograph@]{inner}[true]{}
\psset[pst-spirograph]{inner}
%% === pour dessiner cercle de base et cercle primitif
\define@boolkey[psset]{pst-spirograph}[Pst@spirograph@]{circles}[true]{}
\psset[pst-spirograph]{circles=false}
%
\def\psSpirograph{\def\pst@par{}\pst@object{psSpirograph}}
\def\psSpirograph@i{\@ifnextchar({\psSpirograph@ii}{\psSpirograph@ii(0,0)}}
\def\psSpirograph@ii(#1){%
  \begin@SpecialObj
  \pst@@getcoor{#1}%
  \addto@pscode{
    tx@spirographDict begin
    \pst@coor /t@@y ED /t@@x ED
    1 setlinejoin
    /cm { \pst@number\psunit mul } bind def
    /Z1 \psk@ZA def
    /m1 \psk@m def
    /Z2 \psk@ZB def
    /m2 \psk@m def
    /ap \psk@ap def
    /polarAngle  \psk@polarangle def
    /setlinedash { [ \psk@dash\space ] 0 setdash } def 
    /ni \psk@holenumber def % numero du trou
    ni 8 gt {/ni 8 def} if
    /thetamax \psk@thetamax def
%
    /ifinner \ifPst@spirograph@inner true \else false \fi def
    /ifcircles \ifPst@spirograph@circles true \else false \fi def 
    /iffill \ifx\psk@fillstyle\relax false \else true \fi def
    /Fill { \tx@setTransparency fill } def
%
    /ki 1 ni 9 div sub def
    /r2 m1 Z2 mul 2 div def
    /qi Z1 Z2 div 1 sub def
    /qi2 Z1 Z2 div 1 add def
    thetamax 0 le {/pas -1 def}{/pas 1 def} ifelse
% tableau des points de l'hypocycloide
    /tabSpirograph [ 0 pas thetamax {/i_ exch def [i_ coorPen]} for ] def
% tableau des points de l'epicycloide
    /tabSpirograph2 [ 0 pas thetamax {/i_ exch def [i_ coorPen2]} for ] def
    /nPts tabSpirograph length 1 sub def
    /nPts2 tabSpirograph2 length 1 sub def
    /color1 {\pst@usecolor\pscolora } def
    /color2 {\pst@usecolor\pscolorb } def
    /circlescolor {\pst@usecolor\pscolorc } def
    /curvecolor {\pst@usecolor\pscolord } def
    /linecolor  {\pst@usecolor\pslinecolor} def
    /fillopacity \psk@opacityalpha def
    /SetCurveWidth { \pst@number\pscurvewidth SLW } def
%
    Roue1
    Roue2
    end
  }%
  \end@SpecialObj
  \ignorespaces}
%
\catcode`\@=\PstAtCode\relax
%
\endinput 