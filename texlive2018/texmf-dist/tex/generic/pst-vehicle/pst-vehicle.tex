%%
%% This is file `pst-vehicle.tex',
%%
%% IMPORTANT NOTICE:
%%
%% Package `pst-vehicle.tex'
%%
%% Thomas Söll
%% with the collaboration of
%% Juergen Gilg
%% Manuel Luque
%%
%% This program can redistributed and/or modified under %%
%% the terms of the LaTeX Project Public License        %%
%% Distributed from CTAN archives in directory          %%
%% macros/latex/base/lppl.txt; either version 1.3c of   %%
%% the License, or (at your option) any later version.  %%
%%
%% DESCRIPTION:
%%   `pst-vehicle' is a PSTricks package
%%
%%
\csname PSTvehicleLoaded\endcsname
\let\PSTvehicleLoaded\endinput
% Requires PSTricks, pst-xkey, pst-node packages
\ifx\PSTricksLoaded\endinput\else\input pstricks.tex\fi
\ifx\PSTXKeyLoaded\endinput\else\input pst-xkey.tex\fi
\ifx\PSTplotLoaded\endinput\else\input pst-plot.tex\fi
\ifx\PSTnodeLoaded\endinput\else\input pst-node.tex\fi
\def\fileversion{1.2}
\def\filedate{2017/09/16}
\message{`PST' v\fileversion, \filedate}

\edef\PstAtCode{\the\catcode`\@} \catcode`\@=11\relax

\definecolor{greenSlp}{rgb}{0,0.4,0.0}
\definecolor{redSlp}{rgb}{0.7,0.134,0.134}

\pst@addfams{pst-vehicle}
\define@key[psset]{pst-vehicle}{backwheel}[\wheelA]{\def\pst@backwheel{#1 }} %
\define@key[psset]{pst-vehicle}{frontwheel}[\wheelA]{\def\pst@frontwheel{#1 }} %
\define@key[psset]{pst-vehicle}{ownvehicle}{\def\pst@ownvehicle{#1 }} %
\define@key[psset]{pst-vehicle}{vehicle}[\Bike]{\def\pst@vehicle{#1}} %
\define@key[psset]{pst-vehicle}{GravNode}[]{\def\pst@GravNode{#1 }} %
\define@key[psset]{pst-vehicle}{d}[5.8]{\def\pst@d{#1 }} %
\define@key[psset]{pst-vehicle}{rF}[1.6]{\def\pst@rF{#1 }} %
\define@key[psset]{pst-vehicle}{rB}[1.6]{\def\pst@rB{#1 }} %
\define@key[psset]{pst-vehicle}{gang}[1]{\def\pst@gang{#1 }} %
\define@key[psset]{pst-vehicle}{epsilon}[1e-6]{\def\pst@epsilon{#1 }} %
\define@key[psset]{pst-vehicle}{startPos}[0]{\def\pst@startPos{#1 }} %
\define@boolkey[psset]{pst-vehicle}[Pst@]{showSlope}[true]{} %
\define@boolkey[psset]{pst-vehicle}[Pst@]{MonoAxis}[false]{} %
%\define@boolkey[psset]{pst-vehicle}[Pst@]{EqValveStartPos}[true]{} %
%%
%% This is file `ListVehicles.tex',
%%
%% IMPORTANT NOTICE:
%%
%% Package `pst-vehicle.tex'
%%
%% Thomas Söll
%% with the collaboration of
%% Juergen Gilg
%% Manuel Luque
%%
%% This program can redistributed and/or modified under %%
%% the terms of the LaTeX Project Public License        %%
%% Distributed from CTAN archives in directory          %%
%% macros/latex/base/lppl.txt; either version 1.3c of   %%
%% the License, or (at your option) any later version.  %%
%%
%% DESCRIPTION:
%%   `pst-vehicle' is a PSTricks package
%%
%%
\newpsstyle{segway}{rB=1.4,backwheel=\segWheel}
\newpsstyle{unicycle}{rB=1.6,backwheel=\SpokesWheelB}
\newpsstyle{tractor}{d=4,rB=1.4,rF=1.0}
\newpsstyle{truck}{backwheel=\TruckWheel,frontwheel=\TruckWheel,d=6.28,rB=1.9,rF=1.9}
\newpsstyle{bike}{backwheel=\SpokesWheelB,frontwheel=\SpokesWheelB,d=5.8,rB=1.6,rF=1.6}
%\wheelA,\wheelB,\wheelC,\segWheel,\arrowWheel,\TruckWheel,\TractorFrontWheel,\TractorRearWheel,\SpokesWheelCrossed,\SpokesWheelA
\def\Bike{% ------------------------------ Vehicle  Nr. 1: normal bike ----------------------------
\rput{!phiB}(0,0){% ------------ backwheel --- Hinterrad
\pst@backwheel
}%
\rput{!phiF}(!/rB rF def AF1x AF1y){%------ frontwheel --- Vorderrad
\pst@frontwheel
}%
\rput(!0 rB neg){%
\psline[linewidth=3pt](2.5,1.95)(0,1.7)%
\psline[linewidth=3pt](2.5,1.25)(0,1.45)%
\rput{!phiB Gang div}(!2.5 rB){%   Kurbel
\pscircle*(0,0){0.4}%
\psline[linewidth=4pt](-0.3,-0.9)(0.3,0.9)%
\rput{!phiB Gang div neg}(-0.3,-0.9){\psline[linewidth=4pt](0,0)(-0.3,0)}%
\rput{!phiB Gang div neg}(0.3,0.9){\psline[linewidth=4pt](0,0)(0.3,0)}%
}%
\psline[linewidth=5pt](5.8,1.6)(4.75,4.85)(3.8,4.85)%
\psline[linewidth=5pt](0,1.6)(2.5,1.6)(5.2,3.65)%
\psline[linewidth=5pt](0,1.6)(1.6,4.05)(5,4.05)%
\psline[linewidth=5pt](2.5,1.6)(1.5,4.45)%
\psline[linewidth=7pt](0.8,4.45)(2,4.45)%
\pspolygon[linecolor=\pslinecolor,fillstyle=solid,fillcolor=\pslinecolor](5.05,3.85)(5.65,4.1)(5.65,3.6)%
}%
}%
\def\HighWheeler{% ------------------------------ Vehicle  Nr. 2: high wheeler -- Hochrad  Nr 2 ----------
%------ pedal behind the frontwheel -- Pedal hinter dem Rad ---------------------------------------
\rput{!phiF 5 mul Gang div}(!AF1x AF1y){%
\psline[linewidth=3pt,border=0.5pt,bordercolor=white](0,0)(1.15;135)
\rput{!phiF 5 mul Gang div 180 add neg}(1.11;135){\pspolygon*[linearc=0.08,border=0.5pt,bordercolor=white]%
(-0.25,0.06)(-0.1,0.08)(-0.08,0.16)(0.08,0.16)(0.1,0.08)(0.25,0.06)%
(0.25,-0.06)(0.1,-0.08)(0.08,-0.16)(-0.08,-0.16)(-0.1,-0.08)(-0.25,-0.06)}
\pscircle[linewidth=0.75pt,dimen=outer,linecolor=white](1.1;135){0.075}
}%
%--------------------------------------------------------------------------------------------
\rput{!phiF}(!AF1x AF1y){% % frontwheel -----  Vorderrad
\multido{\iA=0+10}{36}{%
\rput(0,0){\psline[linewidth=0.7pt,border=0.35pt,bordercolor=white](0,0)(!rF \iA\space PtoC)}%
}
\pscircle[linewidth=7.5pt,dimen=outer](0,0){!rF}%
\pscircle[linewidth=0.6pt,dimen=outer,linecolor=white](0,0){!rF 0.955 mul}%
\pscircle*(0,0){0.3}%
\pscircle*[linecolor=white](0,0){0.2}%
\pscircle*(0,0){0.14}%
}%
%-------------------------------------------------------------------------------------------------
%--------------------------------------------------------------------------------------------
\rput{!phiF}(!AF1x AF1y){%
\pscircle*(0,0){0.14}%
\pscircle[linewidth=0.6pt,linecolor=white](0,0){0.07}
}%
%--------------------------------------------------------------------------------------------------
\rput{!phiB}(0,0){%  backwheel ------  Hinterrad
\multido{\iA=0+24}{15}{%
\rput(0,0){\psline[linewidth=0.7pt,border=0.35pt,bordercolor=white](0,0)(!rB \iA\space PtoC)}%
}%
\pscircle[linewidth=4.3pt,dimen=outer](0,0){!rB}%
\pscircle[linewidth=0.6pt,dimen=outer,linecolor=white](0,0){!rB 0.91 mul}%
\pscircle*[linewidth=0.5pt](0,0){0.14}%
}%
%-------- front to back connection -- Verbindung vordere Gabel zur Hinterachse ----------------------
\rput(!AF1x AF1y){\pnode(!rF 1.06 mul 170 PtoC){ZA}}%
\let\myfillcolor\pslinecolor
\rput(!AF1x AF1y){%
\pscustom[linewidth=0.4pt,linecolor=white,fillstyle=solid,fillcolor=\myfillcolor]{%
\parametricplot[linewidth=0.5pt]{88}{10}{-(rF+0.36)*cos(Pi*t/180)|(rF+0.36)*sin(Pi*t/180)}%
\psarc(!AF1x 0.1 sub neg AF1y 0.03 add neg){0.08}{200}{300}
\parametricplot[linewidth=0.5pt]{10}{88}{-(rF+0.08-(t-100)*0.0353*2.5/78)*cos(Pi*t/180)|(rF+0.08-(t-100)*0.0353*2.5/78)*sin(Pi*t/180)}
\closepath
}%
}%
\rput(0,0){%
\pscircle[linestyle=none,fillstyle=solid,fillcolor=\pslinecolor](0,0){0.08}%
\pscircle[linewidth=0.9pt,linecolor=white](0,0){0.08}}
%--------------------------------------------------------------------------------------------------
%-------front part and handle bar --- Vordere Gabel und Lenker -------------------------------------------
\rput{1.5}(!AF1x AF1y){%
\psline[linewidth=5.5pt,border=0.5pt,bordercolor=white](0,0.1)(!0 rF 1.265 mul)%
\pscircle*[linewidth=0.6pt](!0 rF 1.28 mul){0.075}%
\psline[linewidth=5.5pt,linecolor=white](!0 rF 1.265 mul)(!0 rF 1.27 mul)
\psline[linewidth=3pt,linearc=0.02](!0 rF 1.24 mul)(!0.193 rF 1.24 mul)(!0.33 rF 1.31 mul)(!0.38 rF 1.31 mul)
\psline[linewidth=4.7pt,linearc=0.02,linecap=1,border=0.5pt,bordercolor=white](!0.42 rF 1.32 mul)(!0.455 rF 1.22 mul)
\pscircle*[linewidth=0.6pt](!0.455 rF 1.223 mul){0.145}%
\pscircle[linecolor=white,linewidth=0.5pt](!0.455 rF 1.223 mul){0.085}%
\pscircle[linecolor=white,linewidth=0.65pt](!0.455 rF 1.223 mul){0.15}%
\psline[linewidth=5.5pt,linecolor=white](!0 rF 1.208 mul)(!0 rF 1.211 mul)
\psline[linewidth=8.5pt](!0 rF 1.186 mul)(!0 rF 1.208 mul)
\psline[linewidth=5.5pt,linecolor=white](!0 rF 1.183 mul)(!0 rF 1.186 mul)
}%
%--------------------------------------------------------------------------------------------------
%--------- pedal in front of the frontwheel --- Pedal vor dem Rad ---------------------------------------------------------------
\rput{!phiF 5 mul Gang div 180 add}(!AF1x AF1y){%
\psline[linewidth=3pt,border=0.5pt,bordercolor=white](0.12;135)(1.15;135)
\rput{!phiF 5 mul Gang div 180 add neg}(1.11;135){\pspolygon*[linearc=0.08,border=0.5pt,bordercolor=white]%
(-0.25,0.06)(-0.1,0.08)(-0.08,0.16)(0.08,0.16)(0.1,0.08)(0.25,0.06)%
(0.25,-0.06)(0.1,-0.08)(0.08,-0.16)(-0.08,-0.16)(-0.1,-0.08)(-0.25,-0.06)}
\pscircle[linewidth=0.75pt,dimen=outer,linecolor=white](1.1;135){0.075}
\psline[linewidth=3pt](0,0)(0.12;135)
\pscircle[linecolor=white,linewidth=0.3pt](0,0){0.065}%
}
%---------------------------------------------------------------------------------
\rput(2,5.87){%
%--------- special garniture frontwheel ---- Geschweifte Linie am Vorderrad -------------------------------------------------------
\rput(0,0){%
\pscustom[linewidth=1pt]{%
\psarc(3.16,1.53){0.1}{0}{180}
\psbezier(3.06,1.2)(3.2,1.05)(3.5,1.05)
\psbezier(3.63,1.05)(3.6,1.16)(3.53,1.17)
}%
\pscircle*[linewidth=0.6pt](3.2,1.5){0.075}%
\psellipse*[rot=0,linewidth=0.6pt](3.47,1.15)(0.085,0.05)%
}%
%------------ part of saddle ------ Sattelhalterung -----------------------------------------------------------
\pscircle*[linewidth=0.6pt](2.63,1.6){0.068}%
\rput(0,0){%
\pscustom[showpoints=true,linewidth=1pt]{%
\psarcn(2.66,1.63){0.097}{235}{10}
\psbezier(2.8,1.3)(2.1,1.32)(1.4,1.32)
\psbezier(1.1,1.32)(0.9,1.1)(0.82,1)
\psbezier(0.74,0.9)(0.6,0.54)(0.3,0.54)
\psbezier(0.15,0.54)(0.17,0.7)(0.17,0.7)
\psarcn(0.31,0.7){0.14}{180}{0}
\psbezier(0.44,0.61)(0.34,0.61)(0.32,0.61)
}%
\pscircle*[linewidth=0.6pt](0.33,0.673){0.075}%
}%
\rput(0,0){%
\pscustom[showpoints=true,linewidth=1pt]{%
\psbezier(0.35,0.53)(0.3,0.45)(0.1,0.48)(0.1,0.4)
\psbezier(0.1,0.31)(0.18,0.345)(0.19,0.39)
}%
\psellipse*[rot=50,linewidth=0.6pt](0.2,0.4)(0.055,0.03)%
}%
%-------------------------------------------------------------------------------------------------------
%------------------ saddle ----- Sattel ----------------------------------------------------------------
\rput(0,0){%
\pscustom[showpoints=true,linewidth=0.4pt,fillstyle=solid,fillcolor=\pslinecolor]{%
\psbezier(1.34,1.25)(1.42,1.25)(1.43,1.25)(1.58,1.25)
\psbezier(1.58,1.35)(1.62,1.38)(1.65,1.4)% <<----- Start the white line -- Startpunkt für die weiße Linie
\psbezier(1.7,1.37)(1.8,1.38)(1.91,1.38)
\psbezier(1.98,1.38)(2.35,1.48)(2.35,1.55)
\psbezier(2.35,1.59)(2.2,1.6)(2.15,1.6)
\psbezier(2.05,1.6)(1.8,1.55)(1.74,1.55)
\psbezier(1.56,1.55)(1.41,1.75)(1.1,1.75)
\psbezier(0.9,1.75)(0.79,1.65)(0.79,1.59)
\lineto(0.835,1.59)
\psbezier(0.83,1.5)(1.12,1.45)(1.2,1.42)
\psbezier(1.25,1.4)(1.34,1.38)(1.34,1.25)
\closepath
}%
\psbezier[linecolor=white,linewidth=0.6pt](0.8,1.584)(1.1,1.75)(1.4,1.58)(1.67,1.38)
}%
}}%
\def\Truck{% \psset{radH=1.9,radV=1.9,d=6.28,vehicle=\Truck,ownvehicle=\ownTestB,frontwheel=\segWheel,backwheel=\segWheel}
% ------------------------------ Vehicle Nr. 3: truck -----------------------------------
\rput(0,-2){%
\pscustom[linecolor=blue,fillstyle=solid,fillcolor=blue!20]{%
\psline(8.6,2)(8.38,2)
\psarc(6.28,2){2.1}{0}{180}
\psline(4.18,2)(2.1,2)
\psarc(0,2){2.1}{0}{180}
\psline(-2,2)(-2.2,2)
\moveto(-2.2,2)
\psline[linearc=0.2](-2.2,4.4)(2.5,4.4)(2.5,6)(4.4,6)(5.2,4.4)(7.8,4)(8.6,2)
\moveto(4.2,5.7)
\pspolygon[linearc=0.1](4.4,5.7)(2.8,5.7)(2.8,4.2)(5.1,4.2)
}%
\pswedge[fillstyle=solid,fillcolor=gray!20](6.28,2){2.1}{0}{180}
\pswedge[fillstyle=solid,fillcolor=gray!20](0,2){2.1}{0}{180}
\psarc[doubleline=true,doublecolor=blue!50](0,2){2.1}{0}{180}
\psarc[doubleline=true,doublecolor=blue!50](6.28,2){2.1}{0}{180}
% le phare
\pswedge[linecolor=blue,fillstyle=solid,fillcolor=blue!20](8.65,3){0.4}{90}{270}
% le conducteur
\pstVerb{%
    /r1 0.4 def
    /a1 -50 def
    /b1 50 def
% r2=r1*sqrt(2-sin(2*a1))
    /r2 r1 2 1 a1 cos sub mul sqrt mul def
    /b2 a1 sin neg 1 a1 cos sub atan def
}%
\rput(3.8,4.6){
\pscustom[fillstyle=solid,fillcolor={[RGB]{253 191 183}}]{\psarc(0,0){!r1}{20}{-20}
           \psarc(!r1 0){0.075}{-90}{90}
           \closepath}
\psarcn(!r1 0){!r2}{!b2 180 add}{!b2 180 add b1 sub}
\pscircle*(0.2,0.1){0.05}
\psarc(0.2,0.1){0.1}{60}{130}
\psarc(-0.1,0){0.1}{120}{240}
\pscustom[fillstyle=solid,fillcolor=red]{\psarc(0,0){!r1}{60}{160}\closepath}
\pcline[nodesepB=1](!r1 60 cos mul r1 60 sin mul)(!r1 160 cos mul r1 160 sin mul)}
}%
\rput{!phiB}(0,0){% ----------- backwheel --- Hinterrad
\pst@backwheel
}%
\rput{!phiF}(!/rB rF def AF1x AF1y){%----- frontwheel --- Vorderrad
\pst@frontwheel
}%
}%

%%%%%%%%%% Le tracteur %%%%%%%%%%%%
\definecolor{couleurtracteur}{RGB}{130 196 108}
\def\Tractor{% le tracteur seul
\psset{fillcolor=couleurtracteur}
\pscustom[fillstyle=solid]{%
\psline(!rB 160 cos mul rB 160 sin mul)(!rB 1.25 mul 160 cos mul rB 1.25 mul 160 sin mul)(-0.8,1.6)(1.2,1.5)(1.7,0.6)(1.7,0)(! rB 0)
\psarc(0,0){!rB}{0}{160}
\closepath}
\pscustom[fillstyle=solid]{
\psline(1.4,1.14)(1.7,0.6)(1.7,-0.6)(2.4,-0.6)(2.4,-0.4)(3,-0.4)
\psarcn(!dA12 rF rB sub){!rF}{180}{126.87}
\psline(!dA12 rF 126.87 cos mul add rF rB sub rF 126.87 sin mul add)(2.4,0.4)(2.4,1.6)(2,1.14)(1.4,1.14)
\closepath}
\psline[linecolor=blue](2.4,1.6)(2,2.8)(0.4,2.8)(0.2,1.55)
\pscustom[fillstyle=solid]{%
\psline(2,2.8)(0.4,2.8)(0.2,1.55)(-0.2,1.57)(0.2,3.1)(1.932,3.004)
\closepath}
\pscustom[fillstyle=solid]{%
\psline(!dA12 rF 126.87 cos mul add rF rB sub rF 126.87 sin mul add)(2.4,0.4)(2.4,1.6)(4,1.4)(!dA12 rF 80 cos mul add rF rB sub rF 80 sin mul add)
\psarc(!dA12 rF rB sub){!rF}{80}{126.87}
\closepath}
\pscustom[fillstyle=solid]{%
\psline(3.9,1.43)(3.9,1.8)(3.7,1.8)(3.7,1.4555)
\closepath}
\pspolygon[fillstyle=solid,fillcolor=magenta](4,1.8)(3.6,1.8)(3.8,2)
\psline(2.4,-0.4)(1.7,-0.4)
\psdiamond[linejoin=1,fillstyle=solid,fillcolor=lightgray,doubleline](2.55,1)(0.12,0.2)
\rput(3.3,1){\textsf{\textbf{Renault}}}%
\rput{!phiB}(0,0){% ----------- backwheel --- Hinterrad
%\pst@backwheel
\TractorRearWheel
}%
\rput{!phiF}(!AF1x AF1y){%----- frontwheel --- Vorderrad
%\pst@frontwheel
\TractorFrontWheel
}%
}

\def\Segway{% ------------------------------ Vehicle Nr. 4: Segway -----------------------------------
\rput{!gamma neg}(0,0){%
\psframe*(-0.6,1)(0.2,4.5)
\rput{-10}(-0.8,0){\psframe*[framearc=0.6](-0.9,4.3)(0.4,7.4)}
\pscircle*(0.3,8.2){0.78}
\psline[linewidth=7pt](1,0.5)(2.1,5.2)
%% ARM
\psline[linewidth=12pt](2.1,5.4)(0.6,5.8)(0.2,6.8)
\pscircle*(2.1,5.4){0.3}
\pscircle[linecolor=white](2.1,5.4){0.25}
\psarc[linewidth=5pt](0,0){1.55}{0}{180}
%% la ROUE
}%
\rput{!phiB}(0,0){% ----------- backwheel --- Hinterrad
\pst@backwheel
}%
}%

\def\UniCycle{% ------------------------------ Vehicle UniCycle -----------------------------------
\rput{!gamma neg}(0,0){%
\rput{!phiB 180 add}(0,0){%!phiB
\rput{!phiB 180 add neg}(0,1){\psframe*[linecolor=black!80,framearc=0.15,linestyle=none,linewidth=0pt](-0.25,-0.1)(0.25,0.1)}%!phiB neg
\pscustom[linecolor=black!80,linewidth=0.015,fillstyle=solid,fillcolor=black!60]{%
\psarc(0,1){0.11}{0}{180}
\psbezier(-0.11,0.95)(-0.08,0.8)(-0.08,0.7)
\lineto(-0.08,0.4)
\psbezier(-0.08,0.25)(-0.12,0.05)(-0.12,0)
\psarc(0,0){0.12}{180}{0}
\psbezier(0.12,0.05)(0.08,0.25)(0.08,0.4)
\lineto(0.08,0.7)
\psbezier(0.08,0.8)(0.11,0.95)(0.11,1)
\closepath
}%
\pscircle[linecolor=black!90,linewidth=0.015](0,1){0.09}%
\pscircle*[linecolor=black](0,1){0.02}
}
\rput{!phiB}(0,0){% ----------- backwheel --- Hinterrad
\pst@backwheel
}%
%--- Gabel nach oben und Sattel
\rput{0}(0,0){%!gamma neg
\psframe*[linecolor=black!80,framearc=0.1,linestyle=none,linewidth=0pt](-0.085,2.9)(0.085,4)
\psframe*[linecolor=black!90,framearc=0.1,linestyle=none,linewidth=0pt](-0.1,1.9)(0.1,3)
\psframe*[linecolor=black,framearc=0.1,linestyle=none,linewidth=0pt](-0.16,2.85)(0.12,3.05)
\psframe*[linecolor=black!80,framearc=0.15,linestyle=none,linewidth=0pt](-0.12,0)(0.12,2)
%------------ Sattel ----------------------
\pspolygon[fillstyle=solid,fillcolor=black!90,linearc=0.1,linestyle=none](-0.7,4)(1,4)(1,4.4)(0.6,4.35)(-0.2,4.35)(-0.7,4.4)
}%
%Pedale vorne
\rput{!phiB}(0,0){%!phiB
\pscustom[linecolor=black!80,linewidth=0.015,fillstyle=solid,fillcolor=black!60]{%
\psarc(0,1){0.11}{0}{180}
\psbezier(-0.11,0.95)(-0.08,0.8)(-0.08,0.7)
\lineto(-0.08,0.4)
\psbezier(-0.08,0.25)(-0.12,0.05)(-0.12,0)
\psarc(0,0){0.12}{180}{0}
\psbezier(0.12,0.05)(0.08,0.25)(0.08,0.4)
\lineto(0.08,0.7)
\psbezier(0.08,0.8)(0.11,0.95)(0.11,1)
\closepath
}%
\rput{!phiB neg}(0,1){\psframe*[linecolor=black!80,framearc=0.15,linestyle=none,linewidth=0pt](-0.25,-0.1)(0.25,0.1)}%!phiB neg
\pscircle[linecolor=black!60,linewidth=0.015](0,1){0.09}%
\pscircle*[linecolor=black](0,1){0.02}
\pscircle[linecolor=black,linewidth=0.015](0,0){0.1}%
\pscircle*[linecolor=black](0,0){0.02}
}%
}%
}%

\def\SelfDefinedVehicle{% ------------- Vehicle Nr. 5: self defined vehicle --  Eigenes Fahrzeug Nr 5
\pst@ownvehicle
\rput{!phiB}(0,0){% ----------- backwheel --- Hinterrad
\pst@backwheel
}%
\rput{!phiF}(!/rB rF def AF1x AF1y){%----- frontwheel --- Vorderrad
\pst@frontwheel
}%
}%

\def\wheelA{%
\multido{\iA=0+36}{10}{%
\rput(0,0){\psline[linewidth=2pt](0,0)(!rB \iA\space PtoC)}%
}%
\pscircle[linewidth=5pt,dimen=outer](0,0){!rB}%
\pscircle*(0,0){0.25}%
}%

\def\wheelB{%
\multido{\iA=0+36}{10}{%
\definecolor[ps]{couleurrayons}{hsb}{\iA\space 360 div 1 1 }%
\rput(0,0){\psline[linecolor=couleurrayons,linewidth=2pt](0,0)(!rB 0.9 mul \iA\space PtoC)}
}%
\pscircle[linewidth=5pt,dimen=outer](0,0){!rB}%
\pscircle*(0,0){0.25}%
}%

\def\wheelC{%
\pscircle[fillstyle=solid,fillcolor=gray!20,dimen=outer](0,0){!rB}%
\multido{\iA=0+36}{10}{%
\definecolor[ps]{couleurrayons}{hsb}{\iA\space 360 div 1 1 }%
\rput(0,0){\psline[linecolor=couleurrayons,linewidth=2pt](0,0)(!rB 0.9 mul \iA\space PtoC)}
}%
\pscircle[linewidth=15pt,dimen=outer](0,0){!rB}%
\pscircle*(0,0){0.25}%
}%

\def\wheelD{%
\multido{\iA=0+36}{10}{%
\definecolor[ps]{couleurrayons}{hsb}{\iA\space 360 div 1 1 }%
\rput(0,0){\psline[linecolor=couleurrayons,linewidth=1pt](0,0)(!rB \iA\space PtoC)}
}%
\pscircle[linewidth=1pt,dimen=outer](0,0){!rB}%
\pscircle*(0,0){0.1}%
}

\def\arrowWheel{%
\pscircle*(0,0){!rB}
\pscircle*[linecolor=white](0,0){0.2}
\multido{\iA=0+30}{12}{%
\psline[linecolor=white](0,0)(!rB 0.7 mul \iA\space PtoC)
}%
\pscircle[linecolor=white,linewidth=2pt](0,0){!rB 0.7 mul}
\psline[linecolor=magenta,linewidth=1.5pt]{->}(0,0)(!rB -90 PtoC)
}%

\def\TruckWheel{%
\pscircle*(0,0){!rB}
\pscircle*[linecolor=white](0,0){0.2}
\multido{\iA=0+30}{12}{%
\psline[linecolor=white](0,0)(!rB 0.65 mul \iA\space PtoC)
}%
\pscircle[linecolor=white,linewidth=2pt](0,0){!rB 0.65 mul}
}%

\def\segWheel{%
\pscircle*(0,0){!rB}
\pscircle*[linecolor=white](0,0){0.2}
\multido{\iA=0+30}{12}{%
\psline[linecolor=white](0,0)(!rB 0.9 mul \iA\space PtoC)
}%
\pscircle[linecolor=white,linewidth=2pt](0,0){!rB 0.9 mul}
}%

\def\SpokesWheelCrossed{
\multido{\iM=0+40,\iJ=60+40}{10}{\psline[linewidth=0.1](!rB 0.16 mul \iM\space PtoC)(!rB \iJ\space PtoC)
                                 \psline[linewidth=0.1](!rB 0.16 mul \iM\space PtoC)(!rB \iJ\space 10 add PtoC)}
\pscircle[fillstyle=solid,fillcolor=white,linewidth=0.1]{!rB 0.16 mul 0.2 add}
\multido{\i=0+40}{9}{\pscircle[linestyle=dashed,linecolor=gray!50](!rB 0.16 mul \i\space PtoC){! 0.1 rB 0.16 mul mul}
                     \pscircle*(!rB 0.16 mul \i\space 20 add PtoC){! 0.1 rB 0.16 mul mul}}
\multido{\iM=20+40,\iJ=-30+40}{10}{\psline[linewidth=0.1](!rB 0.16 mul \iM\space PtoC)(!rB \iJ\space PtoC)
                                   \psline[linewidth=0.1](!rB 0.16 mul \iM\space PtoC)(!rB \iJ\space 10 sub PtoC)}
\pscircle[dimen=outer,linewidth=0.5,linecolor=black!90]{!rB}%
\pscircle[dimen=outer,linewidth=0.1,linecolor=white]{!rB 0.3 sub}%
}

\def\SpokesWheelA{
\multido{\iM=0+40,\iJ=60+40}{10}{\psline[linewidth=0.025](!rB 0.1 mul \iM\space PtoC)(!rB \iJ\space PtoC)
                                 \psline[linewidth=0.025](!rB 0.1 mul \iM\space PtoC)(!rB \iJ\space 10 add PtoC)}
\pscircle[fillstyle=solid,fillcolor=white,linewidth=0.025]{!rB 0.1 mul 0.1 add}
\multido{\i=0+40}{9}{\pscircle[linewidth=0.025,linecolor=gray!50](!rB 0.1 mul \i\space PtoC){! 0.03 rB 0.16 mul mul}
                     \pscircle*[linewidth=0.025](!rB 0.1 mul \i\space 20 add PtoC){! 0.03 rB 0.16 mul mul}}
\multido{\iM=20+40,\iJ=-30+40}{10}{\psline[linewidth=0.025](!rB 0.1 mul \iM\space PtoC)(!rB \iJ\space PtoC)
                                   \psline[linewidth=0.025](!rB 0.1 mul \iM\space PtoC)(!rB \iJ\space 10 sub PtoC)}
\pscircle[dimen=outer,linewidth=0.2,linecolor=black!90]{!rB}%
\pscircle[dimen=outer,linewidth=0.02,linecolor=white]{!rB 0.15 sub}%
}

\def\SpokesWheelB{
\multido{\iM=0+40,\iJ=60+40}{10}{\psline[linewidth=0.015,linecolor=black!80](!rB 0.11 mul \iM\space PtoC)(!rB \iJ\space PtoC)
                                 \psline[linewidth=0.015,linecolor=black!80](!rB 0.11 mul \iM\space PtoC)(!rB \iJ\space 10 add PtoC)}
\pscircle[fillstyle=solid,fillcolor=white,linewidth=0.025,linecolor=black!80]{!rB 0.11 mul 0.06 add}
\multido{\i=0+40}{9}{\pscircle[linewidth=0.01,linecolor=black!50](!rB 0.11 mul \i\space PtoC){! 0.09 rB 0.16 mul mul}
                     \pscircle*[linewidth=0.01,linecolor=black](!rB 0.11 mul \i\space 20 add PtoC){! 0.09 rB 0.16 mul mul}}
\multido{\iM=20+40,\iJ=-30+40}{10}{\psline[linewidth=0.015,linecolor=black!80](!rB 0.11 mul \iM\space PtoC)(!rB \iJ\space PtoC)
                                   \psline[linewidth=0.015,linecolor=black!80](!rB 0.11 mul \iM\space PtoC)(!rB \iJ\space 10 sub PtoC)}
\psline[linewidth=0.04,linecolor=black!90](!rB 0.3 sub -85 PtoC)(!rB 0.4 sub -85 PtoC)%
\psline[linewidth=0.048,linecolor=black!90](!rB 0.4 sub -85 PtoC)(!rB 0.44 sub -85 PtoC)%
\pscircle[dimen=outer,linewidth=0.3,linecolor=black!90]{!rB}%
\pscircle[dimen=outer,linewidth=0.09,linecolor=gray!20]{!rB 0.18 sub}%
}

\def\TractorFrontWheel{%
\pscircle*(0,0){0.25}%
% 0.5=rF/2 0.2=rF/5
\pscircle[linewidth=0.5,dimen=outer](0,0){!rF}
\multido{\iA=0+36}{10}{%
\psline[linewidth=2pt,linecolor=red](0,0)(!rF 2 div \iA\space PtoC)(!rF \iA\space PtoC)}%
\pscircle[dimen=outer,linewidth=0.3,linecolor={[rgb]{0.95 0.95 0}}](0,0){!rF 2 div}
\pscircle(0,0){!rF 5 div}
\multido{\iA=0+36}{10}{%
\psline[linewidth=2pt](0,0)(!rF 2 div \iA\space PtoC)}%
}%
\def\TractorRearWheel{%
\pscircle*(0,0){0.25}%
% 0.7=rB/2 0.28=rb/5
\pscircle[linewidth=0.7,dimen=outer](0,0){!rB}
\multido{\iA=0+36}{10}{%
\psline[linewidth=2pt,linecolor=red](0,0)(!rB 2 div \iA\space PtoC)(!rB \iA\space PtoC)}%
\pscircle[dimen=outer,linewidth=0.42,linecolor={[rgb]{0.95 0.95 0}}](0,0){!rB 2 div}
\pscircle(0,0){!rB 5 div}
\multido{\iA=0+36}{10}{%
\psline[linewidth=2pt](0,0)(!rB 2 div \iA\space PtoC)}%
}% 

\psset[pst-vehicle]{gang=1,epsilon=1e-6,rB=1.6,rF=1.6,d=5.8,vehicle=\Bike,ownvehicle=,backwheel=\wheelA,frontwheel=\wheelA,showSlope=true,%
startPos=0,MonoAxis=false,GravNode=dA12 2 div 1}
\psset{algebraic}
\def\psVehicle{\def\pst@par{}\pst@object{psVehicle}}%
\def\psVehicle@i#1#2#3{%
\pst@killglue%
\begingroup%
\use@par
%-----------------------------------------------------------------------------------------------------
\expandafter\ifx\pst@vehicle\HighWheeler\psset{rF=4,rB=1.13,d=5.8}\fi
\expandafter\ifx\pst@vehicle\Bike\psset{rF=1.6,rB=1.6,d=5.8}\fi
\expandafter\ifx\pst@vehicle\Truck\psset{rF=1.9,rB=1.9,d=6.28}\fi
\expandafter\ifx\pst@vehicle\Tractor\psset{rF=1,rB=1.4,d=4}\fi
\expandafter\ifx\pst@vehicle\UniCycle\psset{rB=1.6,MonoAxis=true}\fi
\expandafter\ifx\pst@vehicle\Segway\psset{MonoAxis=true}\fi
\begin@SpecialObj
\pst@Verb{%
 /rpn { tx@AlgToPs begin AlgToPs end cvx } def%
 /x0 #2 def % -------------- x-Wert des Punktes auf der Kurve, wo das Hinterrad die Kurve berührt; von diesem Wert startet die Rechnung
 /XST \pst@startPos\space def %   Untergrenze für die Integration bei der Rotationswinkelbestimmung
%---- % Definition of the function f(x), its first derivative f'(x) and  \sqrt{1+f'(x)^2}   Definition de la fonction et la premiere derivee
  /func (#3) rpn def
  /Diff (Derive(1,#3)) rpn def
  /DiffI (Derive(2,#3)) rpn def
  /dAB (sqrt(1+Diff^2)) rpn def
  /dABdiff (Derive(1,sqrt(1+(Derive(1,#3))^2))) rpn def
  /x XST def func /funcxST exch def %---- f(XST)
  /x XST def Diff /DiffxST exch def %---- f'(XST)
  /x x0 def func /funcx0 exch def % ----- f(x0)
  /x x0 def Diff /Diffx0 exch def % ----- f'(x0)
  /x x0 def DiffI /DiffIx0 exch def % --- f''(x0)
%-----------------------------------------------------------------------------------------------------------------------
/eps \pst@epsilon def
% Definition of a transmission between frontwheel and backwheel (interesting for vehicles with pedals) --- Gangschaltung
/Gang \pst@gang def
%% Definition of a scaling factor for the vehicles
#1 /skal exch def
%------------------ Radius frontwheel -----------------------------------------
/rF \pst@rF def
/rB \pst@rB def
/dA12 \pst@d def
/rFs rF skal mul def
/rBs rB skal mul def
%--------% dA12 = Distance between the axes of the wheels ----- Achsabstand --- distance entre les axes
/dA12s dA12 skal mul def
/tA 1 1 Diffx0 dup mul add sqrt div def%
/deltax0 tA Diffx0 mul rBs mul def
/deltay0 tA rBs mul def
%-----------------------------------------------------------------------------------------------------------------------------------------
/Function ((x-x0+rBs*Diffx0/(sqrt(1+(Diffx0)^2))-rFs*(Diff)/(sqrt(1+(Diff)^2)))^2+%
            (func-funcx0+rFs/(sqrt(1+(Diff)^2))-rBs/(sqrt(1+(Diffx0)^2)))^2-dA12s^2) rpn def
%-----------------------------------------------------------------------------------------------------------------------------------------
/FunctionST ((x-XST+rBs*DiffxST/(sqrt(1+(DiffxST)^2))-rFs*(Diff)/(sqrt(1+(Diff)^2)))^2+%
            (func-funcxST+rFs/(sqrt(1+(Diff)^2))-rBs/(sqrt(1+(DiffxST)^2)))^2-dA12s^2) rpn def
%--------- inferior value to search for the intersection point of the frontwheel with the curve
/Zeros { %%  Funktion xinf
1 dict begin
/Xinf exch def % ---------------------------------- Untergrenze für die Schnittpunktsuche des Vorderrades mit der Kurve
% superior value to search for the intersection point of the frontwheel with the curve = x0 + distance axes + radius frontwheel + radius backwheel
/Xsup Xinf dA12s add rBs rFs add add def % ---------- Obergrenze für die Schnittpunktsuche des Vorderrades mit der Kurve
%---% Calculating the intersection point of the frontwheel with the function------ Schnittpunktberechnung ---------------------------------
/NB 0 def %-----------------loop-variable  -----  Laufvariable für loop
/NbreIterations 200 def % ---------- Maximum of iterations for the loop
{ %------------------------------------ loop begin ---------------------
 /xM Xinf Xsup add 2 div def %--------- Mittelwert von xM = (Xinf + Xsup):2
   /x Xinf def
   /F_1 FUNK def %----------------- F(Xinf)
   /x xM def %-------------------------
   /F_M FUNK def %----------------- F(xM)
   F_M 0 eq {exit} if %---------------- if F(xM) = 0 --> exit
    F_1 F_M mul 0 ge {/Xinf xM def} %-- F(Xinf) * F(xM) >= 0 Xinf = xM, else Xsup = xM
                     {/Xsup xM def}
    ifelse
 Xinf Xsup sub abs 1e-8 le {exit} if %- if Xinf - Xsup <= 10^-8 --> exit
  /NB NB 1 add def %------------------- else loopvariable NB = NB + 1  Loopvariable um eins erhöhen
 NB NbreIterations ge {exit} if %------ if number of iterations >= 200 --> exit
 } loop
xM
 end
} def
% \ifPst@EqValveStartPos
/FUNK {FunctionST} def XST Zeros /FWxST exch def
% \else /FWxST XST def \fi
/FUNK {Function} def x0 Zeros /FWx exch def
/FWy /x FWx def func def  %----------------------- Berührpunkt des Vorderrades (FWx,FWy)
/mFWy /x FWx def Diff def %----------------------- mFWy = Tangentensteigung in (FWx,FWy)
/TermFW 1 1 mFWy dup mul add sqrt div def %             1/sqrt(1+f'(x_Q)^2)
/deltaxFW TermFW mFWy mul rFs mul def %                  skal*rF*f'(x_Q)*1/sqrt(1+f'(x_Q)^2)
/deltayFW TermFW rFs mul def %                           skal*rF*1/sqrt(1+f'(x_Q)^2)
%------------------------------------------------------------------------------------
/KWRho {DiffI 1 Diff dup mul add 3 exp sqrt div} def
/dPhiB {1 rBs div KWRho sub dAB mul abs} def
/AngleCumB { %
 X1 X2 /x {dPhiB} eps SIMPSON
 } def
 /dPhiF {1 rFs div KWRho sub dAB mul abs} def
/AngleCumF { %
 X1 X2 /x {dPhiF} eps SIMPSON
 } def
% % length of the curve for the wheels  -- La longueur de la courbe pour la roue avant  --- Kurvenlänge
/X1 XST def /X2 x0 def %  Integral_{0}^{x0}
/sB AngleCumB def % ---backwheel - length from 0 to x0 ---- roue arrière ---- Kurvenlänge von 0 bis x0 (Hinterrad)
/X1 FWxST def /X2 FWx def
/sF AngleCumF def % --frontwheel - length from 0 to the abscissa of the intersection frontwheel with curve - roue avant -- Kurvenlänge von 0 bis zum SP Vorderrad - Kurve
%--------------------------------------------------------------------------------------------------------------------
%---% Definition angle of rotation for the backwheel --- Definition de l'angle de roue arriere ---- Rotationswinkel des Hinterrades
/phiB sB RadToDeg neg def
%---% Definition angle of rotation for the frontwheel --- Definition de l'angle de roue avant  ---- Rotationswinkel des Vorderrades
/phiF sF RadToDeg neg def
%---------------------------------------------------------------------------------------------------------------------
%--% coordinates for the axes of the wheels within the non-scaled system -- Koordinaten der Vorderradachse im nicht skalierten System ( außerhalb der \psscalebox )
%----------------------------------------------------------------------------------------------------------------------
/AFx FWx deltaxFW sub def %          % x-coordinate front axis   x-Koordinate der Vorderradachse
/AFy FWy deltayFW add def %          % y-coordinate front axis   y-Koordinate der Vorderradachse
/ABx x0 deltax0 sub def %            % x-coordinate back axis    x-Koordinate der Hinterradachse
/ABy funcx0 deltay0 add def %        % y-coordinate back axis    y-Koordinate der Hinterradachse
%--------------- Koordinaten der Vorderradachse im System des Fahrzeugs, also unskalierte Größen verwenden --( innerhalb der \psscalebox )
%--coordinates for the front axis of the wheels within \psscalebox  -- Die Koordinaten der Hinterradachse sind im System des Fahrzeugs  (0,0)
/AF1x dA12 dup mul rF rB sub dup mul sub sqrt def
/AF1y rF rB sub def
%------% slope and angle of the vehicle-----------------------------------------------------------------------------
/m-vehicle AFy ABy sub AFx ABx sub div def
/beta m-vehicle 1 atan def %          % angle for both radii of front- and backwheel are equal -- Neigungswinkel des Fahrzeugs bei gleich großen Rädern
/alpha AF1y neg AF1x atan def %       % additional angle if radii of front- and backwheel are not equal --  zusätzlicher Winkel bei unterschiedlich großen Rädern
% whole angle (correction with +180 in case the whole angle gets 90 degrees which can be possible with not equal radii)
/gamma beta alpha add AFx ABx lt { 180 add } if def %   gesamter Neigungswinkel des Fahrzeugs
%--------% Special case (mono-axis vehicle) Nr 4 --> segway ---------------------------------------------------------------------------
\ifPst@MonoAxis Diffx0 1 atan /omega exch def Diffx0 /mVehicle exch def \else FWy funcx0 sub FWx x0 sub div /mVehicle exch def mVehicle 1 atan /omega exch def \fi
/normNorm FWy funcx0 eq { 1 } { x0 FWx sub FWy funcx0 sub div dup mul 1 add sqrt } ifelse def
/mTgy FWy funcx0 eq { 1 } { x0 FWx sub FWy funcx0 sub div normNorm div } ifelse def
/mTgx FWy funcx0 eq { 0 } { 1 normNorm div } ifelse def
/xMTg x0 FWx add 2 div def
/yMTg funcx0 FWy add 2 div def
}%
%------------------------%%%%  END OF PS-CODE %%%%%--------------------------------------------------------
\pnode(!FWx FWy){FW}%
% % angle between the axes   angle de droite entre les axes
\rput{!gamma}(!ABx ABy){%  Das Fahrzeug wird mit Hilfe der Hinterradachse (ABx,ABy) und Gesamtdrehwinkel gamma gesetzt
%------------------------- % SETTING SOME VEHICLES  -------------------------------------------------------
\psscalebox{#1}{%---------------Das Fahrzeug kann skaliert werden -----------------------------------------
%\psset{linecolor=#2}%
\pnode(!\pst@GravNode){GravC}%
\pst@vehicle}}%
\ifPst@showSlope
\rput{!omega}(!x0 mVehicle 0 ge { 0 add } { 100 add } ifelse funcx0 0.75 sub){%
\psframebox[fillstyle=solid,fillcolor=greenSlp,linestyle=none]{\footnotesize\color{white}$m\geq 0$}}
\rput{!omega}(!x0 mVehicle 0 lt { 0 add } { 100 add } ifelse funcx0 0.75 sub){%
\psframebox[fillstyle=solid,fillcolor=redSlp,linestyle=none]{\footnotesize\color{white}$m < 0$}}
\ifPst@MonoAxis
\psplotTangent[linewidth=0.75\pslinewidth,linecolor=orange,nodesep=-2]{x0}{1}{#3}
\psplotTangent[linewidth=0.75\pslinewidth,linecolor=greenSlp,Tnormal,nodesep=-2]{x0}{1}{#3}
\psdot[linecolor=red](!x0 funcx0)
\else
\pcline[linecolor=greenSlp,nodesepA=-1,nodesepB=-3](!xMTg yMTg)(!xMTg mTgy 0 ge { mTgx } { mTgx neg } ifelse add yMTg mTgy abs add)
\pcline[linecolor=magenta,nodesep=-2](FW)(!x0 funcx0)
\psdot[linecolor=red](FW)
\psdot[linecolor=red](!x0 funcx0)
\fi%
\fi%
\showpointsfalse
\end@SpecialObj
\endgroup\ignorespaces}%
%
\def\SlopeoMeter#1#2{%
\colorlet{slpmColor}{#1}
\rput{0}(0,0){%
\pscircle[fillstyle=solid,fillcolor=black!90,linewidth=0.5pt,linestyle=solid](0,0){2.2}
\rput{!#2}(0,0){%
\multido{\r=.500+-.008,\rC=0+0.02}{25}{%
\pscircle[linewidth=0.008,linecolor=slpmColor,strokeopacity=\rC](0,0){!\r}
}
\multido{\i=0+2,\r=.400+-.008,\rC=0+0.02}{50}{%
\pswedge[linestyle=none,linewidth=0,fillstyle=solid,fillcolor=slpmColor,opacity=\r](0,0){2}{\i}{!\i\space 2 add}
}}
\multido{\nA=-90+10}{19}{%
\psline[linecolor=slpmColor](1.8;\nA)(1.95;\nA)
\rput(1.65;\nA){\psscalebox{0.4}{\color{slpmColor!10}\nA}}
}
\rput(-0.95,0.85){\psscalebox{0.6}{\color{slpmColor!20}\texttt{Steigungswinkel}}}
\rput(-0.95,0.5){\psscalebox{0.75}{\color{slpmColor!40}\texttt{Slope-o-Meter}}}
\rput{!#2}(0,0){%
\pspolygon[fillstyle=solid,fillcolor=slpmColor!50,linecolor=slpmColor,linejoin=2,linewidth=0.1pt](0.1;15)(0.1;-15)(1.75;0)
\pscircle[fillstyle=solid,fillcolor=black,linestyle=none,linewidth=0pt](0,0){0.3}
}}}
\catcode`\@=\PstAtCode\relax
\endinput
