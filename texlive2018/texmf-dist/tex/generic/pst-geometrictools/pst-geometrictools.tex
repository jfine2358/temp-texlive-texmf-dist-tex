%% This is file `pst-geometrictools.tex',
%%
%% Thomas Söll
%%
%% This program can redistributed and/or modified under %%
%% the terms of the LaTeX Project Public License        %%
%% Distributed from CTAN archives in directory          %%
%% macros/latex/base/lppl.txt; either version 1.3c of   %%
%% the License, or (at your option) any later version.  %%
%%
%% DESCRIPTION:
%%   `pst-geometrictools' is a PSTricks package to draw a protractor, a ruler, a compass and pencils 
%%
%%
\csname PSTgeometrictoolsLoaded\endcsname
\let\PSTgeometrictoolsLoaded\endinput
% Requires PSTricks, pst-xkey, pst-node packages
\ifx\PSTricksLoaded\endinput\else\input pstricks.tex\fi
\ifx\PSTXKeyLoaded\endinput\else\input pst-xkey.tex\fi
\ifx\PSTnodeLoaded\endinput\else\input pst-node.tex\fi
\def\fileversion{1.1}
\def\filedate{2017/09/16}
\message{`PST' v\fileversion, \filedate}

\edef\PstAtCode{\the\catcode`\@} \catcode`\@=11\relax


\pst@addfams{pst-geometrictools}

\define@boolkey[psset]{pst-geometrictools}[Pst@]{Ghost}[true]{}
\define@key[psset]{pst-geometrictools}{ProLineCol}[cyan]{\pst@getcolor{#1}\ProLineCol}
\define@key[psset]{pst-geometrictools}{ProFillCol}[gray!60]{\pst@getcolor{#1}\ProFillCol}
\define@key[psset]{pst-geometrictools}{ProScale}[1]{\def\pst@ProScale{#1 }}%
\define@key[psset]{pst-geometrictools}{OwnerTxt}[Euklid]{\def\pst@OwnerTxt{#1}}%
\define@key[psset]{pst-geometrictools}{MadeTxt}[made in NES]{\def\pst@MadeTxt{#1}}%
\define@key[psset]{pst-geometrictools}{PSfontO}[Symbol]{\def\pst@PSfontO{/#1 }}
\define@key[psset]{pst-geometrictools}{fontsizeO}[10]{\def\pst@fontsizeO{#1 }}
\define@key[psset]{pst-geometrictools}{PSfontM}[NimbusRomNo9L-Regu]{\def\pst@PSfontM{/#1 }}
\define@key[psset]{pst-geometrictools}{fontsizeM}[6]{\def\pst@fontsizeM{#1 }}
\define@key[psset]{pst-geometrictools}{country}[Germany]{\def\pst@country{#1}}
\psset[pst-geometrictools]{ProScale=1,OwnerTxt={Euklid},MadeTxt={made in NES},
  country=Germany,PSfontO=StandardSymL,PSfontM=NimbusRomNo9L-Regu,fontsizeO=10,fontsizeM=6,
  ProFillCol=gray!60,ProLineCol=cyan,Ghost=false}


\def\psProtractor{\def\pst@par{}\pst@object{psProtractor}}
\def\psProtractor@i#1(#2){\@ifnextchar(%
{\psProtractor@ii{#1}(#2)}%
{\psProtractor@ii{#1}(#2)(#2)}}%
\def\psProtractor@ii#1(#2)(#3){%-------------------------------    #1 Winkel,  #2 Ursprung P,  zweiter Punkt Q für Winkel
\begin@SpecialObj
\pst@killglue
\begingroup%
\pnode(#2){@PGD}%
\pnode(#3){@QGD}%
\use@par%
\pst@getcoor{@QGD}\pst@tempB % --------------------------   Hohle die Koordinaten und speichere sie in \pst@tempB
\pst@getcoor{@PGD}\pst@tempA % --------------------------   Hohle die Koordinaten und speichere sie in \pst@tempA
\addto@pscode{%
/sysfill {systemdict /fill get exec} def
/Country (\pst@country) def
    /cm {28.45274 mul} def %  ----------------------------------------  Definiere cm als Längeneinheit mit der multipliziert wird
    \pst@tempA \tx@UserCoor /@y1 ED /@x1 ED % ----------  Definiere die Koordinaten des Ursprungs (zur Winkelberechnung benötigt)
    \pst@tempB \tx@UserCoor /@y2 ED /@x2 ED % ----------  Definiere die Koordinaten des zweiten Punktes (zur Winkelberechnung benötigt)
   @y1 @y2 sub @x1 @x2 sub Atan /@delta ED %------------   Winkelberechnung
    \pst@tempA \pst@coor T % ------------------------------------   Tranformiere den Koordinatenursprung in P
    @delta #1 add 180 add rotate % ------------------------------     Drehe das gesamte Geodreieck
   \pst@ProScale dup scale % --------    Skaliere das ganze Geodreieck
%-------------------------------------------------------------------------------------------------------------
\ifPst@Ghost\else
gsave
%-------------------------------------------------------------------------------------------------------------
%-------------- Äußere Begrenzung des Geodreiecks -----------------------------------------------
gsave
     newpath
       -8 cm 0 moveto
       16 cm 0 rlineto
       -8 cm 8 cm rlineto
       closepath
     gsave
       0.7 setlinewidth
       0.1 setgray
       stroke
    grestore
grestore
%-------------------------------------------------------------------------------------------------------------
%-------------- Gefüllter innerer Teil des Geodreiecks -----------------------------------------------
gsave
    newpath
        -7.5 cm 0.5 22.5 tan mul cm moveto
       15 cm 0 rlineto
       -7.5 cm 7.5 cm rlineto
    closepath
    gsave
         \pst@usecolor\ProFillCol
       [ /ca .1 /SetTransparency  pdfmark %    set  transparency
       0.1 .setopacityalpha sysfill
    grestore
grestore
%-------------------------------------------------------------------------------------------------------------
%-------------- Millimetereinteilung auf der Grundseite -------------------------------------------
 /mmBlinie { newpath
                   -7 cm 0 cm moveto
                    0 cm 0.17 cm rlineto
%                   closepath
                     gsave
                     0.5 setlinewidth
                     0 setgray
                     stroke
                     grestore
                      0.1 cm 0 cm translate
                    } def
gsave
140 {mmBlinie} repeat
grestore
%-------------------------------------------------------------------------------------------------------------
%-------------- 0,5- cm-Einteilung auf der Grundseite ----------------------------------------------
   /cmBlinie { newpath
                   -7 cm 0 cm moveto
                    0 cm 0.27 cm rlineto
%                     closepath
                     gsave
                     0.55 setlinewidth
                     0 setgray
                     stroke
                     grestore
                      0.5 cm 0 cm translate
                    } def
gsave
   29 {cmBlinie} repeat
grestore
%-------------------------------------------------------------------------------------------------------------
%-------------- mm-Einteilung von der Grundseite nach oben ------------------------------------
/mmlinie { newpath
                   -2.3 cm 0.3 cm moveto
                    0.2 cm 0 rlineto
                    2.2 cm 0.3 cm moveto
                    0.2 cm 0 rlineto
%                    closepath
                     gsave
                     0.5 setlinewidth
                     0 setgray
                     stroke
                     grestore
                      0 cm 0.1 cm translate
                    } def
gsave
33 {mmlinie} repeat
grestore
%-------------------------------------------------------------------------------------------------------------
%-------------- 0,5 cm-Einteilung von der Grundseite nach oben --------------------------------
/cmlinie { newpath
                   -4.5 cm 0.5 cm moveto
                    2 cm 0 rlineto
                    0.75 cm 0 rmoveto
                    1.5 cm 0 rlineto
                    0.5 cm 0 rmoveto
                    1.5 cm 0 rlineto
                    0.75 cm 0 rmoveto
                    2 cm 0 rlineto
%                 closepath
                     gsave
                     0.6 setlinewidth
                     0 setgray
                     stroke
                     grestore
                      0 cm 0.5 cm translate
                   } def
%------------------------------------------------------------------------------------------------------------------------------------------------------------
%--------------- Clippen entlang eines Kreises  der 0,5 cm-Einteilung von der Grundseite nach oben ------------------------------------
/kreisring { newpath
                      0 0 translate
                      0 0 4.2 cm 0 180 arc
                     closepath
                    } def
               gsave
                        kreisring clip
                        7 {cmlinie} repeat
              grestore
%---------------------------------------------------------------------------------------------------------------------------------------------
%--------------- farbiger und transparenter Halb-Ring -----------------------------------------------------------------------------
   gsave
                 newpath
                      0 0 translate
                      0 0 5.1 cm 6 174 arc
                      0 0 4.8 cm 174 6 arcn
                 closepath
                   gsave
                    \pst@usecolor\ProLineCol
                      [ /ca .4 /SetTransparency  pdfmark %    set  transparency
                       0.4 .setopacityalpha
                       sysfill
                 grestore
   grestore
%---------------------------------------------------------------------------------------------------------------------------------------------
%--------------- 1-Grad-Winkeleinteilung an der linken Kathete -----------------------------------------------------------------
 gsave
   91 1 179 { /Wk exch def % ---------------------------------------------------- Schleifenvariable festlegen, die von 91° bis 179° geht
                       {8 Wk sin Wk cos div 1 sub div} /xWS1 exch def %---------- xWS1=8/(tan(alpha)-1) --> SP der Geraden linke Kathete mit Ursprungsgeraden im Winkel alpha
                       {7.8 Wk sin Wk cos div 1 sub div} /xWS2 exch def % ------- xWS2=7,8/(tan(alpha)-1) --> SP einer Parallelen zur Geraden linke Kathete mit Ursprungsgeraden im Winkel alpha
                        newpath
                        xWS1 cm xWS1 8 add cm moveto % --------------------------- zu xWS1 gehöriger y-Wert ist xWS1 + 8
                        xWS2 cm xWS2 7.8 add cm % ----------------------------------- zu xWS2 gehöriger y-Wert ist xWS1 + 7,8
                       lineto
%                      closepath
                       0.5 setlinewidth
                       0 setgray
                       stroke
                   } for
  grestore
%---------------------------------------------------------------------------------------------------------------------------------------------
%--------------- 5-Grad-Winkeleinteilung an der linken Kathete -----------------------------------------------------------------
  gsave
     95 5 175 { /Wk exch def
                        {8 Wk sin Wk cos div 1 sub div} /xWS1 exch def
                        {7.6 Wk sin Wk cos div 1 sub div} /xWS2 exch def
                        newpath
                        xWS1 cm xWS1 8 add cm moveto
                        xWS2 cm xWS2 7.6 add cm
                        lineto
%                       closepath
                        0.5 setlinewidth
                        0 setgray
                        stroke
                      } for
  grestore
%---------------------------------------------------------------------------------------------------------------------------------------------
%--------------- 10-Grad-Winkeleinteilung an der linken Kathete -----------------------------------------------------------------
  gsave
    100 10 170 { /Wk exch def
                          {8 Wk sin Wk cos div 1 sub div} /xWS1 exch def
                          {5.1 Wk cos mul  Wk cos div 1 sub div} /xWS2 exch def
                          newpath
                          xWS1 cm xWS1 8 add cm moveto
                         5.1 Wk cos mul cm 5.1 Wk sin mul cm
                         lineto
%                       closepath
                         0.5 setlinewidth
                         0 setgray
                        stroke
                     } for
  grestore
%---------------------------------------------------------------------------------------------------------------------------------------------
%--------------- 1-Grad-Winkeleinteilung an der rechten Kathete ---------------------------------------------------------------
  gsave
     1 1 89 { /Wk exch def
                   {8 Wk sin Wk cos div 1 add div} /xWS1 exch def
                   {7.8 Wk sin Wk cos div 1 add div} /xWS2 exch def
                   newpath
                   xWS1 cm xWS1 neg 8 add cm moveto
                   xWS2 cm xWS2 neg 7.8 add cm
                  lineto
%                  closepath
                  0.5 setlinewidth
                  0 setgray
                  stroke
               } for
  grestore
%---------------------------------------------------------------------------------------------------------------------------------------------
%--------------- 5-Grad-Winkeleinteilung an der rechten Kathete ---------------------------------------------------------------
  gsave
      5 5 85 { /Wk exch def
                    {8 Wk sin Wk cos div 1 add div} /xWS1 exch def
                    {7.6 Wk sin Wk cos div 1 add div} /xWS2 exch def
                    newpath
                    xWS1 cm xWS1 neg 8 add cm moveto
                    xWS2 cm xWS2 neg 7.6 add cm
                   lineto
%                   closepath
                   0.5 setlinewidth
                   0 setgray
                   stroke
                } for
  grestore
%---------------------------------------------------------------------------------------------------------------------------------------------
%--------------- 10-Grad-Winkeleinteilung an der rechten Kathete --------------------------------------------------------------
  gsave
      10 10 80 { /Wk exch def
                        {8 Wk sin Wk cos div 1 add div} /xWS1 exch def
                        {5.1 Wk cos mul  Wk cos div 1 add div} /xWS2 exch def
                        newpath
                        xWS1 cm xWS1 neg 8 add cm moveto
                        5.1 Wk cos mul cm 5.1 Wk sin mul cm
                        lineto
%                        closepath
                        0.5 setlinewidth
                        0 setgray
                        stroke
                   } for
  grestore
%---------------------------------------------------------------------------------------------------------------------------------------------
%--------------- 1-Grad-Winkeleinteilung auf Halbkreis ------------ ---------------------------------------------------------------
  gsave
          5 1 175 { /Wk exch def
                           newpath
                           4.3 Wk cos mul cm 4.3 Wk sin mul cm moveto
                           4.45 Wk cos mul cm 4.45 Wk sin mul cm
                           lineto
%                           closepath
                           0.5 setlinewidth
                           0 setgray
                           stroke
                        } for
  grestore
%---------------------------------------------------------------------------------------------------------------------------------------------
%--------------- 5-Grad-Winkeleinteilung auf Halbkreis ------------ ---------------------------------------------------------------
  gsave
         5 5 175 { /Wk exch def
                          newpath
                         4.3 Wk cos mul cm 4.3 Wk sin mul cm moveto
                         4.5 Wk cos mul cm 4.5 Wk sin mul cm
                         lineto
%                         closepath
                         0.5 setlinewidth
                         0 setgray
                         stroke
                     } for
  grestore
%---------------------------------------------------------------------------------------------------------------------------------------------
%-------------------------- Linke Diagonallinie-dashed -------------------------------------------------------------------------------
  gsave
              newpath /Wk 135 def
               {8 Wk sin Wk cos div 1 sub div} /xWS1 exch def
               0 0 moveto
               4.3 Wk cos mul cm 4.3 Wk sin mul cm lineto
%               closepath
               gsave
                 [2 sqrt 2 div 0.24 sub cm 0.24 cm] -0.12 cm setdash
                 0.5 setlinewidth
                 0 setgray stroke
              grestore
  grestore
%---------------------------------------------------------------------------------------------------------------------------------------------
%-------------------------- Rechte Diagonallinie-dashed -----------------------------------------------------------------------------
 gsave
              newpath /Wk 45 def
               {8 Wk sin Wk cos div 1 add div} /xWS1 exch def
               0 0 moveto
               4.3 Wk cos mul cm 4.3 Wk sin mul cm lineto
%               closepath
             gsave
               [2 sqrt 2 div 0.24 sub cm 0.24 cm] -0.12 cm setdash
               0.5 setlinewidth
               0 setgray stroke
             grestore
 grestore
%---------------------------------------------------------------------------------------------------------------------------------------------
%---------------------------------------- Mittel-Linie -------------------------------------------------------------------------------------
  gsave
              newpath
              0 0.5 cm moveto
              0 3.5 cm lineto
              0 5.1 cm moveto
              0 6.0 cm lineto
              0 6.4 cm moveto
              0 8.0 cm lineto
%              closepath
              gsave
              0.5 setlinewidth
               0 setgray stroke
              grestore
  grestore
%--------------------------------------------------------------------------------------------------------------------------------------------------
%------------------------- Definition für horizontal-zentriertes - Setzen der cm- Zahlen (180° gedreht) ------------------------
  /cmshow {% (string)  x y
                      gsave
                            moveto 180 rotate  % s
                            /NimbusRomNo9L-Regu findfont 7 scalefont setfont
                            dup % s s
                            stringwidth pop 2 div neg 0 % s -dx/2
                            rmoveto 0.2 setgray show
                      grestore
                 } bind def
%-------------------------  Setzen der Zahlen für die cm-Einteilungen -----------------------------------------------------------------
      0 1 7 { /MZ exch def MZ 2 string cvs MZ cm 0.47 cm cmshow } for %---------------- Zahlen 0 bis 7 (nach rechts)
      1 1 7 { /MZ exch def MZ 2 string cvs MZ neg cm 0.47 cm cmshow } for %----------- Zahlen 1 bis 7 (nach links)
      1 1 3 { /MZ exch def MZ 2 string cvs 2 cm MZ 0.1 add cm cmshow } for %---------- Zahlen 1 bis 3 (nach oben, rechte Seite)
      1 1 3 { /MZ exch def MZ 2 string cvs -2 cm MZ 0.1 add cm cmshow } for %--------- Zahlen 1 bis 3 (nach oben, linke Seite)
%--------------------------------------------------------------------------------------------------------------------------------------------------
%------------------------- Definition für horizontal-zentriertes - Setzen der Winkel- Zahlen (mitgedreht) ----------------------
  /cmRotshow {% ----------------------------------   (string),  Winkel,   x  y
                             gsave
                                    moveto 90 add rotate  % s
                                    /NimbusRomNo9L-Regu findfont 7 scalefont setfont
                                    dup % s s
                                    stringwidth pop 2 div neg 0 % s -dx/2
                                    rmoveto 0.2 setgray show
                             grestore
                          } bind def
  gsave
%-------------------------  Setzen der Zahlen für die Winkel-Einteilungen -------------------------------------------------------------------------------------------------
      10 10 170 { /MZ exch def MZ 3 string cvs  MZ 5.03 cm MZ cos mul 5.03 cm MZ sin mul  cmRotshow } for % ------------------ äußerer Kreisbogen
      10 10 170 { /MZ exch def 180 MZ sub 3 string cvs  MZ 4.75 cm MZ cos mul 4.75 cm MZ sin mul  cmRotshow } for % ------- innerer Kreisbogen
 grestore
%---------------------------------------------------------------------------------------------------------------------------------------------------------------------------------------
 /TSshow {% (string)  x y
                         gsave
                         moveto 180 rotate  % s
                         \pst@PSfontO\space findfont \pst@fontsizeO scalefont setfont
                         dup % s s
                         stringwidth pop 2 div neg 0 % s -dx/2
                         rmoveto 0.4 setgray show
                         grestore
                      } bind def
   gsave
      (\pst@OwnerTxt) 0 cm 4 cm TSshow
   grestore
%--------------------------------------------------------------------------------------------------------------------------------------------------------------------------------------
 /MadeShow {% (string)  x y
                         gsave
                         moveto 180 rotate  % s
                         \pst@PSfontM\space findfont \pst@fontsizeM scalefont setfont
                         dup % s s
                         stringwidth pop 2 div neg 0 % s -dx/2
                         rmoveto 0.7 setgray show
                         grestore
                      } bind def
   gsave
      (\pst@MadeTxt) 0 cm 6.25 cm MadeShow
   grestore
%---------------------------------------- Fahne -------------------------------------------------------------------------------------
  gsave
           /Fahne {
                             newpath
                             0 0 moveto
                             0 1 rlineto
                             1 0 rlineto
                             0 -1 rlineto
                             closepath
                             setrgbcolor
                             [ /ca .5 /SetTransparency  pdfmark %    set  transparency
                              0.5 .setopacityalpha sysfill
                           } def
Country (France) eq {
      0.73 cm 6.1 cm translate
      3 6 scale
      1 0 0 Fahne
      1 0 translate
      1 1 1 Fahne
      1 0 translate
      0 0 1 Fahne
      } if
Country (Germany) eq {
      0.775 cm 6.22 cm translate
      7.5 1.5 scale
      1 0.588 0 Fahne
      0 -1 translate
      0.75 0.07 0.11 Fahne
      0 -1 translate
      0 0 0 Fahne
      } if
  grestore
%--------------------------------------------------------------------------------------------------------------------------------------------------
%--------------- Ende nur grestore von gsave ganz am Anfang ------------------
grestore
\fi
}%
\pnode(!
    /cm {28.45274 mul} def %  ----------------------------------------  Definiere cm als Längeneinheit mit der multipliziert wird
    \pst@tempA \tx@UserCoor /@y1 ED /@x1 ED % ----------  Definiere die Koordinaten des Ursprungs (zur Winkelberechnung benötigt)
    \pst@tempB \tx@UserCoor /@y2 ED /@x2 ED % ----------  Definiere die Koordinaten des zweiten Punktes (zur Winkelberechnung benötigt)
   @y1 @y2 sub @x1 @x2 sub Atan /@delta ED %------------   Winkelberechnung
    @x1 cm @y1 cm  T % ------------------------------------   Tranformiere den Koordinatenursprung in P
    @delta #1 add 180 add rotate % ------------------------------     Drehe den oberen Punkt des Geodreiecks
   \pst@ProScale\space \pst@ProScale\space scale % --------    Skalieren!
   0 8){GeodrC}%
\pnode(!
    @x1 cm @y1 cm  T % ------------------------------------   Tranformiere den Koordinatenursprung in P
    @delta #1 add 180 add rotate % ------------------------------     Drehe die linke untere  Ecke des Geodreiecks
   \pst@ProScale\space \pst@ProScale\space scale % --------    Skalieren!
   -8 0){GeodrA}%
\pnode(!
    @x1 cm @y1 cm  T % ------------------------------------   Tranformiere den Koordinatenursprung in P
    @delta #1 add 180 add rotate % ------------------------------     Drehe die rechte untere Ecke des Geodreiecks
   \pst@ProScale\space \pst@ProScale\space scale % --------    Skalieren!
   8 0){GeodrB}%
\endgroup%
\end@SpecialObj\ignorespaces
}%


\definecolor{HolzCol}{rgb}{1,0.937,0.835}%
\define@key[psset]{pst-geometrictools}{PenScale}[1]{\def\pst@PenScale{#1 }}%
\define@key[psset]{pst-geometrictools}{PenLength}[5]{\def\pst@PenLength{#1 }}%
\define@key[psset]{pst-geometrictools}{pencilColA}[red]{\pst@getcolor{#1}\pencilColA}
\define@key[psset]{pst-geometrictools}{pencilColB}[HolzCol]{\pst@getcolor{#1}\pencilColB}
\psset[pst-geometrictools]{PenLength=5,pencilColA=red,pencilColB=HolzCol,PenScale=1}
\def\psPencil{\def\pst@par{}\pst@object{psPencil}}
\def\psPencil@i#1(#2){%-------------------------------    #1 Winkel,  #2 Ursprung P
\begin@SpecialObj
\pst@killglue
\begingroup
\pnode(#2){@PGD}%
\use@par
\pst@getcoor{@PGD}\pst@tempA % --------------------------   Hohle die Koordinaten und speichere sie in \pst@tempA
\addto@pscode{
/sysfill {systemdict /fill get exec} def
%/Country (\pst@country) def
    /cm {28.45274 mul} def %  ----------------------------------------  Definiere cm als Längeneinheit mit der multipliziert wird
    /Plength \pst@PenLength def
%    \pst@tempA \tx@UserCoor /@y1 ED /@x1 ED % ----------  Definiere die Koordinaten des Ursprungs (zur Winkelberechnung benötigt)
    \pst@tempA \pst@coor T % ------------------------------------   Tranformiere den Koordinatenursprung in P
     #1 rotate % ------------------------------     Drehe das gesamte Geodreieck
     \pst@PenScale dup scale % --------    Skaliere das ganze Geodreieck
%-------------------------------------------------------------------------------------------------------------
gsave
%-------------------------------------------------------------------------------------------------------------
%-------------- Unterer Holzteil des Stifts -------------------------------------------------------------
gsave
     newpath
       0 cm 0 moveto
       -0.3 cm 1 cm rlineto
       0 cm 0.2 cm rlineto
       0.6 cm 0 cm rlineto
       0 cm -0.2 cm rlineto
        closepath
     gsave
       \pst@usecolor\pencilColB
       fill
    grestore
    gsave
    0.4 setlinewidth
    0.5 setgray
    1 setlinejoin
    stroke
    grestore
grestore
%-------------------------------------------------------------------------------------------------------------
%-------------- langer farbiger Teil des Stifts ----------------------------------------------------------
gsave
     newpath
       -0.3 cm 1 cm moveto
       0 cm  Plength cm rlineto
       0.6 cm 0 cm rlineto
       0 cm  Plength neg cm rlineto
       0.3 cm 1.13 cm 0.13 cm 270 180 arcn
       0 cm 1.13 cm 0.17 cm 0 180 arcn
       -0.3 cm 1.13 cm 0.13 cm 0 270 arcn
       closepath
     gsave
       \pst@usecolor\pencilColA
       fill
    grestore
     gsave
       0.5 setlinewidth
       0.1 setgray
      stroke
    grestore
grestore
%-------------------------------------------------------------------------------------------------------------
%--------------------------- Spitze in Farbe --------------------------------------------------------------
gsave
     newpath
       0 cm 0 moveto
       -0.1 cm 1 3 div cm rlineto
       0.2 cm 0 cm rlineto
     closepath
     gsave
       \pst@usecolor\pencilColA
       fill
    grestore
grestore
%-------------------------------------------------------------------------------------------------------------
%--------------------------- Striche im Stift --------------------------------------------------------------
gsave
     newpath
       -0.17 cm 1.13 cm moveto
       0 cm  Plength 0.13 sub cm rlineto
       0.17 cm 1.13 cm moveto
       0 cm  Plength 0.13 sub cm rlineto
       closepath
     gsave
      0.5 setlinewidth
      0.1 setgray
       stroke
    grestore
grestore
%-------------------------------------------------------------------------------------------------------------
%--------------------------- Elliptisches Stift-Ende in Holz --------------------------------------------
gsave
     newpath
       0 cm Plength 1 add cm T
       1 0.2 scale
       0 cm 0 cm 0.3 cm 0 360 arc
     closepath
     gsave
       \pst@usecolor\pencilColB
       fill
      grestore
       gsave
       0.5 setlinewidth
       0.3 setgray
      stroke
    grestore
grestore
%-------------------------------------------------------------------------------------------------------------
%--------------------------- Elliptisches Stift-Ende in Farbe -------------------------------------------
gsave
     newpath
       0 cm Plength 1 add cm T
       1 0.2 scale
       0 cm 0 cm 0.1 cm 0 360 arc
     closepath
     gsave
       \pst@usecolor\pencilColA
       fill
      grestore
       gsave
       0.3 setlinewidth
       0.3 setgray
       stroke
    grestore
grestore
%--------------------------------------------------------------------------------------------------------
%--------------------------------------------------------------------------------------------------------
grestore
}%
\endgroup
\end@SpecialObj\ignorespaces
}%


\define@key[psset]{pst-geometrictools}{RulerFillCol}[cyan!60]{\pst@getcolor{#1}\RulerFillCol}
\define@key[psset]{pst-geometrictools}{RulerScale}[1]{\def\pst@RulerScale{#1 }}%
\psset[pst-geometrictools]{RulerFillCol=gray,RulerScale=1}

\def\psRuler{\def\pst@par{}\pst@object{psRuler}}
\def\psRuler@i#1(#2){\@ifnextchar(%
{\psRuler@ii{#1}(#2)}%
{\psRuler@ii{#1}(#2)(#2)}}%
\def\psRuler@ii#1(#2)(#3){%-------------------------------    #1 Winkel,  #2 Ursprung P,  zweiter Punkt Q für Winkel
\begin@SpecialObj
\pst@killglue
\begingroup%
\pnode(#2){@PGD}%
\pnode(#3){@QGD}%
\use@par%
\pst@getcoor{@QGD}\pst@tempB % --------------------------   Hohle die Koordinaten und speichere sie in \pst@tempB
\pst@getcoor{@PGD}\pst@tempA % --------------------------   Hohle die Koordinaten und speichere sie in \pst@tempA
\addto@pscode{
/sysfill {systemdict /fill get exec} def
    /cm {28.45274 mul} def %  --------------------------------------- Definiere cm als Längeneinheit mit der multipliziert wird
    \pst@tempA \tx@UserCoor /@y1 ED /@x1 ED % ----------  Definiere die Koordinaten des Ursprungs (zur Winkelberechnung benötigt)
    \pst@tempB \tx@UserCoor /@y2 ED /@x2 ED % ----------  Definiere die Koordinaten des zweiten Punktes (zur Winkelberechnung benötigt)
   @y1 @y2 sub @x1 @x2 sub Atan /@delta ED %------------   Winkelberechnung
    \pst@tempA \pst@coor T % ------------------------------------  Tranformiere den Koordinatenursprung in P
    @delta #1 add 180 add rotate % ------------------------------    Drehe das gesamte Geodreieck
   \pst@RulerScale dup scale % -----------------------------------    Skaliere das ganze Geodreieck
%-------------------------------------------------------------------------------------------------------------
gsave
%-------------------------------------------------------------------------------------------------------------
%-------------- Füllung der beiden inneren Bereiche und äußere Begrenzung des Lineals -----------------------------------------------
gsave
     newpath
       -0.2 cm 0 moveto
       30.4 cm 0 rlineto
       0 cm -0.9 cm rlineto
       -30.4 cm 0 cm rlineto
       closepath
        gsave
         \pst@usecolor\RulerFillCol
       [ /ca .25 /SetTransparency  pdfmark %    set  transparency
       0.25 .setopacityalpha sysfill
    grestore
grestore
%-------------------------------------------------------------------------------------------------------------
gsave
     newpath
       -0.2 cm 0 moveto
       30.2 cm 0 lineto
       0 cm -3 cm rlineto
       -30.4 cm 0 cm rlineto
     closepath
       3.75 cm -1.5 cm moveto
       3.5 cm -1.5 cm 0.25 cm 0 360 arc
     closepath
    gsave
         \pst@usecolor\RulerFillCol
       [ /ca .15 /SetTransparency  pdfmark %    set  transparency
       0.15 .setopacityalpha sysfill
    grestore
         gsave
%         2 setlinejoin
       0.7 setlinewidth
       0.1 setgray
      stroke
    grestore
grestore
%-------------------------------------------------------------------------------------------------------------
%-------------- Millimetereinteilung auf der Grundseite -------------------------------------------
 /mmBlinie { newpath
                    0 cm 0 cm moveto
                    0 cm -0.25 cm rlineto
                     gsave
                     0.5 setlinewidth
                     0 setgray
                     stroke
                     grestore
                      0.1 cm 0 cm translate
                    } def
gsave
300 {mmBlinie} repeat
grestore
%-------------------------------------------------------------------------------------------------------------
%-------------- 0,5- cm-Einteilung auf der Grundseite ----------------------------------------------
   /cmBlinie { newpath
                   0.5 cm 0 cm moveto
                    0 cm -0.375 cm rlineto
%                     closepath
                     gsave
                     0.55 setlinewidth
                     0 setgray
                     stroke
                     grestore
                      1 cm 0 cm translate
                    } def
gsave
   30 {cmBlinie} repeat
grestore
%-------------- 1- cm-Einteilung auf der Grundseite ----------------------------------------------
   /cmBlinie { newpath
                   0 cm 0 cm moveto
                    0 cm -0.5 cm rlineto
%                     closepath
                     gsave
                     0.6 setlinewidth
                     0 setgray
                     stroke
                     grestore
                      1 cm 0 cm translate
                    } def
gsave
   31 {cmBlinie} repeat
grestore
%--------------------------------------------------------------------------------------------------------------------------------------------------
%------------------------- Definition für horizontal-zentriertes - Setzen der cm- Zahlen (180° gedreht) ------------------------
  /cmshow {% (string)  x y
                      gsave
                            moveto   % s
                            /NimbusRomNo9L-Regu findfont 8 scalefont setfont
                            dup % s s
                            stringwidth pop 2 div neg 0 % s -dx/2
                            rmoveto 0.2 setgray show
                      grestore
                 } bind def
%-------------------------  Setzen der Zahlen für die cm-Einteilungen -----------------------------------------------------------------
      0 1 30 { /MZ exch def MZ 2 string cvs MZ cm -0.78 cm cmshow } for %---------------- Zahlen 0 bis 30 (nach rechts)
%-------------------------------------------------------------------------------------------------------------
%--------------------------------------------------------------------------------------------------------------------------------------------------
%--------------- Ende nur grestore von gsave ganz am Anfang ------------------
grestore
}%
\endgroup%
\end@SpecialObj\ignorespaces
}%

\definecolor{PoCRed}{rgb}{0.7,0.134,0.134}
\define@boolkey[psset]{pst-geometrictools}[Pst@]{MCAngle}[true]{}
\define@key[psset]{pst-geometrictools}{PoCLength}[5]{\def\pst@PoCLength{#1 }}
\define@key[psset]{pst-geometrictools}{PoCAngle}[0]{\def\pst@PoCAngle{#1 }}
\define@key[psset]{pst-geometrictools}{PoCFillCol}[PoCRed]{\pst@getcolor{#1}\PoCFillCol}
\define@key[psset]{pst-geometrictools}{PoCMineCol}[black!80]{\pst@getcolor{#1}\PoCMineCol}
\define@key[psset]{pst-geometrictools}{PoCScale}[1]{\def\pst@PoCScale{#1 }}%
\define@key[psset]{pst-geometrictools}{RadVS}[RVS]{\def\pst@RadVS{#1}}%
\define@key[psset]{pst-geometrictools}{AngleVS}[AVS]{\def\pst@AngleVS{#1}}%
\define@key[psset]{pst-geometrictools}{RadMul}[1]{\def\pst@RadMul{#1 }}%
\psset[pst-geometrictools]{PoCScale=1,PoCFillCol=PoCRed,PoCAngle=0,PoCMineCol=black!80,PoCLength=5,MCAngle=true,RadVS=RVS,AngleVS=AVS,RadMul=1}

\def\psCompass{\def\pst@par{}\pst@object{psCompass}}%  Pair of Compasses
\def\psCompass@i#1(#2){\@ifnextchar(%
{\psCompass@ii{0}(#2)}%
{\psCompass@ii{#1}(#2)(#2)}}%
\def\psCompass@ii#1(#2)(#3){%-------------------------------    #1 Winkel,  #2 Ursprung P,  zweiter Punkt Q für Winkel
\begin@SpecialObj
\pst@killglue
\begingroup%
\pnode(#2){@PGD}%
\pnode(#3){@QGD}%
\use@par%
\pst@getcoor{@QGD}\pst@tempB % --------------------------   Hohle die Koordinaten und speichere sie in \pst@tempB
\pst@getcoor{@PGD}\pst@tempA % --------------------------   Hohle die Koordinaten und speichere sie in \pst@tempA
\addto@pscode{%
     \pst@RadMul /faktor ED
     \pst@PoCScale /skal ED
     #1 skal div 2 div /Radius ED
     skal dup scale % --------    Skaliere das ganze Geodreieck
    /sysfill {systemdict /fill get exec} def
    /SLg \pst@PoCLength def
    /cm {28.45274 mul} def %  ----------------------------------------  Definiere cm als Längeneinheit mit der multipliziert wird
    \pst@tempA\tx@UserCoor /@y1 ED /@x1 ED % ----------  Definiere die Koordinaten des Ursprungs (zur Winkelberechnung benötigt)
    \pst@tempB\tx@UserCoor /@y2 ED /@x2 ED % ----------  Definiere die Koordinaten des zweiten Punktes (zur Winkelberechnung benötigt)
    \pst@tempA\pst@coor /@y3 ED /@x3 ED
    @y1 @y2 sub dup mul @x1 @x2 sub dup mul add faktor dup mul mul /quadlg ED
     quadlg 0 eq { 180 } { @y1 @y2 sub @x1 @x2 sub atan } ifelse /@delta ED %------------   Winkelberechnung
     quadlg sqrt 2 div /RadiusPktB ED %------------   Radiusberechnung mit zweitem Punkt
     SLg RadiusPktB Radius add  gt { RadiusPktB Radius add } { SLg } ifelse  /RadPoC ED
    /hoehe SLg dup mul RadPoC dup mul sub sqrt  def
    /phiS RadPoC hoehe atan def
    @x3 @y3 T % -------------------------------------------------------------------------------    Transformiere den Koordinatenursprung in P
    \pst@PoCAngle\space \ifPst@MCAngle\space @delta add 180 add \fi rotate % ------------------------------     Drehe das gesamte Geodreieck
%-------------------------------------------------------------------------------------------------------------
gsave
%-------------------------------------------------------------------------------------------------------------
%-------------- Linke Zirkelspitze-----------------------------------------------     0 cm 0 cm  0.15 cm 0 360 arc
gsave
     newpath
     phiS neg rotate
       0 cm 0 cm moveto
       -0.1 cm 0.33 cm lineto
       0.07 cm 0 cm rlineto
       closepath
    gsave
         0.5 setgray
         fill
    grestore
grestore
%---------------------------------------- rechte Zirkel-Mine --------------------------------------
gsave
     newpath
     phiS neg rotate
       0 cm SLg cm T
        phiS 2 mul rotate
         0 cm SLg  neg cm moveto
         0.06 cm 0.1 cm rlineto
         0.013 cm 0.2 cm rlineto
        -0.06 cm 0 cm rlineto
       closepath
     gsave
       \pst@usecolor\PoCMineCol
         fill
    grestore
grestore
%---------------------------------------- Spindel --------------------------------------
gsave
     newpath
        phiS neg rotate
        0 cm SLg cm moveto
        0 cm SLg cm T
        phiS rotate
        -2 cm -1 phiS cos mul cm rmoveto
        4 cm 0 cm rlineto
    gsave
         0.5 setgray
                1.6 setlinewidth
         stroke
    grestore
grestore
gsave
     newpath
        phiS neg rotate
        0 cm SLg cm moveto
        0 cm SLg cm T
        phiS rotate
        -2 cm -1 phiS cos mul cm rmoveto
        0.2 -0.75 rmoveto
       189 { 0.5 1.5 rlineto
        0.1 -1.5 rmoveto } repeat
    gsave
         0.2 setgray
                0.2 setlinewidth
         stroke
    grestore
grestore
%---------------------------------------- Drehrad an der Spindel --------------------------------------
gsave
     newpath
        phiS neg rotate
        0 cm SLg cm moveto
        0 cm SLg cm T
        phiS rotate
        0 cm -1.3 phiS cos mul cm rmoveto
        0 cm 0.6 cm rlineto
    gsave
         0.5 setgray
                0.08 cm setlinewidth
         stroke
    grestore
grestore
gsave
gsave
     newpath
        phiS neg rotate
        0 cm SLg cm moveto
        0 cm SLg cm T
        phiS rotate
        -0.04 cm -1.28 phiS cos mul cm rmoveto
        20 { 0.08 cm 0 cm rlineto
        -0.08 cm 0.03 cm rmoveto } repeat
    gsave
         0.3 setgray
                0.2 setlinewidth
         stroke
    grestore
grestore
gsave
%---------------------------------------- Halterung zum Anfassen --------------------------------------
gsave
     newpath
        phiS neg rotate
        0 cm SLg cm moveto
        0 cm SLg cm T
        phiS rotate
        0 cm 0.75 0.54 phiS sin mul add cm moveto
        0 cm 0.2 cm rlineto
    gsave
         0.5 setgray
                3 setlinewidth
         stroke
    grestore
grestore
gsave
     newpath
        phiS neg rotate
        0 cm SLg cm moveto
        0 cm SLg cm T
        phiS rotate
        0 cm 0.95 0.54 phiS sin mul add cm moveto
        0 cm 0.05 cm rlineto
    gsave
         \pst@usecolor\PoCFillCol
                5 setlinewidth
         stroke
    grestore
grestore
gsave
     newpath
        phiS neg rotate
        0 cm SLg cm moveto
        0 cm SLg cm T
        phiS rotate
        0 cm 1.0 0.54 phiS sin mul add cm moveto
        0 cm 0.5 cm rlineto
    gsave
         0.5 setgray
                5 setlinewidth
         stroke
    grestore
grestore
gsave
     newpath
        phiS neg rotate
        0 cm SLg cm moveto
        0 cm SLg cm T
        phiS rotate
        -0.08 cm 1.0 0.6 phiS sin mul add cm moveto
        0 cm 0.48 cm rlineto
        6 { 0.025 cm -0.48 cm rmoveto
         0 cm 0.48 cm rlineto } repeat
    gsave
         0.65 setgray
                0.15 setlinewidth
         stroke
    grestore
grestore
gsave
     newpath
        phiS neg rotate
        0 cm SLg cm moveto
        0 cm SLg cm T
        phiS rotate
        -0.075 cm 1.0 0.6 phiS sin mul add cm moveto
        0 cm 0.48 cm rlineto
        6 { 0.025 cm -0.48 cm rmoveto
         0 cm 0.48 cm rlineto } repeat
    gsave
         0.25 setgray
                0.2 setlinewidth
         stroke
    grestore
grestore
%---------------------------------------- Kreis um oberen Teil --------------------------------------
gsave
     newpath
        phiS neg rotate
        0 cm SLg cm moveto
        0 cm SLg cm T
        phiS rotate
        0 cm 0.1 0.54 phiS sin mul add cm moveto
        0.65 -50 PtoC exch cm exch cm  rmoveto
        0 cm 0.1 0.54 phiS sin mul add cm  0.65 cm -50 230 arc
    gsave
         0.1 setgray
                1.3 setlinewidth
         stroke
    grestore
grestore
%---------------------------------------- Kreis zwischen den Schenkeln --------------------------------------
gsave
     newpath
        phiS neg rotate
        0 cm SLg cm moveto
        0 cm SLg cm  0.15 cm 0 360 arc
      closepath
    gsave
         0.1 setgray
         fill
    grestore
grestore
%---------------------------------------- Schenkel des Zirkels ----------------------------------------------------
gsave
     newpath
     phiS neg rotate
       0 cm 0.3 cm moveto
       -0.3 cm SLg 0.1 add cm lineto
       0 cm SLg cm T
        phiS 2 mul rotate
       0.3 cm 0.1 cm moveto
       0 cm SLg 0.3 sub neg cm lineto
%       closepath
     gsave
       0.285 cm setlinewidth
       2 setlinejoin
       \pst@usecolor\PoCFillCol
        stroke
    grestore
grestore
gsave
     newpath
     phiS neg rotate
       -0.17 cm SLg 0.6 mul cm moveto
       -0.25 cm  SLg 0.85 mul cm lineto
       0 cm SLg cm T
        phiS 2 mul rotate
       0.17 cm SLg -0.4 mul cm moveto
       0.25 cm  SLg -0.15 mul cm lineto
%       closepath
     gsave
       0.17 cm setlinewidth
       2 setlinejoin
       0.4 setgray
        stroke
    grestore
grestore
%---------------------------------------- Kreis an der linken Spitze --------------------------------------
gsave
     newpath
        phiS neg rotate
       -0.03 cm 0.8 cm moveto
        -0.03 cm 0.8 cm 0.175 cm 0 360 arc
      closepath
    gsave
         0.25 setgray
         fill
    grestore
grestore
gsave
     newpath
        phiS neg rotate
       -0.03 cm 0.8 cm moveto
        -0.03 cm 0.8 cm 0.025 cm 0 360 arc
      closepath
    gsave
         0.8 setgray
         fill
    grestore
grestore
%---------------------------------------- Kreis an der rechten Spitze --------------------------------------
gsave
     newpath
     phiS neg rotate
       0 cm SLg cm T
        phiS 2 mul rotate
               0.03 cm 0 cm T
         0 cm SLg 0.8 sub neg cm moveto
         0 cm SLg 0.8 sub cm neg 0.175 cm 0 360 arc
      closepath
    gsave
         0.25 setgray
         fill
    grestore
grestore
gsave
     newpath
     phiS neg rotate
       0 cm SLg cm T
        phiS 2 mul rotate
               0.03 cm 0 cm T
         0 cm SLg 0.8 sub neg cm moveto
         0 cm SLg 0.8 sub cm neg 0.025 cm 0 360 arc
      closepath
    gsave
         0.8 setgray
         fill
    grestore
grestore
%--------------------------------------------------------------------------------------------------------------------------------------------------
%--------------- Ende nur grestore von gsave ganz am Anfang ------------------
grestore
}%
\pnode(!
   \pst@tempA \tx@UserCoor /@@y1 ED /@@x1 ED
   \pst@tempB \tx@UserCoor  /@@y2 ED /@@x2 ED
    @@x1 @@x2 sub @@y1 @@y2 sub Pyth \pst@RadMul\space mul /\pst@RadVS\space exch def
    @@y1 @@y2 sub dup mul @@x1 @@x2 sub dup mul add \pst@RadMul dup mul mul /quadlg ED
    quadlg 0 eq { 180 } { @@y1 @@y2 sub @@x1 @@x2 sub atan } ifelse /\pst@AngleVS\space exch def  %------------   Winkelberechnung /\pst@AngleVS\space exch def
  0 0){MoneyforNothing}%
\endgroup%
\end@SpecialObj\ignorespaces%
}%

%\define@boolkey[psset]{pst-geometrictools}[Pst@]{Ruler}[true]{}
%\define@boolkey[psset]{pst-geometrictools}[Pst@]{Ppoint}[false]{}
%\define@boolkey[psset]{pst-geometrictools}[Pst@]{LineOffset}[false]{}
%\define@key[psset]{pst-geometrictools}{AddAngle}[0]{\def\pst@AddAngle{#1 }}%
%\psset[pst-geometrictools]{Ruler=true,Ppoint=false,AddAngle=0,LineOffset=false}

%% Distance between two points => #3
\def\psDistAB(#1)(#2)#3{%
   \pst@getcoor{#1}\pst@tempA
   \pst@getcoor{#2}\pst@tempB
\pnode(!
   \pst@tempA \tx@UserCoor
   \pst@tempB \tx@UserCoor
  3 -1 roll sub dup mul 3 1 roll sub dup mul add sqrt
  /#3 ED
  0 0
  ){XX}
}
%% calcule l'angle entre Ox et AB et le place en #3
\def\psAngleAB(#1)(#2)#3{%
   \pst@getcoor{#1}\pst@tempA
   \pst@getcoor{#2}\pst@tempB
\pnode(!
   \pst@tempA \tx@UserCoor
   \pst@tempB \tx@UserCoor
   3 -1 roll sub 3 1 roll exch sub Atan
  /#3 ED
  0 0
  ){YY}
}%
%% angle defined by three points
\def\psAngleAOB(#1)(#2)(#3)#4{%
   \pst@getcoor{#1}\pst@tempA
   \pst@getcoor{#2}\pst@tempB
   \pst@getcoor{#2}\pst@tempC
\pnode(!
   \pst@tempA \tx@UserCoor
   \pst@tempB \tx@UserCoor
   \pst@tempC \tx@UserCoor
  4 copy exch 4 -1 roll mul
  3 -2 roll mul add
  4 -2 roll mul 4 -2 roll mul sub exch Atan
   /#4 ED
   0 0
   ){ZZ}
}

% la nouvelle version de \psParallels
\define@key[psset]{pst-geometrictools}{DistCoeff}[1]{\def\pst@DistCoeff{#1 }}%
\psset[pst-geometrictools]{DistCoeff=1}
%
\def\psParallels{\def\pst@par{}\pst@object{psParallels}}
\def\psParallels@i(#1)(#2)(#3){%
\begin@SpecialObj %
\pnode(#1){pointA}%
\pnode(#2){pointB}%
\pnode(#3){pointC}%
\pcline[nodesep=-3](pointA)(pointB)
\psdots(pointA)(pointB)(pointC)
\uput[l](pointA){A}
\uput[l](pointB){B}
\uput[r](pointC){C}
%
\psProtractor[Ghost]{0}(pointA)%
\psAngleAB(pointA)(pointB){alpha1}
\psAngleAB(pointA)(GeodrA){alpha2}
\psProtractor[Ghost]{alpha1 alpha2 sub}(pointA)%
\psRuler{0}(GeodrB)(GeodrC)
%
\pstProjection[PointName={},PointSymbol=none]{pointA}{pointB}{pointC}[pointI]%
\pstRotation[RotAngle=45,PointSymbol=none,PointName={}]{pointC}{pointI}[pointJ]%
\pstInterLL[PointSymbol=none,PointName={}]{pointA}{pointB}{pointC}{pointJ}{pointK}%
\pstTranslation[PointSymbol=none,PointName={},DistCoef=\pst@DistCoeff]{pointK}{pointC}{pointA}[pointA*]
\psProtractor{alpha1 alpha2 sub}(pointA*)%
\end@SpecialObj\ignorespaces%
}%

%\define@boolkey[psset]{pst-geometrictools}[Pst@]{Ruler}[true]{}
%\define@boolkey[psset]{pst-geometrictools}[Pst@]{Ppoint}[false]{}
%\define@boolkey[psset]{pst-geometrictools}[Pst@]{LineOffset}[false]{}
%\define@key[psset]{pst-geometrictools}{AddAngle}[1]{\def\pst@AddAngle{#1 }}%
%\psset[pst-geometrictools]{Ruler=true,Ppoint=false,AddAngle=0,LineOffset=false}
%\def\psParallels{\def\pst@par{}\pst@object{psParallels}}
%\def\psParallels@i(#1)(#2)(#3){%
%\psset{saveNodeCoors,NodeCoorPrefix=n}%
%\@ifundefined{rCounter}{\def\rCounter{1}}{}%
%\begin@SpecialObj %
%\pnode(#1){pointA}%
%\pnode(#2){pointB}%
%\pnode(#3){pointC}%
%\pstProjection[PointName={},PointSymbol=none]{pointA}{pointB}{pointC}[pointI]%
%\midAB(pointA)(pointB){pointM}%
%\pstRotation[RotAngle=45,PointSymbol=none,PointName={}]{pointC}{pointI}[pointJ]%
%\pstInterLL[PointSymbol=none,PointName={}]{pointA}{pointB}{pointC}{pointJ}{pointK}%
%\AtoB(pointI)(pointC){IC}%
%\AtoB(pointK)(pointC){KC}%
%\AplusB(pointM)(KC){MT}%
%\AplusB(pointB)(KC){BT}%
%\psProtractor[Ghost]{0}(pointM)(pointB)%
%\pnode(GeodrB){ProB}%
%\pnode(GeodrC){ProC}%
%\psLDNode(ProB)(ProC){-1cm}{pointE}%
%\ifPst@Ruler\psRuler{0}(pointE)(GeodrC)\fi%
%\AplusB(ProB)(KC){ProB*}%
%\pst@Verb{ nKCx nKCy Pyth /laenge exch def %
%nKCx laenge div \pst@number\pslinewidth\space 2 div 0.35 \pst@ProScale mul add mul 28.45274 div 2 sqrt mul /delX exch def %
%nKCy laenge div \pst@number\pslinewidth\space 2 div 0.35 \pst@ProScale mul add mul 28.45274 div 2 sqrt mul /delY exch def
%}%
%\pnode(!nProB*x delX add nProB*y delY add ){GProB*}%
%\pnode(!/AddA \pst@AddAngle def nProB*x  AddA 0 eq { delX add } { AddA 180 eq { delX sub } { 0 add } ifelse } ifelse nProB*y AddA 0 eq { delY add } { AddA 180 eq { delY sub } { 0 add } ifelse } ifelse ){GProBC*}%
%\pnode(!nMTx  delX add nMTy  delY add){GMT*}%
%\pnode(!/AddA \pst@AddAngle def npointCx  AddA 0 eq { delX add } { AddA 180 eq { delX sub } { 0 add } ifelse } ifelse npointCy AddA 0 eq { delY add } { AddA 180 eq { delY sub } { 0 add } ifelse } ifelse ){pointC*}%
%\psLNode(pointM)(GMT*){\rCounter}{MPro}%
%\psLNode(ProB)(GProB*){\rCounter}{BPro}%
%\ifPst@Ppoint
%\ifPst@LineOffset
%\pst@Verb{ nICx nICy Pyth /laenge exch def %
%nICx laenge div \pst@number\pslinewidth\space 2 div 0.35 \pst@ProScale mul add mul 28.45274 div 2 sqrt mul /delX exch def %
%nICy laenge div \pst@number\pslinewidth\space 2 div 0.35 \pst@ProScale mul add mul 28.45274 div 2 sqrt mul /delY exch def %
%}%
%\pnode(!/AddA \pst@AddAngle def nProB*x  AddA 0 eq { delX add } { AddA 180 eq { delX sub } { 0 add } ifelse } ifelse nProB*y AddA 0 eq { delY add } { AddA 180 eq { delY sub } { 0 add } ifelse } ifelse ){GProBC*}%
%\pnode(!/AddA \pst@AddAngle def npointCx  AddA 0 eq { delX add } { AddA 180 eq { delX sub } { 0 add } ifelse } ifelse npointCy AddA 0 eq { delY add } { AddA 180 eq { delY sub } { 0 add } ifelse } ifelse ){pointC*}%
%\psProtractor{\pst@AddAngle}(pointC*)(GProBC*)\else%
%\psProtractor{\pst@AddAngle}(pointC)(ProB*)\fi%
%\else
%\psProtractor{\pst@AddAngle}(MPro)(BPro)\fi % Um die Liniendicke 0.8pt nach rechts versetzt
%\psLNode(pointC)(ProB*){0.98}{ProB**}%
%\psLNode(pointC)(ProB**){2}{C*}%
%\end@SpecialObj\ignorespaces%
%}%

\catcode`\@=\PstAtCode\relax
\endinput 