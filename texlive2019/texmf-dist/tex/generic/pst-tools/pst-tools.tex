%% $Id: pst-tools.tex 674 2017-12-02 22:04:53Z herbert $
%%
%% This is file `pst-tools.tex',
%%
%% IMPORTANT NOTICE:
%%
%% Package `pst-tools.tex'
%%
%% Herbert Voss <hvoss@tug.org>
%%
%% This program can be redistributed and/or modified under the terms
%% of the LaTeX Project Public License Distributed from CTAN archives
%% in directory macros/latex/base/lppl.txt.
%%
%% DESCRIPTION:
%%   `pst-tools' is a PSTricks package for helper functions
%%
%%
\csname PSTtoolsLoaded\endcsname
\let\PSTtoolsLoaded\endinput

\ifx\PSTricksLoaded\endinput\else\input pstricks.tex\fi
\ifx\PSTXKeyLoaded\endinput\else \input pst-xkey.tex \fi
%
\edef\PstAtCode{\the\catcode`\@} \catcode`\@=11\relax
% interface to the `xkeyval' package
\pst@addfams{pst-tools}
%
\def\fileversion{0.09b}
\def\filedate{2017/12/03}
\message{`PST-tools' v\fileversion, \filedate\space (hv)}
%
\define@key[psset]{pst-tools}{decimalSeparator}[.]{\def\psk@decimalSeparator{#1}}%
\define@boolkey[psset]{pst-tools}[Pst@]{comma}[true]{%
  \ifPst@comma\def\psk@decimalSeparator{,}\else\def\psk@decimalSeparator{.}\fi}
\define@boolkey[psset]{pst-tools}[Pst@]{trimSpaces}[true]{}
\define@key[psset]{pst-tools}{xShift}[0]{\def\psk@xShift{#1}}
\define@key[psset]{pst-tools}{yShift}[0]{\def\psk@yShift{#1}}
\define@key[psset]{pst-tools}{postString}[{}]{\def\psk@postString{#1}}
\define@key[psset]{pst-tools}{VarName}[]{\def\psk@VarName{#1 }}
\define@key[psset]{pst-tools}{PSfont}[NimbusRomNo9L-Regu]{\def\psk@PSfont{/#1 }}
\define@key[psset]{pst-tools}{valuewidth}[10]{\pst@getint{#1}\psk@valuewidth }
\define@key[psset]{pst-tools}{fontscale}[10]{\pst@checknum{#1}\psk@fontscale }
\define@key[psset]{pst-tools}{decimals}[-1]{\pst@getint{#1}\psk@decimals }
\psset[pst-tools]{PSfont=NimbusRomNo9L-Regu,fontscale=10,valuewidth=10,decimals=-1,
                  xShift=0,yShift=0,trimSpaces=false,postString={},VarName={},
                  comma=false,decimalSeparator=.}
%
\def\txG{ true setglobal globaldict begin }
\def\etxG{ end false setglobal }
%
\newpsstyle{mmpaper}{subgriddiv=5,gridlabels=0,gridwidth=1pt,gridcolor=orange,subgridwidth=0.1pt,subgridcolor=orange!90}
%
\def\psPrintValue{\pst@object{psPrintValue}}
\def\psPrintValue@i#1{\expandafter\psPrintValue@ii#1,,\@nil}
\def\psPrintValue@ii#1,#2,#3\@nil{%  #1,#2 only for algebraic code
  \begin@SpecialObj
  \addto@pscode{  
     gsave \psk@PSfont findfont \psk@fontscale scalefont setfont 
     \ifPst@algebraic 
       /x #1 def 
       /Func (#2) tx@AlgToPs begin AlgToPs end cvx def 
       Func 
     \else #1 \fi
     \ifx\relax\psk@VarName\relax\else dup \txG 10 string cvs /\psk@VarName\space exch def \etxG \fi
     \psk@decimals -1 gt { 10 \psk@decimals exp dup 3 1 roll mul cvi exch div } if
     \psk@valuewidth string cvs %/Output exch def % save output
%/concatstringarray{  %  [(a) (b) ... (z)] --> (ab...z)  20100422
     \ifx\relax\psk@postString\relax\else [ exch (\psk@postString) ] concatstringarray \fi
     \ifPst@trimSpaces rightTrim \fi
     \ifPst@comma dot2comma \fi        % do we have to change dot to comma
     \psk@xShift\space \psk@yShift\space moveto  %Output 
     show grestore
  }%
  \end@SpecialObj%
}

\define@boolkey[psset]{pst-tools}[Pst@]{round}[true]{}%
\define@boolkey[psset]{pst-tools}[Pst@]{science}[true]{%
  \ifPst@science\def\psk@Scin{true }\else\def\psk@Scin{false }\fi}
\psset[pst-tools]{science=false,round=false}
%
\def\psPrintValueNew{\pst@object{psPrintValueNew}}
\def\psPrintValueNew@i#1{\expandafter\psPrintValueNew@ii#1,,!!}
\def\psPrintValueNew@ii#1,#2,#3!!{%  #1,#2 only for algebraic code
  \begin@SpecialObj
  \addto@pscode{  %		thanks to Buddy Ledger
     /mfont { \psk@PSfont findfont \psk@fontscale scalefont setfont } bind def
     /mfontexp { \psk@PSfont findfont \psk@fontscale 1.2 div scalefont setfont } bind def
     /s1 { /StandardSymL findfont \psk@fontscale scalefont setfont } bind def
     \ifPst@algebraic 
        /x #1 def
        /Func (#2) tx@AlgToPs begin AlgToPs end cvx def 
        Func  
     \else #1 \fi
     /value exch def
     \psk@Scin {
       value 0 ne { value log floor cvi /expon exch def }{ /expon 0 def } ifelse
       value 10 expon exp div 
       \psk@decimals -1 gt { 10  \psk@decimals exp dup 3 1 roll mul 
         \ifPst@round round \else cvi \fi  exch div } if
       \psk@decimals 0 eq { cvi } if /numb exch def
       expon \psk@valuewidth string cvs /expon exch def
       numb \psk@valuewidth string cvs 
       \ifPst@comma dot2comma \fi  % do we have to change dot to comma
       /Output exch def
       /txspc \psk@fontscale 4 div def
       \psk@xShift\space 0 moveto mfont Output show
       txspc 0 rmoveto s1 (\string\264) show
       txspc 0 rmoveto mfont (10) show
       txspc 2 div txspc 1.5 mul rmoveto mfontexp expon show }
     { value
       \psk@decimals -1 gt { 10 \psk@decimals exp dup 3 1 roll mul 
         \ifPst@round round \else cvi \fi exch div } if
       \psk@decimals 0 eq { cvi } if %inserted to handle decimals=0
       \psk@valuewidth string cvs 
       \ifPst@comma dot2comma \fi         % do we have to change dot to comma
       \psk@xShift\space 0 moveto mfont %Output 
       show
     } ifelse
  }%
  \end@SpecialObj
}
%
\def\psRegisterList#1#2{%
  \@namedef{#1}##1{\@nameuse{#1##1}}%
  \pst@cnto=1
  \@for\pst@tempA:=#2\do{%
    \edef\pst@tempB{\@namedef{#1\the\pst@cnto}{\pst@tempA}}%	
    \pst@tempB
    \advance\pst@cnto by 1
  }%
}

\catcode`\@=\PstAtCode\relax
%
%% END: pst-tools.tex
\endinput
%
