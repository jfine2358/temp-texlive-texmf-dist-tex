%% $Id: pst-arrow.tex 328 2016-09-01 19:00:31Z herbert $
%%
%% This is file `pst-arrow.tex',
%%
%% IMPORTANT NOTICE:
%%
%% Package `pst-arrow.tex'
%%
%% Herbert Voss <hvoss@tug.org>
%%
%% This program can be redistributed and/or modified under the terms
%% of the LaTeX Project Public License Distributed from CTAN archives
%% in directory macros/latex/base/lppl.txt.
%%
%% DESCRIPTION:
%%   `pst-arrow' is a PSTricks package for additional arrows 
%%
\csname PSTarrowLoaded\endcsname
\let\PSTarrowLoaded\endinput
%
% Requires some packages
\ifx\PSTricksLoaded\endinput\else\input pstricks \fi
\ifx\PSTXKeyLoaded\endinput\else \input pst-xkey \fi
%
\def\fileversion{0.01}
\def\filedate{2016/09/01}
\message{`pst-arrow' v\fileversion, \filedate\space (dr,hv)}
%
\edef\PstAtCode{\the\catcode`\@} \catcode`\@=11\relax
\pst@addfams{pst-arrow}
%
\def\psBigArrow{\pst@object{psBigArrow}}
\def\psBigArrow@i(#1)(#2){%
  \addbefore@par{doublesep=1cm}
  \begin@ClosedObj
  \pssetlength\pst@dimm{\psdoublesep}
  \pst@getcoor{#1}\pst@tempA 
  \pst@getcoor{#2}\pst@tempB  
  \addto@pscode{
    /Width \pst@number\pst@dimm def
    \pst@tempA % x y 
    \pst@tempB % x y
    exch       % x y y x
    4 -1 roll   % y y x x
    sub        % y y dx
    3 1 roll   % dx y y
    sub        % dx dy
    exch       % dy dx
    atan neg      % alpha
    \pst@tempA
    translate
    rotate     
    0 0 moveto
    0 Width 2 div rlineto % |
    \pst@tempB \pst@tempA Pyth2 Width 1.5 mul sub 0 rlineto
    0 Width 1.5 div rlineto
    Width 1.5 mul dup neg rlineto
    Width 1.5 mul neg dup rlineto
    0 Width 1.5 div rlineto
    \pst@tempB \pst@tempA Pyth2 neg Width 1.5 mul add 0 rlineto
    closepath 
  }%
  \end@ClosedObj
}
%  the original table
% \def\pst@arrowtable{,<->,<<->>,>-<,>>-<<,(-),[-],)-(,]-[,|>-<|}
%
% v : Vee arrow (inside)                 v,V,f and F by Christophe FOUREY
% V : Vee arrow (outside)
% f : Filled vee arrow (inside)
% F : Filled vee arrow (outside)
\edef\pst@arrowtable{\pst@arrowtable,v-v,V-V,f-f,F-F,t-t,T-T}

% Vee arrow
\define@key[psset]{pst-arrow}{veearrowlength}[3mm]{\pst@getlength{#1}\psk@veearrowlength}
\psset[pst-arrow]{veearrowlength=3mm} % default projected length
\define@key[psset]{pst-arrow}{veearrowangle}[30]{\pst@getangle{#1}\psk@veearrowangle}
\psset[pst-arrow]{veearrowangle=30} % default angle
\define@key[psset]{pst-arrow}{veearrowlinewidth}[0.35mm]{\pst@getlength{#1}\psk@veearrowlinewidth}
\psset[pst-arrow]{veearrowlinewidth=0.35mm} % default vee arrow line width

% Filled vee arrow
\define@key[psset]{pst-arrow}{filledveearrowlength}[3mm]{\pst@getlength{#1}\psk@filledveearrowlength}
\psset[pst-arrow]{filledveearrowlength=3mm} % default projected length
\define@key[psset]{pst-arrow}{filledveearrowangle}[15]{\pst@getangle{#1}\psk@filledveearrowangle}
\psset[pst-arrow]{filledveearrowangle=15} % default angle
\define@key[psset]{pst-arrow}{filledveearrowlinewidth}[0.35mm]{\pst@getlength{#1}\psk@filledveearrowlinewidth}
\psset[pst-arrow]{filledveearrowlinewidth=0.35mm} % default vee arrow line width
\define@key[psset]{pst-arrow}{arrowlinestyle}[solid]{%
  \@ifundefined{psls@#1}%
    {\@pstrickserr{Line style `#1' not defined}\@eha}%
    {\def\psarrowlinestyle{#1}}}
\psset[pst-arrow]{arrowlinestyle=solid} % default
\pst@def{VeeArrow}<%
    1 setlinecap            % round caps
    1 setlinejoin            % round join
    setlinewidth            % vee arrow line width
    /y ED                % projected length
    2 div /a ED                % angle (divide by 2)
    /t ED                % false = inside, true = outside
    a sin a cos div y mul /x ED        % perpendicular length : x=tan(a).y
    t { 1 -1 scale } if            % if outside : symmetry
    x neg y moveto            % point #1
    0 0 L                % point #2
    x y L                % point #3
    { closepath gsave fill grestore } if    % if filled : close and fill
    \@nameuse{psls@\psarrowlinestyle}
    stroke                % draw line
    0 t { y 2 mul } { 0 } ifelse moveto>    % if outside : twice longer line

% VeeArrow : filled?   outside?   (total) angle   (projected) length   (arrow) line width

\@namedef{psas@v}{%
  false false \psk@veearrowangle \psk@veearrowlength \psk@veearrowlinewidth \tx@VeeArrow}
\@namedef{psas@V}{%
  false true \psk@veearrowangle \psk@veearrowlength \psk@veearrowlinewidth \tx@VeeArrow}
\@namedef{psas@f}{%
  true false \psk@filledveearrowangle \psk@filledveearrowlength \psk@filledveearrowlinewidth \tx@VeeArrow}
\@namedef{psas@F}{%
  true true \psk@filledveearrowangle \psk@filledveearrowlength \psk@filledveearrowlinewidth \tx@VeeArrow}

% And An another arrowhead
% architectural tick / oblique arrow

% Tick arrow
\define@key[psset]{pst-arrow}{tickarrowlength}[1.5mm]{\pst@getlength{#1}\psk@tickarrowlength}
\psset[pst-arrow]{tickarrowlength=1.5mm} % default projected length
\define@key[psset]{pst-arrow}{tickarrowlinewidth}[0.35mm]{\pst@getlength{#1}\psk@tickarrowlinewidth}
\psset[pst-arrow]{tickarrowlinewidth=0.35mm} % default tick arrow line width

\pst@def{TickArrow}<%
    1 setlinecap            % round caps
    1 setlinejoin            % round join
    setlinewidth            % tick line width
    /y ED                % projected length
    /t ED                % false = normal, true = reversed
    t { 1 -1 scale } if            % if reversed : symmetry
    y neg y moveto            % point #1
    y y neg L                % point #2
    \@nameuse{psls@\psarrowlinestyle}
    stroke                % draw line
    0 0 moveto>                % origin

\@namedef{psas@t}{ false \psk@tickarrowlength \psk@tickarrowlinewidth \tx@TickArrow }
\@namedef{psas@T}{ true \psk@tickarrowlength \psk@tickarrowlinewidth \tx@TickArrow }
%
% HookLeft/RightArrow
\newdimen\pshooklength
\newdimen\pshookwidth
\define@key[psset]{pst-arrow}{hooklength}[3mm]{\pssetlength\pshooklength{#1}}
\define@key[psset]{pst-arrow}{hookwidth}[1mm]{\pssetlength\pshookwidth{#1}}
%\psset{hooklength=3mm,hookwidth=1mm}
%
\edef\pst@arrowtable{\pst@arrowtable,H-H,h-h} % add new arrow
\def\tx@RHook{RHook }         % PostScript name
\def\tx@Rhook{Rhook }         % PostScript name
\@namedef{psas@H}{%
  /RHook {
    /x ED                     % hook width
    /y ED                     % hook length 
    /z CLW 2 div def          % save it
    x y moveto                % goto first point
    x 0 0 0 0 y 
    curveto                   % draw Bezier
    stroke 
    0 y moveto                % define current point
  } def
  \pst@number\pshooklength
  \pst@number\pshookwidth
  \tx@RHook 
}
\@namedef{psas@h}{%
  /Rhook {
    CLW mul 			% size * CLW
    add dup 			% +length  size*CLW+length size*CLW+length 
    2 div /w ED	 		% (size*CLW+length)/2  -> w 
    mul dup /h ED mul 		% (size*CLW+length)
    /a ED  
    w neg h abs moveto 0 0 L 
    gsave 
    stroke grestore 
  } def
  0 \psk@arrowlength \psk@arrowsize \tx@Rhook 
}
% New parameter "arrowfill", with default as "true"
\define@boolkey[psset]{pst-arrow}[ps]{ArrowFill}[true]{}
%
% Modification of the PostScript macro Arrow to choose to fill or not the arrow
% (it require to restore the current linewidth, despite of the scaling)
\pst@def{Arrow}<{%
    CLW mul add dup 2 div
    /w ED mul dup
    /h ED mul
    /a ED { 0 h T 1 -1 scale } if
    gsave
    \ifpsArrowFill\else\pst@number\pslinewidth \pst@arrowscale\space div SLW \fi
    w neg h moveto
    0 0 L w h L w neg a neg rlineto
    \ifpsArrowFill gsave 
       \tx@setStrokeTransparency
       fill  
       grestore \else gsave closepath stroke grestore \fi
    grestore
    0 h a sub moveto
}>
%
\define@key[psset]{pst-arrow}{nArrowsA}[2]{\def\psk@nArrowsA{#1}}
\define@key[psset]{pst-arrow}{nArrowsB}[2]{\def\psk@nArrowsB{#1}}
\define@key[psset]{pst-arrow}{nArrows}[2]{\def\psk@nArrowsA{#1}\def\psk@nArrowsB{#1}}
\psset{nArrows=2}
%
\@namedef{psas@>>}{%
    \psk@nArrowsA\space 1 sub {
      false \psk@arrowinset \psk@arrowlength \psk@arrowsize \tx@Arrow
      0 h a sub T
    } repeat
    gsave
    newpath
    false \psk@arrowinset \psk@arrowlength \psk@arrowsize \tx@Arrow
    CP
    grestore
    moveto
}
%
\@namedef{psas@<<}{%
    true \psk@arrowinset \psk@arrowlength \psk@arrowsize \tx@Arrow
    0 h neg a add T
  \psk@nArrowsB\space 2 sub {
    false \psk@arrowinset \psk@arrowlength \psk@arrowsize \tx@Arrow
    0 h neg a add T
  } repeat
  false \psk@arrowinset \psk@arrowlength \psk@arrowsize \tx@Arrow
  0 h a 5 mul 2 div sub moveto
}
%
% DG addition begin - Dec. 18/19, 1997 and Oct. 11, 2002
% Adapted from \psset@arrows
\define@key[psset]{pst-arrow}{ArrowInside}{%
  \def\pst@tempArrow{#1}%
  \ifx\pst@tempArrow\@empty \def\psk@ArrowInside{} %
  \else%
    \begingroup%
      \pst@activearrows%
      \xdef\pst@tempg{<#1}%
    \endgroup%
    \expandafter\psset@@ArrowInside\pst@tempg\@empty-\@empty\@nil%
    \if@pst\else\@pstrickserr{Bad intermediate arrow specification: #1}\@ehpa\fi%
  \fi%
}
% Adapted from \psset@@arrows
\def\psset@@ArrowInside#1-#2\@empty#3\@nil{%
  \@psttrue
  \def\next##1,#1-##2,##3\@nil{\def\pst@tempg{##2}}%
  \expandafter\next\pst@arrowtable,#1-#1,\@nil
  \@ifundefined{psas@#2}%
    {\@pstfalse\def\psk@ArrowInside{}}%
    {\def\psk@ArrowInside{#2}}%
}
% Default value empty
\psset{ArrowInside={}}
% Modified version of \pst@addarrowdef
\def\pst@addarrowdef{%
  \addto@pscode{%
    /ArrowA {
      \ifx\psk@arrowA\@empty
        \pst@oplineto
      \else
	\pst@arrowdef{A}
	moveto
      \fi
    } def
    /ArrowB { \ifx\psk@arrowB\@empty \else \pst@arrowdef{B} \fi } def
% DG addition
    /ArrowInside { 
      \ifx\psk@ArrowInside\@empty \else \pst@arrowdefA{Inside} \fi 
    } def
  }%
}
% Adapted from \pst@arrowdef
\def\pst@arrowdefA#1{%
  \ifnum\pst@repeatarrowsflag>\z@ /Arrow#1c [ 6 2 roll ] cvx def Arrow#1c\fi 
  \tx@BeginArrow
  \psk@arrowscale
  \@nameuse{psas@\@nameuse{psk@Arrow#1}}
  \tx@EndArrow%
}
% ArrowInsidePos parameter (default value 0.5)
\define@key[psset]{pst-arrow}{ArrowInsidePos}[0.5]{\pst@checknum{#1}\psk@ArrowInsidePos}%
%\psset{ArrowInsidePos=0.5}
%
%
% Redefinition of the PostScript /Line macro to print the intermediate
% arrow on each segment of the line
%
\define@key[psset]{pst-arrow}{ArrowInsideNo}[1]{\pst@checknum{#1}\psk@ArrowInsideNo}% hv 20031001
\define@key[psset]{pst-arrow}{ArrowInsideOffset}[0]{\pst@checknum{#1}\psk@ArrowInsideOffset}% hv 20031001
%\psset{ArrowInsideNo=1,ArrowInsideOffset=0}
%
\def\arrowType@H{H}
\pst@def{Line}<
  NArray n 0 eq not { n 1 eq { 0 0 /n 2 def } if
  (\psk@ArrowInside) length 0 gt { 
    \ifx\psk@arrowA\arrowType@H   % do we have a Hook arrow at the beginning?
      \pst@number\pshooklength  % yes 
    \else
      \psk@arrowsize\space CLW mul add dup \psk@arrowlength\space mul exch \psk@arrowinset mul neg add  
    \fi
    /arrowlength exch def 
    4 copy 				% copy all four values for the arrow line
    /y1 ED /x1 ED /y2 ED /x2 ED 	% save them
    /Alpha y2 y1 sub x2 x1 sub Atan def % the gradient of the line
%    2 copy /y1 ED /x1 ED ArrowA x1 y1  
    ArrowA 				% draw arrowA
    x1 Alpha cos arrowlength mul add	% dx add
    y1 Alpha sin arrowlength mul add	% dy add, to get the current point at the end of the arrow tip
    /n n 1 sub def
    n {
      4 copy
      /y1 ED /x1 ED /y2 ED /x2 ED
      x1 y1
      \psk@ArrowInsidePos\space 1 gt {
        /Alpha y2 y1 sub x2 x1 sub Atan def
        /ArrowPos \psk@ArrowInsideOffset\space def
        /dArrowPos \psk@ArrowInsidePos\space abs def
%        /Length x2 x1 sub y2 y1 sub Pyth def
        \psk@ArrowInsideNo\space cvi {
          /ArrowPos ArrowPos dArrowPos add def
%          ArrowPos Length gt { exit } if
          x1 Alpha cos ArrowPos mul add
          y1 Alpha sin ArrowPos mul add
          ArrowInside
          pop pop
        } repeat
      }{
        /ArrowPos \psk@ArrowInsideOffset\space def
        /dArrowPos \psk@ArrowInsideNo 1 gt {%
          1.0 \psk@ArrowInsideNo 1.0 add div
        }{\psk@ArrowInsidePos } ifelse def
          \psk@ArrowInsideNo\space cvi {
            /ArrowPos ArrowPos dArrowPos add def
            x2 x1 sub ArrowPos mul x1 add
            y2 y1 sub ArrowPos mul y1 add
            ArrowInside
            pop pop
          } repeat
      } ifelse
      pop pop Lineto
    } repeat
  }{ ArrowA /n n 2 sub def n { Lineto } repeat } ifelse
  CP 4 2 roll ArrowB L pop pop } if >
%
% Redefinition of the PostScript /Polygon macro to print the intermediate
% arrow on each segment of the line
\pst@def{Polygon}<{%
    NArray n 2 eq { 0 0 /n 3 def } if
    n 3 lt {
	n { pop pop } repeat
    }{
	n 3 gt { CheckClosed } if
	n 2 mul	-2 roll
	/y0 ED
	/x0 ED
    	/y1 ED
    	/x1 ED
    	/xx1 x1 def
    	/yy1 y1 def
    	x1 y1
    	/x1 x0 x1 add 2 div def
    	/y1 y0 y1 add 2 div def
    	x1 y1 moveto
    	/n n 2 sub def
	/drawArrows {
	    x11 y11
	    \psk@ArrowInsidePos\space 1 gt {
		/Alpha y12 y11 sub x12 x11 sub atan def
		/ArrowPos \psk@ArrowInsideOffset\space def
		/Length x12 x11 sub y12 y11 sub Pyth def
		/dArrowPos \psk@ArrowInsidePos\space abs def
		{
		    /ArrowPos ArrowPos dArrowPos add def
		    ArrowPos Length gt { exit } if
		    x11 Alpha cos ArrowPos mul add
		    y11 Alpha sin ArrowPos mul add
		    currentdict /ArrowInside known { ArrowInside } if
		    pop pop
		} loop
	    }{
		/ArrowPos \psk@ArrowInsideOffset\space def
		/dArrowPos \psk@ArrowInsideNo\space 1 gt {%
	    	    1.0 \psk@ArrowInsideNo\space 1.0 add div
		}{ \psk@ArrowInsidePos } ifelse def
		\psk@ArrowInsideNo\space cvi {
		    /ArrowPos ArrowPos dArrowPos add def
		    x12 x11 sub ArrowPos mul x11 add
		    y12 y11 sub ArrowPos mul y11 add
		    currentdict /ArrowInside known { ArrowInside } if
		    pop pop
		} repeat
	    } ifelse
	    pop pop Lineto
	} def
	n {
	    4 copy
	    /y11 ED /x11 ED /y12 ED /x12 ED
	    drawArrows
	} repeat
	x1 y1 x0 y0
	6 4 roll
	2 copy
	/y11 ED /x11 ED /y12 y0 def /x12 x0 def
	drawArrows
	/y11 y0 def /x11 x0 def /y12 yy1 def /x12 xx1 def
	drawArrows
	pop pop
    	closepath
    } ifelse %
}>
%
%
% Redefinition of the PostScript /OpenBezier macro to print the intermediate
% arrow
\pst@def{OpenBezier}<{%
  /dArrowPos \psk@ArrowInsideNo 1 gt {%
    1.0 \psk@ArrowInsideNo 1.0 add div
    }{ \psk@ArrowInsidePos } ifelse def
      BezierNArray
      n 1 eq { pop pop
      }{ 2 copy
        /y0 ED /x0 ED
        ArrowA
        n 4 sub 3 idiv { 6 2 roll 4 2 roll curveto } repeat
        6 2 roll
        4 2 roll
        ArrowB
        /y3 ED /x3 ED /y2 ED /x2 ED /y1 ED /x1 ED
        /cx x1 x0 sub 3 mul def
        /cy y1 y0 sub 3 mul def
        /bx x2 x1 sub 3 mul cx sub def
        /by y2 y1 sub 3 mul cy sub def
        /ax x3 x0 sub cx sub bx sub def
        /ay y3 y0 sub cy sub by sub def
        /getValues {
          ax t0 3 exp mul bx t0 t0 mul mul add cx t0 mul add x0 add
          ay t0 3 exp mul by t0 t0 mul mul add cy t0 mul add y0 add
          ax t 3 exp mul bx t t mul mul add cx t mul add x0 add
          ay t 3 exp mul by t t mul mul add cy t mul add y0 add
        } def
        /getdL {
          getValues
          3 -1 roll sub 3 1 roll sub Pyth
        } def
        /CurveLength {
          /u 0 def
          /du 0.01 def
          0 100 {
            /t0 u def
            /u u du add def
            /t u def
            getdL add
          } repeat } def
          /GetArrowPos {
            /ende \psk@ArrowInsidePos\space 1 gt
              {ArrowPos}
              {ArrowPos CurveLength mul} ifelse def
            /u 0 def
            /du 0.01 def
            /sum 0 def
            { /t0 u def
              /u u du add def
              /t u def
              /sum getdL sum add def
              sum ende gt {exit} if
            } loop u
          } def
          /ArrowPos \psk@ArrowInsideOffset\space def
          /loopNo \psk@ArrowInsidePos\space 1 gt {%
            CurveLength \psk@ArrowInsidePos\space div cvi
          }{ \psk@ArrowInsideNo } ifelse def
            loopNo cvi {
              /ArrowPos ArrowPos dArrowPos add def
              /t GetArrowPos def
              /t0 t 0.95 mul def
              getValues
              ArrowInside pop pop pop pop
            } repeat
            x1 y1 x2 y2 x3 y3 curveto
  } ifelse
}>
%
% Redefinition of the PostScript /NCLine macro to print the intermediate
% arrow of the line
\pst@def{NCLine}<{%
	NCCoor
	tx@Dict begin
	ArrowA CP 4 2 roll ArrowB
	4 copy
	/y2 ED /x2 ED /y1 ED /x1 ED
	x1 y1
	\psk@ArrowInsidePos\space 1 gt {
		/Alpha y2 y1 sub x2 x1 sub atan def
		/ArrowPos \psk@ArrowInsideOffset\space def
		/Length x2 x1 sub y2 y1 sub Pyth def
		/dArrowPos \psk@ArrowInsidePos\space abs def
		{%
			/ArrowPos ArrowPos dArrowPos add def
			ArrowPos Length gt { exit } if
			x1 Alpha cos ArrowPos mul add
			y1 Alpha sin ArrowPos mul add
			ArrowInside
			pop pop
		} loop
	}{%
		/ArrowPos \psk@ArrowInsideOffset\space def
		/dArrowPos \psk@ArrowInsideNo 1 gt {%
			1.0 \psk@ArrowInsideNo 1.0 add div
		}{ \psk@ArrowInsidePos } ifelse def
		\psk@ArrowInsideNo\space cvi {
			/ArrowPos ArrowPos dArrowPos add def
			x2 x1 sub ArrowPos mul x1 add
			y2 y1 sub ArrowPos mul y1 add
			ArrowInside
			pop pop
		} repeat
	} ifelse
	pop pop lineto pop pop
	end%
}>
%
\pst@def{NCCurve}<{%
	GetEdgeA GetEdgeB
	xA1 xB1 sub yA1 yB1 sub
	Pyth 2 div dup 3 -1 roll mul
	/ArmA ED
	mul
	/ArmB ED
	/ArmTypeA 0 def
	/ArmTypeB 0 def
	GetArmA GetArmB
	xA2 yA2 xA1 yA1
	2 copy
	/y0 ED /x0 ED
	tx@Dict begin
		ArrowA
	end
	xB2 yB2 xB1 yB1
	tx@Dict begin
		ArrowB
	end
	/y3 ED /x3 ED /y2 ED /x2 ED /y1 ED /x1 ED
	/cx x1 x0 sub 3 mul def
	/cy y1 y0 sub 3 mul def
	/bx x2 x1 sub 3 mul cx sub def
	/by y2 y1 sub 3 mul cy sub def
	/ax x3 x0 sub cx sub bx sub def
	/ay y3 y0 sub cy sub by sub def
	/getValues {
		ax t0 3 exp mul bx t0 t0 mul mul add cx t0 mul add x0 add
		ay t0 3 exp mul by t0 t0 mul mul add cy t0 mul add y0 add
		ax t 3 exp mul bx t t mul mul add cx t mul add x0 add
	ay t 3 exp mul by t t mul mul add cy t mul add y0 add
	} def
	/getdL {
		getValues
		3 -1 roll sub 3 1 roll sub Pyth
	} def
	/CurveLength {
		/u 0 def
		/du 0.01 def
		0 100 {
			/t0 u def
			/u u du add def
			/t u def
			getdL add
		} repeat } def
	/GetArrowPos {
		/ende \psk@ArrowInsidePos\space 1 gt {ArrowPos}{ArrowPos CurveLength mul} ifelse def
		/u 0 def
		/du 0.01 def
		/sum 0 def
		{
			/t0 u def
			/u u du add def
			/t u def
			/sum getdL sum add def
			sum ende gt {exit} if
		} loop u
	} def
	/dArrowPos \psk@ArrowInsideNo 1 gt {%
		1.0 \psk@ArrowInsideNo 1.0 add div
	}{ \psk@ArrowInsidePos } ifelse def
	/ArrowPos \psk@ArrowInsideOffset\space def
	/loopNo \psk@ArrowInsidePos\space 1 gt {%
		CurveLength \psk@ArrowInsidePos\space div cvi
		}{ \psk@ArrowInsideNo } ifelse def
	loopNo cvi {
		/ArrowPos ArrowPos dArrowPos add def
		/t GetArrowPos def
		/t0 t 0.95 mul def
		getValues
		ArrowInside pop pop pop pop
	} repeat
	x1 y1 x2 y2 x3 y3 curveto
	/LPutVar [ xA1 yA1 xA2 yA2 xB2 yB2 xB1 yB1 ] cvx def
	/LPutPos { t LPutVar BezierMidpoint } def
	/HPutPos { { HPutLines } HPutCurve } def
	/VPutPos { { VPutLines } HPutCurve } def
}>
%

\def\resetArrowOptions{%
  \def\pst@linetype{0}%
  \psset[pst-arrow]{%
    hooklength=3mm, hookwidth=1mm,
    ArrowFill=true,
    ArrowInside={}, ArrowInsidePos=0.5,
    ArrowInsideNo=1, ArrowInsideOffset=0,
}}
%
\resetArrowOptions
%
\catcode`\@=\PstAtCode\relax
%
%% END: pst-arrow.tex
\endinput
