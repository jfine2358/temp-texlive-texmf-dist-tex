%%%%%%%%%%%%%%%%%%%%%%%%%%%%%%%%%%%%%%%%%%%%%%%%%%%%%%%%%%%%%%%%%%%%%%%
%%                                                                   %%
%% This is file `pst-moire.tex'                                      %%
%%                                                                   %%
%% IMPORTANT NOTICE:                                                 %%
%%                                                                   %%
%% Package `pst-moire'                                               %%
%%                                                                   %%
%% J�rgen Gilg, Manuel Luque, Jean-Michel Sarlat                     %%
%%                                                                   %%
%% Copyright (C) 2018                                                %%
%%                                                                   %%
%% This program can redistributed and/or modified under              %%
%% the terms of the LaTeX Project Public License                     %%
%% Distributed from CTAN archives in directory                       %%
%% macros/latex/base/lppl.txt; either version 1.3c of                %%
%% the License, or (at your option) any later version.               %%
%%                                                                   %%
%% DESCRIPTION:                                                      %%
%%   `pst-moire' is a PSTricks package to draw moire patterns        %%
%%                                                                   %%
%%%%%%%%%%%%%%%%%%%%%%%%%%%%%%%%%%%%%%%%%%%%%%%%%%%%%%%%%%%%%%%%%%%%%%%
\def\fileversion{2.1}%
\def\filedate{2018/11/22}%
\message{`PST-MOIRE v\fileversion, \filedate\space (ML)}%
\csname PSTMoireLoaded\endcsname
% Requires PSTricks, pst-xkey and pst-node packages
\ifx\PSTricksLoaded\endinput\else\input pstricks.tex\fi
\ifx\PSTXKeyLoaded\endinput\else\input pst-xkey \fi
\ifx\MultidoLoaded\endinput\else\input multido.tex \fi
%
\edef\PstAtCode{\the\catcode`\@} \catcode`\@=11\relax
%
\pstheader{pst-moire.pro}
%
\pst@addfams{pst-moire}
\define@key[psset]{pst-moire}{Rmax}{\edef\psk@moire@R{#1}} % en cm
\define@key[psset]{pst-moire}{scale}{\edef\psk@moire@scale{#1}} % echelle
\define@key[psset]{pst-moire}{Alpha}{\edef\psk@moire@Alpha{#1}} % pente des traits pour Gauss en degr�s
\define@key[psset]{pst-moire}{rotate}{\edef\psk@moire@rot{#1 }} % rotation de la grille
\define@key[psset]{pst-moire}{E}{\edef\psk@moire@E{#1 }} % espacement entre 2 traits de Gauss
\define@key[psset]{pst-moire}{n}{\edef\psk@moire@n{#1 }} % nombre de cercles
\define@key[psset]{pst-moire}{T}{\edef\psk@moire@T{#1 }} % p�riode : dr en mm
% random-dots parameters
\define@key[psset]{pst-moire}{r}{\edef\psk@moire@r{#1 }} % r des trous en pts
\define@key[psset]{pst-moire}{p}{\edef\psk@moire@p{#1 }} % pas de la grille
\define@key[psset]{pst-moire}{k}{\edef\psk@moire@k{#1 }} % k : facteur qui module la dispersion des trous
\define@key[psset]{pst-moire}{symbole}{\def\psk@moire@symbol{#1}} % motif (lettre ou chiffre)
\define@key[psset]{pst-moire}{PSfont}{\def\pst@moirage@PSfont{/#1 }}
\define@key[psset]{pst-moire}{fontsize}{\def\pst@moirage@fontsize{#1 }}
\define@key[psset]{pst-moire}{vadjust}{\def\pst@moirage@vadjust{#1 }}
\define@key[psset]{pst-moire}{hadjust}{\def\pst@moirage@hadjust{#1 }}
\define@key[psset]{pst-moire}{hole}{\def\pst@moirage@hole{#1}}
%
\psset[pst-moire]{Rmax=6,scale=1,Alpha=70,rotate=0,E=0.25,n=30,T=2}
\psset[pst-moire]{r=0.5,p=4,k=0.2,symbole=2,rotate=0,PSfont=Helvetica-Bold,fontsize=4.7,vadjust=-1.5,hadjust=0,hole=round}
%
\def\variablesMoirages{%
 /R \psk@moire@R\space def
 /reduction {\psk@moire@scale\space mul} def
 /unit {\pst@number\psunit mul reduction} def
 /moire  {\psk@moire@type} def
 /rot \psk@moire@rot def
 /Runit R unit def
 /E1 \psk@moire@E def
 /nr \psk@moire@n def
 /Tr \psk@moire@T \pst@number\psunit 10 div mul def
 /circle { newpath 0 0 radius 0 360 arc closepath} def
 /NombreTraits  Runit \pst@number\psrunit div 10 mul cvi def
 /NombreTraitsLinear Runit 2.845 div cvi def
 /reduc \psk@moire@scale\space def
 /Alpha \psk@moire@Alpha\space def
 /m Alpha dup sin exch cos div def % tan(Alpha)
 /linecolor {\pst@usecolor\pslinecolor} def
 /linewidth {\pst@number\pslinewidth SLW} def
 /mm {\pst@number\psunit 10 div} bind def % le millimetre
 /DotSize  {\psk@dotsize} def
 /DotStyle {\@nameuse{psds@\psk@dotstyle}} def
 }
%% Reservation des noms de macros
% adaptation de la macro de pst-solides3d
% #1=type de reservation associee a une liste qui
% doit forcement etre du type pst@moire@list@#1
\def\pst@moire@reservednames#1{%
\edef\@tempa{\csname pst@moire@list@#1\endcsname}
\@for \name:=\@tempa\do{%
  \expandafter\def\csname pst@moire@#1@\name\endcsname{}%
}}
% nature du moir�
\def\pst@moire@list@type{%
  circle,Fresnel,radial,square,Gauss,Newton,Bouasse,linear,dot,chess}
%
%% Reservation de macros correspondants aux types
\pst@moire@reservednames{type}
\define@key[psset]{pst-moire}{type}{%
 \ifcsname pst@moire@type@#1\endcsname
   \def\psk@moire@type{pst-#1}%
 \else
    \@pstrickserr{The moire #1 is not defined, the moire by default is
    drawn}{}%
 \fi}
% moir� par defaut
\psset[pst-moire]{type=Fresnel}%
%
%  01 novembre 2018
%% Ajout de moir� �  la liste des types
\def\addtomoirelisttype#1{%
% ajouter les nouveaux objets  la liste
\expandafter\def\expandafter\pst@moire@list@type\expandafter%
 {\pst@moire@list@type,#1}%
% Reserver le nom des nouveaux types
 \edef\@tempa{#1}%
 \@for \name:=\@tempa\do{%
  \expandafter\def\csname pst@moire@type@\name\endcsname{}%
 }}
%
\def\psmoire{\def\pst@par{}\pst@object{psmoire}}
\def\psmoire@i{\@ifnextchar({\psmoire@do}{\psmoire@do(0,0)}}
\def\psmoire@do(#1){%
\begingroup%
\use@par%
  \begin@SpecialObj%
  \pst@@getcoor{#1}%
    \addto@pscode{
 moireDict begin
  \variablesMoirages
  \pst@coor  translate
 rot rotate
  \psk@moire@type\space
 end
    }%
  \end@SpecialObj%
\endgroup%
}%
%
\def\psRandomDot{\def\pst@par{}\pst@object{psRandomDot}}
\def\psRandomDot@i{\@ifnextchar({\psRandomDot@ii}{\psRandomDot@ii(10,10)}}
\def\psRandomDot@ii(#1,#2){%
\begin@SpecialObj
\addto@pscode{%
/cm {\pst@number\psunit mul } bind def
/pagewidth #1 cm def % en cm
/pageheight #2 cm def % en cm
% pas de la grille
/gridwidth \psk@moire@p def
% rayon des trous
/holeradius \psk@moire@r def
/symbol (\psk@moire@symbol) def
/alpha \psk@moire@rot def
/PSfont {\pst@moirage@PSfont} def
/fontsize \pst@moirage@fontsize def
/vadjust \pst@moirage@vadjust def
/hadjust \pst@moirage@hadjust def
/hole (\pst@moirage@hole) def
PSfont findfont fontsize scalefont setfont
% nombre al�atoire compris entre 0.0000 et 1.0000
/rand01 {rand 10000 mod 10000 div} def
% holedispersion permet de choisir le rayon du disque o� on placera al�atoirement le centre du trou
/holedispersion {gridwidth \psk@moire@k neg \psk@moire@k 2 mul rand01 mul add mul} def
/loy pageheight -2 div def
/hiy pageheight 2 div def
/hix pagewidth 2 div def
/lox pagewidth -2 div def
/squarehole {
  xi holeradius neg add
  yi holeradius neg add moveto
  0 holeradius 2 mul rlineto
  holeradius 2 mul 0 rlineto
  0 holeradius 2 mul neg rlineto
}
def
%
/revealer {
2 srand
  newpath
  lox gridwidth sub loy gridwidth sub moveto
 pagewidth  2 gridwidth mul add 0 rlineto
  0 pageheight 2 gridwidth mul add  rlineto
  pagewidth neg  2 gridwidth mul sub 0 rlineto
loy gridwidth hiy
  {
    /y exch def
lox gridwidth hix
    {
      /x exch def
      /xi x holedispersion add def
      /yi y holedispersion add def
hole (round) eq {
      xi holeradius add yi moveto
      xi yi holeradius 360 0 arcn
      }{
  squarehole
      } ifelse
    }
    for
  }
  for
  closepath
  fill
} def
symbol stringwidth /wy exch def /wx exch def
/baselayer
{
1 setgray
lox gridwidth sub loy gridwidth sub pagewidth 2 gridwidth mul add pageheight 2 gridwidth mul add rectstroke
% 0 setgray
\pst@usecolor\pslinecolor
2 srand
loy gridwidth hiy
  {
    /y exch def
lox gridwidth hix
    {
      /x exch def
      /xi x holedispersion add def
      /yi y holedispersion add def
   gsave
      xi yi moveto
      wx -2 div hadjust add wy -2 div vadjust add rmoveto
      symbol show
   grestore
    }
    for
  }
  for
} def
-90 rotate
baselayer
alpha rotate
revealer
 }% fin du code ps
 \end@SpecialObj
 }% % fin de la commande PSTricks
% https://lspwww.epfl.ch/publications/books/moire/figs1II.html
%
% Figure "star1" from the Moire Demonstration Kit accompanying the book:
%		"The Theory of the Moire Phenomenon"
% by I. Amidror, published by Springer, 2007.
%
%		* * *  Copyright (c) 2007 EPFL  * * *
%
% Author: I. Amidror
%
% Modified: March 30, 2007
%
% ********************************************************************************
%
% An aperiodic dot screen that has undergone a non-linear transformation
%
% ********************************************************************************
% adapted by Manuel Luque for PSTricks
% November 15, 2018
%
\define@key[psset]{pst-moire}{function}{\edef\psk@moire@function{#1 }} %  la fonction
\define@boolkey[psset]{pst-moire}[Pst@moire@]{layers}[true]{}
% Default values
\psset[pst-moire]{function=5 r mul t 5 mul sin neg 0.5 mul 1 add mul 2.5 sub,layers=true}
%
\def\psGlassPattern{\def\pst@par{}\pst@object{psGlassPattern}}
\def\psGlassPattern@i{%
\addbefore@par{dotsize=1pt,fillcolor=black}
\begin@SpecialObj
\addto@pscode{%
 \ifPst@algebraic
      /fonction (\psk@moire@function) tx@AlgToPs begin AlgToPs end cvx def
    \else
      /fonction  { \psk@moire@function } def
    \fi
  \psk@dotsize
  \@nameuse{psds@\psk@dotstyle}
   /dotcolor {\pst@usecolor\pslinecolor } def
   /dotcolor2  {\pst@usecolor\psfillcolor} def
\ifPst@moire@layers /layers true def \else /layers false def \fi
/star-shapped {
20 dict begin
/Atan { /atan load stopped { pop pop 0 } if } def % return 0 if atan not known
/cm {\pst@number\psunit mul} bind def
/dim 15 cm def
/dim2 7.5 cm def
/myrand {rand 2147483647 div 2 mul 1 sub} def	% random number between -1...1
/pr 2 def		% period for dot screen
% ************ Draw the dot-screen:
% draw first (reference) dot screen:
10 srand		% seed for rand
/y dim2 neg def
1 setgray
-7.5 cm -7.5 cm 15 cm 15 cm rectfill
%0 setgray
%-7.5 cm -7.5 cm 15 cm 15 cm rectstroke
gsave
-7.5 cm -7.5 cm 14.75 cm 14.75 cm rectclip
dotcolor
0 pr dim 	% draw horizontal lines of dots
{
	/x dim2 neg def
	/ysave exch dim2 sub def
	0 pr dim 		% draw a horizontal line of dots
		{/xsave exch dim2 sub def
		/x1 x dim2 div def
		/y1 y dim2 div def
		/r2 x1 x1 mul y1 y1 mul add def
		/r r2 sqrt def
		/t y1 x1  Atan def	% angle in degrees!
		/cost fonction  def
		xsave
		fonction t cos mul 		% g1(x,y)
		add
		myrand  sub
		ysave
		fonction t sin mul 		% g2(x,y)
		add
		myrand sub Dot
		/x x pr add def
	} for
	/y y pr add def
} for
grestore
layers {
gsave
-7.25 cm -7.25 cm 14.75 cm 14.75 cm rectclip
% draw second dot screen:
%0 setgray
dotcolor2
10 srand		% same seed for rand
/y dim2 neg def
0 pr dim		% draw horizontal lines of dots
{
	/x dim2 neg def
	/ysave exch dim2 sub def
	0 pr dim		
		{/xsave exch dim2 sub def
		xsave
		myrand sub
		ysave
		myrand sub Dot
		/x x pr add def
	} for
	/y y pr add def
} for
grestore
} if
end
 } def
star-shapped
 }% fin du code ps
 \end@SpecialObj
 }% % fin de la commande PSTricks
%
\define@key[psset]{pst-moire}{NumberDots}{\edef\psk@moire@NumberDots{#1 }} %  Number of dots
\define@key[psset]{pst-moire}{coefficients}{\edef\psk@moire@coefficients{#1 }} % [a b theta tx ty]
\define@key[psset]{pst-moire}{srand}{\edef\psk@moire@srand{#1 }} % graine al�atoire pour postscript
\define@boolkey[psset]{pst-moire}[Pst@moire@]{revlayer}[true]{}
\define@boolkey[psset]{pst-moire}[Pst@moire@]{layer}[true]{}
% Default values
\psset{NumberDots=2000,coefficients=1.05 1.05 0 0 0,revlayer=true,layer=true,srand=10}
%
\def\psRandomDotPatterns{\def\pst@par{}\pst@object{psRandomDotPatterns}}
\def\psRandomDotPatterns@i{%
\addbefore@par{dotsize=2pt,fillcolor=black}
\begin@SpecialObj
\addto@pscode{%
  /N_ \psk@moire@NumberDots def
  \psk@dotsize
  \@nameuse{psds@\psk@dotstyle}
  /dotcolor {\pst@usecolor\pslinecolor } def
  /dotcolor2  {\pst@usecolor\psfillcolor} def
  /DOT {r 0 360 arc fill} def
\ifPst@moire@revlayer /revealinglayer true def \else /revealinglayer false def \fi
\ifPst@moire@layer /layer true def \else /layer false def \fi
/cm {\pst@number\psunit mul} bind def
/xmax 5 cm round cvi def
/ymax xmax def
/GlassPattern {
7 dict begin
/ty exch def
/tx exch def
/t exch def % theta degrees
/b exch def
/a exch def
layer {
\psk@moire@srand srand
N_ {
    /x rand 2 xmax mul mod xmax sub def % -xmax<x<xmax
    /y rand 2 ymax mul mod ymax sub def % -ymax<y<ymax
    /r rand 2147483647 div 2 mul def
 dotcolor x y DOT
   } repeat
 } if
% transformation
a b scale t rotate tx ty translate
\psk@moire@srand srand
revealinglayer {
N_{
    /x rand 2 xmax mul mod xmax sub def % -xmax<x<xmax
    /y rand 2 ymax mul mod ymax sub def % -ymax<y<ymax
    /r rand 2147483647 div 2 mul def
 dotcolor2 x y DOT
    } repeat
 } if
grestore
end
 } def
[\psk@moire@coefficients] length 3 eq {\psk@moire@coefficients 0 0} {\psk@moire@coefficients} ifelse
GlassPattern
 }% fin du code ps
 \end@SpecialObj
 }% % fin de la commande PSTricks
 \catcode`\@=\PstAtCode\relax
%% END: pst-moire.tex
\endinput

