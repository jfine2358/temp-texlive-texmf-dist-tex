%& --translate-file=il1-t1.tcx
% Documentatioin File of the package fontch.tex   V2.2 2010/04/12
% Copyright 2010 R. Medina
% This work may be distributed and/or modified under the
% conditions of the LaTeX Project Public License, version 1.3c
%
%% Options of fontch
 \let\AMSfont\relax
 \let\DStroke\relax 
 \let\LMTone\relax
 \let\LMTSone\relax
%% Input fontch and bsymbols
 \input fontch.tex
 \input bsymbols.tex
\leftline{ } %% To start the page
\bigskip\bigskip
\twentyfourpoint
\centerline{\bf fontch}
\fourteenpoint\sf
\bigskip
\centerline{ Macros for changing fonts and sizes in plain \TeX}
\centerline{Rodrigo Medina rmedina@ivic.gob.ve}
\centerline{V2.2 2010/04/12}
\rm
\bigskip\bigskip
\twelvepoint
\leftline{\bf Description}
\tenpoint
       This package allows, in plain \TeX, to change with a single
command the size of all fonts that are used for text input and math input.
Main text fonts of sizes 8, 10, 12, 14, 20 and 24 points are available.
Fonts of sizes 5, 6, 7 and 9 points are used in subscripts and subscripts of
subscripts of other main sizes.
In addition fontch gives support to:
\item{\bull} Boldface math-italic and boldface math symbols,
\item{\bull} T1 Latin Modern fonts,
\item{\bull} TS1 companion symbol fonts.
\item{\bull} AMS fonts for boldface math,
\item{\bull} Double Stroke fonts for blackboard bold symbols
\medskip 
\twelvepoint
\leftline{\bf Requirements}
\tenpoint
No special requirements are needed for using the original OT1 Computer Modern fonts.

\item{\bull}  In order to use the options for  T1 Latin Modern fonts and the
companion TS1 symbol fonts, you
 need to have installed
  the Latin Modern fonts of version 1.000 or larger. Older versions of LM fonts,
  where the cork encoded fonts had names as cork-lmr10 (instead of ec-lmr10) 
  are compatible only with fontch-2.0.
 You can find the old package  in

\centerline{\sf CTAN:/tex-archive/obsolete/macros/plain/contrib/fontch20.tar.gz}
\item{\bull} In order to use the option for boldface math symbols you need to have
 the AMS fonts installed.
\item{\bull} In order to use the option for blackboard bold symbols you need to
 have the Double Stroke fonts installed.
\medskip
\twelvepoint
\leftline{\bf Components}
\tenpoint
The fontch V2.2 package is composed of the following files:\par\noindent
\settabs 10 \columns
\+&{\tt README}      &&-- text file with this same information \cr
\+&{\tt fontch.tex}  &&-- main macros\cr
\+&{\tt bsymbols.tex}&&-- macros for boldface  symbols\cr
\+&{\tt TS1mac.texr} &&-- macros for the TS1 companion symbols\cr
\+&{\tt DSmac.tex}   &&-- auxiliary file for the DStroke option\cr
\+&{\tt fontch.pdf}  &&-- manual of fontch\cr
\+&{\tt fontch\_doc.tex}  &&-- source of manual \cr

\medskip 
\twelvepoint
\leftline{\bf Installation}
\tenpoint
        You have only to put the macro files {\tt fontch.tex}, {\tt
 bsymbols.tex}, {\tt TS1mac.tex} and {\tt DSmac.tex} in any sensible place
 inside the texmf tree, like\par

\centerline{\tt  \dots/texmf/tex/plain/fontch/}

It is also commendable to put the files {\tt README} and {\tt fontch.pdf}
in a proper place such as

 \centerline{\tt \dots/texmf/doc/plain/fontch/}
\medskip 
\twelvepoint
\leftline{\bf Usage}
\tenpoint
   For using {\tt fontch.tex} with the original TeX OT1 fonts (Computer Modern)
just put at the beginning of the document:\par

{\tt $\backslash$input fontch.tex}

The file {\tt bsymbols.tex} provides macro definitions for boldface versions
of math symbols. For using it just input the file where you need it.

The fontch package has four options for handling different kinds of fonts:\par

\+&{\tt LMTone}  &&for Latin Modern T1 fonts.\cr
\+&{\tt LMTSone} &&for the TS1 companion symbol font.\cr
\+&{\tt AMSfont} &&for AMS fonts used for boldface math.\cr
\+&{\tt DStroke} &&for Doublestroke fonts providing blackboard-boldface
 symbols.\cr
\par

The four options are independent. They are activated by setting the
corresponding variable before calling  fontch.tex. For example for
activating all the four options and using the macros for bold math
put at the beginning of the document:
\smallskip\par
{\tt
$\backslash$let$\backslash$AMSfont$\backslash$relax\par
$\backslash$let$\backslash$DStroke$\backslash$relax\par
$\backslash$let$\backslash$LMTone$\backslash$relax\par
$\backslash$let$\backslash$LMTSone$\backslash$relax\par
$\backslash$input fontch.tex\par
$\backslash$input bsymbols.tex\par
}
\medskip 
{\twelvepoint
\leftline{\bf General Commands}
}
Here we treat those features of fontch that are common to all options. First
we have to say that there is one modification of the standard behavior of
plain \TeX. Plain
     \TeX\ defines for the font \#3 the same font (tenex) for text, script and
     scriptscript. That is awful. The fontch package uses sevenex for
     script and fiveex for scriptscript, as \TeX\ does for the other fonts.

The package fontch defines the following commands valid for any option.
\par\noindent
Commands for changing font size:\par

\+&{\tt $\backslash$eightpoint}      &&-- Change to small type\cr
\+&{\tt $\backslash$tenpoint}        &&-- Change to normal type\cr
\+&{\tt $\backslash$twelvepoint}     &&-- Change to large type\cr
\+&{\tt $\backslash$fourteenpoint}   &&-- Change to very large type\cr
\+&{\tt $\backslash$twentypoint}     &&-- Change to huge type\cr
\+&{\tt $\backslash$twentyfourpoint} &&-- Change to immense type\cr
\smallskip
\noindent
Commands for changing family, most of them already in plain \TeX:

\+&{\tt $\backslash$rm}         &&-- Roman\cr
\+&{\tt $\backslash$sl}         &&-- Slanted\cr
\+&{\tt $\backslash$it}         &&-- Italic\cr
\+&{\tt $\backslash$bf}         &&-- Boldface\cr
\+&{\tt $\backslash$tt}         &&-- Teletype\cr
\+&{\tt $\backslash$sf}         &&-- Sans Serif (new)\cr
\+&{\tt $\backslash$sc}         &&-- Small Caps\cr
\+&{\tt $\backslash$cal}        &&-- Calligraphic\cr
\+&{\tt $\backslash$mit}        &&-- Math Italic\cr
\+&{\tt $\backslash$mb}         &&-- Math Boldface (new)\cr
\+&{\tt $\backslash$bcal}       &&-- Boldface calligraphic (new)\cr
\+&{\tt $\backslash$oldstyle}   &&-- Old style digits\cr
\+&{\tt $\backslash$boldstyle}  &&-- Boldface old style digits (new)\cr
\+&{\tt $\backslash$setmathbold}&&-- Set families 1 and 2 to cmmib and
 cmmbsy (new)\cr
\+&{\tt $\backslash$unsetmathbold}&&-- Reset families 1 and 2 to cmmi and
 cmmsy (new)\cr
\smallskip
\noindent
Command for changing line spacing:
\+&{\tt $\backslash$doublespace}&&-- Double space for ten and twelve points\cr
\medskip
The scope of these commands is the group defined by the braces.
The commands for changing size are active until a new changing size command is
given. The initial size is tenpoint and the initial family is roman. The
commands for changing family 
are active until a new changing family command is given. The changing size
commands reset the family to {\tt $\backslash$rm} and the double space
to single space.

 Since the changing size commands modify the parameters that
control the spacing between lines and between words, unpredictable results
are obtained if they are given in the middle of a paragraph. Nevertheless
inside a paragraph 
it is still possible to use the \TeX\ font commands of the form
 {\it $\backslash$size+family}, where {\it size} is one from {\tt five,
six, seven, eight, nine, ten, twelve, fourteen, twenty} or {\tt  twentyfour}
and {\it family} is one from  {\tt rm, sl, it, bf, tt, ss} or {\tt csc}
({\it e.g.}{\tt $\backslash$twelvess}). Here {\tt ss} represents the sans serifi
family and {\tt csc} the small caps family.

\noindent
For example the following text\par
{\tt
\smallskip
$\backslash$eightpoint\par
This  well known formula of physics is due to $\{\backslash$it
 Newton$\}$\par
\$\$ $\backslash$vec F = m $\backslash$vec a \$\$\par
\medskip
$\backslash$twentyfourpoint\par
This  well known formula of physics is due to $\{\backslash$it
 Newton$\}$\par
\$\$ $\backslash$vec F = m $\backslash$vec a \$\$\par
\medskip
$\backslash$fourteenpoint\par
Many books use boldface letters instead of arrows for indicating vectors,\par
\$\$ $\{\backslash$mb F$\}$ = m $\{\backslash$mb a$\}$ \$\$\par
}
\par \noindent
yields:\par
\eightpoint
This  well known formula of physics is due to {\it Newton}
$$ \vec F = m \vec a $$

\twentyfourpoint
This  well known formula of physics is due to {\it Newton}
$$ \vec F = m \vec a $$

\fourteenpoint
Many books use boldface letters instead of arrows for indicating  vectors,\par
$$ {\mb F} = m {\mb a} $$

\tenpoint
\medskip\noindent
Another example using the sans serif family,

{\tt
The orthogonal matrix \$$\backslash$sf A\$ satifies the relation\par
\$\$ $\{\backslash$sf A$\}$\^{}t $\backslash$sf A = I\$\$
}
\par\noindent
yields

The orthogonal matrix $\sf A$ satifies the relation
$$ {\sf A}^t \sf A = I$$

\medskip
{\twelvepoint
\leftline{\bf AMS fonts}
}
 Only the cmcs, cmmib, cmbsy and cmex AMS fonts are supported. These
     are used for small caps and for boldface math-italic and boldface
     greek symbols. The other AMS fonts like the Euler, Cyrillic and extra
     symbols are not supported.

\medskip
{\twelvepoint
\leftline{\bf Boldface math}}

  There are two main uses of boldface math. One case is the use of
     isolated boldface  characters or symbols inside a formula. For this case
     fontch provides bold versions of math symbols and the macro
  {\tt $\backslash$mb} for bold
     math-italic characters.  The plain \TeX\  macro
{\tt $\backslash$bf}  yields roman-bold
     characters in math mode.  The name of the bold version of a math symbol
     is obtained adding the prefix {\tt bf} to the name of the normal symbol.
    For
     example a boldface italic ``a'' ($\mb a$) is {\tt $\{\backslash$mb a$\}$},
 a boldface roman ``P'' ({\bf P}) is {\tt $\{\backslash$bf P$\}$},
      a boldface $\sigma$ ($\bfsigma$) is {\tt $\backslash$bfsigma}, a boldface
{\tt $\backslash$iff} ($\bfiff$) is {\tt $\backslash$bfiff}.

     The other case is when one wants to write a complete formula in
     boldface, for example inside a title. For this case fontch has the
     macro {\tt $\backslash$setmathbold} that changes the font families 1 (cmmi) and 2 (cmsy)
     to the bold versions cmmib and cmmbsy. The macro
 {\tt $\backslash$unsetmathbold} resets
     the families 1 and 2 to their original values. These macros should be
     put before and after the formula.

  Usually the macro  {\tt $\backslash$setmathbold} does not change the whole
 formula to
     boldface. This is due to the fact that in math mode some symbols
     come from family \#0 (cmr) or \#3 (cmex) such as ``+''
 or {\tt $\backslash$int}. Fontch
     provides macros for the symbols that come from family \#0, For
     example the bold version of ``('' is {\tt $\backslash$bflparen}.

  Math symbols of family \#3 (cmex) do not have bold versions. In particular
    {\tt $\backslash$int}, {\tt $\backslash$sum} and {\tt $\backslash$prod}.
 Nevertheless {\tt $\backslash$smallint} is of cmsy and does have
     a bold version.


The macros for boldface math symbols are defined in the file
 {\tt bsymbols.tex}. The macros for the bold version of
symbols that don't have a keyword are
\smallskip
\+&$\backslash$bfexcl      &&$\bfexcl$\cr
\+&$\backslash$bflparen    &&$\bflparen$\cr
\+&$\backslash$bfrparen    &&$\bfrparen$\cr 
\+&$\backslash$bfplus      &&$\bfplus$\cr  
\+&$\backslash$bfcomma     &&$\bfcomma$\cr 
\+&$\backslash$bfcolon     &&$\bfcolon$\cr 
\+&$\backslash$bfsemicolon &&$\bfsemicolon$\cr 
\+&$\backslash$bfequal     &&$\bfequal$\cr 
\+&$\backslash$bflbraket   &&$\bflbraket$\cr 
\+&$\backslash$bfrbraket   &&$\bfrbraket$\cr 
\+&$\backslash$bflt        &&$\bflt$\cr  
\+&$\backslash$bfslash     &&$\bfslash$\cr 
\+&$\backslash$bfgt        &&$\bfgt$\cr   
\+&$\backslash$bfminus     && $\bfminus$\cr 
\+&$\backslash$bfvert      &&$\bfvert$\cr  

\smallskip\noindent
For example the following text

{\tt
$\backslash$twelvepoint\par
$\backslash$centerline$\{\backslash$bf Newton's Sencod Law, 
$\backslash$setmathbold \$$\backslash$vec F$\backslash$bfequal m
$\backslash$vec a\$\par
$\backslash$unsetmathbold$\}$}\par
\smallskip\noindent
yields

\twelvepoint
\centerline{\bf Newton's Second Law, \setmathbold$\vec F\bfequal m\vec a$
\unsetmathbold}
\tenpoint

\medskip
{\twelvepoint
\leftline{\bf Double Stroke option}
}
The {\tt DStroke} option loads the Double Stroke fonts, that provide
blackboard-boldface capital letters. 
 Only the roman Double Stroke font is supported. The sans serif version
     is not supported. There is only a command associated to this option

\+&{\tt $\backslash$ds} &&-- Double Stroke\cr

\smallskip\noindent
For example\par
{\tt \$\$ $\{\backslash$ds N$\} \backslash$subset
$\{\backslash$ds Z$\} \backslash$subset
$\{\backslash$ds Q$\} \backslash$subset
$\{\backslash$ds R$\} \backslash$subset
$\{\backslash$ds C$\}$ \$\$}

\noindent
yields

$$ {\ds N}\subset{\ds Z}\subset{\ds Q}\subset{\ds R}\subset{\ds C}$$


\medskip
{\twelvepoint
\leftline{\bf T1 encoded Latin Modern fonts}}

The {\tt LMTone} option loads the T1 encoded Latin Modern fonts, which
are used for text input. Math input is still done with the CM OT1 fonts.
Using the translation files {\tt il1-t1.tcx} or {\tt il2-t1.tcx}
it is possible to write a latin1 or latin2 encoded {\tt .tex} file, without
bothering about accents and special characters.  This is not only more
convenient for writing the
file, but usually it is also required for a proper behavior of the
hyphenation procedures. For example

{\tt
\%\& --translate-file=il1-t1.tcx\par
$\backslash$let$\backslash$LMTone$\backslash$relax\par
$\backslash$input fontch.tex\par
\medskip
�Esto est� escrito en espa�ol! �no es c�modo?}\par
\smallskip\noindent
yields

�Esto est� escrito en espa�ol! �no es c�modo? 

\medskip
{\twelvepoint
\leftline{\bf Commands of the LM T1 option}}

The {\tt LMTone} option of fontch redefines the following plain
 \TeX\ commands: {\tt
  $\backslash$i, $\backslash$j, $\backslash$ae, $\backslash$AE,
 $\backslash$oe, $\backslash$OE, $\backslash$o, $\backslash$O,
  $\backslash$ss, $\backslash$SS, $\backslash$aa, $\backslash$AA,
 $\backslash$l, $\backslash$L,
  $\backslash$Gamma, $\backslash$Delta, $\backslash$Theta,
 $\backslash$Lambda, $\backslash$Xi, $\backslash$Pi, $\backslash$Phi,\par
\noindent
  $\backslash$Sigma, $\backslash$Upsilon,
 $\backslash$Psi, $\backslash$Omega, $\backslash$hbar.}

\noindent
Fontch defines new T1 commands for non-latin1 characters

\+&{\tt $\backslash$lh} &&\lh\cr
\+&{\tt $\backslash$Lh} &&\Lh\cr
\+&{\tt $\backslash$dh} &&\dh\cr
\+&{\tt $\backslash$th}  &&\th \cr
\+&{\tt $\backslash$dbar}  &&\dbar\cr 
\+&{\tt $\backslash$Edh} &&\Edh\cr
\+&{\tt $\backslash$edh} &&\edh\cr
\+&{\tt $\backslash$ij} &&\ij\cr
\+&{\tt $\backslash$IJ} &&\IJ\cr
\+&{\tt $\backslash$nj} &&\nj\cr
\+&{\tt $\backslash$NJ} &&\NJ\cr
\+&{\tt $\backslash$thorn} &&\thorn\cr
\+&{\tt $\backslash$Thorn} &&\Thorn\cr
\+&{\tt $\backslash$smallzero} &&\smallzero\cr
\+&{\tt $\backslash$lguille} &&\lguille &&&(don't appear in il1-t1.tcx)\cr
\+&{\tt $\backslash$rguille} &&\rguille  &&&(don't appear in il1-t1.tcx)\cr
\+&{\tt $\backslash$vispace}  &&\vispace\cr 

\noindent
The following plain \TeX\ text accents are redefined:

{\tt
 $\backslash$\`{}, $\backslash$\'{}, $\backslash$\^{}, $\backslash$\~{},
 $\backslash$\"{}, $\backslash$H, $\backslash$v, $\backslash$u,
 $\backslash$=, $\backslash$., $\backslash$b, $\backslash$c}

\noindent
The following  plain \TeX\ math-accents are redefined:

{\tt
  $\backslash$acute, $\backslash$grave,
 $\backslash$ddot,
$\backslash$tilde, $\backslash$bar, $\backslash$breve,
$\backslash$check, $\backslash$hat, $\backslash$dot.}

\noindent
New T1 text accents re defined:

\+&{\tt $\backslash$r}  &&-- ring accent\cr
\+&{\tt $\backslash$k}  &&-- ogonek\cr

\noindent
Finally a new T1 math-accent is defined

\+&{\tt $\backslash$ring}\cr

\medskip
{\twelvepoint
\leftline{\bf Commands of the LMTS1 option}}

The TS1 option defines a macro for accessing the TS1 symbols by their position
in the table,\par
 
\+&{\tt $\backslash$tcchar$\{$\#1$\}$} &&-- Character of TS1 table\cr

\noindent
Some plain \TeX\ commands are redefined:\par

\+&{\tt $\backslash$t$\{$\#1$\}$} &&--  tie-after-accent\cr
\+&{\tt $\backslash$P}     &&\P\cr
\+&{\tt $\backslash$S}     &&\S\cr
\+&{\tt $\backslash$dag}   &&\dag\cr
\+&{\tt $\backslash$ddag}  &&\ddag\cr
\+&{\tt $\backslash$copyright} &&\$copyright\cr

\noindent
There is a macro for most of the TS1 symbols,
Some are text versions of symbols of the math mode. The macros are
in increasing charcode order:

\+&{\tt $\backslash$arrowl}     &&\arrowl\cr 
\+&{\tt $\backslash$arrowr}     &&\arrowr\cr 
\+&{\tt $\backslash$blank}      &&\blank\cr
\+&{\tt $\backslash$dollar}     &&\dollar\cr     
\+&{\tt $\backslash$caster}     &&\caster\cr 
\+&{\tt $\backslash$sequals}    &&\sequals\cr 
\+&{\tt $\backslash$anglel}     &&\anglel\cr
\+&{\tt $\backslash$angler}     &&\angler\cr
\+&{\tt $\backslash$mho}        &&\mho\cr
\+&{\tt $\backslash$bigcircle}  &&\bigcircle\cr
\+&{\tt $\backslash$Ohm}        &&\Ohm\cr
\+&{\tt $\backslash$bbracketl}  &&\bbracketl\cr 
\+&{\tt $\backslash$bbracketr}  &&\bbracketr\cr
\+&{\tt $\backslash$arrowu}     &&\arrowu \cr
\+&{\tt $\backslash$arrowd}     &&\arrowd \cr
\+&{\tt $\backslash$textstar}   &&\textstar\cr
\+&{\tt $\backslash$born}       &&\born\cr
\+&{\tt $\backslash$died}       &&\died\cr
\+&{\tt $\backslash$leaf}       &&\leaf\cr
\+&{\tt $\backslash$married}    &&\married\cr
\+&{\tt $\backslash$music}      &&\music\cr
\+&{\tt $\backslash$doublevert} &&\doublevert\cr 
\+&{\tt $\backslash$perthousand} &&\perthousand\cr
\+&{\tt $\backslash$bull}        &&\bull\cr   
\+&{\tt $\backslash$centigrade}  &&\centigrade\cr
\+&{\tt $\backslash$olddollar}   &&\olddollar\cr
\+&{\tt $\backslash$oldcent}     &&\oldcent\cr
\+&{\tt $\backslash$florin}      &&\florin\cr
\+&{\tt $\backslash$Colon}       &&\Colon\cr
\+&{\tt $\backslash$won}         &&\won\cr
\+&{\tt $\backslash$naira}       &&\naira\cr
\+&{\tt $\backslash$guarani}     &&\guarani\cr
\+&{\tt $\backslash$peso}        &&\peso\cr
\+&{\tt $\backslash$lira}        &&\lira\cr
\+&{\tt $\backslash$recipe}      &&\recipe\cr
\+&{\tt $\backslash$interrobang} &&\interrobang\cr 
\+&{\tt $\backslash$gnaborretni} &&\gnaborretni\cr
\+&{\tt $\backslash$dong}        &&\dong\cr
\+&{\tt $\backslash$TM}          &&\TM\cr
\+&{\tt $\backslash$pertenth}    &&\pertenth\cr
\+&{\tt $\backslash$npilcrow}    &&\npilcrow \cr
\+&{\tt $\backslash$baht}        &&\baht\cr
\+&{\tt $\backslash$numero}      &&\numero\cr
\+&{\tt $\backslash$abzueglich}  &&\abzueglich\cr
\+&{\tt $\backslash$aestimated}  &&\aestimated \cr
\+&{\tt $\backslash$openbull}    &&\openbull\cr
\+&{\tt $\backslash$SM}          &&\SM\cr
\+&{\tt $\backslash$qbrackl}     &&\qbrackl\cr 
\+&{\tt $\backslash$qbrackr}     &&\qbrackr \cr
\+&{\tt $\backslash$cent}        &&\cent\cr
\+&{\tt $\backslash$sterling}    &&\sterling\cr
\+&{\tt $\backslash$currency}    &&\currency\cr
\+&{\tt $\backslash$yen}         &&\yen\cr
\+&{\tt $\backslash$brokenvert}  &&\brokenvert\cr 
\+&{\tt $\backslash$feminine}    &&\feminine\cr
\+&{\tt $\backslash$copyleft}    &&\copyleft\cr
\+&{\tt $\backslash$lognot}      &&\lognot \cr
\+&{\tt $\backslash$circledP}    &&\circledP\cr
\+&{\tt $\backslash$registered}  &&\registered\cr
\+&{\tt $\backslash$degree}      &&\degree\cr
\+&{\tt $\backslash$plusminus}   &&\plusminus\cr
\+&{\tt $\backslash$twoup}       &&\twoup\cr
\+&{\tt $\backslash$threeup}     &&\threeup\cr
\+&{\tt $\backslash$micro}       &&\micro\cr
\+&{\tt $\backslash$centereddot} &&\centereddot\cr
\+&{\tt $\backslash$reference}   &&\reference\cr
\+&{\tt $\backslash$oneup}       &&\oneup\cr
\+&{\tt $\backslash$masculine}   &&\masculine\cr
\+&{\tt $\backslash$root}        &&\root\cr
\+&{\tt $\backslash$onequarter}  &&\onequarter\cr
\+&{\tt $\backslash$onehalf}     &&\onehalf \cr
\+&{\tt $\backslash$threequarters} &&\threequarters\cr
\+&{\tt $\backslash$euro}        &&\euro\cr
\+&{\tt $\backslash$texttimes}   &&\texttimes\cr

\medskip
{\twelvepoint
\leftline{\bf History}}
\leftline{\sl Modifications from version 2.1}
Version 2.2 is a minor correction version. Some minor
corrections of the README file.
The main change is the inclusion of the {\tt fontch.pdf} 
documentation file.

\leftline{\sl  Modifications from version 2.0}
Version 2.1 is a minor bug correction version.
The main change is the change in the name of
the cork-encoded LM font files to the new
naming conventions (cork-lmr10 -> ec-lmr10, etc.).
Fontch-2.1 is compatible with LM-1.000 or newer versions.
Older versions of LM fonts are not compatible.

\leftline{\sl Modifications from version 1.3}
Version 2.0 is a mayor revision.
There are many improvements such as

\item{--} Proper handling of smallcaps
\item{--} Support for bold-math
\item{--} Support for AMS fonts
\item{--} Support for blackboard boldface
\item{--} Correction of bugs and scaling.

\medskip
{\twelvepoint
\leftline{\bf Bugs}}
  Comments and bugs reports are welcome at {\sf rmedina@ivic.gob.ve}.

\medskip
{\twelvepoint
\leftline{\bf Copyright}}
 Copyright 2010 Rodrigo Medina\par
The fontch package v2.2, including this manual, may be distributed or modified
under the conditions of the LaTeX Project Public License, version 1.3c.

This software is copyright but you are granted a license which gives you, the
``user'' of the software, legal permission to copy, distribute, and/or modify
the software. However, if you modify the software and then distribute it (even
just locally) you must change the name of the software, or use other technical
means to avoid confusion.

\bye
