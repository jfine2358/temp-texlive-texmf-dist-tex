%%
%% This is file `pgfmolbio.domains.tex',
%% generated with the docstrip utility.
%%
%% The original source files were:
%%
%% pgfmolbio.dtx  (with options: `pmb-dom-tex')
%% 
%% Copyright (C) 2013 by Wolfgang Skala
%% 
%% This work may be distributed and/or modified under the
%% conditions of the LaTeX Project Public License, either version 1.3
%% of this license or (at your option) any later version.
%% The latest version of this license is in
%%   http://www.latex-project.org/lppl.txt
%% and version 1.3 or later is part of all distributions of LaTeX
%% version 2005/12/01 or later.
%% 
\ProvidesFile{pgfmolbio.domains.tex}[2013/08/01 v0.21 Protein domains]


\ProvidesFile{pgfmolbio.domains.tex}[2012/10/01 v0.2 Protein Domains]

\ifluatex
  \RequireLuaModule{pgfmolbio.domains}
  \directlua{pmbSpecialKeys = pgfmolbio.domains.SpecialKeys:new()}
\fi

\def\@pmb@dom@keydef#1#2{%
  \pgfkeyssetvalue{/pgfmolbio/domains/#1}{#2}%
}

\def\pmbdomvalueof#1{%
  \pgfkeysvalueof{/pgfmolbio/domains/#1}%
}

\@pmb@dom@keydef{name}{Protein}
\newif\ifpmb@dom@showname
\pgfmolbioset[domains]{%
  show name/.is if=pmb@dom@showname,
  show name
}
\@pmb@dom@keydef{description}{}

\@pmb@dom@keydef{x unit}{.5mm}
\@pmb@dom@keydef{y unit}{6mm}
\@pmb@dom@keydef{residues per line}{200}
\@pmb@dom@keydef{baseline skip}{3}
\@pmb@dom@keydef{residue numbering}{auto}
\@pmb@dom@keydef{residue range}{auto-auto}
\@pmb@dom@keydef{enlarge left}{0cm}
\@pmb@dom@keydef{enlarge right}{0cm}
\@pmb@dom@keydef{enlarge top}{1cm}
\@pmb@dom@keydef{enlarge bottom}{0cm}

\pgfmolbioset[domains]{%
  style/.code=\pgfmolbioset[domains]{current style/.style={#1}}
}

\@pmb@dom@keydef{domain font}{\footnotesize}

\@pmb@dom@keydef{level}{}
\@pmb@dom@keydef{disulfide base distance}{1}
\@pmb@dom@keydef{disulfide level distance}{.2}
\@pmb@dom@keydef{range font}{\sffamily\scriptsize}

\newif\ifpmb@dom@showruler
\pgfmolbioset[domains]{%
  show ruler/.is if=pmb@dom@showruler,
  show ruler
}
\@pmb@dom@keydef{ruler range}{auto-auto}
\@pmb@dom@keydef{default ruler step size}{50}
\@pmb@dom@keydef{ruler distance}{-.5}

\@pmb@dom@keydef{sequence}{}
\@pmb@dom@keydef{magnified sequence font}{\ttfamily\footnotesize}

\newif\ifpmb@dom@showsecstructure
\pgfmolbioset[domains]{%
  show secondary structure/.is if=pmb@dom@showsecstructure,
  show secondary structure=false
}
\@pmb@dom@keydef{secondary structure distance}{1}
\pgfmolbioset[domains]{%
  helix back border color/.code=\colorlet{helix back border color}{#1},
  helix back main color/.code=\colorlet{helix back main color}{#1},
  helix back middle color/.code=\colorlet{helix back middle color}{#1},
  helix front border color/.code=\colorlet{helix front border color}{#1},
  helix front main color/.code=\colorlet{helix front main color}{#1},
  helix front middle color/.code=\colorlet{helix front middle color}{#1},
  helix back border color=white!50!black,
  helix back main color=white!90!black,
  helix back middle color=white,
  helix front border color=red!50!black,
  helix front main color=red!90!black,
  helix front middle color=red!10!white
}

\@pmb@dom@keydef{sequence length}{}

\@pmb@dom@keydef{@layer}{}

\newcommand\setfeatureshape[2]{%
  \expandafter\def\csname @pmb@dom@feature@#1@shape\endcsname{#2}%
}

\newcommand\setfeatureshapealias[2]{%
  \expandafter\def\csname @pmb@dom@feature@#1@shape\endcsname{%
    \@nameuse{@pmb@dom@feature@#2@shape}%
  }%
}

\ifluatex
  \newcommand\setfeaturestylealias[2]{%
    \directlua{
      if pmbProtein then
        pmbProtein.specialKeys:aliasFeatureStyle("#1", "#2")
      else
        pmbSpecialKeys:aliasFeatureStyle("#1", "#2")
      end
    }%
  }
  \newcommand\setfeaturealias[2]{%
    \setfeatureshapealias{#1}{#2}%
    \setfeaturestylealias{#1}{#2}%
  }
\else
  \let\setfeaturealias\setfeatureshapealias%
\fi

\newcommand\pmbdomdrawfeature[1]{%
  \@ifundefined{@pmb@dom@feature@#1@shape}{%
    \PackageWarning{pgfmolbio}%
      {Feature shape `#1' unknown, using `default'.}%
    \@pmb@dom@feature@default@shape%
  }{%
    \@nameuse{@pmb@dom@feature@#1@shape}%
  }%
}

\setfeatureshape{default}{%
  \path [/pgfmolbio/domains/current style]
    (\xLeft, \yMid + .5 * \pmbdomvalueof{y unit}) rectangle
    (\xRight, \yMid - .5 * \pmbdomvalueof{y unit});
}

\setfeatureshape{domain}{
  \draw [/pgfmolbio/domains/current style, rounded corners=2pt]
    (\xLeft, \yMid + .5 * \pmbdomvalueof{y unit}) rectangle
    (\xRight, \yMid - .5 * \pmbdomvalueof{y unit});
  \node at (\xMid, \yMid)
    {\pmbdomvalueof{domain font}{\pmbdomvalueof{description}}};
}
\setfeaturealias{DOMAIN}{domain}

\setfeatureshape{signal peptide}{%
  \path [/pgfmolbio/domains/current style]
    (\xLeft, \yMid + \pmbdomvalueof{y unit} / 5) rectangle
    (\xRight, \yMid - \pmbdomvalueof{y unit} / 5);
}
\setfeaturealias{SIGNAL}{signal peptide}

\setfeatureshape{propeptide}{%
  \path [/pgfmolbio/domains/current style]
    (\xLeft, \yMid + .5 * \pmbdomvalueof{y unit}) rectangle
    (\xRight, \yMid - .5 * \pmbdomvalueof{y unit});
}
\setfeaturealias{PROPEP}{propeptide}

\setfeatureshape{carbohydrate}{%
  \draw [/pgfmolbio/domains/current style]
    (\xMid, \yMid) --
    (\xMid, \yMid + .7 * \pmbdomvalueof{y unit})
    node [above] {\tiny\strut\pmbdomvalueof{description}};
  \fill [/pgfmolbio/domains/current style]
    (\xMid, \yMid + .7 * \pmbdomvalueof{y unit}) circle [radius=1pt];
}
\setfeaturealias{CARBOHYD}{carbohydrate}

\setfeatureshape{other/main chain}{%
  \ifpmb@dom@showsecstructure%
    \pgfmathsetmacro\yUpper{%
      \yMid + \pmbdomvalueof{secondary structure distance}
        * \pmbdomvalueof{y unit}%
    }
    \draw [thin]
      (\xLeft, \yUpper pt) --
      (\xRight, \yUpper pt);%
  \fi%
  \path [/pgfmolbio/domains/current style]
    (\xLeft, \yMid) --
    (\xRight, \yMid);%
}

\setfeatureshape{other/name}{%
  \ifpmb@dom@showname%
    \node [/pgfmolbio/domains/current style]
      at (\xMid, \pmbdomvalueof{baseline skip} * \pmbdomvalueof{y unit})
      {\pmbdomvalueof{name} (\pmbdomvalueof{sequence length} residues)};
  \fi%
}

\setfeatureshape{disulfide}{%
  \pgfmathsetmacro\yUpper{%
    \yMid + (
      \pmbdomvalueof{disulfide base distance} +
      (\pmbdomvalueof{level} - 1) *
      \pmbdomvalueof{disulfide level distance}
    ) * \pmbdomvalueof{y unit}
  }
  \path [/pgfmolbio/domains/current style]
    (\xLeft, \yMid) --
    (\xLeft, \yUpper pt) --
    (\xRight, \yUpper pt) --
    (\xRight, \yMid);
}
\setfeaturealias{DISULFID}{disulfide}

\setfeatureshape{range}{%
  \pgfmathsetmacro\yUpper{%
    \yMid + (
      \pmbdomvalueof{disulfide base distance} +
      (\pmbdomvalueof{level} - 1) *
      \pmbdomvalueof{disulfide level distance}
    ) * \pmbdomvalueof{y unit}
  }
  \path [/pgfmolbio/domains/current style]
    (\xLeft, \yUpper pt) --
    (\xRight, \yUpper pt)
    node [pos=.5, above]
      {\pmbdomvalueof{range font}{\pmbdomvalueof{description}}};
}

\setfeatureshape{other/ruler}{%
  \draw [/pgfmolbio/domains/current style]
    (\xMid,
      \yMid + \pmbdomvalueof{ruler distance} *
              \pmbdomvalueof{y unit}) --
    (\xMid,
      \yMid + \pmbdomvalueof{ruler distance} *
              \pmbdomvalueof{y unit} - 1mm)
    node [below=-1mm] {\tiny\sffamily\strut\residueNumber};
}

\setfeatureshape{other/sequence}{%
  \node [/pgfmolbio/domains/current style]
    at (\xMid, \yMid) {\strut\currentResidue};
}

\newlength\pmb@magnifiedsequence@width

\setfeatureshape{other/magnified sequence above}{%
  \settowidth\pmb@magnifiedsequence@width{%
    \begin{pgfinterruptpicture}%
      \pmbdomvalueof{magnified sequence font}%
      \featureSequence%
      \end{pgfinterruptpicture}%
    }%
  \pgfmathsetmacro\xUpperLeft{\xMid - \pmb@magnifiedsequence@width / 2}
  \pgfmathsetmacro\xUpperRight{\xMid + \pmb@magnifiedsequence@width / 2}

  \draw [/pgfmolbio/domains/current style]
    (\xLeft, \yMid) --
    (\xLeft, \yMid + \pmbdomvalueof{y unit} / 6) --
    (\xUpperLeft pt, \yMid + \pmbdomvalueof{y unit} * 4/6) --
    (\xUpperLeft pt, \yMid + \pmbdomvalueof{y unit} * 5/6)
    (\xUpperRight pt, \yMid + \pmbdomvalueof{y unit} * 5/6) --
    (\xUpperRight pt, \yMid + \pmbdomvalueof{y unit} * 4/6) --
    (\xRight, \yMid + \pmbdomvalueof{y unit} / 6) --
    (\xRight, \yMid);
  \node [anchor=mid]
    at (\xMid, \yMid + \pmbdomvalueof{y unit})
    {\pmbdomvalueof{magnified sequence font}\featureSequence};
}

\setfeatureshape{other/magnified sequence below}{%
  \settowidth\pmb@magnifiedsequence@width{%
    \begin{pgfinterruptpicture}%
      \pmbdomvalueof{magnified sequence font}%
      \featureSequence%
      \end{pgfinterruptpicture}%
    }%
  \pgfmathsetmacro\xLowerLeft{\xMid - \pmb@magnifiedsequence@width / 2}
  \pgfmathsetmacro\xLowerRight{\xMid + \pmb@magnifiedsequence@width / 2}

  \draw [/pgfmolbio/domains/current style]
    (\xLeft, \yMid) --
    (\xLeft, \yMid - \pmbdomvalueof{y unit} / 6) --
    (\xLowerLeft pt, \yMid - \pmbdomvalueof{y unit}) --
    (\xLowerLeft pt, \yMid - \pmbdomvalueof{y unit} * 7/6)
    (\xLowerRight pt, \yMid - \pmbdomvalueof{y unit} * 7/6) --
    (\xLowerRight pt, \yMid - \pmbdomvalueof{y unit}) --
    (\xRight, \yMid - \pmbdomvalueof{y unit} / 6) --
    (\xRight, \yMid);
  \node [anchor=mid]
    at (\xMid, \yMid - \pmbdomvalueof{y unit} * 8/6)
    {\pmbdomvalueof{magnified sequence font}\featureSequence};
}

\newcommand\@pmb@dom@helixsegment[1]{%
  svg [scale=#1] "%
    c  0.30427  0
       0.62523  0.59174
       0.79543  0.96646
    c  0.97673  2.15039
       1.34005  4.49858
       1.84538  6.6178
    c  0.56155  2.35498
       0.99602  4.514
       1.82948  6.72355
    c  0.11069  0.29346
       0.23841  0.69219
       0.56172  0.69219
    l -5        0
    c -0.27235  0.0237
      -0.55793 -0.51373
      -0.65225 -0.76773
    c -0.98048 -2.64055
      -1.40233 -5.46534
      -2.06809 -8.00784
    c -0.50047 -1.91127
      -0.94696 -3.73368
      -1.68631 -5.43929
    c -0.14066 -0.3245
      -0.34516 -0.78514
      -0.69997 -0.78514
    z"
}

\newcommand\@pmb@dom@helixhalfsegment[1]{%
  svg [scale=#1] "%
    c  0.50663  2.18926
       0.96294  4.51494
       1.78125  6.71875
    c  0.09432  0.254
       0.35265  0.80495
       0.625    0.78125
    l  5        0
    c -0.32331  0
      -0.45181 -0.42529
      -0.5625  -0.71875
    c -0.83346 -2.20955
      -1.2822  -4.36377
      -1.84375 -6.78125
    l -5        0
    z"
}

\pgfdeclareverticalshading[%
    helix back border color,%
    helix back main color,%
    helix back middle color%
  ]{helix half upper back}{100bp}{
  color(0bp)=(helix back middle color);
  color(5bp)=(helix back middle color);
  color(45bp)=(helix back main color);
  color(75bp)=(helix back border color);
  color(100bp)=(helix back border color)
}

\pgfdeclareverticalshading[%
    helix back border color,%
    helix back main color,%
    helix back middle color%
  ]{helix half lower back}{100bp}{
  color(0bp)=(helix back border color);
  color(25bp)=(helix back border color);
  color(35bp)=(helix back main color);
  color(55bp)=(helix back middle color);
  color(95bp)=(helix back main color);
  color(100bp)=(helix back main color)
}

\pgfdeclareverticalshading[%
    helix back border color,%
    helix back main color,%
    helix back middle color%
  ]{helix full back}{100bp}{
  color(0bp)=(helix back border color);
  color(25bp)=(helix back border color);
  color(30bp)=(helix back main color);
  color(40bp)=(helix back middle color);
  color(60bp)=(helix back main color);
  color(75bp)=(helix back border color);
  color(100bp)=(helix back border color)
}

\pgfdeclareverticalshading[%
    helix front border color,%
    helix front main color,%
    helix front middle color%
  ]{helix half upper front}{100bp}{
  color(0bp)=(helix front main color);
  color(5bp)=(helix front main color);
  color(45bp)=(helix front middle color);
  color(65bp)=(helix front main color);
  color(75bp)=(helix front border color);
  color(100bp)=(helix front border color)
}

\pgfdeclareverticalshading[%
    helix front border color,%
    helix front main color,%
    helix front middle color%
  ]{helix full front}{100bp}{
  color(0bp)=(helix front border color);
  color(25bp)=(helix front border color);
  color(40bp)=(helix front main color);
  color(60bp)=(helix front middle color);
  color(70bp)=(helix front main color);
  color(75bp)=(helix front border color);
  color(100bp)=(helix front border color)
}

\setfeatureshape{helix/half upper back}{%
  \ifpmb@dom@showsecstructure%
    \pgfmathsetmacro\yShift{%
      \pmbdomvalueof{secondary structure distance} *
      \pmbdomvalueof{y unit}%
    }
    \draw [shading=helix half upper back]
      (\xLeft, \yMid + \yShift pt)
      \@pmb@dom@helixhalfsegment{\pmbdomvalueof{x unit} / 5};
  \fi%
}

\setfeatureshape{helix/half lower back}{%
  \ifpmb@dom@showsecstructure%
    \pgfmathsetmacro\yShift{%
      \pmbdomvalueof{secondary structure distance} *
      \pmbdomvalueof{y unit}%
    }
    \draw [shading=helix half lower back]
      (\xRight, \yMid + \yShift pt) [rotate=180]
      \@pmb@dom@helixhalfsegment{\pmbdomvalueof{x unit} / 5};
  \fi%
}

\setfeatureshape{helix/full back}{%
  \ifpmb@dom@showsecstructure%
    \pgfmathsetmacro\yShift{%
      \pmbdomvalueof{secondary structure distance} *
      \pmbdomvalueof{y unit}%
    }
    \draw [shading=helix full back]
      (\xMid, \yLower + \yShift pt)
      \@pmb@dom@helixsegment{\pmbdomvalueof{x unit} / 5};
  \fi%
}

\setfeatureshape{helix/half upper front}{%
  \ifpmb@dom@showsecstructure%
    \pgfmathsetmacro\yShift{%
      \pmbdomvalueof{secondary structure distance} *
      \pmbdomvalueof{y unit}%
    }
    \draw [shading=helix half upper front]
      (\xRight, \yMid + \yShift pt) [xscale=-1]
      \@pmb@dom@helixhalfsegment{\pmbdomvalueof{x unit} / 5};
  \fi%
}

\setfeatureshape{helix/full front}{%
  \ifpmb@dom@showsecstructure%
    \pgfmathsetmacro\yShift{%
      \pmbdomvalueof{secondary structure distance} *
      \pmbdomvalueof{y unit}%
    }
    \draw [shading=helix full front]
      (\xMid, \yLower + \yShift pt) [xscale=-1]
      \@pmb@dom@helixsegment{\pmbdomvalueof{x unit} / 5};
  \fi%
}

\definecolor{strand left color}{RGB}{42,127,255}
\definecolor{strand right color}{RGB}{128,179,255}

\setfeatureshape{beta strand}{%
  \ifpmb@dom@showsecstructure%
    \pgfmathsetmacro\yShift{%
      \pmbdomvalueof{secondary structure distance} *
      \pmbdomvalueof{y unit}%
    }
    \draw [/pgfmolbio/domains/current style]
      (\xLeft, \yMid + \pmbdomvalueof{x unit} + \yShift pt) --
      (\xRight - 1.5 * \pmbdomvalueof{x unit},
        \yMid + \pmbdomvalueof{x unit} + \yShift pt) --
      (\xRight - 1.5 * \pmbdomvalueof{x unit},
        \yMid + 1.5 * \pmbdomvalueof{x unit} + \yShift pt) --
      (\xRight, \yMid + \yShift pt) --
      (\xRight - 1.5 * \pmbdomvalueof{x unit},
        \yMid - 1.5 * \pmbdomvalueof{x unit} + \yShift pt) --
      (\xRight - 1.5 * \pmbdomvalueof{x unit},
        \yMid - \pmbdomvalueof{x unit} + \yShift pt) --
      (\xLeft, \yMid - \pmbdomvalueof{x unit} + \yShift pt) --
      cycle;%
  \fi%
}
\setfeaturealias{STRAND}{beta strand}

\setfeatureshape{beta turn}{%
  \ifpmb@dom@showsecstructure%
    \pgfmathsetmacro\yShift{%
      \pmbdomvalueof{secondary structure distance} *
      \pmbdomvalueof{y unit}%
    }
    \pgfmathsetmacro\turnXradius{(\xRight - \xLeft) / 2}%
    \pgfmathsetmacro\turnYradius{\pmbdomvalueof{x unit} * 1.5}%
    \fill [white]
      (\xLeft, \yMid + 1mm + \yShift pt) rectangle
      (\xRight, \yMid - 1mm + \yShift pt);%
    \draw [/pgfmolbio/domains/current style]
      (\xLeft - .5pt, \yMid + \yShift pt) --
      (\xLeft, \yMid + \yShift pt) arc
        [start angle=180, end angle=0,
        x radius=\turnXradius pt, y radius=\turnYradius pt] --
      (\xRight + .5pt, \yMid + \yShift pt);%
  \fi%
}
\setfeaturealias{TURN}{beta turn}

\setfeatureshape{beta bridge}{%
  \ifpmb@dom@showsecstructure%
    \pgfmathsetmacro\yShift{%
      \pmbdomvalueof{secondary structure distance} *
      \pmbdomvalueof{y unit}%
    }
    \draw [/pgfmolbio/domains/current style]
      (\xLeft, \yMid + .25 * \pmbdomvalueof{x unit} + \yShift pt) --
      (\xRight - 1.5 * \pmbdomvalueof{x unit},
        \yMid + .25 * \pmbdomvalueof{x unit} + \yShift pt) --
      (\xRight - 1.5 * \pmbdomvalueof{x unit},
        \yMid + 1.5 * \pmbdomvalueof{x unit} + \yShift pt) --
      (\xRight, \yMid + \yShift pt) --
      (\xRight - 1.5 * \pmbdomvalueof{x unit},
        \yMid - 1.5 * \pmbdomvalueof{x unit} + \yShift pt) --
      (\xRight - 1.5 * \pmbdomvalueof{x unit},
        \yMid - .25 * \pmbdomvalueof{x unit} + \yShift pt) --
      (\xLeft, \yMid - .25 * \pmbdomvalueof{x unit} + \yShift pt) --
      cycle;%
  \fi%
}

\setfeatureshape{bend}{%
  \ifpmb@dom@showsecstructure%
    \pgfmathsetmacro\yShift{%
      \pmbdomvalueof{secondary structure distance} *
      \pmbdomvalueof{y unit}%
    }
    \fill [white]
      (\xLeft, \yMid + 1mm + \yShift pt) rectangle
      (\xRight, \yMid - 1mm + \yShift pt);%
    \draw [/pgfmolbio/domains/current style]
      (\xLeft - .5pt, \yMid + \yShift pt) --
      (\xLeft, \yMid + \yShift pt) --
      (\xMid, \yMid + .5 * \pmbdomvalueof{y unit} + \yShift pt) --
      (\xRight, \yMid + \yShift pt) --
      (\xRight + .5pt, \yMid + \yShift pt);%
  \fi%
}

\ifluatex\else\expandafter\endinput\fi

\newcommand\pmb@dom@inputuniprot[1]{%
  \directlua{
    pmbProtein:readUniprotFile("#1")
    pmbProtein:getParameters()
    pmbProtein:setParameters{
      residueNumbering = "\pmbdomvalueof{residue numbering}"
    }
  }%
}

\newcommand\pmb@dom@inputgff[1]{%
  \directlua{
    pmbProtein:readGffFile("#1")
    pmbProtein:setParameters{
      residueNumbering = "\pmbdomvalueof{residue numbering}"
    }
  }%
}

\newcommand\pmb@dom@addfeature[4][]{%
  \begingroup%
  \pgfmolbioset[domains]{#1}%
  \@pmb@toksa{#1}%
  \directlua{
    pmbProtein:addFeature{
      key = "#2",
      start = "#3",
      stop = "#4",
      kvList = "\luaescapestring{\the\@pmb@toksa}",
      level = tonumber("\pmbdomvalueof{level}"),
      layer = tonumber("\pmbdomvalueof{@layer}")
    }
  }%
  \endgroup%
}

\newif\ifpmb@dom@tikzpicture

\newenvironment{pmbdomains}[2][]{%
  \@ifundefined{useasboundingbox}%
    {\pmb@dom@tikzpicturefalse\begin{tikzpicture}}%
    {\pmb@dom@tikzpicturetrue}%
  \pgfmolbioset[domains]{sequence length=#2, #1}%
  \let\inputuniprot\pmb@dom@inputuniprot%
  \let\inputgff\pmb@dom@inputgff%
  \let\addfeature\pmb@dom@addfeature%
  \directlua{
    pmbProtein = pgfmolbio.domains.Protein:new()
    pmbProtein.specialKeys =
      pgfmolbio.domains.SpecialKeys:new(pmbSpecialKeys)
    pmbProtein:setParameters{
      sequenceLength = "\pmbdomvalueof{sequence length}"
    }
    pmbProtein:setParameters{
      residueNumbering = "\pmbdomvalueof{residue numbering}"
    }
  }%
}{%
  \pmb@dom@addfeature[@layer=1]{other/main chain}%
    {(1)}{(\pmbdomvalueof{sequence length})}%
  \@pmb@toksa=%
    \expandafter\expandafter\expandafter\expandafter%
    \expandafter\expandafter\expandafter{%
      \pgfkeysvalueof{/pgfmolbio/domains/name}%
    }%
  \directlua{
    pmbProtein:setParameters{
      residueRange = "\pmbdomvalueof{residue range}",
      defaultRulerStepSize = "\pmbdomvalueof{default ruler step size}"
    }
    pmbProtein:setParameters{
      name = "\luaescapestring{\the\@pmb@toksa}",
      xUnit = "\pmbdomvalueof{x unit}",
      yUnit = "\pmbdomvalueof{y unit}",
      residuesPerLine = "\pmbdomvalueof{residues per line}",
      baselineSkip = "\pmbdomvalueof{baseline skip}",
      showRuler = "\ifpmb@dom@showruler true\else false\fi",
      rulerRange = "\pmbdomvalueof{ruler range}",
      sequence = "\pmbdomvalueof{sequence}"
    }
    pmbProtein:calculateDisulfideLevels()
    pgfmolbio.setCoordinateFormat(
      "\pgfkeysvalueof{/pgfmolbio/coordinate unit}",
      "\pgfkeysvalueof{/pgfmolbio/coordinate format string}"
    )
    \ifpmb@loadmodule@convert
      local filename =
        "\pgfkeysvalueof{/pgfmolbio/convert/output file name}"
      if filename == "(auto)" then
        filename = "pmbconverted" .. pgfmolbio.outputFileId
      end
      filename = filename ..
        ".\pgfkeysvalueof{/pgfmolbio/convert/output file extension}"
      outputFile, ioError = io.open(filename, "w")
      if ioError then
        tex.error(ioError)
      end
      \ifpmb@con@outputtikzcode
        tex.sprint = function(a) outputFile:write(a) end
        pmbProtein:getParameters()
        tex.sprint("\string\n\string\\begin{tikzpicture}")
        pmbProtein:printTikzDomains()
        tex.sprint("\string\n\string\\end{tikzpicture}")
      \else
        \ifpmb@con@includedescription
          pmbProtein.includeDescription = true
        \fi
        outputFile:write(tostring(pmbProtein))
      \fi
      outputFile:close()
      pgfmolbio.outputFileId = pgfmolbio.outputFileId + 1
    \else
      pmbProtein:printTikzDomains()
    \fi
    pmbProtein = nil
  }%
  \ifpmb@dom@tikzpicture\else\end{tikzpicture}\fi%
}

\newcommand\setdisulfidefeatures[1]{%
  \directlua{
    if pmbProtein then
      pmbProtein.specialKeys:clearKeys("disulfideKeys")
      pmbProtein.specialKeys:setKeys("disulfideKeys", "#1", true)
    else
      pmbSpecialKeys:clearKeys("disulfideKeys")
      pmbSpecialKeys:setKeys("disulfideKeys", "#1", true)
    end
  }%
}

\newcommand\adddisulfidefeatures[1]{%
  \directlua{
    if pmbProtein then
      pmbProtein.specialKeys:setKeys("disulfideKeys", "#1", true)
    else
      pmbSpecialKeys:setKeys("disulfideKeys", "#1", true)
    end
  }%
}

\newcommand\removedisulfidefeatures[1]{%
  \directlua{
    if pmbProtein then
      pmbProtein.specialKeys:setKeys("disulfideKeys", "#1", nil)
    else
      pmbSpecialKeys:setKeys("disulfideKeys", "#1", nil)
    end
  }%
}

\setdisulfidefeatures{DISULFID, disulfide, range}

\newcommand\setfeatureprintfunction[2]{%
  \directlua{
    if pmbProtein then
      pmbProtein.specialKeys:setKeys("printFunctions", "#1", #2)
    else
      pmbSpecialKeys:setKeys("printFunctions", "#1", #2)
    end
  }%
}

\newcommand\removefeatureprintfunction[1]{%
  \directlua{
    if pmbProtein then
      pmbProtein.specialKeys:setKeys("printFunctions", "#1", nil)
    else
      pmbSpecialKeys:setKeys("printFunctions", "#1", nil)
    end
  }%
}

\setfeatureprintfunction{other/sequence}%
  {pgfmolbio.domains.printSequenceFeature}
\setfeatureprintfunction{alpha helix, pi helix, 310 helix, HELIX}%
  {pgfmolbio.domains.printHelixFeature}

\newcommand\setfeaturestyle[2]{%
  \@pmb@toksa{#2}%
  \directlua{
    if pmbProtein then
      pmbProtein.specialKeys:setFeatureStyle(
        "#1", "\luaescapestring{\the\@pmb@toksa}"
      )
    else
      pmbSpecialKeys:setFeatureStyle(
        "#1", "\luaescapestring{\the\@pmb@toksa}"
      )
    end
  }%
}

\setfeaturestyle{default}{draw}
\setfeaturestyle{domain}%
  {fill=Chartreuse,fill=LightSkyBlue,fill=LightPink,fill=Gold!50}
\setfeaturestyle{signal peptide}{fill=black}
\setfeaturestyle{propeptide}%
  {*1{fill=Gold, opacity=.5, rounded corners=4pt}}
\setfeaturestyle{carbohydrate}{red}
\setfeaturestyle{other/main chain}{*1{draw, line width=2pt, black!25}}
\setfeaturestyle{other/name}{font=\sffamily}
\setfeaturestyle{disulfide}{draw=olive}
\setfeaturestyle{range}{*1{draw,decorate,decoration=brace}}
\setfeaturestyle{other/ruler}{black, black!50}
\setfeaturestyle{other/sequence}{*1{font=\ttfamily\tiny}}%
\setfeaturestyle{other/magnified sequence above}%
  {*1{draw=black!50, densely dashed}}
\setfeaturestylealias{other/magnified sequence below}%
  {other/magnified sequence above}
\setfeaturestyle{alpha helix}{%
  *1{helix front border color=red!50!black,%
  helix front main color=red!90!black,%
  helix front middle color=red!10!white}%
}
\setfeaturestylealias{HELIX}{alpha helix}
\setfeaturestyle{pi helix}{%
  *1{helix front border color=yellow!50!black,%
  helix front main color=yellow!70!red,%
  helix front middle color=yellow!10!white}%
}
\setfeaturestyle{310 helix}{%
  *1{helix front border color=magenta!50!black,%
  helix front main color=magenta!90!black,%
  helix front middle color=magenta!10!white}%
}
\setfeaturestyle{beta strand}{%
  *1{left color=strand left color, right color=strand right color}%
}
\setfeaturestyle{beta turn}{*1{draw=violet, thick}}
\setfeaturestyle{beta bridge}{*1{fill=MediumBlue}}
\setfeaturestyle{bend}{*1{draw=magenta, thick}}
\endinput
%%
%% End of file `pgfmolbio.domains.tex'.
