% utf8
% author Alain Matthes d'après des travaux de F Fradin et H Voss  sur un fichier
% 21/01/2012
\makeatletter

% dimensions des motifs
\def\@pgfornamentDim#1{% dim en bp
\ifcase#1\relax%
\or\def\@pgfornamentX{136}\def\@pgfornamentY{107}% 1
\or\def\@pgfornamentX{133}\def\@pgfornamentY{48}%  2
\fi%
}%
% appels des motifs 
\def\pgf@@ornament#1{%
\begingroup
\def\i{\pgfusepath{clip}}
\let\o\pgfpathclose
\let\s\pgfusepathqfillstroke
\def\p ##1##2{\pgfqpoint{##1bp}{##2bp}}
\def\m ##1 ##2 {\pgfpathmoveto{\p{##1}{##2}}}
\def\l ##1 ##2 {\pgfpathlineto{\p{##1}{##2}}}
\def\r ##1 ##2 ##3 ##4 {\pgfpathrectangle{\p{##1}{##2}}{\p{##3}{##4}}}
\def\c ##1 ##2 ##3 ##4 ##5 ##6 {%
\pgfpathcurveto{\p{##1}{##2}}{\p{##3}{##4}}{\p{##5}{##6}}}%
\@@input am#1.pgf 
%\@nameuse{pgf@@am@#1}%
\endgroup}%
\makeatother
\endinput   