%%
%% This is file `apmptbl.tex',
%% generated with the docstrip utility.
%%
%% The original source files were:
%%
%% stepe.dtx  (with options: `apmptbl')
%% 
%%     This work has been partially funded by the US government
%%  and is not subject to copyright.
%% 
%%     This program is provided under the terms of the
%%  LaTeX Project Public License distributed from CTAN
%%  archives in directory macros/latex/base/lppl.txt.
%% 
%%  Author: Peter Wilson (CUA and NIST)
%%          now at: peter.r.wilson@boeing.com
%% 
\ProvidesFile{apmptbl.tex}[2002/01/22 AP mapping table boilerplate]
\typeout{apmptbl.tex [2002/01/22 STEP AP mapping table boilerplate]}

  This clause contains the mapping table that shows how each
UoF and application object of this part of ISO~10303
(see \cref{;sireq}) maps to one or more AIM constructs
(see \aref{;saeel}).
The mapping table is organized in five columns.

 Column 1) Application element: Name of an application
    element as it appears in the application object definition
    in~\ref{;sao}. Application object names are written in uppercase.
    Attribute names and assertions are listed after the application
    object to which they belong and are written in lower case.

 Column 2) AIM element: Name of an AIM element as it
    appears in the AIM (see \aref{;saeel}), the term ``IDENTICAL MAPPING'',
    or the term ``PATH''. AIM entities are written in lower case.
    Attribute names of AIM entities are referred to as
    $<$entity name$>$.$<$attribute name$>$. The mapping of an
    application element may result in several related AIM
    elements. Each of these AIM elements requires a line of its
    own in the table. The term ``IDENTICAL MAPPING'' indicates
    that both application objects of an application assertion
    map to the same AIM element. The term ``PATH'' indicates
    that the application assertion maps to the entire reference
    path.

 Column 3) Source: For those AIM elements that are
    interpreted from the integrated resources or the application
    interpreted constructs, this is the
    number of the corresponding part of ISO~10303. For those
    AIM elements that are created for the purpose of this part
    of ISO~10303, this is the number of this part.
    Entities or types that are defined within the integrated
    resources have an AIC as the source reference if the use
    of the entity or type for the mapping is within the scope
    of the AIC.

 Column 4) Rules: One or more numbers may be given that
    refer to rules that apply to the current AIM element or
    reference path. For rules that are derived from
    relationships between application objects, the same rule
    is referred to by the mapping entries of all the involved AIM
    elements. The expanded names of the rules are listed after
    the table.

 Column 5) Reference path: To describe fully the mapping
    of an application object, it may be necessary to specify a
    reference path through several related AIM elements. The
    reference path column documents the role of an AIM element
    relative to the AIM element in the row succeeding it.
    Two or more such related AIM elements define the
    interpretation of the integrated resources that satisfies
    the requirement specified by the application object.
    For each AIM element that has been created for use within this
    part of ISO~10303, a reference path up to its supertype from
    an integrated resource is specified.

  For the expression of reference paths the following notational
conventions apply:
\begin{enumerate}
\item \verb|[]| : enclosed section constrains multiple AIM elements
    or sections of the
    reference path are required to satisfy an information
    requirement;
\item \verb|()| : enclosed section constrains multiple AIM elements
    or sections of the
    reference path are identified as alternatives within the
    mapping to satisfy an information requirement;
\item \verb|{}| : enclosed section constrains the reference path
    to satisfy an information requirement;
\item \verb|<>| : enclosed section constrains at one or more
     required reference path;
\item \verb+||+ : enclosed section constrains the supertype entity;
\item \verb|->| : attribute references the entity or select type
    given in the following row;
\item \verb|<-| : entity or select type is referenced by the
     attribute in the following row;
\item \verb|[i]| : attribute is an aggregation of which a
     single member is given in the following row;
\item \verb|[n]| : attribute is an aggregation of which
     member \verb|n| is given in the following row;
\item \verb|=>| : entity is a supertype of the entity given in the
    following row;
\item \verb|<=| : entity is a subtype of the entity given in
    the following row;
\item \verb|=| : the string, select, or enumeration type is
    constrained to a choice or value;
\item \verb|\| : the reference path expression continues on
    the next line;
\item \verb|*| : used in conjunction with braces to indicate that any
    number of relationship entity data types may be assembled in a
    relationship tree structure.
\end{enumerate}

\endinput
%%
%% End of file `apmptbl.tex'.
