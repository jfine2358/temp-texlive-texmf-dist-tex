% $StyleId: cwebbase.doc,v 1.4 1995/11/07 17:55:19 schrod Exp $
%----------------------------------------------------------------------
% Written by Joachim Schrod <schrod@iti.informatik.th-darmstadt.de>.
% Copyright conditions see below (GPL).

%
% LaTeX macros cwebbase
% support for incorporation of CWEAVE markup in LaTeX documents
%
% [LaTeX in MAKEPROG]
% (history at end)




%%%%
%%%%
%%%% These TeX macros were documented with the documentation system
%%%% MAKEPROG and automatically converted to the current form.
%%%% If you have MAKEPROG available you may transform it back to
%%%% the original input: Remove every occurence of three percents
%%%% and one optional blank from the beginning of a line and remove
%%%% every line which starts with four percents.  The following lex
%%%% program will do this:
%%%%
%%%%	%%
%%%%
%%%%	^%%%\ ?   ;
%%%%	^%%%%.*\n ;
%%%%
%%%%	If you just want to print the documentation you may fetch
%%%% the archive print-makeprog.tar.Z from ftp.th-darmstadt.de (directory
%%%% pub/tex/latex). It contains *all* used styles -- but beware, they
%%%% may not be in a documented form...
%%%%
%%%%
%%% \documentclass{progltx}

%%% \usepackage{cweb-doc}		% document-specific markup
%%% \usepackage{fullpage}


%%% \RCS $StyleRevision: 1.4 $
%%% \RCS $StyleDate: 1995/11/07 17:55:19 $


%%% \begin{document}


%%% \title{Typesetting \cweave{} Output}
%%% \author{%			% LaTeX does not discard unnecessary glue...
%%%     Joachim Schrod%
%%%     \thanks{%
%%% 	\protect\raggedright
%%% 	TU~Darmstadt, Computer Science Department, WG Systems Programming,
%%% 	Alexanderstr.~10, D-64283~Darmstadt, Germany.
%%% 	Email: \texttt{schrod@iti.informatik.th-darmstadt.de}.%
%%% 	}%
%%%     }
%%% \date{%
%%%     Revision \RCSStyleRevision\\
%%%     (as of \RCSStyleDate)%
%%%     }

%%% \maketitle


%%% % doesn't work with progltx yet
%%% %\tableofcontents




%%% \chap Introduction.

%%% This code shall interpret and render the tag set output by \cweave{}.
%%% That's realized in this module as it's needed in multiple places --
%%% when we want to use \LaTeX{} markup in \cweb{}, when we want to
%%% incorporate \cweb{} pieces in \LaTeX{} documents (as examples, or in
%%% textbooks), and when we want to incorporate whole \cweb{} documents in
%%% a \LaTeX{} document. (That's by far the most difficult task.)

%%% As \cweb{} operates in modes, for obvious reasons our implementation
%%% is based on a finite state automaton (FSA). The whole tag set is
%%% separated into subsets that are valid in certain states.

%%% The complete set of tags does not obey any name conventions, some tags
%%% are even in conflict with standard \LaTeX{} tags (e.g.\ |\\|). Most
%%% tags are used to render code pieces. With the introduction of states
%%% we can restrict the visibilities of these tags to those states where
%%% we typeset program parts. The needed functionality will be available
%%% in cseqs of our namespace, we'll map that to \cweave{}'s names in a
%%% group.

%%% Then we have to implement the rendering of chunks and sections, that's
%%% (more or less) a straight-forward task. Actually, we only set up the
%%% basic code pieces, the integration of section divisions in the actual
%%% \LaTeX{} document is a task of higher-level code; that integration
%%% depends heavily on the type of document. That also implies that we
%%% won't implement front or back matter material (document markup, table
%%% of contents, index, list of refinements, etc.)\ in this module.


%%% \sect Let's identify this module.

%%% \noindent The code below is explained in the implementation
%%% documentation of the \pkg{rcs} package.

%%% \beginprog
\begingroup
    \def\RCSFile#1#2 $#3: #4 #5\endRCS $#6: #7 #8\endRCS{%
	\def\date{#4}\def\id{v#7}%
	\ProvidesFile{#1}[\date\space\id\space #2]%
	}

  \RCSFile{cwebbase.tex}{CWEAVE tags for LaTeX markup}
  $StyleDate: 1995/11/07 17:55:19 $: 9999/00/00 \endRCS
  $StyleRevision: 1.4 $: 0.0 \endRCS
\endgroup
%%% \endprog


%%% \sect This module is part of the supported \textsl{cweb-sty} bundle.
%%% Send bug reports, comments and repairs.

%%% It assumes that the canonical setup of my modules has happened and can
%%% be used herein. (In particular, the underscore is a letter.)


%%% \sect This module reserves the namespace |cwbb|. It assumes that
%%% |\cweb_warning| is bound either to |\PackageWarningNoLine| or
%%% |\ClassWarningNoLine|, depending on the environment.

%%% \beginprog
\ifx \cwbb@loaded\undefined
    \def\cwbb@loaded{$StyleRevision: 1.4 $}
\else
    \cweb_warning{cwebbase}%
        {Some other package already uses namespace `cwbb'}
\fi
%%% \endprog


%%% \sect This is freely distributable software; you can redistribute it
%%% and/or modify it under the terms of the GNU General Public License as
%%% published by the Free Software Foundation; either version~2 of the
%%% License, or (at your option) any later version.

%%% This software is distributed in the hope that it will be useful, but
%%% \textbf{without any warranty}; without even the implied warranty of
%%% \textbf{merchantability} or \textbf{fitness for a particular purpose}.
%%% See the GNU General Public License for more details.

%%% You should have received a copy of the GNU General Public License in
%%% the file |License| along with this package; if not, write to the Free
%%% Software Foundation, Inc., 675~Mass Ave, Cambridge, MA~02139,~USA.




%%% % ------------------------------------------------------------
%%% %
%%% % subdocument: The interface between CWEAVE and TeX
%%% %

%%% \input{cweave-spec}

%%% %
%%% % ------------------------------------------------------------




%%% \chap Interface of \pkg{cwebbase} module.

%%% \pkg{cwebbase} implements most of \cweave{} tags. It sets up tons of
%%% redefinable cseqs, all of them start with |Cweb|.
%%% %
%%% \begin{fixme}
%%%   How do I document them? We need a configuration document anyhow. Do
%%%   I present here just a list of those cseqs realized in this module
%%%   and refer to the config docs? Need to wait until I have it, then
%%%   we'll see how the stuff here will be added.
%%% \end{fixme}


%%% \sect This module supports both hierarchic and flat section
%%% structuring. You have to bind |\cweb@structure| to 0 (hierarchic) or 1
%%% (flat) before importing it, to show which structure is wanted.


%%% \sect This module supports suppression of different document parts.
%%% You have to define flags |@cweb@suppress@|\<foo>|@| with
%%% $\mbox{\<foo>} \in \{ |changehints|, |unchanged|, |format| \}$ to
%%% denote if suppression shall happen.


%%% \sect The counter |secnumdepth| must be~11 if chunks shall be
%%% numbered. That counter doesn't need to be set when this module is
%%% imported, it can be set locally during typesetting \cweb{} code.

%%% The definitions for flat structure assume that |secnumdepth| is~11
%%% during typesetting.


%%% \sect |\part|, |\chapter| (if available), and |\section| are used for
%%% \cweb{} section divisions (those with the lowest ranks). These
%%% commands will change the respective counters globally and will
%%% add toc lines and marks. You might want to prevent that by redefining
%%% |\addcontentsline| and |\mark|.

%%% Table of contents macros are not defined for the section divisions
%%% introduced by this module. You must either prevent that they are
%%% written to the |AUX| file at all; or you must define them.
%%% %
%%% \begin{fixme}
%%%   Add explanation of introduced section divisions.
%%% \end{fixme}

%%% For flat structure, all section divisions are of type `|chunk|', like
%%% chunks. The section names start with `|\cwbbRank{|\<rank>|}|'. By
%%% default, |\cwbbRank| takes one argument and expands to the empty list
%%% -- this can be redefined for the table of contents.


%%% \sect The document end, as explained in the \cweave{} interface specs
%%% (chunk~\ref{spec:sec:docend}), is not implemented. That must be done
%%% by the user of this module -- if the back matter is to be processed at
%%% all.




%%% \chap Processing states.

%%% We have to typeset five different categories of material:
%%% Documentation, program pieces---embedded within the documentation and
%%% as large chunks, \TeX{} material within program pieces (i.e., comments
%%% and refinement names), and cross reference information. Since we need
%%% a complete other environment for the program pieces than for the rest
%%% we design ``states'' where we switch to appropriately.
%%% %
%%% \begin{enumerate}

%%% \item A chunk starts in the \textsl{documentation state}.

%%% \item |\B| switches to \textsl{program state}. This can happen in
%%% documentation and program state.

%%% \item While we process the argument of |\PB| we're in \textsl{restricted
%%% program state}; |\PB| may appear in documentation and in \TeX{}
%%% state. Since \TeX{} state can be switched on within restricted
%%% program state, |\PB| can appear within the argument of itself.

%%% \item In the arguments of |\C|, |\SHC|, and |\X| we switch to
%%% \textsl{\TeX{} state}. All these cseqs appear only in (restricted)
%%% program state, their official names are actually different. I.e., only
%%% in (restricted) program state these cseq are bound to the meaning
%%% described here.

%%% \item Cross reference information are attached to most chunks with
%%% refinements. This information is processed in \textsl{CR state}.
%%% After CR state material
%%% comes always the next chunk or the document end, i.e., material in
%%% documentation state.

%%% \end{enumerate}
%%% %
%%% This FSA is illustrated by the diagram in figure~\ref{fig:fsa}.

%%% \begin{figure}

%%% \begin{center}
%%% \chardef\\=`\\
%%% \input{cweb-fsa.latex}
%%% \end{center}

%%% \caption{The processing state's FSA. The automaton starts and ends in
%%% the ``documentation'' state.}
%%% \label{fig:fsa}

%%% \end{figure}

%%% |\cwbb@documentation| will switch to documentation state,
%%% |\cwbb@program| to program state, |\cwbb@Rprogram| to restricted
%%% program state, |\cwbb@tex| to \TeX{} state, and |\cwbb@CR| to CR state.
%%% If we're already in a state, the switch to this state shall be a
%%% permissible null operation.


%%% \sect The basic difference between these states can be named with
%%% two parameters: (1)~The cseq bindings in effect and (2)~the layout
%%% parameters used for paragraph makeup.

%%% In (restricted) program state and in CR state the text is output
%%% under the control of \cweave{}, and tagged by \cweave{}. The used
%%% tags are from a global namespace and should only be in effect during
%%% these states. We call this tag set the \textsl{\cweave{} bindings}. In the
%%% other two states the tags are largely defined by the user, the tag
%%% set is called the \textsl{user bindings}. The switch to another binding
%%% is always done locally, i.e., if we switch from documentation to
%%% restricted program state within a group we don't have to bother about
%%% the restauration of the user binding; it will be done automatically
%%% by \TeX{} at the end of the group. Nevertheless we must be able to
%%% switch from the \cweave{} bindings back to the user bindings which
%%% were in effect when we activated the \cweave{} bindings. This is
%%% needed for the \TeX{} state which is always activated within
%%% (restricted) program state.

%%% The parameters for \textsl{program layout} are really special ones
%%% since they need to support the indentation which shows the program
%%% structure. These parameters are used in program and in \TeX{} state.
%%% The \textsl{document layout} parameters established by the user are
%%% used in the other three states.

%%% The following table shall summarize this. $C$~denotes \cweave{}
%%% bindings, $U$~user bindings, $P$~program layout, and $D$~document
%%% layout. If an entry is empty, its value is not changed on entry in
%%% this state.
%%% %
%%% \begin{center}
%%%   \DeleteShortVerb\|
%%% \begin{tabular}{r|c|c}
%%% \multicolumn1{c|}{\textsc{State}} & \textsc{Binding} & \textsc{Layout} \\[1pt]
%%% \hline
%%% documentation & $U$ & $D$ \\
%%% program & $C$ & $P$ \\
%%% restricted program & $C$ & \\
%%% \TeX{} & $U$ & \\
%%% CR & $C$ & $D$ \\
%%% \end{tabular}
%%%   \MakeShortVerb\|
%%% \end{center}


%%% \sect Since the user bindings and the document layout is defined
%%% initially, we don't have to do anything if it's requested. Only if we
%%% change it, i.e., within |\cwbb@cweave_bindings| and
%%% |\cwbb@program_layout|, we redefine |\cwbb@user_bindings| and
%%% |\cwbb@doc_layout|. If they are eventually executed, they shall
%%% rebind themselves back to |\relax|. This way we can switch to
%%% documentation state as often as we want.

%%% \beginprog
\let\cwbb@user_bindings=\relax
\let\cwbb@doc_layout=\relax

\def\cwbb@documentation{%
    \cwbb@user_bindings
    \cwbb@doc_layout
    }
\def\cwbb@Rprogram{\cwbb@cweave_bindings}

\def\cwbb@program{%
    \cwbb@cweave_bindings
    \cwbb@program_layout
    }
\def\cwbb@tex{\cwbb@user_bindings}

\def\cwbb@CR{%
    \cwbb@cweave_bindings
    \cwbb@doc_layout
    }
%%% \endprog




%%% \chap Saving and restoring control sequences.

%%% We have a lot of cseqs which are defined within the namespace of this
%%% module and which will be used with other names. This usage is in a
%%% controlled environment, namely neither in documentation nor in \TeX{}
%%% state. (I.e., the text processed is tagged by \cweave{}, not by humans;
%%% therefore we have a precise specification of the cseqs we have to
%%% accept.) We cannot work with groups where a cseq is just redefined
%%% and \TeX{} takes care for establishing the old binding again; when we
%%% switch from program state to \TeX{} state all the bindings which were
%%% in effect before the program state get active, i.e., in the
%%% documentation state, must be in effect again. We cannot simply
%%% consider the \TeX{} state as something parallel to program state, it
%%% must be a hierarchical relationship: In the program state values are
%%% set up which must be available after switching back from \TeX{} to
%%% program state.

%%% We save the current binding of a cseq in another cseq, but only if
%%% there exists a binding currently. This is done to save valuable
%%% \TeX{} main memory. (Why don't all those people who talk about
%%% extending \TeX{} not just implement dynamic memory management for the
%%% existing \TeX{}? It would still be \TeX{} and would \emph{really} be
%%% a worthy activity. But not enough fun, I assume\,\dots) Actually, one
%%% can assume that nearly no cseq bindings must be saved at all---the
%%% used names are strange enough. The bindings of |\foo|, i.e., of the
%%% cseq with the name \fbox{\ttfamily foo}, is saved as the cseq with the
%%% name \fbox{\ttfamily cweb\char `\\ foo}, i.e., as
%%% |\csname cweb\string\foo \endcsname|.


%%% \sect The save process is not done statically, but by the macro
%%% |\cwbb@rebind| which interprets a list of tuples $({\it old\_name},
%%% {\it new\_name})$, terminated by the tuple $(|\stop|, |\stop|)$.
%%% Eventually it constructs two new lists, |\cwbb@to_restore| with the
%%% cseqs which had a binding, and |\cwbb@undefined| with the names which
%%% didn't have one.

%%% The saving is actually done by |\cwbb@save_binding|, |\cwbb@rebind|
%%% is responsible for the initialization, |\cwbb@do_rebind| for the
%%% effective rebinding and the tail recursion on the list.

%%% \begin{fixme}
%%%   We could pull the |\next| assignment in the |\else| branch out of
%%%   the loop to get a better performance. Should measure if this is of
%%%   interest.
%%% \end{fixme}

%%% \beginprog
\newtoks\cwbb@undefined
\newtoks\cwbb@to_restore

\def\cwbb@rebind{%
    \cwbb@undefined{}%
    \cwbb@to_restore{}%
    \cwbb@do_rebind
    }
\def\cwbb@do_rebind#1#2{%
    \ifx #1\stop
	\let\next\relax
    \else
	\cwbb@save_binding #2%
	\let #2=#1%
	\let\next\cwbb@do_rebind
    \fi
    \next
    }
%%% \endprog


%%% \sect If the cseq to be saved is undefined, it may just be added to
%%% the ``undefined list.'' Otherwise its binding is saved and it's added
%%% to the ``to-be-restored list.''

%%% \TeX{}nical note: The cseq-name for the saved binding must be created
%%% before the |\let| is executed.

%%% \beginprog
\def\cwbb@save_binding#1{%
    \ifx #1\undefined
	\cwbb@undefined \expandafter{\the\cwbb@undefined #1}%
    \else
	\expandafter\let \csname cweb\string#1\endcsname =#1%
	\cwbb@to_restore \expandafter{\the\cwbb@to_restore #1}%
    \fi
    }
%%% \endprog


%%% \sect The restoration of rebound cseqs is a two-tied activity: All
%%% previously undefined cseqs must be made undefined again, and all saved
%%% cseqs must be restored. Actually, we don't need to reset the two token
%%% lists, but we do it to save space.

%%% Both |\cwbb@undefine| and |\cwbb@restore_binding| iterate over a list
%%% of cseqs terminated by |\stop|.

%%% \begin{fixme}
%%%   And here the |\next| assignment could be prepended to the loop as
%%%   well.
%%% \end{fixme}

%%% \beginprog
\def\cwbb@restore_bindings{%
    \expandafter\cwbb@undefine \the\cwbb@undefined \stop
    \cwbb@undefined{}%
    \expandafter\cwbb@restore_binding \the\cwbb@to_restore \stop
    \cwbb@to_restore{}%
    }

\def\cwbb@undefine#1{%
    \ifx #1\stop
	\let\next\relax
    \else
	\let#1\undefined
	\let\next\cwbb@undefine
    \fi
    \next
    }
%%% \endprog


%%% \sect \TeX{}nical note: As in |\cwbb@save_binding|, the cseq-name for
%%% the saved binding must be created before the |\let|s are executed.

%%% \begin{fixme}
%%%   Another |\next| assignment.
%%% \end{fixme}

%%% \beginprog
\def\cwbb@restore_binding#1{%
    \ifx #1\stop
	\let\next\relax
    \else
	\expandafter\let \expandafter#1\csname cweb\string#1\endcsname
	\expandafter\let \csname cweb\string#1\endcsname \undefined
	\let\next\cwbb@restore_binding
    \fi
    \next
    }
%%% \endprog




%%% \chap Chunks.

%%% All chunks are numbered subsequently. The chunk number is supplied as
%%% part of the markup. In addition, the `changed' tag |\*| may be
%%% appended to the chunk number.

%%% By default, chunks have approximately two picas vertical space in
%%% front.

%%% If we use an hierarchic structure, we output the chunk number at the
%%% refinement. (It's only needed to identify the refinement for cross
%%% references.) But chunks are important structural units and thus should
%%% be marked by typographic means that's more visible than the space
%%% mentioned above, we add a `chunk start marker'.

%%% For a flat structure, we output the chunk number as the chunk start
%%% marker, that is the traditional \WEB{} layout.

%%% In any case, the documentation part shall be run in after the chunk
%%% start marker.

%%% \beginprog
\newskip\CwebChunkPreSkip
	\CwebChunkPreSkip=2pc plus 1pc minus 6pt
%%% \endprog


%%% \sect In \LaTeX{}, structural divisions are best realized by
%%% |\@startsection|, we'll use it also for chunk starts. The chunk start
%%% marker is the section heading. For hierarchic structure, we use a
%%% traditional paragraph sign, used since medieval times to tag new text
%%% units. Our first test showed that the paragraph sign has typically an
%%% underlength, that doesn't look good at the start of a chunk. We raise
%%% the sign by its depth.

%%% For flat structure, we don't have text as the section heading,
%%% but we have to cope for a deficiency of \LaTeX{}: It always leaves a
%%% quad after the section number (by the way, this is not mentioned in
%%% the documentation---you have to look in the macros to figure this
%%% out). So we use a backskip of $\rm -1\,em$ as the chunk start marker
%%% then.

%%% Changed chunks are marked by a changeflag, by default a star lapping
%%% to the right. We need the information if that chunk is changed to
%%% decide if we shall output the changeflag. We demand from the chunk
%%% macros that they record that info in the flag |@cwbb_changed_chunk@|.

%%% Each structural division must be categorized, it belongs to a
%%% \textsl{type}. This type determines also the name of the counter, the
%%% mark handling for this category, and how this section tag is
%%% represented in the table of contents. The type of chunks is `|chunk|'
%%% (trivial, isn't it?).

%%% Furthermore, a structural division in \LaTeX{} has an associated
%%% \textsl{level}. That level is used to determine if that division get
%%% numbered in text or table of contents. For sections, we'll compute the
%%% level from the rank, it will not be larger than the rank.%
%%%     \footnote{Check the next section for a more precise definition of
%%% 	`rank' and `level'.}
%%% Since the highest section rank is~10, we may use 11 as chunk level.

%%% For flat structure, we output the number for all section headings,
%%% including chunks. For hierarchic structure we use the default maximum
%%% level as supplied by our document class (e.g., 3~for \cls{article},
%%% and 2~for \cls{report} or \cls{book}).

%%% As a last point, we have to care already for sections in flat
%%% structure. They will use the |chunk| category as well, we have to
%%% distinguish them in the table of contents. For that, we'll record the
%%% level in the text, it will get written to the table of contents.

%%% \beginprog
\newif\if@cwbb_changed_chunk@
\let\CwebChangeFlag=*
\if@cweb@suppress@changehints@
    \let\cwbb@change_flag=\relax
\else
    \let\cwbb@change_flag=\CwebChangeFlag
\fi
\def\cwbb@lap_change_flag{\rlap{\cwbb@change_flag}}

\ifcase \cweb@structure
    %% hierarchic
    \let\CwebChunkHeadingStyle\relax
    \newbox\cweb@chunk_marker
    \setbox\cweb@chunk_marker=\hbox{\P}	% tradition from medivial times
    \setbox\cweb@chunk_marker=\hbox{%
	\raise\dp\cweb@chunk_marker \box\cweb@chunk_marker
	}
    \def\CwebChunkHeadingMarker{\copy\cweb@chunk_marker}
    \def\CwebChunkHeading#1{%
	\CwebChunkHeadingMarker
	\if@cwbb_changed_chunk@ \cwbb@lap_change_flag \fi
	}
  \or
    %% flat
    \def\cwbbRank#1{}
    \def\CwebChunkHeadingStyle{\bfseries}
    \def\CwebChunkHeading#1{%
	\protect\cwbbRank{#1}%
	\kern -1em
	\if@cwbb_changed_chunk@ \cwbb@lap_change_flag \fi
	}
\fi
%%% \endprog


%%% \sect Chunks are tagged with |\M| and get passed the chunk number as
%%% argument. Before we evaluate |\@startsection|, we'll start the chunk.
%%% That's common code to be used by sections as well, as they also
%%% implicitely start a chunk.

%%% \beginprog
\def\M#1{%
    \cwbb@start_chunk{#1}%
    \@startsection
	{chunk}%		% category, == counter etc.
	{11}%			% level
	{\z@}%			% no indentation for heading
	{\CwebChunkPreSkip}%	% skip above heading
	{\m@ne em}%		% run-in heading, 1em distance to text
	{\normalfont\CwebChunkHeadingStyle}%	% layout
	{\CwebChunkHeading{11}}% % text, level might go to toc file
    }
%%% \endprog


%%% \sect But let's first declare how a chunk number is represented. The
%%% principle rendering is controlled by |\thechunk|, namely, we get
%%% arabic numbers.  This rendering is used both for creation of the
%%% section header (if in flat mode) and for cross references. I.e.,
%%% |\ref| expands into it.

%%% But for flat structure we need different expansions: In the section
%%% header a period shall be appended, a cross reference shall appear
%%% without such a period. We use two different macros: |\thechunk|
%%% creates the number with the period, |\cwbb@refchunk| only the number.
%%% The latter is used for a redefinition of |\@currentlabel|. Of course,
%%% for hierarchic structure, both are the same.

%%% \beginprog
\newcounter{chunk}

\def\cwbb@refchunk{\arabic{chunk}}

\ifcase \cweb@structure
    %% hierarchic
    \let\thechunk=\cwbb@refchunk
  \or
    %% flat
    \def\thechunk{\arabic{chunk}.}
\fi
%%% \endprog


%%% \sect Each chunk is started by |\cwbb@start_chunk|, which does the
%%% common actions. First of all, we switch to documentation state.

%%% Then we store the chunk number in the |chunk| counter. That implies
%%% that the chunk start must not step this counter! We'll take care for
%%% that later by redefining |\refstepcounter|. The actual number is
%%% passed as the argument, including an optional |\*| used to denote
%%% changed chunks. We'll redefine that cseq to update the flag
%%% |@cwbb_changed_chunk@|.

%%% Each chunk must be typeset with an open |\if| since it is finished by
%%% |\fi|. If we shall suppress the output of unchanged chunks, we have
%%% to use |\iffalse| if not |@cwbb_changed_chunk@|, otherwise |\iftrue|.
%%% The flag |@cwbb_print_chunk@| is set up to provide an appropriate
%%% test.

%%% When we typeset a chunk, we evaluate |\cwbb@init_print_chunk|. For
%%% hierarchic structure, that cseq will be used to initialize the
%%% typesetting of the chunk number at program parts.

%%% \beginprog
\if@cweb@suppress@unchanged@
    \def\if@cwbb_print_chunk@{\if@cwbb_changed_chunk@}
    \@gobble\fi				% close \if in case it's skipped
\else
    \let\if@cwbb_print_chunk@=\iftrue
    \@gobble\fi				% same here
\fi

\def\cwbb@start_chunk#1{%
    \cwbb@documentation
    \begingroup
        \def\*{\global\@cwbb_changed_chunk@true}%
        \global\@cwbb_changed_chunk@false
        \global \c@chunk #1\relax	% might expand \*
    \endgroup
    \if@cwbb_print_chunk@
        \cwbb@init_print_chunk
    }
%%% \endprog


%%% \sect |\@startsection| will step the section counter (i.e., |chunk|).
%%% Since we just assigned it in |\cwbb@start_chunk|, that would leave us
%%% with a counter that's one too high. And our output in cross references
%%% may be wrong, as a period is appended in flat structure.

%%% That damage all comes from one macro, |\refstepcounter|; let's
%%% advice it to do the Right Thing for |chunk| counters.

%%% \beginprog
\let\cwbb@refstepcounter=\refstepcounter

\def\cwbb@string@chunk{chunk}
\def\refstepcounter#1{%
    \def\@tempa{#1}%
    \ifx \@tempa\cwbb@string@chunk
        \protected@edef\@currentlabel{\cwbb@refchunk}%
    \else
    	\cwbb@refstepcounter{#1}%
    \fi
    }
%%% \endprog




%%% \chap Sections.

%%% Important sections are started on a new page, normal sections have the
%%% same space as chunks in front. The very first section does not
%%% automatically start on a new page since a title may be in front of it.

%%% Each section division has a \textsl{rank}, that denotes the section
%%% depth as seen by \cweave{}.

%%% For this depth we compute a \textsl{level}, used to denote the section
%%% depth for \LaTeX{}. This section depth is used to decide where
%%% numbering shall stop, or if that entry shall be presented in the table
%%% of contents, etc.

%%% These two numbers are not the same, due to the support of document
%%% classes with chapters.

%%% We consider a section as important if it's rank is smaller than~3,
%%% i.e., then we'll start a new page. This limit is controlled by the
%%% number |\CwebRankNoEject|. The rank is used as the section level
%%% and is therefore available to configure depth of typeset numbering and
%%% appearance in table of contents.

%%% \beginprog
\def\CwebRankNoEject{3}
%%% \endprog


%%% \sect Sections are started with the macro |\N|. It's first parameter
%%% is the section's rank. The second parameter is the chunk number, the
%%% third the title of the section. This last parameter must be terminated
%%% by a dot.

%%% Each section implicitly starts a chunk. As explained above, major
%%% sections may eject the page, and we add our actual section heading
%%% with |\CwebSection|.

%%% We also tell the user that we have reached the next section by telling
%%% him about the current chunk number. The message is output after the
%%% section header is set, it shall be on the correct page. Since the
%%% token list to be output is expanded at shipout time, we must take care
%%% for the immediate expansion ourselves.

%%% \beginprog
\def\N#1#2#3.{%
    \cwbb@start_chunk{#2}%
    \CwebEjectSection{#1}%
    \CwebSection{#1}{#3}%
    \begingroup
        \def\*{}%
        \edef\tmp{*#2}%
        \message \expandafter{\tmp}%
    \endgroup
    }
%%% \endprog


%%% \sect At the start of a section we usually eject the page if our rank
%%% is not too low. But if the user had a title above the first section,
%%% he won't want this page break. So we try to guess it. If we guess
%%% wrong, the user may either add |\newpage| himself or he may redefine
%%% |\CwebEjectSection| in a package.

%%% \beginprog
\def\CwebEjectSection#1{%		% #1 == rank
    \gdef\CwebEjectSection##1{%
	\relax
        \ifnum ##1<\CwebRankNoEject\relax
            \newpage
        \fi
	}%
    }

\if@titlepage
    \CwebEjectSection{}
\fi
%%% \endprog


%%% \sect Now comes a heary piece. We have 11~section ranks, and need 11
%%% different section categories, section counters, etc. For flat
%%% structure, they shall all do the same; for hierarchic structure, they
%%% shall be really different.

%%% \noindent Let's have a look at the hierarchic structure first, it's
%%% more complex.
%%% %
%%% \begin{itemize}

%%% \item We associate section rank~0 (that's `|@**|') with document
%%%   parts, i.e., map it to |\part|. Many smaller programs won't use it.

%%% \item If we are used in a class that has a bound cseq |\chapter|, we
%%%   associate rank~1 (`|@*0|') with it and rank~2 (`|@*1|') with
%%%   |\section|. Otherwise we associate rank~1 with |\section|. These
%%%   tags (|\chapter| and/or |\section|) must be defined because many
%%%   macros that want to create section divisions will use them.

%%% \item All other section ranks are placed below |\section|.
%%%   |\subsection| and other \LaTeX{} style markup is not supported.

%%%   For hierarchic structure, we identify these sections with category
%%%   names `|section|$l$', where $l$ is the respective level, computed
%%%   from the rank. Those section names do not exist yet, we have to
%%%   create them.

%%% \end{itemize}

%%% Now we come to the point why we can't use the rank as the section
%%% level passed to \LaTeX{}: When we assume the level usage of \LaTeX{}'s
%%% standard classes, |\section| is \emph{always} on level~1, independent
%%% of the existence of |\chapter|. The latter is placed on level~0, its
%%% existence only influences the section level of |\part|, which is
%%% either $-1$ or~0.

%%% This implies that |\subsection| in a standard class is always on
%%% level~2, but our equivalent section division is either `|@*0|'
%%% (rank~1) or `|@*1|' (rank~2). I.e., when we have chapters, we use a
%%% level that's one lower than the rank, otherwise we use the rank.

%%% For flat structure, we use type |chunk|. This way we'll use the chunk
%%% numbers, they are already set up.

%%% For flat structure, we will map all ranks to the same section markup.
%%% This markup will also be accessible by |\section| or |\chapter|. They
%%% will be declared to be on a level that is the same as the rank.


%%% \sect The printed representation of a section divison number is simply
%%% the presentation from the next upper level, a full stop, and an arabic
%%% section number. When we compute the name of the previous level we must
%%% pay attention to our first section division created this way. This
%%% division must be placed below |section|, without any following number.

%%% Since registers are a valuable resource in \TeX{}, we have to take
%%% care not to allocate a superfluous one.

%%% \beginprog
\ifcase \cweb@structure
    %% hierarchic
    \newcounter{section2}[section]
    \@namedef{thesection2}{\@nameuse{thesection}.\arabic{section2}}
    \newcounter{section3}[section2]
    \@namedef{thesection3}{\@nameuse{thesection2}.\arabic{section3}}
    \newcounter{section4}[section3]
    \@namedef{thesection4}{\@nameuse{thesection3}.\arabic{section4}}
    \newcounter{section5}[section4]
    \@namedef{thesection5}{\@nameuse{thesection4}.\arabic{section5}}
    \newcounter{section6}[section5]
    \@namedef{thesection6}{\@nameuse{thesection5}.\arabic{section6}}
    \newcounter{section7}[section6]
    \@namedef{thesection7}{\@nameuse{thesection6}.\arabic{section7}}
    \newcounter{section8}[section7]
    \@namedef{thesection8}{\@nameuse{thesection7}.\arabic{section8}}
    \newcounter{section9}[section8]
    \@namedef{thesection9}{\@nameuse{thesection8}.\arabic{section9}}
    \ifx \chapter\undefined
        \newcounter{section10}[section9]
	\@namedef{thesection10}{\@nameuse{thesection9}.\arabic{section10}}
    \fi
\fi
%%% \endprog


%%% \sect Sections are typeset by |\CwebSection|; rank and title are the
%%% arguments.

%%% For hierarchic structure, we select the proper cseq according to rank:
%%% |\part|, |\chapter| if this is a structure with chapters, |\section|,
%%% and |\cwbb@section{|\<rank>|}| for other ranks. For the last cseq we
%%% also have to supply the section category, it's `|section|$l$' where
%%% $l$ is the rank~$r$ if we don't have chapters, $r-1$ otherwise.

%%% That category cannot be computed by |\cwbb@section| itself since we
%%% want to use that macro also for flat structure. But there all sections
%%% are basically chunks, we have to use the |chunk| category there. Since
%%% we want to differentiate them in the table of contents nevertheless,
%%% we record the current rank in the title.

%%% \beginprog
\ifcase \cweb@structure
    %% hierarchic
    \ifx \chapter\undefined
        \def\CwebSection#1#2{%
	    \ifcase #1
	        %% 0
	        \let\next\part
	      \or  %% 1
		\let\next\section
	    \else  %% level = rank
		\def\cwbb@sect_name{section#1}%
		\def\next{\cwbb@section{#1}}%
	    \fi
	    \next{#2}%
	    }
    \else
        \def\CwebSection#1#2{%
	    \ifcase #1
	        %% 0
	        \let\next\part
	      \or  %% 1
		\let\next\chapter
	      \or  %% 2
		\let\next\section
	    \else  %% level = rank - 1
		\count@=#1  \advance \count@\m@ne
		\edef\cwbb@sect_name{section\number\count@}%
		\edef\next{\noexpand\cwbb@section{\number\count@}}%
	    \fi
	    \next{#2}%
	    }
    \fi
  \or
    %% flat
    \def\CwebSection#1#2{%
	\def\cwbb@sect_name{chunk}%
	\cwbb@section{#1}{\protect\cwbbRank{#1}#2}%
	}
    \def\section{%
	\def\cwbb@sect_name{chunk}%
	\cwbb@section{1}%
	}
\fi
%%% \endprog


%%% \sect We use the same layout for all minor sections. That layout can
%%% be changed by ammended by defining |\CwebSect|$r$|Hook| to redefine
%%% |\CwebSectPreSkip| and/or |\CwebSectLayout| locally. (Of couse, it may
%%% be redefined once globally, too.) These are the values to be used for
%%% arguments 4 and~5 of |\@startsection|, pay attention to the skip sign,
%%% telling about indentation of the first paragraph. And don't forget
%%% that the skip is only used for sections on ranks below
%%% |\CwebRankNoEject|, the others start on a new page anyhow.

%%% The default definition sets the section title in bold and leaves the
%%% same space as before chunks. The first paragraph of the following
%%% documentation has no indentation.

%%% Other layout changes (indentation for heading, etc.)\ must be made by
%%% redefining |\cwbb@section|. Or---email me, if you want to do it. Then
%%% I might be enclined to introduce further possibilities.

%%% \beginprog
\newskip\CwebSectPreSkip
	\CwebSectPreSkip = -\CwebChunkPreSkip
\def\CwebSectLayout{\normalfont\bfseries}

\def\cwbb@section#1{%			% #2 (next token) is title
    \csname CwebSect#1Hook\endcsname	% \relax if undefined
    \expandafter\@startsection
	\expandafter{\cwbb@sect_name}%	% category; for counter, etc.
	{#1}%				% level is rank
	\z@				% no indentation for heading
	\CwebSectPreSkip		% skip above, maybe no next par indent
	\medskipamount			% skip below heading
        \CwebSectLayout
    }
%%% \endprog




%%% \chap Marks.

%%% \begin{fixme}
%%%   Since I don't know yet how to handle marks I ignore most of them for
%%%   the time being.

%%%   Have to change this later.
%%% \end{fixme}

%%% \beginprog
\let\chunkmark\@gobble

\count@=2
\@whilenum \count@<11 \do {
    \expandafter\let \csname section\number\count@ mark\endcsname \@gobble
    \advance \count@ by 1
    }
%%% \endprog




%%% \chap Typesetting programs.

%%% Program pieces come in two flavours: as argument of |\PB|, or as
%%% material after |\B|. In the former case we can use a group for the
%%% switch to the restricted program state, the group end will restore the
%%% previous state again. In the latter case we use |\B| for the switch to
%%% program state, the cross reference list or the next chunk will go to
%%% another state. Since we have to do more for the material after |\B|,
%%% we define this cseq later.

%%% \beginprog
\def\PB#1{%
    \begingroup
	\cwbb@Rprogram
	\leavevmode
	#1%
    \endgroup
    }
%%% \endprog


%%% \sect Note, that |\Y| cannot just be |\smallskip|, as in Plain
%%% \cweb{}. We must assert that the current paragraph is ended before the
%%% vertical glue is inserted, and that's not done by the \LaTeX{}
%%% definition of |\smallskip|. In addition we add some negative penalty,
%%% here is a good place for a page break. As the penalty value we use
%%% half the chunk break penalty---of course a chunk start is an even
%%% better place for a page break\,\dots

%%% It might be that |\Y| is the very first token in a chunk. Then the
%%% chunk start marker isn't set already since it is regarded as a run-in
%%% section heading. In that case we don't set the vertical skip but
%%% simply start the chunk with the program part.

%%% \TeX{}nical note: The flag |@noskipsec| may be used to test if we're
%%% immediately after a run-in section heading.

%%% \beginprog
\newcount\CwebProgPenalty
	\CwebProgPenalty=\@secpenalty
	\divide\CwebProgPenalty by 2
\def\Y{%
    \if@noskipsec
    \else
	\par
	\penalty\CwebProgPenalty
	\smallskip
    \fi
    }
%%% \endprog


%%% \sect Yep, let's unfold the ``official'' names of the cseqs used in
%%% program state.

%%% If we turn on the \cweave{} bindings, they might be in effect already,
%%% we don't need to establish them again. We can test
%%% |\cwbb@user_bindings| for this case, it will be redefined then.

%%% \beginprog
\def\cwbb@cweave_bindings{%
    \ifx \cwbb@user_bindings\relax
	\cwbb@rebind
	    % indentation and paragraph layout
	    \cwbb@incr_indent	\1%
	    \cwbb@decr_indent	\2%
	    \cwbb@expr_break	\3%
	    \cwbb@backup	\4%
	    \cwbb@opt_break	\5%
	    \cwbb@break		\6%
	    \cwbb@big_break	\7%
	    \cwbb@noindent	\8%
	    % C/C++ tokens
	    \CwebRel		\?%
	    \CwebAddress	\AND
	    \CwebComplement	\CM
	    \CwebScope		\DC
	    \CwebEquiv		\E
	    \CwebGe		\G
	    \CwebRightShift	\GG
	    \CwebNe		\I
	    \CwebAssign		\K
	    \CwebLeftShift	\LL
	    \CwebMod		\MOD
	    \CwebNull		\NULL
	    \CwebNot		\R
	    \CwebBinOr		\OR
	    \CwebMemberRef	\PA
	    \CwebThis		\this
	    \CwebOr		\V
	    \CwebAnd		\W
	    \CwebXor		\XOR
	    \CwebLe		\Z
	    \CwebPointer	\MG
	    \CwebPointerMemberRef \MGA
	    \CwebDecr		\MM
	    \CwebIncr		\PP
	    % more tokens
	    \CwebId		\\%
	    \CwebIdLetter	\|%
	    \CwebRes		\&%
	    \CwebString		\.%		%% ( ...Emacs...
	    \CwebStringBreak	\)%
	    \CwebNumber		\T
	    \CwebCombinedOp	\MRL
	    % goes to TeX state
	    \CwebComment	\C
	    \CwebCxxComment	\SHC
	    \CwebRefName	\X
	    % CWEB tokens
	    \CwebMacrosHere	\ATH
	    \CwebDefine		\D
	    \cwbb@format	\F
	    \CwebIdCat		\J
	    \CwebVerbString	\vb
	    % cross reference tags
	    \cwbb@change_flag	\*%
	    \CwebCRAlso		\A
	    \CwebCRsAlso	\As
	    \CwebCRCite		\Q
	    \CwebCRsCite	\Qs
	    \CwebCRUse		\U
	    \CwebCRsUse		\Us
	    \CwebCREt		\ET
	    \CwebCRsEt		\ETs
	    % finish the list
	    \stop\stop
	\def\cwbb@user_bindings{%
	    \cwbb@restore_bindings
	    \let\cwbb@user_bindings\relax
	    }%
    \fi
    }
%%% \endprog


%%% \sect Since most of the \cweave{} bindings are simple and tedious
%%% coding, we'll have a look at the program layout next. Between two
%%% paragraphs there must not be any skip, the skip used in document
%%% layout is saved in |\cwbb@save@parskip|. A few other layout
%%% parameters from the document layout must be saved as well.

%%% \beginprog
\newskip\cwbb@save@parskip
\newskip\cwbb@save@rightskip
\newcount\cwbb@save@sem_sfcode
\newcount\cwbb@save@pretolerance
\newcount\cwbb@save@hyphenpenalty
\newcount\cwbb@save@exhyphenpenalty

\def\cwbb@save_doc_layout{%
    \cwbb@save@parskip\parskip
    \cwbb@save@rightskip\rightskip
    \cwbb@save@sem_sfcode\sfcode`\;
    \cwbb@save@pretolerance\pretolerance
    \cwbb@save@hyphenpenalty\hyphenpenalty
    \cwbb@save@exhyphenpenalty\exhyphenpenalty
    }
\def\cwbb@restore_doc_layout{%
    \parskip\cwbb@save@parskip
    \rightskip\cwbb@save@rightskip
    \sfcode`\;=\cwbb@save@sem_sfcode
    \pretolerance\cwbb@save@pretolerance
    \hyphenpenalty\cwbb@save@hyphenpenalty
    \exhyphenpenalty\cwbb@save@exhyphenpenalty
    }
%%% \endprog


%%% \sect When |\cwbb@program_layout| is called for the very first time
%%% we have to set up some variables. This cannot be done in advance
%%% since they depend on values which might be changed by the user in the
%%% preamble. This setup is done in |\cwbb@layout_init|.

%%% If the program layout is already active, we must not switch it on
%%% another time. This can be tested by the current binding of
%%% |\cwbb@doc_layout|.

%%% \beginprog
\def\cwbb@program_layout{%
    \ifx \cwbb@doc_layout\relax
	\cwbb@layout_init
	\cwbb@save_doc_layout
	\def\cwbb@doc_layout{%
	    \cwbb@restore_doc_layout
	    \let\cwbb@doc_layout\relax
	    }%
	% set new values
	\parskip\z@skip
	\rightskip\z@ plus 100\p@ minus 10\p@
	\sfcode`\;=3000
	\pretolerance\@M
	\hyphenpenalty 1000	% strings can be broken this way
	\exhyphenpenalty\@M
    \fi
    }
%%% \endprog


%%% \sect The unit of the basic indentation is stored in
%%% |\CwebIndentUnit|. Continuation lines are indented two units further,
%%% i.e., the initial hanging indentation is 3~units. The current hanging
%%% indentation is kept in |\cwbb@indent|.

%%% We need undiscardable items which can be used as backspaces, of one
%%% and two units, respectively. This is done best by boxes. The
%%% initialization of the boxes must be postponed until the user had the
%%% chance to change |\CwebIndentUnit|, it's done in the program state
%%% initialization.

%%% \beginprog
\newdimen\CwebIndentUnit
	\CwebIndentUnit=1em
\newdimen\cwbb@indent

\newbox\cwbb@bak	% backspace one unit
\newbox\cwbb@bakk	% backspace two units

\def\cwbb@layout_init{%
    \global\setbox\cwbb@bak \hbox to -1\CwebIndentUnit{}%
    \global\setbox\cwbb@bakk \hbox to -2\CwebIndentUnit{}%
    \global\let\cwbb@layout_init\relax
    }
%%% \endprog


%%% \sect Now we can formulate how to start typesetting program pieces: We
%%% switch to program state, set the initial indentation, and add the
%%% basic indentation of the first code line. Then we might add a marker
%%% to this program part; for hierarchic structure we want to typeset the
%%% chunk number in the margin of the chunk's program part. That's not
%%% really possible, 'though: definitions and format requests are also
%%% introduced with |\B|, so the chunk number will happen to be at the
%%% start of the definition part if one exists.

%%% If the code line is the very first text to be typeset in this chunk,
%%% we don't add the basic indentation---we're already indented from the
%%% run-in chunk start marker. But we must not start our chunk with
%%% |\noindent|, the marker would be typeset in the left margin then. (I
%%% consider this as a \LaTeX{} error.) So we just emulate an empty
%%% documentation part, the marker is now set correctly. |\B| can test for
%%% this case: |@noskipsec| is still true then.

%%% \beginprog
\def\B{%
    \cwbb@program
    \cwbb@indent 3\CwebIndentUnit  \hangindent\cwbb@indent
    \ifvmode
	\if@noskipsec
	    \indent		% add an empty documentation part
	\else
	    \noindent \kern\CwebIndentUnit
	\fi
    \fi
    \CwebNumberProgramPart
    }
%%% \endprog


%%% \sect The chunk number shall be added at the first program part. We
%%% have a hook in the chunk start where we can set up
%%% |\CwebNumberProgramPart|. It will redefine itself at first call, to
%%% disable further output on more |\B| occurences in a chunk.

%%% The chunk number should always be in the left margin. If it's in the
%%% right margin, there would be too much horizontal space between the
%%% code start and the actual number. (The first code line is typically
%%% not very long.) \LaTeX{} does not allow to select the margin for
%%% marginpars, therefore we have to typeset the marginal note `by hand'.
%%% As we know that we're in horizontal mode when we're called, |\vadjust|
%%% is used to create the note. The standard \LaTeX{} dimen register for
%%% the distance between marginal notes and text matter is used, of
%%% course.

%%% Note that marginal notes may overwrite the chunk number. As it's very
%%% unlikely that somebody issues marginal notes in program mode (that's
%%% quite difficult and involves ingenious usage of the implementation of
%%% this macros), I won't care for that case.

%%% \begin{fixme}
%%%   The baseline of the marginal note shall be aligned to the baseline
%%%   of the preceding line. That does not happen by default. Currently, I
%%%   assume that the preceding line has a depth of appropriately the
%%%   depth of a~`g'. Experiments showed that this gives good results for
%%%   most \cweb{} documents. But nevertheless one should determine the
%%%   exact depth of the preceding line, to know how much |box0| should be
%%%   raised. If anybody knows how this can be done, please tell me. (I
%%%   know about the route over the output routine, implementing an own
%%%   marginal note category, but this seems too much effort.)
%%% \end{fixme}

%%% For flat structure, we don't add any program part marker and we
%%% don't need to initialize it.

%%% \beginprog
\ifcase \cweb@structure
    %% hierarchic
    \def\cwbb@marginal_chunkno{%
	\vadjust{%
	    \setbox\z@ \hbox to \z@{%
		\hss \vphantom{g}%
		\small \thechunk
		\hskip \marginparsep
		}%
	    \vbox to \z@{%
		\hsize\z@
		\vss
		\noindent \raise \dp\z@ \box\z@
		}%
	    }%
	}
    \def\cwbb@init_print_chunk{%
       \def\CwebNumberProgramPart{%
	   \cwbb@marginal_chunkno
	   \let\CwebNumberProgramPart\relax
	   }%
       }
  \or
    %% flat
    \let\cwbb@init_print_chunk\relax
    \let\CwebNumberProgramPart\relax
\fi
%%% \endprog


%%% \sect If a statement is finished, a new paragraph with the basic
%%% indentation has to be started.

%%% An optional statement break is implemented by a low penalty which
%%% will be selected if the line has to be broken. We assume that the
%%% hanging indentation of the new line is already set correctly, and
%%% have to backup two units to get the basic indentation. It might be
%%% that the line break is not chosen by \TeX{}, we compensate the
%%% backspace for this case. This compensation is discarded at the start
%%% of a new line.

%%% \beginprog
\def\cwbb@break{%	% forced break, between statements
    \ifmmode\else
	\endgraf	% in LaTeX it isn't sure what \par is really...
	\noindent
	\hangindent\cwbb@indent  \kern\cwbb@indent
	\copy\cwbb@bakk	% go back to basic indentation
	\ignorespaces
    \fi
    }
\def\cwbb@opt_break{%	% optional break between statements
    \hfil \penalty\m@ne
    \hfilneg  \kern .5em  \kern 2\CwebIndentUnit   % discarded on line break
    \copy\cwbb@bakk
    \ignorespaces
    }
\def\cwbb@big_break{%	% forced break and a little extra space
    \Y
    \cwbb@break
    }
\def\cwbb@expr_break#1{%	% break with penalty #1 * 10
    \hfil \penalty#10
    \hfilneg		% discarded on line break
    }
%%% \endprog


%%% \sect When we increment the indentation, we must not forget to set
%%% the hanging indentation immediately since |\cwbb@opt_break| relies on
%%% the new value. If a continuation line is needed it will also be
%%% indented one unit more, which is ok since it should be distinguishable
%%% from the next line.
%%% %
%%% \begin{fixme}
%%%   Shouldn't the hanging indentation be set anew at the end of
%%%   |\cwbb@decr_indent| as well?
%%% \end{fixme}

%%% \beginprog
\def\cwbb@incr_indent{%
    \global\advance\cwbb@indent\CwebIndentUnit
    \hangindent\cwbb@indent
    }
\def\cwbb@decr_indent{\global\advance\cwbb@indent -\CwebIndentUnit}
\def\cwbb@backup{\copy\cwbb@bak}
\def\cwbb@noindent{%				% no indentation
    \hskip -\cwbb@indent \hskip 2\CwebIndentUnit
    }
%%% \endprog




%%% \chap Program (C or \C++) tokens.

%%% Since the user might want to change the (somewhat unusual) way some
%%% of the operators are typeset, we supply names for them.

%%% \beginprog
\let\CwebAnd=\land		% logical and, &&
\let\CwebEquiv=\equiv		% equiv sign, for ==
\let\CwebGe=\ge			% greater or equal
\let\CwebLe=\le			% less or equal
\let\CwebNe=\ne			% unequal, !=
\let\CwebNull=\Lambda		% NULL pointer
\let\CwebNot=\lnot		% logical not, !
\let\CwebOr=\lor		% logical or, ||
\let\CwebXor=\oplus		% bitwise exclusive or, ^
%%% \endprog


%%% \sect Some symbols have to be shifted around, to save computation time
%%% we put them in boxes.

%%% \begin{fixme}
%%%   It's not sure if the following symbols still look good in other font
%%%   families than Computer Modern. The movements might need to get
%%%   adapted when more experience is available. They were chosen from
%%%   Plain \cweb{}, where these were explicit movements measured in
%%%   points. I changed the dimension to quads, to achieve at least the
%%%   support for different sizes in the Computer Modern family.
%%% \end{fixme}

%%% \TeX{}nical note: The pointer symbol must not be declared as a math
%%% symbol (e.g., with |\DeclareMathSymbol|). Then \TeX{} would insert an
%%% italic correction behind it, according to rule~17 from appendix~G.
%%% (Please send email if you know of an easy way to prevent that
%%% addition.) So I use the symbol as an accent, but then I have to supply
%%% a nucleus for this math list. That nucleus also determines the total
%%% width of this construct. As the vector accent has a width of
%%% approximately 5\,pt in |cmmi10|, I'm going to use 0.5\,em. (That
%%% heuristic is as good as any other I can think of.)

%%% \beginprog
%% pointer to struct component (`->'), use symbol of \vec accent
\newbox\cwbb@pointer
\setbox\cwbb@pointer=\hbox{%
	    \kern -.2em
	    \lower .3em \hbox{$\vec{\kern .5em}$}%
	    \kern .1em
	    }

\newbox\cwbb@decr		% decrement, --
\setbox\cwbb@decr=\hbox{%
	    \kern .05em
	    \raise .1em \hbox{$\scriptstyle {-}\kern -.1em{-}$}%
	    \kern .05em
	    }
\newbox\cwbb@incr		% increment, ++
\setbox\cwbb@incr=\hbox{%
	    \kern .05em
	    \raise .1em \hbox{$\scriptstyle {+}\kern -.1em{+}$}%
	    \kern .05em
	    }
%%% \endprog


%%% \sect Although the following symbols are typeset like an ``ordinary
%%% \C++ programmer'' would expect them, we provide own module names
%%% nevertheless. They can now be changed as well, i.e., orthogonality is
%%% enhanced. (And we can use |\cwbb@rebind| for assigning them to their
%%% names while we switch to program state\,\dots).

%%% Note: The comment at |\CwebMod| says ``modulo/remainder'', as this is
%%% not defined by the C standard, i.e., implementation-dependent.

%%% \beginprog
\mathchardef\CwebAddress="2026	% `&', as binary op
\let\CwebAssign==		% assignment
\let\CwebBinOr=\mid		% binary or
\def\CwebComplement{{\sim}}	% `~', as ordinary symbol
\def\CwebDecr{\copy\cwbb@decr}	% decrement
\def\CwebIncr{\copy\cwbb@incr}	% increment
\let\CwebLeftShift=\ll		% left shift, <<
\def\CwebMod{\mathbin{\hbox{\footnotesize\rm\%}}}	% modulo/remainder, %
\def\CwebMemberRef{\mathbin{.*}}	% ptr to member (on object)
\def\CwebPointer{\copy\cwbb@pointer}
\def\CwebPointerMemberRef{\mathbin{\CwebPointer*}} % ptr to member (on ptr)
\def\CwebRel{\mathrel?}		% relation operator
\let\CwebRightShift=\gg		% right shift, >>
\def\CwebScope{\kern.1em{::}\kern.1em}	% scope resolution

\def\CwebThis{\CwebRes{this}}	% reserved identifier `this'
%%% \endprog


%%% \sect The special identifier |\TeX| remains. In math mode it shall be
%%% the identifier, in other modes the usual logo. We use the definition
%%% supplied by DEK (or \textsc{Silvio Levy}?)\ from Plain \cweb{}.

%%% \beginprog
\def\TeX{%
    {%
	\ifmmode\it\fi
	\mbox{T\kern-.1667em\lower.424ex\hbox{E}\hskip-.125emX}%
    }}
%%% \endprog


%%% \sect Some tokens don't have constant names, the name is supplied as
%%% the argument.

%%% Identifiers are typeset in italics, reserved words and type names in
%%% boldface, and strings in typewriter. For an underscore in the bold
%%% words we use a line that's 50\,\% thicker as the usual line thickness.
%%% -- But what is the usual line thickness. As Knuth uses 0.4\,pt for a
%%% 10\,pt Computer Modern font, we'll use 0.06\,em for a CM font in an
%%% arbitrary size. If that line thickness is still OK for other fonts is
%%% doubtful, as it should fit to the stem width. (CM is a rather light
%%% font, after all.) But since there is no font dimension that gives a
%%% hint to an appropriate line thickness, it's our best try.

%%% \beginprog
\def\CwebCombinedOp#1{\mathrel{\let\K==#1}}  % e.g., += operator
\def\CwebId#1{\hbox{\it#1\/\kern.05em}} % identifier, more than one char
\def\CwebIdLetter#1{\hbox{$#1$}}	% identifier, one letter
\def\CwebRes#1{%			% reserverd words and type names
    \hbox{\bf
	\def\_{\kern.04em\vbox{\hrule width.3em height.06em}\kern.08em}%
	#1\/\kern .05em
	}%
    }
%%% \endprog


%%% \sect In Plain \cweb{} strings are typeset in the font |cmtex10|,
%%% which is a typewriter font with extended ASCII characters. This font
%%% does not have the usual accents and can therefore not used for
%%% typesetting national characters. (Very often they are input in some
%%% 8-bit encoding, the respective character code is made active and is
%%% substituted by a cseq which expands to the correct glyph. Of course,
%%% there are better ways, but that's the reality we have to cope with.)
%%% Instead of this special font we use the standard typewriter font.
%%% This has the further advantage that |cmtex10| may not be introduced
%%% to NFSS\@.

%%% \beginprog
\def\CwebString{%
    \hbox \bgroup
        \tt
	\cwbb@string_setup
	\cwbb@string
    }
\def\cwbb@string#1{%
	#1\kern .05em
	\egroup
    }
%%% \endprog


%%% \sect Within strings certain cseqs have a special meaning; this is
%%% introduced by |\cwbb@string_setup|. The cseqs within strings are
%%% mostly special \TeX{} characters that are to be typeset by cseqs with
%%% the same name.

%%% But a problem remains: active characters. They will still expand, and
%%% will mess up the string visualization. This is a particular problem
%%% with |german.sty| where the double quote is active. (Of course, that's
%%% a problem -- I'm German. :--) When one typesets strings, a double
%%% quote shall be typeset as double quotes, and must not expand to the
%%% special meaning. Remember that double quotes happen to appear quite
%%% often in strings. As the string argument is already tokenized, one has
%%% to redefine the binding of $(\textit{\textup{symbol}}\ \textbf{.}\
%%% |#\"|)$.

%%% \begin{fixme}
%%%   Perhaps one might use a loop over all characters except the braces,
%%%   and set their catcode to other. That would give us enough security.

%%%   For now, we handle only the double quote.
%%% \end{fixme}

%%% Every once in a while we have discretionary breaks in a string,
%%% denoted by |\CwebStringBreak|. This break is shown with the C
%%% convention of escaped newlines.

%%% \beginprog
\def\cwbb@string_setup{%
    \chardef\ =`\  %		% <-- two spaces !
    \chardef\&=`\&
    \chardef\\=`\\
    \chardef\^=`\^
    \chardef\_=`\_
    \chardef\{=`\{
    \chardef\}=`\}
    \chardef\~=`\~
    \catcode`\"=\CatOther
    }

\def\CwebStringBreak{\discretionary{\hbox{\tt\char`\\}}{}{}}
%%% \endprog


%%% \sect Numbers are typeset in different ways. We use the definition of
%%% Plain \cweb{}.
%%% %
%%% \begin{fixme}
%%%   Should add a specification of the possible input and an explanation
%%%   of the macros below. In particular, that the closing brace after
%%%   |\aftergroup| is used much later is probably not grokked by most
%%%   \TeX{} programmers.
%%% \end{fixme}

%%% \beginprog
\def\cwbb@oct{\hbox{$^\circ$\kern-.1em\it\aftergroup\?\aftergroup}}
\def\cwbb@hex{\hbox{$^{\scriptscriptstyle\#}$\tt\aftergroup}}

\def\CwebNumber#1{%		% octal, hex, or decimal constant
    \hbox{%
	$%
	    \def\?{\kern.2em}%
	    \def\$##1{\egroup\sb{\,\rm##1}\bgroup}% suffix to constant
	    \def\_{\cdot 10^{\aftergroup}}% power of ten (via dirty trick)
	    \let\~\cwbb@oct \let\^\cwbb@hex
	    {#1}%
	$}%
    }
%%% \endprog


%%% \sect Comments are typeset in \TeX{} state. We add a hook, the user
%%% shall be able to change the layout (e.g., he might want another font).
%%% %
%%% \begin{fixme}
%%%   Currently \C++ comments are typeset like C comments. This is
%%%   horrible in usual circumstances, i.e., when complete blocks of text
%%%   are prefixed with |//|. We should simply catenate all these text and
%%%   typeset it as one paragraph, each line prefixed by |//|. But then we
%%%   have to implement an |\everyline| first, and since that's not so
%%%   easy we postpone it\,\dots
%%% \end{fixme}
%%% %
%%% In front of a C comment there is an optional stmt break, with 2\,quad
%%% in front if the line is not broken and 1.5\,quad if the line is broken.

%%% \beginprog
\def\CwebComment#1{%
    \5%				% 0.5em will be discarded on line break
    \hskip 1.5em
    $/\ast\,$%
    {\cwbb@tex
	\CwebCommentHook
	#1%
    }%
    $\,\ast/$%
    }
\let\CwebCommentHook\relax
\let\CwebCxxComment\CwebComment
%%% \endprog





%%% \chap \cweb{} tokens.

%%% We distinguish three categories of \cweb{} tokens: (1) Those which
%%% output constant text, (2) those which have attributes to be displayed
%%% in a special way, and (3) those which start a new structure element,
%%% namely |@d| and~|@f|. Let's consider them in this order.


%%% \sect \cweb{} tokens which expand in a constant string are the
%%% identifier catenation operator~(`|@&|') and the macro placement
%%% directive~(`|@h|').

%%% \beginprog
\def\CwebIdCat{\CwebString{@\&}}
\def\CwebMacrosHere{%
    \begingroup
	\def\CwebRefNumber##1{}%
	\CwebRefName :Preprocessor definitions\X
    \endgroup
    }
%%% \endprog


%%% \sect Verbatim program strings, i.e., strings passed verbatim by
%%% \ctangle{}~(`|@=|') are typeset like normal strings, but within a box.
%%% We use 2\,pt as the separating distance, this is set locally.

%%% \beginprog
\def\CwebVerbString#1{{\fboxsep\tw@\p@ \fbox{\CwebString{#1}}}}
%%% \endprog


%%% \sect The refinement names are typeset in angles, this has a long
%%% tradition. We burry the typesetting of the chunk number (which is
%%% the first parameter) in a macro call; the user may change this to
%%% achieve special effects.
%%% %
%%% \begin{fixme}
%%%   One could think about handling special values of the chunk numbers
%%%   differently. E.g., an empty argument is not typeset at all, a~0
%%%   triggers a marginal note about a missing definition, etc.
%%% \end{fixme}

%%% A refinement name may be typeset both in math and horizontal mode.
%%% The name itself is typeset in horizontal mode, of course; for the
%%% angles we need math mode. Therefore we assert at the start of the
%%% macro that we're not in math mode any more. At the end we switch back
%%% to math mode if we've started in it. This conditional switch from and
%%% back to math mode is done by |\cwbb@toggle_text|.

%%% \TeX{}nical note: If we're in math mode |\cwbb@toggle_text| must be
%%% defined globally, as it will turn off math mode and the definition
%%% would be un-made then. The second invocation of |\cwbb@toggle_text|
%%% would be undefined then.

%%% A refinement may also consist of the file name the expansion of this
%%% refinement shall be written to.
%%% %
%%% \label{sec:refname-dot}
%%% %
%%% This file name is tagged with |\.|, it shall be typeset like a string.
%%% But the user shall be able to use the dot accent in the refinement
%%% name as well. We check if the text consists solely of the tag and its
%%% argument; in this case we substitute |\.| with |\CwebString|.
%%% Otherwise we leave it as it is.  Then a refinement name may not
%%% consist of a single dot-accented expression---well, that's highly
%%% unlikely. (Nevertheless I'll document it in the user's manual\,\dots)

%%% \begin{fixme}
%%%   One should be able to use |\EnsureMath| here. But the refinement
%%%   name must not be set in math mode and must not be set in an hbox;
%%%   the latter would prevent line breaks.
%%% \end{fixme}

%%% \beginprog
\def\CwebRefName#1:#2\X{%
    \ifmmode  \gdef\cwbb@toggle_text{\null$\null}%
    \else  \let\cwbb@toggle_text\relax
    \fi
    \cwbb@toggle_text
    $\langle\,${\cwbb@tex \cwbb@check_dot{#2}\CwebRefNumber{#1}}$\,\rangle$%
    \cwbb@toggle_text
    }
\def\CwebRefNumber#1{%
    {\reset@font \footnotesize \kern .5em#1}%
    }
%%% \endprog


%%% \sect The next macro, |\cwbb@check_dot| is a hassle to implement. We
%%% must check if it's argument consists solely of a `|\.|'~tag, together
%%% with the argument to this tag. Then the `|\.|'-argument shall be
%%% typeset like a string, |\.| itself must be ignored.

%%% The check for |\.| must not evaluate the parameter. Especially |\PB|
%%% must not be evaluated within, it would lead to havoc with the
%%% redefinition of all bindings, etc. (At least that's the empirical
%%% result---it happened; although I don't know where the problems are.
%%% Anyhow, it wasn't in the specs of the rebinding process.) All other
%%% cseqs which shouldn't be used in moving arguments may cause problems
%%% here, too. Note that |\protect| is no solution here, the user often is
%%% not aware of the presence of \cweave{}-generated tags. As an example,
%%% consider the refinement \verb"@< local variables of |foo| @>|" which
%%% will lead to a call with the argument `|local variables of \PB{\\{foo}}|'.
%%% |\\| must not be evaluated here.

%%% I use an approach where I hope that it will work---but to be honest,
%%% I'm not sure. I want to define a macro |\next| which shall have an
%%% empty expansion iff |#1| was a `|\.|'~tag with its argument. Empty
%%% refinement names are not allowed so we can assume that |#1| consists
%%% of at least one token. I evaluate the first token of |#1| before the
%%% |\def| is done. With an appropriate binding a |\.| will disappear,
%%% perhaps leaving an empty token list. I also take care that |\PB| is
%%% not evaluated then, by temporary rebinding it to |\relax|. Thereby I
%%% have taken care for all cseqs \cweave{} might have introduced at the
%%% very front of |#1|. Other cseqs are user-tags. If they cause
%%% problems, the user might circumstance them by adding |\protect|.

%%% \beginprog
\def\cwbb@check_dot#1{%
    \begingroup
	\let\.\@gobble
	\let\PB\relax
	\expandafter\def \expandafter\next \expandafter{#1}%
	\ifx \next\empty
	    \gdef\next##1{\CwebString}%
	\else
	    \global\let\next\relax
	\fi
    \endgroup
    \next #1%
    }
%%% \endprog


%%% \sect \cweb{} macro definitions (done by~|@d|) and format directives are
%%% basically program text. When they are started we are already in
%%% program state, i.e., |\B| has appeared in front of it. But since we have
%%% an introductionary identifier (either |#define| or |format|) to set at
%%% the very front we increase the indentation by three units.

%%% The output of format directives may be suppressed.  Then we have to
%%% gather both identifiers that will appear after |\F|.  Both of them
%%% consist of one group-token (a cseq and the identifier name), after
%%% each identifier one other token separates them. I.e., we have to
%%% gobble six group-tokens. As the very last token is |\par| if a format
%%% directive follows directly on another one, the definition must be
%%% long.

%%% \TeX{}nical note: |\cwbb@format| must expand to |\CwebFormat|, not to
%%% its expansion. Otherwise a redefinition of |\CwebFormat| in a package
%%% would not be used.

%%% \beginprog
\def\CwebDefine{\cwbb@macro{\#define}}
\def\CwebFormat{\cwbb@macro{format}}
\if@cweb@suppress@format@
    \long\def\cwbb@format#1#2#3#4#5#6{}
\else
    \def\cwbb@format{\CwebFormat}
\fi

\def\cwbb@macro#1{%
    \global\advance\cwbb@indent \tw@\CwebIndentUnit
    \cwbb@incr_indent
    \CwebRes{#1 }%		% <-- blank!
    }
%%% \endprog




%%% \chap Refinement cross references and changes.

%%% At the end of chunks with refinements, \cweave{} may output cross
%%% reference information: In which chunks additional definitions for
%%% this refinement are found, where this refinement is used, and where it
%%% is cited. This cross reference information is always introduced by a
%%% tag followed by a non-empty list of chunk numbers. Different tags
%%% are used for lists with only one element and for lists with more than
%%% one element, this way the introductionary text may be adapted.

%%% Within a list of $n$~chunk numbers the first $n-1$~numbers are
%%% separated by commas with one following blank. The last two numbers
%%% are separated by |\ET| if~$n=2$, and by |\ETs| if~$n>2$. If a chunk
%%% is changed by the changefile, a changeflag~(`|\*|') is appended to
%%% its number. The list is eventually terminated by a full stop.


%%% \sect We separate the cross reference information by a smallskip from
%%% the refinement or from a previous cross reference information. The
%%% information itself is typeset in a smaller font, as it is auxilliary,
%%% inserted stuff. The number list has a hanging indentation of
%%% |\CwebNumberListHangindent|. But beware: This isn't a dimension
%%% register, it's a macro. This way one can use ems as dimensions.

%%% The actual size change is represented by a macro. We will redefine it
%%% when we need an other size, e.g., in the lists at the end of the
%%% document.

%%% We must assert that we're in CR state while we set a cross reference.

%%% The first parameter of a cross reference information unit is the
%%% introductionary text, the second is the number list. The parameters
%%% must be evaluated in a group---local parameter changes therein must
%%% not influence the environment.

%%% \beginprog
\def\CwebNumberListHangindent{2em}
\let\CwebCRSize\footnotesize
\def\CwebCrossRef#1#2.{%
    \par\smallskip
    \cwbb@CR
    \begingroup
	\reset@font \CwebCRSize
	\noindent \hangindent\CwebNumberListHangindent\relax
	#1~#2.\par
    \endgroup
    }
%%% \endprog


%%% \sect Well, let's define all introducing tags, which in fact start the
%%% cross reference information unit. The number list is gathered by
%%% |\CwebCrossRef|.

%%% \beginprog
\def\CwebCRAlso{\CwebCrossRef{See also chunk}}
\def\CwebCRsAlso{\CwebCrossRef{See also chunks}}

\def\CwebCRCite{\CwebCrossRef{This code is cited in chunk}}
\def\CwebCRsCite{\CwebCrossRef{This code is cited in chunks}}

\def\CwebCRUse{\CwebCrossRef{This code is used in chunk}}
\def\CwebCRsUse{\CwebCrossRef{This code is used in chunks}}

\def\CwebCREt{ and~}
\def\CwebCRsEt{, and~}
%%% \endprog




%%% \chap Bells and whistles.

%%% \begin{fixme}
%%%   Just copied from Plain \cweb{}. It doesn't work
%%%   anyhow---|\char"|\<xy> usually doesn't typeset the correct
%%%   character. That's because this character is most probably an active
%%%   character; at least that's typical for the way \TeX{} systems are
%%%   used in Europe. I have to think about the implementation.
%%% \end{fixme}

%%% \beginprog
\def\ATL{\par\noindent\bgroup\catcode`\_=12 \postATL} % print @l in limbo
\def\postATL#1 #2 {\bf letter \\{\uppercase{\char"#1}}
   tangles as \tentex "#2"\egroup\par}
\def\noATL#1 #2 {}
\def\noatl{\let\ATL=\noATL} % suppress output from @l
%%% \endprog



%%% \chap The end.

%%% Well, after all we're finished with this module

%%% \beginprog
\endinput
%%% \endprog


%%% 
%%% %%%%%%%%%%%%%%%%%%%%%%%%%%%%%%%%%%%%%%%%%%%%%%%%%%%%%%%%%%%%%%%%%%%%%%

%%% \vskip \PltxPreSectSkip

%%% \rcsLogIntro{After the actual revision log of this module, we give an
%%%     extract of log messages of the \cls{cweb} class. This module was
%%%     extracted from revision~2.8 of that class, we present to you the
%%%     log entries that concern the code of the new module.}

%%% \begin{rcslog}
%%% $StyleLog: cwebbase.doc,v $
%%% \Revision 1.4 1995/11/07 17:55:19 schrod
%%%     \LaTeX{} \cweb{} should work at least with the last two \LaTeX{}
%%% versions; make it work with the previous-to-last one, version
%%% \mbox{$\langle$1994/12/01$\rangle$}. For that, one has to install the
%%% \mbox{$\langle$1995/06/01$\rangle$} (non-outer) definition of
%%% |\newif|, and |\hb@xt@| must not be used.\\
%%% Problem reported by Laurent Desnogues
%%% \path|<laurent.desnogues@aiguemarine.unice.fr>| and somebody else (XXX
%%% -- add name).

%%% \Revision 1.3 1995/11/07 14:13:12 schrod
%%% Reset section numbers when higher-level section begins.\\
%%% Error reported by Bronne Louis \path|<bronne@montefiore.ulg.ac.be>|.

%%% \Revision 1.2 1995/11/06 10:58:47 schrod
%%%     Support active double quotes in strings. That happens when
%%% |german.sty| is used. Problems with other active characters may still
%%% appear.

%%% \Revision 1.1 1995/09/12 23:02:22 schrod
%%% Moved all code that does the actual typesetting of \cweave{} tags and
%%% is therefore also needed for a |cweb| environment. It's now an own
%%% module named \pkg{cwebbase}.

%%% Chunk markers and chunk marginal numbers don't hang below the
%%% baseline any more.


%%% \item[] \noindent\hrulefill\break
%%%   \textbf{Extract of the revision log for |cweb.doc|:}

%%%   \vskip 2ex

%%% \Revision 2.5 1995/08/29 02:07:26 schrod
%%% Discard dependencies on 10\,pt fonts.

%%% Support suppression of format directives.

%%% \Revision 2.4 1995/08/27 19:31:43 schrod
%%% Make configuration of change flag easier.

%%% Put section title of changed chunks list in marks, and tell the user
%%% that it got typeset.

%%% \Revision 2.3 1995/08/27 17:24:46 schrod
%%% Make usage of baseclass with chapters work.

%%% \Revision 2.2 1995/08/27 13:26:22 schrod
%%% Add possibility to suppress change hints.

%%% \Revision 2.1 1995/08/25 19:11:18 schrod
%%% Hierarchic structures are supported now, in addition to the flat
%%% structure of the beta-test version. One can choose with an option. For
%%% that step, the terminology was cleaned up, too: Chunks are not named
%%% sections any more. (That change involved reimplementation of almost
%%% all the structure and toc stuff.)

%%% The chunk number supplied by \cweave{} is used now, not some computed
%%% number. Change flags are printed, too.

%%% One can suppress output of unchanged sections.

%%% \Revision 1.12  1993/08/10  14:15:43  schrod
%%% New page on main section only if group level $<$ |\cwebSecNoEject|.
%%% Default for the latter is 3.

%%% \Revision 1.11  1993/08/10  11:21:07  schrod
%%% Reference to section number does not render a period after the
%%% number any more.

%%% \Revision 1.10  1993/08/09  20:08:20  schrod
%%% |\cweb@cweave_bindings| is now a no-op if \cweave{} bindings are in
%%% effect already.\\
%%% (Problem reported by Michael M\"uller \path|<mimu@mpi-sb.mpg.de>|.)

%%% \Revision 1.9  1993/08/09  18:05:28  schrod
%%% Left shift operator wasn't defined correctly.\\
%%% (Problem reported by Michael M\"uller \path|<mimu@mpi-sb.mpg.de>|.)

%%% \Revision 1.6  1993/06/15  08:49:23  schrod
%%% |\cweb@check_dot| must not evaluate its argument in an |\edef|, this
%%% causes problems if a |\PB| is within. Now I try hard not to evaluate any
%%% tokens outside of my control.

%%% Can use |\@defpar| for an empty line in |\cweb@has_entries|, don't
%%% need an own macro.

%%% \Revision 1.4  1993/06/14  15:54:18  schrod
%%% Add FSA diagram about processing states. CR state is also switched
%%% to from \TeX{} state (that happens with |\ch| at the document end).

%%% \Revision 1.3  1993/05/13  17:51:21  schrod
%%% Refinements may also be filenames (`|@(|'). Then the complete name
%%% consists of a |\.| macro call, which is handled now.\\
%%% (Problem reported by Michael M\"uller \path|<mimu@mpi-sb.mpg.de>|.)

%%% Made the detection of `|@.|' index entries more robust.

%%% \Revision 1.2  1993/05/12  18:28:59  schrod
%%% Adapted to recent changes of \cweave{} (of April 93):

%%% Main sections have a group level, represented in the table of
%%% contents. This changed the complete implementation of section tags.

%%% New C token cseqs: |\Z| and |\MRL|, implemented as |\CwebLe| and
%%% |\CwebCombinedOp|.

%%% \Revision 1.1  1993/04/09  15:00:37  schrod
%%% Initial revision

%%% \end{rcslog}



%%% \end{document}


%%% 
%%% %%%%%%%%%%%%%%%%%%%%%%%%%%%%%%%%%%%%%%%%%%%%%%%%%%%%%%%%%%%%%%%%%%%%%%
%%% Local Variables:
%%% mode: LaTeX
%%% TeX-brace-indent-level: 4
%%% indent-tabs-mode: t
%%% TeX-auto-untabify: nil
%%% TeX-auto-regexp-list: LaTeX-auto-regexp-list
%%% compile-command: "make cwebbase.tex"
%%% End:
