% Copyright 2006 by Till Tantau
%
% This file may be distributed and/or modified
%
% 1. under the LaTeX Project Public License and/or
% 2. under the GNU Public License.
%
% See the file doc/generic/pgf/licenses/LICENSE for more details.

\ProvidesFileRCS $Header: /cvsroot/pgf/pgf/generic/pgf/basiclayer/pgfcorepathusage.code.tex,v 1.24 2014/11/02 21:38:01 cfeuersaenger Exp $


% Stroke/fill/clip/etc. the current path. Depending on the options,
% the current path will be stroked/filled/clipped/etc. If no options
% are given, the path is stroked. If multiple options are given, all
% of them are performed (in a sensible order).
%
% #1 = action(s) to be applied to the current path. Valid actions are: 
%      stroke -  strokes the path.
%      draw -    strokes the path and adds arrow tips.
%      tips -    adds arrow tips.
%      fill -    fills the path.
%      clip -    clip the path.
%      discard - Discards the path. Same effect as having an empty
%                options list.
%
% Example:
%
% % Draws an edge.
% \pgfpathmoveto{\pgfxy(0,0)}
% \pgfpathlineto{\pgfxy(0,1)}
% \pgfpathlineto{\pgfxy(1,0)}
% \pgfusepath{stroke}

\pgfkeys{
  /pgf/stroke/.code=\let\pgf@up@stroke\pgf@up@stroke@text\pgf@up@path@neededtrue,
  /pgf/draw/.code=\let\pgf@up@stroke\pgf@up@stroke@text\pgf@up@path@neededtrue,
  /pgf/fill/.code=\let\pgf@up@fill\pgf@up@fill@text\pgf@up@path@neededtrue,
  /pgf/clip/.code=\let\pgf@up@clip\pgf@up@clip@text\pgf@up@path@neededtrue,
  /pgf/discard/.code=,
  /pgf/use as bounding box/.code=\let\pgf@up@bb\pgf@do@up@bb,
  /pgf/arrow keys/flex/.code=\pgferror{You need to say \string\usetikzlibrary{bending} for flexing and bending arrows},
  /pgf/arrow keys/flex'/.code=\pgferror{You need to say \string\usetikzlibrary{bending} for flexing and bending arrows},
  /pgf/arrow keys/bend/.code=\pgferror{You need to say \string\usetikzlibrary{bending} for flexing and bending arrows},
  /pgf/arrow keys/quick/.code=\pgfarrowsaddtooptions{\let\pgf@arrow@flex@mode\pgf@arrow@mode@is@quick},
}

\pgfkeys{
  /pgf/.cd,
  tips/.is choice,
  tips/.default=true,
  tips/proper/.code=\let\pgf@tips@mode\pgf@tips@mode@onproper,
  tips/on proper draw/.code=\let\pgf@tips@mode\pgf@tips@mode@onproperdraw,
  tips/true/.code=\let\pgf@tips@mode\pgf@tips@mode@true,
  tips/on draw/.code=\let\pgf@tips@mode\pgf@tips@mode@ondraw,
  tips/never/.code=\let\pgf@tips@mode\pgf@tips@mode@false,
  tips/false/.code=\let\pgf@tips@mode\pgf@tips@mode@false,
}
\def\pgf@tips@mode@onproper{4}
\def\pgf@tips@mode@onproperdraw{3}
\def\pgf@tips@mode@true{2}
\def\pgf@tips@mode@ondraw{1}
\def\pgf@tips@mode@false{0}
\let\pgf@tips@mode\pgf@tips@mode@ondraw


% The following can only be set to "true" using the options in the
% module "bending"
\newif\ifpgf@precise@shortening
\def\pgf@arrow@mode@is@quick{0}
\let\pgf@arrow@flex@mode\pgf@arrow@mode@is@quick

\newif\ifpgf@up@path@needed
\newif\ifpgf@up@draw@arrows
\def\pgf@up@stroke@text{stroke}
\def\pgf@up@fill@text{fill}
\def\pgf@up@clip@text{clip}
\def\pgf@do@up@bb{\pgf@relevantforpicturesizefalse}

\def\pgfusepath#1{%
  \pgf@up@path@neededfalse%
  \pgf@up@draw@arrowsfalse%
  \let\pgf@up@stroke\pgfutil@empty%
  \let\pgf@up@fill\pgfutil@empty%
  \let\pgf@up@clip\pgfutil@empty%
  \let\pgf@up@bb\pgfutil@empty%
  \pgfset{#1}%
  \expandafter\def\expandafter\pgf@up@action\expandafter{\csname pgfsys@\pgf@up@fill\pgf@up@stroke\endcsname}%
  \ifnum\pgf@tips@mode=2\relax%
    \pgf@up@path@neededtrue%
  \fi%
  \ifnum\pgf@tips@mode=4\relax%
    \pgf@up@path@neededtrue%
  \fi%
  \ifpgf@up@path@needed%
  \else%
    \pgfsyssoftpath@setcurrentpath\pgfutil@empty%
  \fi%
  \ifx\pgf@up@stroke\pgfutil@empty%
    \ifx\pgf@up@fill\pgfutil@empty%
      \let\pgf@up@action=\pgfutil@empty%
      \ifx\pgf@up@clip\pgfutil@empty%
      \else%
        % only clipping  
        \let\pgf@up@action=\pgfsys@discardpath%
      \fi%
    \fi%
  \fi%  
  \pgfsyssoftpath@getcurrentpath\pgf@last@processed@path
  \pgfprocessround{\pgf@last@processed@path}{\pgf@last@processed@path}% change the path
  \pgfsyssoftpath@setcurrentpath\pgf@last@processed@path%
  %
  % Check whether the path is stroked. If so, add half the line width
  % to the bounding box.
  %
  \ifpgf@relevantforpicturesize%
    \ifx\pgf@up@stroke\pgfutil@empty%
    \else%
      \ifdim\pgf@picmaxx=-16000pt\relax%
      \else%
        \pgf@x=\pgf@pathminx\advance\pgf@x by-.5\pgflinewidth%
        \ifdim\pgf@x<\pgf@picminx\global\pgf@picminx\pgf@x\fi%
        \pgf@y=\pgf@pathminy\advance\pgf@y by-.5\pgflinewidth%
        \ifdim\pgf@y<\pgf@picminy\global\pgf@picminy\pgf@y\fi%
        \pgf@x=\pgf@pathmaxx\advance\pgf@x by.5\pgflinewidth%
        \ifdim\pgf@x>\pgf@picmaxx\global\pgf@picmaxx\pgf@x\fi%
        \pgf@y=\pgf@pathmaxy\advance\pgf@y by.5\pgflinewidth%
        \ifdim\pgf@y>\pgf@picmaxy\global\pgf@picmaxy\pgf@y\fi%
      \fi%
    \fi%
  \fi%
  %
  \ifx\pgf@up@clip\pgfutil@empty%
    \ifx\pgf@up@stroke\pgfutil@empty%
      \ifx\pgf@up@action\pgfutil@empty%
        \ifnum\pgf@tips@mode=2\relax%
          \pgf@up@draw@arrows@only%
        \fi%
        \ifnum\pgf@tips@mode=4\relax%
          \pgf@up@draw@arrows@only%
        \fi%
      \else%  
        \pgfsyssoftpath@invokecurrentpath%
        \pgf@up@action%
      \fi%
    \else%
      \pgfgetpath\pgf@arrowpath%
      \pgf@path@setup@tempswa%
      \pgfprocesscheckclosed{\pgf@arrowpath}{\pgfutil@tempswafalse}%
      \pgf@path@check@proper%
      \ifpgfutil@tempswa%
        \pgf@prepare@end@of@path%
        \begingroup%
          \pgf@prepare@start@of@path%
      \fi%
         \pgfsyssoftpath@invokecurrentpath%
          \pgf@up@action%
          \pgf@stroke@inner@line@if@needed%
      \ifpgfutil@tempswa%
          \pgf@add@arrow@at@start%
        \endgroup%
        \pgf@add@arrow@at@end%
      \fi%
    \fi%
  \else%
    \pgfsyssoftpath@invokecurrentpath%
    \pgfsys@clipnext%
    \pgf@up@action%
    \pgf@relevantforpicturesizefalse%
  \fi%
  \pgf@up@bb%
  \pgfsyssoftpath@setcurrentpath\pgfutil@empty%
  \pgf@resetpathsizes%
  \ignorespaces%
}
\def\pgf@path@setup@tempswa{%
  \ifnum\pgf@tips@mode>0\relax
    \pgfutil@tempswatrue%
  \else
    \pgfutil@tempswafalse%
  \fi%
}
\def\pgf@up@draw@arrows@only{%
  \pgfgetpath\pgf@arrowpath%
  \pgfutil@tempswatrue%
  \pgfprocesscheckclosed{\pgf@arrowpath}{\pgfutil@tempswafalse}%
  \pgf@path@check@proper%
  \ifpgfutil@tempswa%
    \pgf@prepare@end@of@path%
    \begingroup%
      \pgf@prepare@start@of@path%
      \pgf@add@arrow@at@start%
    \endgroup%
    \pgf@add@arrow@at@end%
  \fi%
}

\def\pgf@path@check@proper{%
  \ifpgfutil@tempswa%
    \ifnum\pgf@tips@mode>2\relax%
      \pgf@path@check@proper@%
    \fi%
  \fi%
}


\def\pgf@path@check@proper@{%
  {%
    \pgf@x0pt\pgf@y\pgf@x%
    \pgfutil@tempswatrue%
    \let\pgfsyssoftpath@movetotoken\pgf@path@proper@init%
    \let\pgfsyssoftpath@linetotoken\pgf@path@proper@test%%
    \let\pgfsyssoftpath@curvetosupportatoken\pgf@path@proper@test%
    \let\pgfsyssoftpath@curvetosupportbtoken=\pgf@path@proper@test%
    \let\pgfsyssoftpath@curvetotoken\pgf@path@proper@test%
    \let\pgfsyssoftpath@rectcornertoken\pgf@path@proper@test%
    \let\pgfsyssoftpath@rectsizetoken\pgf@path@proper@zero@test%
    \let\pgfsyssoftpath@closepathtoken\pgf@path@proper@test%
    % Execute the path:
    \pgf@arrowpath%
    \expandafter
  }%
  \ifpgfutil@tempswa\pgfutil@tempswafalse\else\pgfutil@tempswatrue\fi%
}

\def\pgf@path@proper@init#1#2{\pgfutil@tempswatrue\pgf@x#1\pgf@y#2}%
\def\pgf@path@proper@test#1#2{%
  \ifdim\pgf@x=#1\relax\else\pgfutil@tempswafalse\fi%
  \ifdim\pgf@y=#2\relax\else\pgfutil@tempswafalse\fi}
\def\pgf@path@proper@zero@test#1#2{%
  \ifdim\pgf@x=0pt\relax\else\pgfutil@tempswafalse\fi%
  \ifdim\pgf@y=0pt\relax\else\pgfutil@tempswafalse\fi}




\def\pgf@stroke@inner@line@if@needed{%
  \ifdim\pgfinnerlinewidth>0pt\relax%
    \let\pgf@temp@save=\pgf@strokecolor@global
    \pgfsys@beginscope%
    {%
      \pgfsys@setlinewidth{\pgfinnerlinewidth}%
      \pgfsetstrokecolor{\pgfinnerstrokecolor}%
      \pgfsyssoftpath@invokecurrentpath%
      \pgfsys@stroke%
    }%
    \pgfsys@endscope%
    \global\let\pgf@strokecolor@global=\pgf@temp@save
  \fi%
}

\let\pgf@prepare@start@of@path\relax%
\let\pgf@add@arrow@at@end\relax%
\let\pgf@add@arrow@at@start\relax%




% Shorten start/end of paths by a certain amount.
%
% #1 = amount by which paths should be shortened.
%
% Example:
%
% \pgfpathmoveto{\pgfpointorigin}
% \pgfpathlineto{\pgfpoint{10pt}{0pt}
%
% % The following has the same effect:
% \pgfsetshortenstart{1pt}
% \pgfpathmoveto{\pgfpointorigin}
% \pgfpathlineto{\pgfpoint{11pt}{0pt}

\def\pgfsetshortenstart#1{\pgfmathsetlength\pgf@shorten@start@additional{#1}}
\def\pgfsetshortenend#1{\pgfmathsetlength\pgf@shorten@end@additional{#1}}

\newdimen\pgf@shorten@end@additional
\newdimen\pgf@shorten@start@additional



% 
% 
% Handling the end of a path 
% 
%  
% The "handling" consists of first testing whether we need to do
% anything at all, namely because either an arrow tip should be drawn
% at the end or because the path should be shortened at the end. If
% so, we shorten the path if needed and, later on, add the arrow tip.
%

\newif\ifpgf@worry


% Prepare the end of the path: Test whether anything must be done
% (step 0) and, if so, split the path (step 1), extract the interesting points from
% the path (step 2), prepare further computations (step 3), shorten
% the path (step 4) if necessary, and add the arrow tip (step 4 in
% macro \pgf@add@arrow@at@end, which is called later). 

\def\pgf@prepare@end@of@path{%
  \let\pgfprocessresultpathsuffix\relax% flag that nothing has happened...
  \let\pgfprocessresultsubpathsuffix\relax%
  \pgfsyssoftpath@getcurrentpath\pgf@arrowpath%
  % 
  % Step 0 start:
  % 
  % Do we need to worry about the end? 
  % 
  \ifx\pgf@arrowpath\pgfutil@empty\else%
    \pgf@worryfalse%
    \ifx\pgf@end@tip@sequence\pgfutil@empty\else\pgf@worrytrue\fi% Yes, worry if we have to draw an arrow
    \pgf@precise@shorteningfalse%
    \pgf@arrow@compute@shortening\pgf@end@tip@sequence%
    \advance\pgf@xa by\pgf@shorten@end@additional%
    \advance\pgf@xb by\pgf@shorten@end@additional%
    \ifdim\pgf@xa=0pt\relax\else\pgf@worrytrue\fi% Also, worry if shortening is requested
    \edef\pgf@path@shortening@distance{\the\pgf@xa}%
    \edef\pgf@arrow@tip@total@length{\the\pgf@xb}%
    % 
    % Step 0 done.
    % 
    \ifpgf@worry%
      % Ok, need to "worry" about the end, either because we need to
      % shorten it or to draw an arrow head.
      %
      % Step 1: Split
      % 
      \pgfprocesssplitpath{\pgf@arrowpath}%
      \pgfprocesssplitsubpath{\pgfprocessresultpathsuffix}%
      % 
      % Step 2: extract
      %
      \expandafter\pgf@parse@end\pgfprocessresultsubpathsuffix\pgf@stop\pgf@stop\pgf@stop%
      % 
      % Step 3: prep 
      % 
      \pgf@prep@end%
      % 
      % Step 4: shorten
      % 
      \ifdim\pgf@path@shortening@distance=0pt\else\pgf@do@shorten@end\fi%
      \expandafter\expandafter\expandafter\def%
      \expandafter\expandafter\expandafter\pgf@arrowpath%
      \expandafter\expandafter\expandafter{\expandafter\pgfprocessresultpathprefix\pgfprocessresultpathsuffix}%
      \pgfsyssoftpath@setcurrentpath\pgf@arrowpath%
    \fi%
  \fi%
}

\def\pgf@parse@end#1#2#3#4#5#6{%
  \ifx#4\pgf@stop% Only a moveto! -> do nothing!
    \def\pgfpointlastonpath{\pgfqpoint{#2}{#3}}%
    \let\pgf@next\relax%
    \let\pgf@do@shorten@end\relax%
    \let\pgf@do@draw@end\pgf@do@draw@straightend%
    \let\pgf@prep@end\pgf@prep@movetoend%
  \else\ifx#4\pgfsyssoftpath@curvetosupportatoken% A curve -> this will get complicated...
    \def\pgfsubpathfourthlasttoken{#1}%
    \def\pgfpointfourthlastonpath{\pgfqpoint{#2}{#3}}%
    \def\pgfpointthirdlastonpath{\pgfqpoint{#5}{#6}}%
    \ifpgf@precise@shortening%
      \let\pgf@do@shorten@end\pgf@do@shorten@curvedend%
      \let\pgf@do@draw@end\pgf@do@draw@curvedend%
      \let\pgf@prep@end\pgf@prep@curveend%
    \else%
      \def\pgfsubpathstart{\noexpand#1{#2}{#3}\noexpand#4{#5}{#6}}%
      \let\pgf@do@shorten@end\pgf@do@shorten@straightend%
      \let\pgf@do@draw@end\pgf@do@draw@straightend%
      \let\pgf@prep@end\pgf@prep@straightend%
    \fi%
    \let\pgf@next\pgf@parse@end@curve@cont%
  \else% A straight line -> great!
    \def\pgfsubpathstart{\noexpand#1{#2}{#3}\noexpand#4}%
    \def\pgfpointsecondlastonpath{\pgfqpoint{#2}{#3}}%
    \def\pgfpointlastonpath{\pgfqpoint{#5}{#6}}%
    \let\pgfpointfourthlastonpath\relax%
    \let\pgfpointthirdlastonpath\relax%
    \let\pgf@next\pgf@parse@end@gobble@three%  
    \let\pgf@do@shorten@end\pgf@do@shorten@straightend%
    \let\pgf@do@draw@end\pgf@do@draw@straightend%
    \let\pgf@prep@end\pgf@prep@straightend%
  \fi\fi%
  \pgf@next% Needed for reading rest of path...
}

\def\pgf@parse@end@gobble@three#1#2#3{}%
\def\pgf@parse@end@curve@cont#1#2#3#4#5#6#7#8#9{%
  \def\pgfpointsecondlastonpath{\pgfqpoint{#2}{#3}}%
  \def\pgfpointlastonpath{\pgfqpoint{#5}{#6}}%
  \ifpgf@precise@shortening%
  \else%
    \expandafter\def\expandafter\pgfsubpathstart\expandafter{\pgfsubpathstart\noexpand#1{#2}{#3}\noexpand#4}%
  \fi%
}

% 
% Preps 
%
\def\pgf@prep@movetoend{%
  \pgf@process{\pgfpointlastonpath}%
  \pgf@xb\pgf@x
  \pgf@yb\pgf@y
  \pgf@xc\pgf@x
  \pgf@yc\pgf@y
}
\def\pgf@prep@straightend{%
  \let\pgf@ref\pgfpointsecondlastonpath
  \ifx\pgfpointlastonpath\pgfpointsecondlastonpath% degenerate!
    \ifx\pgfpointthirdlastonpath\relax% cannot help it: really a zero-length path
    \else%
      \ifx\pgfpointlastonpath\pgfpointthirdlastonpath% double degenerate!
        \let\pgf@ref\pgfpointfourthlastonpath% Use third point for reference.
      \else
        \let\pgf@ref\pgfpointthirdlastonpath% Use third point for reference.
      \fi%
    \fi%
  \fi%
  \pgfpointlineatdistance{\pgf@path@shortening@distance}{\pgfpointlastonpath}{\pgf@ref}%
  \pgf@xa\pgf@x%
  \pgf@ya\pgf@y%
}


% 
% Line shortening for straight lines:
%
\def\pgf@do@shorten@straightend{%
  \edef\pgfprocessresultsubpathsuffix{\pgfsubpathstart{\the\pgf@xa}{\the\pgf@ya}}%
  \expandafter\expandafter\expandafter\def%
  \expandafter\expandafter\expandafter\pgfprocessresultpathsuffix%
  \expandafter\expandafter\expandafter{\expandafter\pgfprocessresultsubpathprefix\pgfprocessresultsubpathsuffix}%
}


% 
% Draw an end arrow by calling an appropriate subfunction, if necessary
% 

\def\pgf@add@arrow@at@end{%
  \ifx\pgf@arrowpath\pgfutil@empty\else%
    \ifx\pgf@end@tip@sequence\pgfutil@empty\else%
      \pgf@do@draw@end%
    \fi%    
  \fi%
}

% 
% Draw an end arrow at the end of a straight line
% 
\def\pgf@do@draw@straightend{%
  {%
    \pgftransformreset%
    \pgftransformarrow{\pgfqpoint{\pgf@xc}{\pgf@yc}}{\pgfqpoint{\pgf@xb}{\pgf@yb}}%
    \pgf@arrow@draw@arrow\pgf@end@tip@sequence\pgf@arrow@tip@total@length%
  }%
}



% 
% 
% Handling the start of a path 
% 
%  
% The "handling" is similar to the case for the start of the path. We
% may be able to skip the splitting if that was done already for the
% end. Otherwise, things are basically the same.
%

% Prepare the start of the path: Test whether anything must be done
% (step 0) and, if so, split the path (step 1) if necesssary, extract
% the interesting points from the path (step 2), prepare computations
% (step 3) needed for both shortening and tip adding, shorten the path
% (step 4), and add the arrow tip (step 5 in macro
% \pgf@add@arrow@at@start, which is called later). 

\def\pgf@prepare@start@of@path{%
  % 
  % Step 0 start:
  % 
  % Do we need to worry about the start? 
  % 
  \ifx\pgf@arrowpath\pgfutil@empty\else%
    \pgf@worryfalse%
    \ifx\pgf@start@tip@sequence\pgfutil@empty\else\pgf@worrytrue\fi% Yes, worry if we have to draw an arrow
    \pgf@precise@shorteningfalse%
    \pgf@arrow@compute@shortening\pgf@start@tip@sequence%
    \advance\pgf@xa by\pgf@shorten@start@additional%
    \advance\pgf@xb by\pgf@shorten@start@additional%
    \ifdim\pgf@xa=0pt\relax\else\pgf@worrytrue\fi% Also, worry if shortening is requested
    \edef\pgf@path@shortening@distance{\the\pgf@xa}%
    \edef\pgf@arrow@tip@total@length{\the\pgf@xb}%
    % 
    % Step 0 done.
    % 
    \ifpgf@worry%
      % Ok, need to "worry" about the start, either because we need to
      % shorten it or to draw an arrow head.
      %
      % Step 1: Split
      % 
      \ifx\pgfprocessresultpathsuffix\relax%
        % Ok, still need to compute the split:
        \pgfprocesssplitpath{\pgf@arrowpath}%
      \fi%
      % 
      % Step 2: extract
      %
      \expandafter\pgf@parse@start\pgfprocessresultpathsuffix\pgf@stop\pgf@stop\pgf@stop%
      % 
      % Step 3: prep
      %
      \pgf@prep@start%
      % 
      % Step 4: shorten
      % 
      \ifdim\pgf@path@shortening@distance=0pt\else\pgf@do@shorten@start\fi%
      \expandafter\expandafter\expandafter\def%
      \expandafter\expandafter\expandafter\pgf@arrowpath%
      \expandafter\expandafter\expandafter{\expandafter\pgfprocessresultpathprefix\pgfprocessresultpathsuffix}%
      \pgfsyssoftpath@setcurrentpath\pgf@arrowpath%
    \fi%
  \fi%
}

\def\pgf@parse@start#1#2#3#4#5#6{%
  \def\pgfpointfirstonpath{\pgfqpoint{#2}{#3}}%
  \def\pgfpointsecondonpath{\pgfqpoint{#5}{#6}}%
  \let\pgfpointthirdonpath\relax%
  \let\pgfpointfourthonpath\relax%
  \def\pgfsubpathfirsttoken{\noexpand#1}%
  \def\pgfsubpathsecondtoken{\noexpand#4}%
  \ifx#4\pgf@stop% Only a moveto! -> do nothing!
    \let\pgf@next\relax%
    \let\pgf@do@shorten@start\relax%
    \let\pgf@prep@start\pgf@prep@movetostart%
    \let\pgf@do@draw@start\pgf@do@draw@straightstart%
  \else\ifx#4\pgfsyssoftpath@curvetosupportatoken% A curve -> this will get complicated...
    \ifpgf@precise@shortening%
      \let\pgf@next\pgf@parse@start@curve@cont%
      \let\pgf@do@shorten@start\pgf@do@shorten@curvedstart%
      \let\pgf@do@draw@start\pgf@do@draw@curvedstart%
      \let\pgf@prep@start\pgf@prep@curvedstart%
    \else% can treat the end like a straight line...
      \let\pgf@next\pgf@parse@start@curve@cont@straight%
      \let\pgf@do@shorten@start\pgf@do@shorten@straightstart%
      \let\pgf@do@draw@start\pgf@do@draw@straightstart%
      \let\pgf@prep@start\pgf@prep@straightstart%
    \fi%
  \else% A straight line -> great!
    \let\pgf@next\pgf@parse@start@till@stop%  
    \let\pgf@do@shorten@start\pgf@do@shorten@straightstart%
    \let\pgf@do@draw@start\pgf@do@draw@straightstart%
    \let\pgf@prep@start\pgf@prep@straightstart%
  \fi\fi%
  \pgf@next% Needed for reading rest of path...
}

\def\pgf@parse@start@till@stop#1\pgf@stop\pgf@stop\pgf@stop{\def\pgfsubpathend{#1}}%

\def\pgf@parse@start@curve@cont#1#2#3#4#5#6#7\pgf@stop\pgf@stop\pgf@stop{%
  \def\pgfpointthirdonpath{\pgfqpoint{#2}{#3}}%
  \def\pgfpointfourthonpath{\pgfqpoint{#5}{#6}}%
  \def\pgfsubpathend{#7}%
}

\def\pgf@parse@start@curve@cont@straight#1#2#3#4#5#6#7\pgf@stop\pgf@stop\pgf@stop{%
  \def\pgfpointthirdonpath{\pgfqpoint{#2}{#3}}%
  \def\pgfpointfourthonpath{\pgfqpoint{#5}{#6}}%
  \def\pgfsubpathend{#1{#2}{#3}#4{#5}{#6}#7}%
}

% 
% Preps 
%
\def\pgf@prep@movetostart{%
  \pgf@process{\pgfpointfirstonpath}%
  \pgf@xb\pgf@x
  \pgf@yb\pgf@y
  \pgf@xc\pgf@x
  \pgf@yc\pgf@y
}
\def\pgf@prep@straightstart{%
  \let\pgf@ref\pgfpointsecondonpath
  \ifx\pgfpointfirstonpath\pgfpointsecondonpath% degenerate!
    \ifx\pgfpointthirdonpath\relax% cannot help it: really a zero-length path
    \else%
      \ifx\pgfpointfirstonpath\pgfpointthirdonpath% double degenerate!
        \let\pgf@ref\pgfpointfourthonpath% Use third point for reference.
      \else
        \let\pgf@ref\pgfpointthirdonpath% Use third point for reference.
      \fi%
    \fi%
  \fi%
  \pgfpointlineatdistance{\pgf@path@shortening@distance}{\pgfpointfirstonpath}{\pgf@ref}%
  \pgf@xa\pgf@x%
  \pgf@ya\pgf@y%
}

% 
% Line shortening for straight lines:
%
\def\pgf@do@shorten@straightstart{%
  \edef\pgfprocessresultpathsuffix{\pgfsubpathfirsttoken{\the\pgf@xa}{\the\pgf@ya}\pgfsubpathsecondtoken{\the\pgf@xc}{\the\pgf@yc}}%
  \expandafter\expandafter\expandafter\def%
  \expandafter\expandafter\expandafter\pgfprocessresultpathsuffix%
  \expandafter\expandafter\expandafter{\expandafter\pgfprocessresultpathsuffix\pgfsubpathend}%
}


% 
% Draw a start arrow by calling an appropriate subfunction, if necessary
% 

\def\pgf@add@arrow@at@start{%
  \ifx\pgf@arrowpath\pgfutil@empty\else%
    \ifx\pgf@start@tip@sequence\pgfutil@empty\else%
      \pgf@do@draw@start%
    \fi%    
  \fi%
}

% 
% Draw an start arrow at the start of a straight line
% 
\def\pgf@do@draw@straightstart{%
  {%
    \pgftransformreset%
    \pgftransformarrow{\pgfqpoint{\pgf@xc}{\pgf@yc}}{\pgfqpoint{\pgf@xb}{\pgf@yb}}%
    \pgf@arrow@draw@arrow\pgf@start@tip@sequence\pgf@arrow@tip@total@length%
  }%
}



\let\pgf@shorten@end=\pgfutil@empty
\let\pgf@shorten@start=\pgfutil@empty


\endinput%





