%--------------------------------------------
%
% Package pgfplots
%
% Provides a user-friendly interface to create function plots (normal
% plots, semi-logplots and double-logplots).
% 
% It is based on Till Tantau's PGF package.
%
% Copyright 2013 by Christian Feuersaenger
%
% This program is free software: you can redistribute it and/or modify
% it under the terms of the GNU General Public License as published by
% the Free Software Foundation, either version 3 of the License, or
% (at your option) any later version.
% 
% This program is distributed in the hope that it will be useful,
% but WITHOUT ANY WARRANTY; without even the implied warranty of
% MERCHANTABILITY or FITNESS FOR A PARTICULAR PURPOSE.  See the
% GNU General Public License for more details.
% 
% You should have received a copy of the GNU General Public License
% along with this program.  If not, see <http://www.gnu.org/licenses/>.
%
%--------------------------------------------


% This library adds support for high-level instructions for "fill area
% between two arbitrary plots of functions".
%
% It activates the syntax \pgfpathfillbetween
% where A and B are two plots named by 'name path='.
%
% In fact, this here is not much more than a low-level invocation of
% \pgfpathfillbetween
% and a couple of styles. It could become a TikZ library because it
% actually works on any two named paths, but it has its restrictions
% regarding the supported input paths: both need to be plots of
% functions (non-intersecting, should have at most one function value
% for each canvas X coord)

\pgfutil@IfUndefined{pgfplotsset}{%
	\pgferror{Please load pgfplots before pgfplots.fillbetween.}%
	\endinput
}{}%

\usetikzlibrary{intersections}
\usetikzlibrary{decorations.softclip}

% COMPATIBILITY WITH PGF RELEASE:
\pgfutil@IfUndefined{pgfintersectiongetsolutiontimes}{%
	\pgfplotsusecompatibilityfile{pgflibraryintersections.code.tex}%
}{}%
\def\tikz@key@name@path@wrong#1#2{%
    \tikz@addmode{%
      \pgfsyssoftpath@getcurrentpath\tikz@intersect@temppath@round%
      \pgfprocessround\tikz@intersect@temppath@round\tikz@intersect@temppath%
      \ifx\tikz@intersect@namedpaths\pgfutil@empty%
      \else%
        \tikz@intersect@namedpaths%
        \pgfutil@ifundefined{tikz@intersect@path@name@#1}{}%
        {%
          \expandafter\expandafter\expandafter\def\expandafter\expandafter\expandafter\tikz@intersect@@temppath%
          \expandafter\expandafter\expandafter{\csname tikz@intersect@path@name@#1\endcsname}%
          \expandafter\expandafter\expandafter\def\expandafter\expandafter\expandafter\tikz@intersect@temppath%
          \expandafter\expandafter\expandafter{\expandafter\tikz@intersect@temppath\tikz@intersect@temppath}%
        }%
      \fi%
      \tikz@intersect@addto@path@names{#1}{#2}%
    }%
}%

\def\tikz@key@name@path@new#1#2{%
    \tikz@addmode{%
      \pgfsyssoftpath@getcurrentpath\tikz@intersect@temppath@round%
      \pgfprocessround\tikz@intersect@temppath@round\tikz@intersect@temppath%
      \ifx\tikz@intersect@namedpaths\pgfutil@empty%
      \else%
        \tikz@intersect@namedpaths%
      \fi%
      \tikz@intersect@addto@path@names{#1}{#2}%
    }%
}%

\ifx\tikz@key@name@path@wrong\tikz@key@name@path
	\immediate\write16{Package pgfplots: loading complementary 'name path' implementation for your pgf version...}
	\let\tikz@key@name@path=\tikz@key@name@path@new
\fi

% ------------------------------------------------------------

\newif\iftikzfillbetween@optimize@name@intersections

\pgfkeys{%
	/tikz/fill between/of/.code=\tikzlibraryfillbetween@parse#1\pgf@stop,
	%
	/tikz/fill between/on layer/.initial=pre main,
	/tikz/fill between/every segment/.style={},
	/tikz/fill between/every odd segment/.style={},
	/tikz/fill between/every even segment/.style={},
	/tikz/fill between/every last segment/.style={},
	%
	% Allows to add path instructions *after* the segment.
	% If you want to add some *before* the segment, you can rely on
	% 'every segment no 0/.style={ ... suitable tikz options ... }'
	/tikz/fill between/path after segment/.initial={},
	%
	% soft clip={(axis cs:0,0) rectangle (axis cs:1,1)}
	/tikz/fill between/soft clip/.style={
		/tikz/fill between/soft clip first={#1},%
		/tikz/fill between/soft clip second={#1},%
	},%
	/tikz/fill between/soft clip first/.initial=,%
	/tikz/fill between/soft clip second/.initial=,
	%
	% #1: drawing options.
	/tikz/fill between/@draw style/.style={
		/pgf/fill between/result stream/begin/.code={%
			\gdef\tikzsegmentindex{0}%
			\xdef\tikzsegmentindices{##1}%
			\c@pgf@countc=##1 %
			\advance\c@pgf@countc by-1 %
			\xdef\tikzsegmentlastindex{\the\c@pgf@countc}%
		},%
		/pgf/fill between/result stream/next ready/.code={%
			\let\pgflibraryfill@path=\pgfretval
			\def\pgfplots@loc@TMPa{%
				/tikz/fill between/every segment,
				#1,%
				/tikz/fill between/every segment no \tikzsegmentindex/.try,
			}%
			\ifodd\tikzsegmentindex\relax
				\expandafter\def\expandafter\pgfplots@loc@TMPa\expandafter{\pgfplots@loc@TMPa
					/tikz/fill between/every odd segment,
				}%
			\else
				\expandafter\def\expandafter\pgfplots@loc@TMPa\expandafter{\pgfplots@loc@TMPa
					/tikz/fill between/every even segment,
				}%
			\fi
			%
			\ifnum\tikzsegmentindex=\tikzsegmentlastindex\relax
				\expandafter\def\expandafter\pgfplots@loc@TMPa\expandafter{\pgfplots@loc@TMPa
					/tikz/fill between/every last segment,
				}%
			\fi
			%
			\expandafter\fill\expandafter[\pgfplots@loc@TMPa]
				\pgfextra 
				\pgfsetpathandBB{\pgflibraryfill@path}%
				\pgfkeysgetvalue{/tikz/fill between/path after segment}\tikz@fillbetween@post@segment
				\expandafter
				\endpgfextra
				\tikz@fillbetween@post@segment
			;%
			\pgfplotsutil@advancestringcounter@global\tikzsegmentindex
		},%
		/pgf/fill between/result stream/end/.code=,%
	},
	/tikz/fill between/.search also={/pgf/fill between,/pgfplots},
	/tikz/fill between/optimize name intersections/.is if=tikzfillbetween@optimize@name@intersections,
	% 
	% FIXME : this optimization needs much more work... I believe it
	% would be stable enough, but it covers too few cases.
	%/tikz/fill between/optimize name intersections=true,
	%
	%
	%
	%--------------------------------------------------
	% /pgfplots/execute at begin axis@@/.add={%
	% 	\def\b@pgfplotslibraryfill@added
	% }{%
	% 	
	% },%
	%-------------------------------------------------- 
}

\def\tikzlibraryfillbetween@parse#1 and #2\pgf@stop{%
	\def\tikz@fillbetween@a{#1}%
	\def\tikz@fillbetween@b{#2}%
}%

% \tikzfillbetween[<options>]{<draw style>}
%
% <options> must contain 'of=<A> and <B>' and may configure how the
% area is computed.
%
% <draw style> affects every drawn region.
%
\def\tikzfillbetween{\pgfutil@ifnextchar[{\tikzfillbetween@opt}{\tikzfillbetween@opt[]}}
\def\tikzfillbetween@opt[#1]#2{%
	\begingroup
	\pgfqkeys{/tikz/fill between}{% 
		% prepare the low-level path generation instruction(s):
		/tikz/fill between/@draw style={#2},%
		%
		% set up input options:
		#1,%
	}%
	% automagically try to use the correct layer:
	\pgfkeysgetvalue{/tikz/fill between/on layer}\tikzlibraryfillbetween@path@layer@name
	\tikzlibraryfillbetween@determine@layer
	%
	\ifx\tikzlibraryfillbetween@path@layer@name\pgfutil@empty \else
		\pgfonlayer{\tikzlibraryfillbetween@path@layer@name}%
	\fi
	%
		\tikzlibraryfillbetween@path@generatepath
	%
	\ifx\tikzlibraryfillbetween@path@layer@name\pgfutil@empty \else
		\endpgfonlayer%
	\fi
	%
	\endgroup
}%

% Defines \tikzlibraryfillbetween@path@layer@name
\def\tikzlibraryfillbetween@determine@layer{%
	\ifx\tikzlibraryfillbetween@path@layer@name\pgfutil@empty
	\else
		\pgfutil@IfUndefined{pgf@layerlist}{%
			% hm. No layers active!? A pity...
			\tikzlibraryfillbetween@path@warn@layer
			\let\tikzlibraryfillbetween@path@layer@name\pgfutil@empty%
		}{%
			\edef\pgfplots@loc@TMPa{\noexpand\pgfutil@in@{\tikzlibraryfillbetween@path@layer@name}}%
			\expandafter\pgfplots@loc@TMPa\expandafter{\pgf@layerlist}%
			\ifpgfutil@in@
			\else
				\tikzlibraryfillbetween@path@warn@layer
				\let\tikzlibraryfillbetween@path@layer@name\pgfutil@empty%
			\fi
		}%
	\fi
}%

\def\tikzlibraryfillbetween@path@warn@layer{%
	\pgfplots@warning{'fill between': Could not activate graphics layer '\tikzlibraryfillbetween@path@layer@name'. Filled path will be on top of the other ones. Please ensure that '\tikzlibraryfillbetween@path@layer@name' is somewhere in the layer list (or set '/tikz/fill between/on layer=').}%
}%

\def\tikzlibraryfillbetween@parse@softclip{%
	\pgfkeysgetvalue{/tikz/fill between/soft clip first}\pgf@temp
	\pgfkeysgetvalue{/tikz/fill between/soft clip second}\pgf@tempb
	\ifx\pgf@temp\pgf@tempb
		% Ah - both have the same value!
		\ifx\pgf@temp\pgfutil@empty
			% ... and both are empty.
			\pgffillbetweensetsoftclippath{\pgfutil@empty}%
		\else
			% ... and both contain some path! Process it (once):
			\def\tikz@marshal{\tikzlibsoftclip@setkey{\pgffillbetweensetsoftclippath}}%
			\expandafter\tikz@marshal\pgf@temp\pgf@stop
		\fi
	\else
		% handle 'soft clip first':
		\ifx\pgf@temp\pgfutil@empty
			\pgffillbetweensetsoftclippathfirst{\pgfutil@empty}%
		\else
			\def\tikz@marshal{\tikzlibsoftclip@setkey{\pgffillbetweensetsoftclippathfirst}}%
			\expandafter\tikz@marshal\pgf@temp\pgf@stop
		\fi
		%
		% handle 'soft clip second':
		\pgfkeysgetvalue{/tikz/fill between/soft clip second}\pgf@tempb
		\ifx\pgf@tempb\pgfutil@empty
			\pgffillbetweensetsoftclippathsecond{\pgfutil@empty}%
		\else
			\def\tikz@marshal{\tikzlibsoftclip@setkey{\pgffillbetweensetsoftclippathsecond}}%
			\expandafter\tikz@marshal\pgf@tempb\pgf@stop
		\fi
	\fi
}%

\def\tikzlibraryfillbetween@path@generatepath{%
	%
	\tikzlibraryfillbetween@parse@softclip
	%
	\tikzlibraryfillbetween@path@check
	\expandafter\let\expandafter\tikz@fillbetween@a@path
		\csname tikz@intersect@path@name@\tikz@fillbetween@a\endcsname
	\expandafter\let\expandafter\tikz@fillbetween@b@path
		\csname tikz@intersect@path@name@\tikz@fillbetween@b\endcsname
	%
	\iftikzfillbetween@optimize@name@intersections
		\ifpgfpathfillbetween@split
			\tikzfillbetween@optimize@name@intersections\tikz@fillbetween@a\tikz@fillbetween@b
		\fi
	\fi
	%
	\pgfpathfillbetween{\tikz@fillbetween@a@path}{\tikz@fillbetween@b@path}%
}%

\def\tikzlibraryfillbetween@path@check{%
	\pgfutil@IfUndefined{tikz@intersect@path@name@\tikz@fillbetween@a}{%
		\pgferror
		{fill between: the mandatory argument 'of=<name path A> and <name path B> is missing or has empty arguments. Please ensure that the option has been set and that both path names have been assigned (perhaps you need 'name path global=\tikz@fillbetween@a' somewhere?)}%
	}{}%
	\pgfutil@IfUndefined{tikz@intersect@path@name@\tikz@fillbetween@b}{%
		\pgferror
		{fill between: the mandatory argument 'of=<name path A> and <name path B> is missing or has empty arguments. Please ensure that the option has been set and that both path names have been assigned (perhaps you need 'name path global=\tikz@fillbetween@b' somewhere?)}%
	}{}%
}


%-----------------------------------------
%
% Utilities to work with path segments
%
%-----------------------------------------

% Defines \pgfretval such that it contains the named path '#1'
\def\tikzgetnamedpath#1{%
	\pgfutil@IfUndefined{tikz@intersect@path@name@#1}{%
		\pgferror{There is no named path called '#1'. Perhaps you misspelled it?}%
	}{%
		\expandafter\let\expandafter\pgfretval
			\csname tikz@intersect@path@name@#1\endcsname
	}%
}%

\def\tikznamecurrentpath#1{%
	\pgfgetpath\pgf@temp
	\pgfprocessround\pgf@temp\pgf@tempb%
	\expandafter\global\expandafter\let\csname tikz@intersect@path@name@#1\endcsname=\pgf@tempb
}%

% ---------------------------------------------------------------------------------------

\newif\iftikzpathsegments@reverse
\def\tikzlibrarysegments@parse#1 and #2\pgf@stop{%
	\def\tikz@path@segments@a{#1}%
	\def\tikz@path@segments@b{#2}%
}%


\pgfkeys{
	% allows to append intersection segments to the current path by
	% writing
	% \path[intersection segments={of=first and second, sequence=A0 -- B1 -- B3 A3[reverse] -- A1}];
	/tikz/intersection segments/.code={%
		\tikzpathintersectionsegments[#1]%
	},
	/tikz/segments/of/.code=\tikzlibrarysegments@parse#1\pgf@stop,
	/tikz/segments/of={} and {},
	/tikz/segments/sequence/.initial=A0 -- B1,
	/tikz/segments/reverse/.is if=tikzpathsegments@reverse,
	/tikz/segments/reverse/.default=true,
}

% Programmatic method to do the same as 'intersection
% segments={<options>}
%
% #1: <options>
\def\tikzpathintersectionsegments[#1]{%
	\begingroup
	\pgfqkeys{/tikz/segments}{#1}%
	\tikzgetnamedpath{\tikz@path@segments@a}%
	\let\tikz@path@segments@A=\pgfretval
	\tikzgetnamedpath{\tikz@path@segments@b}%
	\let\tikz@path@segments@B=\pgfretval
	%
	% this macro will be called whenever we need an intersection
	% segment. Perhaps we do not need it at all
	\def\tikz@ensurehascomputedintersection{%
		% compute intersections using the PGF intersection lib...
		\pgfintersectionofpaths{\pgfsetpath\tikz@path@segments@A}{\pgfsetpath\tikz@path@segments@B}%
		%
		\ifnum\pgfintersectionsolutions=0 %
			\pgferror{The argument of 'sequence' requests an intersection segment -- but the two input paths do not intersect. Please use A* or B* to select the whole path}%
			\def\b@tikz@select@all{1}%
		\else
			%
			% ... and compute the intersection *segments* for both input
			% paths...
			\pgfcomputeintersectionsegments1%
			\pgfcomputeintersectionsegments2%
		\fi
		\let\tikz@ensurehascomputedintersection=\relax
	}%
	%
	\pgfkeysgetvalue{/tikz/segments/sequence}\tikz@sequence
	%
	\def\b@tikz@isect@nextismoveto{1}%
	\let\pgfpointlastofsetpath=\pgfutil@empty
	\expandafter\tikzpathintersectionsegments@parse@loop\tikz@sequence\pgf@stop
	\pgfmath@smuggleone\pgfpointlastofsetpath
	\endgroup
	\tikz@make@last@position{\pgfpointlastofsetpath}%
}%

\def\tikzpathintersectionsegments@parse@loop{%
	\pgfutil@ifnextchar-{%
		\tikz@isect@p@lineto
	}{%
		\pgfutil@ifnextchar A{%
			\tikz@isect@p@next
		}{%
			\pgfutil@ifnextchar B{%
				\tikz@isect@p@next
			}{%
				\pgfutil@ifnextchar L{%
					\tikz@isect@p@next
				}{%
					\pgfutil@ifnextchar R{%
						\tikz@isect@p@next
					}{%
						\pgfutil@ifnextchar\pgf@stop{%
							\tikz@isect@finish
						}{%
							\tikz@isect@p@error
						}%
					}%
				}%
			}%
		}%
	}%
}%

\def\tikz@isect@finish\pgf@stop{}%

\def\tikz@isect@p@lineto--{%
	\def\b@tikz@isect@nextismoveto{0}%
	\tikzpathintersectionsegments@parse@loop
}%

\def\tikz@isect@p@error#1\pgf@stop{%
	\pgferror{The argument of 'sequence' has an unexpected format near '#1'. Please write something like A0 -- B1 -- A1}%
}

\def\tikz@isect@p@next#1#2{%
	\def\tikz@temp{#2}%
	\def\tikz@@temp{-}%
	\ifx\tikz@@temp\tikz@temp
		% also accept minus signs without curly braces, i.e. 
		% L-2 instead of L{-2}
		\def\tikz@next{\tikz@isect@p@next@{#1}{#2}}%
	\else
		\def\tikz@next{\tikz@isect@p@next@{#1}{#2}{}}%
	\fi
	\tikz@next
}%
\def\tikz@isect@p@next@#1#2#3{%
	\pgfutil@ifnextchar[{%
		\tikz@isect@p@next@opt{#1}{#2#3}%
	}{%
		\tikz@isect@p@next@opt{#1}{#2#3}[]%
	}%
}%

\def\tikz@isect@p@next@opt#1#2[#3]{%
	\begingroup
	%
	% set keys (if any):
	\def\tikz@temp{#3}%
	\ifx\tikz@temp\pgfutil@empty
	\else
		\pgfqkeys{/tikz/segments}{#3}%
	\fi
	%
	\def\tikz@indexshift{}%
	\if A#1%
		% FIRST function (0-based index)
		\def\tikz@path{1}%
		\def\tikz@path@ab{A}%
	\else
		\if B#1%
			% SECOND function (0-based index)
			\def\tikz@path{2}%
			\def\tikz@path@ab{B}%
		\else
			\if L#1%
				% FIRST function (1-based index)
				\def\tikz@path{1}%
				\def\tikz@indexshift{-1}%
				\def\tikz@path@ab{A}%
			\else
				\if R#1%
					% SECOND function (1-based index)
					\def\tikz@path{2}%
					\def\tikz@indexshift{-1}%
					\def\tikz@path@ab{B}%
				\else
					\def\tikz@path{}%
					\pgferror{The argument of 'sequence' has an unexpected format near '#1#2': expected L#2 or R#2}%
				\fi
			\fi
		\fi
	\fi
	%
	% parse arguments:
	\def\pgfmathresult{#2}%
	\def\tikz@temp{*}%
	\ifx\pgfmathresult\tikz@temp
		\def\b@tikz@select@all{1}%
	\else
		\def\b@tikz@select@all{0}%
		\tikz@ensurehascomputedintersection
		\pgfmathparse{round(#2)}%
		\let\tikz@index=\pgfmathresult
		\ifx\tikz@indexshift\pgfutil@empty
		\else
			\afterassignment\pgfutil@gobble@until@relax
			\c@pgf@countc=\tikz@index\relax
			\ifnum\c@pgf@countc<0 \else
				% the index shift is ONLY for positive numbers: we
				% want to start indexing at 1, not at 0 -- and the
				% negative ones start at -1 anyway.
				\advance\c@pgf@countc by\tikz@indexshift\relax
				\edef\tikz@index{\the\c@pgf@countc}%
			\fi
		\fi
	\fi
	%
	%
	% PROCESS IT:
	\ifx\tikz@path\pgfutil@empty
		\let\pgfpointlastofsetpath=\pgfutil@empty
	\else
		\if1\b@tikz@select@all%
			% ok... select the *entire* path. 
			% #1 = A|B :
			\expandafter\let\expandafter\pgfretval\csname tikz@path@segments@\tikz@path@ab\endcsname
		\else
			\pgfgetintersectionsegmentpath{\tikz@path}{\tikz@index}%
		\fi
		\iftikzpathsegments@reverse
			\pgf@reverse@path\pgfretval
		\fi
		\if0\b@tikz@isect@nextismoveto
			\pgfpathreplacefirstmoveto\pgfretval
		\fi
		\pgfaddpathandBB\pgfretval
	\fi
	\pgfmath@smuggleone\pgfpointlastofsetpath
	\endgroup
	\def\b@tikz@isect@nextismoveto{1}%
	\tikzpathintersectionsegments@parse@loop
}%

\def\tikzfillbetween@optimize@name@intersections#1#2{%
	\edef\tikzfillbetween@precached@intersectionofpaths@A{#1}%
	\edef\tikzfillbetween@precached@intersectionofpaths@B{#2}%
	\pgfkeys{%
		/tikz/name intersections/.add code={%
			\let\pgfintersectionofpaths=\tikzfillbetween@precached@intersectionofpaths
		}{%
			\let\pgfintersectionofpaths=\pgfintersectionofpaths@orig
		}
	}%
}%

\let\pgfintersectionofpaths@orig=\pgfintersectionofpaths

\def\tikzfillbetween@precached@intersectionofpaths@log{%
	\immediate\write-1{fill between: outcome of 'name intersections={of=\tikz@intersect@path@a\space and \tikz@intersect@path@b}' has been computed from available information of fill between}%
}
\def\tikzfillbetween@precached@intersectionofpaths#1#2{%
	\def\pgf@loc@TMPa{0}%
	\ifx\tikz@intersect@path@a\tikzfillbetween@precached@intersectionofpaths@A
		\ifx\tikz@intersect@path@b\tikzfillbetween@precached@intersectionofpaths@B
			\def\pgf@loc@TMPa{1}%
		\fi
	\fi
	\ifx\tikz@intersect@path@b\tikzfillbetween@precached@intersectionofpaths@A
		\ifx\tikz@intersect@path@a\tikzfillbetween@precached@intersectionofpaths@B
			\def\pgf@loc@TMPa{1}%
		\fi
	\fi
	%
	\if1\pgf@loc@TMPa
		\tikzfillbetween@precached@intersectionofpaths@log
		\relax
	\else
		\pgfintersectionofpaths@orig{#1}{#2}%
	\fi
}%
\endinput
