% This is MEX2.TEX - a part of the Polish plain TeX and LaTeX formats.
% Version 1.05,  18 December 1993
%
% Authors: Marek Ry\'cko & Bogus\l{}aw Jackowski
%
% This macro file belongs to the public domain
% under the conditions specified by the author of TeX:
%
%   ``Macro files like PLAIN.TEX should not be changed in any way,
%     except with respect to preloaded fonts,
%     unless the changes are authorized by the authors of the macros.''
%
%                                           Donald E. Knuth
%
% For details see MEXINFO.ENG or MEXINFO.POL.

\n@wlanguages

\message{redefinitions of plain rules,}

% Polish \dospecials sometimes must \do something also with active /
\let\plaindospecials\dospecials % save the original meaning
\expandafter\def\expandafter\dospecials\expandafter
{\dospecials \ifx\pr@fix T\do\/\fi}
% see also the redefinition of \@sanitize in LaMeX

% The definition of \=
% If you type
%    `niebiesko\=czerwony'
% to the input file then it will be hyphenated
%    niebiesko-
%    -czerwony
% instead of:
%    niebiesko-
%    czerwony
% The possibility of hyphenating both `niebiesko' and `czerwony' is retained.
\exhyphenpenalty=10000 % breaking after explicit hyphen disabled
\let\macron\=
\def\={\kern0sp\discretionary{-}{-}{-}\penalty10000\hskip0sp\relax}
% \kern0sp allows for the hyphenation of a word to the left of \=
% \hskip0sp starts a new potentialy hyphenatable word
% \penalty10000 disables line breaking at the glue.
% (See The TeXbook, p. 454)
% \penalty10000 is used instead of \nobreak deliberately because
% \nobreak expands to \penalty\@MM and may cause troubles when written
% in the expanded version to a file which is read while \catcode`@=12.
% Please note that \= will work independently of the current value
% of \hyphenchar. Always the character `-' is used. In particular, when
% \hyphenchar does not lay between 0 and 255 breaking at \= is still allowed.

% logos:
\def\f@sl{\fontdimen1\the\font}% font slant
% MeX
\def\nM@X{M\kern-.111em\lower.6ex\hbox{E}\kern-.075emX}
% to be used with slanted typefaces:
\def\sM@X{M\kern-.25em\lower.6ex\hbox{E}\kern-.075emX}
% general macro:
\def\MeX{\ifnum\f@sl=0 \nM@X \else \sM@X \fi}

% slightly different Polish mathematical customs
% (write \arc\tg to obtain `arc tg' (with a tiny space))
\def\arc#1{\mathop{\rm arc\,\escapechar-1 \string#1}\nolimits}

\def\tg{\mathop{\rm tg}\nolimits}
\def\tgh{\mathop{\rm tgh}\nolimits}
\def\ctg{\mathop{\rm ctg}\nolimits}
\def\ctgh{\mathop{\rm ctgh}\nolimits}

% different shapes of less-or-equal and greater-or-equal:
% in new positions in the font PLSY10 (family 2)
\mathchardef\xleq="32AC \let\xle=\xleq
\mathchardef\xgeq="32AD \let\xge=\xgeq

% ``transparent'' space factor of the French closing quote:
\sfcode`\>=\z@% 0
% but NOT: \sfcode"\P@lqq>>=\z@% (following PLAIN TeX conventions)

% Emulate the original plain and LaTeX formats using PL fonts:
\def\emulateplain
    {%
    \exhyphenpenalty50
    \let\=\macron
    \let\slash\normalslash
    \language\english
    \nonfrenchspacing
    \nonprefixing
    }

\message{defaults,}
% (see the configuration file `MEXCONF.TEX'):

\d@faultspacing
\layout{\d@faultlayout} % defines also the meaning of `\polish'
\language\d@faultlanguage
\d@faultprefixing

\endinput

