% The author of this work is Amit Raj Dhawan
% This work has been released under
% Creative Commons Attribution-Share Alike 3.0 Unported License
% on July 19, 2010. For details visit:
% http://creativecommons.org/licenses/by-sa/3.0/.
%
%%%%%%%%%   Fonts   %%%%%%%%%%
\font\webomints=WebOMintsGd at40pt
\font\titlefont=artemisiarg8a at40pt
\font\titleone=mdputr7t at18pt
\font\titletwo=mdputri7t at18pt
\font\sectionfont=rm-kurierh at18pt
\font\subsectionfont=rm-kurierb at14pt
\font\subsubsectionfont=rm-kurierb at12pt
\font\sansrm=rm-kurierl at10pt
\font\foliofont=mdbchr7t at9pt
\font\amstexfont=cmsy10 at7.5pt
\font\dev=dvng10 scaled \magstep1
\font\letterone=RoyalIn at36pt


%%%%%%%%%%   Packages   %%%%%%%%%%
%%% Eplain
\input eplain % Add Eplain before AmSTeX
\beginpackages
\usepackage{url}
\usepackage{color}
\usepackage{graphicx}
\endpackages
\enablehyperlinks[dvipdfm] % Enables hyperlinks using Eplain
%\hlopts{colormodel=named,color=Black} % Produces links in black color
%% Color
\definecolor{brown}{rgb}{.7,.2,.2}


%%% AmSTeX
\input amstex
\UseAMSsymbols

\catcode`@=12

%%% eps files
\input epsf  % Includes files like pictures and figures in eps format




%%%%%%%%%%  Page characteristics  %%%%%%%%%%
\magnification=1120
\parindent=20pt
\parfillskip=\parindent plus1fil
\everypar{\looseness=-1}
\headline{} \footline{}
\vsize=24truecm
\hoffset-2.9mm
\settabs 20 \columns
\exhyphenpenalty10000 % stops TeX to break words at dashes
\hyphenpenalty200



%%%%%%%%%%   Definitions   %%%%%%%%%%
\def\bs{\bigskip}%
\def\ms{\medskip}%
\def\sk{\smallskip}%
\def\cl{\centerline}%
\def\ii{\noindent}%
\def\pic#1#2#3{\bigskip\cl{\epsfxsize#1\epsfbox{#2.eps}}\par\cl{\eightrm #3}}%
\def\amstex{{\amstexfont  A\kern-.1667em\lower.5ex\hbox{M}\kern-.125em S}-\capstex}%
\def\xetex{X\lower0.5ex\hbox{\kern-0.11em\reflectbox{E}}\kern-0.165em\TeX}%
\def\latex{L\setbox0=\hbox{\sc A}\kern-0.5766\wd0\raise0.41ex\hbox{\sc A}\setbox1=\hbox{T}\kern-.177\wd1\TeX}%
\def\xelatex{X\lower0.5ex\hbox{\kern-0.11em\reflectbox{E}}\kern-0.13em\latex}%
\def\capstex{{\caps t\kern-.122em\lower.38ex\hbox{e}\kern-.11em x}}%
\def\capslatex{{\caps l\setbox0=\hbox{\sevencaps a}\kern-0.5888\wd0{\raise0.33ex\hbox{\sevencaps a}}\setbox1=\hbox{t}\kern-.2\wd1\capstex}}%
\def\capsxetex{{\caps x\lower0.38ex\hbox{\kern-0.1em\reflectbox{e}}\kern-0.13em\capstex}}%
\def\capsxelatex{{\caps x\lower0.38ex\hbox{\kern-0.1em\reflectbox{e}}\kern-0.122em\capslatex}}%
\def\fontss{\baselineskip=2.7ex plus0pt minus0pt \spaceskip=0.28em plus0.08em minus0.08em}%
\def\footnote#1{\numberedfootnote{\hskip-7mm\hbox to 20cm{\vtop{\hangindent\parindent\hangafter1\eightrm\fontss #1}}}}%
\def\quote#1{\sk\leftskip10mm{\sl\noindent #1}\sk \leftskip0mm\rightskip0mm}%
\def\emdash{\hbox{\kern0.15em---\kern0.15em}\relax}%
\def\endash{\hbox{\kern0.15em--\kern0.15em}\relax}%


\def\sample{\hrule\vbox{\noindent\vrule\NoBlackBoxes\vbox{\vskip2mm\leftskip7mm\rightskip7mm
\noindent{\bf Euler Formula}: The Euler formula, also known as {\bf Euler identity}, states
$$e^{\iota x} =\cos(x) + \iota \sin(x), $$
where $\iota$~is the {\sl imaginary unit}.

The Euler formula can be expanded as a series:
$$\eqalign {e^{\iota x}
&= \sum_{n=0}^{\infty} {(\iota x)^n\over{n!}}\cr
&= \sum_{n=0}^{\infty}{(-1)^{n}x^{2n}\over (2n)!} + \iota\sum_1^{\infty}{(-1)^{n-1}x^{2n-1}\over(2n-1)!}\cr
&= \cos(x) + \iota\sin(x).\cr}$$

\bigskip\bigskip
\noindent{\bf Cauchy Integral Theorem}: If $f(z)$ is analytic and its partial derivatives are continuous throughout some simply connected region~$R$,~then
$$\oint_\gamma f(z)\,dz  = 0$$
for any closed contour~$\gamma$ completely contained in~$R$.\vskip2mm
}\vrule}\hrule\BlackBoxes\bigskip\bigskip
%\input mathcharacters
}


%%%%%%%%%%   Chapter, Section   %%%%%%%%%%
\newcount\sectionno\sectionno=0 % For sections
\newcount\subsectionno\subsectionno=0 % For subsections
\newcount\subsubsectionno\subsubsectionno=0 % For subsubsections
\definecolor{sectioncolor}{rgb}{0.22,0.38,0.62}
% NOTE: No two cross-references should be the same. This will cause chaos. In other words, no two chapters can the same "label name"

\def\section#1#2{\newpage
{\definexref{#2}{#2}{}}
{\writetocentry{section}{\refs{#2}}}
{\special{pdf: outline 1 << /Title (#2)/F 0 /Dest [@thispage /FitH @ypos ]  >> }}
\subsectionno=0 % For subsections
\centerline{\textcolor{sectioncolor}{\sectionfont\fontss #1}}
\nopagebreak\bigskip\nopagebreak\noindent}

\def\subsection#1{
{\definexref{#1}{#1}{}}
{\writetocentry{subsection}{\kern5mm\refs{#1}}}
{\special{pdf: outline 2 << /Title (#1)/F 0 /Dest [@thispage /FitH @ypos ]  >> }}
\bigskip\bigskip\medskip\goodbreak
{\global\advance\subsectionno by 1
 \noindent{\textcolor{sectioncolor}{\subsectionfont\fontss #1}}}
\nopagebreak\medskip\nopagebreak\noindent}

\def\subsubsection#1{
{\definexref{#1}{#1}{}}
{\writetocentry{subsubsection}{\kern5mm\refs{#1}}}
{\special{pdf: outline 2 << /Title (#1)/F 0 /Dest [@thispage /FitH @ypos ]  >> }}
\bigskip\bigskip\medskip\goodbreak
{\global\advance\subsubsectionno by 1
 \noindent{\textcolor{sectioncolor}{\subsubsectionfont\fontss #1}}}
\nopagebreak\medskip\nopagebreak\noindent}

















\input font_charter
\fontss % font change and spacing
\frenchspacing




















%%%%%%%%%%  Cover   %%%%%%%%%%
{\special{pdf: outline 1 << /Title (Cover)/F 0 /Dest [@thispage /FitH @ypos ]  >> }}
{\centerline{{\color{brown}\webomints\char'160}\hskip8mm\titlefont\color{sectioncolor} font-change\hskip8mm{\color{brown}\webomints\char'161}}\bs
\centerline{{\color{brown}\webomints\char'125\char'126}}
\bs
\centerline{Version \ 2010.1}
\bs\bs
{{\color{sectioncolor}\titleone\fontss\centerline{Macros to Change Text \& Math fonts in \TeX}\kern3mm
\centerline{{\titleone 45}{\titletwo\fontss \ Beautiful Variants}}}
\vskip2cm
\centerline{\color{brown}\webomints \char'063}
\vskip2cm
\centerline{\twelvebf\fontss Amit Raj Dhawan}\sk
\centerline{\href{mailto:amitrajdhawan@gmail.com}{\sansrm amitrajdhawan\@gmail.com}}\sk
\centerline{\rm July 19, 2010}

\vskip6cm

% Licence
\hrule\kern1pt\hrule\sk
\ii{\epsfxsize2cm\epsfbox{by-sa.eps}}
\vskip-7mm
\vbox{\leftskip3cm\parindent=0pt\eightrm\fontss This work has been released under  \href{http://creativecommons.org/licenses/by-sa/3.0/}{\eightbf Creative Commons Attribution-Share Alike 3.0 Unported License} on July 19, 2010.

You are free to {\eightitbf Share\/} (to copy, distribute and transmit the work) and to {\eightitbf Remix\/} (to adapt the work) provided you follow the {\eightitbf Attribution\/} and {\eightitbf Share Alike\/} guidelines of the licence. For the full licence text, please visit:
\href{http://creativecommons.org/licenses/by-sa/3.0/legalcode}{\eightrm http://creativecommons.org/licenses/by-sa/3.0/legalcode}.}\leftskip0cm
\sk\hrule\kern1pt\hrule

\BlackBoxes






%%%%%%%%%    Quote    %%%%%%%%%%%%
{\input font_artemisia_euler

\sixteenrm \fontss
\

\vskip1cm
\cl{\webomints\char'064}
\vskip2cm

\leftskip1cm \rightskip4cm

\raggedright
\ii When I reach the destination, more than I realize that I have realized the goal, I~am occupied with the reminiscences of the journey. It strikes to me again and again, ``Isn't the journey to the goal the\break real attainment of the goal?'' In this way even if I miss {\sixteencaps the} goal, I still have attained {\sixteencaps some} goal.


}
\leftskip0cm \rightskip0cm







%%%%%%%%%%   Contents   %%%%%%%%%%
\newpage\pageno=-3\footline{\centerline{\foliofont\folio}}
{\special{pdf: outline 1 << /Title (Contents)/F 0 /Dest [@thispage /FitH @ypos ]  >> }}
\cl{\textcolor{sectioncolor}{\sectionfont Contents}}


{\parskip0pt\bs\bs
\readtocfile}



























%%%%%%%%%%   Introduction   %%%%%%%%%%
\section{Introduction}{Introduction}
\pageno=1


\hbox{\letterone \TeX\ }\vskip-14.5mm\ii \hangindent2.6cm\hangafter-3 typesets documents in Computer Modern fonts by default.\footnote{Typographically, the correct expression is, ``\capstex\ typesets documents in Computer Modern typefaces by default.'' But most people (including me) use the words {\eightit font\/} and {\eightit typeface\/} synonymously. In this manual such distinction has been avoided.} Knuth's Computer Modern fonts are very elegant but sometimes we all look for a change. Many of us want to typeset \capstex\ documents in fonts other than Computer Modern. At the user level, changing the font in \capstex's {\sl text mode}, i.e.\ the text font, is simple and there are many free fonts available with various typefaces like {\rm roman}, {\bf bold}, {\it italic}, {\sl slanted}, {\itbf italic bold}, {\slbf slanted bold}, {\caps Caps}, {\capsbf Bold Caps}, etc. The difficulty lies in changing the math fonts in \capstex\ documents. This is mainly due to the lack of math fonts for \capstex. Another reason is that switching the font in {\sl math mode\/} is not as simple as switching the font in {\sl text mode}. For \capslatex\ there are various packages that can be used to change the font\emdash text and math\emdash with one statement. But for \capstex, I could not find an easy way to change the font in the document\emdash text and math. Using one font in {\sl text mode\/} and another in {\sl math mode\/} can spoil the look of the document. It is always desired to have text and math in the same font; text in New Century and math in Computer Modern do not go well. Though there are some combinations, as we will see later, that go well.

Being able to choose from different fonts is quite advantageous. Computer Modern fonts look very good on paper, esp.\ on inkjet printouts, but they look relatively thin on new computer screens (LCDs) and on laser printouts. For slide shows, most people prefer sans-serif fonts of relatively heavier weight. The idea of changing the entire font family which includes various typefaces like boldface, italics, etc., and the math fonts, with one control statement has been the motivation behind my work. For this purpose I have written 45~\capstex\ macros that instruct \capstex\ to typeset documents in the fonts called by those macros. In this document, the use of the above mentioned 45~font macros has been displayed. Each of these macros changes the fonts in the document globally, and can be used locally too, i.e.\ within a group. Now a \capstex\ document, which is normally produced in Computer Modern, can be produced in 45 other font variants. These macro files can be easily understood, and changed if convenient. Each macro has various typefaces declared at 5, 6, 7, 8, 9, 10, 12, 14, 16, 18, and 20\,pt~sizes. To save \capstex's memory we can delete some of the sizes and typefaces we do not use normally.

To display our 45~font changing macros in action, a sample text has been typeset 45 times but in different fonts. The fonts/font~families called by our macros have almost all the glyphs contained in the Computer Modern family. In general, these fonts have more glyphs than Computer Modern. To see all the glyphs in a font, please use Werner Lemberg's \href{http://www.ctan.org/tex-archive/help/Catalogue/entries/fontchart.html}{fontchart}~utility. In a few cases, e.g., in Epigrafica normal font~(\verbatim epigrafican8r|endverbatim), some important glyphs like $\Gamma$ and~$\Theta$ are missing. Our macro takes care of this; the user need not bother unless something very unusual is demanded from \capstex. These minor issues arise with \capslatex\ packages too.



\subsection{Usage}These macros have been bundled as a package called  {\color{brown}\verbatim font-change|endverbatim} which is included in {\caps m{\eightrm i}k}\capstex\ and \capstex~{\caps l{\eightrm ive}} distributions. The package can also be downloaded from \href{http://www.ctan.org/tex-archive/macros/plain/contrib/font-change/}{\caps ctan}. If our \capstex\ installation has the package {\color{brown}\verbatim font-change|endverbatim} installed then we can readily use it, e.g., to typeset our document in Charter, we have to type {\color{brown}\verbatim \input font_charter|endverbatim} in our source file. In case we do not have {\color{brown}\verbatim font-change|endverbatim} installed on our \capstex\ system and we are lazy to do that, then we can download the package from the internet and follow the following procedure. Please read the following to know about the available options and to see the macros in~effect.

Suppose we would like to typeset our \capstex\ document in Charter font. To do this we have to copy the \capstex\ macro file {\color{brown}\verbatim font_charter.tex|endverbatim} to the directory~(folder) which contains our \capstex\ source file. In our \capstex\ source file, we have to type {\color{brown}\verbatim \input font_charter|endverbatim}. This will change the font to Charter from the point where the statement {\color{brown}\verbatim \input font_charter|endverbatim} was declared. We can declare {\color{brown}\verbatim \input font_charter|endverbatim} in a closed group ({\color{brown}\verbatim {\input font_charter ... }|endverbatim}) to change the font to Charter in that group, provided no other font change is called in that group or its~sub-group.

Another way to use the font changing macro files is to put them in a folder (say ``font-change") in some drive~(say ``C'') and then call these files in our \capstex\ source file. If we want to use the Charter font, we should type {\color{brown}\verbatim \input C:/font-change/font_charter|endverbatim} to get the desired change. If we have put the font changing macro files in a folder that has space(s) in its name~(say ``font change''), then we should type {\color{brown}\verbatim \input "C:/font change/font_charter"|endverbatim} to use the Charter font.

The complete change of font will be at the default size in \capstex\ (10\,pt), though a little manipulation with the macro file will enable us to use the text and math fonts at smaller and larger point~changes.

\goodbreak The basic typeface changing \capstex\ control statements\sk
{\obeylines\leftskip1cm
{\color{brown}\verbatim\rm|endverbatim} \dots  {\rm roman}
{\color{brown}\verbatim\it|endverbatim} \dots  {\it italic}
{\color{brown}\verbatim\bf|endverbatim} \dots  {\bf boldface}
{\color{brown}\verbatim\sl|endverbatim} \dots  {\sl slanted}
{\color{brown}\verbatim\tt|endverbatim} \dots  {\tt typewriter}
\leftskip0cm}\sk

\ii hold their usual meaning. All the macro files that this {\caps pdf} mentions have the above mentioned five options. In addition, most macro files have other useful options too. These~are:\sk
{\obeylines\leftskip1cm
{\color{brown}\verbatim\itbf|endverbatim} \dots  {\itbf italic boldface}
{\color{brown}\verbatim\slbf|endverbatim} \dots  {\slbf slanted boldface}
{\color{brown}\verbatim\caps|endverbatim} \dots  {\caps Caps}
{\color{brown}\verbatim\capsbf|endverbatim} \dots  {\capsbf Caps in Boldface}
\leftskip0cm}\ms

In the {\sl text mode}, the above mentioned typefaces can be used at 5, 6, 7, 8, 9, 10, 12, 14, 16, 18, and 20\,pt~sizes. This is done by typing the size in words between the backslash~($\backslash$) and the words that declare the typeface. For example, if we want to typeset some text in bold at~14\,pt then we have to use the control statement {\color{brown}\verbatim \fourteenbf|endverbatim}.





\newpage\subsection{Example}A sample \capstex\ source file as shown below:

\bigskip\hrule\vbox{\noindent\vrule\NoBlackBoxes\vbox{\vskip2mm\leftskip7mm\rightskip7mm
{\obeylines\parindent=0pt\color{brown}\verbatim
\parindent=0pt
\input C:/font-change/font_cm
This is the {\bf Computer Modern font}. The {\twelveslbf Gamma function\/}
is defined as:
$$\Gamma(z) \equiv \int_0^\infty t^{z-1} e^{-t} dt.$$

\input C:/font-change/font_charter
This is the {\bf Charter font}. The {\twelveslbf Gamma function\/}
is defined as:
$$\Gamma(z) \equiv \int_0^\infty t^{z-1} e^{-t} dt.$$

{ % begin group
\input C:/font-change/font_century
This is the {\bf New Century Schoolbook font}. The {\twelveslbf Gamma
function\/} is defined as:
$$\Gamma(z) \equiv \int_0^\infty t^{z-1} e^{-t} dt.$$
} % end group

Now we are back to Charter.|endverbatim}
\vskip2mm}\vrule}\hrule\BlackBoxes\bigskip\bigskip




\nopagebreak\ii after compilation will produce:\nopagebreak




\bigskip\bigskip\hrule\vbox{\noindent\vrule\NoBlackBoxes\vbox{\vskip2mm\leftskip7mm\rightskip7mm
{\parindent=0pt
\input font_cm \fontss
This is the {\bf Computer Modern font}. The {\twelveslbf Gamma function\/} is defined~as:
$$\Gamma(z) \equiv \int_0^\infty t^{z-1} e^{-t} dt.$$

\input font_charter \fontss
This is the {\bf Charter font}. The {\twelveslbf Gamma function\/} is defined as:
$$\Gamma(z) \equiv \int_0^\infty t^{z-1} e^{-t} dt.$$

{\input font_century \fontss
This is the {\bf New Century font}. The {\twelveslbf Gamma function\/} is defined as:
$$\Gamma(z) \equiv \int_0^\infty t^{z-1} e^{-t} dt.$$}

Now we are back to Charter.}
\vskip2mm}\vrule}\hrule\BlackBoxes\bigskip\bigskip


\subsection{AMS Symbols}Some fonts, e.g., Kp-Fonts, have support for {\caps ams} symbols. Fonts {\verbatim msam|endverbatim} and {\verbatim msbm|endverbatim} of the {\caps ams} font collection contain these symbols. Blackboard letters~($\Bbb A, \Bbb B, \Bbb C, \Bbb R, \dots$) are a part of {\caps ams} symbols. If we are using \amstex, and we are using the preprint style or we have already declared {\color{brown}\verbatim \UseAMSsymbols|endverbatim} (default \amstex\ command), then we can use {\caps ams} symbols with some of the macros of {\color{brown}\verbatim font-change|endverbatim} by declaring {\color{brown}\verbatim \UseAMSsymbols|endverbatim} {\bf again} after calling the macro. In a while we will look at an example of this implementation.

If we have used instructions {\color{brown}\verbatim \loadmsam|endverbatim} or {\color{brown}\verbatim \loadmsbm|endverbatim} of \amstex, we can use the statements {\bf again} after declaring the {\color{brown}\verbatim font-change|endverbatim} macro to obtain the desired results. The control sequence {\color{brown}\verbatim \UseAMSsymbols|endverbatim} subsumes the instructions {\color{brown}\verbatim \loadmsam|endverbatim} and {\color{brown}\verbatim \loadmsbm|endverbatim}.

If we would like to return to the default {\caps ams} fonts\emdash {\verbatim msam|endverbatim} and {\verbatim msbm|endverbatim}\emdash we will have to input the macro file {\color{brown}\verbatim default-amssymbols.tex|endverbatim} by instructing {\color{brown}\verbatim \input default-amssymbols|endverbatim} in our source file. This small file has just the following two defintions:


\bigskip\hrule\vbox{\noindent\vrule\NoBlackBoxes\vbox{\vskip2mm\leftskip7mm\rightskip7mm
{\obeylines\parindent=0pt\color{brown}\verbatim
\def\loadmsam{\font\tenmsa=msam10 \font\sevenmsa=msam7 \font\fivemsa=msam5
\fam\msafam
\textfont\msafam=\tenmsa \scriptfont\msafam=\sevenmsa
\scriptscriptfont\msafam=\fivemsa \global\let\loadmsam\empty}%
\loadmsam
%
\def\loadmsbm{\font\tenmsb=msbm10 \font\sevenmsb=msbm7 \font\fivemsb=msbm5
\fam\msbfam
\textfont\msbfam=\tenmsb \scriptfont\msbfam=\sevenmsb
\scriptscriptfont\msbfam=\fivemsb \global\let\loadmsbm\empty}%
\loadmsbm
|endverbatim}
\vskip2mm}\vrule}\hrule\BlackBoxes\bigskip

It will be mentioned further if a macro of package {\color{brown}\verbatim font-change|endverbatim} offers {\caps ams} symbols support. The following shows the discussed in action (the character in {\color{red}red} color is from {\caps ams} symbols):

\bigskip\hrule\vbox{\noindent\vrule\NoBlackBoxes\vbox{\vskip2mm\leftskip7mm\rightskip7mm
{\obeylines\parindent=0pt\color{brown}\verbatim
\input amstex % Input AmSTeX
\UseAMSsymbols % Calls AMS symbols
$$f:{\color{red}\Bbb R}^3\to R$$

\input font_kp % Call Kp-Fonts
\UseAMSsymbols % Uses jkpsya and jkpsyb of Kp-Fonts instead of msam and msbm of AMS fonts
$$f:{\color{red}\Bbb R}\to R$$

\input default-amssymbols % Return to default
$$f:{\color{red}\Bbb R}^3\to R$$
|endverbatim}
\vskip2mm}\vrule}\hrule\BlackBoxes\bigskip

\nopagebreak\ii upon compilation produces:\nopagebreak

\bigskip\hrule\vbox{\noindent\vrule\NoBlackBoxes\vbox{\vskip2mm\leftskip7mm\rightskip7mm
{\input C:/Users/PARTH/Documents/Vividh/Filhaal/TeX/font-change/font_cm
$$f:{\color{red}\Bbb R}^3\to R$$
\input C:/Users/PARTH/Documents/Vividh/Filhaal/TeX/font-change/font_kp
\UseAMSsymbols
$$f:{\color{red}\Bbb R}^3\to R$$
\input C:/Users/PARTH/Documents/Vividh/Filhaal/TeX/font-change/default-amssymbols
$$f:{\color{red}\Bbb R}^3\to R$$}
\vskip2mm}\vrule}\hrule\BlackBoxes\bigskip

\subsection{Available Weights}
Some font changing macros of the package {\color{brown}\verbatim font-change|endverbatim} offer light, medium, and bold weights. There are many font families that offer the bold weight variant of the math fonts, but we have not all included such variants as they do not supply a heavier font to produce the contrast. If we type all text in boldface then at places where we would like to get bolder we are be left without an option. The philosophy of {\color{brown}\verbatim font-change|endverbatim} says that to use bold for all text and math we need a heavier face available within the type family. Font families Kp-Fonts, Antykwa Toru\'nska, Iwona, and Kurier include such weights and they have been included in {\color{brown}\verbatim font-change|endverbatim}. For instance, macro {\color{brown}\verbatim font_kurier-bold|endverbatim}, which uses boldface as the normal font~(in math and text), uses the heavy weight font as the~boldface.


\subsection{Warning}The fonts used in these 45 macros are included in \href{http://miktex.org/}{{\caps m{\eightrm i}k}\capstex} and \href{http://www.tug.org/texlive/}{\capstex~{\caps l{\eightrm ive}}} distributions. All these macros should work smoothly with a full installation of \href{http://miktex.org/}{{\caps m{\eightrm i}k}\capstex} (Versions 2.7 and~2.8 tested). The macros should work smoothly with \capstex~{\caps l{\eightrm ive}}~too.

These 45~font changing macros have worked successfully with plain~\capstex, and a combination of plain~\capstex\ and other macros designed for plain~\capstex, e.g.,  \href{http://www.ctan.org/tex-archive/help/Catalogue/entries/amstex.html}{\amstex} and~\href{http://www.ctan.org/tex-archive/help/Catalogue/entries/eplain.html}{eplain}. The macros work smoothly with \href{http://www.ctan.org/tex-archive/help/Catalogue/entries/pdftex.html}{pdf\capstex} and \href{http://scripts.sil.org/cms/scripts/page.php?site_id=nrsi&id=xetex}{\capsxetex} too. Please note that these macros do not work with \href{http://www.ctan.org/tex-archive/help/Catalogue/entries/latex.html}{\capslatex}, pdf\capslatex\hfuzz2pt, or \capsxelatex.

\hfuzz1pt
If we are typesetting our document in English with any mathematics, then using these macros would be trouble free. They might demur when we try to type letters like \l, esp.\ when using typefaces like {\slbf slanted boldface\/} or {\caps Caps}. These are issues of missing glyphs and encoding. In the current typeface (Charter, regular roman, {\verbatim mdbchr7t|endverbatim}), {\color{brown}\verbatim\l|endverbatim} produces \l, {\color{brown}\verbatim\slbf \l|endverbatim} produces {\slbf \l}, but {\color{brown}\verbatim\caps\l|endverbatim} produces~{\caps \l}.

Sans-serif fonts do not have {\it italics}\emdash they only have {\sl slanted\/} glyphs. To make the font changing macro files more consistent, both italics and slanted commands, e.g., {\color{brown}\verbatim\it|endverbatim} and {\color{brown}\verbatim\sl|endverbatim}, produce {\sl slanted} typefaces in case of sans-serif fonts and in those fonts that do not have distinct italic and slanted glyphs. Displayed further are samples exhibiting the change of \capstex's text and math fonts using macros of {\color{brown}\verbatim font-change|endverbatim}. All the fonts used in any macro of {\color{brown}\verbatim font-change|endverbatim} are also listed in this~document.

It is hoped that these macros work well and do not raise compatibility issues but it can not be promised. There is no warranty. If the user find any bugs, or has suggestions or complaints, please email them to~me.


















%%%%%%%%%%%%%%%%%%%%%%%%%%%%%%%%
%%%%%%%%%%   Macros   %%%%%%%%%%
%%%%%%%%%%%%%%%%%%%%%%%%%%%%%%%


\section{\sixteenbf\fontss Charter}{Charter}
\sample
\ii The Charter font is declared by typing {\color{brown}\verbatim\input font_charter|endverbatim}. The font family uses fonts from the \href{http://www.ctan.org/tex-archive/help/Catalogue/entries/mathdesign-charter.html}{mdbch} family, which corresponds to \href{http://www.ctan.org/tex-archive/help/Catalogue/entries/charter.html}{Bitstream Charter} text fonts. This family is a part of Paul Pichaureau's \href{http://www.ctan.org/tex-archive/help/Catalogue/entries/mathdesign.html}{MathDesign} project. The \href{http://new.myfonts.com/fonts/bitstream/charter-bt-pro/}{Charter font} was originally designed by Matthew Carter for Bitstream Inc.\ in 1987. Details of this \capstex\ macro are given in the table~below.
\bs
\hfil{Font assignment in {\color{brown}\verbatim font_charter|endverbatim}~macro}\hfil

{\parindent=0pt\settabs4\columns\hfil\vbox{\hrule\hbox{\vrule\hbox{\vbox{\kern1pt\hrule\NoBlackBoxes		 \eightrm\fontss																 \+\hfil	 \textcolor{blue}{ Typeface}	&\strut\vrule\strut\hfil	 \textcolor{blue}{ Font name}	 &\strut\vrule\kern1pt\vrule\strut\hfil	 \textcolor{blue}{ Typeface}	 &\strut\vrule\strut\hfil	 \textcolor{blue}{ Font name}	 &\cr	\hrule
\+\hfil	\eightrm Roman text	&\strut\vrule\strut\hfil	mdbchr7t	 &\strut\vrule\kern1pt\vrule\strut\hfil	 \eightbf Boldface text	 &\strut\vrule\strut\hfil	 mdbchb7t	 &\cr	\hrule
\+\hfil	\eighti Math italic	&\strut\vrule\strut\hfil	mdbchri7m	 &\strut\vrule\kern1pt\vrule\strut\hfil	 \eighttt Typewriter text	 &\strut\vrule\strut\hfil	 rm-inconsolata	&\cr	\hrule
\+\hfil	\eightrm Math symbols	&\strut\vrule\strut\hfil	md-chr7y	 &\strut\vrule\kern1pt\vrule\strut\hfil	 \eightitbf Italic boldface text	 &\strut\vrule\strut\hfil	 mdbchbi7t	&\cr	\hrule
\+\hfil	\eightrm Math extension	&\strut\vrule\strut\hfil	mdbchr7v	 &\strut\vrule\kern1pt\vrule\strut\hfil	 \eightslbf Slanted boldface text	 &\strut\vrule\strut\hfil	 mdbchbo7t	&\cr	\hrule
\+\hfil	\eightit Italic text	&\strut\vrule\strut\hfil	mdbchri7t	 &\strut\vrule\kern1pt\vrule\strut\hfil	 \eightcaps Caps	 &\strut\vrule\strut\hfil	 mdbchrfc8t	&\cr	 \hrule
\+\hfil	\eightsl Slanted text	&\strut\vrule\strut\hfil	mdbchro7t	 &\strut\vrule\kern1pt\vrule\strut\hfil	 \eightcapsbf Caps in Boldface	 &\strut\vrule\strut\hfil	 mdbchbfc8t	&\cr	\hrule
}\vrule}}\hrule}\hfil}									

\BlackBoxes					










\input font_utopia   \fontss
\section{\sixteenbf\fontss Utopia}{Utopia}
\sample
\ii The Utopia font is declared by typing {\color{brown}\verbatim\input font_utopia|endverbatim}. The font family uses most of its fonts from the \href{http://www.ctan.org/tex-archive/help/Catalogue/entries/mathdesign-utopia.html}{mdput} family, which corresponds to \href{http://www.ctan.org/tex-archive/help/Catalogue/entries/utopia.html}{Adobe Utopia} text fonts. This family is a part of Paul Pichaureau's \href{http://www.ctan.org/tex-archive/help/Catalogue/entries/mathdesign.html}{MathDesign}~project. The font family is very complete and includes the math fonts too. For inter-letter spacing reasons, macro {\color{brown}\verbatim font_utopia.tex|endverbatim} uses math italic font and math symbols font from Michel Bovani's \href{http://www.ctan.org/tex-archive/fonts/fourier-GUT/}{fourier} package. The \href{http://new.myfonts.com/fonts/adobe/utopia/}{Utopia font} was originally designed by Robert Slimbach for Adobe in~1989.

Math italic~(\verbatim mdputri7m|endverbatim) and math symbols~(\verbatim md-utr7y|endverbatim) from the \href{http://www.ctan.org/tex-archive/help/Catalogue/entries/mathdesign-utopia.html}{mdput} family can also be used. Details of this \capstex\ macro are given in the table~below.
\bs
\hfil{Font assignment in {\color{brown}\verbatim font_utopia|endverbatim}~macro}\hfil

{\parindent=0pt\settabs4\columns\hfil\vbox{\hrule\hbox{\vrule\hbox{\vbox{\kern1pt\hrule\NoBlackBoxes		 \eightrm\fontss																 \+\hfil	 \textcolor{blue}{ Typeface}	&\strut\vrule\strut\hfil	 \textcolor{blue}{ Font name}	 &\strut\vrule\kern1pt\vrule\strut\hfil	 \textcolor{blue}{ Typeface}	 &\strut\vrule\strut\hfil	\textcolor{blue}{ Font name}	 &\cr	\hrule
\+\hfil	\eightrm Roman text	&\strut\vrule\strut\hfil	mdputr7t	 &\strut\vrule\kern1pt\vrule\strut\hfil	\eightbf Boldface text	 &\strut\vrule\strut\hfil	 mdputb7t	&\cr	\hrule
\+\hfil	\eighti Math italic	&\strut\vrule\strut\hfil	futmii	 &\strut\vrule\kern1pt\vrule\strut\hfil	\eighttt Typewriter text	 &\strut\vrule\strut\hfil	 rm-inconsolata	&\cr	\hrule
\+\hfil	\eightrm Math symbols	&\strut\vrule\strut\hfil	futsy	 &\strut\vrule\kern1pt\vrule\strut\hfil	\eightitbf Italic boldface text	 &\strut\vrule\strut\hfil	mdputbi7t	&\cr	\hrule
\+\hfil	\eightrm Math extension	&\strut\vrule\strut\hfil	mdputr7v	 &\strut\vrule\kern1pt\vrule\strut\hfil	 \eightslbf Slanted boldface text	 &\strut\vrule\strut\hfil	mdputbo7t	&\cr	\hrule
\+\hfil	\eightit Italic text	&\strut\vrule\strut\hfil	mdputri7t	 &\strut\vrule\kern1pt\vrule\strut\hfil	 \eightcaps Caps	 &\strut\vrule\strut\hfil	 mdputrfc8t	&\cr	\hrule
\+\hfil	\eightsl Slanted text	&\strut\vrule\strut\hfil	mdputro7t	 &\strut\vrule\kern1pt\vrule\strut\hfil	 \eightcapsbf Caps in Boldface	 &\strut\vrule\strut\hfil	mdputbfc8t	&\cr	\hrule
}\vrule}}\hrule}\hfil}								

\BlackBoxes								
























\input font_century   \fontss
\section{\sixteenbf\fontss New Century Schoolbook}{New Century Schoolbook}
\sample
\ii The New Century Schoolbook font is declared by typing {\color{brown}\verbatim\input font_century|endverbatim}. The font family uses fonts from the \href{http://www.ctan.org/tex-archive/help/Catalogue/entries/tex-gyre-schola.html}{TeX Gyre Schola} family, which corresponds to \href{http://store1.adobe.com/cfusion/store/html/index.cfm?store=OLS-US&event=displayFontPackage&code=1240}
{Adobe New Century Schoolbook} text~fonts. The \href{http://new.myfonts.com/fonts/adobe/new-century-schoolbook/}{Century Schoolbook} font was created by Morris Fuller Benton between 1918 and~1921.

The macro uses math italic~(fncmii) and math symbols~(fncsy) from Michael Zedler's \href{http://www.ctan.org/tex-archive/help/Catalogue/entries/fouriernc.html}{fouriernc} package. Details of this \capstex\ macro are given in the table~below.
\bs
\hfil{Font assignment in {\color{brown}\verbatim font_century|endverbatim}~macro}\hfil

{\parindent=0pt\settabs4\columns\hfil\vbox{\hrule\hbox{\vrule\hbox{\vbox{\kern1pt\hrule\NoBlackBoxes\eightrm\fontss
\+\hfil	\textcolor{blue}{ Typeface}	&\strut\vrule\strut\hfil	 \textcolor{blue}{ Font name}	 &\strut\vrule\kern1pt\vrule\strut\hfil	 \textcolor{blue}{ Typeface}	 &\strut\vrule\strut\hfil	\textcolor{blue}{ Font name}	 &\cr	\hrule
\+\hfil	\eightrm Roman text	&\strut\vrule\strut\hfil	rm-qcsr	 &\strut\vrule\kern1pt\vrule\strut\hfil	\eightbf Boldface text	 &\strut\vrule\strut\hfil	 rm-qcsb	&\cr	\hrule
\+\hfil	\eighti Math italic	&\strut\vrule\strut\hfil	fncmii	 &\strut\vrule\kern1pt\vrule\strut\hfil	\eighttt Typewriter text	 &\strut\vrule\strut\hfil	 cmtt10	&\cr	\hrule
\+\hfil	\eightrm Math symbols	&\strut\vrule\strut\hfil	fncsy	 &\strut\vrule\kern1pt\vrule\strut\hfil	\eightitbf Italic boldface text	 &\strut\vrule\strut\hfil	rm-qsbi	&\cr	\hrule
\+\hfil	\eightrm Math extension	&\strut\vrule\strut\hfil	cmex10	 &\strut\vrule\kern1pt\vrule\strut\hfil	\eightslbf Slanted boldface text	 &\strut\vrule\strut\hfil	pncbo7t	&\cr	\hrule
\+\hfil	\eightit Italic text	&\strut\vrule\strut\hfil	rm-qcsri	 &\strut\vrule\kern1pt\vrule\strut\hfil	 \eightcaps Caps	 &\strut\vrule\strut\hfil	 rm-qcsr-sc	&\cr	\hrule
\+\hfil	\eightsl Slanted text	&\strut\vrule\strut\hfil	pncro7t	 &\strut\vrule\kern1pt\vrule\strut\hfil	 \eightcapsbf Caps in Boldface	 &\strut\vrule\strut\hfil	rm-qcsb-sc	&\cr	\hrule
}\vrule}}\hrule}\hfil}								
					
\BlackBoxes								









\input font_palatino   \fontss
\UseAMSsymbols
\section{\sixteenbf\fontss Palatino}{Palatino}
\sample
\ii The Palatino font is declared by typing {\color{brown}\verbatim\input font_palatino|endverbatim}. The font family uses fonts from Young Ryu's \href{http://www.ctan.org/tex-archive/help/Catalogue/entries/pxfonts.html}{pxfonts} package, which corresponds to \href{http://www.urwpp.de/cgi-bin1/dalcgi/source/schnellsuche.htd?searchchar=palladio}
{{\caps urw++} Palladio} text fonts designed by Herman Zapf. The {\caps urw++} Palladio font is based on the \href{http://new.myfonts.com/fonts/adobe/palatino/}{Palatino font} which was originally designed by Hermann Zapf for the Stempel foundry in 1950. The fonts of this macro provide their own {\caps ams} symbols. Details of this \capstex\ macro are given in the table~below.
\bs
\hfil{Font assignment in {\color{brown}\verbatim font_palatino|endverbatim}~macro}\hfil

{\parindent=0pt\settabs4\columns\hfil\vbox{\hrule\hbox{\vrule\hbox{\vbox{\kern1pt\hrule\NoBlackBoxes		 \eightrm\fontss \+\hfil	 \textcolor{blue}{ Typeface}	 &\strut\vrule\strut\hfil	 \textcolor{blue}{ Font name}	 &\strut\vrule\kern1pt\vrule\strut\hfil	 \textcolor{blue}{ Typeface}	 &\strut\vrule\strut\hfil	\textcolor{blue}{ Font name}	 &\cr	\hrule
\+\hfil	\eightrm Roman text	&\strut\vrule\strut\hfil	pxr	 &\strut\vrule\kern1pt\vrule\strut\hfil	\eightbf Boldface text	 &\strut\vrule\strut\hfil	 pxb	&\cr	\hrule
\+\hfil	\eighti Math italic	&\strut\vrule\strut\hfil	pxmi	 &\strut\vrule\kern1pt\vrule\strut\hfil	\eighttt Typewriter text	 &\strut\vrule\strut\hfil	 cmtt10	&\cr	\hrule
\+\hfil	\eightrm Math symbols	&\strut\vrule\strut\hfil	pxsy	 &\strut\vrule\kern1pt\vrule\strut\hfil	\eightitbf Italic boldface text	 &\strut\vrule\strut\hfil	pxbi	&\cr	\hrule
\+\hfil	\eightrm Math extension	&\strut\vrule\strut\hfil	pxex	 &\strut\vrule\kern1pt\vrule\strut\hfil	\eightslbf Slanted boldface text	 &\strut\vrule\strut\hfil	pxbsl	&\cr	\hrule
\+\hfil	\eightit Italic text	&\strut\vrule\strut\hfil	pxi	 &\strut\vrule\kern1pt\vrule\strut\hfil	\eightcaps Caps	 &\strut\vrule\strut\hfil	 pxsc	 &\cr	\hrule
\+\hfil	\eightsl Slanted text	&\strut\vrule\strut\hfil	pxsl	 &\strut\vrule\kern1pt\vrule\strut\hfil	 \eightcapsbf Caps in Boldface	 &\strut\vrule\strut\hfil	pxbsc	&\cr	\hrule
}\vrule}}\hrule}\hfil}	
							
\BlackBoxes					

\bs\ii Matching {\caps ams} symbols: \circledR \ \yen \ $\blacksquare \ \approxeq \ \eqslantgtr \ \curlyeqprec \ \curlyeqsucc \ \preccurlyeq \ \leqq \ \leqslant \ \lessgtr \ \nless \ \nleq \ \nleqslant \ \Bbb R \ \Bbb E \ \Bbb C \ \dots$








\input font_pagella  \fontss
\UseAMSsymbols
\section{\sixteenbf\fontss Pagella}{Pagella}
\sample
\ii The Pagella font is declared by typing {\color{brown}\verbatim\input font_pagella|endverbatim}. Most of text is typeset using fonts from \href{http://www.gust.org.pl/projects/e-foundry/tex-gyre/pagella}{\capstex~Gyre Pagella} package and most math typesetting uses Diego Puga's \href{http://www.ctan.org/tex-archive/help/Catalogue/entries/mathpazo.html}{mathpazo} package, and some text~(slanted fonts) and some math~({\caps ams} symbols) is from Young Ryu's \href{http://www.ctan.org/tex-archive/help/Catalogue/entries/pxfonts.html}{pxfonts}\emdash all of these correspond to \href{http://www.urwpp.de/cgi-bin1/dalcgi/source/schnellsuche.htd?searchchar=palladio}
{{\caps urw++} Palladio} text fonts designed by Herman Zapf. The {\caps urw++} Palladio font is based on the \href{http://new.myfonts.com/fonts/adobe/palatino/}{Palatino font} which was originally designed by Hermann Zapf for the Stempel foundry in 1950. The \capstex~Gyre \href{http://www.gust.org.pl/projects/e-foundry/tex-gyre/pagella}{Pagella} fonts can be said to be a bit more refined version of the Palatino fonts and they also have the ff ligature, which is missing in \href{http://www.ctan.org/tex-archive/help/Catalogue/entries/pxfonts.html}{pxfonts} or other Palatino-based fonts. The fonts of this macro provide their own {\caps ams} symbols. Details of this \capstex\ macro are given in the table~below.
\bs
\hfil{Font assignment in {\color{brown}\verbatim font_pagella|endverbatim}~macro}\hfil

{\parindent=0pt\settabs4\columns\hfil\vbox{\hrule\hbox{\vrule\hbox{\vbox{\kern1pt\hrule\NoBlackBoxes		 \eightrm\fontss							 
									
\+\hfil	\textcolor{blue}{ Typeface}	&\strut\vrule\strut\hfil	\textcolor{blue}{ Font name}	 &\strut\vrule\kern1pt\vrule\strut\hfil	 \textcolor{blue}{ Typeface}	 &\strut\vrule\strut\hfil	\textcolor{blue}{ Font name}	&\cr	\hrule
\+\hfil	\eightrm Roman text	&\strut\vrule\strut\hfil	rm-qplr	 &\strut\vrule\kern1pt\vrule\strut\hfil	\eightbf Boldface text	 &\strut\vrule\strut\hfil	 rm-qplb	&\cr	\hrule
\+\hfil	\eighti Math italic	&\strut\vrule\strut\hfil	zplmr7m	 &\strut\vrule\kern1pt\vrule\strut\hfil	\eighttt Typewriter text	 &\strut\vrule\strut\hfil	 cmtt10	&\cr	\hrule
\+\hfil	\eightrm Math symbols	&\strut\vrule\strut\hfil	zplmr7y	 &\strut\vrule\kern1pt\vrule\strut\hfil	\eightitbf Italic boldface text	 &\strut\vrule\strut\hfil	rm-qplbi	&\cr	\hrule
\+\hfil	\eightrm Math extension	&\strut\vrule\strut\hfil	zplmr7v	 &\strut\vrule\kern1pt\vrule\strut\hfil	\eightslbf Slanted boldface text	 &\strut\vrule\strut\hfil	pxbsl	&\cr	\hrule
\+\hfil	\eightit Italic text	&\strut\vrule\strut\hfil	rm-qplri	 &\strut\vrule\kern1pt\vrule\strut\hfil	\eightcaps Caps	 &\strut\vrule\strut\hfil	rm-qplr-sc	 &\cr	\hrule
\+\hfil	\eightsl Slanted text	&\strut\vrule\strut\hfil	pxsl	 &\strut\vrule\kern1pt\vrule\strut\hfil	\eightcapsbf Caps in Boldface	 &\strut\vrule\strut\hfil	rm-qplb-sc	&\cr	\hrule
									
	}\vrule}}\hrule}\hfil}								
									
	\BlackBoxes								
			

\bs\ii Matching {\caps ams} symbols: \circledR \ \yen \ $\blacksquare \ \approxeq \ \eqslantgtr \ \curlyeqprec \ \curlyeqsucc \ \preccurlyeq \ \leqq \ \leqslant \ \lessgtr \ \nless \ \nleq \ \nleqslant \ \Bbb R \ \Bbb E \ \Bbb C \ \dots$















\input font_times \fontss
\UseAMSsymbols
\section{\sixteenbf\fontss Times}{Times}
\sample
\ii The Times font is declared by typing {\color{brown}\verbatim\input font_times|endverbatim}. The font family uses fonts from Young Ryu's \href{http://www.ctan.org/tex-archive/help/Catalogue/entries/txfonts.html}{txfonts} package, which corresponds to \href{http://new.myfonts.com/fonts/adobe/times/}{Adobe Times} text fonts. The \href{http://new.myfonts.com/fonts/adobe/times/}{Times font} was designed in 1931 by Stanley Morison at Monotype Corp. The fonts of this macro provide their own {\caps ams} symbols. Details of this \capstex\ macro are given in the table~below.
\bs
\hfil{Font assignment in {\color{brown}\verbatim font_times|endverbatim}~macro}\hfil

{\parindent=0pt\settabs4\columns\hfil\vbox{\hrule\hbox{\vrule\hbox{\vbox{\kern1pt\hrule\NoBlackBoxes		 \eightrm\fontss																 \+\hfil	 \textcolor{blue}{ Typeface}	&\strut\vrule\strut\hfil	 \textcolor{blue}{ Font name}	 &\strut\vrule\kern1pt\vrule\strut\hfil	 \textcolor{blue}{ Typeface}	 &\strut\vrule\strut\hfil	\textcolor{blue}{ Font name}	 &\cr	\hrule
\+\hfil	\eightrm Roman text	&\strut\vrule\strut\hfil	txr	 &\strut\vrule\kern1pt\vrule\strut\hfil	\eightbf Boldface text	 &\strut\vrule\strut\hfil	 txb	&\cr	\hrule
\+\hfil	\eighti Math italic	&\strut\vrule\strut\hfil	txmi	 &\strut\vrule\kern1pt\vrule\strut\hfil	\eighttt Typewriter text	 &\strut\vrule\strut\hfil	 txtt	&\cr	\hrule
\+\hfil	\eightrm Math symbols	&\strut\vrule\strut\hfil	txsy	 &\strut\vrule\kern1pt\vrule\strut\hfil	\eightitbf Italic boldface text	 &\strut\vrule\strut\hfil	txbi	&\cr	\hrule
\+\hfil	\eightrm Math extension	&\strut\vrule\strut\hfil	txex	 &\strut\vrule\kern1pt\vrule\strut\hfil	\eightslbf Slanted boldface text	 &\strut\vrule\strut\hfil	txbsl	&\cr	\hrule
\+\hfil	\eightit Italic text	&\strut\vrule\strut\hfil	txi	 &\strut\vrule\kern1pt\vrule\strut\hfil	\eightcaps Caps	 &\strut\vrule\strut\hfil	 txsc	 &\cr	\hrule
\+\hfil	\eightsl Slanted text	&\strut\vrule\strut\hfil	txsl	 &\strut\vrule\kern1pt\vrule\strut\hfil	 \eightcapsbf Caps in Boldface	 &\strut\vrule\strut\hfil	txbsc	&\cr	\hrule
}\vrule}}\hrule}\hfil}								

\BlackBoxes								

\bs\ii Matching {\caps ams} symbols: \circledR \ \yen \ $\blacksquare \ \approxeq \ \eqslantgtr \ \curlyeqprec \ \curlyeqsucc \ \preccurlyeq \ \leqq \ \leqslant \ \lessgtr \ \nless \ \nleq \ \nleqslant \ \Bbb R \ \Bbb E \ \Bbb C \ \dots$








\input font_bookman   \fontss
\section{\sixteenbf\fontss Bookman Font}{Bookman Font}
\sample
\ii The Bookman font is declared by typing {\color{brown}\verbatim\input font_bookman|endverbatim}. The font family uses fonts from Jackowski and Nowacki's (\capstex\ Gyre) \href{http://www.ctan.org/tex-archive/help/Catalogue/entries/tex-gyre-bonum.html}{bonum} family, and Antonis Tsolomitis' \href{http://www.ctan.org/tex-archive/help/Catalogue/entries/kerkis.html}{kerkis} package; both these packages correspond to \href{http://new.myfonts.com/fonts/adobe/itc-bookman/}{ITC Bookman} text fonts. The math symbols and extension characters are taken from Young Ryu's \href{http://www.ctan.org/tex-archive/help/Catalogue/entries/txfonts.html}{txfonts} package. The \href{http://new.myfonts.com/fonts/adobe/itc-bookman/}{Bookman} font was originally designed by Alexander Phemister in 1860 for the Miller \& Richard foundry in Scotland. Details of this \capstex\ macro are given in the table~below.
\bs
\hfil{Font assignment in {\color{brown}\verbatim font_bookman|endverbatim}~macro}\hfil

{\parindent=0pt\settabs4\columns\hfil\vbox{\hrule\hbox{\vrule\hbox{\vbox{\kern1pt\hrule\NoBlackBoxes		 \eightrm\fontss							 \+\hfil	\textcolor{blue}{ Typeface}	 &\strut\vrule\strut\hfil	\textcolor{blue}{ Font name}	 &\strut\vrule\kern1pt\vrule\strut\hfil	 \textcolor{blue}{ Typeface}	 &\strut\vrule\strut\hfil	\textcolor{blue}{ Font name}	 &\cr	\hrule
\+\hfil	\eightrm Roman text	&\strut\vrule\strut\hfil	rm-qbkr	 &\strut\vrule\kern1pt\vrule\strut\hfil	\eightbf Boldface text	 &\strut\vrule\strut\hfil	 rm-qbkb	&\cr	\hrule
\+\hfil	\eighti Math italic	&\strut\vrule\strut\hfil	kmath8r	 &\strut\vrule\kern1pt\vrule\strut\hfil	\eighttt Typewriter text	 &\strut\vrule\strut\hfil	 txtt	&\cr	\hrule
\+\hfil	\eightrm Math symbols	&\strut\vrule\strut\hfil	txsy	 &\strut\vrule\kern1pt\vrule\strut\hfil	\eightitbf Italic boldface text	 &\strut\vrule\strut\hfil	rm-qbkbi	&\cr	\hrule
\+\hfil	\eightrm Math extension	&\strut\vrule\strut\hfil	txex	 &\strut\vrule\kern1pt\vrule\strut\hfil	\eightslbf Slanted boldface text	 &\strut\vrule\strut\hfil	pbkdo7t	&\cr	\hrule
\+\hfil	\eightit Italic text	&\strut\vrule\strut\hfil	rm-qbkri	 &\strut\vrule\kern1pt\vrule\strut\hfil	 \eightcaps Caps	 &\strut\vrule\strut\hfil	 rm-qbkr-sc	&\cr	\hrule
\+\hfil	\eightsl Slanted text	&\strut\vrule\strut\hfil	pbklo7t	 &\strut\vrule\kern1pt\vrule\strut\hfil	 \eightcapsbf Caps in Boldface	 &\strut\vrule\strut\hfil	rm-qbkb-sc	&\cr	\hrule
}\vrule}}\hrule}\hfil}								

\BlackBoxes								
						












\input font_kp   \fontss
\UseAMSsymbols
\section{\sixteenbf\fontss Kp-Fonts}{Kp-Fonts}
\sample
\ii Kp-Fonts are declared by typing {\color{brown}\verbatim\input font_kp|endverbatim}. The font family uses fonts from Chris\-tophe \hfuzz4pt Caignaert's \href{http://www.ctan.org/tex-archive/help/Catalogue/entries/kpfonts.html}{Kp-Fonts} family. The fonts of this macro provide their own {\caps ams} symbols. Details of this \capstex\ macro are given in the table~below.
\bs\hfuzz=1pt
\hfil{Font assignment in {\color{brown}\verbatim font_kp|endverbatim}~macro}\hfil
					
{\parindent=0pt\settabs4\columns\hfil\vbox{\hrule\hbox{\vrule\hbox{\vbox{\kern1pt\hrule\NoBlackBoxes		 \eightrm\fontss																 \+\hfil	 \textcolor{blue}{ Typeface}	&\strut\vrule\strut\hfil	\textcolor{blue}{ Font name}	 &\strut\vrule\kern1pt\vrule\strut\hfil	 \textcolor{blue}{ Typeface}	 &\strut\vrule\strut\hfil	\textcolor{blue}{ Font name}	 &\cr	\hrule
\+\hfil	\eightrm Roman text	&\strut\vrule\strut\hfil	jkpmn7t	 &\strut\vrule\kern1pt\vrule\strut\hfil	\eightbf Boldface text	 &\strut\vrule\strut\hfil	 jkpbn7t	&\cr	\hrule
\+\hfil	\eighti Math italic	&\strut\vrule\strut\hfil	jkpmi	 &\strut\vrule\kern1pt\vrule\strut\hfil	\eighttt Typewriter text	 &\strut\vrule\strut\hfil	 jkpttmn7t	&\cr	\hrule
\+\hfil	\eightrm Math symbols	&\strut\vrule\strut\hfil	jkpsy	 &\strut\vrule\kern1pt\vrule\strut\hfil	\eightitbf Italic boldface text	 &\strut\vrule\strut\hfil	jkpbit7t	&\cr	\hrule
\+\hfil	\eightrm Math extension	&\strut\vrule\strut\hfil	jkpex	 &\strut\vrule\kern1pt\vrule\strut\hfil	\eightslbf Slanted boldface text	 &\strut\vrule\strut\hfil	jkpbsl7t	&\cr	\hrule
\+\hfil	\eightit Italic text	&\strut\vrule\strut\hfil	jkpmit7t	 &\strut\vrule\kern1pt\vrule\strut\hfil	\eightcaps Caps	 &\strut\vrule\strut\hfil	jkpmsc7t	 &\cr	\hrule
\+\hfil	\eightsl Slanted text	&\strut\vrule\strut\hfil	jkpmsl7t	 &\strut\vrule\kern1pt\vrule\strut\hfil	\eightcapsbf Caps in Boldface	 &\strut\vrule\strut\hfil	jkpbsc7t	&\cr	\hrule
									
	}\vrule}}\hrule}\hfil}								
									
	\BlackBoxes								

\bs\ii Matching {\caps ams} symbols: \circledR \ \yen \ $\blacksquare \ \approxeq \ \eqslantgtr \ \curlyeqprec \ \curlyeqsucc \ \preccurlyeq \ \leqq \ \leqslant \ \lessgtr \ \nless \ \nleq \ \nleqslant \ \Bbb R \ \Bbb E \ \Bbb C \ \dots$






\input font_kp-light   \fontss
\UseAMSsymbols
\section{{\sixteenbf\fontss Kp}-{\sixteenslbf Light}}{Kp-Light}
\sample
\ii Kp-{\sl Light\/} fonts are declared by typing {\color{brown}\verbatim\input font_kp-light|endverbatim}. The font family uses fonts from Christophe Caignaert's \href{http://www.ctan.org/tex-archive/help/Catalogue/entries/kpfonts.html}{Kp-Fonts} family. This is the light version of Kp-Fonts. The difference between the medium~(regular) and light versions is visible in the text {\sl color} and of course, upon magnification of characters. The {\sl light\/} option, which certainly saves the printer tones, is claimed by the author of Kp-Fonts to be better on print than display. The fonts of this macro provide their own {\caps ams} symbols. Details of this \capstex\ macro are given in the table~below.
\bs
\hfil{Font assignment in {\color{brown}\verbatim font_kp-light|endverbatim}~macro}\hfil

{\parindent=0pt\settabs4\columns\hfil\vbox{\hrule\hbox{\vrule\hbox{\vbox{\kern1pt\hrule\NoBlackBoxes		 \eightrm\fontss															 \+\hfil	 \textcolor{blue}{ Typeface}	&\strut\vrule\strut\hfil	\textcolor{blue}{ Font name}	 &\strut\vrule\kern1pt\vrule\strut\hfil	 \textcolor{blue}{ Typeface}	 &\strut\vrule\strut\hfil	\textcolor{blue}{ Font name}	 &\cr	\hrule
\+\hfil	\eightrm Roman text	&\strut\vrule\strut\hfil	jkplmn7t	 &\strut\vrule\kern1pt\vrule\strut\hfil	\eightbf Boldface text	 &\strut\vrule\strut\hfil	 jkplbn7t	&\cr	\hrule
\+\hfil	\eighti Math italic	&\strut\vrule\strut\hfil	jkplmi	 &\strut\vrule\kern1pt\vrule\strut\hfil	\eighttt Typewriter text	 &\strut\vrule\strut\hfil	 jkpttmn7t	&\cr	\hrule
\+\hfil	\eightrm Math symbols	&\strut\vrule\strut\hfil	jkplsy	 &\strut\vrule\kern1pt\vrule\strut\hfil	\eightitbf Italic boldface text	 &\strut\vrule\strut\hfil	jkplbit7t	&\cr	\hrule
\+\hfil	\eightrm Math extension	&\strut\vrule\strut\hfil	jkpex	 &\strut\vrule\kern1pt\vrule\strut\hfil	\eightslbf Slanted boldface text	 &\strut\vrule\strut\hfil	jkplbsl7t	&\cr	\hrule
\+\hfil	\eightit Italic text	&\strut\vrule\strut\hfil	jkplmit7t	 &\strut\vrule\kern1pt\vrule\strut\hfil	\eightcaps Caps	 &\strut\vrule\strut\hfil	jkplmsc7t	 &\cr	\hrule
\+\hfil	\eightsl Slanted text	&\strut\vrule\strut\hfil	jkplmsl7t	 &\strut\vrule\kern1pt\vrule\strut\hfil	\eightcapsbf Caps in Boldface	 &\strut\vrule\strut\hfil	jkplbsc7t	&\cr	\hrule
									
	}\vrule}}\hrule}\hfil}								
									
	\BlackBoxes								

\bs\ii Matching {\caps ams} symbols: \circledR \ \yen \ $\blacksquare \ \approxeq \ \eqslantgtr \ \curlyeqprec \ \curlyeqsucc \ \preccurlyeq \ \leqq \ \leqslant \ \lessgtr \ \nless \ \nleq \ \nleqslant \ \Bbb R \ \Bbb E \ \Bbb C \ \dots$












\input font_antt   \fontss
\section{\sixteenbf\fontss Antykwa Toru\'nska}{Antykwa Torunska}
\sample
\ii The Antykwa Toru\'nska font is declared by typing {\color{brown}\verbatim\input font_antt|endverbatim}. The font family uses fonts from J.\;M.\;Nowacki's \href{http://www.ctan.org/tex-archive/help/Catalogue/entries/antt.html}{antt} package, which corresponds to Zygfryd Gardzielewski's \href{http://nowacki.strefa.pl/torunska-e.html}{Antykwa Toru\'nska} text fonts. Zygfryd Gardzielewski designed Antykwa Toru\'nska in 1960 for Grafmasz typefoundry in Warsaw. L with stroke~(\Lstroke) is displayed by {\color{brown}\verbatim\Lstroke|endverbatim} and l with stroke (\lstroke) is displayed by {\color{brown}\verbatim\lstroke|endverbatim}. When this macro is in use the default plain \capstex\ control statements {\color{brown}\verbatim\L|endverbatim} or {\color{brown}\verbatim\l|endverbatim} do not work. Details of this \capstex\ macro are given in the table~below.
\bs
\hfil{Font assignment in {\color{brown}\verbatim font_antt|endverbatim}~macro}\hfil

{\parindent=0pt\settabs4\columns\hfil\vbox{\hrule\hbox{\vrule\hbox{\vbox{\kern1pt\hrule\NoBlackBoxes		 \eightrm\fontss							 \+\hfil	\textcolor{blue}{ Typeface}	 &\strut\vrule\strut\hfil	\textcolor{blue}{ Font name}	 &\strut\vrule\kern1pt\vrule\strut\hfil	 \textcolor{blue}{ Typeface}	 &\strut\vrule\strut\hfil	\textcolor{blue}{ Font name}	 &\cr	\hrule
\+\hfil	\eightrm Roman text	&\strut\vrule\strut\hfil	rm-anttr	 &\strut\vrule\kern1pt\vrule\strut\hfil	\eightbf Boldface text	 &\strut\vrule\strut\hfil	 rm-anttb	&\cr	\hrule
\+\hfil	\eighti Math italic	&\strut\vrule\strut\hfil	mi-anttri	 &\strut\vrule\kern1pt\vrule\strut\hfil	\eighttt Typewriter text	 &\strut\vrule\strut\hfil	 rm-inconsolata	&\cr	\hrule
\+\hfil	\eightrm Math symbols	&\strut\vrule\strut\hfil	sy-anttrz	 &\strut\vrule\kern1pt\vrule\strut\hfil	 \eightitbf Italic boldface text	 &\strut\vrule\strut\hfil	rm-anttbi	&\cr	\hrule
\+\hfil	\eightrm Math extension	&\strut\vrule\strut\hfil	ex-anttr	 &\strut\vrule\kern1pt\vrule\strut\hfil	 \eightslbf Slanted boldface text	 &\strut\vrule\strut\hfil	rm-anttbi	&\cr	\hrule
\+\hfil	\eightit Italic text	&\strut\vrule\strut\hfil	rm-anttri	 &\strut\vrule\kern1pt\vrule\strut\hfil	 \eightcaps Caps	 &\strut\vrule\strut\hfil	 qx-anttrcap	&\cr	\hrule
\+\hfil	\eightsl Slanted text	&\strut\vrule\strut\hfil	rm-anttri	 &\strut\vrule\kern1pt\vrule\strut\hfil	 \eightcapsbf Caps in Boldface	 &\strut\vrule\strut\hfil	rx-anttbcap	&\cr	\hrule
}\vrule}}\hrule}\hfil}								
									
\BlackBoxes								









\input font_antt-light   \fontss
\section{{\sixteenbf\fontss Antykwa Toru\'nska}-{\sixteenslbf Light}}{Antykwa Torunska-Light}
\sample
\ii The Antykwa Toru\'nska-{\sl Light} font is declared by typing {\color{brown}\verbatim\input font_antt-light|endverbatim}. The font family uses light and medium weight fonts from J.\;M.\;Nowacki's \href{http://www.ctan.org/tex-archive/help/Catalogue/entries/antt.html}{antt} package, which corresponds to Zygfryd Gardzielewski's \href{http://nowacki.strefa.pl/torunska-e.html}{Antykwa Toru\'nska} text fonts. Zygfryd Gardzielewski designed Antykwa Toru\'nska in 1960 for Grafmasz typefoundry in Warsaw. L with stroke~(\Lstroke) is displayed by {\color{brown}\verbatim\Lstroke|endverbatim} and l with stroke~(\lstroke) is displayed by {\color{brown}\verbatim\lstroke|endverbatim}. When this macro is in use the default plain \capstex\ control statements {\color{brown}\verbatim\L|endverbatim} or {\color{brown}\verbatim\l|endverbatim} do not work. Details of this \capstex\ macro are given in the table below.
\bs
\hfil{Font assignment in {\color{brown}\verbatim font_antt-light|endverbatim}~macro}\hfil	
								
{\parindent=0pt\settabs4\columns\hfil\vbox{\hrule\hbox{\vrule\hbox{\vbox{\kern1pt\hrule\NoBlackBoxes		 \eightrm\fontss																 
\+\hfil	\textcolor{blue}{ Typeface}	&\strut\vrule\strut\hfil	\textcolor{blue}{ Font name}	 &\strut\vrule\kern1pt\vrule\strut\hfil	 \textcolor{blue}{ Typeface}	 &\strut\vrule\strut\hfil	\textcolor{blue}{ Font name}	 &\cr	\hrule
\+\hfil	\eightrm Roman text	&\strut\vrule\strut\hfil	rm-anttl	 &\strut\vrule\kern1pt\vrule\strut\hfil	\eightbf Boldface text	 &\strut\vrule\strut\hfil	 rm-anttm	&\cr	\hrule
\+\hfil	\eighti Math italic	&\strut\vrule\strut\hfil	mi-anttli	 &\strut\vrule\kern1pt\vrule\strut\hfil	\eighttt Typewriter text	 &\strut\vrule\strut\hfil	 rm-inconsolata	&\cr	\hrule
\+\hfil	\eightrm Math symbols	&\strut\vrule\strut\hfil	sy-anttlz	 &\strut\vrule\kern1pt\vrule\strut\hfil	\eightitbf Italic boldface text	 &\strut\vrule\strut\hfil	rm-anttmi	&\cr	\hrule
\+\hfil	\eightrm Math extension	&\strut\vrule\strut\hfil	ex-anttl	 &\strut\vrule\kern1pt\vrule\strut\hfil	\eightslbf Slanted boldface text	 &\strut\vrule\strut\hfil	rm-anttmi	&\cr	\hrule
\+\hfil	\eightit Italic text	&\strut\vrule\strut\hfil	rm-anttli	 &\strut\vrule\kern1pt\vrule\strut\hfil	\eightcaps Caps	 &\strut\vrule\strut\hfil	 qx-anttlcap	&\cr	\hrule
\+\hfil	\eightsl Slanted text	&\strut\vrule\strut\hfil	rm-anttli	 &\strut\vrule\kern1pt\vrule\strut\hfil	\eightcapsbf Caps in Boldface	 &\strut\vrule\strut\hfil	qx-anttmcap	&\cr	\hrule
									
	}\vrule}}\hrule}\hfil}								
									
	\BlackBoxes								








\input font_antt-medium   \fontss
\section{{\sixteenbf\fontss Antykwa Toru\'nska}-{\sixteenslbf Medium}}{Antykwa Torunska-Medium}
\sample
\ii The Antykwa Toru\'nska-{\sl Medium} font is declared by typing {\color{brown}\verbatim\input font_antt-medium|endverbatim}. The font family uses medium and bold weight fonts from J.\;M.\;Nowacki's \href{http://www.ctan.org/tex-archive/help/Catalogue/entries/antt.html}{antt} package, which corresponds to Zygfryd Gardzielewski's \href{http://nowacki.strefa.pl/torunska-e.html}{Antykwa Toru\'nska} text fonts. Zygfryd Gardzie\-lewski designed Antykwa Toru\'nska in 1960 for Grafmasz typefoundry in Warsaw. L with stroke~(\Lstroke) is displayed by {\color{brown}\verbatim\Lstroke|endverbatim} and l with stroke~(\lstroke) is displayed by {\color{brown}\verbatim\lstroke|endverbatim}. When this macro is in use the default plain \capstex\ control statements {\color{brown}\verbatim\L|endverbatim} or {\color{brown}\verbatim\l|endverbatim} do not work. Details of this \capstex\ macro are given in the table~below.
\bs
\hfil{Font assignment in {\color{brown}\verbatim font_antt-medium|endverbatim}~macro}\hfil	
								
{\parindent=0pt\settabs4\columns\hfil\vbox{\hrule\hbox{\vrule\hbox{\vbox{\kern1pt\hrule\NoBlackBoxes		 \eightrm\fontss							 
									
\+\hfil	\textcolor{blue}{ Typeface}	&\strut\vrule\strut\hfil	\textcolor{blue}{ Font name}	 &\strut\vrule\kern1pt\vrule\strut\hfil	 \textcolor{blue}{ Typeface}	 &\strut\vrule\strut\hfil	\textcolor{blue}{ Font name}	 &\cr	\hrule
\+\hfil	\eightrm Roman text	&\strut\vrule\strut\hfil	rm-anttm	 &\strut\vrule\kern1pt\vrule\strut\hfil	\eightbf Boldface text	 &\strut\vrule\strut\hfil	 rm-anttb	&\cr	\hrule
\+\hfil	\eighti Math italic	&\strut\vrule\strut\hfil	mi-anttmi	 &\strut\vrule\kern1pt\vrule\strut\hfil	\eighttt Typewriter text	 &\strut\vrule\strut\hfil	 rm-inconsolata	&\cr	\hrule
\+\hfil	\eightrm Math symbols	&\strut\vrule\strut\hfil	sy-anttmz	 &\strut\vrule\kern1pt\vrule\strut\hfil	\eightitbf Italic boldface text	 &\strut\vrule\strut\hfil	rm-anttbi	&\cr	\hrule
\+\hfil	\eightrm Math extension	&\strut\vrule\strut\hfil	ex-anttm	 &\strut\vrule\kern1pt\vrule\strut\hfil	\eightslbf Slanted boldface text	 &\strut\vrule\strut\hfil	rm-anttbi	&\cr	\hrule
\+\hfil	\eightit Italic text	&\strut\vrule\strut\hfil	rm-anttmi	 &\strut\vrule\kern1pt\vrule\strut\hfil	\eightcaps Caps	 &\strut\vrule\strut\hfil	 qx-anttmcap	&\cr	\hrule
\+\hfil	\eightsl Slanted text	&\strut\vrule\strut\hfil	rm-anttmi	 &\strut\vrule\kern1pt\vrule\strut\hfil	\eightcapsbf Caps in Boldface	 &\strut\vrule\strut\hfil	qx-anttbcap	&\cr	\hrule
									
	}\vrule}}\hrule}\hfil}								
									
	\BlackBoxes								













\input font_antt-condensed   \fontss
\section{{\sixteenbf\fontss Antykwa Toru\'nska}-{\sixteenslbf Condensed}}{Antykwa Torunska-Condensed}
\sample
\ii The Antykwa Toru\'nska-{\sl Condensed} font is declared by typing {\color{brown}\verbatim\input font_antt-condensed|endverbatim}. The font family uses condensed width regular and bold weight fonts from J.\;M.\;Nowacki's \href{http://www.ctan.org/tex-archive/help/Catalogue/entries/antt.html}{antt} package, which corresponds to Zygfryd Gardzielewski's \href{http://nowacki.strefa.pl/torunska-e.html}{Antykwa Toru\'nska} text fonts. Zygfryd Gardzie\-lewski designed Antykwa Toru\'nska in 1960 for Grafmasz typefoundry in Warsaw. L with stroke~(\Lstroke) is displayed by {\color{brown}\verbatim\Lstroke|endverbatim} and l with stroke~(\lstroke) is displayed by {\color{brown}\verbatim\lstroke|endverbatim}. When this macro is in use the default plain \capstex\ control statements {\color{brown}\verbatim\L|endverbatim} or {\color{brown}\verbatim\l|endverbatim} do not work. Details of this \capstex\ macro are given in the table~below.
\bs
\hfil{Font assignment in {\color{brown}\verbatim font_antt-condensed|endverbatim}~macro}\hfil	
													
{\parindent=0pt\settabs4\columns\hfil\vbox{\hrule\hbox{\vrule\hbox{\vbox{\kern1pt\hrule\NoBlackBoxes		 \eightrm\fontss							 
									
\+\hfil	\textcolor{blue}{ Typeface}	&\strut\vrule\strut\hfil	\textcolor{blue}{ Font name}	 &\strut\vrule\kern1pt\vrule\strut\hfil	 \textcolor{blue}{ Typeface}	 &\strut\vrule\strut\hfil	\textcolor{blue}{ Font name}	 &\cr	\hrule
\+\hfil	\eightrm Roman text	&\strut\vrule\strut\hfil	rm-anttcr	 &\strut\vrule\kern1pt\vrule\strut\hfil	\eightbf Boldface text	 &\strut\vrule\strut\hfil	 rm-anttcb	&\cr	\hrule
\+\hfil	\eighti Math italic	&\strut\vrule\strut\hfil	mi-anttcri	 &\strut\vrule\kern1pt\vrule\strut\hfil	\eighttt Typewriter text	 &\strut\vrule\strut\hfil	 rm-inconsolata	&\cr	\hrule
\+\hfil	\eightrm Math symbols	&\strut\vrule\strut\hfil	sy-anttcrz	 &\strut\vrule\kern1pt\vrule\strut\hfil	\eightitbf Italic boldface text	 &\strut\vrule\strut\hfil	rm-anttcbi	&\cr	\hrule
\+\hfil	\eightrm Math extension	&\strut\vrule\strut\hfil	ex-anttcr	 &\strut\vrule\kern1pt\vrule\strut\hfil	\eightslbf Slanted boldface text	 &\strut\vrule\strut\hfil	rm-anttcbi	&\cr	\hrule
\+\hfil	\eightit Italic text	&\strut\vrule\strut\hfil	rm-anttcri	 &\strut\vrule\kern1pt\vrule\strut\hfil	\eightcaps Caps	 &\strut\vrule\strut\hfil	 qx-anttcrcap	&\cr	\hrule
\+\hfil	\eightsl Slanted text	&\strut\vrule\strut\hfil	rm-anttcri	 &\strut\vrule\kern1pt\vrule\strut\hfil	\eightcapsbf Caps in Boldface	 &\strut\vrule\strut\hfil	qx-anttcbcap	&\cr	\hrule
									
	}\vrule}}\hrule}\hfil}								
									
	\BlackBoxes								













\input font_antt-condensed-light   \fontss
\section{{\sixteenbf\fontss Antykwa Toru\'nska}-{\sixteenslbf Condensed Light}}{Antykwa Torunska-Condensed Light}
\sample
\ii Antykwa Toru\'nska-{\sl Condensed Light} font is declared by typing {\color{brown}\verbatim\input font_antt-condensed-light|endverbatim}. The font family uses condensed width light and medium weight fonts from J.\;M.\;Nowacki's \href{http://www.ctan.org/tex-archive/help/Catalogue/entries/antt.html}{antt} package, which corresponds to Zygfryd Gardzielewski's \href{http://nowacki.strefa.pl/torunska-e.html}{Antykwa Toru\'nska} text fonts. Zygfryd Gardzie\-lewski designed Antykwa Toru\'nska in 1960 for Grafmasz typefoundry in Warsaw. L with stroke~(\Lstroke) is displayed by {\color{brown}\verbatim\Lstroke|endverbatim} and l with stroke~(\lstroke) is displayed by {\color{brown}\verbatim\lstroke|endverbatim}. When this macro is in use the default plain \capstex\ control statements {\color{brown}\verbatim\L|endverbatim} or {\color{brown}\verbatim\l|endverbatim} do not work. Details of this \capstex\ macro are given in the table~below.
\bs
\hfil{Font assignment in {\color{brown}\verbatim font_antt-condensed-light|endverbatim}~macro}\hfil

			% Antykwa Toru?ska-Light						
{\parindent=0pt\settabs4\columns\hfil\vbox{\hrule\hbox{\vrule\hbox{\vbox{\kern1pt\hrule\NoBlackBoxes		 \eightrm\fontss							 
									
\+\hfil	\textcolor{blue}{ Typeface}	&\strut\vrule\strut\hfil	\textcolor{blue}{ Font name}	 &\strut\vrule\kern1pt\vrule\strut\hfil	 \textcolor{blue}{ Typeface}	 &\strut\vrule\strut\hfil	\textcolor{blue}{ Font name}	 &\cr	\hrule
\+\hfil	\eightrm Roman text	&\strut\vrule\strut\hfil	rm-anttcl	 &\strut\vrule\kern1pt\vrule\strut\hfil	\eightbf Boldface text	 &\strut\vrule\strut\hfil	 rm-anttcm	&\cr	\hrule
\+\hfil	\eighti Math italic	&\strut\vrule\strut\hfil	mi-anttcli	 &\strut\vrule\kern1pt\vrule\strut\hfil	\eighttt Typewriter text	 &\strut\vrule\strut\hfil	 rm-inconsolata	&\cr	\hrule
\+\hfil	\eightrm Math symbols	&\strut\vrule\strut\hfil	sy-anttclz	 &\strut\vrule\kern1pt\vrule\strut\hfil	\eightitbf Italic boldface text	 &\strut\vrule\strut\hfil	rm-anttcmi	&\cr	\hrule
\+\hfil	\eightrm Math extension	&\strut\vrule\strut\hfil	ex-anttcl	 &\strut\vrule\kern1pt\vrule\strut\hfil	\eightslbf Slanted boldface text	 &\strut\vrule\strut\hfil	rm-anttcmi	&\cr	\hrule
\+\hfil	\eightit Italic text	&\strut\vrule\strut\hfil	rm-anttcli	 &\strut\vrule\kern1pt\vrule\strut\hfil	\eightcaps Caps	 &\strut\vrule\strut\hfil	 qx-anttclcap	&\cr	\hrule
\+\hfil	\eightsl Slanted text	&\strut\vrule\strut\hfil	rm-anttcli	 &\strut\vrule\kern1pt\vrule\strut\hfil	\eightcapsbf Caps in Boldface	 &\strut\vrule\strut\hfil	qx-anttcmcap	&\cr	\hrule
									
	}\vrule}}\hrule}\hfil}								
									
	\BlackBoxes								


















\input font_antt-condensed-medium   \fontss
\section{{\sixteenbf\fontss Antykwa Toru\'nska}-{\sixteenslbf Condensed Medium}}{Antykwa Torunska-Condensed Medium}
\sample
\ii The Antykwa Toru\'nska-{\sl Condensed Medium} font can be used in \capstex\ documents after typing\break {\color{brown}\verbatim\input font_antt-condensed-medium|endverbatim}. The font family uses condensed width medium and bold weight fonts from J.\,M.\,Nowacki's \href{http://www.ctan.org/tex-archive/help/Catalogue/entries/antt.html}{antt} package, which corresponds to Zygfryd Gardzielewski's \href{http://nowacki.strefa.pl/torunska-e.html}{Antykwa Toru\'nska} text fonts. Zygfryd Gardzie\-lewski designed Antykwa Toru\'nska in 1960 for Grafmasz typefoundry in Warsaw. L with stroke~(\Lstroke) is displayed by {\color{brown}\verbatim\Lstroke|endverbatim} and l with stroke~(\lstroke) is displayed by {\color{brown}\verbatim\lstroke|endverbatim}. When this macro is in use the default plain \capstex\ control statements {\color{brown}\verbatim\L|endverbatim} or {\color{brown}\verbatim\l|endverbatim} do not work. Details of this \capstex\ macro are given in the table~below.
\bs
\hfil{Font assignment in {\color{brown}\verbatim font_antt-condensed-medium|endverbatim}~macro}\hfil	

							
{\parindent=0pt\settabs4\columns\hfil\vbox{\hrule\hbox{\vrule\hbox{\vbox{\kern1pt\hrule\NoBlackBoxes		 \eightrm\fontss							 
									
\+\hfil	\textcolor{blue}{ Typeface}	&\strut\vrule\strut\hfil	\textcolor{blue}{ Font name}	 &\strut\vrule\kern1pt\vrule\strut\hfil	 \textcolor{blue}{ Typeface}	 &\strut\vrule\strut\hfil	\textcolor{blue}{ Font name}	 &\cr	\hrule
\+\hfil	\eightrm Roman text	&\strut\vrule\strut\hfil	rm-anttcm	 &\strut\vrule\kern1pt\vrule\strut\hfil	\eightbf Boldface text	 &\strut\vrule\strut\hfil	 rm-anttcb	&\cr	\hrule
\+\hfil	\eighti Math italic	&\strut\vrule\strut\hfil	mi-anttcmi	 &\strut\vrule\kern1pt\vrule\strut\hfil	\eighttt Typewriter text	 &\strut\vrule\strut\hfil	 rm-inconsolata	&\cr	\hrule
\+\hfil	\eightrm Math symbols	&\strut\vrule\strut\hfil	sy-anttcmz	 &\strut\vrule\kern1pt\vrule\strut\hfil	\eightitbf Italic boldface text	 &\strut\vrule\strut\hfil	rm-anttcbi	&\cr	\hrule
\+\hfil	\eightrm Math extension	&\strut\vrule\strut\hfil	ex-anttcm	 &\strut\vrule\kern1pt\vrule\strut\hfil	\eightslbf Slanted boldface text	 &\strut\vrule\strut\hfil	rm-anttcbi	&\cr	\hrule
\+\hfil	\eightit Italic text	&\strut\vrule\strut\hfil	rm-anttcmi	 &\strut\vrule\kern1pt\vrule\strut\hfil	\eightcaps Caps	 &\strut\vrule\strut\hfil	 qx-anttcmcap	&\cr	\hrule
\+\hfil	\eightsl Slanted text	&\strut\vrule\strut\hfil	rm-anttcmi	 &\strut\vrule\kern1pt\vrule\strut\hfil	\eightcapsbf Caps in Boldface	 &\strut\vrule\strut\hfil	qx-anttcbcap	&\cr	\hrule
									
	}\vrule}}\hrule}\hfil}								
									
	\BlackBoxes								















\input font_iwona   \fontss
\section{\sixteenbf\fontss Iwona}{Iwona}
\sample
\ii The Iwona font is declared by typing {\color{brown}\verbatim\input font_iwona|endverbatim}. The font family uses fonts from J.\;M.\break Nowacki's \href{http://tug.ctan.org/cgi-bin/ctanPackageInformation.py?id=iwona}{iwona} package, which corresponds to Ma\lstroke{}gorzata Budyta's text fonts. L with stroke~(\Lstroke) is displayed by {\color{brown}\verbatim\Lstroke|endverbatim} and l with stroke~(\lstroke) is displayed by {\color{brown}\verbatim\lstroke|endverbatim}. When this macro is in use the default plain \capstex\ control statements {\color{brown}\verbatim\L|endverbatim} or {\color{brown}\verbatim\l|endverbatim} do not work. Details of this \capstex\ macro are given in the table~below.
\bs
\hfil{Font assignment in {\color{brown}\verbatim font_iwona|endverbatim}~macro}\hfil

{\parindent=0pt\settabs4\columns\hfil\vbox{\hrule\hbox{\vrule\hbox{\vbox{\kern1pt\hrule\NoBlackBoxes		 \eightrm\fontss							 \+\hfil	\textcolor{blue}{ Typeface}	 &\strut\vrule\strut\hfil	\textcolor{blue}{ Font name}	 &\strut\vrule\kern1pt\vrule\strut\hfil	 \textcolor{blue}{ Typeface}	 &\strut\vrule\strut\hfil	 \textcolor{blue}{ Font name}	 &\cr	\hrule
\+\hfil	\eightrm Roman text	&\strut\vrule\strut\hfil	rm-iwonar	 &\strut\vrule\kern1pt\vrule\strut\hfil	 \eightbf Boldface text	 &\strut\vrule\strut\hfil	 rm-iwonab	 &\cr	\hrule
\+\hfil	\eighti Math italic	&\strut\vrule\strut\hfil	mi-iwonari	 &\strut\vrule\kern1pt\vrule\strut\hfil	 \eighttt Typewriter text	 &\strut\vrule\strut\hfil	 rm-inconsolata	&\cr	\hrule
\+\hfil	\eightrm Math symbols	&\strut\vrule\strut\hfil	sy-iwonarz	 &\strut\vrule\kern1pt\vrule\strut\hfil	 \eightitbf Italic boldface text	 &\strut\vrule\strut\hfil	 rm-iwonabi	&\cr	\hrule
\+\hfil	\eightrm Math extension	&\strut\vrule\strut\hfil	ex-iwonar	 &\strut\vrule\kern1pt\vrule\strut\hfil	 \eightslbf Slanted boldface text	 &\strut\vrule\strut\hfil	 rm-iwonabi	&\cr	\hrule
\+\hfil	\eightit Italic text	&\strut\vrule\strut\hfil	rm-iwonari	 &\strut\vrule\kern1pt\vrule\strut\hfil	 \eightcaps Caps	 &\strut\vrule\strut\hfil	 qx-iwonarcap	 &\cr	\hrule
\+\hfil	\eightsl Slanted text	&\strut\vrule\strut\hfil	rm-iwonari	 &\strut\vrule\kern1pt\vrule\strut\hfil	 \eightcapsbf Caps in Boldface	 &\strut\vrule\strut\hfil	 qx-iwonabcap	&\cr	\hrule
}\vrule}}\hrule}\hfil}								
									
\BlackBoxes	










\input font_iwona-light   \fontss
\section{\sixteenbf\fontss Iwona-{\sixteenslbf Light}}{Iwona-Light}
\sample
\ii The Iwona-{\sl Light\/} font is declared by typing {\color{brown}\verbatim\input font_iwona-light|endverbatim}. The font family uses light and bold weight Iwona fonts from J.\;M.\;Nowacki's \href{http://tug.ctan.org/cgi-bin/ctanPackageInformation.py?id=iwona}{iwona} package, which corresponds to Ma\lstroke{}gorzata Budyta's text fonts. L with stroke~(\Lstroke) is displayed by {\color{brown}\verbatim\Lstroke|endverbatim} and l with stroke~(\lstroke) is displayed by {\color{brown}\verbatim\lstroke|endverbatim}. When this macro is in use the default plain \capstex\ control statements {\color{brown}\verbatim\L|endverbatim} or {\color{brown}\verbatim\l|endverbatim} do not work. Details of this \capstex\ macro are given in the table~below.
\bs
\hfil{Font assignment in {\color{brown}\verbatim font_iwona-light|endverbatim}~macro}\hfil

{\parindent=0pt\settabs4\columns\hfil\vbox{\hrule\hbox{\vrule\hbox{\vbox{\kern1pt\hrule\NoBlackBoxes		 \eightrm\fontss							 
									
\+\hfil	\textcolor{blue}{ Typeface}	&\strut\vrule\strut\hfil	\textcolor{blue}{ Font name}	 &\strut\vrule\kern1pt\vrule\strut\hfil	 \textcolor{blue}{ Typeface}	 &\strut\vrule\strut\hfil	\textcolor{blue}{ Font name}	&\cr	\hrule
\+\hfil	\eightrm Roman text	&\strut\vrule\strut\hfil	rm-iwonal	 &\strut\vrule\kern1pt\vrule\strut\hfil	\eightbf Boldface text	 &\strut\vrule\strut\hfil	 rm-iwonam	&\cr	\hrule
\+\hfil	\eighti Math italic	&\strut\vrule\strut\hfil	mi-iwonali	 &\strut\vrule\kern1pt\vrule\strut\hfil	\eighttt Typewriter text	 &\strut\vrule\strut\hfil	 rm-inconsolata	&\cr	\hrule
\+\hfil	\eightrm Math symbols	&\strut\vrule\strut\hfil	sy-iwonalz	 &\strut\vrule\kern1pt\vrule\strut\hfil	\eightitbf Italic boldface text	 &\strut\vrule\strut\hfil	rm-iwonami	&\cr	\hrule
\+\hfil	\eightrm Math extension	&\strut\vrule\strut\hfil	ex-iwonal	 &\strut\vrule\kern1pt\vrule\strut\hfil	\eightslbf Slanted boldface text	 &\strut\vrule\strut\hfil	rm-iwonami	&\cr	\hrule
\+\hfil	\eightit Italic text	&\strut\vrule\strut\hfil	rm-iwonali	 &\strut\vrule\kern1pt\vrule\strut\hfil	\eightcaps Caps	 &\strut\vrule\strut\hfil	 qx-iwonalcap	&\cr	\hrule
\+\hfil	\eightsl Slanted text	&\strut\vrule\strut\hfil	rm-iwonali	 &\strut\vrule\kern1pt\vrule\strut\hfil	\eightcapsbf Caps in Boldface	 &\strut\vrule\strut\hfil	qx-iwonamcap	&\cr	\hrule
									
	}\vrule}}\hrule}\hfil}								
									
	\BlackBoxes								










\input font_iwona-medium   \fontss
\section{\sixteenbf\fontss Iwona-{\sixteenslbf Medium}}{Iwona-Medium}
\sample
\ii The Iwona-{\sl Medium\/} font is declared by typing {\color{brown}\verbatim\input font_iwona-medium|endverbatim}. The font family uses\break medium and heavy weight Iwona fonts from J.\;M.\;Nowacki's \href{http://tug.ctan.org/cgi-bin/ctanPackageInformation.py?id=iwona}{iwona} package, which corresponds to Ma\lstroke{}gorzata Budyta's text fonts. L with stroke~(\Lstroke) is displayed by {\color{brown}\verbatim\Lstroke|endverbatim} and l with stroke~(\lstroke) is displayed by {\color{brown}\verbatim\lstroke|endverbatim}. When this macro is in use the default plain \capstex\ control statements {\color{brown}\verbatim\L|endverbatim} or {\color{brown}\verbatim\l|endverbatim} do not work. Details of this \capstex\ macro are given in the table~below.
\bs
\hfil{Font assignment in {\color{brown}\verbatim font_iwona-medium|endverbatim}~macro}\hfil

{\parindent=0pt\settabs4\columns\hfil\vbox{\hrule\hbox{\vrule\hbox{\vbox{\kern1pt\hrule\NoBlackBoxes		 \eightrm\fontss							 
									
\+\hfil	\textcolor{blue}{ Typeface}	&\strut\vrule\strut\hfil	\textcolor{blue}{ Font name}	 &\strut\vrule\kern1pt\vrule\strut\hfil	 \textcolor{blue}{ Typeface}	 &\strut\vrule\strut\hfil	\textcolor{blue}{ Font name}	&\cr	\hrule
\+\hfil	\eightrm Roman text	&\strut\vrule\strut\hfil	rm-iwonam	 &\strut\vrule\kern1pt\vrule\strut\hfil	\eightbf Boldface text	 &\strut\vrule\strut\hfil	 rm-iwonah	&\cr	\hrule
\+\hfil	\eighti Math italic	&\strut\vrule\strut\hfil	mi-iwonami	 &\strut\vrule\kern1pt\vrule\strut\hfil	\eighttt Typewriter text	 &\strut\vrule\strut\hfil	 rm-inconsolata	&\cr	\hrule
\+\hfil	\eightrm Math symbols	&\strut\vrule\strut\hfil	sy-iwonamz	 &\strut\vrule\kern1pt\vrule\strut\hfil	\eightitbf Italic boldface text	 &\strut\vrule\strut\hfil	rm-iwonahi	&\cr	\hrule
\+\hfil	\eightrm Math extension	&\strut\vrule\strut\hfil	ex-iwonam	 &\strut\vrule\kern1pt\vrule\strut\hfil	\eightslbf Slanted boldface text	 &\strut\vrule\strut\hfil	rm-iwonahi	&\cr	\hrule
\+\hfil	\eightit Italic text	&\strut\vrule\strut\hfil	rm-iwonami	 &\strut\vrule\kern1pt\vrule\strut\hfil	\eightcaps Caps	 &\strut\vrule\strut\hfil	 qx-iwonamcap	&\cr	\hrule
\+\hfil	\eightsl Slanted text	&\strut\vrule\strut\hfil	rm-iwonami	 &\strut\vrule\kern1pt\vrule\strut\hfil	\eightcapsbf Caps in Boldface	 &\strut\vrule\strut\hfil	qx-iwonahcap	&\cr	\hrule
									
	}\vrule}}\hrule}\hfil}								
									
	\BlackBoxes								






\input font_iwona-bold   \fontss
\section{\sixteenbf\fontss Iwona-{\sixteenslbf Bold}}{Iwona-Bold}
\sample
\ii The Iwona-{\sl Bold\/} font is declared by typing {\color{brown}\verbatim\input font_iwona-bold|endverbatim}. The font family uses bold and heavy weight Iwona fonts from J.\;M.\;Nowacki's \href{http://tug.ctan.org/cgi-bin/ctanPackageInformation.py?id=iwona}{iwona} package, which corresponds to Ma\lstroke{}gorzata Budyta's text fonts. L with stroke~(\Lstroke) is displayed by {\color{brown}\verbatim\Lstroke|endverbatim} and l with stroke~(\lstroke) is displayed by {\color{brown}\verbatim\lstroke|endverbatim}. When this macro is in use the default plain \capstex\ control statements {\color{brown}\verbatim\L|endverbatim} or {\color{brown}\verbatim\l|endverbatim} do not work. Details of this \capstex\ macro are given in the table~below.
\bs
\hfil{Font assignment in {\color{brown}\verbatim font_iwona-medium|endverbatim}~macro}\hfil

{\parindent=0pt\settabs4\columns\hfil\vbox{\hrule\hbox{\vrule\hbox{\vbox{\kern1pt\hrule\NoBlackBoxes		 \eightrm\fontss							 
									
\+\hfil	\textcolor{blue}{ Typeface}	&\strut\vrule\strut\hfil	\textcolor{blue}{ Font name}	 &\strut\vrule\kern1pt\vrule\strut\hfil	 \textcolor{blue}{ Typeface}	 &\strut\vrule\strut\hfil	\textcolor{blue}{ Font name}	&\cr	\hrule
\+\hfil	\eightrm Roman text	&\strut\vrule\strut\hfil	rm-iwonab	 &\strut\vrule\kern1pt\vrule\strut\hfil	\eightbf Boldface text	 &\strut\vrule\strut\hfil	 rm-iwonah	&\cr	\hrule
\+\hfil	\eighti Math italic	&\strut\vrule\strut\hfil	mi-iwonabi	 &\strut\vrule\kern1pt\vrule\strut\hfil	\eighttt Typewriter text	 &\strut\vrule\strut\hfil	 rm-inconsolata	&\cr	\hrule
\+\hfil	\eightrm Math symbols	&\strut\vrule\strut\hfil	sy-iwonabz	 &\strut\vrule\kern1pt\vrule\strut\hfil	\eightitbf Italic boldface text	 &\strut\vrule\strut\hfil	rm-iwonahi	&\cr	\hrule
\+\hfil	\eightrm Math extension	&\strut\vrule\strut\hfil	ex-iwonab	 &\strut\vrule\kern1pt\vrule\strut\hfil	\eightslbf Slanted boldface text	 &\strut\vrule\strut\hfil	rm-iwonahi	&\cr	\hrule
\+\hfil	\eightit Italic text	&\strut\vrule\strut\hfil	rm-iwonabi	 &\strut\vrule\kern1pt\vrule\strut\hfil	\eightcaps Caps	 &\strut\vrule\strut\hfil	 qx-iwonabcap	&\cr	\hrule
\+\hfil	\eightsl Slanted text	&\strut\vrule\strut\hfil	rm-iwonabi	 &\strut\vrule\kern1pt\vrule\strut\hfil	\eightcapsbf Caps in Boldface	 &\strut\vrule\strut\hfil	qx-iwonahcap	&\cr	\hrule
									
	}\vrule}}\hrule}\hfil}								
									
	\BlackBoxes								








\input font_iwona-condensed   \fontss
\section{\sixteenbf\fontss Iwona-{\sixteenslbf Condensed}}{Iwona-Condensed}
\sample
\ii The Iwona-{\sl Condensed\/} font is declared by typing {\color{brown}\verbatim\input font_iwona-condensed|endverbatim}. The font family uses condensed width regular and bold weight Iwona fonts from J.\;M.\;Nowacki's \href{http://tug.ctan.org/cgi-bin/ctanPackageInformation.py?id=iwona}{iwona} package, which corresponds to Ma\lstroke{}gorzata Budyta's text fonts. L with stroke~(\Lstroke) is displayed by {\color{brown}\verbatim\Lstroke|endverbatim} and l with stroke~(\lstroke) is displayed by {\color{brown}\verbatim\lstroke|endverbatim}. When this macro is in use the default plain \capstex\ control statements {\color{brown}\verbatim\L|endverbatim} or {\color{brown}\verbatim\l|endverbatim} do not work. Details of this \capstex\ macro are given in the table~below.
\bs
\hfil{Font assignment in {\color{brown}\verbatim font_iwona-condensed|endverbatim}~macro}\hfil

{\parindent=0pt\settabs4\columns\hfil\vbox{\hrule\hbox{\vrule\hbox{\vbox{\kern1pt\hrule\NoBlackBoxes		 \eightrm\fontss							 
									
\+\hfil	\textcolor{blue}{ Typeface}	&\strut\vrule\strut\hfil	\textcolor{blue}{ Font name}	 &\strut\vrule\kern1pt\vrule\strut\hfil	 \textcolor{blue}{ Typeface}	 &\strut\vrule\strut\hfil	\textcolor{blue}{ Font name}	&\cr	\hrule
\+\hfil	\eightrm Roman text	&\strut\vrule\strut\hfil	rm-iwonacr	 &\strut\vrule\kern1pt\vrule\strut\hfil	\eightbf Boldface text	 &\strut\vrule\strut\hfil	 rm-iwonacb	&\cr	\hrule
\+\hfil	\eighti Math italic	&\strut\vrule\strut\hfil	mi-iwonacri	 &\strut\vrule\kern1pt\vrule\strut\hfil	\eighttt Typewriter text	 &\strut\vrule\strut\hfil	 rm-inconsolata	&\cr	\hrule
\+\hfil	\eightrm Math symbols	&\strut\vrule\strut\hfil	sy-iwonacrz	 &\strut\vrule\kern1pt\vrule\strut\hfil	\eightitbf Italic boldface text	 &\strut\vrule\strut\hfil	rm-iwonacbi	&\cr	\hrule
\+\hfil	\eightrm Math extension	&\strut\vrule\strut\hfil	ex-iwonacr	 &\strut\vrule\kern1pt\vrule\strut\hfil	\eightslbf Slanted boldface text	 &\strut\vrule\strut\hfil	rm-iwonacbi	&\cr	\hrule
\+\hfil	\eightit Italic text	&\strut\vrule\strut\hfil	rm-iwonacri	 &\strut\vrule\kern1pt\vrule\strut\hfil	\eightcaps Caps	 &\strut\vrule\strut\hfil	 qx-iwonacrcap	&\cr	\hrule
\+\hfil	\eightsl Slanted text	&\strut\vrule\strut\hfil	rm-iwonacri	 &\strut\vrule\kern1pt\vrule\strut\hfil	\eightcapsbf Caps in Boldface	 &\strut\vrule\strut\hfil	qx-iwonacbcap	&\cr	\hrule
									
	}\vrule}}\hrule}\hfil}								
									
	\BlackBoxes								







\input font_iwona-condensed-light   \fontss
\section{\sixteenbf\fontss Iwona-{\sixteenslbf Condensed-Light}}{Iwona-Condensed-Light}
\sample
\ii The Iwona-{\sl Condensed-Light\/} font is declared by typing {\color{brown}\verbatim\input font_iwona-condensed-light|endverbatim}. The font family uses condensed width light and medium weight Iwona fonts from J.\;M.\;Nowacki's \href{http://tug.ctan.org/cgi-bin/ctanPackageInformation.py?id=iwona}{iwona} package, which corresponds to Ma\lstroke{}gorzata Budyta's text fonts. L with stroke~(\Lstroke) is displayed by {\color{brown}\verbatim\Lstroke|endverbatim} and l with stroke~(\lstroke) is displayed by {\color{brown}\verbatim\lstroke|endverbatim}. When this macro is in use the default plain \capstex\ control statements {\color{brown}\verbatim\L|endverbatim} or {\color{brown}\verbatim\l|endverbatim} do not work. Details of this \capstex\ macro are given in the table~below.
\bs
\hfil{Font assignment in {\color{brown}\verbatim font_iwona-condensed-light|endverbatim}~macro}\hfil

{\parindent=0pt\settabs4\columns\hfil\vbox{\hrule\hbox{\vrule\hbox{\vbox{\kern1pt\hrule\NoBlackBoxes		 \eightrm\fontss							 
									
\+\hfil	\textcolor{blue}{ Typeface}	&\strut\vrule\strut\hfil	\textcolor{blue}{ Font name}	 &\strut\vrule\kern1pt\vrule\strut\hfil	 \textcolor{blue}{ Typeface}	 &\strut\vrule\strut\hfil	\textcolor{blue}{ Font name}	&\cr	\hrule
\+\hfil	\eightrm Roman text	&\strut\vrule\strut\hfil	rm-iwonacl	 &\strut\vrule\kern1pt\vrule\strut\hfil	\eightbf Boldface text	 &\strut\vrule\strut\hfil	 rm-iwonacm	&\cr	\hrule
\+\hfil	\eighti Math italic	&\strut\vrule\strut\hfil	mi-iwonacli	 &\strut\vrule\kern1pt\vrule\strut\hfil	\eighttt Typewriter text	 &\strut\vrule\strut\hfil	 rm-inconsolata	&\cr	\hrule
\+\hfil	\eightrm Math symbols	&\strut\vrule\strut\hfil	sy-iwonaclz	 &\strut\vrule\kern1pt\vrule\strut\hfil	\eightitbf Italic boldface text	 &\strut\vrule\strut\hfil	rm-iwonacmi	&\cr	\hrule
\+\hfil	\eightrm Math extension	&\strut\vrule\strut\hfil	ex-iwonacl	 &\strut\vrule\kern1pt\vrule\strut\hfil	\eightslbf Slanted boldface text	 &\strut\vrule\strut\hfil	rm-iwonacmi	&\cr	\hrule
\+\hfil	\eightit Italic text	&\strut\vrule\strut\hfil	rm-iwonacli	 &\strut\vrule\kern1pt\vrule\strut\hfil	\eightcaps Caps	 &\strut\vrule\strut\hfil	 qx-iwonaclcap	&\cr	\hrule
\+\hfil	\eightsl Slanted text	&\strut\vrule\strut\hfil	rm-iwonacli	 &\strut\vrule\kern1pt\vrule\strut\hfil	\eightcapsbf Caps in Boldface	 &\strut\vrule\strut\hfil	qx-iwonacmcap	&\cr	\hrule
									
	}\vrule}}\hrule}\hfil}								
									
	\BlackBoxes								
									








\input font_iwona-condensed-medium  \fontss
\section{\sixteenbf\fontss Iwona-{\sixteenslbf Condensed-Medium}}{Iwona-Condensed-Medium}
\sample
\ii The Iwona-{\sl Condensed-Medium\/} font is declared by typing {\color{brown}\verbatim\input font_iwona-condensed-medium|endverbatim}. The font family uses condensed width medium and heavy weight Iwona fonts from J.\;M.\;Nowacki's \href{http://tug.ctan.org/cgi-bin/ctanPackageInformation.py?id=iwona}{iwona} package, which corresponds to Ma\lstroke{}gorzata Budyta's text fonts. L with stroke~(\Lstroke) is displayed by {\color{brown}\verbatim\Lstroke|endverbatim} and l with stroke~(\lstroke) is displayed by {\color{brown}\verbatim\lstroke|endverbatim}. When this macro is in use the default plain \capstex\ control statements {\color{brown}\verbatim\L|endverbatim} or {\color{brown}\verbatim\l|endverbatim} do not work. Details of this \capstex\ macro are given in the table~below.
\bs
\hfil{Font assignment in {\color{brown}\verbatim font_iwona-condensed-medium|endverbatim}~macro}\hfil

{\parindent=0pt\settabs4\columns\hfil\vbox{\hrule\hbox{\vrule\hbox{\vbox{\kern1pt\hrule\NoBlackBoxes		 \eightrm\fontss							 
									
\+\hfil	\textcolor{blue}{ Typeface}	&\strut\vrule\strut\hfil	\textcolor{blue}{ Font name}	 &\strut\vrule\kern1pt\vrule\strut\hfil	 \textcolor{blue}{ Typeface}	 &\strut\vrule\strut\hfil	\textcolor{blue}{ Font name}	&\cr	\hrule
\+\hfil	\eightrm Roman text	&\strut\vrule\strut\hfil	rm-iwonacm	 &\strut\vrule\kern1pt\vrule\strut\hfil	\eightbf Boldface text	 &\strut\vrule\strut\hfil	 rm-iwonach	&\cr	\hrule
\+\hfil	\eighti Math italic	&\strut\vrule\strut\hfil	mi-iwonacmi	 &\strut\vrule\kern1pt\vrule\strut\hfil	\eighttt Typewriter text	 &\strut\vrule\strut\hfil	 rm-inconsolata	&\cr	\hrule
\+\hfil	\eightrm Math symbols	&\strut\vrule\strut\hfil	sy-iwonacmz	 &\strut\vrule\kern1pt\vrule\strut\hfil	\eightitbf Italic boldface text	 &\strut\vrule\strut\hfil	rm-iwonachi	&\cr	\hrule
\+\hfil	\eightrm Math extension	&\strut\vrule\strut\hfil	ex-iwonacm	 &\strut\vrule\kern1pt\vrule\strut\hfil	\eightslbf Slanted boldface text	 &\strut\vrule\strut\hfil	rm-iwonachi	&\cr	\hrule
\+\hfil	\eightit Italic text	&\strut\vrule\strut\hfil	rm-iwonacmi	 &\strut\vrule\kern1pt\vrule\strut\hfil	\eightcaps Caps	 &\strut\vrule\strut\hfil	 qx-iwonacmcap	&\cr	\hrule
\+\hfil	\eightsl Slanted text	&\strut\vrule\strut\hfil	rm-iwonacmi	 &\strut\vrule\kern1pt\vrule\strut\hfil	\eightcapsbf Caps in Boldface	 &\strut\vrule\strut\hfil	qx-iwonachcap	&\cr	\hrule
									
	}\vrule}}\hrule}\hfil}								
									
	\BlackBoxes								







\input font_iwona-condensed-bold  \fontss
\section{\sixteenbf\fontss Iwona-{\sixteenslbf Condensed-Bold}}{Iwona-Condensed-Bold}
\sample
\ii The Iwona-{\sl Condensed-Bold\/} font is declared by typing {\color{brown}\verbatim\input font_iwona-condensed-bold|endverbatim}. The font\break family uses condensed width bold and heavy weight Iwona fonts from J.\;M.\;Nowacki's \href{http://tug.ctan.org/cgi-bin/ctanPackageInformation.py?id=iwona}{iwona} package, which corresponds to Ma\lstroke{}gorzata Budyta's text fonts. L with stroke~(\Lstroke) is displayed by {\color{brown}\verbatim\Lstroke|endverbatim} and l with stroke~(\lstroke) is displayed by {\color{brown}\verbatim\lstroke|endverbatim}. When this macro is in use the default plain \capstex\ control statements {\color{brown}\verbatim\L|endverbatim} or {\color{brown}\verbatim\l|endverbatim} do not work. Details of this \capstex\ macro are given in the table~below.
\bs
\hfil{Font assignment in {\color{brown}\verbatim font_iwona-condensed-bold|endverbatim}~macro}\hfil

{\parindent=0pt\settabs4\columns\hfil\vbox{\hrule\hbox{\vrule\hbox{\vbox{\kern1pt\hrule\NoBlackBoxes		 \eightrm\fontss							 
									
\+\hfil	\textcolor{blue}{ Typeface}	&\strut\vrule\strut\hfil	\textcolor{blue}{ Font name}	 &\strut\vrule\kern1pt\vrule\strut\hfil	 \textcolor{blue}{ Typeface}	 &\strut\vrule\strut\hfil	\textcolor{blue}{ Font name}	&\cr	\hrule
\+\hfil	\eightrm Roman text	&\strut\vrule\strut\hfil	rm-iwonacb	 &\strut\vrule\kern1pt\vrule\strut\hfil	\eightbf Boldface text	 &\strut\vrule\strut\hfil	 rm-iwonach	&\cr	\hrule
\+\hfil	\eighti Math italic	&\strut\vrule\strut\hfil	mi-iwonacbi	 &\strut\vrule\kern1pt\vrule\strut\hfil	\eighttt Typewriter text	 &\strut\vrule\strut\hfil	 rm-inconsolata	&\cr	\hrule
\+\hfil	\eightrm Math symbols	&\strut\vrule\strut\hfil	sy-iwonacbz	 &\strut\vrule\kern1pt\vrule\strut\hfil	\eightitbf Italic boldface text	 &\strut\vrule\strut\hfil	rm-iwonachi	&\cr	\hrule
\+\hfil	\eightrm Math extension	&\strut\vrule\strut\hfil	ex-iwonacb	 &\strut\vrule\kern1pt\vrule\strut\hfil	\eightslbf Slanted boldface text	 &\strut\vrule\strut\hfil	rm-iwonachi	&\cr	\hrule
\+\hfil	\eightit Italic text	&\strut\vrule\strut\hfil	rm-iwonacbi	 &\strut\vrule\kern1pt\vrule\strut\hfil	\eightcaps Caps	 &\strut\vrule\strut\hfil	 qx-iwonacbcap	&\cr	\hrule
\+\hfil	\eightsl Slanted text	&\strut\vrule\strut\hfil	rm-iwonacbi	 &\strut\vrule\kern1pt\vrule\strut\hfil	\eightcapsbf Caps in Boldface	 &\strut\vrule\strut\hfil	qx-iwonachcap	&\cr	\hrule
									
	}\vrule}}\hrule}\hfil}								
									
	\BlackBoxes								




















\input font_kurier  \fontss
\section{\sixteenbf\fontss Kurier}{Kurier}
\sample
\ii The Kurier font is declared by typing {\color{brown}\verbatim\input font_kurier|endverbatim}. The font family uses fonts from J.\;M.\ Nowacki's \href{http://www.ctan.org/tex-archive/help/Catalogue/entries/kurier.html}{kurier} package, which corresponds to Ma\lstroke{}gorzata Budyta's text fonts. The Kurier font is very similar to Iwona font; Kurier is a bit extended and has ink traps. L with stroke~(\Lstroke) is displayed by {\color{brown}\verbatim\Lstroke|endverbatim} and l with stroke~(\lstroke) is displayed by {\color{brown}\verbatim\lstroke|endverbatim}. When this macro is in use the default plain \capstex\ control statements {\color{brown}\verbatim\L|endverbatim} or {\color{brown}\verbatim\l|endverbatim} do not work. Details of this \capstex\ macro are given in the table~below.
\bs
\hfil{Font assignment in {\color{brown}\verbatim font_kurier|endverbatim}~macro}\hfil

{\parindent=0pt\settabs4\columns\hfil\vbox{\hrule\hbox{\vrule\hbox{\vbox{\kern1pt\hrule\NoBlackBoxes		 \eightrm\fontss							 \+\hfil	\textcolor{blue}{ Typeface}	 &\strut\vrule\strut\hfil	\textcolor{blue}{ Font name}	 &\strut\vrule\kern1pt\vrule\strut\hfil	 \textcolor{blue}{ Typeface}	 &\strut\vrule\strut\hfil	 \textcolor{blue}{ Font name}	 &\cr	\hrule
\+\hfil	\eightrm Roman text	&\strut\vrule\strut\hfil	rm-kurierr	 &\strut\vrule\kern1pt\vrule\strut\hfil	 \eightbf Boldface text	 &\strut\vrule\strut\hfil	 rm-kurierb	 &\cr	\hrule
\+\hfil	\eighti Math italic	&\strut\vrule\strut\hfil	mi-kurierri	 &\strut\vrule\kern1pt\vrule\strut\hfil	 \eighttt Typewriter text	 &\strut\vrule\strut\hfil	 rm-inconsolata	&\cr	\hrule
\+\hfil	\eightrm Math symbols	&\strut\vrule\strut\hfil	sy-kurierrz	 &\strut\vrule\kern1pt\vrule\strut\hfil	 \eightitbf Italic boldface text	 &\strut\vrule\strut\hfil	 rm-kurierbi	&\cr	\hrule
\+\hfil	\eightrm Math extension	&\strut\vrule\strut\hfil	ex-kurierr	 &\strut\vrule\kern1pt\vrule\strut\hfil	 \eightslbf Slanted boldface text	 &\strut\vrule\strut\hfil	 rm-kurierbi	&\cr	\hrule
\+\hfil	\eightit Italic text	&\strut\vrule\strut\hfil	rm-kurierri	 &\strut\vrule\kern1pt\vrule\strut\hfil	 \eightcaps Caps	 &\strut\vrule\strut\hfil	 qx-kurierrcap	 &\cr	\hrule
\+\hfil	\eightsl Slanted text	&\strut\vrule\strut\hfil	rm-kurierri	 &\strut\vrule\kern1pt\vrule\strut\hfil	 \eightcapsbf Caps in Boldface	 &\strut\vrule\strut\hfil	 qx-kurierbcap	&\cr	\hrule
}\vrule}}\hrule}\hfil}								
									
\BlackBoxes								
									














\input font_kurier-light   \fontss
\section{\sixteenbf\fontss Kurier-{\sixteenslbf Light}}{Kurier-Light}
\sample
\ii The Kurier-{\sl Light\/} font is declared by typing {\color{brown}\verbatim\input font_kurier-light|endverbatim}. The font family uses light and medium weight Kurier fonts from J.\;M.\;Nowacki's \href{http://tug.ctan.org/cgi-bin/ctanPackageInformation.py?id=kurier}{kurier} package, which corresponds to Ma\lstroke{}gorzata Budyta's text fonts. The Kurier font is very similar to Iwona font; Kurier is a bit extended and has ink traps. L with stroke~(\Lstroke) is displayed by {\color{brown}\verbatim\Lstroke|endverbatim} and l with stroke~(\lstroke) is displayed by {\color{brown}\verbatim\lstroke|endverbatim}. When this macro is in use the default plain \capstex\ control statements {\color{brown}\verbatim\L|endverbatim} or {\color{brown}\verbatim\l|endverbatim} do not work. Details of this \capstex\ macro are given in the table~below.
\bs
\hfil{Font assignment in {\color{brown}\verbatim font_kurier-light|endverbatim}~macro}\hfil

{\parindent=0pt\settabs4\columns\hfil\vbox{\hrule\hbox{\vrule\hbox{\vbox{\kern1pt\hrule\NoBlackBoxes		 \eightrm\fontss							 
									
\+\hfil	\textcolor{blue}{ Typeface}	&\strut\vrule\strut\hfil	\textcolor{blue}{ Font name}	 &\strut\vrule\kern1pt\vrule\strut\hfil	 \textcolor{blue}{ Typeface}	 &\strut\vrule\strut\hfil	\textcolor{blue}{ Font name}	&\cr	\hrule
\+\hfil	\eightrm Roman text	&\strut\vrule\strut\hfil	rm-kurierl	 &\strut\vrule\kern1pt\vrule\strut\hfil	\eightbf Boldface text	 &\strut\vrule\strut\hfil	 rm-kurierm	&\cr	\hrule
\+\hfil	\eighti Math italic	&\strut\vrule\strut\hfil	mi-kurierli	 &\strut\vrule\kern1pt\vrule\strut\hfil	\eighttt Typewriter text	 &\strut\vrule\strut\hfil	 rm-inconsolata	&\cr	\hrule
\+\hfil	\eightrm Math symbols	&\strut\vrule\strut\hfil	sy-kurierlz	 &\strut\vrule\kern1pt\vrule\strut\hfil	\eightitbf Italic boldface text	 &\strut\vrule\strut\hfil	rm-kuriermi	&\cr	\hrule
\+\hfil	\eightrm Math extension	&\strut\vrule\strut\hfil	ex-kurierl	 &\strut\vrule\kern1pt\vrule\strut\hfil	\eightslbf Slanted boldface text	 &\strut\vrule\strut\hfil	rm-kuriermi	&\cr	\hrule
\+\hfil	\eightit Italic text	&\strut\vrule\strut\hfil	rm-kurierli	 &\strut\vrule\kern1pt\vrule\strut\hfil	\eightcaps Caps	 &\strut\vrule\strut\hfil	 qx-kurierlcap	&\cr	\hrule
\+\hfil	\eightsl Slanted text	&\strut\vrule\strut\hfil	rm-kurierli	 &\strut\vrule\kern1pt\vrule\strut\hfil	\eightcapsbf Caps in Boldface	 &\strut\vrule\strut\hfil	qx-kuriermcap	&\cr	\hrule
									
	}\vrule}}\hrule}\hfil}								
									
	\BlackBoxes								










\input font_kurier-medium   \fontss
\section{\sixteenbf\fontss Kurier-{\sixteenslbf Medium}}{Kurier-Medium}
\sample
\ii The Kurier-{\sl Medium\/} font is declared by typing {\color{brown}\verbatim\input font_kurier-medium|endverbatim}. The font family uses medium and heavy weight Kurier fonts from J.\;M.\;Nowacki's \href{http://tug.ctan.org/cgi-bin/ctanPackageInformation.py?id=kurier}{kurier} package, which corresponds to Ma\lstroke{}gorzata Budyta's text fonts. The Kurier font is very similar to Iwona font; Kurier is a bit extended and has ink traps. L with stroke~(\Lstroke) is displayed by {\color{brown}\verbatim\Lstroke|endverbatim} and l with stroke~(\lstroke) is displayed by {\color{brown}\verbatim\lstroke|endverbatim}. When this macro is in use the default plain \capstex\ control statements {\color{brown}\verbatim\L|endverbatim} or {\color{brown}\verbatim\l|endverbatim} do not work. Details of this \capstex\ macro are given in the table~below.
\bs
\hfil{Font assignment in {\color{brown}\verbatim font_kurier-medium|endverbatim}~macro}\hfil

{\parindent=0pt\settabs4\columns\hfil\vbox{\hrule\hbox{\vrule\hbox{\vbox{\kern1pt\hrule\NoBlackBoxes		 \eightrm\fontss							 
									
\+\hfil	\textcolor{blue}{ Typeface}	&\strut\vrule\strut\hfil	\textcolor{blue}{ Font name}	 &\strut\vrule\kern1pt\vrule\strut\hfil	 \textcolor{blue}{ Typeface}	 &\strut\vrule\strut\hfil	\textcolor{blue}{ Font name}	&\cr	\hrule
\+\hfil	\eightrm Roman text	&\strut\vrule\strut\hfil	rm-kurierm	 &\strut\vrule\kern1pt\vrule\strut\hfil	\eightbf Boldface text	 &\strut\vrule\strut\hfil	 rm-kurierh	&\cr	\hrule
\+\hfil	\eighti Math italic	&\strut\vrule\strut\hfil	mi-kuriermi	 &\strut\vrule\kern1pt\vrule\strut\hfil	\eighttt Typewriter text	 &\strut\vrule\strut\hfil	 rm-inconsolata	&\cr	\hrule
\+\hfil	\eightrm Math symbols	&\strut\vrule\strut\hfil	sy-kuriermz	 &\strut\vrule\kern1pt\vrule\strut\hfil	\eightitbf Italic boldface text	 &\strut\vrule\strut\hfil	rm-kurierhi	&\cr	\hrule
\+\hfil	\eightrm Math extension	&\strut\vrule\strut\hfil	ex-kurierm	 &\strut\vrule\kern1pt\vrule\strut\hfil	\eightslbf Slanted boldface text	 &\strut\vrule\strut\hfil	rm-kurierhi	&\cr	\hrule
\+\hfil	\eightit Italic text	&\strut\vrule\strut\hfil	rm-kuriermi	 &\strut\vrule\kern1pt\vrule\strut\hfil	\eightcaps Caps	 &\strut\vrule\strut\hfil	 qx-kuriermcap	&\cr	\hrule
\+\hfil	\eightsl Slanted text	&\strut\vrule\strut\hfil	rm-kuriermi	 &\strut\vrule\kern1pt\vrule\strut\hfil	\eightcapsbf Caps in Boldface	 &\strut\vrule\strut\hfil	qx-kurierhcap	&\cr	\hrule
									
	}\vrule}}\hrule}\hfil}								
									
	\BlackBoxes								






\input font_kurier-bold   \fontss
\section{\sixteenbf\fontss Kurier-{\sixteenslbf Bold}}{Kurier-Bold}
\sample
\ii The Kurier-{\sl Bold\/} font is declared by typing {\color{brown}\verbatim\input font_kurier-bold|endverbatim}. The font family uses bold and heavy weight Kurier fonts from J.\;M.\;Nowacki's \href{http://tug.ctan.org/cgi-bin/ctanPackageInformation.py?id=kurier}{kurier} package, which corresponds to Ma\lstroke{}gorzata Budyta's text fonts. The Kurier font is very similar to Iwona font; Kurier is a bit extended and has ink traps. L with stroke~(\Lstroke) is displayed by {\color{brown}\verbatim\Lstroke|endverbatim} and l with stroke~(\lstroke) is displayed by {\color{brown}\verbatim\lstroke|endverbatim}. When this macro is in use the default plain \capstex\ control statements {\color{brown}\verbatim\L|endverbatim} or {\color{brown}\verbatim\l|endverbatim} do not work. Details of this \capstex\ macro are given in the table~below.
\bs
\hfil{Font assignment in {\color{brown}\verbatim font_kurier-medium|endverbatim}~macro}\hfil

{\parindent=0pt\settabs4\columns\hfil\vbox{\hrule\hbox{\vrule\hbox{\vbox{\kern1pt\hrule\NoBlackBoxes		 \eightrm\fontss							 
									
\+\hfil	\textcolor{blue}{ Typeface}	&\strut\vrule\strut\hfil	\textcolor{blue}{ Font name}	 &\strut\vrule\kern1pt\vrule\strut\hfil	 \textcolor{blue}{ Typeface}	 &\strut\vrule\strut\hfil	\textcolor{blue}{ Font name}	&\cr	\hrule
\+\hfil	\eightrm Roman text	&\strut\vrule\strut\hfil	rm-kurierb	 &\strut\vrule\kern1pt\vrule\strut\hfil	\eightbf Boldface text	 &\strut\vrule\strut\hfil	 rm-kurierh	&\cr	\hrule
\+\hfil	\eighti Math italic	&\strut\vrule\strut\hfil	mi-kurierbi	 &\strut\vrule\kern1pt\vrule\strut\hfil	\eighttt Typewriter text	 &\strut\vrule\strut\hfil	 rm-inconsolata	&\cr	\hrule
\+\hfil	\eightrm Math symbols	&\strut\vrule\strut\hfil	sy-kurierbz	 &\strut\vrule\kern1pt\vrule\strut\hfil	\eightitbf Italic boldface text	 &\strut\vrule\strut\hfil	rm-kurierhi	&\cr	\hrule
\+\hfil	\eightrm Math extension	&\strut\vrule\strut\hfil	ex-kurierb	 &\strut\vrule\kern1pt\vrule\strut\hfil	\eightslbf Slanted boldface text	 &\strut\vrule\strut\hfil	rm-kurierhi	&\cr	\hrule
\+\hfil	\eightit Italic text	&\strut\vrule\strut\hfil	rm-kurierbi	 &\strut\vrule\kern1pt\vrule\strut\hfil	\eightcaps Caps	 &\strut\vrule\strut\hfil	 qx-kurierbcap	&\cr	\hrule
\+\hfil	\eightsl Slanted text	&\strut\vrule\strut\hfil	rm-kurierbi	 &\strut\vrule\kern1pt\vrule\strut\hfil	\eightcapsbf Caps in Boldface	 &\strut\vrule\strut\hfil	qx-kurierhcap	&\cr	\hrule
									
	}\vrule}}\hrule}\hfil}								
									
	\BlackBoxes								








\input font_kurier-condensed   \fontss
\section{\sixteenbf\fontss Kurier-{\sixteenslbf Condensed}}{Kurier-Condensed}
\sample
\ii The Kurier-{\sl Condensed\/} font is declared by typing {\color{brown}\verbatim\input font_kurier-condensed|endverbatim}. The font family uses condensed width regular and bold weight Kurier fonts from J.\;M.\;Nowacki's \href{http://tug.ctan.org/cgi-bin/ctanPackageInformation.py?id=kurier}{kurier} package, which corresponds to Ma\lstroke{}gorzata Budyta's text fonts. The Kurier font is very similar to Iwona font; Kurier is a bit extended and has ink traps. L with stroke~(\Lstroke) is displayed by {\color{brown}\verbatim\Lstroke|endverbatim} and l with stroke~(\lstroke) is displayed by {\color{brown}\verbatim\lstroke|endverbatim}. When this macro is in use the default plain \capstex\ control statements {\color{brown}\verbatim\L|endverbatim} or {\color{brown}\verbatim\l|endverbatim} do not work. Details of this \capstex\ macro are given in the table~below.
\bs
\hfil{Font assignment in {\color{brown}\verbatim font_kurier-condensed|endverbatim}~macro}\hfil

{\parindent=0pt\settabs4\columns\hfil\vbox{\hrule\hbox{\vrule\hbox{\vbox{\kern1pt\hrule\NoBlackBoxes		 \eightrm\fontss							 
									
\+\hfil	\textcolor{blue}{ Typeface}	&\strut\vrule\strut\hfil	\textcolor{blue}{ Font name}	 &\strut\vrule\kern1pt\vrule\strut\hfil	 \textcolor{blue}{ Typeface}	 &\strut\vrule\strut\hfil	\textcolor{blue}{ Font name}	&\cr	\hrule
\+\hfil	\eightrm Roman text	&\strut\vrule\strut\hfil	rm-kuriercr	 &\strut\vrule\kern1pt\vrule\strut\hfil	\eightbf Boldface text	 &\strut\vrule\strut\hfil	 rm-kuriercb	&\cr	\hrule
\+\hfil	\eighti Math italic	&\strut\vrule\strut\hfil	mi-kuriercri	 &\strut\vrule\kern1pt\vrule\strut\hfil	\eighttt Typewriter text	 &\strut\vrule\strut\hfil	 rm-inconsolata	&\cr	\hrule
\+\hfil	\eightrm Math symbols	&\strut\vrule\strut\hfil	sy-kuriercrz	 &\strut\vrule\kern1pt\vrule\strut\hfil	\eightitbf Italic boldface text	 &\strut\vrule\strut\hfil	rm-kuriercbi	&\cr	\hrule
\+\hfil	\eightrm Math extension	&\strut\vrule\strut\hfil	ex-kuriercr	 &\strut\vrule\kern1pt\vrule\strut\hfil	\eightslbf Slanted boldface text	 &\strut\vrule\strut\hfil	rm-kuriercbi	&\cr	\hrule
\+\hfil	\eightit Italic text	&\strut\vrule\strut\hfil	rm-kuriercri	 &\strut\vrule\kern1pt\vrule\strut\hfil	\eightcaps Caps	 &\strut\vrule\strut\hfil	 qx-kuriercrcap	&\cr	\hrule
\+\hfil	\eightsl Slanted text	&\strut\vrule\strut\hfil	rm-kuriercri	 &\strut\vrule\kern1pt\vrule\strut\hfil	\eightcapsbf Caps in Boldface	 &\strut\vrule\strut\hfil	qx-kuriercbcap	&\cr	\hrule
									
	}\vrule}}\hrule}\hfil}								
									
	\BlackBoxes								







\input font_kurier-condensed-light   \fontss
\section{\sixteenbf\fontss Kurier-{\sixteenslbf Condensed-Light}}{Kurier-Condensed-Light}
\sample
\ii The Kurier-{\sl Condensed-Light\/} font is declared by typing {\color{brown}\verbatim\input font_kurier-condensed-light|endverbatim}. The font family uses condensed width light and medium weight Kurier fonts from J.\;M.\;Nowacki's \href{http://tug.ctan.org/cgi-bin/ctanPackageInformation.py?id=kurier}{kurier} package, which corresponds to Ma\lstroke{}gorzata Budyta's text fonts. The Kurier font is very similar to Iwona font; Kurier is a bit extended and has ink traps. L with stroke~(\Lstroke) is displayed by {\color{brown}\verbatim\Lstroke|endverbatim} and l with stroke~(\lstroke) is displayed by {\color{brown}\verbatim\lstroke|endverbatim}. When this macro is in use the default plain \capstex\ control statements {\color{brown}\verbatim\L|endverbatim} or {\color{brown}\verbatim\l|endverbatim} do not work. Details of this \capstex\ macro are given in the table~below.
\bs
\hfil{Font assignment in {\color{brown}\verbatim font_kurier-condensed-light|endverbatim}~macro}\hfil

{\parindent=0pt\settabs4\columns\hfil\vbox{\hrule\hbox{\vrule\hbox{\vbox{\kern1pt\hrule\NoBlackBoxes		 \eightrm\fontss							 
									
\+\hfil	\textcolor{blue}{ Typeface}	&\strut\vrule\strut\hfil	\textcolor{blue}{ Font name}	 &\strut\vrule\kern1pt\vrule\strut\hfil	 \textcolor{blue}{ Typeface}	 &\strut\vrule\strut\hfil	\textcolor{blue}{ Font name}	&\cr	\hrule
\+\hfil	\eightrm Roman text	&\strut\vrule\strut\hfil	rm-kuriercl	 &\strut\vrule\kern1pt\vrule\strut\hfil	\eightbf Boldface text	 &\strut\vrule\strut\hfil	 rm-kuriercm	&\cr	\hrule
\+\hfil	\eighti Math italic	&\strut\vrule\strut\hfil	mi-kuriercli	 &\strut\vrule\kern1pt\vrule\strut\hfil	\eighttt Typewriter text	 &\strut\vrule\strut\hfil	 rm-inconsolata	&\cr	\hrule
\+\hfil	\eightrm Math symbols	&\strut\vrule\strut\hfil	sy-kurierclz	 &\strut\vrule\kern1pt\vrule\strut\hfil	\eightitbf Italic boldface text	 &\strut\vrule\strut\hfil	rm-kuriercmi	&\cr	\hrule
\+\hfil	\eightrm Math extension	&\strut\vrule\strut\hfil	ex-kuriercl	 &\strut\vrule\kern1pt\vrule\strut\hfil	\eightslbf Slanted boldface text	 &\strut\vrule\strut\hfil	rm-kuriercmi	&\cr	\hrule
\+\hfil	\eightit Italic text	&\strut\vrule\strut\hfil	rm-kuriercli	 &\strut\vrule\kern1pt\vrule\strut\hfil	\eightcaps Caps	 &\strut\vrule\strut\hfil	 qx-kurierclcap	&\cr	\hrule
\+\hfil	\eightsl Slanted text	&\strut\vrule\strut\hfil	rm-kuriercli	 &\strut\vrule\kern1pt\vrule\strut\hfil	\eightcapsbf Caps in Boldface	 &\strut\vrule\strut\hfil	qx-kuriercmcap	&\cr	\hrule
									
	}\vrule}}\hrule}\hfil}								
									
	\BlackBoxes								
									








\input font_kurier-condensed-medium  \fontss
\section{\sixteenbf\fontss Kurier-{\sixteenslbf Condensed-Medium}}{Kurier-Condensed-Medium}
\sample
\ii The Kurier-{\sl Condensed-Medium\/} font is declared by typing {\color{brown}\verbatim\input font_kurier-condensed-medium|endverbatim}. The font family uses condensed width medium and heavy weight Kurier fonts from J.\;M.\;Nowacki's \href{http://tug.ctan.org/cgi-bin/ctanPackageInformation.py?id=kurier}{kurier} package, which corresponds to Ma\lstroke{}gorzata Budyta's text fonts. The Kurier font is very similar to Iwona font; Kurier is a bit extended and has ink traps. L with stroke~(\Lstroke) is displayed by {\color{brown}\verbatim\Lstroke|endverbatim} and l with stroke~(\lstroke) is displayed by {\color{brown}\verbatim\lstroke|endverbatim}. When this macro is in use the default plain \capstex\ control statements {\color{brown}\verbatim\L|endverbatim} or {\color{brown}\verbatim\l|endverbatim} do not work. Details of this \capstex\ macro are given in the table~below.
\bs
\hfil{Font assignment in {\color{brown}\verbatim font_kurier-condensed-medium|endverbatim}~macro}\hfil

{\parindent=0pt\settabs4\columns\hfil\vbox{\hrule\hbox{\vrule\hbox{\vbox{\kern1pt\hrule\NoBlackBoxes		 \eightrm\fontss							 
									
\+\hfil	\textcolor{blue}{ Typeface}	&\strut\vrule\strut\hfil	\textcolor{blue}{ Font name}	 &\strut\vrule\kern1pt\vrule\strut\hfil	 \textcolor{blue}{ Typeface}	 &\strut\vrule\strut\hfil	\textcolor{blue}{ Font name}	&\cr	\hrule
\+\hfil	\eightrm Roman text	&\strut\vrule\strut\hfil	rm-kuriercm	 &\strut\vrule\kern1pt\vrule\strut\hfil	\eightbf Boldface text	 &\strut\vrule\strut\hfil	 rm-kurierch	&\cr	\hrule
\+\hfil	\eighti Math italic	&\strut\vrule\strut\hfil	mi-kuriercmi	 &\strut\vrule\kern1pt\vrule\strut\hfil	\eighttt Typewriter text	 &\strut\vrule\strut\hfil	 rm-inconsolata	&\cr	\hrule
\+\hfil	\eightrm Math symbols	&\strut\vrule\strut\hfil	sy-kuriercmz	 &\strut\vrule\kern1pt\vrule\strut\hfil	\eightitbf Italic boldface text	 &\strut\vrule\strut\hfil	rm-kurierchi	&\cr	\hrule
\+\hfil	\eightrm Math extension	&\strut\vrule\strut\hfil	ex-kuriercm	 &\strut\vrule\kern1pt\vrule\strut\hfil	\eightslbf Slanted boldface text	 &\strut\vrule\strut\hfil	rm-kurierchi	&\cr	\hrule
\+\hfil	\eightit Italic text	&\strut\vrule\strut\hfil	rm-kuriercmi	 &\strut\vrule\kern1pt\vrule\strut\hfil	\eightcaps Caps	 &\strut\vrule\strut\hfil	 qx-kuriercmcap	&\cr	\hrule
\+\hfil	\eightsl Slanted text	&\strut\vrule\strut\hfil	rm-kuriercmi	 &\strut\vrule\kern1pt\vrule\strut\hfil	\eightcapsbf Caps in Boldface	 &\strut\vrule\strut\hfil	qx-kurierchcap	&\cr	\hrule
									
	}\vrule}}\hrule}\hfil}								
									
	\BlackBoxes								







\input font_kurier-condensed-bold  \fontss
\section{\sixteenbf\fontss Kurier-{\sixteenslbf Condensed-Bold}}{Kurier-Condensed-Bold}
\sample
\ii The Kurier-{\sl Condensed-Bold\/} font is declared by typing {\color{brown}\verbatim\input font_kurier-condensed-bold|endverbatim}. The font family uses condensed width bold and heavy weight Kurier fonts from J.\;M.\;Nowacki's \href{http://tug.ctan.org/cgi-bin/ctanPackageInformation.py?id=kurier}{kurier} package, which corresponds to Ma\lstroke{}gorzata Budyta's text fonts. The Kurier font is very similar to Iwona font; Kurier is a bit extended and has ink traps. L with stroke~(\Lstroke) is displayed by {\color{brown}\verbatim\Lstroke|endverbatim} and l with stroke~(\lstroke) is displayed by {\color{brown}\verbatim\lstroke|endverbatim}. When this macro is in use the default plain \capstex\ control statements {\color{brown}\verbatim\L|endverbatim} or {\color{brown}\verbatim\l|endverbatim} do not work. Details of this \capstex\ macro are given in the table~below.
\bs
\hfil{Font assignment in {\color{brown}\verbatim font_kurier-condensed-bold|endverbatim}~macro}\hfil

{\parindent=0pt\settabs4\columns\hfil\vbox{\hrule\hbox{\vrule\hbox{\vbox{\kern1pt\hrule\NoBlackBoxes		 \eightrm\fontss							 
									
\+\hfil	\textcolor{blue}{ Typeface}	&\strut\vrule\strut\hfil	\textcolor{blue}{ Font name}	 &\strut\vrule\kern1pt\vrule\strut\hfil	 \textcolor{blue}{ Typeface}	 &\strut\vrule\strut\hfil	\textcolor{blue}{ Font name}	&\cr	\hrule
\+\hfil	\eightrm Roman text	&\strut\vrule\strut\hfil	rm-kuriercb	 &\strut\vrule\kern1pt\vrule\strut\hfil	\eightbf Boldface text	 &\strut\vrule\strut\hfil	 rm-kurierch	&\cr	\hrule
\+\hfil	\eighti Math italic	&\strut\vrule\strut\hfil	mi-kuriercbi	 &\strut\vrule\kern1pt\vrule\strut\hfil	\eighttt Typewriter text	 &\strut\vrule\strut\hfil	 rm-inconsolata	&\cr	\hrule
\+\hfil	\eightrm Math symbols	&\strut\vrule\strut\hfil	sy-kuriercbz	 &\strut\vrule\kern1pt\vrule\strut\hfil	\eightitbf Italic boldface text	 &\strut\vrule\strut\hfil	rm-kurierchi	&\cr	\hrule
\+\hfil	\eightrm Math extension	&\strut\vrule\strut\hfil	ex-kuriercb	 &\strut\vrule\kern1pt\vrule\strut\hfil	\eightslbf Slanted boldface text	 &\strut\vrule\strut\hfil	rm-kurierchi	&\cr	\hrule
\+\hfil	\eightit Italic text	&\strut\vrule\strut\hfil	rm-kuriercbi	 &\strut\vrule\kern1pt\vrule\strut\hfil	\eightcaps Caps	 &\strut\vrule\strut\hfil	 qx-kuriercbcap	&\cr	\hrule
\+\hfil	\eightsl Slanted text	&\strut\vrule\strut\hfil	rm-kuriercbi	 &\strut\vrule\kern1pt\vrule\strut\hfil	\eightcapsbf Caps in Boldface	 &\strut\vrule\strut\hfil	qx-kurierchcap	&\cr	\hrule
									
	}\vrule}}\hrule}\hfil}								
									
	\BlackBoxes								
















\input font_arev   \fontss
\section{\sixteenbf\fontss Arev}{Arev}
\sample
\ii The Arev font is declared by typing {\color{brown}\verbatim\input font_arev|endverbatim}. The font family uses fonts from S.\;G.\; Hartke's \href{http://www.ctan.org/tex-archive/help/Catalogue/entries/arev.html}{arev} package, which corresponds to \href{http://www.bitstream.com/font_rendering/products/dev_fonts/vera.html}{Bitstream Vera Sans} text fonts. \href{http://www.gnome.org/fonts/}{Bitstream Vera} font was designed by Jim Lyles. Details of this {\eightrm\TeX} macro are given in the table~below.
\bs
\hfil{Font assignment in {\color{brown}\verbatim font_arev|endverbatim}~macro}\hfil

{\parindent=0pt\settabs4\columns\hfil\vbox{\hrule\hbox{\vrule\hbox{\vbox{\kern1pt\hrule\NoBlackBoxes		 \eightrm\fontss							 \+\hfil	\textcolor{blue}{ Typeface}	 &\strut\vrule\strut\hfil	\textcolor{blue}{ Font name}	 &\strut\vrule\kern1pt\vrule\strut\hfil	 \textcolor{blue}{ Typeface}	 &\strut\vrule\strut\hfil	 \textcolor{blue}{ Font name}	 &\cr	\hrule
\+\hfil	\eightrm Roman text	&\strut\vrule\strut\hfil	zavmr7t	 &\strut\vrule\kern1pt\vrule\strut\hfil	 \eightbf Boldface text	 &\strut\vrule\strut\hfil	 zavmb7t	&\cr	\hrule
\+\hfil	\eighti Math italic	&\strut\vrule\strut\hfil	zavmri7m	 &\strut\vrule\kern1pt\vrule\strut\hfil	 \eighttt Typewriter text	 &\strut\vrule\strut\hfil	 fvmr8t	&\cr	\hrule
\+\hfil	\eightrm Math symbols	&\strut\vrule\strut\hfil	zavmr7y	 &\strut\vrule\kern1pt\vrule\strut\hfil	 \eightitbf Italic boldface text	 &\strut\vrule\strut\hfil	 favbi8t	&\cr	\hrule
\+\hfil	\eightrm Math extension	&\strut\vrule\strut\hfil	ex-kurierr	 &\strut\vrule\kern1pt\vrule\strut\hfil	 \eightslbf Slanted boldface text	 &\strut\vrule\strut\hfil	 favbi8t	&\cr	\hrule
\+\hfil	\eightit Italic text	&\strut\vrule\strut\hfil	favri8t	 &\strut\vrule\kern1pt\vrule\strut\hfil	No caps	 &\strut\vrule\strut\hfil	 \emdash  	 &\cr	 \hrule
\+\hfil	\eightsl Slanted text	&\strut\vrule\strut\hfil	favri8t	 &\strut\vrule\kern1pt\vrule\strut\hfil	No caps in bold	 &\strut\vrule\strut\hfil	 \emdash  	 &\cr	 \hrule
}\vrule}}\hrule}\hfil}								
									
\BlackBoxes								
						
									
				
							
									
		



\input font_cmbright   \fontss
\UseAMSsymbols
\section{\sixteenbf\fontss Computer Modern Bright}{Computer Modern Bright}
\sample
\ii The Computer Modern Bright font is declared by typing {\color{brown}\verbatim\input font_cmbright|endverbatim}. The font family uses fonts from Walter Schmidt's \href{http://www.ctan.org/tex-archive/help/Catalogue/entries/cmbright.html}{cmbright} package, which corresponds to Donald Knuth's Computer Modern Sans Serif text fonts. Computer Modern Bright fonts are lighter than Knuth's Computer Modern Sans Serif fonts. The fonts of this macro provide their own {\eightrm AMS} symbols. Details of this {\eightrm\TeX} macro are given in the table~below.
\bs
\hfil{Font assignment in {\color{brown}\verbatim font_cmbright|endverbatim}~macro}\hfil

{\parindent=0pt\settabs4\columns\hfil\vbox{\hrule\hbox{\vrule\hbox{\vbox{\kern1pt\hrule\NoBlackBoxes		 \eightrm\fontss							 \+\hfil	\textcolor{blue}{ Typeface}	 &\strut\vrule\strut\hfil	\textcolor{blue}{ Font name}	 &\strut\vrule\kern1pt\vrule\strut\hfil	 \textcolor{blue}{ Typeface}	 &\strut\vrule\strut\hfil	 \textcolor{blue}{ Font name}	 &\cr	\hrule
\+\hfil	\eightrm Roman text	&\strut\vrule\strut\hfil	cmbr10	 &\strut\vrule\kern1pt\vrule\strut\hfil	 \eightbf Boldface text	 &\strut\vrule\strut\hfil	 cmbrbx10	&\cr	\hrule
\+\hfil	\eighti Math italic	&\strut\vrule\strut\hfil	cmbrmi10	 &\strut\vrule\kern1pt\vrule\strut\hfil	 \eighttt Typewriter text	 &\strut\vrule\strut\hfil	 rm-inconsolata	&\cr	\hrule
\+\hfil	\eightrm Math symbols	&\strut\vrule\strut\hfil	cmbrsy10	 &\strut\vrule\kern1pt\vrule\strut\hfil	 \eightitbf Italic boldface text	 &\strut\vrule\strut\hfil	 rm-lmssbo10	&\cr	\hrule
\+\hfil	\eightrm Math extension	&\strut\vrule\strut\hfil	ex-kurierr	 &\strut\vrule\kern1pt\vrule\strut\hfil	 \eightslbf Slanted boldface text	 &\strut\vrule\strut\hfil	 rm-lmssbo10	&\cr	\hrule
\+\hfil	\eightit Italic text	&\strut\vrule\strut\hfil	cmbrsl10	 &\strut\vrule\kern1pt\vrule\strut\hfil	 No caps	 &\strut\vrule\strut\hfil	 \emdash  	 &\cr	\hrule
\+\hfil	\eightsl Slanted text	&\strut\vrule\strut\hfil	cmbrsl10	 &\strut\vrule\kern1pt\vrule\strut\hfil	 No caps in bold	 &\strut\vrule\strut\hfil	 \emdash  	&\cr	 \hrule
}\vrule}}\hrule}\hfil}								
									
\BlackBoxes								

\bs\ii Matching {\eightrm AMS} symbols: \circledR \ \yen \ $\blacksquare \ \approxeq \ \eqslantgtr \ \curlyeqprec \ \curlyeqsucc \ \preccurlyeq \ \leqq \ \leqslant \ \lessgtr \ \nless \ \nleq \ \nleqslant \ \Bbb R \ \Bbb E \ \Bbb C \ \dots$












\input font_epigrafica_euler   \fontss
\section{\sixteenbf\fontss Epigrafica with Euler}{Epigrafica with Euler}
\sample
\ii This macro enables us to type text in Epigrafica font and math in Euler font. The macro is declared by typing {\color{brown}\verbatim\input font_epigrafica_euler|endverbatim}. The macro typesets text in fonts from Antonis Tsolomitis's \href{http://www.ctan.org/tex-archive/help/Catalogue/entries/epigrafica.html}{epigrafica} package (based on Hermann Zapf's \href{http://new.myfonts.com/fonts/adobe/optima/}{Optima} text font) and math in Walter Schmidt's \href{http://www.ctan.org/tex-archive/help/Catalogue/entries/eulervm.html}{Euler-VM} fonts (based on Hermann Zapf's Euler and Knuth's CM fonts). Details of this \capstex\ macro are given in the table~below.
\bs
\hfil{Font assignment in {\color{brown}\verbatim font_epigrafica_euler|endverbatim}~macro}\hfil

{\parindent=0pt\settabs4\columns\hfil\vbox{\hrule\hbox{\vrule\hbox{\vbox{\kern1pt\hrule\NoBlackBoxes		 \eightrm\fontss							 \+\hfil	\textcolor{blue}{ Typeface}	 &\strut\vrule\strut\hfil	\textcolor{blue}{ Font name}	 &\strut\vrule\kern1pt\vrule\strut\hfil	 \textcolor{blue}{ Typeface}	 &\strut\vrule\strut\hfil	 \textcolor{blue}{ Font name}	 &\cr	\hrule
\+\hfil	\eightrm Roman text	&\strut\vrule\strut\hfil	epigrafican8r	 &\strut\vrule\kern1pt\vrule\strut\hfil	 \eightbf Boldface text	 &\strut\vrule\strut\hfil	 epigraficab8r	&\cr	\hrule
\+\hfil	\eighti Math italic	&\strut\vrule\strut\hfil	eurm10	 &\strut\vrule\kern1pt\vrule\strut\hfil	 \eighttt Typewriter text	 &\strut\vrule\strut\hfil	 rm-inconsolata	&\cr	\hrule
\+\hfil	\eightrm Math symbols	&\strut\vrule\strut\hfil	cmsy10	 &\strut\vrule\kern1pt\vrule\strut\hfil	\eightitbf Italic boldface text	 &\strut\vrule\strut\hfil	 epigraficabi8r	&\cr	\hrule
\+\hfil	\eightrm Math extension	&\strut\vrule\strut\hfil	euex10	 &\strut\vrule\kern1pt\vrule\strut\hfil	\eightslbf Slanted boldface text	 &\strut\vrule\strut\hfil	 epigraficabi8r	&\cr	\hrule
\+\hfil	\eightit Italic text	&\strut\vrule\strut\hfil	epigraficai8r	 &\strut\vrule\kern1pt\vrule\strut\hfil	 \eightcaps Caps	 &\strut\vrule\strut\hfil	 epigraficac8r	 &\cr	\hrule
\+\hfil	\eightsl Slanted text	&\strut\vrule\strut\hfil	epigraficai8r	 &\strut\vrule\kern1pt\vrule\strut\hfil	No caps in bold	 &\strut\vrule\strut\hfil	\emdash  	 &\cr	 \hrule
}\vrule}}\hrule}\hfil}								
									
\BlackBoxes								
									








\input font_epigrafica_palatino   \fontss
\section{\sixteenbf\fontss Epigrafica with Palatino}{Epigrafica with Palatino}
\sample
\ii This macro enables us to type text in Epigrafica font and math in PX~Fonts. The macro is declared by typing {\color{brown}\verbatim\input font_epigrafica_palatino|endverbatim}. The macro typesets text in fonts from Antonis Tsolomitis's \href{http://www.ctan.org/tex-archive/help/Catalogue/entries/epigrafica.html}{epigrafica} package (based on Hermann Zapf's \href{http://new.myfonts.com/fonts/adobe/optima/}{Optima} text font) and math in Young Ryu's \href{http://www.ctan.org/tex-archive/help/Catalogue/entries/pxfonts.html}{pxfonts} package (which corresponds to \href{http://www.adobe.com/type/browser/html/readmes/PalatinoStdReadMe.html#A2}
{Adobe Palatino} text fonts). Details of this \capstex\ macro are given in the table~below.
\bs
\hfil{Font assignment in {\color{brown}\verbatim font_epigrafica_palatino|endverbatim}~macro}\hfil

{\parindent=0pt\settabs4\columns\hfil\vbox{\hrule\hbox{\vrule\hbox{\vbox{\kern1pt\hrule\NoBlackBoxes		 \eightrm\fontss							 \+\hfil	\textcolor{blue}{ Typeface}	 &\strut\vrule\strut\hfil	\textcolor{blue}{ Font name}	 &\strut\vrule\kern1pt\vrule\strut\hfil	 \textcolor{blue}{ Typeface}	 &\strut\vrule\strut\hfil	 \textcolor{blue}{ Font name}	 &\cr	\hrule
\+\hfil	\eightrm Roman text	&\strut\vrule\strut\hfil	epigrafican8r	 &\strut\vrule\kern1pt\vrule\strut\hfil	 \eightbf Boldface text	 &\strut\vrule\strut\hfil	 epigraficab8r	&\cr	\hrule
\+\hfil	\eighti Math italic	&\strut\vrule\strut\hfil	pxmi	 &\strut\vrule\kern1pt\vrule\strut\hfil	 \eighttt Typewriter text	 &\strut\vrule\strut\hfil	 rm-inconsolata	&\cr	\hrule
\+\hfil	\eightrm Math symbols	&\strut\vrule\strut\hfil	pxsy	 &\strut\vrule\kern1pt\vrule\strut\hfil	\eightitbf Italic boldface text	 &\strut\vrule\strut\hfil	 epigraficabi8r	&\cr	\hrule
\+\hfil	\eightrm Math extension	&\strut\vrule\strut\hfil	pxex	 &\strut\vrule\kern1pt\vrule\strut\hfil	\eightslbf Slanted boldface text	 &\strut\vrule\strut\hfil	 epigraficabi8r	&\cr	\hrule
\+\hfil	\eightit Italic text	&\strut\vrule\strut\hfil	epigraficai8r	 &\strut\vrule\kern1pt\vrule\strut\hfil	 \eightcaps Caps	 &\strut\vrule\strut\hfil	 epigraficac8r	 &\cr	\hrule
\+\hfil	\eightsl Slanted text	&\strut\vrule\strut\hfil	epigraficai8r	 &\strut\vrule\kern1pt\vrule\strut\hfil	No caps in bold	 &\strut\vrule\strut\hfil	\emdash  	 &\cr	 \hrule
}\vrule}}\hrule}\hfil}								
									
\BlackBoxes								
									








\input font_antp_euler   \fontss
\section{\sixteenbf\fontss Antykwa P\'o\char'252tawskiego with Euler}{Antykwa Poltawskiego with Euler}
\sample
\ii This macro enables us to type text in Antykwa P\'o\lstroke{}tawskiego font and math in Euler font. The macro is declared by typing {\color{brown}\verbatim\input font_antp_euler|endverbatim}. The macro typesets text in fonts from J.\;M.\;No\-wacki's \href{http://www.ctan.org/tex-archive/help/Catalogue/entries/antp.html}{antp} package (based on Polish typographer, Adam P\'o\lstroke{}tawski's \href{http://nowacki.strefa.pl/poltawski-e.html}{Antykwa P\'o\lstroke{}tawskiego} text fonts) and math in Walter Schmidt's \href{http://www.ctan.org/tex-archive/help/Catalogue/entries/eulervm.html}{Euler-VM} fonts (based on Hermann Zapf's Euler and Knuth's CM fonts). L with stroke~(\Lstroke) is displayed by {\color{brown}\verbatim\Lstroke|endverbatim} and l with stroke~(\lstroke) is displayed by {\color{brown}\verbatim\lstroke|endverbatim}. When this macro is in use the default {\eightrm \TeX} control statement {\color{brown}\verbatim\L|endverbatim} or {\color{brown}\verbatim\l|endverbatim} do not work. Details of this {\eightrm \TeX} macro are given in the table~below.
\bs
\hfil{Font assignment in {\color{brown}\verbatim font_antp_euler|endverbatim}~macro}\hfil

{\parindent=0pt\settabs4\columns\hfil\vbox{\hrule\hbox{\vrule\hbox{\vbox{\kern1pt\hrule\NoBlackBoxes		 \eightrm\fontss							 \+\hfil	\textcolor{blue}{ Typeface}	 &\strut\vrule\strut\hfil	\textcolor{blue}{ Font name}	 &\strut\vrule\kern1pt\vrule\strut\hfil	 \textcolor{blue}{ Typeface}	 &\strut\vrule\strut\hfil	\textcolor{blue}{ Font name}	 &\cr	\hrule
\+\hfil	\eightrm Roman text	&\strut\vrule\strut\hfil	antpr	 &\strut\vrule\kern1pt\vrule\strut\hfil	\eightbf Boldface text	 &\strut\vrule\strut\hfil	 antpb	&\cr	\hrule
\+\hfil	\eighti Math italic	&\strut\vrule\strut\hfil	eurm10	 &\strut\vrule\kern1pt\vrule\strut\hfil	\eighttt Typewriter text	 &\strut\vrule\strut\hfil	 rm-inconsolata	&\cr	\hrule
\+\hfil	\eightrm Math symbols	&\strut\vrule\strut\hfil	cmsy10	 &\strut\vrule\kern1pt\vrule\strut\hfil	\eightitbf Italic boldface text	 &\strut\vrule\strut\hfil	antpbi	&\cr	\hrule
\+\hfil	\eightrm Math extension	&\strut\vrule\strut\hfil	euex10	 &\strut\vrule\kern1pt\vrule\strut\hfil	\eightslbf Slanted boldface text	 &\strut\vrule\strut\hfil	antpbi	&\cr	\hrule
\+\hfil	\eightit Italic text	&\strut\vrule\strut\hfil	antpri	 &\strut\vrule\kern1pt\vrule\strut\hfil	No caps	 &\strut\vrule\strut\hfil	 \emdash  	 &\cr	 \hrule
\+\hfil	\eightsl Slanted text	&\strut\vrule\strut\hfil	antpri	 &\strut\vrule\kern1pt\vrule\strut\hfil	No caps in bold	 &\strut\vrule\strut\hfil	 \emdash  	 &\cr	\hrule
}\vrule}}\hrule}\hfil}								
									
\BlackBoxes								









\input font_bera_concrete   \fontss
\section{\sixteenbf\fontss Bera Serif with Concrete}{Bera Serif with Concrete}
\sample
\ii This macro enables us to type text in Bera serif and math in Concrete. The macro is declared by typing {\color{brown}\verbatim\input font_bera_concrete|endverbatim}. The macro typesets text in Bera serif fonts from Walter Schmidt's \href{http://www.ctan.org/tex-archive/help/Catalogue/entries/bera.html}{bera} package (based on \href{http://www.urbanfonts.com/fonts/Bitstream_Vera.htm}{Bitstream Vera serif} font designed by Jim Lyles of Bitstream Inc.) and math is typeset using in Jackowski, Ry\char'242ko and Bzyl's \href{http://www.ctan.org/tex-archive/help/Catalogue/entries/cc-pl.html}{cc-pl} package (based on Knuth's \href{http://www.ctan.org/tex-archive/help/Catalogue/entries/concrete.html}{Concrete Roman} fonts). Details of this {\eightrm \TeX} macro are given in the table~below.
\bs
\hfil{Font assignment in {\color{brown}\verbatim font_bera_concrete|endverbatim}~macro}\hfil

{\parindent=0pt\settabs4\columns\hfil\vbox{\hrule\hbox{\vrule\hbox{\vbox{\kern1pt\hrule\NoBlackBoxes		 \eightrm\fontss							 \+\hfil	\textcolor{blue}{ Typeface}	 &\strut\vrule\strut\hfil	\textcolor{blue}{ Font name}	 &\strut\vrule\kern1pt\vrule\strut\hfil	 \textcolor{blue}{ Typeface}	 &\strut\vrule\strut\hfil	\textcolor{blue}{ Font name}	 &\cr	\hrule
\+\hfil	\eightrm Roman text	&\strut\vrule\strut\hfil	fver8t	 &\strut\vrule\kern1pt\vrule\strut\hfil	\eightbf Boldface text	 &\strut\vrule\strut\hfil	 fveb8t	&\cr	\hrule
\+\hfil	\eighti Math italic	&\strut\vrule\strut\hfil	pcmi10	 &\strut\vrule\kern1pt\vrule\strut\hfil	\eighttt Typewriter text	 &\strut\vrule\strut\hfil	 fvmr8t	&\cr	\hrule
\+\hfil	\eightrm Math symbols	&\strut\vrule\strut\hfil	cmsy10	 &\strut\vrule\kern1pt\vrule\strut\hfil	\eightitbf Italic boldface text	 &\strut\vrule\strut\hfil	fvebo8t	&\cr	\hrule
\+\hfil	\eightrm Math extension	&\strut\vrule\strut\hfil	cmex10	 &\strut\vrule\kern1pt\vrule\strut\hfil	\eightslbf Slanted boldface text	 &\strut\vrule\strut\hfil	fvebo8t	&\cr	\hrule
\+\hfil	\eightit Italic text	&\strut\vrule\strut\hfil	fvero8t	 &\strut\vrule\kern1pt\vrule\strut\hfil	No caps	 &\strut\vrule\strut\hfil	 \emdash  	 &\cr	 \hrule
\+\hfil	\eightsl Slanted text	&\strut\vrule\strut\hfil	fvero8t	 &\strut\vrule\kern1pt\vrule\strut\hfil	No caps in bold	 &\strut\vrule\strut\hfil	 \emdash  	 &\cr	\hrule
}\vrule}}\hrule}\hfil}								
									
\BlackBoxes								









\input font_bera_euler   \fontss
\section{\sixteenbf\fontss Bera Serif with Euler}{Bera Serif with Euler}
\sample
\ii This macro enables us to type text in Bera serif and math in Euler. The macro is declared by typing {\color{brown}\verbatim\input font_bera_euler|endverbatim}. The macro typesets text in Bera serif fonts from Walter Schmidt's \href{http://www.ctan.org/tex-archive/help/Catalogue/entries/bera.html}{bera} package (based on \href{http://www.urbanfonts.com/fonts/Bitstream_Vera.htm}{Bitstream Vera serif} font designed by Jim Lyles of Bitstream Inc.) and math in Walter Schmidt's \href{http://www.ctan.org/tex-archive/help/Catalogue/entries/eulervm.html}{Euler-VM} fonts (based on Hermann Zapf's Euler and Knuth's CM fonts). Details of this {\eightrm \TeX} macro are given in the table~below.
\bs
\hfil{Font assignment in {\color{brown}\verbatim font_bera_euler|endverbatim}~macro}\hfil

{\parindent=0pt\settabs4\columns\hfil\vbox{\hrule\hbox{\vrule\hbox{\vbox{\kern1pt\hrule\NoBlackBoxes		 \eightrm\fontss							 \+\hfil	\textcolor{blue}{ Typeface}	 &\strut\vrule\strut\hfil	\textcolor{blue}{ Font name}	 &\strut\vrule\kern1pt\vrule\strut\hfil	 \textcolor{blue}{ Typeface}	 &\strut\vrule\strut\hfil	\textcolor{blue}{ Font name}	 &\cr	\hrule
\+\hfil	\eightrm Roman text	&\strut\vrule\strut\hfil	fver8t	 &\strut\vrule\kern1pt\vrule\strut\hfil	\eightbf Boldface text	 &\strut\vrule\strut\hfil	 fveb8t	&\cr	\hrule
\+\hfil	\eighti Math italic	&\strut\vrule\strut\hfil	eurm10	 &\strut\vrule\kern1pt\vrule\strut\hfil	\eighttt Typewriter text	 &\strut\vrule\strut\hfil	 fvmr8t	&\cr	\hrule
\+\hfil	\eightrm Math symbols	&\strut\vrule\strut\hfil	cmsy10	 &\strut\vrule\kern1pt\vrule\strut\hfil	\eightitbf Italic boldface text	 &\strut\vrule\strut\hfil	fvebo8t	&\cr	\hrule
\+\hfil	\eightrm Math extension	&\strut\vrule\strut\hfil	euex10	 &\strut\vrule\kern1pt\vrule\strut\hfil	\eightslbf Slanted boldface text	 &\strut\vrule\strut\hfil	fvebo8t	&\cr	\hrule
\+\hfil	\eightit Italic text	&\strut\vrule\strut\hfil	fvero8t	 &\strut\vrule\kern1pt\vrule\strut\hfil	No caps	 &\strut\vrule\strut\hfil	 \emdash  	 &\cr	 \hrule
\+\hfil	\eightsl Slanted text	&\strut\vrule\strut\hfil	fvero8t	 &\strut\vrule\kern1pt\vrule\strut\hfil	No caps in bold	 &\strut\vrule\strut\hfil	 \emdash  	 &\cr	\hrule
}\vrule}}\hrule}\hfil}								
									
\BlackBoxes								









\input font_bera_fnc   \fontss
\section{\sixteenbf\fontss Bera Serif with Fouriernc}{Bera Serif with Fouriernc}
\sample
\ii This macro enables us to type text in Bera serif and math in Fouriernc (originally used with New Century). The macro is declared by typing {\color{brown}\verbatim\input font_bera_fnc|endverbatim}. The macro typesets text in Bera serif fonts from Walter Schmidt's \href{http://www.ctan.org/tex-archive/help/Catalogue/entries/bera.html}{bera} package (based on \href{http://www.urbanfonts.com/fonts/Bitstream_Vera.htm}{Bitstream Vera serif} font designed by Jim Lyles of Bitstream Inc.) and math using in Michael Zedler's \href{http://www.ctan.org/tex-archive/help/Catalogue/entries/fouriernc.html}{fouriernc} package. Details of this {\eightrm \TeX} macro are given in the table~below.
\bs
\hfil{Font assignment in {\color{brown}\verbatim font_bera_fnc|endverbatim}~macro}\hfil

{\parindent=0pt\settabs4\columns\hfil\vbox{\hrule\hbox{\vrule\hbox{\vbox{\kern1pt\hrule\NoBlackBoxes		 \eightrm\fontss							 \+\hfil	\textcolor{blue}{ Typeface}	 &\strut\vrule\strut\hfil	\textcolor{blue}{ Font name}	 &\strut\vrule\kern1pt\vrule\strut\hfil	 \textcolor{blue}{ Typeface}	 &\strut\vrule\strut\hfil	\textcolor{blue}{ Font name}	 &\cr	\hrule
\+\hfil	\eightrm Roman text	&\strut\vrule\strut\hfil	fver8t	 &\strut\vrule\kern1pt\vrule\strut\hfil	\eightbf Boldface text	 &\strut\vrule\strut\hfil	 fveb8t	&\cr	\hrule
\+\hfil	\eighti Math italic	&\strut\vrule\strut\hfil	fncmii	 &\strut\vrule\kern1pt\vrule\strut\hfil	\eighttt Typewriter text	 &\strut\vrule\strut\hfil	 fvmr8t	&\cr	\hrule
\+\hfil	\eightrm Math symbols	&\strut\vrule\strut\hfil	fncsy	 &\strut\vrule\kern1pt\vrule\strut\hfil	\eightitbf Italic boldface text	 &\strut\vrule\strut\hfil	fvebo8t	&\cr	\hrule
\+\hfil	\eightrm Math extension	&\strut\vrule\strut\hfil	cmex10	 &\strut\vrule\kern1pt\vrule\strut\hfil	\eightslbf Slanted boldface text	 &\strut\vrule\strut\hfil	fvebo8t	&\cr	\hrule
\+\hfil	\eightit Italic text	&\strut\vrule\strut\hfil	fvero8t	 &\strut\vrule\kern1pt\vrule\strut\hfil	No caps	 &\strut\vrule\strut\hfil	 \emdash  	 &\cr	 \hrule
\+\hfil	\eightsl Slanted text	&\strut\vrule\strut\hfil	fvero8t	 &\strut\vrule\kern1pt\vrule\strut\hfil	No caps in bold	 &\strut\vrule\strut\hfil	 \emdash  	 &\cr	\hrule
}\vrule}}\hrule}\hfil}								
									
\BlackBoxes								

















\input font_artemisia_euler   \fontss
\section{\sixteenbf\fontss Artemisia with Euler}{Artemisia with Euler}
\sample
\ii This macro enables us to type text in GFS Artemisia and math in Euler. The macro is declared by typing {\color{brown}\verbatim\input font_artemisia_euler|endverbatim}. The macro typesets text in Antonis Tsolomitis, George D.\ Matthiopoulos and The Greek Font Society's \href{http://www.ctan.org/tex-archive/help/Catalogue/entries/gfsartemisia.html}{GFS Artemisia fonts} and math in Walter\break Schmidt's \href{http://www.ctan.org/tex-archive/help/Catalogue/entries/eulervm.html}{Euler-VM} fonts (based on Hermann Zapf's Euler and Knuth's CM fonts). Details of this {\eightrm \TeX} macro are given in the table~below.
\bs
\hfil{Font assignment in {\color{brown}\verbatim font_artemisia_euler|endverbatim}~macro}\hfil

{\parindent=0pt\settabs4\columns\hfil\vbox{\hrule\hbox{\vrule\hbox{\vbox{\kern1pt\hrule\NoBlackBoxes		 \eightrm\fontss							 
									
\+\hfil	\textcolor{blue}{ Typeface}	&\strut\vrule\strut\hfil	\textcolor{blue}{ Font name}	 &\strut\vrule\kern1pt\vrule\strut\hfil	 \textcolor{blue}{ Typeface}	 &\strut\vrule\strut\hfil	\textcolor{blue}{ Font name}	&\cr	\hrule
\+\hfil	\eightrm Roman text	&\strut\vrule\strut\hfil	artemisiarg8a	 &\strut\vrule\kern1pt\vrule\strut\hfil	\eightbf Boldface text	 &\strut\vrule\strut\hfil	 artemisiab8a	&\cr	\hrule
\+\hfil	\eighti Math italic	&\strut\vrule\strut\hfil	zeurm10	 &\strut\vrule\kern1pt\vrule\strut\hfil	\eighttt Typewriter text	 &\strut\vrule\strut\hfil	 rm-inconsolata	&\cr	\hrule
\+\hfil	\eightrm Math symbols	&\strut\vrule\strut\hfil	zeusm10	 &\strut\vrule\kern1pt\vrule\strut\hfil	\eightitbf Italic boldface text	 &\strut\vrule\strut\hfil	artemisiabi8a	&\cr	\hrule
\+\hfil	\eightrm Math extension	&\strut\vrule\strut\hfil	zeuex10	 &\strut\vrule\kern1pt\vrule\strut\hfil	\eightslbf Slanted boldface text	 &\strut\vrule\strut\hfil	artemisiabo8a	&\cr	\hrule
\+\hfil	\eightit Italic text	&\strut\vrule\strut\hfil	artemisiai8a	 &\strut\vrule\kern1pt\vrule\strut\hfil	\eightcaps Caps &\strut\vrule\strut\hfil	 artemisiasc8a	&\cr	\hrule
\+\hfil	\eightsl Slanted text	&\strut\vrule\strut\hfil	artemisiao8a	 &\strut\vrule\kern1pt\vrule\strut\hfil	No caps in bold	 &\strut\vrule\strut\hfil	\emdash  	 &\cr	\hrule
									
	}\vrule}}\hrule}\hfil}								
									
	\BlackBoxes								
								











\input font_libertine_kp   \fontss
\UseAMSsymbols
\section{\sixteenbf\fontss Libertine with Kp-Fonts}{Libertine with Kp-Fonts}
\sample
\ii This macro enables us to type text in Linux-Libertine and math in Kp-Fonts. The macro is declared by typing {\color{brown}\verbatim\input font_libertine_kp|endverbatim}. The macro typesets text in Michael Niedermair's  \href{http://www.tex.ac.uk/tex-archive/help/Catalogue/entries/libertine.html}{Linux-Libertine} font and math in Chris\-tophe Caignaert's \href{http://www.ctan.org/tex-archive/help/Catalogue/entries/kpfonts.html}{Kp-Fonts}. The fonts of this macro provide their own {\caps ams} symbols. Details of this {\eightrm \TeX} macro are given in the table~below.
\bs
\hfil{Font assignment in {\color{brown}\verbatim font_libertine_kp|endverbatim}~macro}\hfil
					
{\parindent=0pt\settabs4\columns\hfil\vbox{\hrule\hbox{\vrule\hbox{\vbox{\kern1pt\hrule\NoBlackBoxes		 \eightrm\fontss							 
									
\+\hfil	\textcolor{blue}{ Typeface}	&\strut\vrule\strut\hfil	\textcolor{blue}{ Font name}	 &\strut\vrule\kern1pt\vrule\strut\hfil	 \textcolor{blue}{ Typeface}	 &\strut\vrule\strut\hfil	\textcolor{blue}{ Font name}	&\cr	\hrule
\+\hfil	\eightrm Roman text	&\strut\vrule\strut\hfil	fxlr-t1	 &\strut\vrule\kern1pt\vrule\strut\hfil	\eightbf Boldface text	 &\strut\vrule\strut\hfil	 fxlb-t1	&\cr	\hrule
\+\hfil	\eighti Math italic	&\strut\vrule\strut\hfil	jkpmi	 &\strut\vrule\kern1pt\vrule\strut\hfil	\eighttt Typewriter text	 &\strut\vrule\strut\hfil	 rm-inconsolata	&\cr	\hrule
\+\hfil	\eightrm Math symbols	&\strut\vrule\strut\hfil	jkpsy	 &\strut\vrule\kern1pt\vrule\strut\hfil	\eightitbf Italic boldface text	 &\strut\vrule\strut\hfil	fxlbi-t1	&\cr	\hrule
\+\hfil	\eightrm Math extension	&\strut\vrule\strut\hfil	jkpex	 &\strut\vrule\kern1pt\vrule\strut\hfil	\eightslbf Slanted boldface text	 &\strut\vrule\strut\hfil	fxlbi-t1	&\cr	\hrule
\+\hfil	\eightit Italic text	&\strut\vrule\strut\hfil	fxlri-t1	 &\strut\vrule\kern1pt\vrule\strut\hfil	\eightcaps Caps	 &\strut\vrule\strut\hfil	fxlrc-t1	 &\cr	\hrule
\+\hfil	\eightsl Slanted text	&\strut\vrule\strut\hfil	fxlri-t1	 &\strut\vrule\kern1pt\vrule\strut\hfil	\eightcapsbf Caps in Boldface	 &\strut\vrule\strut\hfil	fxlbc-t1	&\cr	\hrule
									
	}\vrule}}\hrule}\hfil}								
									
	\BlackBoxes								

\bs\ii Matching {\eightrm AMS} symbols: \circledR \ \yen \ $\blacksquare \ \approxeq \ \eqslantgtr \ \curlyeqprec \ \curlyeqsucc \ \preccurlyeq \ \leqq \ \leqslant \ \lessgtr \ \nless \ \nleq \ \nleqslant \ \Bbb R \ \Bbb E \ \Bbb C \ \dots$






\input font_libertine_palatino   \fontss
\UseAMSsymbols
\section{\sixteenbf\fontss Libertine with Palatino}{Libertine with Palatino}
\sample
\ii This macro enables us to type text in Linux-Libertine and math in PX~Fonts. The macro is declared by typing {\color{brown}\verbatim\input font_libertine_palatino|endverbatim}. The macro typesets text in Michael Niedermair's  \href{http://www.tex.ac.uk/tex-archive/help/Catalogue/entries/libertine.html}{Linux-Liber\-tine} font and math in Young Ryu's \href{http://www.ctan.org/tex-archive/help/Catalogue/entries/pxfonts.html}{pxfonts}, which corresponds to \href{http://www.urwpp.de/cgi-bin1/dalcgi/source/schnellsuche.htd?searchchar=palladio}
{{\caps urw++} Palladio} text fonts designed by Herman Zapf. The {\caps urw++} Palladio font is based on the \href{http://new.myfonts.com/fonts/adobe/palatino/}{Palatino font} which was originally designed by Hermann Zapf for the Stempel foundry in 1950. The fonts of this macro provide their own {\caps ams} symbols. Details of this {\eightrm \TeX} macro are given in the table~below.
\bs
\hfil{Font assignment in {\color{brown}\verbatim font_libertine_palatino|endverbatim}~macro}\hfil
					
{\parindent=0pt\settabs4\columns\hfil\vbox{\hrule\hbox{\vrule\hbox{\vbox{\kern1pt\hrule\NoBlackBoxes		 \eightrm\fontss							 
									
\+\hfil	\textcolor{blue}{ Typeface}	&\strut\vrule\strut\hfil	\textcolor{blue}{ Font name}	 &\strut\vrule\kern1pt\vrule\strut\hfil	 \textcolor{blue}{ Typeface}	 &\strut\vrule\strut\hfil	\textcolor{blue}{ Font name}	&\cr	\hrule
\+\hfil	\eightrm Roman text	&\strut\vrule\strut\hfil	fxlr-t1	 &\strut\vrule\kern1pt\vrule\strut\hfil	\eightbf Boldface text	 &\strut\vrule\strut\hfil	 fxlb-t1	&\cr	\hrule
\+\hfil	\eighti Math italic	&\strut\vrule\strut\hfil	pxmi	 &\strut\vrule\kern1pt\vrule\strut\hfil	\eighttt Typewriter text	 &\strut\vrule\strut\hfil	 rm-inconsolata	&\cr	\hrule
\+\hfil	\eightrm Math symbols	&\strut\vrule\strut\hfil	pxsy	 &\strut\vrule\kern1pt\vrule\strut\hfil	\eightitbf Italic boldface text	 &\strut\vrule\strut\hfil	fxlbi-t1	&\cr	\hrule
\+\hfil	\eightrm Math extension	&\strut\vrule\strut\hfil	pxex	 &\strut\vrule\kern1pt\vrule\strut\hfil	\eightslbf Slanted boldface text	 &\strut\vrule\strut\hfil	fxlbi-t1	&\cr	\hrule
\+\hfil	\eightit Italic text	&\strut\vrule\strut\hfil	fxlri-t1	 &\strut\vrule\kern1pt\vrule\strut\hfil	\eightcaps Caps	 &\strut\vrule\strut\hfil	fxlrc-t1	 &\cr	\hrule
\+\hfil	\eightsl Slanted text	&\strut\vrule\strut\hfil	fxlri-t1	 &\strut\vrule\kern1pt\vrule\strut\hfil	\eightcapsbf Caps in Boldface	 &\strut\vrule\strut\hfil	fxlbc-t1	&\cr	\hrule
									
	}\vrule}}\hrule}\hfil}								
									
	\BlackBoxes								

\bs\ii Matching {\eightrm AMS} symbols: \circledR \ \yen \ $\blacksquare \ \approxeq \ \eqslantgtr \ \curlyeqprec \ \curlyeqsucc \ \preccurlyeq \ \leqq \ \leqslant \ \lessgtr \ \nless \ \nleq \ \nleqslant \ \Bbb R \ \Bbb E \ \Bbb C \ \dots$







\input font_libertine_times-x   \fontss
\UseAMSsymbols
\section{\sixteenbf\fontss Libertine with Times}{Libertine with Times}
\sample
\ii This macro enables us to type text in Linux Libertine and math in TX~Fonts. The macro is declared by typing {\color{brown}\verbatim\input font_libertine_times|endverbatim}. The macro typesets text in Michael Niedermair's  \href{http://www.tex.ac.uk/tex-archive/help/Catalogue/entries/libertine.html}{Linux-Libertine} font and math in Young Ryu's \href{http://www.ctan.org/tex-archive/help/Catalogue/entries/txfonts.html}{txfonts}, which corresponds to \href{http://new.myfonts.com/fonts/adobe/times/}{Adobe Times} text fonts. The fonts of this macro provide their own {\caps ams} symbols. Details of this {\eightrm \TeX} macro are given in the table~below.
\bs
\hfil{Font assignment in {\color{brown}\verbatim font_libertine_times|endverbatim}~macro}\hfil

{\parindent=0pt\settabs4\columns\hfil\vbox{\hrule\hbox{\vrule\hbox{\vbox{\kern1pt\hrule\NoBlackBoxes		 \eightrm\fontss							 
									
\+\hfil	\textcolor{blue}{ Typeface}	&\strut\vrule\strut\hfil	\textcolor{blue}{ Font name}	 &\strut\vrule\kern1pt\vrule\strut\hfil	 \textcolor{blue}{ Typeface}	 &\strut\vrule\strut\hfil	\textcolor{blue}{ Font name}	&\cr	\hrule
\+\hfil	\eightrm Roman text	&\strut\vrule\strut\hfil	fxlr-t1	 &\strut\vrule\kern1pt\vrule\strut\hfil	\eightbf Boldface text	 &\strut\vrule\strut\hfil	 fxlb-t1	&\cr	\hrule
\+\hfil	\eighti Math italic	&\strut\vrule\strut\hfil	txmi	 &\strut\vrule\kern1pt\vrule\strut\hfil	\eighttt Typewriter text	 &\strut\vrule\strut\hfil	 rm-inconsolata	&\cr	\hrule
\+\hfil	\eightrm Math symbols	&\strut\vrule\strut\hfil	txsy	 &\strut\vrule\kern1pt\vrule\strut\hfil	\eightitbf Italic boldface text	 &\strut\vrule\strut\hfil	fxlbi-t1	&\cr	\hrule
\+\hfil	\eightrm Math extension	&\strut\vrule\strut\hfil	txex	 &\strut\vrule\kern1pt\vrule\strut\hfil	\eightslbf Slanted boldface text	 &\strut\vrule\strut\hfil	fxlbi-t1	&\cr	\hrule
\+\hfil	\eightit Italic text	&\strut\vrule\strut\hfil	fxlri-t1	 &\strut\vrule\kern1pt\vrule\strut\hfil	\eightcaps Caps	 &\strut\vrule\strut\hfil	fxlrc-t1	 &\cr	\hrule
\+\hfil	\eightsl Slanted text	&\strut\vrule\strut\hfil	fxlri-t1	 &\strut\vrule\kern1pt\vrule\strut\hfil	\eightcapsbf Caps in Boldface	 &\strut\vrule\strut\hfil	fxlbc-t1	&\cr	\hrule
									
	}\vrule}}\hrule}\hfil}								
									
	\BlackBoxes								

\bs\ii Matching {\eightrm AMS} symbols: \circledR \ \yen \ $\blacksquare \ \approxeq \ \eqslantgtr \ \curlyeqprec \ \curlyeqsucc \ \preccurlyeq \ \leqq \ \leqslant \ \lessgtr \ \nless \ \nleq \ \nleqslant \ \Bbb R \ \Bbb E \ \Bbb C \ \dots$























\input font_concrete   \fontss
\section{\fourteenrm Concrete}{Concrete}

{\hrule\vbox{\noindent\vrule\NoBlackBoxes\vbox{\vskip2mm\leftskip7mm\rightskip7mm
\noindent\underbar{\rm Euler Formula}: The Euler formula, also known as {\rm Euler identity}, states
$$e^{\iota x} =\cos(x) + \iota \sin(x), $$
where $\iota$~is the {\sl imaginary unit}.

The Euler formula can be expanded as a series:
$$\eqalign {e^{\iota x}
&= \sum_{n=0}^{\infty} {(\iota x)^n\over{n!}}\cr
&= \sum_{n=0}^{\infty}{(-1)^{n}x^{2n}\over (2n)!} + \iota\sum_1^{\infty}{(-1)^{n-1}x^{2n-1}\over(2n-1)!}\cr
&= \cos(x) + \iota\sin(x).\cr}$$
\bigskip\bigskip
\noindent\underbar{\rm Cauchy Integral Theorem}: If $f(z)$ is analytic and its partial derivatives are continuous throughout some simply connected region~$R$,~then
$$\oint_\gamma f(z)\,dz  = 0$$
for any closed contour~$\gamma$ completely contained in~$R$.\vskip2mm
}\vrule}\hrule\BlackBoxes\bigskip\bigskip}

\ii This macro enables us to type text and math in Donald Knuth's \href{http://www.ctan.org/tex-archive/help/Catalogue/entries/concrete.html}{Concrete} fonts. The macro is declared by typing {\color{brown}\verbatim\input font_concrete|endverbatim}. The macro uses Jackowski, Ry\'cko and Bzyl's \href{http://www.ctan.org/tex-archive/help/Catalogue/entries/cc-pl.html}{cc-pl} package which is based on Knuth's \href{http://www.ctan.org/tex-archive/help/Catalogue/entries/concrete.html}{Concrete Roman} fonts. Details of this \capstex\ macro are given in the table~below.
\bs
\hfil{Font assignment in {\color{brown}\verbatim font_concrete|endverbatim}~macro}\hfil

{\parindent=0pt\settabs4\columns\hfil\vbox{\hrule\hbox{\vrule\hbox{\vbox{\kern1pt\hrule\NoBlackBoxes		 \eightrm\fontss							 \+\hfil	\textcolor{blue}{ Typeface}	 &\strut\vrule\strut\hfil	\textcolor{blue}{ Font name}	 &\strut\vrule\kern1pt\vrule\strut\hfil	 \textcolor{blue}{ Typeface}	 &\strut\vrule\strut\hfil	\textcolor{blue}{ Font name}	 &\cr	\hrule
\+\hfil	\eightrm Roman text	&\strut\vrule\strut\hfil	pcr10	 &\strut\vrule\kern1pt\vrule\strut\hfil	No boldface text	 &\strut\vrule\strut\hfil	 \emdash  	&\cr	\hrule
\+\hfil	\eighti Math italic	&\strut\vrule\strut\hfil	pcmi10	 &\strut\vrule\kern1pt\vrule\strut\hfil	\eighttt Typewriter text	 &\strut\vrule\strut\hfil	 cmtt10	&\cr	\hrule
\+\hfil	\eightrm Math symbols	&\strut\vrule\strut\hfil	cmsy10	 &\strut\vrule\kern1pt\vrule\strut\hfil	No italic boldface text	 &\strut\vrule\strut\hfil	 \emdash  	&\cr	\hrule
\+\hfil	\eightrm Math extension	&\strut\vrule\strut\hfil	cmex10	 &\strut\vrule\kern1pt\vrule\strut\hfil	No slanted boldface text	 &\strut\vrule\strut\hfil	 \emdash  	&\cr	\hrule
\+\hfil	\eightit Italic text	&\strut\vrule\strut\hfil	pcti10	 &\strut\vrule\kern1pt\vrule\strut\hfil	\eightcaps Caps	 &\strut\vrule\strut\hfil	 pccsc10	 &\cr	\hrule
\+\hfil	\eightsl Slanted text	&\strut\vrule\strut\hfil	pcsl10	 &\strut\vrule\kern1pt\vrule\strut\hfil	No caps in bold	 &\strut\vrule\strut\hfil	 \emdash  	 &\cr	\hrule
}\vrule}}\hrule}\hfil}								

\BlackBoxes								
									








\input font_cm   \fontss
\section{\sixteenbf\fontss Computer Modern}{Computer Modern}
\sample
\ii This macro enables us to type text in Computer Modern font (serif). Though \capstex\ typesets documents in Donald Knuth's Computer Modern fonts by default, this macro is being supplied so that the user can use the different sizes as discussed in this document and in case the main font of any \capstex\ document is other than Computer Modern then by using this macro we can set the font to Computer Modern in some group. The macro is declared by typing {\color{brown}\verbatim\input font_cm|endverbatim}. Details of this \capstex\ macro are given in the table~below.
\bs
\hfil{Font assignment in {\color{brown}\verbatim font_cm|endverbatim}~macro}\hfil

{\parindent=0pt\settabs4\columns\hfil\vbox{\hrule\hbox{\vrule\hbox{\vbox{\kern1pt\hrule\NoBlackBoxes		 \eightrm\fontss							 \+\hfil	\textcolor{blue}{ Typeface}	 &\strut\vrule\strut\hfil	\textcolor{blue}{ Font name}	 &\strut\vrule\kern1pt\vrule\strut\hfil	 \textcolor{blue}{ Typeface}	 &\strut\vrule\strut\hfil	\textcolor{blue}{ Font name}	 &\cr	\hrule
\+\hfil	\eightrm Roman text	&\strut\vrule\strut\hfil	cmr10	 &\strut\vrule\kern1pt\vrule\strut\hfil	\eightbf Boldface text	 &\strut\vrule\strut\hfil	 cmbx10	&\cr	\hrule
\+\hfil	\eighti Math italic	&\strut\vrule\strut\hfil	cmmi10	 &\strut\vrule\kern1pt\vrule\strut\hfil	\eighttt Typewriter text	 &\strut\vrule\strut\hfil	 cmtt10	&\cr	\hrule
\+\hfil	\eightrm Math symbols	&\strut\vrule\strut\hfil	cmsy10	 &\strut\vrule\kern1pt\vrule\strut\hfil	\eightitbf Italic boldface text	 &\strut\vrule\strut\hfil	cmbxti10	&\cr	\hrule
\+\hfil	\eightrm Math extension	&\strut\vrule\strut\hfil	cmex10	 &\strut\vrule\kern1pt\vrule\strut\hfil	\eightslbf Slanted boldface text	 &\strut\vrule\strut\hfil	cmbxsl10	&\cr	\hrule
\+\hfil	\eightit Italic text	&\strut\vrule\strut\hfil	cmti10	 &\strut\vrule\kern1pt\vrule\strut\hfil	\eightcaps Caps	 &\strut\vrule\strut\hfil	 cmcsc10	 &\cr	\hrule
\+\hfil	\eightsl Slanted text	&\strut\vrule\strut\hfil	cmsl10	 &\strut\vrule\kern1pt\vrule\strut\hfil	No caps in Boldface	 &\strut\vrule\strut\hfil	 \emdash 	&\cr	\hrule
}\vrule}}\hrule}\hfil}								
									
\BlackBoxes								









%%%%%%%%%%   Typefaces and Sizes   %%%%%%%%%%
\input font_charter \fontss
\section{Typefaces and Sizes}{Typefaces and Sizes}

\ii Given below are various typefaces and sizes that our macros offer.\bs\hrule\kern1pt\hrule\bs


{\obeylines
\rightline{Roman}\nopagebreak
{\twentyrm \fontss This text is in 20\,pt size.}
{\eighteenrm \fontss This text is in 18\,pt size.}
{\sixteenrm \fontss This text is in 16\,pt size.}
{\fourteenrm \fontss This text is in 14\,pt size.}
{\twelverm \fontss This text is in 12\,pt size.}
{\rm \fontss This text is in 10\,pt size.}
{\ninerm \fontss This text is in 9\,pt size.}
{\eightrm \fontss This text is in 8\,pt size.}
{\sevenrm \fontss This text is in 7\,pt size.}
{\sixrm \fontss This text is in 6\,pt size.}
{\fiverm \fontss This text is in 5\,pt size.}
\

\rightline{Italic}\nopagebreak
{\twentyit \fontss This text is in 20\,pt size.}
{\eighteenit \fontss This text is in 18\,pt size.}
{\sixteenit \fontss This text is in 16\,pt size.}
{\fourteenit \fontss This text is in 14\,pt size.}
{\twelveit \fontss This text is in 12\,pt size.}
{\it \fontss This text is in 10\,pt size.}
{\nineit \fontss This text is in 9\,pt size.}
{\eightit \fontss This text is in 8\,pt size.}
{\sevenit \fontss This text is in 7\,pt size.}
{\sixit \fontss This text is in 6\,pt size.}
{\fiveit \fontss This text is in 5\,pt size.}
\

\rightline{Slanted}\nopagebreak
{\twentysl \fontss This text is in 20\,pt size.}
{\eighteensl \fontss This text is in 18\,pt size.}
{\sixteensl \fontss This text is in 16\,pt size.}
{\fourteensl \fontss This text is in 14\,pt size.}
{\twelvesl \fontss This text is in 12\,pt size.}
{\sl \fontss This text is in 10\,pt size.}
{\ninesl \fontss This text is in 9\,pt size.}
{\eightsl \fontss This text is in 8\,pt size.}
{\sevensl \fontss This text is in 7\,pt size.}
{\sixsl \fontss This text is in 6\,pt size.}
{\fivesl \fontss This text is in 5\,pt size.}
\

\newpage
\rightline{Boldface}\nopagebreak
{\twentybf \fontss This text is in 20\,pt size.}
{\eighteenbf \fontss This text is in 18\,pt size.}
{\sixteenbf \fontss This text is in 16\,pt size.}
{\fourteenbf \fontss This text is in 14\,pt size.}
{\twelvebf \fontss This text is in 12\,pt size.}
{\bf \fontss This text is in 10\,pt size.}
{\ninebf \fontss This text is in 9\,pt size.}
{\eightbf \fontss This text is in 8\,pt size.}
{\sevenbf \fontss This text is in 7\,pt size.}
{\sixbf \fontss This text is in 6\,pt size.}
{\fivebf \fontss This text is in 5\,pt size.}
\

\rightline{Italic boldface}\nopagebreak
{\twentyitbf \fontss This text is in 20\,pt size.}
{\eighteenitbf \fontss This text is in 18\,pt size.}
{\sixteenitbf \fontss This text is in 16\,pt size.}
{\fourteenitbf \fontss This text is in 14\,pt size.}
{\twelveitbf \fontss This text is in 12\,pt size.}
{\itbf \fontss This text is in 10\,pt size.}
{\nineitbf \fontss This text is in 9\,pt size.}
{\eightitbf \fontss This text is in 8\,pt size.}
{\sevenitbf \fontss This text is in 7\,pt size.}
{\sixitbf \fontss This text is in 6\,pt size.}
{\fiveitbf \fontss This text is in 5\,pt size.}
\

\rightline{Slanted boldface}\nopagebreak
{\twentyslbf \fontss This text is in 20\,pt size.}
{\eighteenslbf \fontss This text is in 18\,pt size.}
{\sixteenslbf \fontss This text is in 16\,pt size.}
{\fourteenslbf \fontss This text is in 14\,pt size.}
{\twelveslbf \fontss This text is in 12\,pt size.}
{\slbf \fontss This is 10 \,pt slanted boldface.}
{\nineslbf \fontss This text is in 9\,pt size.}
{\eightslbf \fontss This text is in 8\,pt size.}
{\sevenslbf \fontss This text is in 7\,pt size.}
{\sixslbf \fontss This text is in 6\,pt size.}
{\fiveslbf \fontss This text is in 5\,pt size.}
\

\rightline{Caps}\nopagebreak
{\twentycaps \fontss This text is in 20\,pt size.}
{\eighteencaps \fontss This text is in 18\,pt size.}
{\sixteencaps \fontss This text is in 16\,pt size.}
{\fourteencaps \fontss This text is in 14\,pt size.}
{\twelvecaps \fontss This text is in 12\,pt size.}
{\caps \fontss This text is in 10\,pt size.}
{\ninecaps \fontss This text is in 9\,pt size.}
{\eightcaps \fontss This text is in 8\,pt size.}
{\sevencaps \fontss This text is in 7\,pt size.}
{\sixcaps \fontss This text is in 6\,pt size.}
{\fivecaps \fontss This text is in 5\,pt size.}

\

\rightline{Caps in boldface}\nopagebreak
{\twentycapsbf \fontss This text is in 20\,pt size.}
{\eighteencapsbf \fontss This text is in 18\,pt size.}
{\sixteencapsbf \fontss This text is in 16\,pt size.}
{\fourteencapsbf \fontss This text is in 14\,pt size.}
{\twelvecapsbf \fontss This text is in 12\,pt size.}
{\capsbf \fontss This text is in 10\,pt size.}
{\ninecapsbf \fontss This text is in 9\,pt size.}
{\eightcapsbf \fontss This text is in 8\,pt size.}
{\sevencapsbf \fontss This text is in 7\,pt size.}
{\sixcapsbf \fontss This text is in 6\,pt size.}
{\fivecapsbf \fontss This text is in 5\,pt size.}
}













%%%%%%%%%%   Inter-Line and Inter-Word Spacing   %%%%%%%%%%
\section{Inter-Line and Inter-Word Spacing}{Inter-Line and Inter-Word Spacing}

\ii As typefaces are very dear to typographic style, so is their arrangement. Of course, the value of the meaning and purpose of text, which holds even if sentences have been scribbled, is beyond comparison, but it is good to arrange good text in a good way. This part of our discussion deals with two salient features of typeset text arrangement\emdash inter-line and inter-word~spacing.

If the text font in \capstex\ is changed, the inter-line and inter-word spacing is not changed accordingly. This is not such a problem if we declare the new font at the same size as the preceding one. But if the new font is declared at a considerably larger or smaller size, the typesetting might not be aesthetically~elegant.

We deal with the inter-line and inter-word spacing problem first-hand by starting with an example. Then an ``acceptable'' solution to the spacing problem has been elaborated. The solution is not perfect but it is handy and a passable compromise. Then we move towards theoretical aspects of spacing. The discussion is fairly brief and can act as a good starting point for re-evaluating the ``space problem''. When it comes to word spacing, the best guide is our own experience. If we try to justify text~(12\,pt) in triple columns on an A4 page, then we are likely to face some problems. Narrower the column, sterner the justification. We will not deliberate on microtypography\emdash a distinctive approach that devotes much to spacing issues and can be used with~pdf\capstex. Curious readers are referred to these three works: \cite{zapf_microtypography}, \cite{thanh_microtypographic}, and~\cite{text_justification}.


\subsection{Example}A sample \capstex\ source file as shown below:

\bigskip\hrule\vbox{\parindent=0pt\vrule\NoBlackBoxes\vbox{\vskip2mm\leftskip7mm\rightskip7mm
{\obeylines\parindent=0pt\color{brown}\verbatim
\parindent=0pt
\input font_epigrafica_euler % the font size is 10pt
Inter-line and inter-word spacing are very important parameters of
typesetting. A text typeset in a beautiful typeface but `bad'
inter-line and inter-word spacing does not look beautiful. Check
the spacing between lines of the paragraph, and words of a line.
\medskip

\sixrm % changes the font size to 6pt
Inter-line and inter-word spacing are very important parameters of
typesetting. A text typeset in a beautiful typeface but `bad'
inter-line and inter-word spacing does not look beautiful. Check
the spacing between lines of the paragraph, and words of a line.
\medskip

\eighteenrm % changes the font size to 18pt
Inter-line and inter-word spacing are very important parameters of
typesetting. A text typeset in a beautiful typeface but `bad'
inter-line and inter-word spacing does not look beautiful. Check
the spacing between lines of the paragraph, and words of a line.|endverbatim}
\vskip2mm}\vrule}\hrule\BlackBoxes\bigskip

\nopagebreak\ii after compilation should produce something like this:\nopagebreak

\bigskip\hrule\vbox{\noindent\vrule\NoBlackBoxes\vbox{\vskip2mm\leftskip7mm\rightskip7mm
{\parindent=0pt
\input font_epigrafica_euler
% the font size is 10pt
Inter-line and inter-word spacing are very important parameters of
typesetting. A text typeset in a beautiful typeface but `bad'
inter-line and inter-word spacing does not look beautiful. Check
the spacing between lines of the paragraph, and words of a line.
\medskip
\sixrm % changes the font size to 6pt
Inter-line and inter-word spacing are very important parameters of
typesetting. A text typeset in a beautiful typeface but `bad'
inter-line and inter-word spacing does not look beautiful. Check
the spacing between lines of the paragraph, and words of a line.
\medskip
\eighteenrm % changes the font size to 18pt
Inter-line and inter-word spacing are very important parameters of
typesetting. A text typeset in a beautiful typeface but `bad'
inter-line and inter-word spacing does not look beautiful. Check
the spacing between lines of the paragraph, and words of a line.}
\vskip2mm}\vrule}\hrule\BlackBoxes\bigskip\bigskip


\ii In the output we can notice that both inter-line and inter-word spacing are quite fine when the font size is 10\,pt. In the 6\,pt text the inter-line space is too much and and the inter-word space is more then needed. In the text at 18\,pt both inter-line and inter-word spacing are less then adequate. This is because \capstex\ is still working according to the default space values, which are declared for 10\,pt font size. To tackle this, \capstex\ offers two very useful primitive control statements~(\cite{knuth_texbook}, pp.\;76, 78). These~are:\ms
{\color{brown}\verbatim \spaceskip|endverbatim} to control the inter-word space,

{\color{brown}\verbatim \baselineskip|endverbatim} to control the inter-line space.




\definexref{solution}{solution}{}
\subsection{An Easy Solution}Here I~am stating a technique that I use to confront spacing problems when using different fonts at different sizes. Let us make a new definition called~{\color{brown}\verbatim \fontspacing|endverbatim}.\ms

\ii {\color{brown}\verbatim\def\fontspacing{\baselineskip=2.8ex plus0pt minus0pt
             \spaceskip=0.333333em plus0.122222em minus0.0999999em}|endverbatim}\sk
             \def\fontspacing{\baselineskip=2.8ex plus0pt minus0pt
             \spaceskip=0.333333em plus0.122222em minus0.0999999em}

\ii The units, {\sl ex\/} and {\sl em\/} are relative~(\cite{knuth_texbook}, pp.\;60). This makes our definition more~general.\ms

{\sl em\/} is the width of a ``quad'' in the current font,

{\sl ex\/} is the ``x-height'' of the current font.\sk

\ii Declaring {\color{brown}\verbatim \fontspacing|endverbatim} would set our inter-line space to 2.8ex~(=~12.05553\,pt in case of font {\verbatim cmr10|endverbatim} at 10\,pt) with no {\sl stretchability\/}~(given after {\sl plus\/}) or {\sl shrinkability}~(given after {\sl minus\/}). Also {\color{brown}\verbatim \fontspacing|endverbatim} would set our inter-word space to 0.333333\,em, with 0.122222\,em of stretchability and 0.0999999\,em of shrinkability allowed. In case of font {\verbatim cmr10|endverbatim}, these values (default) are 3.33333\,pt, 1.66666\,pt, and 1.11111\,pt, respectively.

Let us try to use {\color{brown}\verbatim \fontspacing|endverbatim} in the example given at the beginning of this chapter. A sample \capstex\ source file as given here:


\newpage

\bigskip\hrule\vbox{\noindent\vrule\NoBlackBoxes\vbox{\vskip2mm\leftskip7mm\rightskip7mm
{\obeylines\parindent=0pt\color{brown}
\verbatim
\parindent=0pt
\input font_epigrafica_euler % the font size is 10pt
\fontspacing % \baselineskip and \spaceskip are set accordingly
Inter-line and inter-word spacing are very important parameters of
typesetting. A text typeset in a beautiful typeface but `bad'
inter-line and inter-word spacing does not look beautiful. Check
the spacing between lines of the paragraph, and words of a line.
\medskip
\sixrm % changes the font size to 6pt
\fontspacing % \baselineskip and \spaceskip are set accordingly
Inter-line and inter-word spacing are very important parameters of
typesetting. A text typeset in a beautiful typeface but `bad'
inter-line and inter-word spacing does not look beautiful. Check
the spacing between lines of the paragraph, and words of a line.
\medskip
\eighteenrm % changes the font size to 18pt
\fontspacing % \baselineskip and \spaceskip are set accordingly
Inter-line and inter-word spacing are very important parameters of
typesetting. A text typeset in a beautiful typeface but `bad'
inter-line and inter-word spacing does not look beautiful. Check
the spacing between lines of the paragraph, and words of a line.|endverbatim}
\vskip2mm}\vrule}\hrule\BlackBoxes\bigskip


\nopagebreak\ii after compilation should produce something like this:\nopagebreak


\bigskip\hrule\vbox{\noindent\vrule\NoBlackBoxes\vbox{\vskip2mm\leftskip7mm\rightskip7mm
{\parindent=0pt
\input font_epigrafica_euler % the font size is 10pt
\fontspacing % \baselineskip and \spaceskip are set accordingly
Inter-line and inter-word spacing are very important parameters of
typesetting. A text typeset in a beautiful typeface but `bad'
inter-line and inter-word spacing does not look beautiful. Check
the spacing between lines of the paragraph, and words of a line.
\medskip
\sixrm % changes the font size to 6pt
\fontspacing % \baselineskip and \spaceskip are set accordingly
Inter-line and inter-word spacing are very important parameters of
typesetting. A text typeset in a beautiful typeface but `bad'
inter-line and inter-word spacing does not look beautiful. Check
the spacing between lines of the paragraph, and words of a line.
\medskip
\eighteenrm % changes the font size to 18pt
\fontspacing % \baselineskip and \spaceskip are set accordingly
Inter-line and inter-word spacing are very important parameters of
typesetting. A text typeset in a beautiful typeface but `bad'
inter-line and inter-\break word spacing does not look beautiful. Check
the spacing between lines of the paragraph, and words of a line.}
\vskip2mm}\vrule}\hrule\BlackBoxes\bigskip\bigskip

By using the control primitives {\color{brown}\verbatim \spaceskip|endverbatim} and {\color{brown}\verbatim \baselineskip|endverbatim} we get the desired spacing and these can be declared almost anywhere. For more details on spacing, please refer~to~\cite{knuth_texbook}.


\subsection{Ideal Spacing?}It is a well-known fact that inter-line and inter-word spacing are vital aspects of good typography. Inter-line space is also referred to as {\sl leading}, {\sl line space}, {\sl interlinear space}, and {\sl interline space}. Inter-word space is also known as {\sl word space} and {\sl interword space}. What are the ``best'' values for inter-line and inter-word space? For sure there is no one-line answer to this question. It is subjective; what might be the ``best'' for someone, may look to ``poor'' someone~else.

It can be noted that spacing is certainly dependent on the size of typesetting font. Fonts at larger sizes have different spacing requirements than font at normal or smaller sizes. Also, spacing (inter-line or inter-word) is not directly or inversely proportional to font-size, though it can serve as a good approximation and in our \ref{solution} we had used the proportionality concept. Different typefaces have different spacing demands. The medium of representation also influences spacing values\emdash text on paper is different from text on computer screens or projected slides. Spacing requirements vary if text is a single line and is meant to pass the eye in a glance, e.g., file names in a list, or if it is for continued reading, e.g., this~paragraph.

Let us streamline our discussion by considering the most likely case, i.e.\ normal text; we find it in books, novels, and magazines. In this case the text is designed for continued reading. Even in this case, for a particular font, spacing requisites depend on the width of the text. A text that runs 15\,cm wide should be typeset with different spacing parameters than some text that runs only 6\,cm, e.g., in a column of a multiple-column page. But this is for some other time. For now we focus on the general case\emdash the case of continued normal text, which is mostly in 10\endash 14\,pt. From this point we will discuss inter-line and inter-word space one at a~time.

\subsubsection{Inter-Word Space}
We commence with \href{http://www.linotype.com/794/inhonorofthe100thbirthdayofjantschichold.html}{Jan Tschichold}'s text composition rules which are constituents of \href{http://openlibrary.org/books/OL19449256M/Penguin_composition_rules.}{The Penguin Composition Rules}, which are a compilation of Tschichold's ideas. They can be found \href{http://ronin-group.org/misc_etext_tschichold.html}{here}. On text composition it is mentioned:
\sk{\sl
\itemitem{1.}All text composition should be as closely word-spaced as possible. As a rule, the spacing should be about a middle space or the thickness of an `i' in the type size used.\sk
\itemitem{2.}Wide spaces should be strictly avoided. Words may be freely broken whenever necessary to void wide spacing, as breaking words is less harmful to the appearance of the page than too much space between words.\sk
\itemitem{3.}All major punctuation marks\emdash full point, colon, and semicolon\emdash should be followed by the same spacing as is used throughout the rest of the line.\ms}

\ii In this game there are no rigid rules. \href{http://www.typotheque.com/authors/robert_bringhurst}{Robert Bringhurst} writes in his influential book~(\cite{elements_typographic}):
\quote{For a normal text face in a normal text size, a typical value for the word space is a quarter of an em which can be written M/4. A quarter of an em is typically about the same as, or slightly more than, the set-width of the letter~t.}

\ii The optimum~(without stretching or shrinking) inter-word space in \capstex's default regular font (\hfuzz4pt{\verbatim cmr10|endverbatim} at 10\,pt) is 3.33333\,pt. The width of letter `i' of {\verbatim cmr10|endverbatim} at 10\,pt is 2.77779\,pt and of letter `t' is 3.8889\,pt. One quarter of an em of {\verbatim cmr10|endverbatim} at 10\,pt is 2.5\,pt. A small manipulation in inter-word space, its stretchability or shrinkability, can lead to quite apparent changes.

\hfuzz1pt

Ideally inter-word spacing should be constant in the whole text but in justified text this is almost impossible to attain. The amount of stretching and shrinking of inter-word space and hyphenation of words has its limits. Some people would agree with Tschichold and opt for more hyphenation and less flexible inter-word space to maintain better page color while others would say that excessive hyphenation hinders readability and they would set wider and flexible inter-word space that might lead to rivers. Over the years the inter-word space in text has increased or maybe it is too language dependent or the lack of paper was the issue\emdash compare the inter-word space in \href{http://burton.byu.edu/Bible\%20Site/Gutenberg.htm}{Gutenberg's Bible}, hallmark of excellent typography, and book~\cite{knuth_texbook}, a book typeset with typo\-gra\-phic elegance.

In book~\cite{knuth_texbook}, Knuth instructed \capstex\ to give some extra space after periods, commas, question and exclamation marks, colons and semicolons. By default plain~\capstex\ would do this unless we use the control sequence {\color{brown}\verbatim \frenchspacing|endverbatim}. Tschichold urges not to give such extra space. In this document, I have used {\color{brown}\verbatim \frenchspacing|endverbatim} as the typeset text seemed to have even color without white blocks or rivers. But when I write a scientific report or thesis, which contains mathematics, symbols, variables, etc., I prefer Knuth's way which puts extra space after punctuation\emdash I think it makes text more readable and easier to~understand. In multiple columns with normal size text on an A4 or letter size page, my experience suggests that extra space after punctuation leads to rivers and blocks of~white.

Different fonts demand different inter-word space. Bitstream Charter, the current typeface, can bear and looks better with stiffer and lesser inter-word space than Computer Modern. In case we desire prime typography then we should set inter-word space according to the font in~use.

Book~\cite{elements_typographic} mentions a `reasonable' value of inter-word space with stretch and shrink values. When translated into \capstex's language, it becomes {\color{brown}\verbatim \spaceskip=0.25em plus0.08em minus0.05em|endverbatim}. Try it to find out how good it is. Does it fill the page with black boxes? What effect does it have on hyphenation? What if multiple columns are~used?

Our discussion on inter-word space concludes with the statement: {\sl There are no ideal or perfect or best inter-word space parameters}. We are the judge of our own work and refinement in judgement comes with experience, so let's see what's~coming.

\subsubsection{Inter-Line Space}
Managing inter-line space is usually easy, unless we encounter a club or widow line. In case of normal size text, the inter-line space is usually 0\endash 4\,pt more than the typeface size in points. The regular font this document is {\verbatim mdbchr7t|endverbatim} at 10\,pt, with an inter-line space of {\tt\the\baselineskip}, and for the global magnification, {\verbatim \magnification=1100|endverbatim} has been~used.

On a page, say A4 size, for any particular font, it is acceptable to have lesser inter-line space when using multiple columns. The eye does not loose track of the line and is at ease in stepping down to the next line when the column width is less, e.g., about~6\,cm. In the realm of inter-line spacing there are challenges like grid typesetting and widow lines but we will not discuss them here. \TeX's instruction, {\color{brown}\verbatim \baselineskip|endverbatim}, we have already discussed and more can be found in \cite{knuth_texbook} and~\cite{against_widows}.



%%%%%%%%%%   Acknowledgements   %%%%%%%%%%
\section{Acknowledgements}{Acknowledgements}

\bs

\ii I wholeheartedly thank Donald~E.\;Knuth for giving us \capstex\emdash the best typesetting program till date, which has succeeded the test of time. Also, I~am grateful to dear \capstex\ users who have contributed to make \capstex\ better by giving more freedom to its free feature. The designers of fonts and packages that I have used are a few of those dear \capstex\ users. I~am grateful to Petr Habala for introducing me to \capstex. I~am beholden to my family for their indispensable love and support. I thank my wife, Daphne, for participating in discussions on fonts and for her understanding. There are so many factors that I~am not able to take into account that in the end but above all, I say, ``Thank you~{\dev :}.''









%%%%%%%%%%   References   %%%%%%%%%%
\section{References\raise3mm\hbox{\bf 2}}{References}

\hskip20cm\hbox{\cite{habala_amstex}\cite{knuth_texbook}\cite{fonts_tex_latex}\cite{elements_typographic}}

\bibliography{C:/bib}

\bibliographystyle{ieeetran}


\vskip7.5cm\hrule width 5.70truecm\kern2mm\eightrm\fontss
\hbox{ \raise1.2mm\hbox{\ \ \ 2}\ \;  The reader has been referred to most of the references (they have not been listed on this page) via hyperlinks}
\hbox{\ \ \ \ \ \ \ \ provided in this {\caps pdf} document.}














\special{pdf: docinfo << /Author (Amit Raj Dhawan)
/Title (Macros to Change Text fonts & Math fonts in TeX)
/Creator(XeTeX: Based on TeX---The Genius of Knuth)
/Subject(Text and Math Fonts in TeX)
/Keywords(TeX, free, fonts, math font, text, maths font, font-change, Charter, Utopia, Century, Palatino, Bookman, Times, Euler, Bera, Arev, Vera, Iwona, Kurier, Kp-Fonts, Antykwa Torunska, Libertine, Epigrafica, Computer Modern Bright, CM Bright, Computer Modern, Concrete, macro, macros)>>}

\special{pdf: docview <</PageMode /UseOutlines>> }




\bye 