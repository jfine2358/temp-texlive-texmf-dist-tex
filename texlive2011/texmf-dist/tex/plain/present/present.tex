%%%%%%%%%%%%%%%%%%%%%%%%%%%%%%%%%%%%%%%%%%%%%%%%%%%%%%%%%%%%%%%%%
%
% present.tex
% Some macros for preparing presentations in TeX, using pdftex
% Version 2.0
% Copyright 2010 Matthias Meister
%
% This work may be distributed and/or modified under the
% conditions of the LaTeX Project Public License, either version 1.3
% of this license or (at your option) any later version.
% The latest version of this license is in
%   http://www.latex-project.org/lppl.txt
%
% This work has the LPPL maintenance status `maintained'.
% 
% The Current Maintainer of this work is:
%
%	Matthias Meister (meister_matthias@t-online.de)
%
% The work consists of the following files:
%	present.tex		(the macro collection)
%	HowToTalkTeX.tex 	(an example of present.tex in use and a documentation)
% 	HowToTalkTeX.pdf	(pdf obtained from HowToTalkTex.tex with pdftex) 
%	Stern.png		(auxiliary file)
%	Sagnac.pdf		(auxiliary file)
%	Pfeil1.pdf		(auxiliary file)
%	Pfeil2.pdf		(auxiliary file)
%	Pfeil3.pdf		(auxiliary file)
%	Pfeil4.pdf		(auxiliary file)
% 
%	(The auxiliary files are used in present.tex and HowToTalkTeX.tex, and 
%	 therefore should remain unchanged, too)
%	
% With reference to the definitions in the LaTeX Project Public License, for the
% purposes of this work a 'Base Interpreter' is a program implementing the TeX 
% language.
%
%%%%%%%%%%%%%%%%%%%%%%%%%%%%%%%%%%%%%%%%%%%%%%%%%%%%%%%%%%%%%%%%%

\catcode`\@=11
			

\parindent0pt		% Indentation probably not good for presentation
\parskip0pt		% Neither is automatic extra space between paragraphs

\def\StandardAspect{%   % Define "slide format", basically determined
\pdfpagewidth=120mm	% by aspect ratio of screen
\pdfpageheight=90mm	% StandardAspect gives 4:3 aspect ratio
\vsize=67mm		%
\hsize=110mm		% If you change these values, 
\hoffset=-20.4mm	% do so with care.
\voffset=-12.4mm	%
}

\def\WideAspect{%	% WideAspect gives 144:90 aspect ratio
\pdfpagewidth=144mm	%
\pdfpageheight=90mm	%
\vsize=67mm		% If you change these values,
\hsize=134mm		% do so with care
\hoffset=-20.4mm	%	
\voffset=-12.4mm	%
}


%
% Set some fonts
%
\font\titlefont=cmssbx10 at 20pt % Define fonts for title of presentation 
\font\normalfont=cmss12		 % ordinary text	
\font\slidetitlefont=cmssbx10    % title of a slide
\font\linkbarfont=cmss8		 % the LinkBar (see below)
\font\it=cmti12
%
\baselineskip=15pt
%			%% Below fonts for math are adapted to 12pt size
\font\tfont=cmr12
\font\sfont=cmr9
\font\ssfont=cmr7
\font\tifont=cmmi12
\font\sifont=cmmi9
\font\ssifont=cmmi7
\font\tsyfont=cmsy10 at 12pt
\font\ssyfont=cmsy9
\font\sssyfont=cmsy7
\font\texfont=cmex10 at 12 pt
\font\sexfont=cmex9
\font\ssexfont=cmex7
%
\textfont0=\tfont
\scriptfont0=\sfont
\scriptscriptfont0=\ssfont
\textfont1=\tifont
\scriptfont1=\sifont
\scriptscriptfont1=\ssifont
\textfont2=\tsyfont
\scriptfont2=\ssyfont
\scriptscriptfont2=\sssyfont
\textfont3=\texfont
\scriptfont3=\sexfont
\scriptscriptfont3=\ssexfont
%

%
% Definitions to allow referencing the last page of a document
%
\newwrite\wl@stpage
\newread\rl@stpage
\def\bye{%
\immediate\openout\wl@stpage=\jobname.lp%  The number of the last slide is written in a file
\write\wl@stpage{\the\pageno}%          %   with extension .lp
\closeout\wl@stpage
\par\vfill
\supereject
\end
}
\def\LP{undef} 
\openin\rl@stpage=\jobname.lp
\ifeof\rl@stpage\closein\rl@stpage\message{RUN AGAIN to get correct number of last page!}
\else\read\rl@stpage to \LP\fi

%
% Define some colours
%
% The commands below handle colour without color.tex, directly with pdftex
% Added October 3rd, 2010

% Define colours
\def\backgroundcolour{0.8 0.8 0.8}    % Colour of page background
\def\textcolour{0.0 0.0 0.0}   	      % Colour of text
\def\attentioncolour{1 0 0}	      % Colour for "Attention"

\newdimen\vsh@ft % vertical shift for background image, vsh@ft+\voffset=-25.4mm 
\newdimen\hsh@ft % horizontal shift for background image, hsh@ft+\hoffset=-25.4mm
\vsh@ft=-13mm
\hsh@ft=-5mm

\newcount\colst@ck
\colst@ck=\pdfcolorstackinit page {\textcolour rg \textcolour RG}
\def\popst@ck{\pdfcolorstack\colst@ck pop}
\def\setcolour#1{\pdfcolorstack\colst@ck push {#1 rg #1 RG}\aftergroup\popst@ck}
\long\def\coloured#1#2{{\setcolour#1#2}}
\long\def\att#1{\coloured\attentioncolour{#1}}
\def\bgc@lset{\relax}
\def\setbgcolour#1{\def\bgc@lset{\begingroup\pdfcolorstack\colst@ck push {#1 rg #1 RG}\aftergroup\popst@ck 
\vbox to 0pt{\kern\vsh@ft\hbox to 0pt{%
\kern\hsh@ft\vrule depth0pt height\pdfpageheight width\pdfpagewidth\hss}\vss}
\nointerlineskip
\endgroup}}


\def\NewSlide{\vfill\eject}			 % Begin a new slide; slide number increases

\def\NewFrame{\vfill\eject\advance\pageno by -1} % Begin new frame ("same" slide, with changes)
						 % slide number does not change	

\def\SlideTitle#1{%				 % Set the slide title, show it as the headline
\headline{%					 % of the slide in slidetitlefont, with a 
\vbox to 5mm{%			 		 % horizontal line separating the headline from
\hbox to \hsize{\slidetitlefont #1}%		 % the main body of the slide
\vskip3pt plus 1fil minus 1fil
\hrule height 1pt depth 0pt
}}}

\def\SlideFoot{%				 % Set the footline of the slide, separated by 
\footline{%					 % a horizontal slide from the main body of the
\vbox to 5mm {%                        		 % slide. This macro takes no argument.
\hrule height 1pt depth 0pt			 % Several possibilities are provided below.
\vskip3pt plus 1fil minus 1fil			 % Only one of the following options should be
\hbox to \hsize{%				 % uncommented.
\rlap{\LinkBar}\phantom{\PageBar}\hfil{\normalfont\the\pageno/\LP}\hfil\PageBar%  Links to sections, 
%										  %slide number, page navigation
%\hfil\the\pageno/\LP\hfil% % Slide number; \LP is number of last slide
%% 								     % empty
}}}}

\def\LinkBar{% 		% If you use LinkBar, you have to define it and possibly redefine
LinkBar}		% it at several places in your talk. The definition given 
			% just prints "LinkBar", in order to show where it goes. 

\def\PageBar{%		% This provides page navigation: Next page, previous page,
\hbox{%			% going back and forth in page history
\pagelink{PrevPage}{\image[width4mm]{Pfeil2.pdf}}\kern2mm
\pagelink{NextPage}{\image[width4mm]{Pfeil1.pdf}}\kern2mm
\pagelink{GoBack}{\image[width4mm]{Pfeil3.pdf}}\kern2mm
\pagelink{GoForward}{\image[width4mm]{Pfeil4.pdf}}
}}

%
% Make #2 a link to the destination named #1 in the presentation.
% A destination is created and named with the \target control word.
%
\def\link#1#2{%	
\pdfstartlink attr {/Border [0 0 0]} goto name {#1} #2\pdfendlink}

%
% Creates named destination at position where it is issued. #1 is name of destination.
%
\def\target#1{%
\pdfdest name {#1} xyz\relax}

%
% Used in the PageBar. Uses the predefined actions PrevPage etc.
%
\def\pagelink#1#2{%
\pdfstartlink
attr {/Border [0 0 0]} user {/Subtype /Link /A <</S /Named /N /#1>>}%
#2%
\pdfendlink}

%
% Sets a rectangular area of the screen to be a link to a destination.
% #1 is the destination (in the same document); it can be a named or other destination 
% #2 defines the rectangle which is the active link area, specified by 
%    four space separated numbers corresponding to lower left and upper right corners 
% #3 is the border width (in "default user-space units", probably pixels)
% #4 is the RGB-color, specified as three space separated numbers R G B between 0 and 1 
%
\def\linkarea#1#2#3#4{%
\pdfannot{%
/Subtype /Link
/Rect [#2]
/Border [0 0 #3]
/C [#4]
/A << /S /GoTo /D (#1) >>}}

%
% Make #2 a link to the web destination #1
%
\def\weblink#1#2{%
\pdfstartlink attr {/Border [0 0 0]} user {/Subtype /Link /A << /S /URI /URI (#1) >>}%
#2\pdfendlink}

%
% Make #3 a link to destination #2 in file #1 (a pdf-file).  One of several possible forms of a destination 
% is [page /Fit], where "page" is the page number in the file, counting from 0
% Named destinations are also possible
%
\def\filelink#1#2#3{%
\pdfstartlink attr {/Border [0 0 0]} user {/Subtype /Link /A << /S /GoToR /F (#1) /D #2 /NewWindow true >>}%
#3\pdfendlink}

%
% Includes an image. The image file name is #1, and #1 is a specification of the dimension of the
% image (will be scaled accordingly) in the usual TeX syntax (width, height, depth).
%
\def\image[#1]#2{%
\pdfximage #1 {#2}\pdfrefximage\pdflastximage}

%\def\movie#1#2#3{%
%\pdfstartlink attr {/Border [0 0 0]} user {/Subtype /Link /A << /S /Launch /F (mplayer -noborder -ontop -geometry #2 #1) >>}%
%#3\pdfendlink}

\footline{}	% Set some
\StandardAspect % defaultvalues

%
% Added 28. August 2010. Not part of version 1.0
% 
% Setting / unsetting a background image. If no background image is used, the background shows background colour 
%
\def\h@lpr{\relax}
\def\BIM@G{\relax}
\def\SetBackground#1{\def\BIM@G{#1}}
\def\UnsetBackground{\def\BIM@G{\relax}}
\output{%
\shipout\vbox{%
\bgc@lset
\ifx\BIM@G\h@lpr \else
{\vbox to 0pt{\kern\vsh@ft\hbox to 0pt{\kern\hsh@ft\image[width\pdfpagewidth height\pdfpageheight]{\BIM@G}\hss}\vss}%
\nointerlineskip}\fi
\makeheadline
\pagebody
\makefootline}
\advancepageno
\ifnum\outputpenalty>-20000\else\dosupereject\fi}

\def\rm{\fam=0} %% Added 14. September 2010. Use roman font in math.

\catcode`\@=12
