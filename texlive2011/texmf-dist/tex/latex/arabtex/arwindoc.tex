% arwindoc.tex
% 9 Apr 1997
% 
\documentclass[12pt]{article}
\usepackage{arabtex}

\begin {document}


\begin{table}[htbp]
\begin{center}
\novocalize \setarab
\small \tabcolsep 4.5pt

\def \sun {% sun symbol
\unitlength 0.2em
\begin{picture}(4,4)(-1.4,-1.2)
\put(0.6,0.2){\circle{1.6}}
\put(+1.0,+1.0){.}  \put(+1.0,-1.0){.}
\put(-1.0,-1.0){.}  \put(-1.0,+1.0){.}
\put(+0.8,+0.8){.}  \put(+0.8,-0.8){.}
\put(-0.8,-0.8){.}  \put(-0.8,+0.8){.}
\end{picture}}

\def \mc #1{\multicolumn{2}{c|}{#1}}


\begin{tabular}
{|c||@{}c@{}|@{}c@{}|c|@{}c@{}|@{}c@{}|c|c|c|c|c|c|@{}c@{}|c|c|c|c|@{}c@{%
\vrule height 15pt depth 9pt width 0pt}||}
\hline
   & 00& 01& 02&\mc{03}&04&05 &06&07&08&09&10&11&12&  13  &  14   &  15 \\
\hline
\hline
00 &NUL&DLE&SP &0& \<0> &@& P & ` & p && &NSP& & & \<_d> &  & \<|B"aN> \\
\hline
01 &SOH&DC1&!  &1& \<1> &A& Q & a & q &&&\<,> && \<-'>  & \<r>  & \<l> & \<|B"uN> \\
\hline
02 &STX&DC2&"  &2& \<2> &B& R & b & r &&&&& \<'A>  & \<z>  &   & \<|B"iN> \\
\hline
03 &ETX&DC3&\# &3& \<3> &C& S & c & s &&&&& \<'a>  & \<s>  & \<m>  & \<|B"a> \\
\hline
04 &EOT&DC4&\$ &4& \<4> &D& T & d & t &&&&&\<w"'>& \<^s> & \<n>  &       \\
\hline
05 &ENQ&NAK&\% &5& \<5> &E& U & e & u &&&&& \<'i>  & \<.s> &  \<h> & \<|B"u> \\
\hline
06 &ACK&SYN&\& &6& \<6> &F& V & f & v &&&&& \<Y"'->& \<.d> & \<w>  & \<|B"i>  \\
\hline
07 &BEL&ETB& ' &7& \<7> &G& W & g & w &&&&& \<a>   &  &   &  \\
\hline
08 &BS &CAN& ) &8& \<8> &H& X & h & x &&&&& \<b>   & \<.t> &   & \<|BB>  \\
\hline
09 &HT &EM & ( &9& \<9> &I& Y & i & y &&&&& \<T>   & \<.z> &   &   \\
\hline
10 &LF &SUB&$*$&\mc{:}  &J& Z & j & z &&&&\<;>& \<t>   & \<`> &  &  \<|B"> \\
\hline
11 &VT &ESC&$+$&\mc{;}  &K&\verb"]"&k&\verb"}"&&&&&\<_t>&\<.g>& &  \\
\hline
12 &FF &IS4& , &\mc{$>$}&L&\verb"\"&l&\verb"|"&&&& &\<^g>&\<B|B|B> & \<Y> & \\
\hline
13 &CR &IS3&$-$&\mc{$=$}&M&\verb"["&m&\verb"{"&&&SHY& &\<.h>& \<f> & \<y> & \\
\hline
14 &SO &IS2& . &\mc{$<$}&N&\verb"^"&n&\verb"~"& & & & &\<_h>&\<q>&  & \\
\hline
15 &SI &IS1& / &\mc{?}  &O&\verb"_"&o&  &  & & & \<?> &\<d> &\<k>&  &DEL\\
\hline
\end{tabular}
\end{center}
%\caption{MS Windows with arabic support code table}\label{awin}
\caption{MS Arabic Windows code table (CP 1256)}\label{awin}
\end{table}

%%%%%%%%%%%%%%%%%%%%%%%%%%%%%%%%%%%%%%%%%%%%%%%%%%%%%%%%%%%%%%%%%%%%%%%%

The file \verb"arabwin.sty" contains a reading module for the
{\em MS-Windows with arabic support} code.
It is installed by the \LaTeX\ option \verb"arabwin"
or by \verb"\input arabwin.sty".
The module is activated by \verb"\setcode {arabwin}";
all following Arabic text will be considered to be coded according to
the {\em MS-Windows with arabic support} standard.
The ArabTeX notation may be reactivated by \verb"\setcode {arabtex}".
                                                  \index{code!MS-Windows}
                                                  \index{MS-Windows}
                                                  \index{code!8-bit}
                                                  \index{\setcode{arabwin}}
                                                  \index{\setcode{arabtex}}

The {\em MS-Windows with arabic support} code (see Table \ref{awin})
is an 8-bit code closely related both to 7-bit ASCII; 
whereas the lower 128 positions are identical to
ASCII (ISO 646), some of the upper 128 positions contain the Arabic characters
plus additional graphic and control characters.
                                                  \index{code!ASCII}

We reuse the ASMO 449
reading routines, after suitable modification of the input. 
This only works correctly if the input text does not contain 
genuine ASCII letters, as we
project the Arabic characters onto their locations in ASMO 449. 
Please note that only the characters that appear in Table \ref{awin}
are correctly processed.
Some of the code switching messages in the log file are spurious; 
do not worry.

The notes on vowelization and transliteration of ASMO 449 apply also.

The driver file indicated for ASMO 449 will be usable after the
obvious modifications; however, your \TeX\ installation must be
capable of processing 8-bit data input.
This is nowadays usually the case; otherwise you can try to
locally find some utility program that will strip the highest order bit
off the characters in your file, and process the result via ASMO 449.

\end  {document}
%%%%%%%%%%%%%%%%%%%%%%%% EOF %%%%%%%%%%%%%%%%%%%%%%%%%%%%%%


