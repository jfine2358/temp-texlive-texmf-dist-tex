% ITRANS Devanagari Header for TeX/LaTeX.
% using either Devnac or Devnag font
% Defines devanagari fonts in various sizes
% Also includes some macros for the ITRANS song book....
%-----
% Copy this file to someplace where TeX can find it, the best place
% is usually the ITRANS lib/ directory. Make sure to add that directory
% to the TEXINPUTS environment variable.
%-----
% created: avinash chopde, february 1994
%          avinash@acm.org
% modified: may 2001
% $Header:$
% -----------------------------------------------------------------
%% Usage:
%% Include this file somewhere in the beginning of your input file.
% \input <thisfilename>
%% Then, if you are using Frans's font, add these lines (uncommented):
% #indianifm=dvng_is.ifm  % ITRANS command
% #indianfont=\fransdvng
%% If you are using the PostScript Devnac font that came with ITRANS, add
%% these lines (uncommented):
% #indianifm=dvnc.ifm  % ITRANS command
% #indianfont=\postdvng
%% Then, whenever you need to use some particular size of the font,
%% use the following command:
% \let\usedvng=\largedvng % usedvng is used by \fransdvng or \avidvng
%% sizes available: normaldvng < largedvng < LARGEdvng < hugedvng
% see the file ../doc/sample.itx for an example.
% -----------------------------------------------------------------
\newif\iffrans
% -----------------------------------------------------------------
% Page commands (for LaTeX ONLY, use one of these commands in preamble)
\def\portraitpage{%
    \setlength{\topmargin}{-0.50in} % real margin == this + 1in
    \setlength{\oddsidemargin}{-0.0in} % real margin == this + 1in
    \setlength{\evensidemargin}{-0.0in} % real margin == this + 1in
    \setlength{\columnsep}{20pt}
    \setlength{\columnseprule}{0.4pt}

    % Use Portrait Size Page
    \setlength{\textwidth}{6.5in}
    \setlength{\textheight}{9.0in}%
}
\def\landscapepage{%
    \typeout{Landscape Mode: be sure to print in landscape format}
    \typeout{   (for dvips, use -t landscape option)}
    
    \setlength{\topmargin}{-0.75in} % real margin == this + 1in
    \setlength{\oddsidemargin}{-0.0in} % real margin == this + 1in
    \setlength{\evensidemargin}{-0.0in} % real margin == this + 1in
    \setlength{\columnsep}{20pt}
    \setlength{\columnseprule}{0.4pt}

    % Use Landscape Size Page
    \setlength{\textwidth}{9.5in}
    \setlength{\textheight}{7in}%
}
% -----------------------------------------------------------------
% Font Stuff
\font\sixrm=cmr6
\font\sevenrm=cmr7
\font\eightrm=cmr8
\font\ninerm=cmr9
\font\tenrm=cmr10

\let\smallcmr=\sixrm
\let\titlefont=\eightrm

\def\devnmode{
    \tolerance=10000
    \pretolerance=10000
    \normalbaselines
    \hyphenchar\devnfont=-1 % do not hyphenate words using this font
	%
	% If using Frans's font, need to make punctuation chars active.
	\iffrans
	    \specialsforfrans % turn this on for Frans' font only.
	\fi
    % \englfont % default font is english font.
    \devnfont % default font is devanagari font.
}
% ------------------------------------------------------------------------
% Devanagari font macros
% sizes: normal < large < Large < LARGE < huge < Huge

% 10pt text
\def\normaldvng{
    \iffrans
        \font\devnfont=dvng10
        \font\englfont=cmr10
    \else
        \font\devnfont=dnh at 12pt % size looks like 10pt
        \font\englfont=cmbx10
    \fi
    \normalbaselineskip=15pt \devnmode}

% 10.95pt text
\def\largedvng{\iffrans \font\devnfont=dvng10 scaled\magstephalf
          \font\englfont=cmr10 scaled\magstephalf
    \else \font\devnfont=dnh at 14pt % size looks like 11pt
          \font\englfont=cmbx10 scaled\magstephalf
    \fi
    \normalbaselineskip=16pt \devnmode}

% 12pt text
\def\Largedvng{\iffrans \font\devnfont=dvng10 scaled\magstep1
          \font\englfont=cmr10 scaled\magstep1
    \else \font\devnfont=dnh at 16pt % size looks like 12pt
          \font\englfont=cmbx10 scaled\magstep1
    \fi
    \normalbaselineskip=18pt \devnmode}

% 14.4pt text
\def\LARGEdvng{\iffrans \font\devnfont=dvng10 scaled\magstep2 % 14.4pt
          \font\englfont=cmr10 scaled\magstep2
    \else \font\devnfont=dnh at 19pt % size looks like 14.4 pt
          \font\englfont=cmbx10 scaled\magstep2
    \fi
    \normalbaselineskip=21pt minus2pt \devnmode}

% 17.28pt text
\def\hugedvng{\iffrans \font\devnfont=dvng10 scaled\magstep3 % 17.28pt
          \font\englfont=cmr10 scaled\magstep3
    \else \font\devnfont=dnh at 22pt % size looks like 17.28 pt
          \font\englfont=cmbx10 scaled\magstep3
    \fi
    \normalbaselineskip=25pt \devnmode}

% 20.74pt text
\def\Hugedvng{\iffrans \font\devnfont=dvng10 scaled\magstep4 % 20.74
          \font\englfont=cmr10 scaled\magstep4
    \else \font\devnfont=dnh at 25pt % size looks like 20.74 pt
          \font\englfont=cmbx10 scaled\magstep4
    \fi
    \normalbaselineskip=28pt \devnmode}

% ------------------------------------------------------------------------
% (Must do \let\usedvng after normaldvng is defined...)
%
\let\usedvng=\normaldvng	% default font size definition

\def\fransdvng{\franstrue\usedvng}

\def\postdvng{\fransfalse\usedvng}

% ------------------------------------------------------------------------
% Macro for song listings (verbatim mode---see page 381 of the TeXbook)
\def\obeyspaceslines{\def\par{\leavevmode\null\endgraf\penalty-500\relax}%
                        \obeylines \obeyspaces}
% actually, the TeXbook defn has problems -- according to comp.text.tex
% readers, I needed to add the \null and \relax above to make it correct!
% otherwise TeX will eat up any number that starts a line!!!!!
%{\obeyspaces\global\let =\ } % TeX sets space = "\space", change it to "\ "
{\obeyspaces\global\let =~} % use this for a non-breaking space

\def\threedots{{\englfont%
    \leavevmode\hbox{\hskip2pt .\hskip2pt .\hskip2pt .\hskip2pt}}}

% For Frans's devanagari font, make punctuation chars that are absent from
% the font active characters, and define them to use the roman font punctuation.
% These characters have to be made special
% NO NEED TO DO THIS FOR THE PostScript Devnac font...
\def\specialsforfrans{%
	\catcode`\(=\active
	\catcode`\,=\active
	\catcode`\)=\active
	\catcode`\:=\active
	\catcode`\;=\active
	\catcode`\!=\active
	\catcode`\?=\active
	\catcode`\'=\active
	\catcode`\"=\active
	\catcode`\/=\active
	\catcode`\|=\active%
}
% define the commands for the punctuations
{\specialsforfrans
	\gdef,{{\englfont\char`\,}}
	\gdef({{\englfont\char`\(}}
	\gdef){{\englfont\char`\)}}
	\gdef:{{\englfont\char`\:}}
	\gdef;{{\englfont\char`\;}}
	\gdef!{{\englfont\char`\!}}
	\gdef?{{\englfont\char`\?}}
	\gdef'{{\englfont\char`\'}}
	\gdef"{{\englfont\char`\"}}
	\gdef/{{\englfont\char`\/}}
	\gdef|{.}
	% CANNOT do the same for - (numbers like kern-0.3 get screwed up!)
	% \catcode`\-=\active % NOTE: this implies that minus cannot be used (no numbers)!
	% \def-{{\englfont\char`\-}} % DONT EVER TURN THIS ON!
	% Therefore, have to use \- in the input song
	% Similarly, cannot use . in input text (to get roman period)
	% Use \. instead.
}
% Frans's Font does not have -, but cannot make - an active character,
% so you must use \- to get a dash (same goes for . (dot) use \. instead).
% For PostScript out, the hdr.ips defines a \- procedure, so all works
% there too..... (IF your PostScript intepretor accepts \ in a name!)
% See the comments regarding "specialsforfrans" in this file to see how
% other punctuation characters are handled for Frans's devnag font...
% ------------------
%
% ------ definitions for song headings, etc ------
\def\fileinfo{%
    \ifx\songfile\undefined % songfile is defined when multiple
    			    % .s files are combined by "isongcat" program.
        \ifx\jobname\undefined
            \rightline{\hbox to 120pt{\hfill\smallcmr ITRANS Song Book}}%
	\else
            \rightline{\hbox to 120pt{\smallcmr ITRANS Song Book:\hfill{\jobname}.s}}%
	\fi
    \else
        \rightline{\hbox to 120pt{\smallcmr ITRANS Song Book:\hfill{\songfile}}}%
    \fi
}

% songtitle - use if needed. Invoke this after a \startsong statement
% Assumes that both \startsong and \songtitle lines in the input text
% do not contain a % at the end....
    \def\stitle#1{}	% unused today...
    \def\film#1{\def\vvfilm{#1}}
    \def\starring#1{}
    \def\singer#1{\def\vvsinger{#1}}
    \def\lyrics#1{\def\vvlyrics{#1}}
    \def\music#1{\def\vvmusic{#1}}

    \def\printtitle{{% print out vvfilm, vvsinger, vvlyrics, vvmusic
	\ifx\vvfilm\undefined \def\vvfilm{} \fi
	\ifx\vvsinger\undefined \def\vvsinger{} \fi
	\ifx\vvlyrics\undefined \def\vvlyrics{} \fi
	\ifx\vvmusic\undefined \def\vvmusic{} \fi
	\vskip3pt
	\baselineskip=0pt% local def only, using 2 {'s to define \printtitle..
	%
	% Now, print out legend only if atleast one of the args is non-{}
	\def\vvnonempty{1} % initialize non-empty to TRUE
	\ifx\vvfilm\empty \else\def\vvnonempty{0}\fi % check film non-empty
	\ifx\vvsinger\empty \else\def\vvnonempty{0}\fi % check singers
	\ifx\vvlyrics\empty \else\def\vvnonempty{0}\fi % check lyrics
	\ifx\vvmusic\empty \else\def\vvnonempty{0}\fi % check music
	% if any args present, print out legend:
	\nobreak
	\if0\vvnonempty
	    \setbox0=\hbox{\titlefont {\smallcmr film:} \vvfilm%
	    			   \quad{\smallcmr singer:} \vvsinger%
				   \quad{\smallcmr lyrics:} \vvlyrics%
				   \quad{\smallcmr music:} \vvmusic}%
	    \ifdim\wd0>\hsize
		% too large, split into two lines
		\hbox to\hsize{\titlefont {\smallcmr film:} \vvfilm%
	    			   \quad{\smallcmr singer:} \vvsinger\hfill}%
		\vskip2pt
		\hbox to\hsize{\titlefont {\smallcmr lyrics:} \vvlyrics%
				   \quad{\smallcmr music:} \vvmusic\hfill}%
	    \else
		\hbox to\hsize{\titlefont {\smallcmr film:} \vvfilm%
	    			   \hfill{\smallcmr singer:} \vvsinger%
				   \hfill{\smallcmr lyrics:} \vvlyrics%
				   \hfill{\smallcmr music:} \vvmusic\hfill}%
	    \fi
	\fi
	\nobreak
	}}% printtitle

% ------------------------------------------------
%%%%%%%%%%%%% Shrisha Rao's macros %%%%%%%%%%%%%%
\def\ldq{{\englfont ``\hskip+0.1em}} % Avinash Chopde's definitions
\def\rdq{{\englfont ''}}           % to make quotes possible.

\def\ast{{\englfont *}}      % defined to use an asterisk
\def\dash{{\englfont ---}}   % defined to use a long dash
\def\spl{{\englfont \S}}     % defined to use a special symbol

\def\lsq{{\englfont `\hskip+0.1em}}  % single open-quote
\def\rsq{{\englfont '}}            % single close-quote

\def\?{{\englfont ?}}        % defined to use a question-mark
%%%%%%%%%%%%% End of Shrisha Rao's macros %%%%%%%%%%%%%%
% ------------------------------------------------

\def\startsong{\bgroup 
	\def\-{{\englfont -}}%
	\def\.{{\englfont .}}%
	\usedvng % use whatever font the user wants...default normaldvng
	\englfont % usedvng makes default font indian, restore english font
	\parindent=7pt%
        \obeyspaceslines}

\def\endsong{%
	\nobreak
	\vskip 9pt plus1fill % fill this page if no more material available
	\fileinfo\smallskip
	\hrule height0.4pt%
	\egroup
	}

% ----------------------- End of idevn.tex ------------------------
