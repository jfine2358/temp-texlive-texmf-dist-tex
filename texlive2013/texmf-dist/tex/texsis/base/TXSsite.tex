%% file: TXSsite.tex  (TeXsis version 2.17)
%  $Revision: 18.0 $  :  $Date: 2000/05/17 00:45:04 $  :  $Author: myers $
%======================================================================*
% This is the site dependent info file for TeXsis.  Edit the things
% below to match your installation.  Take out what you don't need or
% want and add any local macros or other stuff you want.  Anything
% between \Ignore and \endIgnore is ignored, but will become active if
% you remove the \Ignore. Remember, this file is only read in once when
% texsis.fmt is made! 

% Change name to your own institution or site here:

\message{Site dependent info.... Generic version.}

\ATunlock				% make @ a letter

%===================================*
% \ORGANIZATION is the name of the issuing organization, which 
% appears at the top of the title page, or on a \Memo.
% You can also change it before you call \titlepage.

\Ignore		% an example using mixed caps
  \def\ORGANIZATION{{\eightrm {\tenrm B}ANSAI {\tenrm I}NSTITUTE}}
\endIgnore

\def\ORGANIZATION{}

%===================================*
% \LandscapeSpecial turns on landscape mode on the laser printer.
% How this is done is very site dependent. Uncomment the version
% that works for you, or insert your own if these don't work.

% 1) This is how dv2ips does it, and you need to have landscape.ps
% accesable to the filter program

\Ignore
  \def\LandscapeSpecial{\special{psfile=landscape.ps}}
\endIgnore

% 2) This is how dvips (Rokicki's version 5.484 and later) does it:

\def\LandscapeSpecial{\special{papersize=11in,8.5in}}
 

%===================================*
% \manualfeed tries to put the printer in Manual Feed mode, for
% printing on letterhead stationary or envelopes.
% How you do this is really installation specific.

% If you use 'dvi2ps' then use \special to read the file manfeed.ps.
% You can get it from ftp://feynman.physics.lsa.umich.edu/texsis/etc/manfeed.ps

\Ignore
  \def\manualfeed{\emsg{\@comment MANUAL FEED MODE SELECTED!}%
        \special{psfile=manfeed.ps}}
\endIgnore

% Rokicki's dvips will do manual feed if you use 'dvips -m'
% but this tries to do the same thing (and doesn't work!  -EAM 98/03/06)

\Ignore
  \def\manualfeed{\emsg{\@comment MANUAL FEED MODE SELECTED!}%
	\special{! @manualfeed}}
\endIgnore


%===================================*
% \@DOCcode is the document code that appears in the upper right
% part of the title page of a \preprint.  It starts out at its default
% value, and is changed by calling \pubcode{<code>} before you call
% \titlepage.  If you don't use anything you'll get \TeXsis.

\Ignore
  \def\@DOCcode{UTREL\hbox to 1.5cm{\hss}}   
\endIgnore

%===================================*
% \banner produces a banner across the top of the title page for
% a \preprint.  Customize this however you like.

\def\banner{\vskip 0pt
  \line{{\ninepoint\@PUBdate \hfil 
    \hbox{\vbox{\hrule\vskip 4pt \hbox{\ORGANIZATION}\vskip 4pt\hrule}}%                       % 
                               \hfil \@DOCcode}}%  
  \vskip 1.0cm 
} 

\def\@PUBdate{\uppercase{\today}}         % default is today


%===================================*
% \letterhead{phone} is for copying onto letterhead stationary.  
% It just puts on the date and a phone number (if given).

% THIS IS JUST AN EXAMPLE - MODIFY FOR YOUR OWN INSTITUTION.

\Ignore
\def\letterhead#1{% letterhead for copying onto Univ. Michigan stationary
  \def\PhoneNumber{#1}%
  \emsg{Positioning date/phone for University of Michigan letterhead.}%
  \begingroup                           % this header stuff is local
    \parskip=0pt \parindent=0pt         %
    \null\vskip 0.6in                   % skip down to date line
    \line{\hfill \hbox to 5in{\tensl \today\hfill}}% Today's Date
    \line{\tenss \hfill (734) 
          \ifx\PhoneNumber\empty 764-4437\else \PhoneNumber\fi}%
    \ifx\Email\undefined\else\line{\hfill \tt \Email}\fi
    \vskip 10pt plus 1fil       
  \endgroup}
\endIgnore

%===================================*
% \UMletterhead{phone} is a facsimile of the University of Michigan
% stationary.  If \Email is defined it is included under the phone number.

% THIS IS JUST AN EXAMPLE - MODIFY FOR YOUR OWN INSTITUTION.
% It uses a PostScript image of the Univ. Michigan seal, umseal.ps

\Ignore
\def\UMletterhead#1{%  Facsimile of UM letterhead
  \def\PhoneNumber{#1}%
  \begingroup                   % make these these things local
    \null\vskip -27pt           % masthead up a bit to match real thing
    \UMmasthead                 % UM seal and name
    \line{\hfill \hbox to 5in{\tensl \today\hfill}}% Today's Date
    \line{\tenss \hfill (734) 
          \ifx\PhoneNumber\empty 764-4437\else \PhoneNumber\fi}%
    \ifx\Email\undefined\else\line{\hfill \tt \Email}\fi
    \vskip 10pt plus 1fil       
  \endgroup}

\def\UMseal{{\hbox{\epsfbox{umseal.eps}}}}%

\def\UMmasthead{\line{\llap{\UMseal\hskip -35pt}\hfill
      \hbox{\vbox{\hsize=13.99cm   \def\LC##1{{\twentyfourpoint\bf ##1}}%
      \line{\hfill \twentypoint\bf \LC{T}HE \LC{U}NIVERSITY OF \LC{M}ICHIGAN}%
         \vskip 0.85cm
         \line{\tenss DEPARTMENT OF PHYSICS \hfill 2071 Randall Laboratory}%
         \line{\tenss \hfill Ann Arbor, MI 48109-1120}%
     }}}}%
\endIgnore

%===================================*
% Mailing Labels:  you may have to change these depending on your
% printer and the kind of labels you use.  

\Ignore

% This is for the "standard" inch high, 3 across the page labels

\lblHsize=7.32cm        % width of standard label
\lblVsize=2.54cm        % height of standard label
\fullHsize=8.5in        % hsize for page
\fullVsize=11.0in       % vsize for page

\lblVoffset=0pt         % shift down this much
\lblHoffset=-1.8cm      % shift over this much

\endIgnore

%>>> EOF TXSsite.tex <<<



