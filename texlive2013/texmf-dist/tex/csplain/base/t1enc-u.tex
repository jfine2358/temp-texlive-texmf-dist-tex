%%% The re-encoding table for Czech/Slovak alphabet using encTeX,
%%% v.Sep.2005 (C) Petr Ol\v s\'ak
%%% input: UTF8, internal TeX: T1Encoding
%%% Oct.2012 added plainTeX csnames mapping and \setucode definition.

% This is very simple setting of Czech and Slovak alphabet only.
% User can reset this value and add his own declaration.
% For more information see cstexman.pdf.

\ifx\mubyte\undefined
   \errhelp={You has specified \let\enc=u but this works only with encTeX
   extension of TeX binary. See ftp://math.feld.cvut.cz/pub/olsak/enctex.}
   \errmessage{The encTeX Feb2003 or later is not detected -- re-encoding is impossible}
   \endinput \fi

%%      TeX   input
\mubyte ^^c1  ^^c3^^81\endmubyte % \'A
\mubyte ^^e1  ^^c3^^a1\endmubyte % \'a
\mubyte ^^c9  ^^c3^^89\endmubyte % \'E
\mubyte ^^e9  ^^c3^^a9\endmubyte % \'e
\mubyte ^^cd  ^^c3^^8d\endmubyte % \'I
\mubyte ^^ed  ^^c3^^ad\endmubyte % \'i
\mubyte ^^d3  ^^c3^^93\endmubyte % \'O
\mubyte ^^f3  ^^c3^^b3\endmubyte % \'o
\mubyte ^^da  ^^c3^^9a\endmubyte % \'U
\mubyte ^^fa  ^^c3^^ba\endmubyte % \'u
\mubyte ^^dd  ^^c3^^9d\endmubyte % \'Y
\mubyte ^^fd  ^^c3^^bd\endmubyte % \'y
\mubyte ^^d4  ^^c3^^94\endmubyte % \^O
\mubyte ^^f4  ^^c3^^b4\endmubyte % \^o
\mubyte ^^c4  ^^c3^^84\endmubyte % \"A
\mubyte ^^e4  ^^c3^^a4\endmubyte % \"a
\mubyte ^^d6  ^^c3^^96\endmubyte % \"O
\mubyte ^^f6  ^^c3^^b6\endmubyte % \"o
\mubyte ^^dc  ^^c3^^9c\endmubyte % \"U
\mubyte ^^fc  ^^c3^^bc\endmubyte % \"u
\mubyte ^^83  ^^c4^^8c\endmubyte % \v C
\mubyte ^^a3  ^^c4^^8d\endmubyte % \v c
\mubyte ^^84  ^^c4^^8e\endmubyte % \v D
\mubyte ^^a4  ^^c4^^8f\endmubyte % \v d
\mubyte ^^85  ^^c4^^9a\endmubyte % \v E
\mubyte ^^a5  ^^c4^^9b\endmubyte % \v e
\mubyte ^^88  ^^c4^^b9\endmubyte % \' L
\mubyte ^^a8  ^^c4^^ba\endmubyte % \' l
\mubyte ^^89  ^^c4^^bd\endmubyte % \v L
\mubyte ^^a9  ^^c4^^be\endmubyte % \v l
\mubyte ^^8c  ^^c5^^87\endmubyte % \v N
\mubyte ^^ac  ^^c5^^88\endmubyte % \v n
\mubyte ^^90  ^^c5^^98\endmubyte % \v R
\mubyte ^^b0  ^^c5^^99\endmubyte % \v r
\mubyte ^^92  ^^c5^^a0\endmubyte % \v S
\mubyte ^^b2  ^^c5^^a1\endmubyte % \v s
\mubyte ^^94  ^^c5^^a4\endmubyte % \v T
\mubyte ^^b4  ^^c5^^a5\endmubyte % \v t
\mubyte ^^97  ^^c5^^ae\endmubyte % \r U
\mubyte ^^b7  ^^c5^^af\endmubyte % \r u
\mubyte ^^9a  ^^c5^^bd\endmubyte % \v Z
\mubyte ^^ba  ^^c5^^be\endmubyte % \v z
\mubyte ^^8f  ^^c5^^94\endmubyte % \' R
\mubyte ^^af  ^^c5^^95\endmubyte % \' r

\def\setcsucode #1 #2 #3 #4 #5{} % to skip the data from utf8lat1.tex
                                 % and utf8lata.tex

% \setucode re-definition on order to map all T1 slots rawly during \input
% utf8lat1.tex or utf8lata.tex. This is usable if the text occurres in 
% \upercase and \lowercase arguments.

\def\setucode #1 #2 #3 #4 #5{% 
   \ifx*#4% The character is not provided by T1 encoding, we need to use csname:
      \expandafter \mubyte \csname#1\endcsname #3\endmubyte
      \expandafter \ifx \csname#1\endcsname \relax
         \expandafter \def\csname#1\endcsname{#5}%
      \fi
   \else % raw UTF-8 to T1 mapping is possible:
      \mubyte #4 #3\endmubyte
   \fi
}

% There exist some character-like control sequences defined in plainTeX.
% It seems to be usable to interpret their corresponding UTF-8 codes:

\mubyte \S   ^^c2^^a7\endmubyte % section sign
\mubyte \P   ^^c2^^b6\endmubyte % paragraph (pilcrow) sign
\mubyte \ss  ^^c3^^9f\endmubyte % german sharp s
\mubyte \l   ^^c5^^82\endmubyte % l slashed
\mubyte \L   ^^c5^^81\endmubyte % L slashed
\mubyte \ae  ^^c3^^a6\endmubyte % ae ligature
\mubyte \oe  ^^c5^^93\endmubyte % oe ligature
\mubyte \o   ^^c3^^b8\endmubyte % o slash
\mubyte \AE  ^^c3^^86\endmubyte % AE ligature
\mubyte \OE  ^^c5^^92\endmubyte % OE ligature
\mubyte \O   ^^c3^^98\endmubyte % O slash
\mubyte \i   ^^c4^^b1\endmubyte % dotless i
\mubyte \j   ^^c8^^b7\endmubyte % dotless j
\mubyte \aa  ^^c3^^a5\endmubyte % a with ring
\mubyte \AA  ^^c3^^85\endmubyte % A with ring
\mubyte \copyright ^^c2^^a9\endmubyte     % copyright
\mubyte \dots      ^^e2^^80^^a6\endmubyte % ellipsis
\mubyte \dag       ^^e2^^80^^a0\endmubyte % single dagger
\mubyte \ddag      ^^e2^^80^^a1\endmubyte % double dagger

% The character-like control sequences defined in csplain:

\mubyte \clqq    ^^e2^^80^^9e\endmubyte  % czech left double quote
\mubyte \crqq    ^^e2^^80^^9c\endmubyte  % czech right double quote
\mubyte \flqq    ^^c2^^ab\endmubyte      % french left (czech right) double guillquote
\mubyte \frqq    ^^c2^^bb\endmubyte      % french right (czech left) double quillquote
\mubyte \promile ^^e2^^80^^b0\endmubyte  % per mille sign

\ifx\baseutfencoding\undefined \else \expandafter\endinput\fi

% if the csplain is used, we cannot process this part of the file. May be
% somebody used it from iniTeX... 

\let\baseutfencoding\relax

\global\everyjob=\expandafter{\the\everyjob
   \message{The utf8->T1 re-encoding of Czech+Slovak alphabet activated by encTeX}}

% Warnings about UTF-8 unknown codes are set by default: 

\input utf8unkn

% UTF-8 input / output is active by default:

\mubytein=1 \mubyteout=3 \mubytelog=1

\endinput

