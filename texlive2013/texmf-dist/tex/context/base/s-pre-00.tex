%D \module
%D   [      file=s-pre-00,
%D        version=1997.07.22,
%D          title=\CONTEXT\ Style File,
%D       subtitle=Presentation Environment 0,
%D         author=Hans Hagen,
%D           date=\currentdate,
%D      copyright={PRAGMA ADE \& \CONTEXT\ Development Team}]
%C
%C This module is part of the \CONTEXT\ macro||package and is
%C therefore copyrighted by \PRAGMA. See mreadme.pdf for
%C details.

%D This module provides some non core functionality that can
%D be used in the presentations styles. This module contains
%D experimental macros.
%D
%D The presentation environments are independent of each other.
%D Although they have much in common, and in many cases can be
%D exchanged, the common components are not collected in this
%D file. That way they can serve as examples of style
%D definitions.

\unprotect

\newbox     \presentationstack
\newcounter \presentationcounter
\newdimen   \presentationheight

\appendtoks\doglobal\newcounter\presentationcounter\to\everyshipout

\defineframedtext
  [presentationshield]

\setupframedtexts
  [presentationshield]
  [\c!background=\v!color,
   \c!backgroundcolor=StepColor,
   \c!offset=\v!overlay,
   \c!frame=\v!off]

\definecolor[StepColor][white]

\def\enablepresentationstep%
  {\let\enablepresentationstep\relax
   \useJSpreamblenow{presentation}%
   \setuptexttexts
     [\vbox to \textheight
        {\dopresentationstep1\box\presentationstack\vfill}]}

\def\presentationstep
  {\dopresentationstep0}

\def\dopresentationstep#1% not ok yet
  {\global\setbox\presentationstack=\vbox
     {\ifnum\presentationcounter=0
        \global\presentationheight=\!!zeropoint
      \fi
      \ifcase#1\relax
        \scratchdimen=\pagetotal
      \else
        \scratchdimen=\textheight
      \fi
      \advance\scratchdimen by -\presentationheight
      \edef\presentationstepheight{\the\scratchdimen}
      \global\presentationheight=\pagetotal
      \forgetall
      \offinterlineskip
      \dontcomplain
      \box\presentationstack
      \doglobal\increment\presentationcounter
      \scratchcounter=\realpageno \multiply\scratchcounter by 100
      \advance\scratchcounter by \presentationcounter
      \edef\presentationtag{presentation:\the\scratchcounter}
      \setupfield
        [presentation]
        [\c!width=\v!fit,\c!height=\v!fit,\c!offset=\v!overlay,
         \c!strut=\v!no,\c!frame=\v!off,\c!option=\v!readonly]
      \definesymbol
        [\presentationtag]
        [\presentationshade]
      \def\presentationshade
        {\presentationshield
           [\c!width=\textwidth,\c!height=\presentationstepheight]{}}
      \definefield[\presentationtag][check][presentation][\presentationtag,\v!none][\presentationtag]
      \fitfield[\presentationtag]}}

\startJSpreamble {presentation} used later
  var presentation = new Array() ;
  for (i=1;i<=\lastpage;i++)
    { presentation[i] = 0 }
  this.dirty = false ;
  function NextPresentation (pagenumber)
    { % ++presentation[pagenumber] ;
      presentation[pagenumber] = presentation[pagenumber] + 1 ;
      var dummy = 100*pagenumber + presentation[pagenumber] ;
      var v = this.getField("presentation:"+dummy) ;
      if (v)
        { v.hidden = true }
      % { v.display = display.hidden }
      else
        { ++pagenumber ;
          if (pagenumber<=\lastpage)
            { presentation[pagenumber] = presentation[pagenumber] + 1 ;
              dummy = 100*(pagenumber) + presentation[pagenumber] ;
              v = this.getField("presentation:"+dummy) ;
              if (v) { v.hidden = true }
            % if (v) { v.display = display.hidden }
              ++this.pageNum } }
      this.dirty = false }
\stopJSpreamble

\definereference[NextStep][JS(NextPresentation{\realfolio})]

%D \macros
%D   {presentationstep}
%D
%D The macro \type {\presentationstep} provides a basic slide
%D show functionality. It sort of records pieces of the page
%D that will show up stepwise. It can be used like:
%D
%D \starttyping
%D \startitemize
%D \item eerste
%D \item tweede
%D \stopitemize
%D
%D \presentationstep
%D
%D \startformula
%D   ax^2+bx+c
%D \stopformula
%D
%D \presentationstep
%D \stoptyping
%D
%D When the document is opened, the two text fragments are
%D covered by a shield. Each page has its own shield stack. The
%D logical reference \type {NextStep} can be used to hide the
%D shield.
%D
%D \starttyping
%D \setupfootertexts[{\button{Show Up}[NextStep]}]
%D \stoptyping
%D
%D In case of presentation style 2, you can say:
%D
%D \starttyping
%D \definereference[Whatever][JS(NextPresentation{\realfolio})]
%D \stoptyping
%D
%D Depending of the viewer, you need to reset the fields at
%D startup:
%D
%D \starttyping
%D \setupinteraction[openaction=ResetForm]
%D \stoptyping
%D
%D Fixed spacing give you the best results:
%D
%D \starttyping
%D \setupwhitespace[fixed]
%D \setupblank[fixed]
%D \stoptyping
%D
%D Sometimes you need to set the color of the background, as
%D with style 2:
%D
%D \starttyping
%D \definecolor[StepColor][Page]
%D \stoptyping
%D
%D Special effects can be reached with:
%D
%D \starttyping
%D \setupframedtexts
%D   [presentationshield]
%D   [background=color,backgroundcolor=red]
%D \stoptyping
%D
%D There is also:
%D
%D \starttyping
%D \autopresentationsteptrue
%D \stoptyping
%D
%D One can enable this feature as \type {step} mode.

\newif\ifautopresentationstep

\appendtoks \ifautopresentationstep\presentationstep\fi \to \everypar

\doifmode{step}                             {\enablepresentationstep}
\doifmode{autostep}{\autopresentationsteptrue\enablepresentationstep}

%D When in \type {comment} mode, embedded page comments are
%D processed and put below the page. The default space of
%D 5cm should be adequate but can of course be changed.

\doifmode{comment}{\setuppagecomment[state=start,location=bottom]}

\protect \endinput
