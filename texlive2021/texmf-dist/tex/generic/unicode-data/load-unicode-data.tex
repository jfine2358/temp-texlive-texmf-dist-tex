% File load-unicode-data.tex
%
% Copyright 2015-2020 The LaTeX3 Project
%
% It may be distributed and/or modified under the conditions of
% the LaTeX Project Public License (LPPL), either version 1.3c of
% this license or (at your option) any later version. The latest
% version of this license is in the file
% http://www.latex-project.org/lppl.txt.
%
% Issues with this file should be reported at
% https://github.com/latex3/unicode-data
%
% This file parses a number of data files provided by the Unicode Consortium
% and when used with used Unicode-capable engine sets up a range of TeX-related
% parameters based on the extracted information.
%
% From the file UnicodeData.txt the following properties are set:
% - \catcode 11 for all letters (Unicode class "L")
% - \catcode 11 for all combining marks (Unicode class "M")
% - \sfcode 999 for all code points of class "Lu" (upper case letters)
% - \lccode for all of class "Ll" (lower case letters) to the code point
%   itself, and \uccode to the upper case mapping (or if not given
%   to the code point itself)
% - \uccode for all of class "Lu" (upper case letters) to the code point
%   itself, and \lccode to the lower case mapping (or if not given
%   to the code point itself)
% - \lccode and \uccode for all of class "Lt" (title case letters) to the
%   lower and upper case mappings (or if not given to the code point itself)
% - \lccode and \uccode for all other letter code points are set to
%   the code point itself
% - \lccode and/or \uccode for non-letter code points for which an upper
%   or lower case mapping is given
% - \sfcode 0 (ignored) for code points of Unicode classes "Pe" (closing
%   punctuation marks) and "Pf" (final quotation marks)
% - \Umathcode for all letters as math type 7 (var)
%
% =============================================================================
%
% The data can only be loaded by Unicode engines. Currently this is limited to
% XeTeX and LuaTeX, both of which define \Umathcode.
\ifx\Umathcode\undefined
  \expandafter\endinput
\fi
% Just in case, check for the e-TeX extensions.
\ifx\eTeXversion\undefined
  \expandafter\endinput
\fi
% This file can be loaded in IniTeX mode so the category codes of |{|, |}| and
% |#| may not be correct. Everything is done in a group so that only the
% settings we want to propagate are made available generally.
\begingroup
  \catcode`\{=1 %
  \catcode`\}=2 %
  \catcode`\#=6 %
% Write some basic information to the log.
  \catcode`\^=7 %
  \newlinechar=`\^^J %
  \message{^^J}%
  \message{load-unicode-data.tex v1.14 (2020-10-24)^^J}%
  \message{Reading Unicode data^^J}%
% The first stage of parsing is dealing with the fact that there are lots of
% data items separated by |;|. Of those, only a few are needed so they are
% picked out and everything else is dropped. There is one complication: there
% are a few cases in the data file of ranges which are marked by the descriptor
% |First| and a matching |Last|. A separate routine is used to handle these
% cases.
  \def\parseunicodedataI#1;#2;#3;#4;#5;#6;#7;#8;#9;{%
    \parseunicodedataII#1;#3;#2 First>\relax
  }%
  \def\parseunicodedataII#1;#2;#3 First>#4\relax{%
    \ifx\relax#4\relax
      \expandafter\parseunicodedataIII
    \else
      \expandafter\parseunicodedataVII
    \fi
    #1;#2;%
  }%
  \def\parseunicodedataIII#1;#2;#3;#4;#5;#6;#7;#8\relax{%
    \parseunicodedataIV{#1}{#2}{#6}{#7}%
  }%
% At this stage we have a `normal' data line with four pieces of information:
% the code point, the Unicode class and the (possibly empty) upper and lower
% case mappings. A few utility macros are defined, then we branch based on the
% Unicode class. Notice that for all letter-like code points we first set the
% |\lccode| and |\uccode| values to the code point itself then test for the
% classes where a different setting might be appropriate. For non-letters
% there is a check to see if any mappings are available, and also for trailing
% punctuation to set the appropriate |\sfcode|.
  \def\Ll{Ll}%
  \def\Lt{Lt}%
  \def\Lu{Lu}%
  \def\Pe{Pe}%
  \def\Pf{Pf}%
  \def\firsttoken#1#2\relax{#1}%
  \def\parseunicodedataIV#1#2#3#4{%
    \ifnum 0%
      \if L\firsttoken#2?\relax 1\fi
      \if M\firsttoken#2?\relax 1\fi
      >0 %
      \parseunicodedataV{"#1}%
      \def\temp{#2}%
      \ifx\Ll\temp
        \parseunicodedataVI\uccode{#1}{#3}%
      \fi
      \ifx\Lt\temp
        \parseunicodedataVI\uccode{#1}{#3}%
        \parseunicodedataVI\lccode{#1}{#4}%
      \fi
      \ifx\Lu\temp
        \parseunicodedataVI\lccode{#1}{#4}%
        \global\sfcode"#1=999 %
      \fi
% All letters in math mode should be variables.
      \global\Umathcode"#1="7"01"#1 %
    \else
      \def\temp{#2}%
      \ifnum 0\ifx\temp\Pe 1\fi\ifx\temp\Pf 1\fi>0 %
        \global\sfcode"#1=0 %
      \fi
      \ifx\relax#3\relax
      \else
        \global\uccode"#1="#3 %
      \fi
      \ifx\relax#4\relax
      \else
        \global\lccode"#1="#4 %
      \fi
    \fi
  }%
% A simple auxiliary for all letter-like code points: the |\lccode| and
% |\uccode| may get reset for cased letters but this means the initial
% setting can't be forgotten.
  \def\parseunicodedataV#1{%
    \global\catcode#1=11 %
    \global\lccode#1=#1 %
    \global\uccode#1=#1 %
  }%
% An auxiliary to deal with the fact that some cased letters don't actually
% have a case mapping available.
  \def\parseunicodedataVI#1#2#3{%
    \ifx\relax#3\relax
    \else
      \global#1"#2="#3 %
    \fi
  }%
% For lines that were the |First>| of a range, read the data source again for
% last line. Lines for letters then trigger a loop over the entire range. These
% are always non-cased letters.
  \def\parseunicodedataVII#1;#2;#3\relax{%
    \read0 to \unicodedataline
    \expandafter\parseunicodedataXII\unicodedataline\relax#1;#2\relax
  }%
  \def\parseunicodedataXII#1;#2\relax#3;#4\relax{%
    \if L\firsttoken#4?\relax
      \begingroup
        \count0="#3 %
        \loop
          \unless\ifnum\count0>"#1 %
            \parseunicodedataV{\count0 }%
            \advance\count0 by 1 %
        \repeat
      \endgroup
    \fi
  }%
% From plain: may not be defined (yet).
  \def\loop#1\repeat{\def\body{#1}\iterate}%
  \def\iterate{%
    \body
      \let\next\iterate
    \else
      \let\next\relax
    \fi
    \next
  }%
  \let\repeat\fi
% There is no version data in |UnicodeData.txt|: log that it is being used with
% a hard-coded date (the modification date from ftp.unicode.org). This obviously
% needs to be updated when a new download takes place!
  \message{\string# UnicodeData-13.0.0.txt^^J}%
  \message{\string# Modified 2020-03-12 11:00:00 GMT [JAW]^^J}%
% Actually loading the file requires an input stream, done directly.
% There is a blank line at the end of the data source so there is a check
% here for a |\par|.
  \def\storedpar{\par}%
  \openin0=UnicodeData.txt %
  \loop\unless\ifeof0 %
    \read0 to \unicodedataline  
    \unless\ifx\unicodedataline\storedpar
      \expandafter\parseunicodedataI\unicodedataline\relax
    \fi
  \repeat
  \closein0 %
\endgroup
