%%
%% This is file `pst-diffraction.tex',
%%
%% IMPORTANT NOTICE:
%%
%% Package `pst-diffraction.tex'
%%
%% Manuel Luque <ml@pstricks.de>
%% Herbert Voss <hv@pstricks.de>
%%
%% with contributions of Julien Cubizolles
%%
%% This program can be redistributed and/or modified under the terms
%% of the LaTeX Project Public License Distributed from CTAN archives
%% in directory macros/latex/base/lppl.txt.
%%
%% DESCRIPTION:
%%   `pst-diffraction' is a PSTricks package to plot special diffractions
%%
%% 
\csname PSTDiffractionLoaded\endcsname
\let\PSTDiffractionLoaded\endinput
% Require PSTricks
\ifx\PSTricksLoaded\endinput\else\input pstricks.tex\fi
\ifx\PSTThreeDplotLoaded\endinput\else\input pst-3dplot.tex\fi
\ifx\PSTXKeyLoaded\endinput\else \input pst-xkey \fi
%
\def\fileversion{2.03}%
\def\filedate{2008/09/03}%
\message{`PST-diffraction v\fileversion, \filedate\space (ML,hv)}%
\edef\PstAtCode{\the\catcode`\@} \catcode`\@=11\relax
\pst@addfams{pst-diff}

\define@key[psset]{pst-diff}{a}{\def\psk@Diffraction@Slit@A{#1 }} % largeur de la fente en m
\define@key[psset]{pst-diff}{k}{\pst@checknum{#1}\psk@Diffraction@Slit@k }
\define@key[psset]{pst-diff}{r}{\def\psk@Diffraction@Circular@r{#1 }} % rayon du trou en m
\define@key[psset]{pst-diff}{d}{\def\psk@Diffraction@Circular@d{#1 }} % demi-distance entre les trous en m
\define@key[psset]{pst-diff}{h}{\def\psk@Diffraction@Triangle@h{#1 }} % hauteur du triangle en m
\define@key[psset]{pst-diff}{scale}{\def\psk@Diffraction@Triangle@scale{#1 }} % hauteur du triangle en m
\define@key[psset]{pst-diff}{s}{\def\psk@Diffraction@Slit@s{#1 }} % distance entre les fentes
\define@key[psset]{pst-diff}{lambda}{\pst@checknum{#1}\psk@Diffraction@Slit@Lambda }% en nm
\define@key[psset]{pst-diff}{f}{\pst@checknum{#1}\psk@Diffraction@Slit@F }% focus en m
\define@key[psset]{pst-diff}{gamma}{\pst@checknum{#1}\psk@Diffraction@gamma@G }
\define@key[psset]{pst-diff}{pixel}{\pst@checknum{#1}\psk@Diffraction@Slit@pixel }
\define@key[psset]{pst-diff}{colorMode}{\pst@getint{#1}\psk@Diffraction@colorMode }
% 0 black and white inverse
% 1 black and white
% 2 color cmyk
%>2 color RGB
%
\define@key[psset]{pst-diff}{contrast}{%
  \pst@cnta=#1\relax 
  \ifnum\pst@cnta>38 
    \typeout{!!! The contrast value is `\the\pst@cnta', but cannot be greater than 38.
     Contrast=38 forced!}%
    \pst@cnta=38 
  \fi%
  \edef\psk@Diffraction@Slit@contrast{\the\pst@cnta\space}}
%
\define@boolkey[psset]{pst-diff}[Pst@Diffraction@]{showFunc}[true]{}
\define@boolkey[psset]{pst-diff}[Pst@Diffraction@]{IIID}[true]{}
\define@boolkey[psset]{pst-diff}[Pst@Diffraction@Circular@]{twoHole}[true]{}
\define@boolkey[psset]{pst-diff}[Pst@Diffraction@Rectangular@]{twoSlit}[true]{}
%
\psset[pst-diff]{a=0.2e-3,f=5,k=1,r=1e-3,
    d=6e-3,s=12e-3,h=0.5e-3,
    lambda=650,pixel=0.5,
    contrast=38,gamma=0.8,twoHole=false,twoSlit=false,colorMode=3, scale=1,
    IIID=false, showFunc=false}
%
% load the pstricks-add.pro only, if not already done
\ifx\PSTricksAddLoaded\endinput\else\pstheader{pstricks-add.pro}\fi
%
\def\tx@RGBtoGRAY{ tx@addDict begin RGBtoGRAY end }
\def\tx@RGBtoCMYK{ tx@addDict begin RGBtoCMYK end }
%
\def\psdiffractionRectangle{\pst@object{psdiffractionRectangle}}
\def\psdiffractionRectangle@i{%
  \begin@SpecialObj%
  \addto@pscode{%
    % les dimensions sont en m�tres
    /focus \psk@Diffraction@Slit@F def
    /widthSlit \psk@Diffraction@Slit@A def
    /heightSlit \psk@Diffraction@Slit@k widthSlit mul def
    /Gamma \psk@Diffraction@gamma@G def
    /pixel \psk@Diffraction@Slit@pixel def
    /SlitSeparation \psk@Diffraction@Slit@s def
    \psk@Diffraction@Slit@Lambda tx@addDict begin wavelengthToRGB Red Green Blue end
    /Blue ED /Green ED /Red ED
    0 0 translate
    /ondeLongueur \psk@Diffraction@Slit@Lambda 1e-9 mul def % en m
    % les bornes sont en pt
    % +- 4 fois le premier minimum
    /bornexpt 1 widthSlit div focus mul ondeLongueur mul 2845 mul def
    /borneypt 1 heightSlit div focus mul ondeLongueur mul 2845 mul def
    \ifPst@Diffraction@IIID 
      \psk@ThreeDplot@zMax dup \tx@ScreenCoor pop /zScale ED 
      tx@3DPlotDict begin \IIIDplot@variables end 
    \fi
    % Les calculs commencent...
    /yEnd borneypt 4 mul def
    yEnd neg pixel yEnd {
     /ordonneept ED
     % y en m
     /ordonnee ordonneept 2845 div def
     /argumenty ordonnee heightSlit mul ondeLongueur div focus div def
     /argumentyRad argumenty Pi mul def
     /argumentyDeg argumenty 180 mul def
     /sincy argumentyRad 0 eq { 1 }{ argumentyDeg sin argumentyRad div } ifelse def
     /xEnd bornexpt 4 mul def
     /OldIntensity 0 def
     xEnd neg pixel xEnd { 
       /abscissept ED
       \ifPst@Diffraction@IIID 
         abscissept pixel sub ordonneept OldIntensity 
         tx@3DPlotDict begin Conv3D2D end moveto 
       \fi % go to first point
       % x en m
       /abscisse abscissept 2845 div def
       /argumentx abscisse widthSlit mul ondeLongueur div focus div def
       /argumentcosx abscisse SlitSeparation mul ondeLongueur div focus div def       
       /argumentxRad argumentx Pi mul def
       /argumentxDeg argumentx 180 mul def
       /argumentcosxDeg argumentcosx 180 mul def
       % sinus cardinal
       /sincx argumentxRad 0 eq { 1 } { argumentxDeg sin argumentxRad div } ifelse def
       %
       1 1e\psk@Diffraction@Slit@contrast sincx dup mul sincy dup mul
       \ifPst@Diffraction@Rectangular@twoSlit mul argumentcosxDeg cos dup mul \fi
       mul neg exp sub dup 
       /intensity ED
       dup dup Red mul 3 -1 roll Green mul 3 -1 roll Blue mul 
       \ifcase\psk@Diffraction@colorMode
         \tx@RGBtoGRAY neg 1 add setgray \or
         \tx@RGBtoGRAY setgray \or
	 \tx@RGBtoCMYK setcmykcolor
       \else setrgbcolor \fi
       %
       % newpath abscissept ordonneept 1 0 360 arc closepath fill stroke
       % 30 juillet 2004
       \ifPst@Diffraction@IIID 
         abscissept ordonneept intensity zScale mul tx@3DPlotDict begin Conv3D2D end 
         lineto stroke 		% draw line in specific color
	 /OldIntensity intensity def
       \else
%         newpath		% done by stroke
         abscissept pixel 2 div sub ordonneept pixel 2 div sub 
         moveto
         pixel 0 rlineto
         0 pixel rlineto
         pixel neg 0 rlineto closepath fill stroke
       \fi
     } for
    } for
  }% \addto@pscode
  \end@OpenObj%
}
%
\def\psdiffractionCircular{\pst@object{psdiffractionCircular}}
\def\psdiffractionCircular@i{%
  \begin@SpecialObj%
  \addto@pscode{%
    % les dimensions sont en m�tres
    /focus \psk@Diffraction@Slit@F def
    /Gamma \psk@Diffraction@gamma@G def
    /pixel \psk@Diffraction@Slit@pixel def
    /contrast 1e\psk@Diffraction@Slit@contrast def
    /r \psk@Diffraction@Circular@r def
    /d \psk@Diffraction@Circular@d def
    \psk@Diffraction@Slit@Lambda tx@addDict begin wavelengthToRGB Red Green Blue end
    /Blue ED /Green ED /Red ED
    \ifPst@Diffraction@IIID 
      \psk@ThreeDplot@zMax dup \tx@ScreenCoor pop /zScale ED 
      tx@3DPlotDict begin \IIIDplot@variables end 
    \fi
%    0 0 translate
    %% Handbook of mathematical functions
    %% ED. M.ABRAMOWITZ & I.A. STEGUN
    %% Chap. 9 : Bessel Functions of Integer Order
    %% Adapt� en ps d'apr�s le code de D.Martin
    %% http://perso.univ-rennes1.fr/daniel.martin/melina/www/code_html/specfct/bejy01.html
    %
    % les coefficients pour le calcul des fonctions de Bessel
    /COSO {0.79788456 1 X div sqrt mul  X 0.78539816 sub RadtoDeg cos mul} def
    /SINO {.79788456 1 X div sqrt mul X 0.78539816 sub RadtoDeg sin mul} def
    %
    /PO { 0.209388721e-6 8  X div dup mul mul 0.207337064e-5 sub 8  X div dup mul mul
          0.273451041e-4 add 8  X div dup mul mul 0.109862863e-2 sub 8  X div dup mul mul 1 add } def
    /QO {-0.934945152e-7 8  X div dup mul mul 0.762109516e-6 add 8  X div dup mul mul
          0.691114765e-5 sub 8  X div dup mul mul 0.143048876e-3 add 8  X div dup mul mul
          0.15624499998e-1 sub 8 X div mul } def
    /P1 { -0.240337019e-6 8  X div dup mul mul 0.2457520174e-5 add 8  X div dup mul mul
           0.351639649e-4 sub 8  X div dup mul mul 0.183105e-2 add 8  X div dup mul mul 1 add } def
    /Q1 {  0.105787412e-6 8  X div dup mul mul 0.88228987e-6   sub 8  X div dup mul mul
           0.8449199096e-5 add 8  X div dup mul mul 0.2002690873e-3 sub 8  X div dup mul mul
           0.4687499995e-1 add 8 X div mul } def
    % les fonctions de Bessel
    /J0 { 
      /X exch def
      X 0 lt { /X X neg def } if
      X 8 le { X 0 eq { 1 }{
        0.33848331e-10 X X dup mul 0.2895532e-7 sub X dup mul mul
        0.113825372e-4 add X dup mul mul 0.258008068e-2 sub X dup mul mul
        0.351527847 add X dup mul mul 27.7849812 sub X dup mul mul
        0.114749848e4 add X dup mul mul 0.198342667e5 sub X dup mul mul
        0.813241206e5 add
        X dup mul 0.49676332e3 add X dup mul mul 0.813241206e5 add div } ifelse 
      }{ PO COSO mul QO SINO mul sub } ifelse 
    } def
    %
    /J1 { 
      /X exch def
      X 0 lt {/X X neg def /Signe {neg} def}{/Signe {} def} ifelse
      X 8 le { X 0 eq {0}{
        -0.47427072618e-9 X dup mul mul 0.35236457401e-6 add X dup mul mul
        -0.11764050281e-3 add X dup mul mul 0.21979683808e-1 add X dup mul mul
        -0.23652383961e+1 add X dup mul mul 0.13812534164e+3 add X dup mul mul
        -0.373650260707e+4 add X dup mul mul 0.31625993793e+5 add
        X dup mul 0.433493017e+3 add X dup mul mul 0.6325198765e+5 add div X mul
        Signe } ifelse 
      }{ P1 SINO mul Q1 COSO mul add Signe } ifelse 
    } def
    % J1 cardinal au carr� : (J1(x)/x)^2
    /J1Card { dup 0 eq {1}{J1 X div dup mul 4 mul} ifelse } def
    %
    /cm { 28.45 mul } def % centim�tres -> pts
    /m2pt { 284.5 mul } def % m�tres -> pts
    /L { \psk@Diffraction@Slit@Lambda 1e-9 mul } bind def % longueur d'onde en m
    /Coeff { TwoPi r mul L focus mul div } bind def
    /Facteur { 180 d mul L focus mul div } bind def
    /R_limite { 20 Coeff div 100 mul } bind def % en cm
    % 05 08 2004
    \ifPst@Diffraction@Circular@twoHole
      R_limite 10 ge { /R_limite 10 def } if
%      newpath
%      R_limite neg 1.2 mul cm dup moveto
%      R_limite 2.4 mul cm 0 rlineto
%      0 R_limite 2.4 mul cm  rlineto
%      R_limite neg 2.4 mul cm 0 rlineto
%      closepath
%      \ifnum\psk@Diffraction@colorMode=\z@ 1 \else 0 \fi
%      setgray fill
      R_limite neg cm 1 R_limite cm {
        /xPts exch def
        /x { xPts 2845 div } bind def
        /OldIntensity 0 def
        R_limite neg cm 1 R_limite cm {
          /yPts exch def
          \ifPst@Diffraction@IIID 
            xPts 1 sub yPts OldIntensity 
            tx@3DPlotDict begin Conv3D2D end moveto 
          \fi % go to first point
          /y { yPts 2845 div } bind def
          /R { x dup mul y dup mul add sqrt } bind def % R en m
          /m Coeff R mul def
  	  1 1e38 m J1Card Facteur x mul cos dup mul mul neg exp sub
          dup /intensity ED
  	  dup dup
	  Red mul 3 -1 roll
	  Green mul 3 -1 roll
	  Blue mul 
	  \ifcase\psk@Diffraction@colorMode
	    \tx@RGBtoGRAY neg 1 add setgray \or
 	    \tx@RGBtoGRAY setgray \or
	    \tx@RGBtoCMYK setcmykcolor
          \else setrgbcolor \fi
          \ifPst@Diffraction@IIID 
            xPts yPts intensity zScale mul tx@3DPlotDict begin Conv3D2D end 
            lineto stroke 		% draw line in specific color
	    /OldIntensity intensity def
          \else
            xPts 0.5 sub yPts 0.5 sub moveto
            1 0 rlineto
            0 1 rlineto
	    1 neg 0 rlineto
	    closepath fill
  	    stroke
	  \fi
        } for
      } for
    %
    \else 			% only one circular hole
      \ifPst@Diffraction@IIID \else 
        newpath
        R_limite neg %1.5 mul 
	cm dup moveto
        R_limite 2 mul cm 0 rlineto
        0 R_limite 2 mul cm  rlineto
        R_limite neg 2 mul cm 0 rlineto
        closepath
      \fi
      \ifnum\psk@Diffraction@colorMode=\z@ 1 \else 0 \fi setgray fill
       /intensityOld 0 def
       R_limite -0.001 0 {
        /Radius exch def
        /m Coeff Radius 0.01 mul mul def
        1 1e38 m J1Card neg exp sub 
	dup /intensity ED
	dup dup
	Red mul 3 -1 roll 	% R
        Green mul 3 -1 roll 	% G
	Blue mul 		% B
	\ifcase\psk@Diffraction@colorMode
	  \tx@RGBtoGRAY neg 1 add setgray \or
	  \tx@RGBtoGRAY setgray \or
	  \tx@RGBtoCMYK setcmykcolor
        \else setrgbcolor \fi
        \ifPst@Diffraction@IIID 
%    	  Radius cm 0 intensity zScale mul tx@3DPlotDict begin Conv3D2D end moveto
	  /notFilled true def
	  0.1 setlinewidth
	  intensity intensityOld gt {
 	    0 1 360 {
	      /angle ED
     	      Radius cm angle cos mul Radius cm angle sin mul intensity zScale mul 
	      tx@3DPlotDict begin Conv3D2D end 
              angle 0 eq { moveto }{ lineto } ifelse	% draw line in specific color
	    } for
	    notFilled { closepath fill /notFilled false def } if
	    stroke
	    /intensityOld intensity def
	  }{ 
 	    0 1 360 {
	      /angle ED
     	      Radius cm angle cos mul Radius cm angle sin mul intensityOld zScale mul 
	      tx@3DPlotDict begin Conv3D2D end 
              angle 0 eq { moveto }{ lineto } ifelse	% draw line in specific color
	    } for
	    closepath stroke
	  } ifelse	% intensity intensityOld gt 
        \else
          0 0 Radius cm 0 360 arc
          stroke
	\fi
      } for
      \ifPst@Diffraction@showFunc 
      newpath
      0 0 moveto
      R_limite 1.5 mul cm 0 rlineto
      0 R_limite 1.5 mul cm  rlineto
      R_limite neg 1.5 mul cm 0 rlineto
      closepath
      clip
      0.95 setgray fill
      newpath
      0 0 J1Card 1000 mul moveto
      0 0.01 R_limite {
        /Radius exch def
	/m Coeff Radius 0.01 mul mul def
	Radius cm m J1Card 1000 mul lineto
      } for
      \ifnum\psk@Diffraction@colorMode=\z@ 1 \else 0 \fi setgray
      \pst@number\pslinewidth setlinewidth
      stroke
      \pst@number\pslinewidth 2 div setlinewidth
      /grille {
        -10 1 10 {
	  /X exch def
	  X cm -10 cm moveto
	  X cm 10 cm lineto
	  X 0 eq { 0 1 0 }{ 0 0 1 } ifelse
  	  \ifcase\psk@Diffraction@colorMode
	    \tx@RGBtoGRAY neg 1 add setgray \or
	    \tx@RGBtoGRAY setgray \or
 	    \tx@RGBtoCMYK setcmykcolor
          \else setrgbcolor \fi
	  stroke
        } for
        -10 1 10 {
	  /Y exch def
	  -10 cm Y cm moveto
	  10 cm Y cm lineto
	  Y 0 eq { 0 1 0 }{ 0 0 1 } ifelse
	  \ifcase\psk@Diffraction@colorMode
	    \tx@RGBtoGRAY neg 1 add setgray \or
	    \tx@RGBtoGRAY setgray \or
  	    \tx@RGBtoCMYK setcmykcolor
          \else setrgbcolor \fi
	  stroke
	} for
      } def
      newpath
      grille
    \fi
  \fi
  }% \addto@pscode
  \end@SpecialObj%
}
%%
%
\def\psdiffractionTriangle{\pst@object{psdiffractionTriangle}}
\def\psdiffractionTriangle@i{%
  \begin@SpecialObj
  \addto@pscode{%
     0 0 translate
     % les dimensions sont en m�tres
    /f \psk@Diffraction@Slit@F def
    /h \psk@Diffraction@Triangle@h def
    /Gamma \psk@Diffraction@gamma@G def
    /L { \psk@Diffraction@Slit@Lambda 1e-9 mul} bind def % longueur d'onde en m
    /pixel \psk@Diffraction@Slit@pixel def
    /k { TwoPi f L mul div } bind def
    /p { 30 dup sin exch cos div } bind def
    \psk@Diffraction@Slit@Lambda tx@addDict begin wavelengthToRGB Red Green Blue end
    /Blue ED /Green ED /Red ED
    % les bornes sont en pt
    % +- 4 fois le premier minimum
    /bornexpt 1 h div f mul L mul 2845 mul def
    /borneypt 1 h div f mul L mul 2845 mul def
    \ifPst@Diffraction@IIID 
      \psk@ThreeDplot@zMax dup \tx@ScreenCoor pop /zScale ED 
      tx@3DPlotDict begin \IIIDplot@variables end 
    \fi
    /P {
      1 k y mul x p y mul sub mul div 
      1 k h mul x p y mul sub mul RadtoDeg cos sub
      mul
      1 k y mul x p y mul add mul div
      1 k h mul x p y mul add mul RadtoDeg cos sub
      mul sub 
    } def
    /Q {
      1 k y mul x p y mul sub mul div
      k h mul x p y mul sub mul RadtoDeg sin
      mul
      1 k y mul x p y mul add mul div
      k h mul x p y mul add mul RadtoDeg sin
      mul sub
    } def
    /I { P dup mul Q dup mul add } def
    /I_max 0 store
    % Les calculs commencent...
    borneypt 4 mul \psk@Diffraction@Triangle@scale mul dup neg pixel 3 -1 roll {
      /ordonneept exch def
      % y en m
      /y ordonneept 2845 div def
      /OldIntensity 0 def
      bornexpt 4 mul \psk@Diffraction@Triangle@scale mul dup neg pixel 3 -1 roll {
        /abscissept exch def
        \ifPst@Diffraction@IIID 
          abscissept ordonneept OldIntensity 
          tx@3DPlotDict begin Conv3D2D end moveto 
        \fi % go to first point
        % x en m
	/x abscissept 2845 div def
	I_max I le { /I_max I def } if
	%
	1.05 1e\psk@Diffraction@Slit@contrast I neg exp sub
        dup /intensity ED
	dup dup
	Red mul 3 -1 roll Green mul 3 -1 roll Blue mul
	\ifcase\psk@Diffraction@colorMode
	  \tx@RGBtoGRAY neg 1 add setgray \or
	  \tx@RGBtoGRAY setgray \or
	  \tx@RGBtoCMYK setcmykcolor
        \else setrgbcolor \fi
	%
        \ifPst@Diffraction@IIID 
          abscissept ordonneept intensity zScale mul tx@3DPlotDict begin Conv3D2D end 
          lineto stroke 		% draw line in specific color
	  /OldIntensity intensity def
        \else
	  newpath abscissept pixel 2 div sub ordonneept pixel 2 div sub moveto
	  pixel 0 rlineto
	  0 pixel rlineto
	  pixel neg 0 rlineto closepath fill stroke
	\fi
      } for 
    } for 
  }% addto@pscode
  \end@SpecialObj%
}
%
\catcode`\@=\PstAtCode\relax
\endinput
