%--------------------------------------------
% $Header: /cvsroot/pgfplots/pgfplots/generic/pgfplots/util/pgfplotsbinary.code.tex,v 1.13 2009/07/21 18:18:48 ludewich Exp $
%
% Package pgfplots
%
% Provides a user-friendly interface to create function plots (normal
% plots, semi-logplots and double-logplots).
% 
% It is based on Till Tantau's PGF package.
%
% Copyright 2007/2008 by Christian Feuersänger.
%
% This program is free software: you can redistribute it and/or modify
% it under the terms of the GNU General Public License as published by
% the Free Software Foundation, either version 3 of the License, or
% (at your option) any later version.
% 
% This program is distributed in the hope that it will be useful,
% but WITHOUT ANY WARRANTY; without even the implied warranty of
% MERCHANTABILITY or FITNESS FOR A PARTICULAR PURPOSE.  See the
% GNU General Public License for more details.
% 
% You should have received a copy of the GNU General Public License
% along with this program.  If not, see <http://www.gnu.org/licenses/>.
%
%--------------------------------------------

% This file provides a self-contained package which does only need
% pgfkeys.
%
% It provides a method to convert TeX numbers (integers and
% dimensions) into binary format (macros with catcode 11 or 12).
%

\edef\pgfplotsbinaryatcode{\the\catcode`\@ }
\catcode`\@=11


% Returns a single character, which has the
% binary ASCII code '#1', with catcode 11.
%
% #1 (expands to) a number between 0 and 255 (inclusive).
%
% @see \pgfplotsgetchar Note that \pgfplotsgetchar is more powerful,
% but can't be used inside of \edef (it is not expandable) whereas
% \pgfplotscharno is.
\def\pgfplotscharno#1{\csname pgfp@bin@#1\endcsname}%
\let\pgfplotscharno@bincatcode=\pgfplotscharno

\def\pgfplotscharno@lualatex#1{#1,}

\input pgfplotsbinary.data.code.tex

% Defines the LUA (!) value pgfplotsretval to be a binary string
% containing the pgfplots binary value #1.
%
% #1 a pgfplots binary value collected with \pgfplotscharno.
% More precisely, it should be a comma-separated sequence of numbers
% of the form '0,255,2,128,' (can be terminated by comma). It will be
% converted to the respective binary numbers 0x0, 0xff, 0x02,..
% 
% example:
%  \pgfplotsbinarytoluabinary{0, 255,2,128}
%  \directlua{
%  		pdf.immediateobj{"stream", pgfplotsretval,"/DataWithBinaryStream"}
%  }
\def\pgfplotsbinarytoluabinary#1{%
	% lualatex does not support binary chars as pdftex does - so we have to resort to LUA
	% methods. The idea is to use
	% string.char(1,2,3) which results in a binary string with chars 0x01, 0x02, 0x03 etc.
	% I only need to get the integer numbers. To this end, I patch \pgfplotscharno 
	% and create the binary string here:
	\pgfplotsutil@directlua{%
		pgfplotsretval = pgfplotsGetLuaBinaryStringFromCharIndices({#1});
	}%
}%

% Defines \pgfplotsretval to be the ASCII character for #1, with
% catcode 11.
%
% #1: either a number between 0 and 255 (inclusive) or a description
% of the character.
%
% Examples:
% \pgfplotsgetchar{35}
% \pgfplotsgetchar{`\#}   % code for '#'
% \pgfplotsgetchar{`\^^M} % Newline
% \pgfplotsgetchar{`\^^ff}% 255
%
% @see \pgfplotscharno
\def\pgfplotsgetchar#1{%
	\begingroup
	\count0=#1\relax
	\edef\pgfplotsretval{\pgfplotscharno{\the\count0 }}%
	\pgfmath@smuggleone\pgfplotsretval
	\endgroup
}%

\def\pgfplotsbinary@apphighorderbytes@BIGENDIAN#1{\xdef\pgfplotsbinaryresult{#1\pgfplotsbinaryresult}}%
\def\pgfplotsbinary@apphighorderbytes@LITTLEENDIAN#1{\xdef\pgfplotsbinaryresult{\pgfplotsbinaryresult#1}}%

\def\pgfplotsbinaryencode@badic@unsigned@PAD@LITTLEENDIAN{%
	% pad with zeros:
	\ifcase\c@pgfplotsbin@byteno
		% ok.
	\or
		% one byte missing.
		\pgfplotsbinary@apphighorderbytes{\pgfplotsbinary@ZERO@HIGHEST}%
	\or
		% two bytes missing.
		\pgfplotsbinary@apphighorderbytes{\pgfplotsbinary@ZERO\pgfplotsbinary@ZERO@HIGHEST}%
	\or
		\pgfplotsbinary@apphighorderbytes{\pgfplotsbinary@ZERO\pgfplotsbinary@ZERO\pgfplotsbinary@ZERO@HIGHEST}%
	\or
		\pgfplotsbinary@apphighorderbytes{\pgfplotsbinary@ZERO\pgfplotsbinary@ZERO\pgfplotsbinary@ZERO\pgfplotsbinary@ZERO@HIGHEST}%
	\or
		\pgfplotsbinary@apphighorderbytes{\pgfplotsbinary@ZERO\pgfplotsbinary@ZERO\pgfplotsbinary@ZERO\pgfplotsbinary@ZERO\pgfplotsbinary@ZERO@HIGHEST}%
	\or
		\pgfplotsbinary@apphighorderbytes{\pgfplotsbinary@ZERO\pgfplotsbinary@ZERO\pgfplotsbinary@ZERO\pgfplotsbinary@ZERO\pgfplotsbinary@ZERO\pgfplotsbinary@ZERO@HIGHEST}%
	\or
		\pgfplotsbinary@apphighorderbytes{\pgfplotsbinary@ZERO\pgfplotsbinary@ZERO\pgfplotsbinary@ZERO\pgfplotsbinary@ZERO\pgfplotsbinary@ZERO\pgfplotsbinary@ZERO\pgfplotsbinary@ZERO@HIGHEST}%
	\or
		\pgfplotsbinary@apphighorderbytes{\pgfplotsbinary@ZERO\pgfplotsbinary@ZERO\pgfplotsbinary@ZERO\pgfplotsbinary@ZERO\pgfplotsbinary@ZERO\pgfplotsbinary@ZERO\pgfplotsbinary@ZERO\pgfplotsbinary@ZERO@HIGHEST}%
	\else
		\pgfplots@error{Sorry, I can't process byte no \the\c@pgfplotsbin@byteno... you may need to change bytes=\pgfplotsbinary@bytes.}%
	\fi
}%
\def\pgfplotsbinaryencode@badic@unsigned@PAD@BIGENDIAN{%
	% pad with zeros:
	\ifcase\c@pgfplotsbin@byteno
		% ok.
	\or
		% one byte missing.
		\pgfplotsbinary@apphighorderbytes{\pgfplotsbinary@ZERO@HIGHEST}%
	\or
		% two bytes missing.
		\pgfplotsbinary@apphighorderbytes{\pgfplotsbinary@ZERO@HIGHEST\pgfplotsbinary@ZERO}%
	\or
		\pgfplotsbinary@apphighorderbytes{\pgfplotsbinary@ZERO@HIGHEST\pgfplotsbinary@ZERO\pgfplotsbinary@ZERO}%
	\or
		\pgfplotsbinary@apphighorderbytes{\pgfplotsbinary@ZERO@HIGHEST\pgfplotsbinary@ZERO\pgfplotsbinary@ZERO\pgfplotsbinary@ZERO}%
	\or
		\pgfplotsbinary@apphighorderbytes{\pgfplotsbinary@ZERO@HIGHEST\pgfplotsbinary@ZERO\pgfplotsbinary@ZERO\pgfplotsbinary@ZERO\pgfplotsbinary@ZERO}%
	\or
		\pgfplotsbinary@apphighorderbytes{\pgfplotsbinary@ZERO@HIGHEST\pgfplotsbinary@ZERO\pgfplotsbinary@ZERO\pgfplotsbinary@ZERO\pgfplotsbinary@ZERO\pgfplotsbinary@ZERO}%
	\or
		\pgfplotsbinary@apphighorderbytes{\pgfplotsbinary@ZERO@HIGHEST\pgfplotsbinary@ZERO\pgfplotsbinary@ZERO\pgfplotsbinary@ZERO\pgfplotsbinary@ZERO\pgfplotsbinary@ZERO\pgfplotsbinary@ZERO}%
	\or
		\pgfplotsbinary@apphighorderbytes{\pgfplotsbinary@ZERO@HIGHEST\pgfplotsbinary@ZERO\pgfplotsbinary@ZERO\pgfplotsbinary@ZERO\pgfplotsbinary@ZERO\pgfplotsbinary@ZERO\pgfplotsbinary@ZERO\pgfplotsbinary@ZERO}%
	\else
		\pgfplots@error{Sorry, I can't process byte no \the\c@pgfplotsbin@byteno... you may need to change bytes=\pgfplotsbinary@bytes.}%
	\fi
}%

\expandafter\def\csname pgfplotsbinarysetbytes@1\endcsname{%
	\def\pgfplotsbinary@add@signed@largest@absolute{\advance\c@pgfplotsbin@input by 127 }%
	\def\pgfplotsbinaryencodesignedmaplinearly@prepare{%
		% warning: \pgfplotsbinary@bytes is NOT necessarily 1 (ASCII
		% encoding features)
		\c@pgfplotsbin@byteno=\pgfplotsbinary@bytes\relax
		\c@pgfplotsbin@basis=\pgfplotsbinary@basis\relax
		\divide\c@pgfplotsbin@input by16909320 % ~= (2^31-1) /  (2^(8*1-1) -1)
	}%
}%
\expandafter\def\csname pgfplotsbinarysetbytes@2\endcsname{%
	\def\pgfplotsbinary@add@signed@largest@absolute{\advance\c@pgfplotsbin@input by 32767 }%
	\def\pgfplotsbinaryencodesignedmaplinearly@prepare{%
		\c@pgfplotsbin@byteno=\pgfplotsbinary@bytes\relax
		\c@pgfplotsbin@basis=\pgfplotsbinary@basis\relax
		\divide\c@pgfplotsbin@input by65538 % ~= (2^31-1) /  (2^(8*2-1) -1)
	}%
}%
\expandafter\def\csname pgfplotsbinarysetbytes@3\endcsname{%
	\def\pgfplotsbinary@add@signed@largest@absolute{\advance\c@pgfplotsbin@input by 8388607 }%
	\def\pgfplotsbinaryencodesignedmaplinearly@prepare{%
		\c@pgfplotsbin@byteno=\pgfplotsbinary@bytes\relax
		\c@pgfplotsbin@basis=\pgfplotsbinary@basis\relax
		\divide\c@pgfplotsbin@input by256 % ~= (2^31-1) /  (2^(8*3-1) -1)
	}%
}%
\expandafter\def\csname pgfplotsbinarysetbytes@4\endcsname{%
	\def\pgfplotsbinary@add@signed@largest@absolute{%
		\advance\c@pgfplotsbin@input by 2147483647 % this is the *absolute* largest int that TeX can handle.
	}%
	\def\pgfplotsbinaryencodesignedmaplinearly@prepare{%
		\c@pgfplotsbin@byteno=\pgfplotsbinary@bytes\relax
		\c@pgfplotsbin@basis=\pgfplotsbinary@basis\relax
	}%
}%
\def\pgfplotsbinarysetbytes@@{%
	\def\pgfplotsbinaryencodesignedmaplinearly@prepare{%
		\c@pgfplotsbin@byteno=\pgfplotsbinary@bytes\relax
		\c@pgfplotsbin@basis=\pgfplotsbinary@basis\relax
		\pgfplots@error{Sorry, but I can't perform \string\pgfplotsbinaryencodesignedmaplinearly\space for bytes=\pgfplotsbinary@bytes\space yet... bytes=4 is the maximum.}%
	}%
	\def\pgfplotsbinary@add@signed@largest@absolute{
		\advance\c@pgfplotsbin@input by 2147483647 
	}%
}%
\expandafter\let\csname pgfplotsbinary@bytes@5\endcsname=\pgfplotsbinarysetbytes@@
\expandafter\let\csname pgfplotsbinary@bytes@6\endcsname=\pgfplotsbinarysetbytes@@
\expandafter\let\csname pgfplotsbinary@bytes@7\endcsname=\pgfplotsbinarysetbytes@@
\expandafter\let\csname pgfplotsbinary@bytes@8\endcsname=\pgfplotsbinarysetbytes@@

\pgfqkeys{/pgfplots/bin}{%
	% ordering not yet implemented; uses always BIG ENDIAN.
	ordering/.is choice,%
	ordering/big endian/.code={%
		\def\pgfplotsbinary@byteorder{0}%
		\let\pgfplotsbinary@apphighorderbytes=\pgfplotsbinary@apphighorderbytes@BIGENDIAN
		\let\pgfplotsbinaryencode@badic@unsigned@PAD=\pgfplotsbinaryencode@badic@unsigned@PAD@BIGENDIAN
	},%
	ordering/net/.style={/pgfplots/bin/ordering/big endian},%
	ordering/little endian/.code={%
		\def\pgfplotsbinary@byteorder{1}%
		\let\pgfplotsbinary@apphighorderbytes=\pgfplotsbinary@apphighorderbytes@LITTLEENDIAN
		\let\pgfplotsbinaryencode@badic@unsigned@PAD=\pgfplotsbinaryencode@badic@unsigned@PAD@LITTLEENDIAN
	},
	ordering/big endian,%
	%
	% The standard method - it results in binary encoded numbers.
	binary encoding/.code={%
		\pgfutil@IfUndefined{directlua}{%
			\let\pgfplotscharno=\pgfplotscharno@bincatcode
		}{%
			% Ah - we use LuaTeX!
			% At the time of this writing, LUA does not allow binary output which has been
			% created by means of catcode modifications & TeX string concatenation.
			% binary output in LUA needs to be (re)implemented in LUA (see inline code 
			% comments below).
			%
			% There are two possible work-arounds: 
			% (a) Base64 encoding
			% (b) binary encoding using special LUA handling. 
			% This is what I do. Set the 'encode filter' such that it reinitializes the encoder:
			% we patch \pgfplotscharno with a special routine which collects
			% only the integer indices:
			\let\pgfplotscharno=\pgfplotscharno@lualatex
			% later, the user has to convert this list into a binary lua
			% string before he can use it. See \pgfplotsbinarytoluabinary
		}%
		\edef\pgfplotsbinary@ZERO{\pgfplotscharno0}%
		\edef\pgfplotsbinary@ZERO@LINEARMAP{\pgfplotscharno{128}}%
		\let\pgfplotsbinary@ZERO@HIGHEST=\pgfplotsbinary@ZERO
		\def\pgfplotsbinary@basis{256}%
		\let\pgfplotsbinary@hook=\relax
		\def\pgfplotsbinary@hook@signed@linearmap{%
			\ifnum\c@pgfplotsbin@byteno=0
				\advance\c@pgfplotsbin@input by128
				\ifnum\c@pgfplotsbin@input>255
					\pgfplotsbinary@hook@signed@linearmap@error
				\fi
			\fi
		}%
		\def\pgfplotsbinarysetbytes##1{%
			\pgfutil@ifundefined{pgfplotsbinarysetbytes@##1}{%
				\pgfplots@error{Sorry, I can't write binary output with '##1' bytes yet...}%
			}{%
				\edef\pgfplotsbinary@bytes{##1}%
				\csname pgfplotsbinarysetbytes@##1\endcsname
			}%
		}%
	},%
	%
	% This applies 'binary encoding' and encodes the resulting bytes
	% in Hex. It corresponds to the ASCIIHexEncode in postscript or
	% pdf.
	% Please note that 'bytes' sets the number of binary bytes - the
	% actual encoding length is exactly twice as large.
	ASCIIHexEncode/.code={%
		\let\pgfplotscharno=\pgfplotscharno@bincatcode
		\edef\pgfplotsbinary@ZERO{\pgfplotscharno{48}}%
		\edef\pgfplotsbinary@ZERO@LINEARMAP{\pgfplotscharno{56}}%
		\let\pgfplotsbinary@ZERO@HIGHEST=\pgfplotsbinary@ZERO
		\pgfkeysalso{/pgfplots/bin/ordering/big endian}%
		\def\pgfplotsbinary@basis{16}%
		\def\pgfplotsbinary@hook@hex{%
			\ifnum\c@pgfplotsbin@input<10
				\advance\c@pgfplotsbin@input by48
			\else
				\advance\c@pgfplotsbin@input by55
			\fi
		}%
		\let\pgfplotsbinary@hook=\pgfplotsbinary@hook@hex
		\def\pgfplotsbinary@hook@signed@linearmap{%
			\ifnum\c@pgfplotsbin@byteno=0
				\advance\c@pgfplotsbin@input by8
				\ifnum\c@pgfplotsbin@input>16
					\pgfplotsbinary@hook@signed@linearmap@error
				\fi
			\fi
			\pgfplotsbinary@hook@hex
		}%
		\def\pgfplotsbinarysetbytes##1{%
			\pgfutil@ifundefined{pgfplotsbinarysetbytes@##1}{%
				\pgfplots@error{Sorry, I can't write binary output with '##1' bytes yet...}%
			}{%
				\csname pgfplotsbinarysetbytes@##1\endcsname
				\begingroup
					\count0=##1\relax
					\multiply\count0 by2
					\xdef\pgfplotsbinary@glob@TMP{\the\count0 }%
				\endgroup
				\let\pgfplotsbinary@bytes=\pgfplotsbinary@glob@TMP
			}%
		}%
	},%
	%
	% 
	% This applies 'binary encoding' and encodes the resulting bytes
	% using a base 85 encoding. It corresponds to the ASCII85Encode in postscript or
	% pdf.
	% Handle this method with care - it works just for ONE number, not for a stream of
	% numbers as in pdf. Therefore, it might not be useful at all.
	% Please note that 'bytes' will be ignored; ASCII85Encode assumes
	% 4 binary bytes and uses 5 bytes to encode them.
	%
	% @ATTENTION bytes is ALWAYS 4, regardless of the setting of
	% 'bytes'!
	ASCII85Encode/.code={%
		\let\pgfplotscharno=\pgfplotscharno@bincatcode
		\edef\pgfplotsbinary@ZERO{\pgfplotscharno{33}}%
		\edef\pgfplotsbinary@ZERO@LINEARMAP{\pgfplotscharno{42}}%
		\let\pgfplotsbinary@ZERO@HIGHEST=\pgfplotsbinary@ZERO
		\pgfkeysalso{/pgfplots/bin/ordering/big endian}%
		\edef\pgfplotsbinary@ASCII@specialzero{\pgfplotsbinary@ZERO\pgfplotsbinary@ZERO\pgfplotsbinary@ZERO\pgfplotsbinary@ZERO\pgfplotsbinary@ZERO}%
		\expandafter\def\expandafter\pgfplotsbinaryencode@badic@unsigned@PAD\expandafter{%
			\pgfplotsbinaryencode@badic@unsigned@PAD
			\ifx\pgfplotsbinaryresult\pgfplotsbinary@ASCII@specialzero
				% PDF standard: 0 is encoded as 'z':
				\gdef\pgfplotsbinaryresult{z}%
			\fi
		}%
		\def\pgfplotsbinary@basis{85}%
		\def\pgfplotsbinary@hook{%
			\advance\c@pgfplotsbin@input by33
		}%
		\def\pgfplotsbinary@hook@signed@linearmap{%
			\advance\c@pgfplotsbin@input by33
			\ifnum\c@pgfplotsbin@byteno=0
				\advance\c@pgfplotsbin@input by42
				\ifnum\c@pgfplotsbin@input>85
					\pgfplotsbinary@hook@signed@linearmap@error
				\fi
			\fi
		}%
		% I know, that does only work efficiently if bytes=4 for every
		% encoded number.
		\def\pgfplotsbinarysetbytes##1{%
			\def\pgfplotsbinary@bytes{5}%
			\csname pgfplotsbinarysetbytes@4\endcsname
		}%
		\pgfplotsbinarysetbytes4%
	},%
	binary encoding,%
	%
	bytes/.code={\pgfplotsbinarysetbytes{#1}},%
	bytes=4,
	%
	% Irreversibly change to VERBATIM output for debugging:
	debug mode/.code={%
		\let\pgfplotsbinary@apphighorderbytes@ORIG=\pgfplotsbinary@apphighorderbytes
		\def\pgfplotsbinary@ZERO{[Pad-0]}%
		\let\pgfplotsbinary@ZERO@HIGHEST=\pgfplotsbinary@ZERO
		\def\pgfplotsbinary@ZERO@LINEARMAP{[Pad-128]}%
		\def\pgfplotsbinary@apphighorderbytes##1{%
			\pgfutil@ifnextchar\pgfplotscharno{%
				\pgfplotsbinary@apphighorderbytes@DEBUG@csname
			}{%
				\pgfplotsbinary@apphighorderbytes@DEBUG@normal
			}%
			##1\relax
		}%
	},%
	% Write pdf objects in binary form. This does only work with
	% pdftex, and its output is only useful in conjunction with
	% \pdfcompresslevel=0
	% and a text editor.
	% Usage:
	% \pgfkeys{/pgfplots/bin/debug to pdf={\pgfplotsbinaryencodeunsigned}{1,2,3,...,16}}
	%
	% works only with pdftex
	debug to pdf/.code 2 args={%
		\foreach \num in {#2} {%
			#1{\num}%
			\immediate \pdfobj stream attr {
				/Decimal \num\space
				/Routine (\string#1)
			} {%
				\pgfplotsbinaryresult
			}%
		}%
	},%
}


\def\pgfplotsbinary@hook@signed@linearmap@error{%
	\pgfplots@error{Sorry, there are not enough bytes to store the current number. I tried to write \the\c@pgfplotsbin@input...}%
}%

\def\pgfplotsbinary@apphighorderbytes@DEBUG@csname\pgfplotscharno#1\relax{%
	\pgfplotsbinary@apphighorderbytes@ORIG{[#1]}%
}%
\def\pgfplotsbinary@apphighorderbytes@DEBUG@normal#1\relax{%
	\pgfplotsbinary@apphighorderbytes@ORIG{#1}%
}%

%\pgfkeys{/pgfplots/bin/debug mode}


\countdef\c@pgfplotsbin@input=0
\countdef\c@pgfplotsbin@tmpa=1
\countdef\c@pgfplotsbin@tmpb=2
\countdef\c@pgfplotsbin@byteno=3
\countdef\c@pgfplotsbin@basis=4
\def\pgfplotsbinaryempty{}

%--------------------------------------------------
% input:    unsigned int x, b, n;
% output:   unsigned int y[n];
% for (i=0; i<n; i++) y[i] = 0;
% i=0;
% while (x > 0)
% { y[i] = x % b; /* entspricht x mod b   */
%  x = x / b;    /* ganzzahlige Division */
%  i++;
%  }
%-------------------------------------------------- 
% with x = #1
% b = basis
% will store stuff into \pgfplotsbinaryresult in binary format
%
% PRECONDITION:
% - \pgfplotsbinaryresult= empty!
% - \c@pgfplotsbin@byteno=\pgfplotsbinary@bytes
\def\pgfplotsbinaryencode@badic@unsigned@{%
	\ifnum\c@pgfplotsbin@input>0
		\c@pgfplotsbin@tmpa=\c@pgfplotsbin@input
		\divide\c@pgfplotsbin@tmpa by\c@pgfplotsbin@basis\relax
		\c@pgfplotsbin@tmpb=\c@pgfplotsbin@tmpa\relax
		\multiply\c@pgfplotsbin@tmpa by\c@pgfplotsbin@basis\relax
		\advance\c@pgfplotsbin@input by -\c@pgfplotsbin@tmpa\relax
		\advance\c@pgfplotsbin@byteno by-1
		\pgfplotsbinary@hook% hooks for modifications.
		\pgfplotsbinary@apphighorderbytes{\pgfplotscharno{\the\c@pgfplotsbin@input}}%
		\c@pgfplotsbin@input=\c@pgfplotsbin@tmpb
%\message{RESULT SO FAR byte no \the\c@pgfplotsbin@byteno: \pgfplotsbinaryresult}%
		\expandafter\pgfplotsbinaryencode@badic@unsigned@
	\else
		\pgfplotsbinaryencode@badic@unsigned@PAD
%\message{RESULT SO FAR byte no \the\c@pgfplotsbin@byteno: \pgfplotsbinaryresult}%
	\fi
}%


% Defines \pgfplotsbinaryresult to be the binary representation of an
% unsigned integer.
%
% The representation will use unsigned dual number representation.
%
% The assignment to \pgfplotsbinaryresult will be globally.
% #1: an unsigned integer. It won't be transformed in any way, so make
% sure it fits into the configured number of bytes. It is an error if
% the number is too large or too small. Please note that only unsigned
% numbers are supported with this method.
%
% FIXME : fix  > 2^30
\def\pgfplotsbinaryencodeunsigned#1{%
	\begingroup
	\global\let\pgfplotsbinaryresult=\pgfplotsbinaryempty
	\c@pgfplotsbin@input=#1 %
	\c@pgfplotsbin@byteno=\pgfplotsbinary@bytes\relax
	\c@pgfplotsbin@basis=\pgfplotsbinary@basis\relax
	\pgfplotsbinaryencode@badic@unsigned@%
	\endgroup
}%

% An implementation for signed integers which maps the signed integer linearly into 
% the unsigned data range before it proceeds.
%
% The idea is thus, to first introduce a linear mapping
%
% phi : [- smallest_possible, +largest_possible ] -> [0, 256^bytes-1 ]
%
% A signed integer in TeX is in [ - (2^31-1), 2^31-1 ].
% Thus, we should map
%
% phi : [  -(2^31-1), 2^31-1 ] -> [ 0, 2^32-1 ].
%
% A simpler case is to employ the symmetry in TeX's registers and
% leave one out, i.e. to map to 2^32-2:
%
% phi : [  -(2^31-1), 2^31-1 ] -> [ 0, 2^32-2 ].
%
% Then,
%
% phi(x) = ( x + 2^31 -1 ) / (2^31-1 + 2^31-1) * (2^32-2) = x+ 2^31-1.
%
% The same map phi(x) = x + 2^31 -1 with target space  [0, 2^32-1 ]
% could be realized with the input space [- (2^31-1), 2^31 ].
%
% I am using this encoding procedure, phi(x) = x + 2^31 -1.
%
% As a consequence, the binary pattern FF FF FF FF does never occur as
% result of the mapping.
%
% To invert the mapping (i.e. to decode the result), set up the unique
% linear map
%
% psi : [ 0, 2^32-1 ] -> [ -(2^31-1), 2^31 ].
%
% Then, psi( phi(x) ) = x and the decoding procedure is correct.
%
% This doesn't need TeX register arithmetics on the whole range.
%
% REMARK: the whole operation does also work if bytes<4 (i.e. we have
% less than 32 bits in the target range). In this case, the mapping is
% phi : [  -(2^31-1), 2^31-1 ] -> [ 0, 2^{8*bytes}-1 ]
% and a further, *lossy* quantization still will be applied. The
% quantization step is an integer division performed in signed number
% arithmetics (i.e. it is symmetric around 0). 
\def\pgfplotsbinaryencodesignedmaplinearly#1{%
	\begingroup
	\global\let\pgfplotsbinaryresult=\pgfplotsbinaryempty
	\c@pgfplotsbin@input=#1\relax%
	\pgfplotsbinaryencodesignedmaplinearly@prepare
	\ifnum\c@pgfplotsbin@input<0 %
		% compute + 2^31 - 1
		\pgfplotsbinary@add@signed@largest@absolute
	\else
		% change zero padding such that positive numbers
		% get the EFFECT of + 2^31.
		\let\pgfplotsbinary@ZERO@HIGHEST=\pgfplotsbinary@ZERO@LINEARMAP
		\let\pgfplotsbinary@hook=\pgfplotsbinary@hook@signed@linearmap
		% and compute the -1 explicitly here:
		\advance\c@pgfplotsbin@input by-1 %
	\fi
	\pgfplotsbinaryencode@badic@unsigned@%
	\endgroup
}%

% Encodes a dimen (like 1pt or \dimen0) in binary form.
%
% The encoding works by mapping #1 linearly into the allowed integer
% range using a quantization technique to respect the (possibly)
% restricted number of bytes.
%
% The implementation is fast and uses only integer arithmetics.
% It relies on \pgfplotsbinaryencodesignedmaplinearly and a scale.
%
% So, what we do is to setup a linear map into binary range with k
% bytes. The range of a TeX dimen is precisely (in units of pt)
%  [ -(2^30-1) / 2^16, (2^30 -1) / 2^16 ] = [-16383.99998, 16383.99998]
%
% Thus, for an input dimen x, we set up the mapping
% 	phi(x) = 2^16 * x * 2
% which maps 
%  phi: [ -(2^30-1) / 2^16, (2^30-1) / 2^16 ] -> [-(2^31 -2), 2^31-2].
%
% I simply use the \pgfplotsbinaryencodesignedmaplinearly to process
% this further. To simplify the computation, I simply compute
%   phi_signed( phi(x) ), 
% where phi_signed denotes an application of
% \pgfplotsbinaryencodesignedmaplinearly:
%   phi_signed( y ) = y +2^31 -1,
%   phi_signed( phi(x) ) = 2^16 * 2 * x + 2^31 - 1.
% This is NOT a linear map to [0,2^32-1] as promised. 
% But, we can setup an inverse transformation PHI (which is linear) anyway
% such that
%   PHI( phi_signed(phi(x)) ) = x
% and that's all I want. Do do that, we use the unique linear decoder map 
%   PHI : [ 0,2^32-1 ] -> [ -16383.999992, 16384 ].
%
% This is not exacty the input range of before, but using it results
% in a proper decoder. The difference is due to the non-unique zero
% representation in TeX's arithmetics.
% 
%
% REMARK: the whole operation does also work if bytes<4 (i.e. we have
% less than 32 bits in the target range). In this case, a further
% *lossy* quantization step is applied in phi_signed. The inverse
% transformations are the same, however. See
% \pgfplotsbinaryencodesignedmaplinearly for details about the
% quantization step (or try it out).
%
	%
	%% DEBUG NOTE: This mapping appears to work correctly according to
	%% my tests.
	%% For bc -l test codes: 
	%% ibase=16; 
	%% -4000 +  809658FA. / (2^20) * 8000
\def\pgfplotsbinaryencodedimenmaplinearly#1{%
	\begingroup
	\dimen0=#1\relax
	\c@pgfplotsbin@input=\dimen0
	\multiply\c@pgfplotsbin@input by2
%\message{LOWLEVEL ENCODING '\the\c@pgfplotsbin@input' with linear map}%
	\pgfplotsbinaryencodesignedmaplinearly\c@pgfplotsbin@input
	\endgroup
}%

\catcode`\@=\pgfplotsbinaryatcode
\endinput
