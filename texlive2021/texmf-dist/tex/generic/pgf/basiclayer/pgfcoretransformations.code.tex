% Copyright 2018 by Till Tantau
%
% This file may be distributed and/or modified
%
% 1. under the LaTeX Project Public License and/or
% 2. under the GNU Public License.
%
% See the file doc/generic/pgf/licenses/LICENSE for more details.

\ProvidesFileRCS{pgfcoretransformations.code.tex}


% Position transformations

\newdimen\pgf@pt@x
\newdimen\pgf@pt@y

\newdimen\pgf@pt@temp

\newif\ifpgf@pt@identity
\pgf@pt@identitytrue

\def\pgf@pt@aa{1.0} \def\pgf@pt@ab{0.0}
\def\pgf@pt@ba{0.0} \def\pgf@pt@bb{1.0}



% Store the current transformation matrix in a macro.
%
% #1 = macro name
%
% Example:
%
% \pgfgettransform\mytransform

\def\pgfgettransform#1{%
  \edef#1{{\pgf@pt@aa}{\pgf@pt@ab}{\pgf@pt@ba}{\pgf@pt@bb}{\the\pgf@pt@x}{\the\pgf@pt@y}}%
}

% Store the current transformation matrix in a set of macros, one for
% each single entry.
%
% #1-#4 are macro which store the matrix entries (0,0), (0,1), (1,0),
% (1,1) (in this order). It is the same order as for \pgftransformcm.
%
% #5 is a macro which will be filled with the x shift and
% #6 is a macro which will be filled with the y shift.
%
% Example:
% \pgfgettransform\aa\ab\ba\bb\shiftx\shifty
%
% ->
%
% \pgftransformcm\aa\ab\ba\bb{\pgfqpoint{\shiftx}{\shifty}}
%
% or
%
% \pgfsettransformentries\aa\ab\ba\bb\shiftx\shifty
\def\pgfgettransformentries#1#2#3#4#5#6{%
  \edef#1{\pgf@pt@aa}%
  \edef#2{\pgf@pt@ab}% attention: this is (1,0), i.e. the point which is usually labelled 'ba'!
  \edef#3{\pgf@pt@ba}% attention: this is (0,1), i.e. the point which is usually labelled 'ab'!
  \edef#4{\pgf@pt@bb}%
  \edef#5{\the\pgf@pt@x}%
  \edef#6{\the\pgf@pt@y}%
}%
\def\pgfsettransformentries#1#2#3#4#5#6{%
  \pgfsettransform{{#1}{#2}{#3}{#4}{#5}{#6}}%
}%

% Restore the current transformation matrix from a macro.
%
% #1 = macro name, previously used with \pgfgettransform
%
% Example:
%
% \pgfsettransform\mytransform

\def\pgfsettransform#1{%
  \edef\pgf@temp{#1}%
  \expandafter\pgf@settransform\pgf@temp%
}
\def\pgf@settransform#1#2#3#4#5#6{%
  \def\pgf@pt@aa{#1}%
  \def\pgf@pt@ab{#2}%
  \def\pgf@pt@ba{#3}%
  \def\pgf@pt@bb{#4}%
  \pgf@pt@x=#5%
  \pgf@pt@y=#6%
  \edef\pgf@idtest{#1,#2,#3,#4}%
  \ifx\pgf@idtest\pgf@idmatrixtext%
    \pgf@pt@identitytrue%
  \else%
    \pgf@pt@identityfalse%
  \fi%
}

\def\pgf@idmatrixtext{1.0,0.0,0.0,1.0}
\def\pgf@zerozerotext{0.0,0.0}
\def\pgf@one@text{1.0}
\def\pgf@zero@text{0.0}

% Transformation command
%
% #1 = x dimension
% #2 = y dimension
%
% Description:
%
% Applies the pgfs transformation matrix to the point (#1,#2). The
% result is once more stored in the point.
%
%
% Example:
%
% \pgf@pos@transform{\pgf@x}{\pgf@y}

\def\pgf@pos@transform#1#2{%
  \ifpgf@pt@identity%
  \else%
    \pgf@pt@temp=#1%
    #1=\pgf@pt@aa#1%
    \advance#1 by\pgf@pt@ba#2%
    #2=\pgf@pt@bb#2%
    \advance#2 by\pgf@pt@ab\pgf@pt@temp%
  \fi%
  \advance#1 by\pgf@pt@x%
  \advance#2 by\pgf@pt@y%
}
\def\pgf@pos@transform@glob{%
  \ifpgf@pt@identity%
  \else%
    \pgf@pt@temp=\pgf@x%
    \global\pgf@x=\pgf@pt@aa\pgf@x%
    \global\advance\pgf@x by\pgf@pt@ba\pgf@y%
    \global\pgf@y=\pgf@pt@bb\pgf@y%
    \global\advance\pgf@y by\pgf@pt@ab\pgf@pt@temp%
  \fi%
  \global\advance\pgf@x by\pgf@pt@x%
  \global\advance\pgf@y by\pgf@pt@y%
}


% Invert the current transformation matrix
%
% Description:
%
% This command replaces the current transformation matrix by its
% inverse. The inversion is not very exact if the matrix is nearly
% singular.
%
% Example:
%
% \pgftransforminvert

\def\pgftransforminvert{%
  % First, invert the shift. That's easy
  \pgf@pt@x=-\pgf@pt@x
  \pgf@pt@y=-\pgf@pt@y
  \ifpgf@pt@identity%
  \else%
    % Sigh. Have to calculate the inverse of the matrix.
    % Perhaps we are lucky and ba and bb are zero?
    \edef\pgf@zerotest{\pgf@pt@ab,\pgf@pt@ba}%
    \ifx\pgf@zerotest\pgf@zerozerotext%
      % Ha! Just invert aa and bb
      \pgfmathreciprocal@{\pgf@pt@aa}%
      \let\pgf@pt@aa\pgfmathresult%
      \pgfmathreciprocal@{\pgf@pt@bb}%
      \let\pgf@pt@bb\pgfmathresult%
    \else%
      % Start with determinant
      % Matrix = (aa ab
      %           ba bb)
      {%
        \pgfutil@tempdima=\pgf@pt@aa pt%
        \pgfutil@tempdima=\pgf@pt@bb\pgfutil@tempdima% aa * bb
        \pgfutil@tempdimb=\pgf@pt@ba pt%
        \pgfutil@tempdimb=\pgf@pt@ab\pgfutil@tempdimb% ba * ab
        \advance\pgfutil@tempdima by -\pgfutil@tempdimb% aa*bb - ba*ab
        % Invert
        \pgfmathreciprocal@{\pgf@sys@tonumber{\pgfutil@tempdima}}%
        \pgfutil@tempdima=\pgfmathresult pt%
        \pgf@x=\pgf@pt@bb\pgfutil@tempdima% bb/(det A)
        \xdef\pgf@new@pt@aa{\pgf@sys@tonumber{\pgf@x}}%
        \pgf@x=\pgf@pt@aa\pgfutil@tempdima% aa/(det A)
        \xdef\pgf@new@pt@bb{\pgf@sys@tonumber{\pgf@x}}%
        \pgfutil@tempdima=-\pgfutil@tempdima%
        \pgf@x=\pgf@pt@ba\pgfutil@tempdima% -ba/(det A)
        \xdef\pgf@new@pt@ba{\pgf@sys@tonumber{\pgf@x}}%
        \pgf@x=\pgf@pt@ab\pgfutil@tempdima% -ab/(det A)
        \xdef\pgf@new@pt@ab{\pgf@sys@tonumber{\pgf@x}}%
      }%
      \let\pgf@pt@aa\pgf@new@pt@aa%
      \let\pgf@pt@ab\pgf@new@pt@ab%
      \let\pgf@pt@ba\pgf@new@pt@ba%
      \let\pgf@pt@bb\pgf@new@pt@bb%
    \fi%
    \pgf@pt@temp=\pgf@pt@x%
    \pgf@pt@x=\pgf@pt@aa\pgf@pt@x%
    \advance\pgf@pt@x by\pgf@pt@ba\pgf@pt@y%
    \pgf@pt@y=\pgf@pt@bb\pgf@pt@y%
    \advance\pgf@pt@y by\pgf@pt@ab\pgf@pt@temp%
  \fi%
}




% PGF-Level concatenation of the transformation matrix with a given
% matrix.
%
% #1 - #5 = a concatenation matrix (see pdf spec). Entry #5 is a
%           translation point.
%
% Description:
%
% All coordinates given to subsequent path construction commands will
% be transformed additionally by the given transformation matrix. If
% the command is followed by a 6th argument, the transformation is
% only applied to this argument.
%
%
% Example:
%
% \pgftransformcm{1}{0}{0}{1}{\pgfpoint{100pt}{0pt}} % 100pt to the right.
%
% \pgftransformcm{2}{0}{0}{2}{\pgfpointorigin} % double in size
% \pgfpathmoveto{\pgfpoint{0cm}{0cm}}
% \pgfpathlineto{\pgflineto{1cm}{1cm}} % actually 2cm/2cm

\def\pgftransformcm#1#2#3#4#5{%
  \edef\pgf@marshal{\noexpand\pgf@transformcm{#1}{#2}{#3}{#4}}%
  \pgf@marshal{#5}%
}
\def\pgf@transformcm#1#2#3#4#5{%
  {%
    \pgf@process{#5}%
    \pgf@xc=\pgf@x%
    \pgf@yc=\pgf@y%
    %
    \pgfmathsetlength\pgf@x{#1}%
    \pgfmathsetlength\pgf@y{#2}%
    %
    \pgf@xa=\pgf@pt@aa\pgf@x%
    \advance\pgf@xa by \pgf@pt@ba\pgf@y%
    %
    \pgf@ya=\pgf@pt@ab\pgf@x%
    \advance\pgf@ya by \pgf@pt@bb\pgf@y%
    %
    \pgfmathsetlength\pgf@x{#3}%
    \pgfmathsetlength\pgf@y{#4}%
    %
    \pgf@xb=\pgf@pt@aa\pgf@x%
    \advance\pgf@xb by \pgf@pt@ba\pgf@y%
    %
    \pgf@yb=\pgf@pt@ab\pgf@x%
    \advance\pgf@yb by \pgf@pt@bb\pgf@y%
    %
    \pgf@x=\pgf@pt@aa\pgf@xc%
    \advance\pgf@x by\pgf@pt@ba\pgf@yc%
    \advance\pgf@x by\pgf@pt@x%
    %
    \pgf@y=\pgf@pt@ab\pgf@xc%
    \advance\pgf@y by\pgf@pt@bb\pgf@yc%
    \advance\pgf@y by\pgf@pt@y%
    %
    \global\pgf@x=\pgf@x%
    \global\pgf@y=\pgf@y%
    \xdef\pgf@tempaa{\pgf@sys@tonumber{\pgf@xa}}%
    \xdef\pgf@tempab{\pgf@sys@tonumber{\pgf@ya}}%
    \xdef\pgf@tempba{\pgf@sys@tonumber{\pgf@xb}}%
    \xdef\pgf@tempbb{\pgf@sys@tonumber{\pgf@yb}}%
  }%
  \pgf@pt@x=\pgf@x%
  \pgf@pt@y=\pgf@y%
  \let\pgf@pt@aa=\pgf@tempaa%
  \let\pgf@pt@ba=\pgf@tempba%
  \let\pgf@pt@ab=\pgf@tempab%
  \let\pgf@pt@bb=\pgf@tempbb%
  \edef\pgf@idtest{\pgf@pt@aa,\pgf@pt@ba,\pgf@pt@ab,\pgf@pt@bb}%
  \ifx\pgf@idtest\pgf@idmatrixtext%
    \pgf@pt@identitytrue%
  \else%
    \pgf@pt@identityfalse%
  \fi%
}



% Transformation into a given triangle. The three corners of the
% triangle are called "origin", "x" and "y" are given. After this
% transformation has been applied, the canvas origin is at "origin",
% the vector (1pt,0pt) lies at "x" and the vector (0pt,1pt) lies at
% "y".
%
% #1 = origin
% #2 = x
% #3 = y
%
% Example:
%
% \pgftransformtriangle{\pgfpointorigin}{\pgfpoint{1cm}{1cm}}{\pgfpoint{-1cm}{1cm}}

\def\pgftransformtriangle#1#2#3{%
  \pgf@process{#2}%
  \pgf@xa=\pgf@x%
  \pgf@ya=\pgf@y%
  \pgf@process{#3}%
  \pgf@xb=\pgf@x%
  \pgf@yb=\pgf@y%
  \pgf@process{#1}%
  \advance\pgf@xa by-\pgf@x%
  \advance\pgf@ya by-\pgf@y%
  \advance\pgf@xb by-\pgf@x%
  \advance\pgf@yb by-\pgf@y%
  \pgftransformcm%
  {\pgf@sys@tonumber\pgf@xa}{\pgf@sys@tonumber\pgf@ya}%
  {\pgf@sys@tonumber\pgf@xb}{\pgf@sys@tonumber\pgf@yb}%
  {\pgfpoint{\pgf@x}{\pgf@y}}%
}



% Undo all transformations
%
% Example:
%
% \pgftansformreset

\def\pgftransformreset{%
  \pgf@pt@x=0pt%
  \pgf@pt@y=0pt%
  \let\pgf@pt@aa\pgf@one@text%
  \let\pgf@pt@ba\pgf@zero@text%
  \let\pgf@pt@ab\pgf@zero@text%
  \let\pgf@pt@bb\pgf@one@text%
  \pgf@pt@identitytrue%
}


% Undo all slanting/rotation, but not translations
%
% Example:
%
% \pgftansformreset

\def\pgftransformresetnontranslations{%
  \let\pgf@pt@aa\pgf@one@text%
  \let\pgf@pt@ba\pgf@zero@text%
  \let\pgf@pt@ab\pgf@zero@text%
  \let\pgf@pt@bb\pgf@one@text%
  \pgf@pt@identitytrue%
}




% Shifting command and environment
%
% #1 = vector by which all subsequent points should be moved.
%
% Example:
%
% \pgftansformshift{\pgfpoint{1cm}{1cm}}

\def\pgftransformshift#1{\pgftransformcm{1}{0}{0}{1}{#1}}
\def\pgftransformxshift#1{\pgftransformcm{1}{0}{0}{1}{\pgfpoint{#1}{+0pt}}}
\def\pgftransformyshift#1{\pgftransformcm{1}{0}{0}{1}{\pgfpoint{+0pt}{#1}}}




% Scaling commands
%
% #1 = scaling for all subsequent points.
%
% Example:
%
% \pgftransformscale{2}
% \pgftransformxscale{2}
% \pgftransformyscale{2}

\def\pgftransformscale#1{\pgftransformcm{#1}{0}{0}{#1}{\pgfpointorigin}}
\def\pgftransformxscale#1{\pgftransformcm{#1}{0}{0}{1.0}{\pgfpointorigin}}
\def\pgftransformyscale#1{\pgftransformcm{1.0}{0}{0}{#1}{\pgfpointorigin}}



% Slanting commands
%
% #1 = slanting in x-direction. 1 means 45 degrees.
%
% Example:
%
% \pgftransformxslant{2}
% \pgftransformyslant{2}

\def\pgftransformxslant#1{\pgftransformcm{1.0}{0}{#1}{1.0}{\pgfpointorigin}}
\def\pgftransformyslant#1{\pgftransformcm{1.0}{#1}{0}{1.0}{\pgfpointorigin}}



% Rotation commands
%
% #1 = degrees for the rotation
%
% Example:
%
% \pgftransformrotate{30}

\def\pgftransformrotate#1{%
  \pgfmathparse{#1}%
  \let\pgftransform@angle=\pgfmathresult%
  \pgfmathsin@{\pgftransform@angle}%
  \let\pgftransform@sin=\pgfmathresult%
  \pgfmathcos@{\pgftransform@angle}%
  \let\pgftransform@cos=\pgfmathresult%
  \pgf@x=\pgftransform@sin pt%
  \pgf@xa=-\pgf@x%
  \pgftransformcm{\pgftransform@cos}{\pgftransform@sin}{\pgf@sys@tonumber{\pgf@xa}}{\pgftransform@cos}{\pgfpointorigin}%
}



% The following if's have the following effect:
%
% \ifpgfslopedattime decides whether the coordinate system of an
% xxxattime transformation command should be rotated such that text
% drawn on the line should be tangent to the line
%
% \ifpgfallowupsidedowattime decides whether the transformation should
% be done in such a way that the text is always ``upright,'' that is,
% text drawn in this coordinate system is never upside-down.
%
% \ifpgfresetnontranslationattime decides whether the coordinate
% system of an xxxattime transformation command should be reset
% (concerning the non-translations) before a possible rotation is
% applied. This is useful, for example, if the main coordinate system
% is scaled by, say, a factor of 2 and you do not want that to apply
% to the text, but you do want the rotation to apply.

\newif\ifpgfslopedattime
\newif\ifpgfallowupsidedownattime
\newif\ifpgfresetnontranslationattime


% Transform to the coordinate system of a point on a line
%
% #1 = a time fraction of line where to put the label, where 0 means start,
%      1 means end, and for example 0.5 means the middle.
% #2 = start of line
% #3 = end of line
%
% Example:
%
% {
%   \pgftransformlineattime{.5}{\pgfxy(0,0)}{\pgfxy(3,2)}
%   \pgftext{Hi!}
% }
%
% {
%   \pgftransformlineattime{.75}{\pgfxy(0,0)}{\pgfxy(3,2)}
%   \pgftransformresetnontranslations
%   \pgftext{Hi!}
% }
%
\def\pgftransformlineattime#1#2#3{%
  \pgf@process{#2}%
  \pgf@xb=\pgf@x% xb/yb = start point
  \pgf@yb=\pgf@y%
  \pgf@process{#3}%
  \pgf@xc=\pgf@x% xc/yc = end point
  \pgf@yc=\pgf@y%
  \pgftransformshift{\pgfpointlineattime{#1}{\pgfqpoint{\pgf@xb}{\pgf@yb}}{\pgfqpoint{\pgf@xc}{\pgf@yc}}}%
  \ifpgfresetnontranslationattime%
    \pgftransformresetnontranslations%
  \fi%
  \ifpgfslopedattime%
    \advance\pgf@xc by-\pgf@xb%
    \advance\pgf@yc by-\pgf@yb%
    \ifpgfallowupsidedownattime%
    \else%
      \ifdim\pgf@xc<0pt%
        \pgf@xc=-\pgf@xc%
        \pgf@yc=-\pgf@yc%
      \fi%
    \fi%
    \pgf@x=\pgf@xc%
    \pgf@y=\pgf@yc%
    \pgfpointnormalised{}% x/y = normalised vector
    \pgf@ya=-\pgf@y%
    \pgftransformcm%
    {\pgf@sys@tonumber{\pgf@x}}{\pgf@sys@tonumber{\pgf@y}}%
    {\pgf@sys@tonumber{\pgf@ya}}{\pgf@sys@tonumber{\pgf@x}}{\pgfpointorigin}%
  \fi%
}


% Transform to the coordinate system of a point on an arc
%
% #1 = a time fraction of line where to put the label, where 0 means start,
%      1 means end, and for example 0.5 means the middle.
% #2 = center of an ellipse
% #3 = 0-degree axis of the ellipse
% #4 = 90-degree axis of the ellipse
% #4 = start angle of the arc on this ellipse
% #5 = end angle of the arc on this ellipse
%
% Example:
%
% {
%   \pgftransformlineattime{.5}{\pgfpoint{1cm}{1cm}}{\pgfpoint{1cm}{0cm}}{\pgfpoint{0cm}{1cm}}{30}{40}
%   \pgftext{Hi!}
% }

\def\pgftransformarcaxesattime#1#2#3#4#5#6{%
  \pgfpointarcaxesattime{#1}{#2}{#3}{#4}{#5}{#6}%
  \pgftransformshift{\pgfqpoint{\pgf@x}{\pgf@y}}%
  \ifpgfresetnontranslationattime%
    \pgftransformresetnontranslations%
  \fi%
  \ifpgfslopedattime%
    \pgf@x=\pgf@xa%
    \pgf@y=\pgf@ya%
    \pgf@process{\pgfpointnormalised{}}%
    \ifpgfallowupsidedownattime%
    \else%
      \ifdim\pgf@x<0pt%
        \pgf@x=-\pgf@x%
        \pgf@y=-\pgf@y%
      \fi%
    \fi%
    \pgf@ya=-\pgf@y%
    \pgftransformcm%
    {\pgf@sys@tonumber{\pgf@x}}{\pgf@sys@tonumber{\pgf@y}}%
    {\pgf@sys@tonumber{\pgf@ya}}{\pgf@sys@tonumber{\pgf@x}}{\pgfpointorigin}%
  \fi%
}




% Transform to the coordinate system of a point on a curve
%
% #1 = a time fraction of line where to put the label, where 0 means start,
%      1 means end, and for example 0.5 means the middle.
% #2 = start of line
% #3 = first control
% #4 = second control
% #5 = end of line
%
% Example:
%
% {
%   \pgftransformcurveattime{.5}{\pgfxy(0,0)}{\pgfxy(3,2)}
%   \pgftext{Hi!}
% }
%
% {
%   \pgftransformcurveattime{.75}{\pgfxy(0,0)}{\pgfxy(3,2)}
%   \pgftransformresetnontranslations
%   \pgftext{Hi!}
% }
%
\def\pgftransformcurveattime#1#2#3#4#5{%
  \pgfpointcurveattime{#1}{#2}{#3}{#4}{#5}%
  \pgftransformshift{\pgfqpoint{\pgf@x}{\pgf@y}}%
  \ifpgfresetnontranslationattime%
    \pgftransformresetnontranslations%
  \fi%
  \ifpgfslopedattime%
    \pgf@x=\pgf@xa%
    \pgf@y=\pgf@ya%
    \advance\pgf@x by-\pgf@xb%
    \advance\pgf@y by-\pgf@yb%
    \ifpgfallowupsidedownattime%
    \else%
      \ifdim\pgf@x<0pt%
        \pgf@x=-\pgf@x%
        \pgf@y=-\pgf@y%
      \fi%
    \fi%
    \pgfpointnormalised{}% x/y = normalised vector
    \pgf@ya=-\pgf@y%
    \pgftransformcm%
    {\pgf@sys@tonumber{\pgf@x}}{\pgf@sys@tonumber{\pgf@y}}%
    {\pgf@sys@tonumber{\pgf@ya}}{\pgf@sys@tonumber{\pgf@x}}{\pgfpointorigin}%
  \fi%
}





% Transform to the coordinate system of an arrow at the end of the
% line going from point #1 to point #2 with the correct rotation.
%
% #1 = a start point an (invisible) line
% #2 = an end point an (invisible) line
%
% Example:
%
% \pgftransformarrow{\pgfpointorigin}{\pgfpoint{1cm}{0cm}}

\def\pgftransformarrow#1#2{%
  \pgf@process{#2}%
  \pgftransformshift{}%
  \pgf@xa=\pgf@x%
  \pgf@ya=\pgf@y%
  \pgf@process{#1}%
  \advance\pgf@xa by-\pgf@x%
  \advance\pgf@ya by-\pgf@y%
  \pgf@x=\pgf@xa%
  \pgf@y=\pgf@ya%
  \pgfpointnormalised{}%
  \pgf@ya=-\pgf@y%
  \pgftransformcm%
  {\pgf@sys@tonumber{\pgf@x}}{\pgf@sys@tonumber{\pgf@y}}%
  {\pgf@sys@tonumber{\pgf@ya}}{\pgf@sys@tonumber{\pgf@x}}{\pgfqpoint{0pt}{0pt}}%
}




% Computes transformations adjustments for the current transformation
% matrix
%
% Description:
%
% This command is used when you install a transformation matrix that
% scales everything by a certain factor, but you still wish to draw
% something with "an absolute size". Suppose for instance that you
% install a transformation matrix that scales everything by a factor
% of 4 and you now wish to draw a horizontal line of length 1cm. Then,
% if you do not reset the transformation matrix, you can draw a line
% of logical length 2.5mm, which will then get scaled to a line of
% 1cm.
%
% Things get more difficult in case you scale things only, say,
% vertically. In this case, the adjustment necessary for horizontal
% lines is different from the one needed for vertical lines.
%
% This function computes the necessary scaling factors and puts them
% in the macros \pgfhorizontaltransformationadjustment and
% \pgfverticaltransformationadjustment.
%
% Note that the "right" way to draw a line of absolute length 1cm in a
% transformed coordinate system is to first compute the start point
% and to then reset the transformation matrix. The transformation
% adjustments computed here are important only in situations where you
% cannot do this, for instance when an "outer xsep" must be set.

\def\pgftransformationadjustments{%
  \ifpgf@pt@identity%
    \let\pgfhorizontaltransformationadjustment\pgf@one@text%
    \let\pgfverticaltransformationadjustment\pgf@one@text%
  \else%
    {%
      \pgf@pt@x0pt%
      \pgf@pt@y0pt%
      \pgfpointtransformed{\pgfqpoint{1pt}{0pt}}%
      \pgfmathveclen@{\pgf@sys@tonumber\pgf@x}{\pgf@sys@tonumber\pgf@y}%
      \pgfmathreciprocal@{\pgfmathresult}%
      \global\let\pgf@temp@hori\pgfmathresult
      \pgfpointtransformed{\pgfqpoint{0pt}{1pt}}%
      \pgfmathveclen@{\pgf@sys@tonumber\pgf@x}{\pgf@sys@tonumber\pgf@y}%
      \pgfmathreciprocal@{\pgfmathresult}%
      \global\let\pgf@temp@vert\pgfmathresult
      \xdef\pgf@marshal@b{\def\noexpand\pgfverticaltransformationadjustment{\pgfmathresult}}
    }%
    \let\pgfhorizontaltransformationadjustment\pgf@temp@hori%
    \let\pgfverticaltransformationadjustment\pgf@temp@vert%
  \fi%
}

\let\pgfhorizontaltransformationadjustment\pgf@one@text
\let\pgfverticaltransformationadjustment\pgf@one@text


% Low-level transformations

% Causes the current high-level transformation command to be applied
% to the low-level.
%
% Description:
%
% All subsequent drawing will be transformed additionally by the
% current high-level transformation. Thus, the high-level
% transformation becomes the low-level transformation. The high-level
% transformation is reset at this point.

\def\pgflowlevelsynccm{%
  \pgfsys@transformcm%
  {\pgf@pt@aa}{\pgf@pt@ab}%
  {\pgf@pt@ba}{\pgf@pt@bb}%
  {\pgf@pt@x}{\pgf@pt@y}%
  \pgftransformreset%
  \pgf@relevantforpicturesizefalse%
}



% Causes a transformation command to be applied to the ``lowlevel''
% transformation matrix.
%
% #1 - a high-level transformation command
%
% Description:
%
% All subsequent drawing will be transformed additionally by the given
% transformation matrix. Note that PGF will no longer be able to
% ``keep track'' of the coordinates. Also, transformations are applied
% to *everything*, including line thickness and line endings. Most
% often, this is not desirable.
%
% Example:
%
% \pgflowlevel{\pgftransformcm{1}{0}{0}{1}{\pgfpoint{100pt}{0pt}}} % 100bp to the right.
%
% \begin{pgflowlevelscope}{\pgftransformcm{2}{0}{0}{2}{\pgfpointorigin}} % double in size
%  \pgfmoveto{\pgfpoint{0cm}{0cm}}
%  \pgflineto{\pgflineto{1cm}{1cm}} % actually 2cm/2cm
% \end{pgflowlevelscope}

\def\pgflowlevel#1{%
  {%
    \pgftransformreset%
    #1%
    \pgflowlevelsynccm%
  }%
  \pgf@relevantforpicturesizefalse%
}

\def\pgflowlevelscope#1{\pgfscope\pgflowlevel{#1}}
\def\endpgflowlevelscope{\endpgfscope}

\long\def\pgflowlevelobj#1#2{\pgfscope{\pgflowlevel{#1}#2}\endpgfscope}



% View boxes

% Establishes a view box scope
%
% #1 and #2 = two opposite corners of the source rectangle
% #3 and #4 = two opposite corners of the target rectangle
% #5        = "meet" or "slice"
%
% Description:
%
% Inside the viewbox scope, the source rectangle will be translated
% and scaled so that it becomes centered on the target rectangle and
% will, for "meet", be as large as possible so that it fits inside the
% target and, for "slice", be as small as possible so that it
% encompasses the target.

\def\pgfviewboxscope#1#2#3#4#5{%
  {% Compute rectangle corners
    \pgf@process{#1}%
    \pgf@pos@transform@glob
    \pgf@xa=\pgf@x%
    \pgf@ya=\pgf@y%
    \pgf@process{#2}%
    \pgf@pos@transform@glob
    \ifdim\pgf@x<\pgf@xa%
      \pgf@xb\pgf@x%
      \pgf@xc\pgf@xa%
    \else%
      \pgf@xb\pgf@xa%
      \pgf@xc\pgf@x%
    \fi%
    \ifdim\pgf@y<\pgf@ya%
      \pgf@yb\pgf@y%
      \pgf@yc\pgf@ya%
    \else%
      \pgf@yb\pgf@ya%
      \pgf@yc\pgf@y%
    \fi%
    \pgf@process{#3}%
    \pgf@pos@transform@glob
    \pgf@xa=\pgf@x%
    \pgf@ya=\pgf@y%
    \pgf@process{#4}%
    \pgf@pos@transform@glob
    \ifdim\pgf@x<\pgf@xa%
    \else%
      \pgfutil@tempdima\pgf@x
      \pgf@x\pgf@xa%
      \pgf@xa\pgfutil@tempdima%
    \fi%
    \ifdim\pgf@y<\pgf@ya%
    \else%
      \pgfutil@tempdima\pgf@y
      \pgf@y\pgf@ya%
      \pgf@ya\pgfutil@tempdima%
    \fi%
    \pgfusetype{.view}%
    \edef\pgf@marshal{\expandafter\noexpand\csname pgfsys@viewbox#5\endcsname{\the\pgf@x}{\the\pgf@y}{\the\pgf@xa}{\the\pgf@ya}{\the\pgf@xb}{\the\pgf@yb}{\the\pgf@xc}{\the\pgf@yc}}%
    \pgf@marshal%
  }%
  \pgf@relevantforpicturesizefalse%
}

\def\endpgfviewboxscope{\pgfsys@endviewbox}

\let\startpgfviewboxscope=\pgfviewboxscope
\let\stoppgfviewboxscope=\endpgfviewboxscope

% Forward declarations for nonlinear stuff (have no effect till module
% nonlineartransformations is loaded)

\let\pgfapproximatenonlineartranslation\relax
\let\pgfapproximatenonlineartransformation\relax
\def\pgftransformnonlinear#1{\pgferror{You need to say
    \string\usepgfmodule{nonlineartransformations} to use nonlinear transformations}}


\endinput
