% Copyright 2018 by Till Tantau and Mark Wibrow
%
% This file may be distributed and/or modified
%
% 1. under the LaTeX Project Public License and/or
% 2. under the GNU Public License.
%
% See the file doc/generic/pgf/licenses/LICENSE for more details.

\ProvidesFileRCS{pgflibraryshapes.geometric.code.tex}

\pgfdeclareshape{ellipse}
%
% Draws a circle around the text
%
{%
  \savedanchor\centerpoint{%
    \pgf@x=.5\wd\pgfnodeparttextbox%
    \pgf@y=.5\ht\pgfnodeparttextbox%
    \advance\pgf@y by-.5\dp\pgfnodeparttextbox%
  }%
  \savedanchor\radius{%
    %
    % Calculate ``height radius''
    %
    \pgf@y=.5\ht\pgfnodeparttextbox%
    \advance\pgf@y by.5\dp\pgfnodeparttextbox%
    \pgfmathsetlength\pgf@yb{\pgfkeysvalueof{/pgf/inner ysep}}%
    \advance\pgf@y by\pgf@yb%
    %
    % Calculate ``width radius''
    %
    \pgf@x=.5\wd\pgfnodeparttextbox%
    \pgfmathsetlength\pgf@xb{\pgfkeysvalueof{/pgf/inner xsep}}%
    \advance\pgf@x by\pgf@xb%
    %
    % Adjust
    %
    \pgf@x=1.4142136\pgf@x%
    \pgf@y=1.4142136\pgf@y%
    %
    % Adjust height, if necessary
    %
    \pgfmathsetlength\pgf@yc{\pgfkeysvalueof{/pgf/minimum height}}%
    \ifdim\pgf@y<.5\pgf@yc%
      \pgf@y=.5\pgf@yc%
    \fi%
    %
    % Adjust width, if necessary
    %
    \pgfmathsetlength\pgf@xc{\pgfkeysvalueof{/pgf/minimum width}}%
    \ifdim\pgf@x<.5\pgf@xc%
      \pgf@x=.5\pgf@xc%
    \fi%
    %
    % Add outer sep
    %
    \pgfmathsetlength{\pgf@xb}{\pgfkeysvalueof{/pgf/outer xsep}}%
    \pgfmathsetlength{\pgf@yb}{\pgfkeysvalueof{/pgf/outer ysep}}%
    \advance\pgf@x by\pgf@xb%
    \advance\pgf@y by\pgf@yb%
  }%

  %
  % Anchors
  %
  \anchor{center}{\centerpoint}%
  \anchor{mid}{\centerpoint\pgfmathsetlength\pgf@y{.5ex}}%
  \anchor{base}{\centerpoint\pgf@y=0pt}%
  \anchor{north}
  {
    \pgf@process{\radius}
    \pgf@ya=\pgf@y%
    \pgf@process{\centerpoint}
    \advance\pgf@y by\pgf@ya
  }%
  \anchor{south}
  {
    \pgf@process{\radius}
    \pgf@ya=\pgf@y%
    \pgf@process{\centerpoint}
    \advance\pgf@y by-\pgf@ya
  }%
  \anchor{west}
  {
    \pgf@process{\radius}
    \pgf@xa=\pgf@x%
    \pgf@process{\centerpoint}
    \advance\pgf@x by-\pgf@xa
  }%
  \anchor{mid west}
  {%
    \pgf@process{\radius}
    \pgf@xa=\pgf@x%
    \pgf@process{\centerpoint}
    \advance\pgf@x by-\pgf@xa%
    \pgfmathsetlength\pgf@y{.5ex}
  }%
  \anchor{base west}
  {%
    \pgf@process{\radius}
    \pgf@xa=\pgf@x%
    \pgf@process{\centerpoint}
    \advance\pgf@x by-\pgf@xa%
    \pgf@y=0pt
  }%
  \anchor{north west}
  {
    \pgf@process{\radius}
    \pgf@xa=\pgf@x%
    \pgf@ya=\pgf@y%
    \pgf@process{\centerpoint}
    \advance\pgf@x by-0.707107\pgf@xa
    \advance\pgf@y by0.707107\pgf@ya
  }%
  \anchor{south west}
  {
    \pgf@process{\radius}
    \pgf@xa=\pgf@x%
    \pgf@ya=\pgf@y%
    \pgf@process{\centerpoint}
    \advance\pgf@x by-0.707107\pgf@xa
    \advance\pgf@y by-0.707107\pgf@ya
  }%
  \anchor{east}
  {%
    \pgf@process{\radius}
    \pgf@xa=\pgf@x%
    \pgf@process{\centerpoint}
    \advance\pgf@x by\pgf@xa
  }%
  \anchor{mid east}
  {%
    \pgf@process{\radius}
    \pgf@xa=\pgf@x%
    \pgf@process{\centerpoint}
    \advance\pgf@x by\pgf@xa%
    \pgfmathsetlength\pgf@y{.5ex}
  }%
  \anchor{base east}
  {%
    \pgf@process{\radius}
    \pgf@xa=\pgf@x%
    \pgf@process{\centerpoint}
    \advance\pgf@x by\pgf@xa%
    \pgf@y=0pt
  }%
  \anchor{north east}
  {
    \pgf@process{\radius}
    \pgf@xa=\pgf@x%
    \pgf@ya=\pgf@y%
    \pgf@process{\centerpoint}
    \advance\pgf@x by0.707107\pgf@xa
    \advance\pgf@y by0.707107\pgf@ya
  }%
  \anchor{south east}
  {
    \pgf@process{\radius}
    \pgf@xa=\pgf@x%
    \pgf@ya=\pgf@y%
    \pgf@process{\centerpoint}
    \advance\pgf@x by0.707107\pgf@xa
    \advance\pgf@y by-0.707107\pgf@ya
  }%
  \anchorborder{%
    \edef\pgf@marshal{%
      \noexpand\pgfpointborderellipse
      {\noexpand\pgfqpoint{\the\pgf@x}{\the\pgf@y}}
      {\noexpand\radius}%
    }%
    \pgf@marshal%
    \pgf@xa=\pgf@x%
    \pgf@ya=\pgf@y%
    \centerpoint%
    \advance\pgf@x by\pgf@xa%
    \advance\pgf@y by\pgf@ya%
  }%

  %
  % Background path
  %
  \backgroundpath
  {
    \pgf@process{\radius}%
    \pgfutil@tempdima=\pgf@x%
    \pgfutil@tempdimb=\pgf@y%
    \pgfmathsetlength{\pgf@xb}{\pgfkeysvalueof{/pgf/outer xsep}}%
    \pgfmathsetlength{\pgf@yb}{\pgfkeysvalueof{/pgf/outer ysep}}%
    \advance\pgfutil@tempdima by-\pgf@xb%
    \advance\pgfutil@tempdimb by-\pgf@yb%
    \pgfpathellipse{\centerpoint}{\pgfqpoint{\pgfutil@tempdima}{0pt}}{\pgfqpoint{0pt}{\pgfutil@tempdimb}}%
  }%
}%




% Set the recommended shape aspect ratio
%
% #1 = aspect ratio
%
% Example:
%
% \pgfsetshapeaspect{1.5}

% Keys for shape aspect ratio.
%
% /pgf/aspect         : recommended width/height quotient.
%
\pgfkeys{/pgf/.cd,
    aspect/.code={\pgfsetshapeaspect{#1}},% this for tikz...
    shape aspect/.initial=1,% but this is consistent with other pgfset stuff.
    shape aspect/.code={%
        \pgfkeys{/pgf/aspect=#1}%
        \pgfkeyssetvalue{/pgf/shape aspect}{#1}
    }%
}%


\def\pgfsetshapeaspect#1{%
  \def\pgfshapeaspect{#1}%
  % Invert
  \pgfutil@tempdima=#1pt%
  \pgfutil@tempdima=.125\pgfutil@tempdima%
  \c@pgf@counta=\pgfutil@tempdima\relax% 8192*determinant
  \pgfutil@tempdima=8192pt%
  \divide\pgfutil@tempdima by\c@pgf@counta%
  \edef\pgfshapeaspectinverse{\pgf@sys@tonumber{\pgfutil@tempdima}}%
}%

\pgfsetshapeaspect{1}%

\pgfdeclareshape{diamond}
{%
  \savedanchor\outernortheast{%
    %
    % Calculate width and height of the inner rectangle
    %
    \pgf@xa=.5\wd\pgfnodeparttextbox%
    \pgfmathsetlength\pgf@xc{\pgfkeysvalueof{/pgf/inner xsep}}%
    \advance\pgf@xa by\pgf@xc%
    \pgf@ya=.5\ht\pgfnodeparttextbox%
    \advance\pgf@ya by.5\dp\pgfnodeparttextbox%
    \pgfmathsetlength\pgf@yc{\pgfkeysvalueof{/pgf/inner ysep}}%
    \advance\pgf@ya by\pgf@yc%
    %
    % Calculate width and height of diamond
    %
    \pgf@x=\pgf@xa%
    \advance\pgf@x by\pgfshapeaspect\pgf@ya%
    \pgf@y=\pgfshapeaspectinverse\pgf@xa%
    \advance\pgf@y by\pgf@ya%
    %
    % Check against minimum height/width
    %
    \pgfmathsetlength\pgf@xb{\pgfkeysvalueof{/pgf/minimum width}}%
    \pgf@xb=.5\pgf@xb%
    \ifdim\pgf@x<\pgf@xb%
      % yes, too small. Enlarge...
      \pgf@x=\pgf@xb%
    \fi%
    \pgfmathsetlength\pgf@yb{\pgfkeysvalueof{/pgf/minimum height}}%
    \pgf@yb=.5\pgf@yb%
    \ifdim\pgf@y<\pgf@yb%
      % yes, too small. Enlarge...
      \pgf@y=\pgf@yb%
    \fi%
    %
    % Add outer border
    %
    \pgfmathsetlength\pgf@xa{\pgfkeysvalueof{/pgf/outer xsep}}%
    \advance\pgf@x by\pgf@xa%
    \pgfmathsetlength\pgf@ya{\pgfkeysvalueof{/pgf/outer ysep}}%
    \advance\pgf@y by\pgf@ya%
  }%
  \savedanchor\text{%
    \pgf@x=-.5\wd\pgfnodeparttextbox%
    \pgf@y=-.5\ht\pgfnodeparttextbox%
    \advance\pgf@y by.5\dp\pgfnodeparttextbox%
  }%

  %
  % Anchors
  %
  \anchor{text}{\text}%
  \anchor{center}{\pgfpointorigin}%
  \anchor{mid}{%
    \pgf@process{\text}%
    \pgf@x=0pt%
    \pgfmathsetlength\pgf@ya{.5ex}%
    \advance\pgf@y by\pgf@ya%
  }%
  \anchor{base}{\pgf@process{\text}\pgf@x=0pt}%
  \anchor{north}{\pgf@process{\outernortheast}\pgf@x=0pt}%
  \anchor{south}{\pgf@process{\outernortheast}\pgf@x=0pt\pgf@y=-\pgf@y}%
  \anchor{west}{\pgf@process{\outernortheast}\pgf@x=-\pgf@x\pgf@y=0pt}%
  \anchor{north west}{\pgf@process{\outernortheast}\pgf@x=-.5\pgf@x\pgf@y=.5\pgf@y}%
  \anchor{south west}{\pgf@process{\outernortheast}\pgf@x=-.5\pgf@x\pgf@y=-.5\pgf@y}%
  \anchor{east}{\pgf@process{\outernortheast}\pgf@y=0pt}%
  \anchor{north east}{\pgf@process{\outernortheast}\pgf@x=.5\pgf@x\pgf@y=.5\pgf@y}%
  \anchor{south east}{\pgf@process{\outernortheast}\pgf@x=.5\pgf@x\pgf@y=-.5\pgf@y}%
  \anchorborder{%
    \pgf@xa=\pgf@x%
    \pgf@ya=\pgf@y%
    \pgf@process{\outernortheast}%
    \ifdim\pgf@xa>0pt%
    \else%
      \pgf@x=-\pgf@x%
    \fi%
    \ifdim\pgf@ya>0pt%
    \else%
      \pgf@y=-\pgf@y%
    \fi%
    \edef\pgf@marshal{%
      \noexpand\pgfpointintersectionoflines
      {\noexpand\pgfpointorigin}
      {\noexpand\pgfqpoint{\the\pgf@xa}{\the\pgf@ya}}
      {\noexpand\pgfqpoint{\the\pgf@x}{0pt}}
      {\noexpand\pgfqpoint{0pt}{\the\pgf@y}}%
    }%
    \pgf@process{\pgf@marshal}%
  }%

  %
  % Background path
  %
  \backgroundpath{
    \pgf@process{\outernortheast}%
    \pgf@xc=\pgf@x%
    \pgf@yc=\pgf@y%
    \pgfmathsetlength{\pgf@xa}{\pgfkeysvalueof{/pgf/outer xsep}}%
    \pgfmathsetlength{\pgf@ya}{\pgfkeysvalueof{/pgf/outer ysep}}%
    \advance\pgf@xc by-1.414213\pgf@xa%
    \advance\pgf@yc by-1.414213\pgf@ya%
    \pgfpathmoveto{\pgfqpoint{\pgf@xc}{0pt}}%
    \pgfpathlineto{\pgfqpoint{0pt}{\pgf@yc}}%
    \pgfpathlineto{\pgfqpoint{-\pgf@xc}{0pt}}%
    \pgfpathlineto{\pgfqpoint{0pt}{-\pgf@yc}}%
    \pgfpathclose%
  }%
}%






\newif\ifpgf@lib@shapes@starouterradiususesratio
\pgf@lib@shapes@starouterradiususesratiotrue

% Keys for star shape
%
% /pgf/star points       : Number of points on the star.
% /pgf/star point height : The height of the outer star points.
% /pgf/star point ratio  : The ratio of the outer point and inner point radii.

\pgfkeys{/pgf/.cd,
    star points/.initial=5,
    star point height/.initial=.5cm,
    star point height/.code={%
        \def\pgf@lib@temp{#1}%
        \pgfkeyslet{/pgf/star point height}{\pgf@lib@temp}%
        \pgf@lib@shapes@starouterradiususesratiofalse%
    },
    star point ratio/.initial=1.5,
    star point ratio/.code={%
        \def\pgf@lib@temp{#1}%
        \pgfkeyslet{/pgf/star point ratio}{\pgf@lib@temp}%
        \pgf@lib@shapes@starouterradiususesratiotrue%
    },%
    star rotate/.style={/pgf/shape border rotate=#1},% For compatibility with 1.18
}%


% Shape star.
%
\pgfdeclareshape{star}{%
    \savedmacro\totalstarpoints{%
        \pgfmathsetcounter{pgf@counta}{\pgfkeysvalueof{/pgf/star points}}%
        \multiply\c@pgf@counta2\relax%
        \edef\totalstarpoints{\the\c@pgf@counta}%
    }%
    \savedmacro\anglestep{%
        \pgfmathdivide{180}{\pgfkeysvalueof{/pgf/star points}}%
        \let\anglestep\pgfmathresult%
    }%
    \savedmacro\calculateradii{%
        %
        % Get the node dimensions.
        %
        \pgfmathsetlength\pgf@x{\pgfkeysvalueof{/pgf/inner xsep}}%
        \advance\pgf@x.5\wd\pgfnodeparttextbox%
        \pgfmathsetlength\pgf@y{\pgfkeysvalueof{/pgf/inner ysep}}%
        \advance\pgf@y.5\ht\pgfnodeparttextbox%
        \advance\pgf@y.5\dp\pgfnodeparttextbox%
        \ifdim\pgf@y>\pgf@x%
            \pgf@x\pgf@y%
        \fi%
        %
        % Calculate the incircle radius.
        %
        \pgf@x1.41421\pgf@x%
        \edef\innerradius{\the\pgf@x}%
        %
        % Calculate the circumcircle radius.
        %
        % Use the star point height or the star point ratio?
        %
        \ifpgf@lib@shapes@starouterradiususesratio%
            \pgfmathparse{\pgfkeysvalueof{/pgf/star point ratio}}%
            \expandafter\pgf@x\pgfmathresult\pgf@x%
        \else%
            \pgfmathaddtolength\pgf@x{\pgfkeysvalueof{/pgf/star point height}}%
        \fi%
        \pgf@xc\pgf@x%
        %
        % Accommodate the larger of the minimum height/width.
        %
        \pgfmathsetlength\pgf@xa{\pgfkeysvalueof{/pgf/minimum width}}%
        \pgfmathsetlength\pgf@ya{\pgfkeysvalueof{/pgf/minimum height}}%
        \ifdim\pgf@ya>\pgf@xa%
            \pgf@xa\pgf@ya%
        \fi%
        \ifdim\pgf@x<.5\pgf@xa%
        \pgf@x.5\pgf@xa%
       \fi%
       \edef\outerradius{\the\pgf@x}%
       %
       % If the outer radius has been enlarged, adjust the inner radius...
       %
       \ifdim\pgf@x>\pgf@xc%
        %
        % ...using the star point ratio, or...
        %
        \ifpgf@lib@shapes@starouterradiususesratio%
            \pgfmathreciprocal{\pgfkeysvalueof{/pgf/star point ratio}}%
            \pgf@xa\pgfmathresult\pgf@x\relax%
            \edef\innerradius{\the\pgf@xa}%
            \else%
                %
                % The star point height.
                %
                \pgf@xa\pgf@x\relax%
                \pgfmathaddtolength\pgf@xa{-\pgfkeysvalueof{/pgf/star point height}}%
                \edef\innerradius{\the\pgf@xa}%
            \fi%
        \fi%
        \edef\outerradius{\the\pgf@x}%
        %
        % Now calculate the anchor radii from the outer sep.
        %
        \pgfmathsetlength\pgf@xa{\pgfkeysvalueof{/pgf/outer xsep}}%
        \pgfmathsetlength\pgf@ya{\pgfkeysvalueof{/pgf/outer ysep}}%
        \ifdim\pgf@ya>\pgf@xa%
            \pgf@xa\pgf@ya%
        \fi
        %
        % Take into account the miter length...
        %
        \pgfmathdivide{180}{\pgfkeysvalueof{/pgf/star points}}%
        \let\angletofirstpoint\pgfmathresult%
        \pgfmathmultiply@{\angletofirstpoint}{2}%
        \let\angletosecondpoint\pgfmathresult%
        %
        % ...for the outer radius...
        %
        \pgfmathanglebetweenlines%
            {\pgfqpointpolar{\angletofirstpoint}{\outerradius}}%
            {\pgfqpointpolar{\angletosecondpoint}{\innerradius}}%
            {\pgfqpointpolar{\angletofirstpoint}{\outerradius}}%
            {\pgfqpointpolar{0}{\innerradius}}%
        \pgfmathdivide@{\pgfmathresult}{2}%
        \pgfmathcosec@{\pgfmathresult}%
        \pgf@x\outerradius\relax%
        \advance\pgf@x\pgfmathresult\pgf@xa%
        \edef\anchorouterradius{\the\pgf@x}%
        %
        % ...and for the inner radius.
        %
        \pgfmathanglebetweenlines%
            {\pgfqpointpolar{\angletofirstpoint}{\innerradius}}%
            {\pgfqpointpolar{0}{\outerradius}}%
            {\pgfqpointpolar{\angletofirstpoint}{\innerradius}}%
            {\pgfqpointpolar{\angletosecondpoint}{\outerradius}}%
        \pgfmathdivide@{\pgfmathresult}{2}%
        \pgfmathcosec@{\pgfmathresult}%
        \pgf@x\innerradius\relax%
        \advance\pgf@x\pgfmathresult\pgf@xa%
        \edef\anchorinnerradius{\the\pgf@x}%
        %
        % Save all radii.
        %
        \addtosavedmacro{\innerradius}%
        \addtosavedmacro{\outerradius}%
        \addtosavedmacro{\anchorinnerradius}%
        \addtosavedmacro{\anchorouterradius}%
    }%
    \savedmacro\startangle{%
        \pgfmathadd{90}{\pgfkeysvalueof{/pgf/shape border rotate}}%
        \let\startangle\pgfmathresult%
    }%
    %
    % Saved anchors.
    %
    \savedanchor{\centerpoint}{%
        \pgf@x.5\wd\pgfnodeparttextbox%
        \pgf@y.5\ht\pgfnodeparttextbox%
        \advance\pgf@y-.5\dp\pgfnodeparttextbox%
    }%
    \savedanchor{\midpoint}{%
        \pgf@x.5\wd\pgfnodeparttextbox%
        \pgfmathsetlength\pgf@y{+.5ex}%
    }%
    %
    %
    % Other anchors.
    %
    \anchor{center}{\centerpoint}%
    \anchor{mid}{\centerpoint\pgfmathsetlength\pgf@y{.5ex}}%
    \anchor{base}{\centerpoint\pgf@y=0pt}%
    \anchor{north}{%
        \calculateradii%
        \csname pgf@anchor@star@border\endcsname{\pgfqpoint{0pt}{\anchorouterradius}}}%
    \anchor{south}{%
        \calculateradii%
        \csname pgf@anchor@star@border\endcsname{\pgfqpoint{0pt}{-\anchorouterradius}}}%
    \anchor{east}{%
        \calculateradii%
        \csname pgf@anchor@star@border\endcsname{\pgfqpoint{\anchorouterradius}{0pt}}}%
    \anchor{west}{%
        \calculateradii%
        \csname pgf@anchor@star@border\endcsname{\pgfqpoint{-\anchorouterradius}{0pt}}}%
    \anchor{north east}{%
        \calculateradii%
        \csname pgf@anchor@star@border\endcsname{\pgfqpoint{\anchorouterradius}{\anchorouterradius}}}%
    \anchor{north west}{%
        \calculateradii%
        \csname pgf@anchor@star@border\endcsname{\pgfqpoint{-\anchorouterradius}{\anchorouterradius}}}%
    \anchor{south east}{%
        \calculateradii%
        \csname pgf@anchor@star@border\endcsname{\pgfqpoint{\anchorouterradius}{-\anchorouterradius}}}%
    \anchor{south west}{%
        \calculateradii%
        \csname pgf@anchor@star@border\endcsname{\pgfqpoint{-\anchorouterradius}{-\anchorouterradius}}}%
    %
    % Background path.
    %
    \backgroundpath{%
        \calculateradii%
        \let\angle\startangle%
        \pgfpathmoveto{\pgfpointadd{\centerpoint}{\pgfqpointpolar{\startangle}{\outerradius}}}%
        \pgfmathloop%
            \pgfmathadd@{\angle}{\anglestep}%
            \let\angle\pgfmathresult%
        \ifnum\pgfmathcounter=\totalstarpoints%
            \pgfpathclose%
        \else%
            \ifodd\pgfmathcounter
                \pgfpathlineto{\pgfpointadd{\centerpoint}{\pgfqpointpolar{\angle}{\innerradius}}}%
            \else%
                \pgfpathlineto{\pgfpointadd{\centerpoint}{\pgfqpointpolar{\angle}{\outerradius}}}%
            \fi%
        \repeatpgfmathloop%
    }%
    %
    % Define points on the anchor border.
    %
    \anchorborder{%
        %
        % Save x and y.
        %
        \edef\externalx{\the\pgf@x}%
        \edef\externaly{\the\pgf@y}%
        %
        % Adjust the location of the external
        % point relative to \centerpoint.
        %
        \centerpoint%
        \pgf@xa\externalx\relax%
        \pgf@ya\externaly\relax%
        \advance\pgf@xa\pgf@x%
        \advance\pgf@ya\pgf@y%
        \edef\externalx{\the\pgf@xa}%
        \edef\externaly{\the\pgf@ya}%
        %
        % Get the angle of the external point to the \centerpoint.
        %
        \pgfmathanglebetweenpoints{\centerpoint}{\pgfqpoint{\externalx}{\externaly}}%
        %
        % Locate the appropriate sides on the star border...
        %
        \pgfmathsubtract@{\pgfmathresult}{\startangle}%
        \ifdim\pgfmathresult pt<0pt\relax%
            \pgfmathadd@{\pgfmathresult}{360}%
        \fi%
        \pgfmathdivide@{\pgfmathresult}{\anglestep}%
        \pgfmathfloor@{\pgfmathresult}%
        \afterassignment\pgfmath@gobbletilpgfmath@
        \c@pgf@counta\pgfmathresult\relax\pgfmath@
        \pgfmathmultiply@{\pgfmathresult}{\anglestep}%
        \pgfmathadd@{\pgfmathresult}{\startangle}%
        \let\firstangle\pgfmathresult%
        \pgfmathadd@{\pgfmathresult}{\anglestep}%
        \let\secondangle\pgfmathresult%
        \calculateradii%
        %
        % ...and thus, the point on the star border.
        %
        \ifodd\c@pgf@counta
            \pgfpointintersectionoflines{\centerpoint}{\pgfpoint{+\externalx}{+\externaly}}%
                {%
                    \pgfpointadd{\centerpoint}%
                        {\pgfqpointpolar{+\firstangle}{+\anchorinnerradius}}%
                    }%
                    {%
                        \pgfpointadd{\centerpoint}%
                            {\pgfqpointpolar{+\secondangle}{+\anchorouterradius}}%
                    }%
        \else%
            \pgfpointintersectionoflines{\centerpoint}{\pgfpoint{+\externalx}{+\externaly}}%
                {%
                    \pgfpointadd{\centerpoint}%
                        {\pgfqpointpolar{+\firstangle}{+\anchorouterradius}}%
                    }%
                    {%
                        \pgfpointadd{\centerpoint}%
                            {\pgfqpointpolar{+\secondangle}{+\anchorinnerradius}}%
                    }%
        \fi%
    }%
    %
    % Some hackery. This is needed if a star is positioned using
    % a `...point <n+1>' anchor, where n is the maximum number of
    % points of any previously drawn star.
    %
    \pgfutil@g@addto@macro\pgf@sh@s@star{%
        \c@pgf@counta\totalstarpoints\relax%
        \divide\c@pgf@counta2\relax%
        \pgfmathloop%
            \ifnum\c@pgf@counta>0\relax%
                \c@pgf@countb\c@pgf@counta\relax%
                \advance\c@pgf@countb-1\relax%
                \pgfutil@ifundefined{pgf@anchor@star@inner point\space\the\c@pgf@counta}{%
                    \expandafter\xdef\csname pgf@anchor@star@inner point\space\the\c@pgf@counta\endcsname{%
                        \noexpand\calculateradii%
                        \noexpand\pgfmathmultiply@{\the\c@pgf@countb}{\noexpand\anglestep}%
                        \noexpand\pgfmathmultiply@{\noexpand\pgfmathresult}{2}%
                        \noexpand\pgfmathadd@{\noexpand\pgfmathresult}{\noexpand\anglestep}%
                        \noexpand\pgfmathadd@{\noexpand\pgfmathresult}{\noexpand\startangle}%
                        \noexpand\let\noexpand\angle\noexpand\pgfmathresult%
                        \noexpand\pgfpointadd{\noexpand\centerpoint}%
                            {\noexpand\pgfqpointpolar{\noexpand\angle}{\noexpand\anchorinnerradius}}%
                    }%
                    \expandafter\xdef\csname pgf@anchor@star@outer point\space\the\c@pgf@counta\endcsname{%
                        \noexpand\calculateradii%
                        \noexpand\pgfmathmultiply@{\the\c@pgf@countb}{\noexpand\anglestep}%
                        \noexpand\pgfmathmultiply@{\noexpand\pgfmathresult}{2}%
                        \noexpand\pgfmathadd@{\noexpand\pgfmathresult}{\noexpand\startangle}%
                        \noexpand\let\noexpand\angle\noexpand\pgfmathresult%
                        \noexpand\pgfpointadd{\noexpand\centerpoint}%
                            {\noexpand\pgfqpointpolar{\noexpand\angle}{\noexpand\anchorouterradius}}%
                    }%
                }{\c@pgf@counta0\relax}%
            \advance\c@pgf@counta-1\relax%
        \repeatpgfmathloop%
    }%
}%




% Keys for shape regular polygon
%
% /pgf/regular polygon sides
%
\pgfkeys{/pgf/.cd,
    regular polygon sides/.initial=5,
    regular polygon rotate/.style={/pgf/shape border rotate=#1}% For compatibility with 1.18
}%


% Shape Regular Polygon.
%
\pgfdeclareshape{regular polygon}{%
    \savedmacro\sides{%
        \pgfmathtruncatemacro\sides{\pgfkeysvalueof{/pgf/regular polygon sides}}%
    }%
    \savedmacro\anglestep{%
        \pgfmathdivide{360}{\pgfkeysvalueof{/pgf/regular polygon sides}}%
        \let\anglestep\pgfmathresult%
    }%
    \savedmacro\calculateradii{%
        %
        % Get the node dimensions.
        %
        \pgfmathsetlength\pgf@x{\pgfkeysvalueof{/pgf/inner xsep}}%
        \advance\pgf@x.5\wd\pgfnodeparttextbox%
        \pgfmathsetlength\pgf@y{\pgfkeysvalueof{/pgf/inner ysep}}%
        \advance\pgf@y.5\ht\pgfnodeparttextbox%
        \advance\pgf@y.5\dp\pgfnodeparttextbox%
        \ifdim\pgf@y>\pgf@x%
            \pgf@x\pgf@y%
        \fi%
        %
        % Calculate i, the incircle radius
        %
        \pgf@x1.41421\pgf@x%
        %
        % Calculate r, the polygon radius
        %
        % r = i / cos(360 / s / 2)
        %
        % (s = polygon sides)
        %
        \pgfmathdivide{180}{\pgfkeysvalueof{/pgf/regular polygon sides}}%
        \pgfmathsec@{\pgfmathresult}%
        \pgf@x\pgfmathresult\pgf@x%
        %
        % Accommodate the minimum width/height.
        %
        \pgfmathsetlength\pgf@xa{\pgfkeysvalueof{/pgf/minimum width}}%
        \pgfmathsetlength\pgf@ya{\pgfkeysvalueof{/pgf/minimum height}}%
        \ifdim\pgf@ya>\pgf@xa%
            \pgf@xa\pgf@ya%
        \fi%
        \ifdim\pgf@x<.5\pgf@xa%
            \pgf@x.5\pgf@xa%
        \fi%
        \edef\radius{\the\pgf@x}%
        %
        % Now calculate the anchor radius from the outer sep.
        %
        \pgfmathsetlength\pgf@xa{\pgfkeysvalueof{/pgf/outer xsep}}%
        \pgfmathsetlength\pgf@ya{\pgfkeysvalueof{/pgf/outer ysep}}%
        \ifdim\pgf@ya>\pgf@xa%
            \pgf@xa\pgf@ya%
        \fi
        %
        % Take into account the miter length.
        %
        % m = o / sin (90 - (360 / s / 2))
        %
        % (o = outer sep, s = sides)
        %
        \pgfmathdivide{180}{\pgfkeysvalueof{/pgf/regular polygon sides}}%
        \pgfmathsubtract@{90}{\pgfmathresult}%
        \pgfmathcosec@{\pgfmathresult}%
        \advance\pgf@x\pgfmathresult\pgf@xa%
        \edef\anchorradius{\the\pgf@x}%
        %
        % Save both radii.
        %
        \addtosavedmacro{\radius}%
        \addtosavedmacro{\anchorradius}%
    }%
    \savedmacro\startangle{%
        \pgfmathdivide{360}{\pgfkeysvalueof{/pgf/regular polygon sides}}%
        \let\anglestep\pgfmathresult%
        \pgfmathtruncatemacro\sides{\pgfkeysvalueof{/pgf/regular polygon sides}}%
        \expandafter\ifodd\sides\relax%
            \edef\pgfmathresult{90}%
        \else%
            \pgfmathdivide@{\anglestep}{2}%
            \pgfmathsubtract@{90}{\pgfmathresult}%
        \fi%
        \pgfmathadd{\pgfmathresult}{\pgfkeysvalueof{/pgf/shape border rotate}}%
        \let\startangle\pgfmathresult%
    }%
    %
    % Saved anchors.
    %
    \savedanchor{\centerpoint}{%
        \pgf@x.5\wd\pgfnodeparttextbox%
        \pgf@y.5\ht\pgfnodeparttextbox%
        \advance\pgf@y-.5\dp\pgfnodeparttextbox%
    }%
    \savedanchor{\midpoint}{%
        \pgf@x.5\wd\pgfnodeparttextbox%
        \pgfmathsetlength\pgf@y{+.5ex}%
    }%
    %
    % Other anchors.
    %
    \anchor{center}{\centerpoint}%
    \anchor{mid}{\midpoint}%
    \anchor{base}{\centerpoint\pgf@y=0pt}%
    \anchor{north}{%
        \calculateradii%
        \csname pgf@anchor@regular polygon@border\endcsname{\pgfqpoint{0pt}{\anchorradius}}}%
    \anchor{south}{%
        \calculateradii%
        \csname pgf@anchor@regular polygon@border\endcsname{\pgfqpoint{0pt}{-\anchorradius}}}%
    \anchor{east}{%
        \calculateradii%
        \csname pgf@anchor@regular polygon@border\endcsname{\pgfqpoint{\anchorradius}{0pt}}}%
    \anchor{west}{%
        \calculateradii%
        \csname pgf@anchor@regular polygon@border\endcsname{\pgfqpoint{-\anchorradius}{0pt}}}%
    \anchor{north east}{%
        \calculateradii%
        \csname pgf@anchor@regular polygon@border\endcsname{\pgfqpoint{\anchorradius}{\anchorradius}}}%
    \anchor{north west}{%
        \calculateradii%
        \csname pgf@anchor@regular polygon@border\endcsname{\pgfqpoint{-\anchorradius}{\anchorradius}}}%
    \anchor{south east}{%
        \calculateradii%
        \csname pgf@anchor@regular polygon@border\endcsname{\pgfqpoint{\anchorradius}{-\anchorradius}}}%
    \anchor{south west}{%
        \calculateradii%
        \csname pgf@anchor@regular polygon@border\endcsname{\pgfqpoint{-\anchorradius}{-\anchorradius}}}%
    %
    % Background path.
    %
    \backgroundpath{%
        \calculateradii%
        \pgfpathmoveto{%
            \pgfpointadd{\centerpoint}{\pgfqpointpolar{\startangle}{\radius}}%
        }%
        \let\angle\startangle%
        \pgfmathloop%
            \ifnum\pgfmathcounter=\sides\relax%
                \pgfpathclose%
            \else%
                \pgfmathadd@{\angle}{\anglestep}%
                \let\angle\pgfmathresult%
                \pgfpathlineto{%
                    \pgfpointadd{\centerpoint}{\pgfqpointpolar{\angle}{\radius}}%
                }%
        \repeatpgfmathloop%
    }%
    \anchorborder{%
        %
        % Save x and y.
        %
        \edef\externalx{\the\pgf@x}%
        \edef\externaly{\the\pgf@y}%
        %
        % Adjust the location of the external
        % point relative to \centerpoint.
        %
        \centerpoint%
        \pgf@xa\externalx\relax%
        \pgf@ya\externaly\relax%
        \advance\pgf@xa\pgf@x%
        \advance\pgf@ya\pgf@y%
        \edef\externalx{\the\pgf@xa}%
        \edef\externaly{\the\pgf@ya}%
        %
        % Get the angle of the external point to the \centerpoint.
        %
        \pgfmathanglebetweenpoints{\centerpoint}{\pgfqpoint{\externalx}{\externaly}}%
        %
        % Locate the appropriate sides on the polygon border...
        %
        \pgfmathsubtract@{\pgfmathresult}{\startangle}%
        \ifdim\pgfmathresult pt<0pt\relax%
            \pgfmathadd@{\pgfmathresult}{360}%
        \fi%
        \pgfmathdivide@{\pgfmathresult}{\anglestep}%
        \pgfmathfloor@{\pgfmathresult}%
        \pgfmathmultiply@{\pgfmathresult}{\anglestep}%
        \pgfmathadd@{\pgfmathresult}{\startangle}%
        \let\firstangle\pgfmathresult%
        \pgfmathadd@{\pgfmathresult}{\anglestep}%
        \let\secondangle\pgfmathresult%
        \calculateradii%
        %
        % ...and thus, the point on the polygon border.
        %
        \pgfpointintersectionoflines{\centerpoint}{\pgfpoint{+\externalx}{+\externaly}}%
            {%
                \pgfpointadd{\centerpoint}%
                    {\pgfqpointpolar{+\firstangle}{+\anchorradius}}%
                }%
                {%
                    \pgfpointadd{\centerpoint}%
                        {\pgfqpointpolar{+\secondangle}{+\anchorradius}}%
                }%
    }%
    %
    % More hackery for when the rectangular polygon is positioned using
    % a `corner <n+1>' or `side <n+1>' anchor, where n is the maximum
    % number of sides of any previously drawn regular polygon.
    %
    \expandafter\pgfutil@g@addto@macro\csname pgf@sh@s@regular polygon\endcsname{%
        \c@pgf@counta\sides\relax%
        \pgfmathloop%
            \ifnum\c@pgf@counta>0\relax%
                \pgfutil@ifundefined{pgf@anchor@regular polygon@corner\space\the\c@pgf@counta}{%
                %
                % ...(manually \xdef as \gdef is normally used by \anchor)...
                %
                \expandafter\xdef\csname pgf@anchor@regular polygon@corner\space\the\c@pgf@counta\endcsname{%
                    \noexpand\calculateradii%
                    \noexpand\pgfmathsubtract@{\the\c@pgf@counta}{1}%
                    \noexpand\pgfmathmultiply@{\noexpand\pgfmathresult}{\noexpand\anglestep}%
                    \noexpand\pgfmathadd@{\noexpand\pgfmathresult}{\noexpand\startangle}%
                    \noexpand\let\noexpand\angle\noexpand\pgfmathresult%
                    \noexpand\pgfpointadd{\noexpand\centerpoint}%
                        {\noexpand\pgfqpointpolar{\noexpand\angle}{\noexpand\anchorradius}}%
                }%
                \expandafter\xdef\csname pgf@anchor@regular polygon@side\space\the\c@pgf@counta\endcsname{%
                    \noexpand\calculateradii%
                    \noexpand\pgfmathsubtract@{\the\c@pgf@counta}{1}%
                    \noexpand\pgfmathmultiply@{\noexpand\pgfmathresult}{\noexpand\anglestep}%
                    \noexpand\pgfmathadd@{\noexpand\pgfmathresult}{\noexpand\startangle}%
                    \noexpand\let\noexpand\firstangle\noexpand\pgfmathresult%
                    \noexpand\pgfmathadd@{\noexpand\pgfmathresult}{\noexpand\anglestep}%
                    \noexpand\let\noexpand\secondangle\noexpand\pgfmathresult%
                    \noexpand\pgfpointlineattime{0.5}%
                        {\noexpand\pgfpointadd{\noexpand\centerpoint}%
                            {\noexpand\pgfqpointpolar{\noexpand\firstangle}{\noexpand\anchorradius}}}%
                        {\noexpand\pgfpointadd{\noexpand\centerpoint}%
                            {\noexpand\pgfqpointpolar{\noexpand\secondangle}{\noexpand\anchorradius}}}%
                }%
            }{\c@pgf@counta0\relax}%
            \advance\c@pgf@counta-1\relax%
        \repeatpgfmathloop%
    }%
}%



% Keys for shape trapezium
%
% /pgf/trapezium left angle  : angle of the left side.
% /pgf/trapezium right angle : angle of the right side.

\pgfkeys{/pgf/.cd,
    trapezium left angle/.initial=60,
    trapezium right angle/.initial=60,
    trapezium angle/.style={
        /pgf/trapezium left angle=#1,
        /pgf/trapezium right angle=#1
    },%
    trapezium stretches/.is if=pgf@lib@sh@trapeziumstretches,%
    trapezium stretches body/.is if=pgf@lib@sh@trapeziumstretchesbody,%
}%

\newif\ifpgf@lib@sh@trapeziumstretchesbody
\newif\ifpgf@lib@sh@trapeziumstretches


% Shape trapezium.
%
\pgfdeclareshape{trapezium}{%
    \savedmacro\installtrapeziumparameters{%
        %
        % Get the node dimensions.
        %
        \pgfmathsetlength\pgf@x{\pgfkeysvalueof{/pgf/inner xsep}}%
        \advance\pgf@x.5\wd\pgfnodeparttextbox%
        \pgfmathsetlength\pgf@y{\pgfkeysvalueof{/pgf/inner ysep}}%
        \advance\pgf@y.5\ht\pgfnodeparttextbox%
        \advance\pgf@y.5\dp\pgfnodeparttextbox%
        \ifpgfshapeborderusesincircle%
            %
            % Get the rotation (no rounding).
            %
            \pgfmathsetmacro\rotate{\pgfkeysvalueof{/pgf/shape border rotate}}%
            %
            % Use the radius of the incircle if required...
            %
            \ifdim\pgf@x<\pgf@y%
                \pgf@x\pgf@y%
            \fi%
            \pgf@x1.41421\pgf@x%
            \pgf@y\pgf@x%
        \else%
            %
            % Get the rotation (with rounding).
            %
            \pgfmathmod{\pgfkeysvalueof{/pgf/shape border rotate}}{360}%
            \afterassignment\pgfmath@gobbletilpgfmath@%
            \expandafter\c@pgf@counta\pgfmathresult\relax\pgfmath@%
            \advance\c@pgf@counta45\relax%
            \divide\c@pgf@counta90\relax%
            \multiply\c@pgf@counta90\relax%
            \ifnum\c@pgf@counta<0\relax%
                \advance\c@pgf@counta360\relax%
            \fi%
            \edef\rotate{\the\c@pgf@counta}%
            %
            % Calculate the width and height of the node
            % contents, according to any border rotation.
            %
            \ifnum\c@pgf@counta=90\relax%
                \pgf@xc\pgf@x%
                \pgf@x\pgf@y%
                \pgf@y\pgf@xc%
            \else%
                \ifnum\c@pgf@counta=270\relax%
                    \pgf@xc\pgf@x%
                    \pgf@x\pgf@y%
                    \pgf@y\pgf@xc%
                \fi%
            \fi%
        \fi%
        \pgfmathmod{\pgfkeysvalueof{/pgf/trapezium left angle}}{360}%
        \ifdim\pgfmathresult pt<0pt\relax%
            \pgfmathadd@{\pgfmathresult}{360}%
        \fi%
        \let\leftangle\pgfmathresult%
        \ifdim\leftangle pt=0pt\relax%
            \def\pgfmathresult{0}%
        \else%
            \pgfmathcot@{\leftangle}%
        \fi%
        \pgf@xa2\pgf@y%
        \pgf@xa\pgfmathresult\pgf@xa%
        \pgfmathmod{\pgfkeysvalueof{/pgf/trapezium right angle}}{360}%
        \ifdim\pgfmathresult pt<0pt\relax%
            \pgfmathadd@{\pgfmathresult}{360}%
        \fi%
        \let\rightangle\pgfmathresult%
        \ifdim\rightangle pt=0pt\relax%
            \def\pgfmathresult{0}%
        \else%
            \pgfmathcot@{\rightangle}%
        \fi%
        \pgf@xb2\pgf@y%
        \pgf@xb\pgfmathresult\pgf@xb%
        %
        % Take into account minimum height and width.
        %
        \pgfmathsetlength\pgf@ya{\pgfkeysvalueof{/pgf/minimum height}}%
        \ifdim\pgf@y<.5\pgf@ya%
            \ifpgf@lib@sh@trapeziumstretches%
                \pgf@y.5\pgf@ya%
            \else%
                \ifpgf@lib@sh@trapeziumstretchesbody%
                    \pgf@y.5\pgf@ya%
                \else%
                    \pgfmathreciprocal@{\pgfmath@tonumber{\pgf@y}}%
                    \pgf@yb.5\pgf@ya%
                    \pgf@yb\pgfmathresult\pgf@yb%
                    \pgf@y.5\pgf@ya%
                    \pgf@x\pgfmath@tonumber{\pgf@yb}\pgf@x%
                    \pgf@xa\pgfmath@tonumber{\pgf@yb}\pgf@xa%
                    \pgf@xb\pgfmath@tonumber{\pgf@yb}\pgf@xb%
                \fi%
            \fi%
        \fi%
        \pgfmathsetlength\pgf@ya{\pgfkeysvalueof{/pgf/minimum width}}%
        \pgf@xc2\pgf@x%
        \ifdim\pgf@xa<0pt\relax%
            \advance\pgf@xc-\pgf@xa%
        \else%
            \advance\pgf@xc\pgf@xa%
        \fi%
        \ifdim\pgf@xb<0pt\relax%
            \advance\pgf@xc-\pgf@xb%
        \else%
            \advance\pgf@xc\pgf@xb%
        \fi%
        \ifdim\pgf@xc<\pgf@ya%
            \ifpgf@lib@sh@trapeziumstretchesbody%
                \pgf@xc-\pgf@xc%
                \advance\pgf@xc\pgf@ya%
                \advance\pgf@x.5\pgf@xc%
            \else%
                \pgfmathreciprocal@{\pgfmath@tonumber{\pgf@xc}}%
                \pgf@ya\pgfmathresult\pgf@ya%
                \pgf@x\pgfmath@tonumber{\pgf@ya}\pgf@x%
                \pgf@xa\pgfmath@tonumber{\pgf@ya}\pgf@xa%
                \pgf@xb\pgfmath@tonumber{\pgf@ya}\pgf@xb%
                \ifpgf@lib@sh@trapeziumstretches%
                \else%
                    \pgf@y\pgfmath@tonumber{\pgf@ya}\pgf@y%
                \fi%
            \fi%
        \fi%
        \edef\halfheight{\the\pgf@y}%
        \edef\halfwidth{\the\pgf@x}%
        \edef\leftextension{\the\pgf@xa}%
        \edef\rightextension{\the\pgf@xb}%
        \pgf@xc2\pgf@x%
        %
        % Take the larger of the outer sep.
        %
        \pgfmathsetlength\pgf@x{\pgfkeysvalueof{/pgf/outer xsep}}%
        \pgfmathsetlength\pgf@y{\pgfkeysvalueof{/pgf/outer ysep}}%
        \ifdim\pgf@y>\pgf@x%
            \pgf@x\pgf@y%
        \fi%
        \edef\outersep{\the\pgf@x}%
        %
        % The \externalradius is a length that is
        % guarenteed to produce a point outside the trapezium.
        %
        \advance\pgf@xc2.0\pgf@x%
        \pgf@yc\halfheight\relax%
        \multiply\pgf@yc2\relax%
        \advance\pgf@yc2.0\pgf@x%
        \ifdim\pgf@xc<\pgf@yc%
            \edef\externalradius{\the\pgf@yc}%
        \else%
            \edef\externalradius{\the\pgf@xc}%
        \fi%
        %
        % Calculate the centre base and mid poins of the node.
        %
        \pgfextract@process\centerpoint{%
            \pgf@x.5\wd\pgfnodeparttextbox%
            \pgf@y.5\ht\pgfnodeparttextbox%
            \advance\pgf@y-.5\dp\pgfnodeparttextbox%
        }%
        \pgfextract@process\basepoint{%
            \pgf@x.5\wd\pgfnodeparttextbox%
            \pgf@y0pt\relax%
        }%
        \pgfextract@process\midpoint{%
            \pgf@x.5\wd\pgfnodeparttextbox%
            \pgfmathsetlength\pgf@y{+.5ex}%
        }%
        %
        % Calculate each point on the trapezium (without rotation).
        %
        \pgfextract@process\lowerleftpoint{%
            \centerpoint%
            \advance\pgf@x-\halfwidth\relax%
            \ifdim\leftextension>0pt\relax%
                \advance\pgf@x-\leftextension\relax%
            \fi%
            \advance\pgf@y-\halfheight\relax%
        }%
        \pgfextract@process\upperleftpoint{%
            \centerpoint%
            \advance\pgf@x-\halfwidth\relax%
            \ifdim\leftextension<0pt\relax%
                \advance\pgf@x\leftextension\relax%
            \fi%
            \advance\pgf@y\halfheight\relax%
        }%
        \pgfextract@process\upperrightpoint{%
            \centerpoint%
            \advance\pgf@x\halfwidth\relax%
            \ifdim\rightextension<0pt\relax%
                \advance\pgf@x-\rightextension\relax%
            \fi%
            \advance\pgf@y\halfheight\relax%
        }%
        \pgfextract@process\lowerrightpoint{%
            \centerpoint%
            \advance\pgf@x\halfwidth\relax%
            \ifdim\rightextension>0pt\relax%
                \advance\pgf@x\rightextension\relax%
            \fi%
            \advance\pgf@y-\halfheight\relax%
        }%
        %
        % Now calculate the adjustment for the miter length at each corner
        % of the trapezium. This ensures more accurate anchor positioning
        % when the line width is particularly thick.
        %
        \pgfextract@process\lowerleftmiter{%
            \pgfmathanglebetweenlines{\lowerleftpoint}{\lowerrightpoint}{\lowerleftpoint}{\upperleftpoint}%
            \pgfmathmultiply@{\pgfmathresult}{.5}%
            \pgfmathtan@{\pgfmathresult}%
            \pgfmathreciprocal@{\pgfmathresult}%
            \pgf@x-\outersep\relax%
            \pgf@x\pgfmathresult\pgf@x%
            \pgf@y-\outersep\relax%
        }%
        \pgfextract@process\upperleftmiter{%
            \pgfmathanglebetweenlines{\upperleftpoint}{\lowerleftpoint}{\upperleftpoint}{\upperrightpoint}%
            \pgfmathmultiply@{\pgfmathresult}{.5}%
            \pgfmathtan@{\pgfmathresult}%
            \pgfmathreciprocal@{\pgfmathresult}%
            \pgf@x-\outersep\relax%
            \pgf@x\pgfmathresult\pgf@x%
            \pgf@y\outersep\relax%
        }%
        \pgfextract@process\upperrightmiter{%
            \pgfmathanglebetweenlines{\upperrightpoint}{\upperleftpoint}{\upperrightpoint}{\lowerrightpoint}%
            \pgfmathmultiply@{\pgfmathresult}{.5}%
            \pgfmathtan@{\pgfmathresult}%
            \pgfmathreciprocal@{\pgfmathresult}%
            \pgf@x\outersep\relax%
            \pgf@x\pgfmathresult\pgf@x%
            \pgf@y\outersep\relax%
        }%
        \pgfextract@process\lowerrightmiter{%
            \pgfmathanglebetweenlines{\lowerrightpoint}{\upperrightpoint}{\lowerrightpoint}{\lowerleftpoint}%
            \pgfmathmultiply@{\pgfmathresult}{.5}%
            \pgfmathtan@{\pgfmathresult}%
            \pgfmathreciprocal@{\pgfmathresult}%
            \pgf@x\outersep\relax%
            \pgf@x\pgfmathresult\pgf@x%
            \pgf@y-\outersep\relax%
        }%
        %
        % Now calculate the corners for determining anchor border
        % points, by adding the adjustment for the miter length.
        %
        \pgfextract@process\lowerleftborderpoint{%
            \pgfpointadd{\lowerleftpoint}{\lowerleftmiter}%
        }%
        \pgfextract@process\upperleftborderpoint{%
            \pgfpointadd{\upperleftpoint}{\upperleftmiter}%
        }%
        \pgfextract@process\upperrightborderpoint{%
            \pgfpointadd{\upperrightpoint}{\upperrightmiter}%
        }%
        \pgfextract@process\lowerrightborderpoint{%
            \pgfpointadd{\lowerrightpoint}{\lowerrightmiter}%
        }%
        %
        % Calculate the angle from the centerpoint to each corner.
        % Rotation is not important here (see \anchorborder code).
        %
        \pgfmathanglebetweenpoints{\centerpoint}{\lowerleftborderpoint}%
        \let\angletolowerleft\pgfmathresult%
        \pgfmathanglebetweenpoints{\centerpoint}{\upperleftborderpoint}%
        \let\angletoupperleft\pgfmathresult%
        \pgfmathanglebetweenpoints{\centerpoint}{\upperrightborderpoint}%
        \let\angletoupperright\pgfmathresult%
        \pgfmathanglebetweenpoints{\centerpoint}{\lowerrightborderpoint}%
        \let\angletolowerright\pgfmathresult%
        %
        % Do the same for the *inversely rotated* base point...
        %
        \pgfextract@process\rotatedbasepoint{%
            \pgfmathrotatepointaround{\basepoint}{\centerpoint}{-\rotate}%
        }%
        \pgfmathanglebetweenpoints{\rotatedbasepoint}{\lowerleftborderpoint}%
        \let\baseangletolowerleft\pgfmathresult%
        \pgfmathanglebetweenpoints{\rotatedbasepoint}{\upperleftborderpoint}%
        \let\baseangletoupperleft\pgfmathresult%
        \pgfmathanglebetweenpoints{\rotatedbasepoint}{\upperrightborderpoint}%
        \let\baseangletoupperright\pgfmathresult%
        \pgfmathanglebetweenpoints{\rotatedbasepoint}{\lowerrightborderpoint}%
        \let\baseangletolowerright\pgfmathresult%
        %
        % And the *inversely rotated* mid point...
        %
        \pgfextract@process\rotatedmidpoint{%
            \pgfmathrotatepointaround{\midpoint}{\centerpoint}{-\rotate}%
        }%
        \pgfmathanglebetweenpoints{\rotatedmidpoint}{\lowerleftborderpoint}%
        \let\midangletolowerleft\pgfmathresult%
        \pgfmathanglebetweenpoints{\rotatedmidpoint}{\upperleftborderpoint}%
        \let\midangletoupperleft\pgfmathresult%
        \pgfmathanglebetweenpoints{\rotatedmidpoint}{\upperrightborderpoint}%
        \let\midangletoupperright\pgfmathresult%
        \pgfmathanglebetweenpoints{\rotatedmidpoint}{\lowerrightborderpoint}%
        \let\midangletolowerright\pgfmathresult%
        %
        % Now rotate the border points around the centerpoint...
        %
        \pgfextract@process\lowerleftborderpoint{%
            \pgfmathrotatepointaround%
                {\lowerleftborderpoint}%
                {\centerpoint}%
                {\rotate}%
        }%
        \pgfextract@process\upperleftborderpoint{%
            \pgfmathrotatepointaround%
                {\upperleftborderpoint}%
                {\centerpoint}%
                {\rotate}%
        }%
        \pgfextract@process\upperrightborderpoint{%
            \pgfmathrotatepointaround%
                {\upperrightborderpoint}%
                {\centerpoint}%
                {\rotate}%
        }%
        \pgfextract@process\lowerrightborderpoint{%
            \pgfmathrotatepointaround%
                {\lowerrightborderpoint}%
                {\centerpoint}%
                {\rotate}%
        }%
        %
        % ...and the points for drawing the border (i.e. no outer sep).
        %
        \pgfextract@process\lowerleftpoint{%
            \pgfmathrotatepointaround{\lowerleftpoint}{\centerpoint}{\rotate}%
        }%
        \pgfextract@process\upperleftpoint{%
            \pgfmathrotatepointaround{\upperleftpoint}{\centerpoint}{\rotate}%
        }%
        \pgfextract@process\upperrightpoint{%
            \pgfmathrotatepointaround{\upperrightpoint}{\centerpoint}{\rotate}%
        }%
        \pgfextract@process\lowerrightpoint{%
            \pgfmathrotatepointaround{\lowerrightpoint}{\centerpoint}{\rotate}%
        }%
        %
        % And finally, save all the relevant stuff.
        %
        \addtosavedmacro{\lowerleftpoint}%
        \addtosavedmacro{\upperleftpoint}%
        \addtosavedmacro{\upperrightpoint}%
        \addtosavedmacro{\lowerrightpoint}%
        %
        \addtosavedmacro{\lowerleftborderpoint}%
        \addtosavedmacro{\upperleftborderpoint}%
        \addtosavedmacro{\upperrightborderpoint}%
        \addtosavedmacro{\lowerrightborderpoint}%
        %
        \addtosavedmacro{\angletolowerleft}%
        \addtosavedmacro{\angletoupperleft}%
        \addtosavedmacro{\angletoupperright}%
        \addtosavedmacro{\angletolowerright}%
        %
        \addtosavedmacro{\baseangletolowerleft}%
        \addtosavedmacro{\baseangletoupperleft}%
        \addtosavedmacro{\baseangletoupperright}%
        \addtosavedmacro{\baseangletolowerright}%
        %
        \addtosavedmacro{\midangletolowerleft}%
        \addtosavedmacro{\midangletoupperleft}%
        \addtosavedmacro{\midangletoupperright}%
        \addtosavedmacro{\midangletolowerright}%
        %
        \addtosavedmacro{\rotate}%
        \addtosavedmacro{\externalradius}%
    }%
    \savedanchor\centerpoint{%
        \pgf@x.5\wd\pgfnodeparttextbox%
        \pgf@y.5\ht\pgfnodeparttextbox%
        \advance\pgf@y-.5\dp\pgfnodeparttextbox%
    }%
    \savedanchor\basepoint{%
        \pgf@x.5\wd\pgfnodeparttextbox%
        \pgf@y0pt\relax%
    }%
    \savedanchor\midpoint{%
        \pgf@x.5\wd\pgfnodeparttextbox%
        \pgfmathsetlength\pgf@y{+.5ex}%
    }%
    \anchor{center}{\centerpoint}%
    \anchor{base}{\basepoint}%
    \anchor{base east}{%
        \installtrapeziumparameters%
        \let\pgf@trapeziumanchorborderreferencepoint\basepoint%
        \csname pgf@anchor@trapezium@border\endcsname{\pgfqpoint{\externalradius}{0pt}}%
    }%
    \anchor{base west}{%
        \installtrapeziumparameters%
        \let\pgf@trapeziumanchorborderreferencepoint\basepoint%
        \csname pgf@anchor@trapezium@border\endcsname{\pgfqpoint{-\externalradius}{0pt}}%
    }%
    \anchor{mid}{\midpoint}%
    \anchor{mid east}{%
        \installtrapeziumparameters%
        \let\pgf@trapeziumanchorborderreferencepoint\midpoint%
        \csname pgf@anchor@trapezium@border\endcsname{\pgfqpoint{\externalradius}{0pt}}%
    }%
    \anchor{mid west}{%
        \installtrapeziumparameters%
        \let\pgf@trapeziumanchorborderreferencepoint\midpoint%
        \csname pgf@anchor@trapezium@border\endcsname{\pgfqpoint{-\externalradius}{0pt}}%
    }%
    \anchor{north}{%
        \installtrapeziumparameters%
        \csname pgf@anchor@trapezium@border\endcsname{\pgfqpoint{0pt}{\externalradius}}%
    }%
    \anchor{south}{%
        \installtrapeziumparameters%
        \csname pgf@anchor@trapezium@border\endcsname{\pgfqpoint{0pt}{-\externalradius}}%
    }%
    \anchor{east}{%
        \installtrapeziumparameters%
        \csname pgf@anchor@trapezium@border\endcsname{\pgfqpoint{\externalradius}{0pt}}%
    }%
    \anchor{west}{%
        \installtrapeziumparameters%
        \csname pgf@anchor@trapezium@border\endcsname{\pgfqpoint{-\externalradius}{0pt}}%
    }%
    \anchor{north east}{%
        \installtrapeziumparameters%
        \csname pgf@anchor@trapezium@border\endcsname{\pgfqpoint{\externalradius}{\externalradius}}%
    }%
    \anchor{south west}{%
        \installtrapeziumparameters%
        \csname pgf@anchor@trapezium@border\endcsname{\pgfqpoint{-\externalradius}{-\externalradius}}%
    }%
    \anchor{south east}{%
        \installtrapeziumparameters%
        \csname pgf@anchor@trapezium@border\endcsname{\pgfqpoint{\externalradius}{-\externalradius}}%
    }%
    \anchor{north west}{%
        \installtrapeziumparameters%
        \csname pgf@anchor@trapezium@border\endcsname{\pgfqpoint{-\externalradius}{\externalradius}}%
    }%
    \anchor{bottom left corner}{%
        \installtrapeziumparameters%
        \lowerleftborderpoint%
    }%
    \anchor{top left corner}{%
        \installtrapeziumparameters%
        \upperleftborderpoint%
    }%
    \anchor{top right corner}{%
        \installtrapeziumparameters%
        \upperrightborderpoint%
    }%
    \anchor{bottom right corner}{%
        \installtrapeziumparameters%
        \lowerrightborderpoint%
    }%
    \anchor{left side}{%
        \installtrapeziumparameters%
        \pgfpointlineattime{0.5}{\lowerleftborderpoint}{\upperleftborderpoint}%
    }%
    \anchor{right side}{%
        \installtrapeziumparameters%
        \pgfpointlineattime{0.5}{\lowerrightborderpoint}{\upperrightborderpoint}%
    }%
    \anchor{top side}{%
        \installtrapeziumparameters%
        \pgfpointlineattime{0.5}{\upperleftborderpoint}{\upperrightborderpoint}%
    }%
    \anchor{bottom side}{%
        \installtrapeziumparameters%
        \pgfpointlineattime{0.5}{\lowerleftborderpoint}{\lowerrightborderpoint}%
    }%
    \backgroundpath{%
        \installtrapeziumparameters%
        \pgfpathmoveto{\lowerleftpoint}%
        \pgfpathlineto{\upperleftpoint}%
        \pgfpathlineto{\upperrightpoint}%
        \pgfpathlineto{\lowerrightpoint}%
        \pgfpathclose%
    }%
    \anchorborder{%
        %
        % Save x and y.
        %
        \edef\externalx{\the\pgf@x}%
        \edef\externaly{\the\pgf@y}%
        %
        % This allows anchors base east, base west, mid east and mid west,
        % to redefine the `center' of the node to correctly calculate the
        % border points.
        %
        \pgfutil@ifundefined{pgf@trapeziumanchorborderreferencepoint}%
            {\let\referencepoint\centerpoint}%
            {\let\referencepoint\pgf@trapeziumanchorborderreferencepoint}%
        %
        % Adjust the location of the external
        % point relative to the reference point.
        %
        \referencepoint%
        \pgf@xa\externalx\relax%
        \pgf@ya\externaly\relax%
        \advance\pgf@xa\pgf@x%
        \advance\pgf@ya\pgf@y%
        \edef\externalx{\the\pgf@xa}%
        \edef\externaly{\the\pgf@ya}%
        %
        % Install the required points and angles.
        %
        \installtrapeziumparameters%
        %
        % Get the angle of the external point to the \referencepoint.
        %
        \pgfmathanglebetweenpoints{\referencepoint}{\pgfqpoint{\externalx}{\externaly}}%
        %
        % *Subtract* the rotation from the external angle. This is
        % why the border point angles do not neeed to be rotated.
        %
        \pgfmathsubtract@{\pgfmathresult}{\rotate}%
        \ifdim\pgfmathresult pt<0pt\relax%
            \pgfmathadd@{\pgfmathresult}{360}%
        \fi%
        %
        % Get the relevant angles for the reference point.
        %
        \let\externalangle\pgfmathresult%
        \ifx\referencepoint\basepoint%
            \let\angletoupperright\baseangletoupperright%
            \let\angletoupperleft\baseangletoupperleft%
            \let\angletolowerright\baseangletolowerright%
            \let\angletolowerleft\baseangletolowerleft%
        \else%
            \ifx\referencepoint\midpoint%
                \let\angletoupperright\midangletoupperright%
                \let\angletoupperleft\midangletoupperleft%
                \let\angletolowerright\midangletolowerright%
                \let\angletolowerleft\midangletolowerleft%
        \fi\fi%
        %
        % Find the line on the border...
        %
        \ifdim\externalangle pt<\angletoupperright pt\relax%
            \let\firstpoint\upperrightborderpoint%
            \let\secondpoint\lowerrightborderpoint%
        \else%
            \ifdim\externalangle pt<\angletoupperleft pt\relax%
                \let\firstpoint\upperleftborderpoint%
                \let\secondpoint\upperrightborderpoint%
            \else%
                \ifdim\externalangle pt<\angletolowerleft pt\relax%
                    \let\firstpoint\upperleftborderpoint%
                    \let\secondpoint\lowerleftborderpoint%
                \else%
                    \ifdim\externalangle pt<\angletolowerright pt\relax%
                        \let\firstpoint\lowerleftborderpoint%
                        \let\secondpoint\lowerrightborderpoint%
                    \else%
                        \let\firstpoint\upperrightborderpoint%
                        \let\secondpoint\lowerrightborderpoint%
                    \fi%
                \fi%
            \fi%
        \fi%
        %
        % ...and thus the point on the border.
        %
        \pgfpointintersectionoflines{\referencepoint}{\pgfqpoint{\externalx}{\externaly}}%
                {\firstpoint}{\secondpoint}%
    }%
}%




% Shape semicircle.
%
\pgfdeclareshape{semicircle}{%
    \savedmacro\installsemicircleparameters{%
        %
        % Get the larger of the outer sep.
        %
        \pgfmathsetlength\pgf@x{\pgfkeysvalueof{/pgf/outer xsep}}%
        \pgfmathsetlength\pgf@y{\pgfkeysvalueof{/pgf/outer ysep}}%
        \ifdim\pgf@x<\pgf@y%
            \pgf@x\pgf@y%
        \fi%
        \edef\outersep{\the\pgf@x}%
        %
        % Get the node dimensions.
        %
        \pgfmathsetlength\pgf@x{\pgfkeysvalueof{/pgf/inner xsep}}%
        \advance\pgf@x+.5\wd\pgfnodeparttextbox%
        \pgfmathsetlength\pgf@y{\pgfkeysvalueof{/pgf/inner ysep}}%
        \advance\pgf@y+.5\ht\pgfnodeparttextbox%
        \advance\pgf@y+.5\dp\pgfnodeparttextbox%
        %
        % Use the radius of the incircle if required...
        %
        \ifpgfshapeborderusesincircle%
            %
            % Get the rotation (no rounding).
            %
            \pgfmathsetmacro\rotate{\pgfkeysvalueof{/pgf/shape border rotate}}%
            \ifdim\pgf@x<\pgf@y%
                \pgf@x\pgf@y%
            \fi%
            \pgf@x1.41421\pgf@x%
            \edef\halfheight{\the\pgf@x}%
            \advance\pgf@x\pgf@x%
        \else%
            %
            % Get the rotation (with rounding).
            %
            \pgfmathmod{\pgfkeysvalueof{/pgf/shape border rotate}}{360}%
            \afterassignment\pgfmath@gobbletilpgfmath@%
            \expandafter\c@pgf@counta\pgfmathresult\relax\pgfmath@%
            \advance\c@pgf@counta45\relax%
            \divide\c@pgf@counta90\relax%
            \multiply\c@pgf@counta90\relax%
            \ifnum\c@pgf@counta<0\relax%
                \advance\c@pgf@counta360\relax%
            \fi%
            \edef\rotate{\the\c@pgf@counta}%
            %
            % Calculate the width and height of the node
            % contents, according to any border rotation.
            %
            \ifnum\c@pgf@counta=90\relax%
                \pgf@xc\pgf@x%
                \pgf@x\pgf@y%
                \pgf@y\pgf@xc%
            \else%
                \ifnum\c@pgf@counta=270\relax%
                    \pgf@xc\pgf@x%
                    \pgf@x\pgf@y%
                    \pgf@y\pgf@xc%
                \fi%
            \fi%
            \advance\pgf@y\pgf@y%
            \pgfmathveclen@{\pgfmath@tonumber{\pgf@x}}{\pgfmath@tonumber{\pgf@y}}%
            \pgf@x\pgfmathresult pt\relax%
            \pgf@y.5\pgf@y%
            \edef\halfheight{\the\pgf@y}%
        \fi%
        \edef\defaultradius{\the\pgf@x}%
        %
        % Take into account minimum height and width.
        %
        \pgfmathsetlength\pgf@xa{\pgfkeysvalueof{/pgf/minimum width}}%
        \ifdim\pgf@x<.5\pgf@xa%
            \pgf@x.5\pgf@xa%
        \fi%
        \pgfmathsetlength\pgf@ya{\pgfkeysvalueof{/pgf/minimum height}}%
        \ifdim\pgf@x<\pgf@ya%
            \pgf@x\pgf@ya%
        \fi%
        \edef\semicircleradius{\the\pgf@x}%
        %
        % Find the center/base/mid of the semi circle node.
        %
        \pgfextract@process\centerpoint{%
            \pgf@x.5\wd\pgfnodeparttextbox%
            \pgf@y.5\ht\pgfnodeparttextbox%
            \advance\pgf@y+-.5\dp\pgfnodeparttextbox%
        }%
        \pgfextract@process\basepoint{%
            \pgf@x.5\wd\pgfnodeparttextbox%
            \pgf@y0pt\relax%
        }%
        \pgfextract@process\midpoint{%
            \pgf@x.5\wd\pgfnodeparttextbox%
            \pgfmathsetlength\pgf@y{+.5ex}%
        }%
        \pgfextract@process\semicirclecenterpoint{%
            \centerpoint%
            \pgf@ya\semicircleradius\relax%
            \advance\pgf@ya-\defaultradius\relax%
            %
            % The point is adjusted to make the node contents
            % appear more centered if the node is enlarged to
            % some minimum size. The factor .4 just seems OK.
            %
            \pgf@ya.4\pgf@ya%
            \advance\pgf@y-\pgf@ya%
            \pgf@ya\halfheight\relax%
            \advance\pgf@y-\pgf@ya%
        }%
        %
        % Add the outer sep to the radius here.
        %
        \pgf@x\semicircleradius\relax%
        \advance\pgf@x\outersep\relax%
        \edef\semicircleradius{\the\pgf@x}%
        \pgfpointdiff{\centerpoint}{\semicirclecenterpoint}%
        \ifdim\pgf@y<0pt\pgf@y-\pgf@y\fi%
        \edef\centerpointdiff{\the\pgf@y}%
        %
        % Calculate the start and end points on the border.
        %
        \pgfextract@process\arcstartborder{%
            \semicirclecenterpoint%
            \advance\pgf@x\semicircleradius\relax%
        }%
        \pgfextract@process\arcendborder{%
            \semicirclecenterpoint%
            \advance\pgf@x-\semicircleradius\relax%
        }%
        %
        % Calculate the start and end *corner* points on the border.
        % This is needed to accommodate the outer sep.
        %
        \pgfextract@process\arcstartcorner{%
            \arcstartborder%
            \advance\pgf@y-\outersep\relax%
        }%
        \pgfextract@process\arcendcorner{%
            \arcendborder%
            \advance\pgf@y-\outersep\relax%
        }
        %
        % Now calculate all the relevant angles.
        %
        % For the center point.
        %
        \pgfmathanglebetweenpoints{\centerpoint}{\arcstartborder}%
        \let\angletoarcstartborder\pgfmathresult%
        \pgfmathanglebetweenpoints{\centerpoint}{\arcendborder}%
        \let\angletoarcendborder\pgfmathresult%
        \pgfmathanglebetweenpoints{\centerpoint}{\arcstartcorner}%
        \let\angletoarcstartcorner\pgfmathresult%
        \pgfmathanglebetweenpoints{\centerpoint}{\arcendcorner}%
        \let\angletoarcendcorner\pgfmathresult%
        %
        % For the basepoint (rotated about the center point).
        %
        \pgfextract@process\rotatedbasepoint{%
            \pgfmathrotatepointaround{\basepoint}{\centerpoint}{-\rotate}%
        }%
        \pgfmathanglebetweenpoints{\rotatedbasepoint}{\arcstartborder}%
        \let\baseangletoarcstartborder\pgfmathresult%
        \pgfmathanglebetweenpoints{\rotatedbasepoint}{\arcendborder}%
        \let\baseangletoarcendborder\pgfmathresult%
        \pgfmathanglebetweenpoints{\rotatedbasepoint}{\arcstartcorner}%
        \let\baseangletoarcstartcorner\pgfmathresult%
        \pgfmathanglebetweenpoints{\rotatedbasepoint}{\arcendcorner}%
        \let\baseangletoarcendcorner\pgfmathresult%
        %
        % For the midpoint (rotated about the center point).
        %
        \pgfextract@process\rotatedmidpoint{%
            \pgfmathrotatepointaround{\midpoint}{\centerpoint}{-\rotate}%
        }%
        \pgfmathanglebetweenpoints{\rotatedmidpoint}{\arcstartborder}%
        \let\midangletoarcstartborder\pgfmathresult%
        \pgfmathanglebetweenpoints{\rotatedmidpoint}{\arcendborder}%
        \let\midangletoarcendborder\pgfmathresult%
        \pgfmathanglebetweenpoints{\rotatedmidpoint}{\arcstartcorner}%
        \let\midangletoarcstartcorner\pgfmathresult%
        \pgfmathanglebetweenpoints{\rotatedmidpoint}{\arcendcorner}%
        \let\midangletoarcendcorner\pgfmathresult%
        %
        % Now, rotate the semicircle points around the centerpoint.
        %
        \pgfextract@process\semicirclecenterpoint{%
            \pgfmathrotatepointaround{\semicirclecenterpoint}{\centerpoint}{\rotate}%
        }%
        \pgfextract@process\arcstartborder{%
            \pgfmathrotatepointaround{\arcstartborder}{\centerpoint}{\rotate}%
        }%
        \pgfextract@process\arcendborder{%
            \pgfmathrotatepointaround{\arcendborder}{\centerpoint}{\rotate}%
        }%
        \pgfextract@process\arcstartcorner{%
            \pgfmathrotatepointaround{\arcstartcorner}{\centerpoint}{\rotate}%
        }%
        \pgfextract@process\arcendcorner{%
            \pgfmathrotatepointaround{\arcendcorner}{\centerpoint}{\rotate}%
        }%
        %
        % Calculate the distance between the base point and the (rotated) semicircle center.
        %
        \pgfpointdiff{\basepoint}{\semicirclecenterpoint}%
        \pgfmathveclen@{\pgfmath@tonumber{\pgf@x}}{\pgfmath@tonumber{\pgf@y}}%
        \edef\basesemicirclecenterdiff{\pgfmathresult pt}%
        %
        % Calculate the distance between the base point and the (rotated) semicircle center.
        %
        \pgfpointdiff{\midpoint}{\semicirclecenterpoint}%
        \pgfmathveclen@{\pgfmath@tonumber{\pgf@x}}{\pgfmath@tonumber{\pgf@y}}%
        \edef\midsemicirclecenterdiff{\pgfmathresult pt}%
        %
        % And save all the stuff.
        %
        \addtosavedmacro{\rotate}%
        \addtosavedmacro{\outersep}%
        \addtosavedmacro{\semicircleradius}%
        %
        \addtosavedmacro{\arcstartborder}%
        \addtosavedmacro{\arcendborder}%
        \addtosavedmacro{\arcstartcorner}%
        \addtosavedmacro{\arcendcorner}%
        \addtosavedmacro{\semicirclecenterpoint}%
        %
        \addtosavedmacro{\angletoarcstartborder}%
        \addtosavedmacro{\angletoarcendborder}%
        \addtosavedmacro{\angletoarcstartcorner}%
        \addtosavedmacro{\angletoarcendcorner}%
        %
        \addtosavedmacro{\centerpointdiff}%
        \addtosavedmacro{\basesemicirclecenterdiff}%
        \addtosavedmacro{\midsemicirclecenterdiff}%
        %
        \addtosavedmacro{\baseangletoarcstartborder}%
        \addtosavedmacro{\baseangletoarcendborder}%
        \addtosavedmacro{\baseangletoarcstartcorner}%
        \addtosavedmacro{\baseangletoarcendcorner}%
        %
        \addtosavedmacro{\midangletoarcstartborder}%
        \addtosavedmacro{\midangletoarcendborder}%
        \addtosavedmacro{\midangletoarcstartcorner}%
        \addtosavedmacro{\midangletoarcendcorner}%
    }%
    \savedanchor{\centerpoint}{%
        \pgfmathsetlength\pgf@x{+.5\wd\pgfnodeparttextbox}%
        \pgfmathsetlength\pgf@y{+.5\ht\pgfnodeparttextbox}%
        \pgfmathaddtolength\pgf@y{-.5\dp\pgfnodeparttextbox}%
    }%
    \savedanchor{\basepoint}{%
        \pgfmathsetlength\pgf@x{+.5\wd\pgfnodeparttextbox}%
        \pgf@y0pt\relax%
    }%
    \savedanchor{\midpoint}{%
        \pgfmathsetlength\pgf@x{+.5\wd\pgfnodeparttextbox}%
        \pgfmathsetlength\pgf@y{+.5ex}%
    }%
    \anchor{center}{\centerpoint}%
    \anchor{base}{\basepoint}%
    \anchor{base west}{%
        \installsemicircleparameters%
        \let\pgf@semicirclereferencepoint\basepoint%
        \csname pgf@anchor@semicircle@border\endcsname{\pgfqpoint{-\semicircleradius}{0pt}}%
    }%
    \anchor{base east}{%
        \installsemicircleparameters%
        \let\pgf@semicirclereferencepoint\basepoint%
        \csname pgf@anchor@semicircle@border\endcsname{\pgfqpoint{\semicircleradius}{0pt}}%
    }%
    \anchor{mid}{\midpoint}%
    \anchor{mid west}{%
        \installsemicircleparameters%
        \let\pgf@semicirclereferencepoint\midpoint%
        \csname pgf@anchor@semicircle@border\endcsname{\pgfqpoint{-\semicircleradius}{0pt}}%
    }%
    \anchor{mid east}{%
        \installsemicircleparameters%
        \let\pgf@semicirclereferencepoint\midpoint%
        \csname pgf@anchor@semicircle@border\endcsname{\pgfqpoint{\semicircleradius}{0pt}}%
    }%
    \anchor{apex}{%
        \installsemicircleparameters%
        \pgfmathadd@{\rotate}{90}%
        \let\angle\pgfmathresult%
        \pgfpointadd{\semicirclecenterpoint}{\pgfqpointpolar{\angle}{\semicircleradius}}%
    }%
    \anchor{arc start}{%
        \installsemicircleparameters%
        \arcstartcorner%
    }%
    \anchor{arc end}{%
        \installsemicircleparameters%
        \arcendcorner%
    }%
    \anchor{chord center}{%
        \installsemicircleparameters%
        \pgfpointlineattime{0.5}{\arcstartcorner}{\arcendcorner}%
    }%
    \anchor{north}{%
        \installsemicircleparameters%
        \csname pgf@anchor@semicircle@border\endcsname{\pgfqpointpolar{90}{\semicircleradius}}%
    }%
    \anchor{south}{%
        \installsemicircleparameters%
        \csname pgf@anchor@semicircle@border\endcsname{\pgfqpointpolar{270}{\semicircleradius}}%
    }%
    \anchor{east}{%
        \installsemicircleparameters%
        \csname pgf@anchor@semicircle@border\endcsname{\pgfqpointpolar{0}{\semicircleradius}}%
    }%
    \anchor{west}{%
        \installsemicircleparameters%
        \csname pgf@anchor@semicircle@border\endcsname{\pgfqpointpolar{180}{\semicircleradius}}%
    }%
    \anchor{north west}{%
        \installsemicircleparameters%
        \csname pgf@anchor@semicircle@border\endcsname{\pgfqpointpolar{135}{\semicircleradius}}%
    }%
    \anchor{south west}{%
        \installsemicircleparameters%
        \csname pgf@anchor@semicircle@border\endcsname{\pgfqpointpolar{225}{\semicircleradius}}%
    }%
    \anchor{north east}{%
        \installsemicircleparameters%
        \csname pgf@anchor@semicircle@border\endcsname{\pgfqpointpolar{45}{\semicircleradius}}%
    }%
    \anchor{south east}{%
        \installsemicircleparameters%
        \csname pgf@anchor@semicircle@border\endcsname{\pgfqpointpolar{315}{\semicircleradius}}%
    }%
    \backgroundpath{%
        \installsemicircleparameters%
        \pgf@x\semicircleradius\relax%
        \advance\pgf@x-\outersep\relax%
        \edef\semicircleradius{\the\pgf@x}%
        \pgfpathmoveto{\pgfpointadd{\semicirclecenterpoint}{\pgfqpointpolar{\rotate}{\semicircleradius}}}%
        {%
            \pgftransformrotate{\rotate}%
            \pgfpatharc{0}{180}{\semicircleradius}%
            \pgfpathclose%
        }%
    }%
    \anchorborder{%
        %
        % Save x and y.
        %
        \edef\externalx{\the\pgf@x}%
        \edef\externaly{\the\pgf@y}%
        %
        % Check if a reference point has been defined (i.e. \midpoint or \basepoint).
        %
        \pgfutil@ifundefined{pgf@semicirclereferencepoint}%
            {\let\referencepoint\centerpoint}%
            {\let\referencepoint\pgf@semicirclereferencepoint}%
        %
        % Adjust the location of the external  point relative to the reference point.
        %
        \referencepoint%
        \pgf@xa\externalx\relax%
        \pgf@ya\externaly\relax%
        \advance\pgf@xa\pgf@x%
        \advance\pgf@ya\pgf@y%
        \edef\externalx{\the\pgf@xa}%
        \edef\externaly{\the\pgf@ya}%
        %
        % Install the required points and angles.
        %
        \installsemicircleparameters%
        %
        % Get the angle of the external point to the \referencepoint.
        %
        \pgfmathanglebetweenpoints{\referencepoint}{\pgfqpoint{\externalx}{\externaly}}%
        \let\externalangle\pgfmathresult%
        %
        % *Subtract* the rotation from the external angle.
        %
        \pgfmathsubtract@{\pgfmathresult}{\rotate}%
        \ifdim\pgfmathresult pt<0pt\relax%
            \pgfmathadd@{\pgfmathresult}{360}%
        \fi%
        \let\angle\pgfmathresult%
        \ifx\referencepoint\basepoint%
            \let\angletoarcstartborder\baseangletoarcstartborder%
            \let\angletoarcendborder\baseangletoarcendborder%
            \let\angletoarcstartcorner\baseangletoarcstartcorner%
            \let\angletoarcendcorner\baseangletoarcendcorner%
            \let\centerpointdiff\basesemicirclecenterdiff%
        \else%
            \ifx\referencepoint\midpoint%
                \let\angletoarcstartborder\midangletoarcstartborder%
                \let\angletoarcendborder\midangletoarcendborder%
                \let\angletoarcstartcorner\midangletoarcstartcorner%
                \let\angletoarcendcorner\midangletoarcendcorner%
                \let\centerpointdiff\midsemicirclecenterdiff%
        \fi\fi%
        %
        % Determine if the line will cross the semicircle arc, or the chord.
        %
        \ifdim\angle pt>\angletoarcstartborder pt\relax%
            \let\firstpoint\pgfutil@empty%
            \let\secondpoint\pgfutil@empty%
        \else%
            \ifdim\angle pt>\angletoarcstartcorner pt\relax%
                \let\firstpoint\arcstartcorner%
                \let\secondpoint\arcstartborder%
            \else%
                \ifdim\angle pt>\angletoarcendcorner pt\relax%
                    \let\firstpoint\arcendcorner%
                    \let\secondpoint\arcstartcorner%
                \else%
                    \ifdim\angle pt>\angletoarcendborder pt\relax%
                        \let\firstpoint\arcendborder%
                        \let\secondpoint\arcendcorner%
                    \else%
                        \let\firstpoint\pgfutil@empty%
                        \let\secondpoint\pgfutil@empty%
                    \fi%
                \fi%
            \fi%
        \fi%
        \ifx\firstpoint\pgfutil@empty
            %
            % Calculate the angle from the centre of the semicircle to the
            % point on the semicircle arc which intersects the line from
            % the external point to the reference point...
            %
            \pgfmathanglebetweenlines{\referencepoint}{\pgfqpoint{\externalx}{\externaly}}%
                {\semicirclecenterpoint}{\referencepoint}%
            \pgfmathsin@{\pgfmathresult}%
            \let\sineangle\pgfmathresult%
            \pgf@x\semicircleradius\relax%
            \pgfmathreciprocal@{\pgfmath@tonumber{\pgf@x}}%
            \let\reciprocalradius\pgfmathresult%
            \pgf@x\centerpointdiff\relax%
            \pgf@x\sineangle\pgf@x%
            \pgf@x\reciprocalradius\pgf@x%
            \pgfmathasin@{\pgfmath@tonumber{\pgf@x}}%
            \pgf@x\pgfmathresult pt\relax%
            \advance\pgf@x\externalangle pt\relax%
            \edef\angle{\pgfmath@tonumber{\pgf@x}}%
            %
            % ...and thus the point on the border.
            %
            \pgfpointadd{\semicirclecenterpoint}{\pgfqpointpolar{\angle}{\semicircleradius}}%
        \else%
            %
            % Calculate the the point where the semicircle chord intersects
            % the line from the external point to the reference point.
            %
            \pgfpointintersectionoflines{\referencepoint}{\pgfqpoint{\externalx}{\externaly}}%
                {\firstpoint}{\secondpoint}%
        \fi%
    }%
}%


% Keys for shape isosceles triangle
%
% /pgf/isosceles triangle apex angle : angle at the apex of the triangle.
\pgfkeys{/pgf/.cd,
    isosceles triangle apex angle/.initial=45,
    isosceles triangle stretches/.is if=pgf@lib@sh@isosceslestrianglestretches
}%
\newif\ifpgf@lib@sh@isosceslestrianglestretches

\pgfdeclareshape{isosceles triangle}{%
    \savedmacro\trianglepoints{%
        \pgfmathsetlength\pgf@x{\pgfkeysvalueof{/pgf/isosceles triangle apex angle}}%
        \divide\pgf@x2\relax%
        \edef\halfapexangle{\pgfmath@tonumber{\pgf@x}}%
        \addtosavedmacro\halfapexangle%
        %
        \pgfmathtan@{\halfapexangle}%
        \let\tanhalfapexangle\pgfmathresult%
        %
        \pgfmathcot@{\halfapexangle}%
        \let\cothalfapexangle\pgfmathresult%
        %
        \pgfmathsin@{\halfapexangle}%
        \let\sinhalfapexangle\pgfmathresult%
        %
        \pgfmathreciprocal@{\pgfmathresult}%
        \let\cosechalfapexangle\pgfmathresult%
        %
        \pgfmathsetlength\pgf@x{\pgfkeysvalueof{/pgf/inner xsep}}%
        \advance\pgf@x.5\wd\pgfnodeparttextbox%
        \pgfmathsetlength\pgf@y{\pgfkeysvalueof{/pgf/inner ysep}}%
        \advance\pgf@y.5\ht\pgfnodeparttextbox%
        \advance\pgf@y.5\dp\pgfnodeparttextbox%
        %
        \ifpgfshapeborderusesincircle%
            \pgfmathsetmacro\rotate{\pgfkeysvalueof{/pgf/shape border rotate}}%
            \ifdim\pgf@x<\pgf@y%
                \pgf@x\pgf@y%
            \else%
                \pgf@y\pgf@x%
            \fi%
            \pgf@x1.414213\pgf@x%
            \pgf@y1.414213\pgf@y%
            %
            \pgf@xa\pgf@x%
            \advance\pgf@xa\cosechalfapexangle\pgf@x%
            \pgf@ya\tanhalfapexangle\pgf@xa%
        \else%
            \pgfmathmod{\pgfkeysvalueof{/pgf/shape border rotate}}{360}%
            \ifdim\pgfmathresult pt<0pt\relax%
                \pgfmathadd@{\pgfmathresult}{360}%
            \fi%
            \pgfmathsetcount\c@pgf@counta{+\pgfmathresult}%
            \advance\c@pgf@counta45\relax%
            \divide\c@pgf@counta90\relax%
            \multiply\c@pgf@counta90\relax%
            \edef\rotate{\the\c@pgf@counta}%
            \ifnum\c@pgf@counta=90\relax%
                \pgf@xa\pgf@x%
                \pgf@x\pgf@y%
                \pgf@y\pgf@xa%
            \else%
                \ifnum\c@pgf@counta=270\relax%
                    \pgf@xa\pgf@x%
                    \pgf@x\pgf@y%
                    \pgf@y\pgf@xa%
                \fi%
            \fi%
            %
            \pgf@xa2.0\pgf@x%
            \pgf@ya\tanhalfapexangle\pgf@xa%
            \advance\pgf@xa\cothalfapexangle\pgf@y%
            \advance\pgf@ya\pgf@y%
        \fi%
        \addtosavedmacro\rotate%
        \ifdim\pgf@xa=0pt\relax%
            \pgfmathsetlength\pgf@xa{\pgfkeysvalueof{/pgf/minimum height}}%
        \fi%
        % Adjust for minimum height and width.
        %
        \pgfmathsetlength\pgf@yb{\pgfkeysvalueof{/pgf/minimum width}}%
        \ifdim\pgf@ya<.5\pgf@yb%
            \ifpgf@lib@sh@isosceslestrianglestretches%
                \pgf@ya.5\pgf@yb%
                %
                \pgfmathdivide@{\pgfmath@tonumber{\pgf@ya}}{\pgfmath@tonumber{\pgf@xa}}%
                \pgfmathatan@{\pgfmathresult}%
                \let\halfapexangle\pgfmathresult%
                %
            \else%
                \pgf@ya.5\pgf@yb%
                \pgf@xa\cothalfapexangle\pgf@ya%
            \fi%
        \fi%
        %
        \pgfmathsetlength\pgf@xb{\pgfkeysvalueof{/pgf/minimum height}}%
        \ifdim\pgf@xa<\pgf@xb%
            \ifpgf@lib@sh@isosceslestrianglestretches%
                \pgf@xa\pgf@xb%
                %
                \pgfmathdivide@{\pgfmath@tonumber{\pgf@ya}}{\pgfmath@tonumber{\pgf@xa}}%
                \pgfmathatan@{\pgfmathresult}%
                \let\halfapexangle\pgfmathresult%
                %
            \else%
                \pgf@xa\pgf@xb%
                \pgf@ya\tanhalfapexangle\pgf@xa%
            \fi%
        \fi%
        %
        %
        % Now find the ``offset height'' (h'). This is the distance from the
        % center of the node contents to the base of the triangle.
        %
        \ifpgfshapeborderusesincircle%
            %
            % h' = H*sin(a/2) / (1+sin(a/2))
            %
            % where:
            %   H = the height of the triangle.
            %   a = the apex angle.
            %
            \pgf@xc\pgf@xa%
            \pgfmathsin@{\halfapexangle}%
            \pgf@xc\pgfmathresult\pgf@xc%
            \pgf@yc1pt\relax%
            \advance\pgf@yc\pgfmathresult pt\relax%
            \pgfmathdivide@{\pgfmath@tonumber{\pgf@xc}}{\pgfmath@tonumber{\pgf@yc}}
            \pgfutil@tempdima\pgfmathresult pt\relax%
        \else%
            %
            % h' = 0.5*h + (0.5*(W-w)*cos(a/2) - h*sin(a/2))/(1+sin(a/2))
            %
            % where:
            %   h = the height of the node contents.
            %   w = the width of the node contents.
            %   W = the width of the triangle.
            %   a = the apex angle.
            %
            \pgf@xc\pgf@ya%
            \advance\pgf@xc-\pgf@y%
            \pgfmathcos@{\halfapexangle}%
            \pgf@xc\pgfmathresult\pgf@xc%
            \pgfmathsin@{\halfapexangle}%
            \advance\pgf@xc-\pgfmathresult\pgf@x%
            \advance\pgf@xc-\pgfmathresult\pgf@x%
            \pgf@yc\pgfmathresult pt\relax%
            \advance\pgf@yc1pt\relax%
            \pgfmathdivide@{\pgfmath@tonumber{\pgf@xc}}{\pgfmath@tonumber{\pgf@yc}}%
            \pgfutil@tempdima\pgfmathresult pt\relax%
            \advance\pgfutil@tempdima\pgf@x%
        \fi%
        \pgfextract@process\apex{%
            \pgf@y0pt%
            \pgf@x\pgf@xa%
            \advance\pgf@x-\pgfutil@tempdima%
        }%
        \addtosavedmacro\apex%
        %
        \pgfextract@process\lowerleft{%
            \pgf@y\pgf@ya%
            \pgf@x-\pgfutil@tempdima%
        }%
        \addtosavedmacro\lowerleft%
        %
        \pgfmathsetlength\pgf@xb{\pgfkeysvalueof{/pgf/outer xsep}}%
        \pgfmathsetlength\pgf@yb{\pgfkeysvalueof{/pgf/outer ysep}}%
        \ifdim\pgf@xb<\pgf@yb%
            \pgf@xb\pgf@yb%
        \fi%
        \pgfextract@process\apexanchor{%
            \apex%
            \advance\pgf@x\cosechalfapexangle\pgf@xb%
        }%
        \addtosavedmacro\apexanchor%
        %
        \pgf@yc-\halfapexangle pt\relax%
        \advance\pgf@yc90pt\relax%
        \divide\pgf@yc2\relax%
        \pgfmathcot@{\pgfmath@tonumber{\pgf@yc}}%
        \pgfextract@process\lowerleftanchor{%
            \lowerleft%
            \advance\pgf@y\pgfmathresult\pgf@xb%
            \advance\pgf@x-\pgf@xb%
        }%
        \addtosavedmacro\lowerleftanchor%
        %
        \pgfextract@process\lowerrightanchor{%
            \lowerleftanchor%
            \pgf@y-\pgf@y%
        }%
        \addtosavedmacro\lowerrightanchor%
    }%
    \anchor{apex}{%
        \trianglepoints%
        \pgfpointadd{\centerpoint}{%
            \pgfmathrotatepointaround{\apexanchor}{\pgfpointorigin}{\rotate}%
        }%
    }%
    \anchor{left corner}{%
        \trianglepoints%
        \pgfpointadd{\centerpoint}{%
            \pgfmathrotatepointaround{\lowerleftanchor}{\pgfpointorigin}{\rotate}%
        }%
    }%
    \anchor{right corner}{%
        \trianglepoints%
        \pgfpointadd{\centerpoint}{%
            \pgfmathrotatepointaround{\lowerrightanchor}{\pgfpointorigin}{\rotate}%
        }%
    }%
    \savedanchor\centerpoint{%
        \pgf@x.5\wd\pgfnodeparttextbox%
        \pgf@y.5\ht\pgfnodeparttextbox%
        \advance\pgf@y-.5\dp\pgfnodeparttextbox%
    }%
    \savedanchor\midpoint{%
        \pgf@x.5\wd\pgfnodeparttextbox%
        \pgfmathsetlength\pgf@y{+.5ex}%
    }%
    \savedanchor\basepoint{%
        \pgf@x.5\wd\pgfnodeparttextbox%
        \pgf@y0pt%
    }%
    \anchor{center}{\centerpoint}%
    \anchor{mid}{\midpoint}%
    \anchor{mid west}{%
        \trianglepoints%
        \let\pgf@isoscelestriangle@referencepoint\midpoint%
        \csname pgf@anchor@isosceles triangle@border\endcsname{\pgfqpoint{-1pt}{0pt}}%
    }%
    \anchor{mid east}{%
        \trianglepoints%
        \let\pgf@isoscelestriangle@referencepoint\midpoint%
        \csname pgf@anchor@isosceles triangle@border\endcsname{\pgfqpoint{1pt}{0pt}}%
    }%
    \anchor{base}{\basepoint}%
    \anchor{base west}{%
        \trianglepoints%
        \let\pgf@isoscelestriangle@referencepoint\basepoint%
        \csname pgf@anchor@isosceles triangle@border\endcsname{\pgfqpoint{-1pt}{0pt}}%
    }%
    \anchor{base east}{%
        \trianglepoints%
        \let\pgf@isoscelestriangle@referencepoint\basepoint%
        \csname pgf@anchor@isosceles triangle@border\endcsname{\pgfqpoint{1pt}{0pt}}%
    }%
    \anchor{north}{%
        \trianglepoints%
        \csname pgf@anchor@isosceles triangle@border\endcsname{\pgfqpoint{0pt}{1pt}}%
    }%
    \anchor{south}{%
        \trianglepoints%
        \csname pgf@anchor@isosceles triangle@border\endcsname{\pgfqpoint{0pt}{-1pt}}%
    }%
    \anchor{east}{%
        \trianglepoints%
        \csname pgf@anchor@isosceles triangle@border\endcsname{\pgfqpoint{1pt}{0pt}}%
    }%
    \anchor{west}{%
        \trianglepoints%
        \csname pgf@anchor@isosceles triangle@border\endcsname{\pgfqpoint{-1pt}{0pt}}%
    }%
    \anchor{north east}{%
        \trianglepoints%
        \csname pgf@anchor@isosceles triangle@border\endcsname{\pgfqpoint{1pt}{1pt}}%
    }%
    \anchor{south west}{%
        \trianglepoints%
        \csname pgf@anchor@isosceles triangle@border\endcsname{\pgfqpoint{-1pt}{-1pt}}%
    }%
    \anchor{south east}{%
        \trianglepoints%
        \csname pgf@anchor@isosceles triangle@border\endcsname{\pgfqpoint{1pt}{-1pt}}%
    }%
    \anchor{north west}{%
        \trianglepoints%
        \csname pgf@anchor@isosceles triangle@border\endcsname{\pgfqpoint{-1pt}{1pt}}%
    }%
    \anchor{left side}{%
        \trianglepoints%
        \pgfmathrotatepointaround{%
            \pgfpointadd{\pgfpointlineattime{0.5}{\lowerleftanchor}{\apexanchor}}{\centerpoint}}%
                {\centerpoint}{\rotate}%
    }%
    \anchor{right side}{%
        \trianglepoints%
        \pgfmathrotatepointaround{%
            \pgfpointadd{\pgfpointlineattime{0.5}{\lowerrightanchor}{\apexanchor}}{\centerpoint}}%
                {\centerpoint}{\rotate}%
    }%
    \anchor{lower side}{%
        \trianglepoints%
        \pgfmathrotatepointaround{%
            \pgfpointadd{\pgfpointlineattime{0.5}{\lowerleftanchor}{\lowerrightanchor}}{\centerpoint}}%
                {\centerpoint}{\rotate}%
    }%
    \backgroundpath{%
        \trianglepoints%
        {%
            \pgftransformshift{\centerpoint}%
            \pgftransformrotate{\rotate}%
            \pgfpathmoveto{\apex}%
            \pgfpathlineto{\lowerleft}%
            \pgfpathlineto{\lowerleft\pgf@y-\pgf@y}%
            \pgfpathclose%
        }%
    }%
    \anchorborder{%
        \pgfutil@ifundefined{pgf@isoscelestriangle@referencepoint}{\let\referencepoint\centerpoint}%
            {\let\referencepoint\pgf@isoscelestriangle@referencepoint}%
        \pgfextract@process\externalpoint{%
            \pgf@xa\pgf@x%
            \pgf@ya\pgf@y%
            \referencepoint%
            \advance\pgf@x\pgf@xa%
            \advance\pgf@y\pgf@ya%
        }%
        \trianglepoints%
        \pgfmathanglebetweenpoints{\referencepoint}{\externalpoint}%
        \pgfmathsubtract@{\pgfmathresult}{\rotate}%
            \pgfmathmod@{\pgfmathresult}{360}%
        \ifdim\pgfmathresult pt<0pt\relax%
            \pgfmathadd@{\pgfmathresult}{360}%
        \fi%
        \let\externalangle\pgfmathresult%
        %
        \pgfmathanglebetweenpoints{\referencepoint}{\pgfpointadd{\centerpoint}{\lowerrightanchor}}%
        \ifdim\externalangle pt<\pgfmathresult pt\relax%
            \pgfmathanglebetweenpoints{\referencepoint}{\pgfpointadd{\centerpoint}{\lowerleftanchor}}%
            \ifdim\externalangle pt<\pgfmathresult pt\relax%
                \pgfmathanglebetweenpoints{\referencepoint}{\pgfpointadd{\centerpoint}{\apexanchor}}%
                \ifdim\externalangle pt>\pgfmathresult pt\relax%
                    \let\firstpoint\apexanchor%
                    \let\secondpoint\lowerleftanchor%
                \else%
                    \let\firstpoint\apexanchor%
                    \let\secondpoint\lowerrightanchor%
                \fi%
            \else%
                \let\firstpoint\lowerleftanchor%
                \let\secondpoint\lowerrightanchor%
            \fi%
        \else%
            \let\firstpoint\lowerrightanchor%
            \let\secondpoint\apexanchor%
        \fi%
        \pgfpointintersectionoflines{\referencepoint}{\externalpoint}%
            {\pgfmathrotatepointaround{\pgfpointadd{\centerpoint}{\firstpoint}}{\centerpoint}{\rotate}}%
            {\pgfmathrotatepointaround{\pgfpointadd{\centerpoint}{\secondpoint}}{\centerpoint}{\rotate}}%
    }%
}%




% Keys for shape kite
%
%
\pgfkeys{/pgf/.cd,
    kite upper vertex angle/.initial=120,
    kite lower vertex angle/.initial=60,
    kite vertex angles/.code={%
        \pgfutil@in@{and}{#1}%
        \ifpgfutil@in@%
            \def\pgf@marshal##1and##2\@@{%
                \pgfkeys{/pgf/kite upper vertex angle=##1}%
                \pgfkeys{/pgf/kite lower vertex angle=##2}%
            }%
            \expandafter\pgf@marshal#1\@@%
        \else%
            \pgfkeys{/pgf/kite upper vertex angle=#1}%
            \pgfkeys{/pgf/kite lower vertex angle=#1}%
        \fi%
    }%
}%

% Shape kite.
%
\pgfdeclareshape{kite}{%
    \savedmacro\installkiteparameters{%
        %
        % Get the larger of the outer sep.
        %
        \pgfmathsetlength\pgf@x{\pgfkeysvalueof{/pgf/outer xsep}}%
        \pgfmathsetlength\pgf@y{\pgfkeysvalueof{/pgf/outer ysep}}%
        \ifdim\pgf@x<\pgf@y%
            \pgf@x\pgf@y%
        \fi%
        \edef\outersep{\the\pgf@x}%
        %
        % Calculate the centre, base and mid points of the node.
        %
        \pgfextract@process\centerpoint{%
            \pgf@x.5\wd\pgfnodeparttextbox%
            \pgf@y.5\ht\pgfnodeparttextbox%
            \advance\pgf@y-.5\dp\pgfnodeparttextbox%
        }%
        \pgfextract@process\basepoint{%
            \pgf@x.5\wd\pgfnodeparttextbox%
            \pgf@y0pt\relax%
        }%
        \pgfextract@process\midpoint{%
            \pgf@x.5\wd\pgfnodeparttextbox%
            \pgfmathsetlength\pgf@y{+.5ex}%
        }%
        %
        % Halve the vertex angles.
        %
        \pgfmathdivide{\pgfkeysvalueof{/pgf/kite upper vertex angle}}{2}%
        \let\halfuppervertexangle\pgfmathresult%
        \pgfmathdivide{\pgfkeysvalueof{/pgf/kite lower vertex angle}}{2}%
        \let\halflowervertexangle\pgfmathresult%
        %
        % Get the node dimensions.
        %
        \pgfmathsetlength\pgf@x{\pgfkeysvalueof{/pgf/inner xsep}}%
        \advance\pgf@x.5\wd\pgfnodeparttextbox%
        \pgfmathsetlength\pgf@y{\pgfkeysvalueof{/pgf/inner ysep}}%
        \advance\pgf@y.5\ht\pgfnodeparttextbox%
        \advance\pgf@y.5\dp\pgfnodeparttextbox%
        \ifpgfshapeborderusesincircle%
            %
            % Get the rotation.
            %
            \pgfmathsetmacro\rotate{\pgfkeysvalueof{/pgf/shape border rotate}}%
            %
            % Get the radius of the incircle.
            %
            \ifdim\pgf@x<\pgf@y%
                \pgf@x\pgf@y%
            \fi%
            \pgf@x1.41421\pgf@x%
            %
            % Get the distances from the upper and lower vertices
            % to the center of the incircle.
            %
            \pgfmathsin@{\halfuppervertexangle}%
            \let\sinehalfuppervertexangle\pgfmathresult%
            \pgfmathreciprocal@{\pgfmathresult}%
            \let\cosechalfuppervertexangle\pgfmathresult%
            \pgf@ya\pgfmathresult\pgf@x%
            \pgfmathsin@{\halflowervertexangle}%
            \let\sinehalflowervertexangle\pgfmathresult%
            \pgfmathreciprocal@{\pgfmathresult}%
            \pgf@yb\pgfmathresult\pgf@x%
            %
            % Now calculate vertical offset (yc) between the center of
            % the incircle, and the intersection of the kite diagonals.
            %
            % yc = x * cosec(a/2) - x * cot(a/2) * (sin(a/2) + sin(b/2)) * cosec(a/2 + b/2)
            %
            % where:
            %   x = the incircle radius.
            %   a = the upper vertex angle.
            %   b = the lower vertex angle.
            %
            \pgf@yc\pgf@x%
            \pgfmathadd@{\halfuppervertexangle}{\halflowervertexangle}%
            \pgfmathcosec@{\pgfmathresult}%
            \pgf@yc\pgfmathresult\pgf@yc%
            \pgfmathadd@{\sinehalfuppervertexangle}{\sinehalflowervertexangle}%
        \pgf@yc\pgfmathresult\pgf@yc%
            \pgfmathcos@{\halfuppervertexangle}%
            \pgf@yc\pgfmathresult\pgf@yc%
            \pgf@yc-\cosechalfuppervertexangle\pgf@yc%
            \advance\pgf@yc\cosechalfuppervertexangle\pgf@x%
            \edef\deltay{\the\pgf@yc}%
            %
            % Now calculate the height of the kite...
            %
            \advance\pgf@ya-\pgf@yc%
            %
            % ...and the depth.
            %
            \advance\pgf@yb\pgf@yc%
            %
            % Get the half width of the widest part of the kite.
            %
            \pgfmathtan@{\halfuppervertexangle}%
            \pgf@xa\pgfmathresult\pgf@ya%
        \else%
            \multiply\pgf@x2\relax%
            \multiply\pgf@y2\relax%
            %
            % Get the rotation (with rounding).
            %
            \pgfmathmod{\pgfkeysvalueof{/pgf/shape border rotate}}{360}%
            \afterassignment\pgfmath@gobbletilpgfmath@%
            \expandafter\c@pgf@counta\pgfmathresult\relax\pgfmath@%
            \advance\c@pgf@counta45\relax%
            \divide\c@pgf@counta90\relax%
            \multiply\c@pgf@counta90\relax%
            \ifnum\c@pgf@counta<0\relax%
                \advance\c@pgf@counta360\relax%
            \fi%
            \edef\rotate{\the\c@pgf@counta}%
            %
            % Calculate the width and height of the node
            % contents, according to any border rotation.
            %
            \ifnum\c@pgf@counta=90\relax%
                \pgf@xc\pgf@x%
                \pgf@x\pgf@y%
                \pgf@y\pgf@xc%
            \else%
                \ifnum\c@pgf@counta=270\relax%
                    \pgf@xc\pgf@x%
                    \pgf@x\pgf@y%
                    \pgf@y\pgf@xc%
                \fi%
            \fi%
            %
            % The node contents (total height y) extends a distance ya into the
            % upper isosceles triangle of the kite and a distance yb into the
            % lower isosceles triangle. Thus, the following relationships hold:
            %
            %   ya/yb = tan(b/2)/tan(a/2)
            %
            % and   y = ya + yb
            %
            % so   ya = y * sin(a/2 + b/2) / (cos(a/2) * sin(b/2))
            %
            % where:
            %   a = upper vertex angle.
            %   b = lower vertex angle.
            %
            \pgfmathadd@{\halfuppervertexangle}{\halflowervertexangle}%
            \pgfmathsin@{\pgfmathresult}%
            \pgfmathreciprocal@{\pgfmathresult}%
            \pgf@ya\pgfmathresult\pgf@y%
            \pgfmathcos@{\halfuppervertexangle}%
            \pgf@ya\pgfmathresult\pgf@ya%
            \pgfmathsin@{\halflowervertexangle}%
            \pgf@ya\pgfmathresult\pgf@ya%
            \pgf@yb\pgf@y%
            \advance\pgf@yb-\pgf@ya%
            %
            % The vertical offset between the center of the node, and
            % the intersection of the kite diagonals is given by:
            %
            % yc = y/2 - ya
            %
            \pgf@yc.5\pgf@y%
            \advance\pgf@yc-\pgf@ya%
            \edef\deltay{\the\pgf@yc}%
            %
            % Get the half width of the widest part of the kite.
            %
            \pgfmathtan@{\halfuppervertexangle}%
            \pgf@xa.5\pgf@x%
            \advance\pgf@xa\pgfmathresult\pgf@ya%
            %
            % Now calculate the height of the kite...
            %
            \pgf@xb.5\pgf@x%
            \pgfmathreciprocal@{\pgfmathresult}%
            \advance\pgf@ya\pgfmathresult\pgf@xb%
            %
            % ...and the depth.
            %
            \pgfmathtan@{\halflowervertexangle}%
            \pgfmathreciprocal@{\pgfmathresult}%
            \advance\pgf@yb\pgfmathresult\pgf@xb%
        \fi%
        %
        % Take into account minimum height and width.
        %
        % ya is the kite height.
        % yb is the kite depth.
        % xa is the kite (half) width.
        %
        \pgfmathsetlength\pgf@yc{\pgfkeysvalueof{/pgf/minimum height}}%
        \pgf@y\pgf@ya%
        \advance\pgf@y\pgf@yb%
        \ifdim\pgf@y<\pgf@yc%
            \pgfmathreciprocal@{\pgfmath@tonumber{\pgf@y}}%
            \pgf@yc\pgfmathresult\pgf@yc%
            \pgf@xa\pgfmath@tonumber{\pgf@yc}\pgf@xa%
            \pgf@ya\pgfmath@tonumber{\pgf@yc}\pgf@ya%
            \pgf@yb\pgfmath@tonumber{\pgf@yc}\pgf@yb%
        \fi%
        \pgf@x2.0\pgf@xa%
        \pgfmathsetlength\pgf@xc{\pgfkeysvalueof{/pgf/minimum width}}%
        \ifdim\pgf@x<\pgf@xc%
            \pgfmathreciprocal@{\pgfmath@tonumber{\pgf@x}}%
            \pgf@xc\pgfmathresult\pgf@xc%
            \pgf@xa\pgfmath@tonumber{\pgf@xc}\pgf@xa%
            \pgf@ya\pgfmath@tonumber{\pgf@xc}\pgf@ya%
            \pgf@yb\pgfmath@tonumber{\pgf@xc}\pgf@yb%
        \fi%
        \edef\kitehalfwidth{\the\pgf@xa}%
        \edef\kiteheight{\the\pgf@ya}%
        \edef\kitedepth{\the\pgf@yb}%
        %
        % Calculate the basic points on the kite (for the background path).
        %
        \pgfextract@process\toppoint{%
            \centerpoint%
            \advance\pgf@y\deltay\relax%
            \advance\pgf@y\kiteheight%
        }%
        \pgfextract@process\bottompoint{%
            \centerpoint%
            \advance\pgf@y\deltay\relax%
            \advance\pgf@y-\kitedepth%
        }%
        \pgfextract@process\leftpoint{%
            \centerpoint%
            \advance\pgf@y\deltay\relax%
            \advance\pgf@x-\kitehalfwidth%
        }%
        \pgfextract@process\rightpoint{%
            \centerpoint%
            \advance\pgf@y\deltay\relax%
                \advance\pgf@x\kitehalfwidth%
        }%
        %
        % Now calculate the miter length. At the top...
        %
        \pgfmathsin@{\halfuppervertexangle}%
        \pgfmathreciprocal@{\pgfmathresult}%
        \pgf@y\outersep\relax%
        \pgf@y\pgfmathresult\pgf@y%
        \edef\topmiter{\the\pgf@y}%
        %
        % ...at the bottom...
        %
        \pgfmathsin@{\halflowervertexangle}%
        \pgfmathreciprocal@{\pgfmathresult}%
        \pgf@y\outersep\relax%
        \pgf@y\pgfmathresult\pgf@y%
        \edef\bottommiter{\the\pgf@y}%
        %
        % ...to the right...
        %
        \pgfmathsubtract@{180}{\halflowervertexangle}%
        \pgfmathsubtract@{\pgfmathresult}{\halfuppervertexangle}%
        \pgfmathdivide@{\pgfmathresult}{2}%
        \pgfmathsin@{\pgfmathresult}%
        \pgfmathreciprocal@{\pgfmathresult}%
        \pgf@xa\outersep\relax%
        \pgf@xa\pgfmathresult\pgf@xa%
        \pgfmathsubtract@{\halfuppervertexangle}{\halflowervertexangle}%
        \pgfmathdivide@{\pgfmathresult}{2}%
        \let\angle\pgfmathresult%
        \pgfextract@process\rightmiter{%
            \pgfqpointpolar{\angle}{\the\pgf@xa}%
        }%
        %
        % ...and to the left.
        %
        \pgfmathsubtract@{180}{\angle}%
        \let\angle\pgfmathresult%
        \pgfextract@process\leftmiter{%
            \pgfqpointpolar{\angle}{\the\pgf@xa}%
        }%
        %
        % Create the border points.
        %
        \pgfextract@process\topborderpoint{%
            \toppoint%
            \advance\pgf@y\topmiter\relax%
        }%
        \pgfextract@process\bottomborderpoint{%
            \bottompoint%
            \advance\pgf@y-\bottommiter\relax%
        }%
        \pgfextract@process\leftborderpoint{%
            \leftpoint%
            \pgf@xa\pgf@x%
            \pgf@ya\pgf@y%
            \leftmiter%
            \advance\pgf@x\pgf@xa%
            \advance\pgf@y\pgf@ya%
        }%
        \pgfextract@process\rightborderpoint{%
            \rightpoint%
            \pgf@xa\pgf@x%
            \pgf@ya\pgf@y%
            \rightmiter%
            \advance\pgf@x\pgf@xa%
            \advance\pgf@y\pgf@ya%
        }%
        %
        % Get the angle from the \centerpoint to the *unrotated points*.
        %
        \pgfmathanglebetweenpoints{\centerpoint}{\topborderpoint}%
        \let\angletotoppoint\pgfmathresult%
        \pgfmathanglebetweenpoints{\centerpoint}{\leftborderpoint}%
        \let\angletoleftpoint\pgfmathresult%
        \pgfmathanglebetweenpoints{\centerpoint}{\bottomborderpoint}%
        \let\angletobottompoint\pgfmathresult%
        \pgfmathanglebetweenpoints{\centerpoint}{\rightborderpoint}%
        \let\angletorightpoint\pgfmathresult%
        %
        % ...from the *inversely rotated* \basepoint...
        %
        \pgfextract@process\rotatedbasepoint{%
            \pgfmathrotatepointaround{\basepoint}{\centerpoint}{-\rotate}%
        }%
        \pgfmathanglebetweenpoints{\rotatedbasepoint}{\topborderpoint}%
        \let\baseangletotoppoint\pgfmathresult%
        \pgfmathanglebetweenpoints{\rotatedbasepoint}{\leftborderpoint}%
        \let\baseangletoleftpoint\pgfmathresult%
        \pgfmathanglebetweenpoints{\rotatedbasepoint}{\bottomborderpoint}%
        \let\baseangletobottompoint\pgfmathresult%
        \pgfmathanglebetweenpoints{\rotatedbasepoint}{\rightborderpoint}%
        \let\baseangletorightpoint\pgfmathresult%
        %
        % ...and from the *inversely rotated* \midpoint.
        %
        \pgfextract@process\rotatedmidpoint{%
            \pgfmathrotatepointaround{\midpoint}{\centerpoint}{-\rotate}%
        }%
        \pgfmathanglebetweenpoints{\rotatedmidpoint}{\topborderpoint}%
        \let\midangletotoppoint\pgfmathresult%
        \pgfmathanglebetweenpoints{\rotatedmidpoint}{\leftborderpoint}%
        \let\midangletoleftpoint\pgfmathresult%
        \pgfmathanglebetweenpoints{\rotatedmidpoint}{\bottomborderpoint}%
        \let\midangletobottompoint\pgfmathresult%
        \pgfmathanglebetweenpoints{\rotatedmidpoint}{\rightborderpoint}%
        \let\midangletorightpoint\pgfmathresult%
        %
        % Calculate a radius guaranteed to be outside the kite.
        %
        \pgfextractx\pgf@xa{\pgfpointdiff{\leftborderpoint}{\rightborderpoint}}%
        \ifdim\pgf@xa<0pt\relax%
            \pgf@xa-\pgf@xa%
        \fi%
        \pgfextracty\pgf@ya{\pgfpointdiff{\topborderpoint}{\bottomborderpoint}}%
        \ifdim\pgf@ya<0pt\relax%
            \pgf@ya-\pgf@ya%
        \fi%
        \ifdim\pgf@ya>\pgf@xa%
            \edef\externalradius{\the\pgf@ya}%
        \else%
            \edef\externalradius{\the\pgf@xa}%
        \fi%
        %
        % Now rotate the points...
        %
        \pgfextract@process\toppoint{%
            \pgfmathrotatepointaround{\toppoint}{\centerpoint}{\rotate}%
        }%
        \pgfextract@process\bottompoint{%
            \pgfmathrotatepointaround{\bottompoint}{\centerpoint}{\rotate}%
        }%
        \pgfextract@process\leftpoint{%
            \pgfmathrotatepointaround{\leftpoint}{\centerpoint}{\rotate}%
        }%
        \pgfextract@process\rightpoint{%
            \pgfmathrotatepointaround{\rightpoint}{\centerpoint}{\rotate}%
        }%
        %
        % ...and the border points.
        %
        \pgfextract@process\topborderpoint{%
            \pgfmathrotatepointaround{\topborderpoint}{\centerpoint}{\rotate}%
        }%
        \pgfextract@process\bottomborderpoint{%
            \pgfmathrotatepointaround{\bottomborderpoint}{\centerpoint}{\rotate}%
        }%
        \pgfextract@process\leftborderpoint{%
            \pgfmathrotatepointaround{\leftborderpoint}{\centerpoint}{\rotate}%
        }%
        \pgfextract@process\rightborderpoint{%
            \pgfmathrotatepointaround{\rightborderpoint}{\centerpoint}{\rotate}%
        }%
        %
        % Now save it all.
        %
        \addtosavedmacro{\rotate}%
        \addtosavedmacro{\externalradius}%
        %
        \addtosavedmacro\toppoint%
        \addtosavedmacro\bottompoint%
        \addtosavedmacro\leftpoint%
        \addtosavedmacro\rightpoint%
        %
        \addtosavedmacro\topborderpoint%
        \addtosavedmacro\bottomborderpoint%
        \addtosavedmacro\leftborderpoint%
        \addtosavedmacro\rightborderpoint%
        %
        \addtosavedmacro\angletotoppoint%
        \addtosavedmacro\angletobottompoint%
        \addtosavedmacro\angletoleftpoint%
        \addtosavedmacro\angletorightpoint%
        %
        \addtosavedmacro\baseangletotoppoint%
        \addtosavedmacro\baseangletobottompoint%
        \addtosavedmacro\baseangletoleftpoint%
        \addtosavedmacro\baseangletorightpoint%
        %
        \addtosavedmacro\midangletotoppoint%
        \addtosavedmacro\midangletobottompoint%
        \addtosavedmacro\midangletoleftpoint%
        \addtosavedmacro\midangletorightpoint%
    }%
    \savedanchor\centerpoint{%
        \pgf@x.5\wd\pgfnodeparttextbox%
        \pgf@y.5\ht\pgfnodeparttextbox%
        \advance\pgf@y-.5\dp\pgfnodeparttextbox%
    }%
    \savedanchor\basepoint{%
        \pgf@x.5\wd\pgfnodeparttextbox%
        \pgf@y0pt\relax%
    }%
    \savedanchor\midpoint{%
        \pgf@x.5\wd\pgfnodeparttextbox%
        \pgfmathsetlength\pgf@y{+.5ex}%
    }%
    \anchor{center}{\centerpoint}%
    \anchor{base}{\basepoint}%
    \anchor{base west}{%
        \installkiteparameters%
        \let\pgf@kiteanchorborderreferencepoint\basepoint%
        \csname pgf@anchor@kite@border\endcsname{\pgfqpoint{-\externalradius}{0pt}}%
    }%
    \anchor{base east}{%
        \installkiteparameters%
        \let\pgf@kiteanchorborderreferencepoint\basepoint%
        \csname pgf@anchor@kite@border\endcsname{\pgfqpoint{\externalradius}{0pt}}%
    }%
    \anchor{mid}{\midpoint}%
    \anchor{mid west}{%
        \installkiteparameters%
        \let\pgf@kiteanchorborderreferencepoint\midpoint%
        \csname pgf@anchor@kite@border\endcsname{\pgfqpoint{-\externalradius}{0pt}}%
    }%
    \anchor{mid east}{%
        \installkiteparameters%
        \let\pgf@kiteanchorborderreferencepoint\midpoint%
        \csname pgf@anchor@kite@border\endcsname{\pgfqpoint{\externalradius}{0pt}}%
    }%
    \anchor{north}{%
        \installkiteparameters%
        \csname pgf@anchor@kite@border\endcsname{\pgfqpoint{0pt}{\externalradius}}%
    }%
    \anchor{south}{%
        \installkiteparameters%
        \csname pgf@anchor@kite@border\endcsname{\pgfqpoint{0pt}{-\externalradius}}%
    }%
    \anchor{east}{%
        \installkiteparameters%
        \csname pgf@anchor@kite@border\endcsname{\pgfqpoint{\externalradius}{0pt}}%
    }%
    \anchor{west}{%
        \installkiteparameters%
        \csname pgf@anchor@kite@border\endcsname{\pgfqpoint{-\externalradius}{0pt}}%
    }%
    \anchor{north east}{%
        \installkiteparameters%
        \csname pgf@anchor@kite@border\endcsname{\pgfqpoint{\externalradius}{\externalradius}}%
    }%
    \anchor{south west}{%
        \installkiteparameters%
        \csname pgf@anchor@kite@border\endcsname{\pgfqpoint{-\externalradius}{-\externalradius}}%
    }%
    \anchor{south east}{%
        \installkiteparameters%
        \csname pgf@anchor@kite@border\endcsname{\pgfqpoint{\externalradius}{-\externalradius}}%
    }%
    \anchor{north west}{%
        \installkiteparameters%
        \csname pgf@anchor@kite@border\endcsname{\pgfqpoint{-\externalradius}{\externalradius}}%
    }%
    \anchor{upper vertex}{%
        \installkiteparameters%
        \topborderpoint}%
    \anchor{lower vertex}{%
        \installkiteparameters%
        \bottomborderpoint}%
    \anchor{left vertex}{%
        \installkiteparameters%
        \leftborderpoint}%
    \anchor{right vertex}{%
        \installkiteparameters%
        \rightborderpoint}%
    \anchor{upper left side}{%
        \installkiteparameters%
        \pgfpointlineattime{0.5}{\topborderpoint}{\leftborderpoint}}%
    \anchor{lower left side}{%
        \installkiteparameters%
        \pgfpointlineattime{0.5}{\bottomborderpoint}{\leftborderpoint}}%
    \anchor{upper right side}{%
        \installkiteparameters%
        \pgfpointlineattime{0.5}{\topborderpoint}{\rightborderpoint}}%
    \anchor{lower right side}{%
        \installkiteparameters%
        \pgfpointlineattime{0.5}{\bottomborderpoint}{\rightborderpoint}}%
    \backgroundpath{%
        \installkiteparameters%
        \pgfpathmoveto{\toppoint}%
        \pgfpathlineto{\leftpoint}%
        \pgfpathlineto{\bottompoint}%
        \pgfpathlineto{\rightpoint}%
        \pgfpathclose%
    }%
    \anchorborder{%
        %
        % Save x and y.
        %
        \edef\externalx{\the\pgf@x}%
        \edef\externaly{\the\pgf@y}%
        %
        % This allows anchors base east, base west, mid east and mid west,
        % to redefine the `center' of the node to correctly calculate the
        % border points.
        %
        \pgfutil@ifundefined{pgf@kiteanchorborderreferencepoint}%
            {\let\referencepoint\centerpoint}%
            {\let\referencepoint\pgf@kiteanchorborderreferencepoint}%
        %
        % Adjust the location of the external
        % point relative to the reference point.
        %
        \referencepoint%
        \pgf@xa\externalx\relax%
        \pgf@ya\externaly\relax%
        \advance\pgf@xa\pgf@x%
        \advance\pgf@ya\pgf@y%
        \edef\externalx{\the\pgf@xa}%
        \edef\externaly{\the\pgf@ya}%
        %
        % Install the required points and angles.
        %
        \installkiteparameters%
        %
        % Get the angle of the external point relative to \referencepoint.
        %
        \pgfmathanglebetweenpoints{\referencepoint}{\pgfqpoint{\externalx}{\externaly}}%
        %
        % *Subtract* the rotation from the external angle.
        %
        \pgfmathsubtract@{\pgfmathresult}{\rotate}%
        \ifdim\pgfmathresult pt<0pt\relax%
            \pgfmathadd@{\pgfmathresult}{360}%
        \fi%
        \let\externalangle\pgfmathresult%
        \ifx\referencepoint\basepoint%
            \let\angletotoppoint\baseangletotoppoint%
            \let\angletobottompoint\baseangletobottompoint%
            \let\angletoleftpoint\baseangletoleftpoint%
            \let\angletorightpoint\baseangletorightpoint%
        \else%
            \ifx\referencepoint\midpoint%
                \let\angletotoppoint\midangletotoppoint%
                \let\angletobottompoint\midangletobottompoint%
                \let\angletoleftpoint\midangletoleftpoint%
                \let\angletorightpoint\midangletorightpoint%
        \fi\fi%
        %
        %  Depending on the rotation, the angle to \rightborderpoint
        %  may be smaller than the angle to \topborderpoint.
        %
        \ifdim\angletorightpoint pt<\angletotoppoint pt\relax%
            \ifdim\externalangle pt<\angletorightpoint pt\relax%
                \let\firstpoint\rightborderpoint%
                \let\secondpoint\bottomborderpoint%
            \else%
                \ifdim\externalangle pt<\angletotoppoint pt\relax%
                    \let\firstpoint\rightborderpoint%
                    \let\secondpoint\topborderpoint%
                \else%
                    \ifdim\externalangle pt<\angletoleftpoint pt\relax%
                        \let\firstpoint\topborderpoint%
                        \let\secondpoint\leftborderpoint%
                    \else%
                        \ifdim\externalangle pt<\angletobottompoint pt\relax%
                            \let\firstpoint\leftborderpoint%
                            \let\secondpoint\bottomborderpoint%
                        \else%
                            \let\firstpoint\rightborderpoint%
                            \let\secondpoint\bottomborderpoint%
                        \fi%
                    \fi%
                \fi%
            \fi%
        \else%
            \ifdim\externalangle pt<\angletotoppoint pt\relax%
                \let\firstpoint\rightborderpoint%
                \let\secondpoint\topborderpoint%
            \else%
                \ifdim\externalangle pt<\angletoleftpoint pt\relax%
                    \let\firstpoint\leftborderpoint%
                    \let\secondpoint\topborderpoint%
                \else%
                    \ifdim\externalangle pt<\angletobottompoint pt\relax%
                        \let\firstpoint\bottomborderpoint%
                        \let\secondpoint\leftborderpoint%
                    \else%
                        \ifdim\externalangle pt<\angletorightpoint pt\relax%
                            \let\firstpoint\rightborderpoint%
                            \let\secondpoint\bottomborderpoint%
                        \else%
                            \let\firstpoint\rightborderpoint%
                            \let\secondpoint\topborderpoint%
                        \fi%
                    \fi%
                \fi%
            \fi%
        \fi%
        \pgfpointintersectionoflines{\referencepoint}{\pgfqpoint{\externalx}{\externaly}}%
                {\firstpoint}{\secondpoint}%
    }%
}%




% Keys for shape dart.
%
%

\pgfkeys{/pgf/.cd,
    dart tip angle/.initial=45,
    dart tail angle/.initial=135,
}%

% Shape dart.
%
\pgfdeclareshape{dart}{%
    \savedmacro\installdartparameters{%
        %
        % Get the halved angles (more useful).
        %
        \pgfmathdivide{\pgfkeysvalueof{/pgf/dart tip angle}}{2}%
        \let\halftipangle\pgfmathresult%
        \pgfmathdivide{\pgfkeysvalueof{/pgf/dart tail angle}}{2}%
        \let\halftailangle\pgfmathresult%
        %
        % Calculate some common results.
        %
        \pgfmathcot@{\halftipangle}%
        \let\cothalftipangle\pgfmathresult%
        %
        % Get the larger of the outer sep.
        %
        \pgfmathsetlength\pgf@x{\pgfkeysvalueof{/pgf/outer xsep}}%
        \pgfmathsetlength\pgf@y{\pgfkeysvalueof{/pgf/outer ysep}}%
        \ifdim\pgf@x<\pgf@y%
            \pgf@x\pgf@y%
        \fi%
        \edef\outersep{\the\pgf@x}%
        %
        % Calculate the centre, base and mid points of the node.
        %
        \pgfextract@process\centerpoint{%
            \pgf@x.5\wd\pgfnodeparttextbox%
            \pgf@y.5\ht\pgfnodeparttextbox%
            \advance\pgf@y-.5\dp\pgfnodeparttextbox%
        }%
        \pgfextract@process\basepoint{%
            \pgf@x.5\wd\pgfnodeparttextbox%
            \pgf@y0pt\relax%
        }%
        \pgfextract@process\midpoint{%
            \pgf@x.5\wd\pgfnodeparttextbox%
            \pgfmathsetlength\pgf@y{+.5ex}%
        }%
        %
        % Get the (halved) dimension of the node.
        %
        \pgfmathsetlength\pgf@x{\pgfkeysvalueof{/pgf/inner xsep}}%
        \advance\pgf@x.5\wd\pgfnodeparttextbox%
        \pgfmathsetlength\pgf@y{\pgfkeysvalueof{/pgf/inner ysep}}%
        \advance\pgf@y.5\ht\pgfnodeparttextbox%
        \advance\pgf@y.5\dp\pgfnodeparttextbox%
        %
        % Get the length of the dart tip.
        %
        \ifpgfshapeborderusesincircle%
            %
            % Get the (unrounded) rotation.
            %
            \pgfmathsetmacro\rotate{\pgfkeysvalueof{/pgf/shape border rotate}}%
            %
            % Get the radius of the incircle.
            %
            \ifdim\pgf@x<\pgf@y%
                \pgf@x\pgf@y%
            \fi%
            \pgf@x1.41421\pgf@x%
            %
            % Calculate the length of the dart tip.
            %
            \pgf@xa\cothalftipangle\pgf@x%
            \advance\pgf@xa\pgf@x%
        \else%
            %
            % Get the rotation (with rounding).
            %
            \pgfmathmod{\pgfkeysvalueof{/pgf/shape border rotate}}{360}%
            \afterassignment\pgfmath@gobbletilpgfmath@%
            \expandafter\c@pgf@counta\pgfmathresult\relax\pgfmath@%
            \advance\c@pgf@counta45\relax%
            \divide\c@pgf@counta90\relax%
            \multiply\c@pgf@counta90\relax%
            \ifnum\c@pgf@counta<0\relax%
                \advance\c@pgf@counta360\relax%
            \fi%
            \edef\rotate{\the\c@pgf@counta}%
            %
            % Flip the width and height of the node contents,
            % according to the appropriate border rotation.
            %
            \ifnum\c@pgf@counta=90\relax%
                \pgf@xc\pgf@x%
                \pgf@x\pgf@y%
                \pgf@y\pgf@xc%
            \else%
                \ifnum\c@pgf@counta=270\relax%
                    \pgf@xc\pgf@x%
                    \pgf@x\pgf@y%
                    \pgf@y\pgf@xc%
                \fi%
            \fi%
            %
            % Calculate the length of the dart tip.
            %
            \pgf@xa\cothalftipangle\pgf@y%
            \advance\pgf@xa2.0\pgf@x%
        \fi%
        %
        % Get the (half) separation of the dart tails.
        %
        \pgfmathsubtract@{\halftailangle}{\halftipangle}%
        \pgfmathcosec@{\pgfmathresult}%
        \pgf@ya\pgfmathresult\pgf@xa%
        \pgfmathsin@{\halftipangle}%
        \pgf@ya\pgfmathresult\pgf@ya%
        \pgfmathcos@{\halftipangle}%
        \pgf@ya\pgfmathresult\pgf@ya%
        %
        % Get the total length of the dart...
        %
        \pgf@xb\cothalftipangle\pgf@ya%
        %
        % and hence the length of the tails.
        %
        \pgf@xc-\pgf@xa%
        \advance\pgf@xc\pgf@xb%
        %
        % Adjust for minimum height (length of the dart).
        %
        \pgfmathsetlength\pgf@y{\pgfkeysvalueof{/pgf/minimum height}}%
        \ifdim\pgf@xb<\pgf@y%
            \pgfmathreciprocal@{\pgfmath@tonumber{\pgf@xb}}%
            \pgf@y\pgfmathresult\pgf@y%
            \pgf@xa\pgfmath@tonumber{\pgf@y}\pgf@xa%
            \pgf@xc\pgfmath@tonumber{\pgf@y}\pgf@xc%
            \pgf@ya\pgfmath@tonumber{\pgf@y}\pgf@ya%
            \pgf@x\pgfmath@tonumber{\pgf@y}\pgf@x%
        \fi%
        %
        % Adjust for minimum width (tail separation length).
        %
        \pgfmathsetlength\pgf@y{\pgfkeysvalueof{/pgf/minimum width}}%
        \pgf@y.5\pgf@y%
        \ifdim\pgf@ya<\pgf@y%
            \pgfmathreciprocal@{\pgfmath@tonumber{\pgf@ya}}%
            \pgf@ya\pgf@y%
            \pgf@y\pgfmathresult\pgf@y%
            \pgf@xa\pgfmath@tonumber{\pgf@y}\pgf@xa%
            \pgf@xc\pgfmath@tonumber{\pgf@y}\pgf@xc%
            \pgf@x\pgfmath@tonumber{\pgf@y}\pgf@x%
        \fi%
        \edef\dartlength{\the\pgf@xa}%
        \edef\deltax{\the\pgf@x}%
        \edef\taillength{\the\pgf@xc}%
        \edef\halftailseparation{\the\pgf@ya}
        %
        % Create the basic points on the dart (for the background path).
        %
        \pgfextract@process\tippoint{%
            \centerpoint%
            \advance\pgf@x\dartlength\relax%
            \advance\pgf@x-\deltax\relax%
        }%
        \pgfextract@process\tailcenterpoint{%
            \centerpoint%
            \advance\pgf@x-\deltax\relax%
        }%
        \pgfextract@process\lefttailpoint{%
            \centerpoint%
            \advance\pgf@x-\deltax\relax%
            \advance\pgf@x-\taillength\relax%
            \advance\pgf@y\halftailseparation\relax%
        }%
        \pgfextract@process\righttailpoint{%
            \centerpoint%
            \advance\pgf@x-\deltax\relax%
            \advance\pgf@x-\taillength\relax%
            \advance\pgf@y-\halftailseparation\relax%
        }%
        %
        % Calculate the miter vectors. At the dart tip...
        %
        \pgfextract@process\tipmiter{%
            \pgfmathcosec@{\halftipangle}%
            \pgf@x\outersep\relax%
            \pgf@x\pgfmathresult\pgf@x%
            \pgf@y0pt\relax%
        }%
        %
        % ...at the tail center...
        %
        \pgfextract@process\tailcentermiter{%
            \pgfmathcosec@{\halftailangle}%
            \pgf@x-\outersep\relax%
            \pgf@x\pgfmathresult\pgf@x%
            \pgf@y0pt\relax%
        }%
        %
        % ...at the left tail...
        %
        \pgfmathsubtract@{\halftailangle}{\halftipangle}%
        \pgfmathdivide@{\pgfmathresult}{2}%
        \let\angle\pgfmathresult%
        \pgfmathcosec@{\pgfmathresult}%
        \pgf@x\outersep\relax%
        \pgf@x\pgfmathresult\pgf@x%
        \pgfmathadd@{\angle}{90}%
        \pgfmathsubtract{\pgfmathresult}{\halftailangle}%
        \pgfmathsincos@{\pgfmathresult}%
        \pgf@ya\pgfmathresultx\pgf@x%
        \pgf@xa\pgfmathresulty\pgf@x%
        \pgfextract@process\lefttailmiter{%
            \pgf@x-\pgf@xa%
            \pgf@y\pgf@ya%
        }%
        %
        % ...and the right tail.
        %
        \pgfextract@process\righttailmiter{%
            \pgf@x-\pgf@xa%
            \pgf@y-\pgf@ya%
        }%
        %
        % Create the border points.
        %
        \pgfextract@process\tipborderpoint{%
            \pgfpointadd{\tippoint}{\tipmiter}
        }%
        \pgfextract@process\tailcenterborderpoint{%
            \pgfpointadd{\tailcenterpoint}{\tailcentermiter}%
        }%
        \pgfextract@process\lefttailborderpoint{%
            \pgfpointadd{\lefttailpoint}{\lefttailmiter}%
        }%
        \pgfextract@process\righttailborderpoint{%
            \pgfpointadd{\righttailpoint}{\righttailmiter}%
        }%
        %
        % Calculate the angles between the centerpoint
        % and the *unrotated* borderpoints.
        %
        \pgfmathanglebetweenpoints{\centerpoint}{\tipborderpoint}%
        \let\angletotip\pgfmathresult%
        \pgfmathanglebetweenpoints{\centerpoint}{\tailcenterborderpoint}%
        \let\angletotailcenter\pgfmathresult%
        \pgfmathanglebetweenpoints{\centerpoint}{\lefttailborderpoint}%
        \let\angletolefttail\pgfmathresult%
        \pgfmathanglebetweenpoints{\centerpoint}{\righttailborderpoint}%
        \let\angletorighttail\pgfmathresult%
        %
        % Calculate the angles between the *inversely rotated*
        % basepoint and the *unrotated* borderpoints.
        %
        \pgfextract@process\rotatedbasepoint{%
            \pgfmathrotatepointaround{\basepoint}{\centerpoint}{-\rotate}%
        }%
        \pgfmathanglebetweenpoints{\rotatedbasepoint}{\tipborderpoint}%
        \let\baseangletotip\pgfmathresult%
        \pgfmathanglebetweenpoints{\rotatedbasepoint}{\tailcenterborderpoint}%
        \let\baseangletotailcenter\pgfmathresult%
        \pgfmathanglebetweenpoints{\rotatedbasepoint}{\lefttailborderpoint}%
        \let\baseangletolefttail\pgfmathresult%
        \pgfmathanglebetweenpoints{\rotatedbasepoint}{\righttailborderpoint}%
        \let\baseangletorighttail\pgfmathresult%
        %
        % Calculate the angles between the *inversely rotated*
        % midpoint and the *unrotated* borderpoints.
        %
        \pgfextract@process\rotatedmidpoint{%
            \pgfmathrotatepointaround{\midpoint}{\centerpoint}{-\rotate}%
        }%
        \pgfmathanglebetweenpoints{\rotatedmidpoint}{\tipborderpoint}%
        \let\midangletotip\pgfmathresult%
        \pgfmathanglebetweenpoints{\rotatedmidpoint}{\tailcenterborderpoint}%
        \let\midangletotailcenter\pgfmathresult%
        \pgfmathanglebetweenpoints{\rotatedmidpoint}{\lefttailborderpoint}%
        \let\midangletolefttail\pgfmathresult%
        \pgfmathanglebetweenpoints{\rotatedmidpoint}{\righttailborderpoint}%
        \let\midangletorighttail\pgfmathresult%
        %
        % Rotate the background path points.
        %
        \pgfextract@process\tippoint{%
            \pgfmathrotatepointaround{\tippoint}{\centerpoint}{\rotate}%
        }%
        \pgfextract@process\tailcenterpoint{%
            \pgfmathrotatepointaround{\tailcenterpoint}{\centerpoint}{\rotate}%
        }%
        \pgfextract@process\lefttailpoint{%
            \pgfmathrotatepointaround{\lefttailpoint}{\centerpoint}{\rotate}%
        }%
        \pgfextract@process\righttailpoint{%
            \pgfmathrotatepointaround{\righttailpoint}{\centerpoint}{\rotate}%
        }%
        %
        % Calculate a radius guaranteed to be outside the dart.
        %
        \pgfextractx\pgf@xa{\pgfpointdiff{\lefttailborderpoint}{\tipborderpoint}}%
        \pgfextracty\pgf@ya{\pgfpointdiff{\lefttailborderpoint}{\righttailborderpoint}}%
        \ifdim\pgf@xa<0pt\relax%
            \pgf@xa-\pgf@xa%
        \fi%
        \ifdim\pgf@ya<0pt\relax%
            \pgf@ya-\pgf@ya%
        \fi%
        \ifdim\pgf@xa>\pgf@ya%
            \edef\externalradius{\the\pgf@xa}%
        \else%
            \edef\externalradius{\the\pgf@ya}%
        \fi%
        %
        % Rotate the border anchor points.
        %
        \pgfextract@process\tipborderpoint{%
            \pgfmathrotatepointaround{\tipborderpoint}{\centerpoint}{\rotate}%
        }%
        \pgfextract@process\tailcenterborderpoint{%
            \pgfmathrotatepointaround{\tailcenterborderpoint}{\centerpoint}{\rotate}%
        }%
        \pgfextract@process\lefttailborderpoint{%
            \pgfmathrotatepointaround{\lefttailborderpoint}{\centerpoint}{\rotate}%
        }%
        \pgfextract@process\righttailborderpoint{%
            \pgfmathrotatepointaround{\righttailborderpoint}{\centerpoint}{\rotate}%
        }%
        %
        % Save everything.
        %
        \addtosavedmacro{\rotate}%
        \addtosavedmacro{\externalradius}%
        %
        \addtosavedmacro{\tippoint}%
        \addtosavedmacro{\tailcenterpoint}%
        \addtosavedmacro{\lefttailpoint}%
        \addtosavedmacro{\righttailpoint}%
        %
        \addtosavedmacro{\tipborderpoint}%
        \addtosavedmacro{\tailcenterborderpoint}%
        \addtosavedmacro{\lefttailborderpoint}%
        \addtosavedmacro{\righttailborderpoint}%
        %
        \addtosavedmacro{\angletotip}%
        \addtosavedmacro{\angletotailcenter}%
        \addtosavedmacro{\angletolefttail}%
        \addtosavedmacro{\angletorighttail}%
        %
        \addtosavedmacro{\baseangletotip}%
        \addtosavedmacro{\baseangletotailcenter}%
        \addtosavedmacro{\baseangletolefttail}%
        \addtosavedmacro{\baseangletorighttail}%
        %
        \addtosavedmacro{\midangletotip}%
        \addtosavedmacro{\midangletotailcenter}%
        \addtosavedmacro{\midangletolefttail}%
        \addtosavedmacro{\midangletorighttail}%
    }%
    \savedanchor\centerpoint{%
        \pgf@x.5\wd\pgfnodeparttextbox%
        \pgf@y.5\ht\pgfnodeparttextbox%
        \advance\pgf@y-.5\dp\pgfnodeparttextbox%
    }%
    \savedanchor\basepoint{%
        \pgf@x.5\wd\pgfnodeparttextbox%
        \pgf@y0pt\relax%
    }%
    \savedanchor\midpoint{%
        \pgf@x.5\wd\pgfnodeparttextbox%
        \pgfmathsetlength\pgf@y{+.5ex}%
    }%
    \anchor{center}{\centerpoint}%
    \anchor{base}{\basepoint}%
    \anchor{base west}{%
        \installdartparameters%
        \let\pgf@dartanchorborderreferencepoint\basepoint%
        \csname pgf@anchor@dart@border\endcsname{\pgfqpoint{-\externalradius}{0pt}}%
    }%
    \anchor{base east}{%
        \installdartparameters%
        \let\pgf@dartanchorborderreferencepoint\basepoint%
        \csname pgf@anchor@dart@border\endcsname{\pgfqpoint{\externalradius}{0pt}}%
    }%
    \anchor{mid}{\midpoint}%
    \anchor{mid west}{%
        \installdartparameters%
        \let\pgf@dartanchorborderreferencepoint\midpoint%
        \csname pgf@anchor@dart@border\endcsname{\pgfqpoint{-\externalradius}{0pt}}%
    }%
    \anchor{mid east}{%
        \installdartparameters%
        \let\pgf@dartanchorborderreferencepoint\midpoint%
        \csname pgf@anchor@dart@border\endcsname{\pgfqpoint{\externalradius}{0pt}}%
    }%
    \anchor{north}{%
        \installdartparameters%
        \csname pgf@anchor@dart@border\endcsname{\pgfqpoint{0pt}{\externalradius}}%
    }%
    \anchor{south}{%
        \installdartparameters%
        \csname pgf@anchor@dart@border\endcsname{\pgfqpoint{0pt}{-\externalradius}}%
    }%
    \anchor{east}{%
        \installdartparameters%
        \csname pgf@anchor@dart@border\endcsname{\pgfqpoint{\externalradius}{0pt}}%
    }%
    \anchor{west}{%
        \installdartparameters%
        \csname pgf@anchor@dart@border\endcsname{\pgfqpoint{-\externalradius}{0pt}}%
    }%
    \anchor{north east}{%
        \installdartparameters%
        \csname pgf@anchor@dart@border\endcsname{\pgfqpoint{\externalradius}{\externalradius}}%
    }%
    \anchor{south west}{%
        \installdartparameters%
        \csname pgf@anchor@dart@border\endcsname{\pgfqpoint{-\externalradius}{-\externalradius}}%
    }%
    \anchor{south east}{%
        \installdartparameters%
        \csname pgf@anchor@dart@border\endcsname{\pgfqpoint{\externalradius}{-\externalradius}}%
    }%
    \anchor{north west}{%
        \installdartparameters%
        \csname pgf@anchor@dart@border\endcsname{\pgfqpoint{-\externalradius}{\externalradius}}%
    }%
    \anchor{tip}{%
        \installdartparameters%
        \tipborderpoint}%
    \anchor{left tail}{%
        \installdartparameters%
        \lefttailborderpoint}%
    \anchor{right tail}{%
        \installdartparameters%
        \righttailborderpoint}%
    \anchor{tail center}{%
        \installdartparameters%
        \tailcenterborderpoint}%
    \anchor{left side}{%
        \installdartparameters%
        \pgfpointlineattime{0.5}{\tipborderpoint}{\lefttailborderpoint}}%
    \anchor{right side}{%
        \installdartparameters%
        \pgfpointlineattime{0.5}{\tipborderpoint}{\righttailborderpoint}}%
    \backgroundpath{%
        \installdartparameters%
        \pgfpathmoveto{\tippoint}%
        \pgfpathlineto{\lefttailpoint}%
        \pgfpathlineto{\tailcenterpoint}%
        \pgfpathlineto{\righttailpoint}%
        \pgfpathclose%
    }%
    \anchorborder{%
        %
        % Save x and y.
        %
        \edef\externalx{\the\pgf@x}%
        \edef\externaly{\the\pgf@y}%
        %
        % This allows anchors base east, base west, mid east and mid west,
        % to redefine the `center' of the node to correctly calculate the
        % border points.
        %
        \pgfutil@ifundefined{pgf@dartanchorborderreferencepoint}%
            {\let\referencepoint\centerpoint}%
            {\let\referencepoint\pgf@dartanchorborderreferencepoint}%
        %
        % Adjust the location of the external
        % point relative to the reference point.
        %
        \referencepoint%
        \pgf@xa\externalx\relax%
        \pgf@ya\externaly\relax%
        \advance\pgf@xa\pgf@x%
        \advance\pgf@ya\pgf@y%
        \edef\externalx{\the\pgf@xa}%
        \edef\externaly{\the\pgf@ya}%
        %
        % Install the required points and angles.
        %
        \installdartparameters%
        %
        % Get the angle of the external point relative to \referencepoint.
        %
        \pgfmathanglebetweenpoints{\referencepoint}{\pgfqpoint{\externalx}{\externaly}}%
        %
        % *Subtract* the rotation from the external angle.
        %
        \pgfmathsubtract@{\pgfmathresult}{\rotate}%
        \ifdim\pgfmathresult pt<0pt\relax%
            \pgfmathadd@{\pgfmathresult}{360}%
        \fi%
        \let\externalangle\pgfmathresult%
        %
        % Get the set of angles for the appropriate border point.
        %
        \ifx\referencepoint\basepoint%
            \let\angletotip\baseangletotip%
            \let\angletotailcenter\baseangletotailcenter%
            \let\angletolefttail\baseangletolefttail%
            \let\angletorighttail\baseangletorighttail%
        \else%
            \ifx\referencepoint\midpoint%
                \let\angletotip\midangletotip%
                \let\angletotailcenter\midangletotailcenter%
                \let\angletolefttail\midangletolefttail%
                \let\angletorighttail\midangletorighttail%
        \fi\fi%
        %
        % Locate the appropriate line on the border...
        %
        \ifdim\externalangle pt<\angletotip pt\relax%
            \let\firstpoint\tipborderpoint%
            \let\secondpoint\righttailborderpoint%
        \else%
            \ifdim\externalangle pt<\angletolefttail pt\relax%
                \let\firstpoint\lefttailborderpoint%
                \let\secondpoint\tipborderpoint%
            \else%
                \ifdim\externalangle pt<\angletotailcenter pt\relax%
                    \let\firstpoint\lefttailborderpoint%
                    \let\secondpoint\tailcenterborderpoint%
                \else%
                    \ifdim\externalangle pt<\angletorighttail pt\relax%
                        \let\firstpoint\righttailborderpoint%
                        \let\secondpoint\tailcenterborderpoint%
                    \else%
                        \let\firstpoint\tipborderpoint%
                        \let\secondpoint\righttailborderpoint%
                    \fi%
                \fi%
            \fi%
        \fi%
        %
        % ...and thus the point on the border.
        %
        \pgfpointintersectionoflines{\referencepoint}{\pgfqpoint{\externalx}{\externaly}}%
                {\firstpoint}{\secondpoint}%
    }%
}%



% Keys for shape circular sector:
%
% /pgf/circular sector angle : the angle at the center of the sector.
%
\pgfkeys{/pgf/circular sector angle/.initial=60}%

% Shape circular sector
%
\pgfdeclareshape{circular sector}{%
    \savedmacro\installcircularsectorparameters{%
        %
        % Define a \centerpoint, \basepoint and \midpoint.
        %
        \pgfextract@process\centerpoint{%
            \pgf@x.5\wd\pgfnodeparttextbox%
            \pgf@y.5\ht\pgfnodeparttextbox%
            \advance\pgf@y-.5\dp\pgfnodeparttextbox%
        }%
        %
        % Get the larger of the outer sep.
        %
        \pgfmathsetlength\pgf@x{\pgfkeysvalueof{/pgf/outer xsep}}%
        \pgfmathsetlength\pgf@y{\pgfkeysvalueof{/pgf/outer ysep}}%
        \ifdim\pgf@x<\pgf@y%
            \pgf@x\pgf@y%
        \fi%
        \edef\outersep{\the\pgf@x}%
        %
        % Half of the sector angle is more useful.
        %
        \pgfmathmod{\pgfkeysvalueof{/pgf/circular sector angle}}{360}%
        \ifdim\pgfmathresult pt<0pt\relax%
            \pgfmathadd@{\pgfmathresult}{360}%
        \fi%
        \let\angle\pgfmathresult%
        \pgfmathdivide@{\pgfmathresult}{2}%
        \let\halfangle\pgfmathresult%
        %
        % Get some useful trigonometric stuff.
        %
        \pgfmathsin@{\halfangle}%
        \let\sinehalfangle\pgfmathresult%
        \pgfmathreciprocal@{\pgfmathresult}%
        \let\cosechalfangle\pgfmathresult%
        \pgfmathabs@{\halfangle}%
        \pgfmathcos@{\pgfmathresult}%
        \let\coshalfangle\pgfmathresult%
        \pgf@x\pgfmathresult pt\relax%
        \pgf@x\cosechalfangle\pgf@x%
        \edef\cothalfangle{\pgfmath@tonumber{\pgf@x}}%
        %
        % Calculate the miter length at the center,
        % according to the outer sep.
        %
        \pgf@x\outersep\relax%
        \pgf@x\cosechalfangle\pgf@x%
        \edef\centermiter{\the\pgf@x}%
        %
        % Get the start and end angles of the arc.
        %
        \pgfmathsubtract@{180}{\halfangle}%
        \let\startangle\pgfmathresult%
        \pgfmathadd@{180}{\halfangle}%
        \let\endangle\pgfmathresult%
        %
        % Get (half) the node dimensions.
        %
        \pgfmathsetlength\pgf@x{\pgfkeysvalueof{/pgf/inner xsep}}%
        \advance\pgf@x.5\wd\pgfnodeparttextbox%
        \pgfmathsetlength\pgf@y{\pgfkeysvalueof{/pgf/inner ysep}}%
        \advance\pgf@y.5\ht\pgfnodeparttextbox%
        \advance\pgf@y.5\dp\pgfnodeparttextbox%
        %
        % Calculate the radius of the sector and the `center offset',
        % which is the distance between the center of the node and the
        % center of the sector.
        %
        \ifpgfshapeborderusesincircle%
            %
            % Get the rotation.
            %
            \pgfmathsetmacro\rotate{\pgfkeysvalueof{/pgf/shape border rotate}}%
            %
            % Calculate the incircle radius.
            %
            \ifdim\pgf@x<\pgf@y%
                \pgf@x\pgf@y%
            \fi%
            \pgf@x1.41421\pgf@x%
            \pgf@xa\cosechalfangle\pgf@x%
            %
            % xa is the radius, xb the offset.
            %
            \pgf@xb\pgf@xa%
            \advance\pgf@xa\pgf@x%
        \else%
            %
            % Get the rotation (with rounding).
            %
            \pgfmathmod{\pgfkeysvalueof{/pgf/shape border rotate}}{360}%
            \afterassignment\pgfmath@gobbletilpgfmath@%
            \expandafter\c@pgf@counta\pgfmathresult\relax\pgfmath@%
            \advance\c@pgf@counta45\relax%
            \divide\c@pgf@counta90\relax%
            \multiply\c@pgf@counta90\relax%
            \ifnum\c@pgf@counta<0\relax%
                \advance\c@pgf@counta360\relax%
            \fi%
            \edef\rotate{\the\c@pgf@counta}%
            %
            % Swap the height and width for relevant angles.
            %
            \ifnum\c@pgf@counta=90\relax%
                \pgf@xc\pgf@x%
                \pgf@x\pgf@y%
                \pgf@y\pgf@xc%
            \else%
                \ifnum\c@pgf@counta=270\relax%
                    \pgf@xc\pgf@x%
                    \pgf@x\pgf@y%
                    \pgf@y\pgf@xc%
                \fi%
            \fi%
            %
            % xa is the radius, xb the offset.
            %
            \pgf@xa\cosechalfangle\pgf@y%
            \pgf@xa\coshalfangle\pgf@xa%
            \advance\pgf@xa\pgf@x%
            \pgf@xb\pgf@xa%
            \advance\pgf@xa\pgf@x%
            \pgfmathveclen@{\pgfmath@tonumber{\pgf@xa}}{\pgfmath@tonumber{\pgf@y}}%
            \pgf@xa\pgfmathresult pt\relax%
        \fi%
        %
        % Adjust for minimum height and width.
        %
        \pgfmathsetlength\pgf@x{\pgfkeysvalueof{/pgf/minimum width}}%
        \pgf@x.5\pgf@x%
        \pgfmathsec@{\halfangle}%
        \pgf@xc\pgfmathresult\pgf@xb%
        \ifdim\pgf@xc<0pt\relax%
            \pgf@xc-\pgf@xc%
        \fi%
        \ifdim\pgf@xc<\pgf@x%
            \pgfmathreciprocal{\pgfmath@tonumber{\pgf@xc}}%
            \pgf@xc\pgfmathresult\pgf@x% Increase by ratio.
            \pgf@xa\pgfmath@tonumber{\pgf@xc}\pgf@xa%
            \pgf@xb\pgfmath@tonumber{\pgf@xc}\pgf@xb%
        \fi%
        \pgfmathsetlength\pgf@y{\pgfkeysvalueof{/pgf/minimum height}}%
        \ifdim\pgf@xa<\pgf@y%
            \pgfmathreciprocal{\pgfmath@tonumber{\pgf@xa}}%
            \pgf@xc\pgfmathresult\pgf@y% Increase by ratio.
            \pgf@xa\pgf@y%
            \pgf@xb\pgfmath@tonumber{\pgf@xc}\pgf@xb%
        \fi%
        \edef\centeroffset{\the\pgf@xb}%
        \edef\radius{\the\pgf@xa}%
        %
        % Calculate the radius at sector border and the arc corners.
        %
        \advance\pgf@xa\outersep\relax%
        \edef\borderradius{\the\pgf@xa}%
        \pgf@x\outersep\relax%
        \pgf@x\cosechalfangle\pgf@x%
        \pgf@x\coshalfangle\pgf@x%
        \advance\pgf@xa\pgf@x%
        \edef\cornerradius{\the\pgf@xa}%
        %
        % Calculate the *unrotated* points for the background path.
        %
        \pgfextract@process\sectorcenter{%
            \centerpoint%
            \advance\pgf@x\centeroffset%
        }%
        \pgfextract@process\arcstart{%
            \pgfqpointpolar{\startangle}{\radius}%
            \pgf@xa\pgf@x%
            \pgf@ya\pgf@y%
            \sectorcenter%
            \advance\pgf@x\pgf@xa%
            \advance\pgf@y\pgf@ya%
        }%
        %
        % Calculate the *unrotated* points for the anchor border.
        %
        \pgfextract@process\sectorcenterborder{%
            \sectorcenter%
            \advance\pgf@x\centermiter\relax%
        }%
        \pgfextract@process\arcstartborder{%
            \pgfqpointpolar{\startangle}{\borderradius}%
            \pgf@xa\pgf@x%
            \pgf@ya\pgf@y%
            \sectorcenter%
            \advance\pgf@x\pgf@xa%
            \advance\pgf@y\pgf@ya%
        }%
        \pgfextract@process\arcendborder{%
            \pgfqpointpolar{\endangle}{\borderradius}%
            \pgf@xa\pgf@x%
            \pgf@ya\pgf@y%
            \sectorcenter%
            \advance\pgf@x\pgf@xa%
            \advance\pgf@y\pgf@ya%
        }%
        \pgfextract@process\arcstartcorner{%
            \pgfqpointpolar{\startangle}{\cornerradius}%
            \pgf@xa\pgf@x%
            \pgf@ya\pgf@y%
            \sectorcenterborder%
            \advance\pgf@x\pgf@xa%
            \advance\pgf@y\pgf@ya%
        }%
        \pgfextract@process\arcendcorner{%
            \pgfqpointpolar{\endangle}{\cornerradius}%
            \pgf@xa\pgf@x%
            \pgf@ya\pgf@y%
            \sectorcenterborder%
            \advance\pgf@x\pgf@xa%
            \advance\pgf@y\pgf@ya%
        }%
        %
        % Calculate the *unrotated* corner angles to the \centerpoint.
        %
        \pgfmathanglebetweenpoints{\centerpoint}{\sectorcenterborder}%
        \let\angletosectorcenterborder\pgfmathresult%
        \pgfmathanglebetweenpoints{\centerpoint}{\arcstartborder}%
        \let\angletoarcstartborder\pgfmathresult%
        \pgfmathanglebetweenpoints{\centerpoint}{\arcendborder}%
        \let\angletoarcendborder\pgfmathresult%
        \pgfmathanglebetweenpoints{\centerpoint}{\arcstartcorner}%
        \let\angletoarcstartcorner\pgfmathresult%
        \pgfmathanglebetweenpoints{\centerpoint}{\arcendcorner}%
        \let\angletoarcendcorner\pgfmathresult%
        %
        % Rotate the background path points around the note center.
        %
        \pgfextract@process\sectorcenter{%
            \pgfmathrotatepointaround{\sectorcenter}{\centerpoint}{\rotate}%
        }%
        \pgfextract@process\arcstart{%
            \pgfmathrotatepointaround{\arcstart}{\centerpoint}{\rotate}%
        }%
        %
        % Rotate the border points around the note center.
        %
        \pgfextract@process\sectorcenterborder{%
            \pgfmathrotatepointaround{\sectorcenterborder}{\centerpoint}{\rotate}%
        }%
        \pgfextract@process\arcstartborder{%
            \pgfmathrotatepointaround{\arcstartborder}{\centerpoint}{\rotate}%
        }%
        \pgfextract@process\arcendborder{%
            \pgfmathrotatepointaround{\arcendborder}{\centerpoint}{\rotate}%
        }%
        \pgfextract@process\arcstartcorner{%
            \pgfmathrotatepointaround{\arcstartcorner}{\centerpoint}{\rotate}%
        }%
        \pgfextract@process\arcendcorner{%
            \pgfmathrotatepointaround{\arcendcorner}{\centerpoint}{\rotate}%
        }%
        %
        % Add the rotation to the start and end angles.
        %
        \pgfmathadd@{\startangle}{\rotate}%
        \let\startangle\pgfmathresult%
        \pgfmathadd@{\endangle}{\rotate}%
        \let\endangle\pgfmathresult%
        %
        % Save everything.
        %
        % NB \addtosavedmacro is currently experimental. May get changed.
        %
        \addtosavedmacro{\rotate}%
        \addtosavedmacro{\centeroffset}%
        %
        \addtosavedmacro{\radius}%
        \addtosavedmacro{\borderradius}%
        \addtosavedmacro{\cornerradius}%
        %
        \addtosavedmacro{\endangle}%
        \addtosavedmacro{\startangle}%
        %
        \addtosavedmacro{\sectorcenter}%
        \addtosavedmacro{\arcstart}%
        %
        \addtosavedmacro{\sectorcenterborder}%
        \addtosavedmacro{\arcstartborder}%
        \addtosavedmacro{\arcendborder}%
        \addtosavedmacro{\arcstartcorner}%
        \addtosavedmacro{\arcendcorner}%
        %
        \addtosavedmacro{\angletosectorcenterborder}%
        \addtosavedmacro{\angletoarcstartborder}%
        \addtosavedmacro{\angletoarcendborder}%
        \addtosavedmacro{\angletoarcstartcorner}%
        \addtosavedmacro{\angletoarcendcorner}%
    }%
    \savedanchor\centerpoint{%
        \pgf@x.5\wd\pgfnodeparttextbox%
        \pgf@y.5\ht\pgfnodeparttextbox%
        \advance\pgf@y-.5\dp\pgfnodeparttextbox%
    }%
    \savedanchor\basepoint{%
        \pgf@x.5\wd\pgfnodeparttextbox%
        \pgf@y0pt\relax%
    }%
    \savedanchor\midpoint{%
        \pgf@x.5\wd\pgfnodeparttextbox%
        \pgfmathsetlength\pgf@y{+.5ex}%
    }%
    \anchor{center}{\centerpoint}%
    \anchor{base}{\basepoint}%
    \anchor{mid}{\midpoint}%
    \anchor{arc start}{%
        \installcircularsectorparameters%
        \arcstartcorner%
    }%
    \anchor{arc end}{%
        \installcircularsectorparameters%
        \arcendcorner%
    }%
    \anchor{sector center}{%
        \installcircularsectorparameters%
        \sectorcenterborder%
    }%
    \anchor{arc center}{%
        \installcircularsectorparameters%
        \pgfmathadd@{\rotate}{180}%
        \let\angle\pgfmathresult%
        \csname pgf@anchor@circular sector@border\endcsname{\pgfqpointpolar{\angle}{\cornerradius}}%
    }%
    \anchor{north}{%
        \installcircularsectorparameters%
        \csname pgf@anchor@circular sector@border\endcsname{\pgfqpoint{0pt}{\cornerradius}}%
    }%
    \anchor{south}{%
        \installcircularsectorparameters%
        \csname pgf@anchor@circular sector@border\endcsname{\pgfqpoint{0pt}{-\cornerradius}}%
    }%
    \anchor{east}{%
        \installcircularsectorparameters%
        \csname pgf@anchor@circular sector@border\endcsname{\pgfqpoint{\cornerradius}{0pt}}%
    }%
    \anchor{west}{%
        \installcircularsectorparameters%
        \csname pgf@anchor@circular sector@border\endcsname{\pgfqpoint{-\cornerradius}{0pt}}%
    }%
    \anchor{north west}{%
        \installcircularsectorparameters%
        \csname pgf@anchor@circular sector@border\endcsname{\pgfqpoint{-\cornerradius}{\cornerradius}}%
    }%
    \anchor{south west}{%
        \installcircularsectorparameters%
        \csname pgf@anchor@circular sector@border\endcsname{\pgfqpoint{-\cornerradius}{-\cornerradius}}%
    }%
    \anchor{north east}{%
        \installcircularsectorparameters%
        \csname pgf@anchor@circular sector@border\endcsname{\pgfqpoint{\cornerradius}{\cornerradius}}%
    }%
    \anchor{south east}{%
        \installcircularsectorparameters%
        \csname pgf@anchor@circular sector@border\endcsname{\pgfqpoint{\cornerradius}{-\cornerradius}}%
    }%
    \backgroundpath{%
        \installcircularsectorparameters%
        \pgfpathmoveto{\sectorcenter}%
        \pgfpathlineto{\arcstart}%
        \ifdim\endangle pt>360pt\relax%
            \ifdim\startangle pt>360pt\relax%
                \pgfpatharc{\startangle}{\endangle}{\radius}%
            \else%
                \pgfpatharc{\startangle}{360}{\radius}%
                \pgfpatharc{0}{\endangle}{\radius}%
            \fi%
        \else%
            \pgfpatharc{\startangle}{\endangle}{\radius}%
        \fi%
        \pgfpathclose%
    }%
    \anchorborder{%
        %
        % Save x and y.
        %
        \edef\externalx{\the\pgf@x}%
        \edef\externaly{\the\pgf@y}%
        %
        % Adjust the location of the external  point relative to \centerpoint.
        %
        \centerpoint%
        \pgf@xa\externalx\relax%
        \pgf@ya\externaly\relax%
        \advance\pgf@xa\pgf@x%
        \advance\pgf@ya\pgf@y%
        \edef\externalx{\the\pgf@xa}%
        \edef\externaly{\the\pgf@ya}%
        %
        % Install the required points and angles.
        %
        \installcircularsectorparameters%
        %
        % Get the angle of the external point to the \centerpoint.
        %
        \pgfmathanglebetweenpoints{\centerpoint}{\pgfqpoint{\externalx}{\externaly}}%
        \let\externalangle\pgfmathresult%
        %
        % *Subtract* the rotation from the external angle.
        %
        \pgfmathsubtract@{\pgfmathresult}{\rotate}%
        \ifdim\pgfmathresult pt<0pt\relax%
            \pgfmathadd@{\pgfmathresult}{360}%
        \fi%
        \let\angle\pgfmathresult%
        %
        % Determine if the line will cross the sector arc.
        %
        \ifdim\angle pt>\angletoarcendcorner pt\relax%
            \let\firstpoint\arcendcorner%
            \let\secondpoint\sectorcenterborder%
        \else%
            \ifdim\angle pt>\angletoarcendborder pt\relax%
                \let\firstpoint\arcendborder%
                \let\secondpoint\arcendcorner%
            \else%
                \ifdim\angle pt>\angletoarcstartborder pt\relax%
                    \let\firstpoint\pgfutil@empty%
                    \let\secondpoint\pgfutil@empty%
                \else%
                    \ifdim\angle pt>\angletoarcstartcorner pt\relax%
                        \let\firstpoint\arcstartborder
                        \let\secondpoint\arcstartcorner
                    \else%
                        \let\firstpoint\sectorcenterborder%
                        \let\secondpoint\arcstartcorner%
                    \fi%
                \fi%
            \fi%
        \fi%
        \ifx\firstpoint\pgfutil@empty
            %
            % Calculate the angle from the centre of the semicircle to the
            % point on the semicircle arc which intersects the line from
            % the external point to the reference point...
            %
            \pgfmathanglebetweenlines{\centerpoint}{\pgfqpoint{\externalx}{\externaly}}%
                {\sectorcenter}{\centerpoint}%
            \pgfmathsin@{\pgfmathresult}%
            \let\sineangle\pgfmathresult%
            \pgf@x\borderradius\relax%
            \pgfmathreciprocal@{\pgfmath@tonumber{\pgf@x}}%
            \let\reciprocalradius\pgfmathresult%
            \pgf@x\centeroffset\relax%
            \pgf@x\sineangle\pgf@x%
            \pgf@x\reciprocalradius\pgf@x%
            \pgfmathasin@{\pgfmath@tonumber{\pgf@x}}%
            \pgf@x\pgfmathresult pt\relax%
            \advance\pgf@x\externalangle pt\relax%
            \edef\angle{\pgfmath@tonumber{\pgf@x}}%
            %
            % ...and thus the point on the border.
            %
            \pgfpointadd{\sectorcenter}{\pgfqpointpolar{\angle}{\borderradius}}%
        \else%
            %
            % Calculate the the point where the semicircle chord intersects
            % the line from the external point to the reference point.
            %
            \pgfpointintersectionoflines{\centerpoint}{\pgfqpoint{\externalx}{\externaly}}%
                {\firstpoint}{\secondpoint}%
        \fi%
    }%
}%



% Keys for shape cylinder
%
% /pgf/shape aspect              : Ratio between the x and y radii of the cylinder end.
% /pgf/cylinder uses custom fill : Use a custom fill for the cylinder.
% /pgf/cylinder end fill         : Custom color for the cylinder end.
% /pgf/cylinder body fill        : Custom color for the cylinder body.
%

\newif\ifpgfcylinderusescustomfill
\pgfkeys{/pgf/.cd,
    cylinder uses custom fill/.is if=pgfcylinderusescustomfill,
    cylinder end fill/.initial=white,
    cylinder body fill/.initial=white,
}%


\pgfdeclareshape{cylinder}{%
    \savedmacro\getcylinderpoints{%
        \pgfmathsetlength\pgf@xc{\pgfkeysvalueof{/pgf/inner xsep}}%
        \pgf@x\pgf@xc%
        \advance\pgf@x.5\wd\pgfnodeparttextbox%
        \pgfmathsetlength\pgf@yc{\pgfkeysvalueof{/pgf/inner ysep}}%
        \pgf@y\pgf@yc%
        \advance\pgf@y.5\ht\pgfnodeparttextbox%
        \advance\pgf@y.5\dp\pgfnodeparttextbox%
        \ifpgfshapeborderusesincircle%
            \pgfmathsetmacro\rotate{\pgfkeysvalueof{/pgf/shape border rotate}}%
            \ifdim\pgf@x<\pgf@y%
                \pgf@x\pgf@y%
            \else%
                \pgf@y\pgf@x%
            \fi%
            \pgf@x1.414213\pgf@x%
            \pgf@y1.414213\pgf@y%
        \else%
            \pgfmathmod{\pgfkeysvalueof{/pgf/shape border rotate}}{360}%
            \ifdim\pgfmathresult pt<0pt\relax%
                \pgfmathadd@{\pgfmathresult}{360}%
            \fi%
            \pgfmathsetcount\c@pgf@counta{+\pgfmathresult}%
            \advance\c@pgf@counta45\relax%
            \divide\c@pgf@counta90\relax%
            \multiply\c@pgf@counta90\relax%
            \edef\rotate{\the\c@pgf@counta}%
            \ifnum\c@pgf@counta=90\relax%
                \pgf@xa\pgf@x%
                \pgf@x\pgf@y%
                \pgf@y\pgf@xa%
                \pgf@yc\pgf@xc%
            \else%
                \ifnum\c@pgf@counta=270\relax%
                    \pgf@xa\pgf@x%
                    \pgf@x\pgf@y%
                    \pgf@y\pgf@xa%
                    \pgf@yc\pgf@xc%
                \fi%
            \fi%
        \fi%
        \addtosavedmacro\rotate%
        \pgf@xa\pgf@x%
        \pgf@ya\pgf@y%
        \pgfutil@tempdima\pgfshapeaspect\pgf@ya%
        \pgfutil@tempdimb\pgf@ya%
        %
        % Adjust for minimum height.
        %
        \pgfmathsetlength\pgf@xc{\pgfkeysvalueof{/pgf/minimum width}}%
        \ifdim\pgfutil@tempdimb<.5\pgf@xc\relax%
            \pgfutil@tempdimb.5\pgf@xc%
            \pgf@ya\pgfutil@tempdimb%
        \fi%
        %
        % Calculate how far the node contents can extend into the cylinder bottom.
        %
        \pgf@yb\pgfutil@tempdimb%
        \advance\pgf@yb-\pgf@yc%
        \pgfmathdivide@{\pgfmath@tonumber{\pgf@yb}}{\pgfmath@tonumber{\pgfutil@tempdimb}}%
        \pgfmathasin@{\pgfmathresult}%
        \pgfmathcos@{\pgfmathresult}%
        \let\angle\pgfmathresult%
        \pgf@xb\pgfmathresult\pgfutil@tempdima%
        %
        % Adjust for minimum width.
        %
        \pgf@x.5\pgflinewidth%
        \advance\pgf@x2.0\pgf@xa%
        \advance\pgf@x3.0\pgfutil@tempdima%
        \advance\pgf@x-\pgf@xb%
        \pgfmathsetlength\pgf@xc{\pgfkeysvalueof{/pgf/minimum height}}%
        \ifdim\pgf@x<\pgf@xc%
            \advance\pgf@xc-\pgf@x%
            \advance\pgf@xa.5\pgf@xc%
        \fi%
        %
        % Add the larger of the outer sep to the radii.
        %
        \pgf@x\pgfutil@tempdima\relax%
        \pgf@y\pgfutil@tempdimb\relax%
        \pgfmathsetlength\pgf@xc{\pgfkeysvalueof{/pgf/outer xsep}}%
        \pgfmathsetlength\pgf@yc{\pgfkeysvalueof{/pgf/outer ysep}}%
        \ifdim\pgf@xc>\pgf@yc%
            \advance\pgf@x\pgf@xc%
            \advance\pgf@y\pgf@xc%
            \edef\outersep{\the\pgf@xc}%
        \else%
            \advance\pgf@x\pgf@yc%
            \advance\pgf@y\pgf@yc%
            \edef\outersep{\the\pgf@yc}%
        \fi%
        \edef\xradius{\the\pgf@x}%
        \edef\yradius{\the\pgf@y}%
        \addtosavedmacro\xradius%
        \addtosavedmacro\yradius%
        \addtosavedmacro\outersep%
        %
        \pgfextract@process\cylindercenter{%
            \pgf@x\pgfutil@tempdima%
            \advance\pgf@x.5\pgflinewidth%
            \advance\pgf@x\pgf@xb%
            \pgf@x.5\pgf@x%
            \pgf@y0pt%
        }%
        \addtosavedmacro\cylindercenter%
        %
        \pgfextract@process\beforetop{%
            \pgf@x\pgf@xa%
            \advance\pgf@x\pgfutil@tempdima%
            \advance\pgf@x.5\pgflinewidth%
            \pgf@y\pgf@ya%
        }%
        \pgfextract@process\afterbottom{%
            \pgf@x-\pgf@xa%
            \advance\pgf@x\pgf@xb%
            \pgf@y\pgf@ya%
        }%
        \addtosavedmacro\beforetop%
        \addtosavedmacro\afterbottom%
        \pgfmathsetlength\pgf@yc{\pgfkeysvalueof{/pgf/outer ysep}}%
        \pgfextract@process\beforetopanchor{%
            \beforetop%
            \advance\pgf@y\pgf@yc%
        }%
        \pgfextract@process\afterbottomanchor{%
            \afterbottom%
            \advance\pgf@y\pgf@yc%
        }%
        \addtosavedmacro\beforetopanchor%
        \addtosavedmacro\afterbottomanchor%
        %
        \beforetopanchor%
        \advance\pgf@x\xradius\relax%
        \ifdim\pgf@x>\pgf@y%
            \edef\externalradius{\the\pgf@x}%
        \else%
            \edef\externalradius{\the\pgf@y}%
        \fi%
        \addtosavedmacro\externalradius%
    }%
    \savedanchor\centerpoint{%
        \pgf@x.5\wd\pgfnodeparttextbox%
        \pgf@y.5\ht\pgfnodeparttextbox%
        \advance\pgf@y-.5\dp\pgfnodeparttextbox%
    }%
    \savedanchor\midpoint{%
        \pgf@x.5\wd\pgfnodeparttextbox%
        \pgfmathsetlength\pgf@y{+0.5ex}%
    }%
    \savedanchor\basepoint{%
        \pgf@x.5\wd\pgfnodeparttextbox%
        \pgf@y0pt%
    }%
    \anchor{center}{\centerpoint}%
    \anchor{shape center}{%
        \getcylinderpoints%
        \pgfmathrotatepointaround{\pgfpointadd{\cylindercenter}{\centerpoint}}%
            {\centerpoint}{\rotate}%
    }%
    \anchor{mid}{\midpoint}%
    \anchor{mid east}{%
        \getcylinderpoints%
        \let\pgf@cylinder@referencepoint\midpoint%
        \pgf@anchor@cylinder@border{\pgfqpoint{\externalradius}{0pt}}%
    }%
    \anchor{mid west}{%
        \getcylinderpoints%
        \let\pgf@cylinder@referencepoint\midpoint%
        \pgf@anchor@cylinder@border{\pgfqpoint{-\externalradius}{0pt}}%
    }%
    \anchor{base}{\basepoint}%
    \anchor{base east}{%
        \getcylinderpoints%
        \let\pgf@cylinder@referencepoint\basepoint%
        \pgf@anchor@cylinder@border{\pgfqpoint{\externalradius}{0pt}}%
    }%
    \anchor{base west}{%
        \getcylinderpoints%
        \let\pgf@cylinder@referencepoint\basepoint%
        \pgf@anchor@cylinder@border{\pgfqpoint{-\externalradius}{0pt}}%
    }%
    \anchor{north}{%
        \getcylinderpoints%
        \pgf@anchor@cylinder@border{\pgfqpoint{0pt}{\externalradius}}%
    }%
    \anchor{south}{%
        \getcylinderpoints%
        \pgf@anchor@cylinder@border{\pgfqpoint{0pt}{-\externalradius}}%
    }%
    \anchor{east}{%
        \getcylinderpoints%
        \pgf@anchor@cylinder@border{\pgfqpoint{\externalradius}{0pt}}%
    }%
    \anchor{west}{%
        \getcylinderpoints%
        \pgf@anchor@cylinder@border{\pgfqpoint{-\externalradius}{0pt}}%
    }%
    \anchor{north east}{%
        \getcylinderpoints%
        \pgf@anchor@cylinder@border{\pgfqpoint{\externalradius}{\externalradius}}%
    }%
    \anchor{south west}{%
        \getcylinderpoints%
        \pgf@anchor@cylinder@border{\pgfqpoint{-\externalradius}{-\externalradius}}%
    }%
    \anchor{south east}{%
        \getcylinderpoints%
        \pgf@anchor@cylinder@border{\pgfqpoint{\externalradius}{-\externalradius}}%
    }%
    \anchor{north west}{%
        \getcylinderpoints%
        \pgf@anchor@cylinder@border{\pgfqpoint{-\externalradius}{\externalradius}}%
    }%
    \anchor{before top}{%
        \getcylinderpoints%
        \pgfmathrotatepointaround{\pgfpointadd{\beforetopanchor}{\centerpoint}}{\centerpoint}{\rotate}%
    }%
    \anchor{top}{%
        \getcylinderpoints%
        \pgfmathrotatepointaround{%
            \pgfpointadd{%
                \beforetop%
                \pgf@y0pt\relax%
                \advance\pgf@x\xradius\relax%
            }{\centerpoint}}{\centerpoint}{\rotate}%
    }%
    \anchor{after top}{%
        \getcylinderpoints%
        \pgfmathrotatepointaround{\pgfpointadd{\beforetopanchor\pgf@y-\pgf@y}{\centerpoint}}{\centerpoint}{\rotate}%
    }%
    \anchor{before bottom}{%
        \getcylinderpoints%
        \pgfmathrotatepointaround{\pgfpointadd{\afterbottomanchor\pgf@y-\pgf@y}{\centerpoint}}{\centerpoint}{\rotate}%
    }%
    \anchor{bottom}{%
        \getcylinderpoints%
        \pgfmathrotatepointaround{%
            \pgfpointadd{%
                \afterbottom%
                \pgf@y0pt\relax%
                \advance\pgf@x-\xradius\relax%
            }{\centerpoint}}{\centerpoint}{\rotate}%
    }%
    \anchor{after bottom}{%
        \getcylinderpoints%
        \pgfmathrotatepointaround{\pgfpointadd{\afterbottomanchor}{\centerpoint}}{\centerpoint}{\rotate}%
    }%
    \backgroundpath{%
        \getcylinderpoints%
        {%
            \pgf@x\xradius\relax%
            \advance\pgf@x-\outersep\relax%
            \edef\xradius{\the\pgf@x}%
            \pgf@y\yradius\relax%
            \advance\pgf@y-\outersep\relax%
            \edef\yradius{\the\pgf@y}%
            \pgftransformshift{\centerpoint}%
            \pgftransformrotate{\rotate}%
            \pgfpathmoveto{\afterbottom}%
            \pgfpatharc{90}{270}{\xradius and \yradius}%
            \pgfpathlineto{\beforetop\pgf@y-\pgf@y}%
            \pgfpatharc{-90}{90}{\xradius and \yradius}%
            \pgfpathclose%
            \pgfpathmoveto{\beforetop}%
            \pgfpatharc{90}{270}{\xradius and \yradius}%
        }%
    }%
    \behindbackgroundpath{%
        \ifpgfcylinderusescustomfill%
            \getcylinderpoints%
            \pgf@x\xradius\relax%
            \advance\pgf@x-\outersep\relax%
            \edef\xradius{\the\pgf@x}%
            \pgf@y\yradius\relax%
            \advance\pgf@y-\outersep\relax%
            \edef\yradius{\the\pgf@y}%
            {%
                \pgftransformshift{\centerpoint}%
                \pgftransformrotate{\rotate}%
                \pgfpathmoveto{\afterbottom}%
                \pgfpatharc{90}{270}{\xradius and \yradius}%
                \pgfpathlineto{\beforetop\pgf@y-\pgf@y}%
                \pgfpatharc{270}{90}{\xradius and \yradius}%
                \pgfpathclose%
                \expandafter\pgfsetfillcolor\expandafter{\pgfkeysvalueof{/pgf/cylinder body fill}}%
                \pgfusepath{fill}%
                %
                \pgfpathmoveto{\beforetop}%
                \pgfpatharc{90}{-270}{\xradius and \yradius}%
                \pgfpathclose
                \expandafter\pgfsetfillcolor\expandafter{\pgfkeysvalueof{/pgf/cylinder end fill}}%
                \pgfusepath{fill}%
            }%
        \fi%
    }%
    \anchorborder{%
        \pgfextract@process\externalpoint{}%
        \getcylinderpoints%
        \pgfutil@ifundefined{pgf@cylinder@referencepoint}{\let\referencepoint\centerpoint}{%
            \let\referencepoint\pgf@cylinder@referencepoint}%
        \pgfextract@process\externalpoint{%
            \externalpoint%
            \pgf@xa\pgf@x%
            \pgf@ya\pgf@y%
            \referencepoint%
            \advance\pgf@x\pgf@xa%
            \advance\pgf@y\pgf@ya%
        }%
        \pgfmathanglebetweenpoints{\centerpoint}{\externalpoint}%
        \pgfmathsubtract@{\pgfmathresult}{\rotate}%
        \ifdim\pgfmathresult pt<0pt\relax%
            \pgfmathadd@{\pgfmathresult}{360}%
        \fi%
        \let\externalangle\pgfmathresult%
        %
        \pgfmathanglebetweenpoints{\referencepoint}{\pgfpointadd{\afterbottomanchor}{\centerpoint}}%
        \ifdim\externalangle pt<\pgfmathresult pt\relax%
            \pgfmathanglebetweenpoints{\referencepoint}{\pgfpointadd{\beforetopanchor}{\centerpoint}}%
            \ifdim\externalangle pt<\pgfmathresult pt\relax%
                \pgfmathrotatepointaround{%
                    \pgfmathpointintersectionoflineandarc%
                        {\pgfmathrotatepointaround{\externalpoint}{\centerpoint}{-\rotate}}%
                        {\pgfmathrotatepointaround{\referencepoint}{\centerpoint}{-\rotate}}%
                        {%
                            \beforetop%
                            \pgf@xa\pgf@x%
                            \centerpoint%
                            \advance\pgf@x\pgf@xa%
                        }%
                        {0}{90}{\xradius and \yradius}%
                    }{\centerpoint}{\rotate}%
            \else%
                \pgfpointintersectionoflines{%
                    \pgfmathrotatepointaround{\pgfpointadd{\afterbottomanchor}{\centerpoint}}%
                        {\centerpoint}{\rotate}}{%
                    \pgfmathrotatepointaround{\pgfpointadd{\beforetopanchor}{\centerpoint}}%
                        {\centerpoint}{\rotate}}%
                    {\referencepoint}{\externalpoint}%
            \fi%
        \else%
            \pgfmathanglebetweenpoints{\referencepoint}{\pgfpointadd{\afterbottomanchor\pgf@y-\pgf@y}{\centerpoint}}%
            \ifdim\externalangle pt>\pgfmathresult pt\relax%
                \pgfmathanglebetweenpoints{\referencepoint}{\pgfpointadd{\beforetopanchor\pgf@y-\pgf@y}{\centerpoint}}%
                \ifdim\externalangle pt>\pgfmathresult pt\relax%
                    \pgfmathrotatepointaround{%
                        \pgfmathpointintersectionoflineandarc%
                            {\pgfmathrotatepointaround{\externalpoint}{\centerpoint}{-\rotate}}%
                            {\pgfmathrotatepointaround{\referencepoint}{\centerpoint}{-\rotate}}%
                            {%
                                \beforetop%
                                \pgf@xa\pgf@x%
                                \centerpoint
                                \advance\pgf@x\pgf@xa%
                            }%
                            {270}{360}{\xradius and \yradius}%
                        }{\centerpoint}{\rotate}%
                \else%
                    \pgfpointintersectionoflines{%
                        \pgfmathrotatepointaround{\pgfpointadd{\afterbottomanchor\pgf@y-\pgf@y}{\centerpoint}}%
                            {\centerpoint}{\rotate}}{%
                        \pgfmathrotatepointaround{\pgfpointadd{\beforetopanchor\pgf@y-\pgf@y}{\centerpoint}}%
                            {\centerpoint}{\rotate}}%
                        {\referencepoint}{\externalpoint}%
                \fi%
            \else%
                \pgfmathrotatepointaround{%
                    \pgfmathpointintersectionoflineandarc%
                        {\pgfmathrotatepointaround{\externalpoint}{\centerpoint}{-\rotate}}%
                        {\pgfmathrotatepointaround{\referencepoint}{\centerpoint}{-\rotate}}%
                        {%
                            \afterbottom%
                            \pgf@xa\pgf@x%
                            \centerpoint
                            \advance\pgf@x\pgf@xa%
                        }%
                        {90}{270}{\xradius and \yradius}%
                    }{\centerpoint}{\rotate}%
            \fi%
        \fi%
    }%
}%
