Tak vám jednou byl jeden chudý pán neboli člověk, on se vlastně jmenoval
František Král, ale tak mu říkali jenom tehdy, když ho sebral strážník a
dovedl ho na komisařství pro potulku, kde ho zapsali do takové tlusté knihy
a nechali ho přespat na pryčně a ráno ho zase poslali dál; na policii mu
tedy říkali František Král, ale ostatní lidé ho jmenovali všelijak jinak:
ten vandrák, ten šupák, ten tulák, ten pobuda, ten lajdák, ten otrapa, ten
hadrník, ten trhan, ten obejda, ten lenoch, ten chudák, ten jindyvyjduum,
ten člověk, ten kdovíkdo, ten poběhlík, ten krajánek, ten štvanec, ten
vošlapa, ten revertent, ten vagabund, ten darmošlap, ten budižkničemu, ten
hlad, ten hunt, ten holota a ještě mnoho jiných jmen mu dávali; kdyby každé
to názvisko platilo aspoň korunu, mohl by si za ně koupit žluté boty a možná
že i klobouk, ale takhle si za to nekoupil nic a měl jenom to, co mu lidé
dali. 
