%%%%%%%%%%%%%%%%%%%%%%%%%%%%%%%%%%%%%%%%%%%%%%%%%%%%%%%%%%%%%%%%%%%%%%%%%%%%%%
%
% Module:    ZzTeX Indexing Facilities, version 1
%
% Synopsis:  This file contains the tools for producing an index.
%
% Author:    Paul C. Anagnostopoulos
% Created:   20 May 2006
%
% Copyright 1989--2020 by Paul C. Anagnostopoulos
% under The MIT License (opensource.org/licenses/MIT)
%
%%%%%%%%%%%%%%%%%%%%%%%%%%%%%%%%%%%%%%%%%%%%%%%%%%%%%%%%%%%%%%%%%%%%%%%%%%%%%%%%

%                       Index Version
%                       ----- -------


\def \indexversion #1{%
  \input zzindexv#1\relax}

%                       Define Index Locator Types
%                       ------ ----- ------- -----


% This list contains the definition of each locator. New ones must be
% added to the end, and any program looking at the definitions must assume
% that duplicates may exist, and that the last one prevails.

\def \zixloclist {}


%~ This command is deprecated in favor of |\definelocator|.

\def \definelocatortype #1#2#3#4#5#6{%  {name}{attributes}{precedence}
                                     %  {beginning-punc}{templates}
                                     %  {ending-punc}
  {\setflag \pg = \true
   \let \page = \relax
   \def \nopage {\setflag \pg = \false}%
   \setflag \txt = \false
   \def \text {\setflag \txt = \true}%
   \let \notext = \relax
   #2%
   \def \bpunc {#4}%
   \def \temp {#5}%
   \def \epunc {#6}%
   \edef \znext {%
     \noexpand\append{{#1}{\pg}{\txt}{#3}%
                      {\expandafter\zdefof \meaning\bpunc\zmark}%
                      {\expandafter\zdefof \meaning\temp\zmark}%
                      {\expandafter\zdefof \meaning\epunc\zmark}}}%
   \znext{\zixloclist}%
   \if \txt
     \withname\gdef{\x#1}##1##2{\zixindex{#1}{##1}{S}{##2}}%
     \if \pg
       \withname\gdef{\x#1begin}##1##2{\zixindex{#1}{##1}{B}{##2}}%
       \withname\gdef{\x#1end}##1##2{\zixindex{#1}{##1}{E}{##2}}%
     \fi
   \else
     \withname\gdef{\x#1}##1{\zixindex{#1}{##1}{S}{}}%
     \if \pg
       \withname\gdef{\x#1begin}##1{\zixindex{#1}{##1}{B}{}}%
       \withname\gdef{\x#1end}##1{\zixindex{#1}{##1}{E}{}}%
     \fi
   \fi}}


% This command is deprecated.

\def \defineindexlocator #1#2#3#4#5#6{% {name}{attributes}{precedence}
                                      % {punctuation}{single-template}
                                      % {range-template}
}

%                       Builtin Locator Types
%                       ------- ------- -----


% Precedences 0--1 and 6--9 are reserved for specific books.

\definelocatortype{term}{\page\notext}{2.3}
                           {, |}{\bold{~P}|\bold{~P--~Q}|, }{. |}
\definelocatortype{biblio}{\page\notext}{2.5}
                           {, |}{~P|~P--~Q|, }{. |}
\definelocatortype{cite}{\page\notext}{2.5}
                           {, |}{~P|~P--~Q|, }{. |}
\definelocatortype{page}{\page\notext}{2.5}
                           {, |}{~P|~P--~Q|, }{. |}
\definelocatortype{figure}{\page\notext}{2.7}
                           {, |}{~P\+\emph{f}|~P--~Q\+\emph{f}|, }{. |}
\definelocatortype{glossary}{\page\notext}{2.7}
                           {, |}{~P\+\emph{g}|~P--~Q\+\emph{g}|, }{. |}
\definelocatortype{note}{\page\notext}{2.8}
                           {, |}{~P\+\emph{n}|~P--~Q\+\emph{n}|, }{. |}
\definelocatortype{sidebar}{\page\notext}{2.7}
                           {, |}{~P\+\emph{s}|~P--~Q\+\emph{s}|, }{. |}
\definelocatortype{table}{\page\notext}{2.7}
                           {, |}{~P\+\emph{t}|~P--~Q\+\emph{t}|, }{. |}
\definelocatortype{see}{\nopage\text}{3}
                           {. |\See\ }{~T||; }{. |}
\definelocatortype{seealso}{\nopage\text}{4}
                           {. |\Seealso\ }{~T||; }{. |}
\definelocatortype{consult}{\nopage\text}{5}
                           {. |\Consult\ }{~T||; }{. |}

%                       Index Preparation File
%                       ----- ----------- ----


% The \produceindex command must appear before the \document command.

\setflag \zindexing = \false


\def \produceindex #1#2{%                               {type}{prep-file}
  {\processdesign{\index}{#1}%
   \immediate\openout \zwritea = #2.zzh\relax
   \immediate\write \zwritea {root \jobname}%
   \immediate\write \zwritea {type #1}%
   \immediate\write \zwritea {divisions {\the\divisions}}%
   \immediate\write \zwritea {locators {\the\locators}}%
   \immediate\write \zwritea {\the\indexparams}%
   \maplist{\immediate\write \zwritea {define-locator ##1}}{\zixloclist}%
   \immediate\closeout \zwritea}%
  \global\setflag \zindexing = \true}

%                       Index File Handling
%                       ----- ---- --------

\definewrite{\zixrootfile}
\definewrite{\zixdivfile}
\setflag \zixdivopen = \false


\def \zindexinit {%
  \if \zindexing
    \immediate\openout \zixrootfile = \jobname.zzi\relax
    \immediate\write \zixrootfile 
      {\zcomment Raw index entry file for book `\jobname'.}%
  \fi}

\def \zindexfinal {%
  \if \zindexing
    \immediate\write \zixrootfile {\zcomment [end]}%
    \immediate\closeout \zixrootfile
  \fi}

\def \zindexdivinit #1{%                                {division}
  \if \zindexing
    \immediate\openout \zixdivfile = #1.zzi\relax
    \immediate\write \zixdivfile
      {\zcomment Raw index entry file for book `\jobname', division `#1'.}%
    \global\setflag \zixdivopen = \true
  \fi}

\def \zindexdivfinal {%
  \if \zixdivopen
    \immediate\write \zixdivfile {\zcomment [end]}%
    \immediate\closeout \zixdivfile
    \global\setflag \zixdivopen = \false
  \fi}

%                       Generate Index Entry
%                       -------- ----- -----


% This macro is called from the indexing macros generated by
% \definelocatortype.

\def \zixindex #1#2#3#4{%                  {locator}{level}{page-code}{text}
  \tcounta = #2\relax
  \if \notp{\rangep{\tcounta}{1}{3}}%
    \error{invindexlvl}{Invalid index level '#2' specified}%
    \tcounta = 1
  \fi
  \name{\zixindex\romannumeral\tcounta}{#1}{#2}{#3}{#4}}

\def \zixindexi #1#2#3#4#5{%
  \zixindexw{#1}{#2}{#3}{#4}{#5}{}{}{\noexpand\folio}}
\def \zixindexii #1#2#3#4#5#6{%
  \zixindexw{#1}{#2}{#3}{#4}{#5}{#6}{}{\noexpand\folio}}
\def \zixindexiii #1#2#3#4#5#6#7{%
  \zixindexw{#1}{#2}{#3}{#4}{#5}{#6}{#7}{\noexpand\folio}}

\catcode"B8 = \catother

\gdef \zixindexw #1#2#3#4#5#6#7#8{% {locator}{level}{page-code}{text}
                                  % {heading1}{heading2}{heading3}{page}
  \if \notp{\zindexing}%
    \warning{notindexing}{An indexing command was encountered,
                          but no \string\produceindex}%
  \fi
  \zbeginhidewrite
    \def \zhdx {#5}%
    \def \zhdy {#6}%
    \def \zhdz {#7}%
    \def \ztxt {#4}%
    \edef \znext {#1^^b8%
                  #2^^b8%
                  \expandafter\zdefof \meaning\zhdx\zmark^^b8%
                  \expandafter\zdefof \meaning\zhdy\zmark^^b8%
                  \expandafter\zdefof \meaning\zhdz\zmark^^b8%
                  #3^^b8%
                  #8^^b8%
                  \expandafter\zdefof \meaning\ztxt\zmark}%
    \write \if \zixdivopen \zixdivfile \else \zixrootfile \fi
      \expandafter{\znext}%
    \if \vmodep \repeatpenalty \fi      % In case glue follows write.
  \zendhidewrite}

\catcode"B8 = \catinvalid

%                       Discussion of Index Formatting
%                       ---------- -- ----- ----------


% An index is formatted inside an index block, which is more or less
% like a text block. Within the index block are occurences of the
% \indexentry macro, one for each entry. In addition, the
% \alphabreak macro can be used to produce breaks preceding the `A'
% entries, `B' entries, etc.
%
% In an index block, parts 4 and 5 of the mark are used to remember
% index entries so they can be carried forward to subsequent columns
% or pages. The mark is used as follows:
%
%   Part  Subpart  Usage
%     4      1     Boolean, true if in the midst of a first-level entry.
%     4      2     The first-level entry text.
%     5      1     Boolean, true if in the midst of a second-level entry.
%     5      2     The second-level entry text.

%                       Index Block
%                       ----- -----


\defineblock{\index}{\endindex}{\false}{}

%~block index Type
% \def \alphaformat ##1{...}            % Alpha break formatter.
% \bodyfont = {\size\style}             % Body font.
% \clubpenalty = penalty                % Club penalty is often different.
% \columngutter = glue                  % Column gutter.
% \divisions = {...}                    % Which divisions to include.
% \leftindent = glue                    % Left indentation.
% \locators = {...}                     % Which locators to include.
% \indexparams = {...}                  % Parameters for index generator.
% \parhang = glue                       % Paragraph hang.
% \parindent = dimen                    % Paragraph indent.
% \parrag = dimen                       % Paragraph raggedness.
% \parskip = glue                       % Paragraph skip. 
% \position = {...}                     % Position and number of columns.
% \rightindent = glue                   % Right indentation.
%~end

\definetoks{\locators}
\definetoks{\indexparams}

\definecount{\previndexlevel}{0}
\definecount{\zixcols}{0}


\def \index #1{%                                        {type}
  \blockcantbein{\index}{\index}%
  \beginblockscope{index}%
  \global\increment \indexdepth
  \clubpenalty = \breakallowed          %~default soft
  \parskip = 0pt                        %~default soft
  \processdesign{\index}{#1}%
  \global\increment \indexnumber
  \def \zixtype {#1}%
  \zixparsepos
  \indexformat}

\def \indexformat {%
  \the\bodyfont
  \alterindentation{\leftindent}{\rightindent}%
  \setparrag{\parrag}%
  \setpagecolumns{\zixpageuse}{\the\zixcols}{\columngutter}%
  \everycolumn = {\zixcarryover}%
  \begingroup
    \let \zfullcolumnbreak = \fullcolumnbreak
    \let \fullcolumnbreak = \indexbreak
    \let \shortcolumnbreak = \indexbreak
    \global\previndexlevel = 0\relax}

\def \endindex {%
    \endgraf                    % End last entry.
    \vbox{}%                    % Force start of column if not already.
  \endgroup
  \clearindexmarks
  \everycolumn = {}%
  \setpagecolumns{\usetextmeasure}{1}{0pt}%
  \global\decrement \indexdepth
  \endblockscope{index}}

\def \zixparsepos {%
  \zixcols = 1
  \let \oneup = \relax
  \def \twoup {\zixcols = 2}%
  \def \threeup {\zixcols = 3}%
  \def \zixpageuse {\usetextmeasure}%
  \let \textarea = \relax
  \def \typearea {\def \zixpageuse {\usetypewidth}}%
  \let \leftmargin = \typearea
  \the\position}

%                       Index Entries
%                       ----- -------


\setflag \zixbreak = \false


\def \indexentry #1#2#3{%                       level{heading}{locators}
  \endgraf
  \ifcase #1%
  \or                                           % Level 1 heading:
    \clearindexmarks
    \zixmaybebreak
    \hangindent = \parhang \hangafter = 1
    {\name{\index\zixtype entryiformat}{#2\setmarkinfo{4}{\true}{#2}}{#3}}%
%%%    \setmarkinfo{4}{\true}{#2}%
  \or                                           % Level 2 heading:
    \setmarkinfo{5}{\false}{}%
    \zixmaybebreak
    \hangindent = \parhang \hangafter = 1
    {\name{\index\zixtype entryiiformat}{#2\setmarkinfo{5}{\true}{#2}}{#3}}%
%%%    \setmarkinfo{5}{\true}{#2}%
  \or                                           % Level 3 heading:
    \zixmaybebreak
    \hangindent = \parhang \hangafter = 1
    {\name{\index\zixtype entryiiiformat}{#2}{#3}}%
  \fi
  \global\previndexlevel = #1\relax}

\def \alphabreak #1{%                           {text}
  \endgraf
  \clearindexmarks
  \zixmaybebreak
  \alphaformat{#1}%
  \endgraf}

\def \indexbreak {%
  \setflag \zixbreak = \true}

\def \clearindexmarks {%
  \setmarkinfo{4}{\false}{}%
  \setmarkinfo{5}{\false}{}}

\def \zixmaybebreak {%
  \if \zixbreak
    \zfullcolumnbreak
    \setflag \zixbreak = \false
  \fi}


% This macro is invoked at the beginning of each column to perform
% any first- and second-level entry carryovers. It only includes
% carryovers on the first column of versos.

\def \zixcarryover {%
  \if \andp{\evenpagep}{\firstcolumnp}%
    {\setflag \znext = {\markinfo{\bottommark}{4}{1}}%
     \if \znext
       \hangindent = \parhang \hangafter = 1
       \name{\index\zixtype carryiformat}{\markinfo{\bottommark}{4}{2}}%
       \setflag \znext = {\markinfo{\bottommark}{5}{1}}%
       \if \znext
         \hangindent = \parhang \hangafter = 1
         \name{\index\zixtype carryiiformat}{\markinfo{\bottommark}{5}{2}}%
       \fi
       \endgraf
       \tdimena = \zoutputcoldepth              % Compensate for depth of
       \advance \tdimena by -\prevdepth         % previous column v. depth
       \kern \tdimena                           % of carryover line(s).
     \fi}%
  \fi}
