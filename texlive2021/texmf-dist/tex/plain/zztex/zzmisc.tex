%%%%%%%%%%%%%%%%%%%%%%%%%%%%%%%%%%%%%%%%%%%%%%%%%%%%%%%%%%%%%%%%%%%%%%%%%%%%%%
%
% Module:    ZzTeX Miscellaneous Facilities
%
% Synopsis:  This module contains miscellaneous facilities that do not
%            seem to belong in any other file.
%
% Author:    Paul C. Anagnostopoulos
% Created:   26 August 1989
%
% Copyright 1989--2020 by Paul C. Anagnostopoulos
% under The MIT License (opensource.org/licenses/MIT)
%
%%%%%%%%%%%%%%%%%%%%%%%%%%%%%%%%%%%%%%%%%%%%%%%%%%%%%%%%%%%%%%%%%%%%%%%%%%%%%%%%

%                       Character Categories
%                       --------- ----------


% The following list contains characters with category codes other
% than endofline, ignored, space, letter, other, and invalid:

\setlist \zcatlist = {}


\def \establishcatcode #1#2{%                           {`\c}{\catname}
  \zdelcat{#1}%
  \catcode #1 = #2%
  {\setflag \znext = \true
   \if \eqlp{#2}{\catendofline}\setflag \znext = \false \fi
   \if \eqlp{#2}{\catignored}\setflag \znext = \false \fi
   \if \eqlp{#2}{\catspace}\setflag \znext = \false \fi
   \if \eqlp{#2}{\catletter}\setflag \znext = \false \fi
   \if \eqlp{#2}{\catother}\setflag \znext = \false \fi
   \if \eqlp{#2}{\catinvalid}\setflag \znext = \false \fi
   \if \znext \append{#1}{\zcatlist}\fi}}

\def \zdelcat #1{%
  {\setlist \znewcatlist = {}%
   \maplist{\def \zcata {##1}\def \zcatb{#1}%
            \if \notp{\tokeqlp{\zcata}{\zcatb}}\append{##1}{\znewcatlist}\fi}%
           {\zcatlist}%
   \setlist \zcatlist = \znewcatlist}}

% Initialize the character list:

\establishcatcode{`\^^L}{\catactive}
\establishcatcode{`\#}{\catparameter}
\establishcatcode{`\$}{\catmath}
\establishcatcode{`\&}{\catalign}
\establishcatcode{`\%}{\catcomment}
\establishcatcode{`\@}{\catactive}
\establishcatcode{`\\}{\catescape}
\establishcatcode{`\^}{\catsuperscript}
\establishcatcode{`\_}{\catsubscript}
\establishcatcode{`\{}{\catbegin}
\establishcatcode{`\}}{\catend}
\establishcatcode{`\~}{\catactive}

\def \uncatcode #1#2#3{%      {allow-commands}{allow-at-commands}{allow-math}
  \maplist{\catcode##1=\catother}{\zcatlist}%
  \if #1%
    \catcode`\\ = \catescape
    \catcode`\{ = \catbegin
    \catcode`\} = \catend
  \fi
  \if #2%
    \catcode`\@ = \catactive
  \fi
  \if #3%
    \catcode`\$ = \catmath
    \catcode`\^ = \catsuperscript
    \catcode`\_ = \catsubscript
  \fi}

\def \catcodemath {%
  \catcode `\  = \catspace
  \catcode `\$ = \catmath
  \catcode `\^ = \catsuperscript
  \catcode `\_ = \catsubscript}

\def \catcoderange #1#2#3{%                     {first}{last}{\catname}
  \tcounta = #1\relax
  \tcountb = #2\relax \increment \tcountb
  \loop
    \ifnum \tcounta < \tcountb
      \catcode \tcounta = #3%
      \increment \tcounta
  \repeat}

% Make the upper half of the ASCII characters invalid.

\catcoderange{"80}{"FF}{\catinvalid}

%                       Conditional Text
%                       ----------- ----


\let \setif = \if
\let \endsetif = \fi

\def \ignore {%
  \if \notp{\zemptyblockstackp}\todo{$\ignore command.}\fi
  \setif \false}

\let \endignore = \endsetif

%                       Date & Time
%                       ---- - ----


% \year, \month, and \day are TeX parameters.

\definecount{\weekday}{0}

% \time is a TeX parameter, the number of minutes since midnight.

\definecount{\hour}{0}
\definecount{\minute}{0}


\def \zdtinit {%
  \zwdinit
  \hour = \time
  \divide \hour by 60
  \minute = \time
  \divide \minute by 60
  \multiply \minute by -60
  \advance \minute by \time}

\def \zwdinit {%
  \tcounta = \year
  \if \lssp{\month}{3}\advance \tcounta by -1 \fi
  \tcountb = \tcounta
  \divide \tcountb by 100
  \tcountc = \tcountb
  \multiply \tcountc by -100
  \advance \tcountc by \tcounta
  \tcounta = \month
  \if \lssp{\month}{3}\advance \tcounta by 9 \else \advance \tcounta by -3 \fi
  \multiply \tcountb by 146097
  \divide \tcountb by 4
  \multiply \tcountc by 1461
  \divide \tcountc by 4
  \advance \tcountb by \tcountc
  \multiply \tcounta by 153
  \advance \tcounta by 2
  \divide \tcounta by 5
  \advance \tcountb by \tcounta
  \advance \tcountb by \day
  \advance \tcountb by -693902
  \weekday = \tcountb
  \divide \weekday by 7
  \multiply \weekday by -7
  \advance \weekday by \tcountb}


\def \formattime {%
  \number\hour:%
  \if \lssp{\minute}{10}0\fi
  \number\minute}

%                       Fancy Names
%                       ----- -----


\def \AMSTeX {AMS\TeX}

\def \BibTeX {Bib\TeX}

\def \LaTeX {L\kern-.3em \raise .4ex \hbox{a}\kern-.05em \TeX}

%                       Hiding Writes
%                       ------ ------


% These macros help to hide the \write's that are performed by commands
% like \tag and \xpage.

\definedimen{\zhidewriteskip}{0pt}

\def \zbeginhidewrite {%
  \zhidewriteskip = \lastskip
  \begingroup
    \if \hmodep \edef \zsf {\spacefactor=\the\spacefactor}\fi}

\def \zendhidewrite {%
    \if \hmodep \zsf \fi
  \endgroup
  \if \andp{\hmodep}{\dimposp{\zhidewriteskip}}\ignorespaces \fi}

%                       Initialization File
%                       -------------- ----


\def \zloadinifile {%
  \gdef \inicompositor {???}%
  \gdef \iniinitials {???}%
  \gdef \ininame {???}
  \gdef \iniorganization {(none)}%
  \checkfile{\zini}{zztex.ini}%
  \if \zini
    \writelog{(loading ini file)}%
    \zloadinifilea
  \else
    \writelog{(no ini file)}%
  \fi}

\def \zloadinifilea {%
  \begingroup
    \catcode \endlinechar = \catcomment
    \catcode `\; = \catcomment
    \openin \zreada zztex.ini\relax
    \setflag \zinireg = \false
    \loop
      \read \zreada to \zline
      \if \notp{\emptydefp{\zline}}\expandafter\zloadinifileb \zline\zmark \fi
    \if \notp{\eofp{\zreada}}\repeat
    \closein \zreada
  \endgroup}

\def \zloadinifileb #1#2\zmark {%
  \if \codeeqlp{#1}{[}%
    \zloadinifilec #2\zmark
  \else\if \zinireg
    \zloadinifiled #1#2\zmark
  \fi\fi}

\def \zloadinifilec #1]#2\zmark {%
  \stringeql{\zinireg}{#1}{registration}}

\def \zloadinifiled #1=#2\zmark {%
  \withname\gdef {\ini#1}{#2}}

%                       Penalties
%                       ---------


% The following symbolic penalties should be used everywhere so they
% can be adjusted without any recoding.  The trailing spaces are
% like \relax but don't screw up \ifnum.

\def \breaknever   {10000 }     %~ Never break here. %^penalty
\def \breakworst   {500 }       %~ Worst place to break. %^penalty
\def \breakworse   {300 }       %~ Worse place to break. %^penalty
\def \breakbad     {100 }       %~ Bad place to break. %^penalty
\def \breakallowed {0 }         %~ Break allowed here. %^penalty
\def \breakgood    {-100 }      %~ Good place to break. %^penalty
\def \breakbetter  {-300 }      %~ Better place to break. %^penalty
\def \breakbest    {-500 }      %~ Best place to break. %^penalty
\def \breakalways  {-10000 }    %~ Always break here. %^penalty


\def \allowbreak {\penalty \breakallowed}       %~ Allow page break here.
\def \break      {\penalty \breakalways}        %~ Force page break here.
\def \nobreak    {\penalty \breaknever}         %~ Disallow page break here.
\let \unbreakable = \nobreak

%                       Versions
%                       --------


% I have no idea what my intention was here.


\def \defineversion #1#2{%                              {\name}{status}
  \if #2%
    \let \znext = \zletiftrue
  \else
    \let \znext = \zletiffalse
  \fi
  \withname\let {\end \expandafter\discardtok\string#1} = \fi
  \znext{#1}}

\def \zletiftrue #1{%
  \let #1 = \iftrue}

\def \zletiffalse #1{%
  \let #1 = \iffalse}


\defineversion{\ignore}{\false}

%                       Translator Flags
%                       ---------- -----


\def \XTran<#1>{%
  \error{xtran}{XTran flag: #1}}

\def \ZzTran[#1]{%
  \error{zztran}{ZzTran flag: #1}}
