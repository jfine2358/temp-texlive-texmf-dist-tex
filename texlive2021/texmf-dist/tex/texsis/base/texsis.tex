%% file: texsis.tex - TeX macros for Physicists - TeXsis version 2.18
%% @(#) $Id: texsis.tex,v 18.1 2001/04/06 22:52:02 myers Exp $
%======================================================================*
% TeXsis - TeX Macros for Physicists   -- by Eric Myers and Frank E. Paige
%
%     This is version 2.18 of TeXsis, a collection of TeX macros
%  for physicists (and anyone else who finds them useful).
%
%     The source code for this format is contained in this file and
% in the files TXS*.tex read by it.  The documentation is contained
% in the files TXS*.doc.  To produce a printed version of the
% documentation, run TeXsis on TXSdoc.doc (or run Manual.tex through
% plain tex).
%
% You can always get the latest version of TeXsis from
%               ftp://ftp.texsis.org/texsis
%
% Eric Myers,  Department of Physics, University of Michigan, Ann Arbor
% Frank Paige, Physics Department, Brookhaven National Laboratory
% ----------
% (C) Copyright 1988-2001 by Eric Myers and Frank E. Paige.
% This file is a part of TeXsis.  Distribution and/or modifications
% are allowed under the terms of the LaTeX Project Public License (LPPL).
% See the file COPYING or ftp://ftp.texsis.org/texsis/LPPL
% THIS SOFTWARE IS PROVIDED ``AS IS'' AND WITHOUT ANY EXPRESS OR
% IMPLIED WARRANTIES, INCLUDING, WITHOUT LIMITATION, THE IMPLIED
% WARRANTIES OF MERCHANTIBILITY AND FITNESS FOR A PARTICULAR PURPOSE.
%======================================================================*
%  This is the master file, which brings together all of the components
%  of TeXsis.  They should all be in the same directory.  This master file
%  can be used to generate TeXsis as a pre-loaded format using initex:
%       initex texsis              (or initex &plain texsis)
%  followed by "\dump".  See The TeXbook for more on INITEX and \dump.
%
%  First check to see if the PLAIN format is pre-loaded.  
%  If \plainoutput is undefined then it is not, so read in plain.tex.
%
\ifx\plainoutput\undefined\input plain.tex\fi

%---------------------------------*
\def\Input #1 {\def\filename{#1}\input #1}      % \Input someday...?
\def\TeXsis{\TeX sis}%                          % the TeXsis logo

%---------------------------------*
%  Fonts: TXSfonts contains the fonts which (almost) all TeX
%  installations should have, but there may be some changes necessary
%  for your own instalation.  It's not a good idea to just replace these
%  though, because there is other info there too.  Read the comments in
%  the file TXSfonts for further info. 
%
\input TXSfonts         % 12, 10, 7, 5, 14, 16 pt fonts, etc...

%---------------------------------*
% Macros: different "packages" are kept in separate files
% 
\input TXSmacs          % main macros common to all components
\input TXSinit          % initialization stuff and some basic macros
\input TXShead          % running headlines and footlines
\input TXSeqns          % equation numbering 
\input TXSprns          % automatic parentheses sizing and balancing
\input TXSrefs          % references and citations
\input TXSsects         % chapter and section divisions
\input TXStags          % tags for ref and eqn numbers, \label-ing
\input TXStitle         % title page macros for physics papers
\input TXSenvmt         % center and flush environments, etc..
\input TXSfmts          % document layouts, \paper, \preprint, etc
\input TXSfigs          % figure, tables, lists of figures
\input TXSruled         % ruled tables
\input TXSdcol          % double column format
\input TXSletr          % letter format
\input TXSmemo          % memos and referee reports
\input TXSconts         % table of contents macros
\input TXSsymb          % extended math symbols for physics

%---------------------------------*
%       Patches and Changes: The two files named below will be read 
% AT RUN TIME by \texsis, if they exist.  See the file TXSinit.tex 
% for more info.
%
\def\TeXsisLib{}% use this to specify an alternate path to where code is kept
\def\TXSpatches{\TeXsisLib TXSpatch}%       % global run-time patches
\def\TXSmods{TXSmods}%                      % local or personal modifications

%---------------------------------*
%       Site dependent info:  See if there is a site info file out
% there called TXSsite.tex.  If so, read it in.  This happens only once 
% when running INITEX to create the format file texsis.fmt.

\LoadSiteFile

%---------------------------------*
\pageno = 1                             % start on page 1
\tenpoint                               % as in Plain TeX
\singlespaced                           % default until \texsis

\def\fmtname{TeXsis}\def\fmtversion{2.18}% 
\def\revdate{21 April 2001}%   

\everyjob={% This stuff is done every job...
  \emsg{\fmtname\space version \fmtversion\space (\revdate) format preloaded.}%
  \SetDate                              % set \adate and \edate
  \ReadPatches                          % read patch file, if it exists
  \ReadAUX                              % read .aux file, if it exists
  \let\filename=\jobname                % for \Input, for starters
  \colwidth=\hsize                      % default column width
  }
\ATlock                                 % lock @ macros
\emsg{\fmtname\space version \fmtversion\space (\revdate) loaded.}%

%%>>> EOF texsis.tex <<<
