\छायांग{पुं}{पुल्लिंग} % Masculine
\छायांग{स्त्री}{स्त्रीलिंग} % Feminine
\छायांग{नपुं}{नपुंसकलिंग} % Neuter
\छायांग{१}{प्रथम व्यक्ती} % First person
\छायांग{२}{द्वितीय व्यक्ती} % Second person
\छायांग{३}{तृतीय व्यक्ती} % Third person
\छायांग{एव}{एकवचन} % Singular
\छायांग{द्विव}{द्विवचन} % Dual
\छायांग{त्रिव}{त्रिव} % Trial 
\छायांग{अव}{अल्पवचन} % Paucal
\छायांग{बव}{बहुवचन} % Plural
\छायांग{अवि}{अभिधानपर विभक्ती} % Nominative
\छायांग{कर्मवि}{कर्मपर विभक्ती} % Accusative
\छायांग{सा}{साधनपर विभक्ती} % Instrumental
\छायांग{दावि}{दानपर विभक्ती} % Dative
\छायांग{वियो}{वियोगपर विभक्ती} % Ablative
\छायांग{संयो}{संबंधयोजक विभक्ती} % Genitive
\छायांग{अधि}{अधिकरण विभक्ती} % Locative
\छायांग{संबो}{संबोधन विभक्ती} % Vocative
\छायांग{साह}{साहचर्यदर्शक विभक्ती} % Associative
\छायांग{कवि}{कर्तृत्वपर विभक्ती} % Ergative
\छायांग{आवि}{आगत विभक्ती} % Oblique
\छायांग{साक्रि}{साहाय्यक क्रियापद} % Auxiliary
\छायांग{गणक}{गणक} % Counter
\छायांग{भूत}{भूतकाळ} % Past
\छायांग{वर्त}{वर्तमान काळ} % Present
\छायांग{भवि}{भविष्यकाळ} % Future
\छायांग{पू}{पूर्ण} % Perfective
\छायांग{अपू}{अपूर्ण} % Imperfective
\छायांग{नि}{नित्य} % Habitual
\छायांग{अखं}{अखंडित} % Continuous
\छायांग{क्र}{क्रमिक} % Progressive
\छायांग{अक्र}{अक्रमिक} % Non-progressive