% Licence    : Released under the LaTeX Project Public License v1.3c
% or later, see http://www.latex-project.org/lppl.txtf
\newcommand{\EquaBase}[5][]{%type ax=d ou b=cx
  \useKVdefault[ClesEquation]%
  \setKV[ClesEquation]{#1}%
  \ifx\bla#2\bla%on teste si le paramètre #2 est vide:
  % si oui, on est dans le cas b=cx. Eh bien on échange :)
  % Mais attention si les deux paramètres a et c sont vides...
  \EquaBase[#1]{#4}{}{}{#3}
  \else
  % si non, on est dans le cas ax=d
  \xintifboolexpr{#2=0}{%
    \xintifboolexpr{#5=0}{%
      L'équation $0\useKV[ClesEquation]{ELettre}=0$ a une infinité de solutions.}{L'équation $0\useKV[ClesEquation]{Lettre}=\num{#5}$ n'a aucune solution.}%
  }{%\else
    \xintifboolexpr{#5=0}{L'équation $\num{#2}\useKV[ClesEquation]{Lettre}=0$ a une unique solution : $\useKV[ClesEquation]{Lettre}=0$.}{%\else
      \begin{align*}%
        \tikzmark{A-\theNbequa}\xintifboolexpr{#2=1}{\useKV[ClesEquation]{Lettre}}{\num{#2}\useKV[ClesEquation]{Lettre}}&=\num{#5}\tikzmark{C-\theNbequa}\\
        \tikzmark{B-\theNbequa}\useKV[ClesEquation]{Lettre}&=\frac{\num{#5}}{\num{#2}}\tikzmark{D-\theNbequa}%\\
        \ifboolKV[ClesEquation]{Fleches}{%
        \leftcomment{A-\theNbequa}{B-\theNbequa}{A-\theNbequa}{$\div\xintifboolexpr{#2<0}{(\num{#2})}{\num{#2}}$}%
        \rightcomment{C-\theNbequa}{D-\theNbequa}{D-\theNbequa}{$\div\xintifboolexpr{#2<0}{(\num{#2})}{\num{#2}}$}%
        }{%
        \ifboolKV[ClesEquation]{FlecheDiv}{%
        \Leftcomment{A-\theNbequa}{B-\theNbequa}{A-\theNbequa}{$\div\xintifboolexpr{#2<0}{(\num{#2})}{\num{#2}}$}%
        \Rightcomment{C-\theNbequa}{D-\theNbequa}{D-\theNbequa}{$\div\xintifboolexpr{#2<0}{(\num{#2})}{\num{#2}}$}%
        }{}%
        }%%
        \ifboolKV[ClesEquation]{Entier}{%
        \SSimpliTest{#5}{#2}%
        \ifboolKV[ClesEquation]{Simplification}{%
        \ifthenelse{\boolean{Simplification}}{\\\useKV[ClesEquation]{Lettre}&=\SSimplifie{#5}{#2}}{}%\\
        }{}
        }{}
        \ifboolKV[ClesEquation]{Fleches}{%
        \stepcounter{Nbequa}}%
        {\ifboolKV[ClesEquation]{FlecheDiv}{\stepcounter{Nbequa}}{}
        }
      \end{align*}
      \ifboolKV[ClesEquation]{Solution}{L'équation $\xintifboolexpr{#2=1}{\useKV[ClesEquation]{Lettre}=\num{#5}}{\num{#2}\useKV[ClesEquation]{Lettre}=\num{#5}}$ a une unique solution : $\displaystyle\ifboolKV[ClesEquation]{LettreSol}{\useKV[ClesEquation]{Lettre}=}{}\opdiv*{#5}{#2}{numequa}{resteequa}\opcmp{resteequa}{0}\ifopeq\opexport{numequa}{\numequa}\num{\numequa}\else\ifboolKV[ClesEquation]{Simplification}{\SSimplifie{#5}{#2}}{\frac{\num{#5}}{\num{#2}}}\fi$.%
      }{}
    }
  }
  \fi
}

\newcommand{\EquaDeuxSoustraction}[5][]{%type ax+b=d ou b=cx+d$
  \useKVdefault[ClesEquation]%
  \setKV[ClesEquation]{#1}%
  \ifx\bla#2\bla%On échange en faisant attention à ne pas boucler : c doit être non vide
    \EquaDeuxSoustraction[#1]{#4}{#5}{#2}{#3}
  \else%cas ax+b=d
    \xintifboolexpr{#2=0}{%
      \xintifboolexpr{#3=#5}{%b=d
        L'équation $\num{#3}=\num{#5}$ a une infinité de solutions.}%
      {%b<>d
        L'équation $\num{#3}=\num{#5}$ n'a aucune solution.%
      }%
    }{%ELSE
      \xintifboolexpr{#3=0}{%ax+b=d
        \EquaBase[#1]{#2}{}{}{#5}%
      }{%ax+b=d$ Ici
        \begin{align*}
          \tikzmark{A-\theNbequa}\xintifboolexpr{#2=1}{}{\num{#2}}\useKV[ClesEquation]{Lettre}\xintifboolexpr{#3>0}{+\num{#3}}{-\num{\fpeval{0-#3}}}&=\num{#5}\tikzmark{E-\theNbequa}\\
          \ifboolKV[ClesEquation]{Decomposition}{%
          \xintifboolexpr{#2=1}{}{\num{#2}}\useKV[ClesEquation]{Lettre}\xintifboolexpr{#3>0}{+\num{#3}}{-\num{\fpeval{0-#3}}}\mathcolor{Cdecomp}{\xintifboolexpr{#3>0}{-\num{#3}}{+\num{\fpeval{0-#3}}}}&=\num{#5}\mathcolor{Cdecomp}{\xintifboolexpr{#3>0}{-\num{#3}}{+\num{\fpeval{0-#3}}}}\\
          }{}%
          \tikzmark{C-\theNbequa}\xdef\Coeffa{#2}\xdef\Coeffb{\fpeval{#5-#3}}\xintifboolexpr{\Coeffa=1}{}{\num{\Coeffa}}\useKV[ClesEquation]{Lettre}&=\num{\Coeffb}\tikzmark{G-\theNbequa}
          \ifboolKV[ClesEquation]{Decomposition}{\\\xintifboolexpr{\Coeffa=1}{}{\frac{\num{\Coeffa}}{\mathcolor{Cdecomp}{\num{\Coeffa}}}\useKV[ClesEquation]{Lettre}&=\frac{\num{\Coeffb}}{\mathcolor{Cdecomp}{\num{\Coeffa}}}}}{}
          \xintifboolexpr{\Coeffa=1}{}{\\}
          \ifboolKV[ClesEquation]{Fleches}{%
          \leftcomment{A-\theNbequa}{C-\theNbequa}{A-\theNbequa}{$\xintifboolexpr{#3>0}{-\num{#3}}{+\num{\fpeval{0-#3}}}$}%
          \rightcomment{E-\theNbequa}{G-\theNbequa}{E-\theNbequa}{$\xintifboolexpr{#3>0}{-\num{#3}}{+\num{\fpeval{0-#3}}}$}%
          }{}
          \xintifboolexpr{\Coeffa=1}{% 
          }{%\ifnum\cmtd>1
          \tikzmark{D-\theNbequa}\useKV[ClesEquation]{Lettre}&=\frac{\num{\Coeffb}}{\num{\Coeffa}}\tikzmark{H-\theNbequa}%\\
          \ifboolKV[ClesEquation]{Fleches}{%
          \leftcomment{C-\theNbequa}{D-\theNbequa}{A-\theNbequa}{$\div\xintifboolexpr{\Coeffa<0}{(\num{\Coeffa})}{\num{\Coeffa}}$}%
          \rightcomment{G-\theNbequa}{H-\theNbequa}{E-\theNbequa}{$\div\xintifboolexpr{\Coeffa<0}{(\num{\Coeffa})}{\num{\Coeffa}}$}%
          }{%ICI ?
          \ifboolKV[ClesEquation]{FlecheDiv}{%
          \leftcomment{C-\theNbequa}{D-\theNbequa}{A-\theNbequa}{$\div\xintifboolexpr{\Coeffa<0}{(\num{\Coeffa})}{\num{\Coeffa}}$}%
          \rightcomment{G-\theNbequa}{H-\theNbequa}{E-\theNbequa}{$\div\xintifboolexpr{\Coeffa<0}{(\num{\Coeffa})}{\num{\Coeffa}}$}%
          }{}
          }
          }
          \ifboolKV[ClesEquation]{Entier}{%
          \SSimpliTest{\Coeffb}{\Coeffa}%
          \ifboolKV[ClesEquation]{Simplification}{%
          \ifthenelse{\boolean{Simplification}}{\\\useKV[ClesEquation]{Lettre}&=\SSimplifie{\Coeffb}{\Coeffa}}{}%\\
          }{}
          }{}
          \ifboolKV[ClesEquation]{Fleches}{\stepcounter{Nbequa}}{\ifboolKV[ClesEquation]{FlecheDiv}{\stepcounter{Nbequa}}{}}
        \end{align*}
        \ifboolKV[ClesEquation]{Solution}{L'équation $\xintifboolexpr{#2=1}{}{\num{#2}}\useKV[ClesEquation]{Lettre}\xintifboolexpr{#3>0}{+\num{#3}}{-\num{\fpeval{0-#3}}}=\num{#5}$ a une unique solution : \opdiv*{\Coeffb}{\Coeffa}{solution}{resteequa}\opcmp{resteequa}{0}$\ifboolKV[ClesEquation]{LettreSol}{\useKV[ClesEquation]{Lettre}=}{}\displaystyle\ifopeq\opexport{solution}{\solution}\num{\solution}\else\ifboolKV[ClesEquation]{Entier}{\SSimplifie{\Coeffb}{\Coeffa}}{\frac{\num{\Coeffb}}{\num{\Coeffa}}}\fi$.
        }{}
      }
    }
  \fi
}

\newcommand{\EquaTroisSoustraction}[5][]{%ax+b=cx ou ax=cx+d
  \useKVdefault[ClesEquation]%
  \setKV[ClesEquation]{#1}%
  \ifx\bla#3\bla%on inverse en faisant attention à la boucle #3<->#5
    \ifx\bla#5\bla%
      %% paramètre oublié
    \else
      \EquaTroisSoustraction[#1]{#4}{#5}{#2}{}%
    \fi
  \else
  \xintifboolexpr{#2=0}{%b=cx
    \EquaBase[#1]{#4}{}{}{#3}
  }{%
    \xintifboolexpr{#4=0}{%ax+b=0
      \EquaDeuxSoustraction[#1]{#2}{#3}{}{0}
      }{%ax+b=cx
        \xintifboolexpr{#2=#4}{%
          \xintifboolexpr{#3=0}{%ax=ax
            L'équation $\xintifboolexpr{#2=1}{}{\num{#2}}\useKV[ClesEquation]{Lettre}=\xintifboolexpr{#4=1}{}{\num{#4}}\useKV[ClesEquation]{Lettre}$ a une infinité de solutions.}%
          {%ax+b=ax
            L'équation $\xintifboolexpr{#2=1}{}{\num{#2}}\useKV[ClesEquation]{Lettre}\xintifboolexpr{#3>0}{+\num{#3}}{-\num{\fpeval{0-#3}}}=\xintifboolexpr{#4=1}{}{\num{#4}}\useKV[ClesEquation]{Lettre}$ n'a aucune solution.%
          }%
        }{%% Cas délicat
          \xintifboolexpr{#2>#4}{%ax+b=cx avec a>c
            \begin{align*}
              \tikzmark{A-\theNbequa}\xintifboolexpr{#2=1}{}{\num{#2}}\useKV[ClesEquation]{Lettre}\xintifboolexpr{#3>0}{+\num{#3}}{-\num{\fpeval{0-#3}}}&=\xintifboolexpr{#4=1}{}{\num{#4}}\useKV[ClesEquation]{Lettre}\tikzmark{E-\theNbequa}\\
                \ifboolKV[ClesEquation]{Decomposition}{%
                \xintifboolexpr{#2=1}{}{\num{#2}}\useKV[ClesEquation]{Lettre}\mathcolor{Cdecomp}{\xintifboolexpr{#4>0}{-\num{#4}\useKV[ClesEquation]{Lettre}}{+\num{\fpeval{0-#4}}\useKV[ClesEquation]{Lettre}}}\xintifboolexpr{#3>0}{+\num{#3}}{-\num{\fpeval{0-#3}}}&=\xintifboolexpr{#4=1}{}{\num{#4}}\useKV[ClesEquation]{Lettre}\mathcolor{Cdecomp}{\xintifboolexpr{#4>0}{-\num{#4}\useKV[ClesEquation]{Lettre}}{+\num{\fpeval{0-#4}}\useKV[ClesEquation]{Lettre}}}\\
                }{}
                \tikzmark{B-\theNbequa}\xdef\Coeffa{\fpeval{#2-#4}}\xintifboolexpr{\Coeffa=1}{}{\num{\Coeffa}}\useKV[ClesEquation]{Lettre}\xintifboolexpr{#3>0}{+\num{#3}}{-\num{\fpeval{0-#3}}}&=0\tikzmark{F-\theNbequa}\\
                \ifboolKV[ClesEquation]{Decomposition}{%
                \xintifboolexpr{\Coeffa=1}{}{\num{\Coeffa}}\useKV[ClesEquation]{Lettre}\xintifboolexpr{#3>0}{+\num{#3}}{-\num{\fpeval{0-#3}}}\mathcolor{Cdecomp}{\xintifboolexpr{#3>0}{-\num{#3}}{+\num{\fpeval{0-#3}}}}&=0\mathcolor{Cdecomp}{\xintifboolexpr{#3>0}{-\num{#3}}{+\num{\fpeval{0-#3}}}}\tikzmark{F-\theNbequa}\\
                }{}%
              \tikzmark{C-\theNbequa}\xdef\Coeffb{\fpeval{0-#3}}\xintifboolexpr{\Coeffa=1}{}{\num{\Coeffa}}\useKV[ClesEquation]{Lettre}&=\num{\Coeffb}\tikzmark{G-\theNbequa}%\\
              %eric
              \ifboolKV[ClesEquation]{Decomposition}{\\\xintifboolexpr{\Coeffa=1}{}{\frac{\num{\Coeffa}}{\mathcolor{Cdecomp}{\num{\Coeffa}}}\useKV[ClesEquation]{Lettre}&=\frac{\num{\Coeffb}}{\mathcolor{Cdecomp}{\num{\Coeffa}}}}}{}
              % eric
                \xintifboolexpr{\Coeffa=1}{}{\\}
                \ifboolKV[ClesEquation]{Fleches}{%
                \leftcomment{A-\theNbequa}{B-\theNbequa}{A-\theNbequa}{$\xintifboolexpr{#4>0}{-\num{#4}}{+\num{\fpeval{0-#4}}}\useKV[ClesEquation]{Lettre}$}
                \rightcomment{E-\theNbequa}{F-\theNbequa}{E-\theNbequa}{$\xintifboolexpr{#4>0}{-\num{#4}}{+\num{\fpeval{0-#4}}}\useKV[ClesEquation]{Lettre}$}
                \leftcomment{B-\theNbequa}{C-\theNbequa}{A-\theNbequa}{$\xintifboolexpr{#3>0}{-\num{#3}}{+\num{\fpeval{0-#3}}}$}%
                \rightcomment{F-\theNbequa}{G-\theNbequa}{E-\theNbequa}{$\xintifboolexpr{#3>0}{-\num{#3}}{+\num{\fpeval{0-#3}}}$}%
                }{}
                \xintifboolexpr{\Coeffa=1}{}{%\ifnum\cmtd>1
                \tikzmark{D-\theNbequa}\useKV[ClesEquation]{Lettre}&=\frac{\num{\Coeffb}}{\num{\Coeffa}}\tikzmark{H-\theNbequa}%\\
                \ifboolKV[ClesEquation]{Fleches}{%
                \leftcomment{C-\theNbequa}{D-\theNbequa}{A-\theNbequa}{$\div\xintifboolexpr{\Coeffa<0}{(\num{\Coeffa})}{\num{\Coeffa}}$}%
                \rightcomment{G-\theNbequa}{H-\theNbequa}{E-\theNbequa}{$\div\xintifboolexpr{\Coeffa<0}{(\num{\Coeffa})}{\num{\Coeffa}}$}%            
                }{
                \ifboolKV[ClesEquation]{FlecheDiv}{%
                \leftcomment{C-\theNbequa}{D-\theNbequa}{A-\theNbequa}{$\div\xintifboolexpr{\Coeffa<0}{(\num{\Coeffa})}{\num{\Coeffa}}$}%
                \rightcomment{G-\theNbequa}{H-\theNbequa}{E-\theNbequa}{$\div\xintifboolexpr{\Coeffa<0}{(\num{\Coeffa})}{\num{\Coeffa}}$}%                 
                }{}
                }
                \ifboolKV[ClesEquation]{Entier}{%
                \SSimpliTest{\Coeffb}{\Coeffa}%
                \ifboolKV[ClesEquation]{Simplification}{%
                \ifthenelse{\boolean{Simplification}}{\\\useKV[ClesEquation]{Lettre}&=\SSimplifie{\Coeffb}{\Coeffa}}{}%\\
                }{}
                }{}
                }
                \ifboolKV[ClesEquation]{Fleches}{\stepcounter{Nbequa}}{\ifboolKV[ClesEquation]{FlecheDiv}{\stepcounter{Nbequa}}{}}
              \end{align*}
              \ifboolKV[ClesEquation]{Solution}{L'équation $\xintifboolexpr{#2=1}{}{\num{#2}}\useKV[ClesEquation]{Lettre}\xintifboolexpr{#3>0}{+\num{#3}}{-\num{\fpeval{0-#3}}}=\xintifboolexpr{#4=1}{}{\num{#4}}\useKV[ClesEquation]{Lettre}$ a une unique solution : \opdiv*{\Coeffb}{\Coeffa}{solution}{resteequa}\opcmp{resteequa}{0}$\ifboolKV[ClesEquation]{LettreSol}{\useKV[ClesEquation]{Lettre}=}{}\displaystyle\ifopeq\opexport{solution}{\solution}\num{\solution}\else\ifboolKV[ClesEquation]{Entier}{\SSimplifie{\Coeffb}{\Coeffa}}{\frac{\num{\Coeffb}}{\num{\Coeffa}}}\fi$.}{}
            }{%ax+b=cx+d avec a<c              % Autre cas délicat
              \begin{align*}%
                \tikzmark{A-\theNbequa}\xintifboolexpr{#2=1}{}{\num{#2}}\useKV[ClesEquation]{Lettre}\xintifboolexpr{#3>0}{+\num{#3}}{-\num{\fpeval{0-#3}}}&=\xintifboolexpr{#4=1}{}{\num{#4}}\useKV[ClesEquation]{Lettre}\tikzmark{E-\theNbequa}\\
                \ifboolKV[ClesEquation]{Decomposition}{%
                \xintifboolexpr{#2=1}{}{\num{#2}}\useKV[ClesEquation]{Lettre}\mathcolor{Cdecomp}{\xintifboolexpr{#2>0}{-\num{#2}\useKV[ClesEquation]{Lettre}}{+\num{\fpeval{0-#2}}\useKV[ClesEquation]{Lettre}}}\xintifboolexpr{#3>0}{+\num{#3}}{-\num{\fpeval{0-#3}}}&=\xintifboolexpr{#4=1}{}{\num{#4}}\useKV[ClesEquation]{Lettre}\mathcolor{Cdecomp}{\xintifboolexpr{#2>0}{-\num{#2}\useKV[ClesEquation]{Lettre}}{+\num{\fpeval{0-#2}}\useKV[ClesEquation]{Lettre}}}\\
                }{}
                \tikzmark{B-\theNbequa}\xdef\Coeffb{#3}\xdef\Coeffa{\fpeval{#4-#2}}\xintifboolexpr{#3>0}{\num{#3}}{-\num{\fpeval{0-#3}}}&=\xintifboolexpr{\Coeffa=1}{}{\num{\Coeffa}}\useKV[ClesEquation]{Lettre}\tikzmark{F-\theNbequa}
                \xintifboolexpr{\Coeffa=1}{}{\\}
                \ifboolKV[ClesEquation]{Fleches}{%
                \leftcomment{A-\theNbequa}{B-\theNbequa}{A-\theNbequa}{$\xintifboolexpr{#2>0}{-\num{#2}}{+\num{\fpeval{0-#2}}}\useKV[ClesEquation]{Lettre}$}
                \rightcomment{E-\theNbequa}{F-\theNbequa}{E-\theNbequa}{$\xintifboolexpr{#2>0}{-\num{#2}}{+\num{\fpeval{0-#2}}}\useKV[ClesEquation]{Lettre}$}
                }{}
                % eric
                \ifboolKV[ClesEquation]{Decomposition}{\\\xintifboolexpr{\Coeffa=1}{}{\frac{\num{\Coeffb}}{\mathcolor{Cdecomp}{\num{\Coeffa}}}&=\frac{\num{\Coeffa}}{\mathcolor{Cdecomp}{\num{\Coeffa}}}\useKV[ClesEquation]{Lettre}}}{}
                % eric
                \xintifboolexpr{\Coeffa=1}{}{%\ifnum\cmtd>1
                \tikzmark{D-\theNbequa}\frac{\num{\Coeffb}}{\num{\Coeffa}}&=\useKV[ClesEquation]{Lettre}\tikzmark{H-\theNbequa}%\\
                \ifboolKV[ClesEquation]{Fleches}{%
                \leftcomment{B-\theNbequa}{D-\theNbequa}{A-\theNbequa}{$\div\xintifboolexpr{\Coeffa<0}{(\num{\Coeffa})}{\num{\Coeffa}}$}%
                \rightcomment{F-\theNbequa}{H-\theNbequa}{E-\theNbequa}{$\div\xintifboolexpr{\Coeffa<0}{(\num{\Coeffa})}{\num{\Coeffa}}$}%            
                }{
                \ifboolKV[ClesEquation]{FlecheDiv}{%
                \leftcomment{B-\theNbequa}{D-\theNbequa}{A-\theNbequa}{$\div\xintifboolexpr{\Coeffa<0}{(\num{\Coeffa})}{\num{\Coeffa}}$}%
                \rightcomment{F-\theNbequa}{H-\theNbequa}{E-\theNbequa}{$\div\xintifboolexpr{\Coeffa<0}{(\num{\Coeffa})}{\num{\Coeffa}}$}%                 
                }{}
                }
                \ifboolKV[ClesEquation]{Entier}{%
                \SSimpliTest{\Coeffb}{\Coeffa}%
                \ifboolKV[ClesEquation]{Simplification}{%
                \ifthenelse{\boolean{Simplification}}{\\\SSimplifie{\Coeffb}{\Coeffa}&=\useKV[ClesEquation]{Lettre}}{}%\\
                }{}
                }{}
                }
                \ifboolKV[ClesEquation]{Fleches}{\stepcounter{Nbequa}}{\ifboolKV[ClesEquation]{FlecheDiv}{\stepcounter{Nbequa}}{}}
              \end{align*}
              \ifboolKV[ClesEquation]{Solution}{L'équation $\xintifboolexpr{#2=1}{}{\num{#2}}\useKV[ClesEquation]{Lettre}\xintifboolexpr{#3>0}{+\num{#3}}{-\num{\fpeval{0-#3}}}=\xintifboolexpr{#4=1}{}{\num{#4}}\useKV[ClesEquation]{Lettre}$ a une unique solution : \opdiv*{\Coeffb}{\Coeffa}{solution}{resteequa}\opcmp{resteequa}{0}$\ifboolKV[ClesEquation]{LettreSol}{\useKV[ClesEquation]{Lettre}=}{}\displaystyle\ifopeq\opexport{solution}{\solution}\num{\solution}\else\ifboolKV[ClesEquation]{Entier}{\SSimplifie{\Coeffb}{\Coeffa}}{\frac{\num{\Coeffb}}{\num{\Coeffa}}}\fi$.}{}%
            }%
          }%
        }%
      }%
    \fi
  }%


\newcommand{\ResolEquationSoustraction}[5][]{%
  \useKVdefault[ClesEquation]%
  \setKV[ClesEquation]{#1}%
  \xintifboolexpr{#2=0}{%
    \xintifboolexpr{#4=0}{%
      \xintifboolexpr{#3=#5}{%b=d
        L'équation $\num{#3}=\num{#5}$ a une infinité de solutions.}%
      {%b<>d
        L'équation $\num{#3}=\num{#5}$ n'a aucune solution.%
      }%
    }%
    {%0x+b=cx+d$
      \EquaDeuxSoustraction[#1]{#4}{#5}{}{#3}%
    }%
  }{%
    \xintifboolexpr{#4=0}{%ax+b=0x+d
      \EquaDeuxSoustraction[#1]{#2}{#3}{}{#5}%
    }
    {%ax+b=cx+d$
      \xintifboolexpr{#3=0}{%
        \xintifboolexpr{#5=0}{%ax=cx
          \EquaTroisSoustraction[#1]{#2}{0}{#4}{}%
        }%
        {%ax=cx+d
          \EquaTroisSoustraction[#1]{#4}{#5}{#2}{}%
        }%
      }%
      {\xintifboolexpr{#5=0}{%ax+b=cx
          \EquaTroisSoustraction[#1]{#2}{#3}{#4}{}%
        }%
        {%ax+b=cx+d -- ici
          \xintifboolexpr{#2=#4}{%
            \xintifboolexpr{#3=#5}{%b=d
              L'équation $\xintifboolexpr{#2=1}{}{\num{#2}}\useKV[ClesEquation]{Lettre}\xintifboolexpr{#3>0}{+\num{#3}}{-\num{\fpeval{0-#3}}}=\xintifboolexpr{#4=1}{}{\num{#4}}\useKV[ClesEquation]{Lettre}\xintifboolexpr{#5>0}{+\num{#5}}{-\num{\fpeval{0-#5}}}$ a une infinité de solutions.}%
            {%b<>d
              L'équation $\xintifboolexpr{#2=1}{}{\num{#2}}\useKV[ClesEquation]{Lettre}\xintifboolexpr{#3>0}{+\num{#3}}{-\num{\fpeval{0-#3}}}=\xintifboolexpr{#4=1}{}{\num{#4}}\useKV[ClesEquation]{Lettre}\xintifboolexpr{#5>0}{+\num{#5}}{-\num{\fpeval{0-#5}}}$ n'a aucune solution.%
            }%
          }{
            %% Cas délicat
            \xintifboolexpr{#2>#4}{%ax+b=cx+d avec a>c
              \begin{align*}
                \tikzmark{A-\theNbequa}\xintifboolexpr{#2=1}{}{\num{#2}}\useKV[ClesEquation]{Lettre}\xintifboolexpr{#3>0}{+\num{#3}}{-\num{\fpeval{0-#3}}}&=\xintifboolexpr{#4=1}{}{\num{#4}}\useKV[ClesEquation]{Lettre}\xintifboolexpr{#5>0}{+\num{#5}}{-\num{\fpeval{0-#5}}}\tikzmark{E-\theNbequa}\\
                \ifboolKV[ClesEquation]{Decomposition}{%
                \xintifboolexpr{#2=1}{}{\num{#2}}\useKV[ClesEquation]{Lettre}\mathcolor{Cdecomp}{\xintifboolexpr{#4>0}{-\num{#4}\useKV[ClesEquation]{Lettre}}{+\num{\fpeval{0-#4}}\useKV[ClesEquation]{Lettre}}}\xintifboolexpr{#3>0}{+\num{#3}}{-\num{\fpeval{0-#3}}}&=\xintifboolexpr{#4=1}{}{\num{#4}}\useKV[ClesEquation]{Lettre}\mathcolor{Cdecomp}{\xintifboolexpr{#4>0}{-\num{#4}\useKV[ClesEquation]{Lettre}}{+\num{\fpeval{0-#4}}\useKV[ClesEquation]{Lettre}}}\xintifboolexpr{#5>0}{+\num{#5}}{-\num{\fpeval{0-#5}}}\\
                }{}
                \tikzmark{B-\theNbequa}\xdef\Coeffa{\fpeval{#2-#4}}\xintifboolexpr{\Coeffa=1}{}{\num{\Coeffa}}\useKV[ClesEquation]{Lettre}\xintifboolexpr{#3>0}{+\num{#3}}{-\num{\fpeval{0-#3}}}&=\num{#5}\tikzmark{F-\theNbequa}\\
                \ifboolKV[ClesEquation]{Decomposition}{%
                \xintifboolexpr{\Coeffa=1}{}{\num{\Coeffa}}\useKV[ClesEquation]{Lettre}\xintifboolexpr{#3>0}{+\num{#3}}{-\num{\fpeval{0-#3}}}\mathcolor{Cdecomp}{\xintifboolexpr{#3>0}{-\num{#3}}{+\num{\fpeval{0-#3}}}}&=\num{#5}\mathcolor{Cdecomp}{\xintifboolexpr{#3>0}{-\num{#3}}{+\num{\fpeval{0-#3}}}}\\
                }{}%
                \tikzmark{C-\theNbequa}\xdef\Coeffb{\fpeval{#5-#3}}\xintifboolexpr{\Coeffa=1}{}{\num{\Coeffa}}\useKV[ClesEquation]{Lettre}&=\num{\Coeffb}\tikzmark{G-\theNbequa}%\\
                % eric
                \ifboolKV[ClesEquation]{Decomposition}{\\\xintifboolexpr{\Coeffa=1}{}{\frac{\num{\Coeffa}}{\mathcolor{Cdecomp}{\num{\Coeffa}}}\useKV[ClesEquation]{Lettre}&=\frac{\num{\Coeffb}}{\mathcolor{Cdecomp}{\num{\Coeffa}}}}}{}
                % eric
                \xintifboolexpr{\Coeffa=1}{}{\\}
                \ifboolKV[ClesEquation]{Fleches}{%
                \leftcomment{A-\theNbequa}{B-\theNbequa}{A-\theNbequa}{$\xintifboolexpr{#4>0}{-\num{#4}}{+\num{\fpeval{0-#4}}}\useKV[ClesEquation]{Lettre}$}
                \rightcomment{E-\theNbequa}{F-\theNbequa}{E-\theNbequa}{$\xintifboolexpr{#4>0}{-\num{#4}}{+\num{\fpeval{0-#4}}}\useKV[ClesEquation]{Lettre}$}
                \leftcomment{B-\theNbequa}{C-\theNbequa}{A-\theNbequa}{$\xintifboolexpr{#3>0}{-\num{#3}}{+\num{\fpeval{0-#3}}}$}%
                \rightcomment{F-\theNbequa}{G-\theNbequa}{E-\theNbequa}{$\xintifboolexpr{#3>0}{-\num{#3}}{+\num{\fpeval{0-#3}}}$}%
                }{}
                \xintifboolexpr{\Coeffa=1}{}{%\ifnum\cmtd>1
                \tikzmark{D-\theNbequa}\useKV[ClesEquation]{Lettre}&=\frac{\num{\Coeffb}}{\num{\Coeffa}}\tikzmark{H-\theNbequa}%\\
                \ifboolKV[ClesEquation]{Fleches}{%
                \leftcomment{C-\theNbequa}{D-\theNbequa}{A-\theNbequa}{$\div\xintifboolexpr{\Coeffa<0}{(\num{\Coeffa})}{\num{\Coeffa}}$}%
                \rightcomment{G-\theNbequa}{H-\theNbequa}{E-\theNbequa}{$\div\xintifboolexpr{\Coeffa<0}{(\num{\Coeffa})}{\num{\Coeffa}}$}%            
                }{
                \ifboolKV[ClesEquation]{FlecheDiv}{%
                \leftcomment{C-\theNbequa}{D-\theNbequa}{A-\theNbequa}{$\div\xintifboolexpr{\Coeffa<0}{(\num{\Coeffa})}{\num{\Coeffa}}$}%
                \rightcomment{G-\theNbequa}{H-\theNbequa}{E-\theNbequa}{$\div\xintifboolexpr{\Coeffa<0}{(\num{\Coeffa})}{\num{\Coeffa}}$}%                 
                }{}
                }
                \ifboolKV[ClesEquation]{Entier}{%
                \SSimpliTest{\Coeffb}{\Coeffa}%
                \ifboolKV[ClesEquation]{Simplification}{%
                \ifthenelse{\boolean{Simplification}}{\\\useKV[ClesEquation]{Lettre}&=\SSimplifie{\Coeffb}{\Coeffa}}{}%\\
                }{}
                }{}
                }
                \ifboolKV[ClesEquation]{Fleches}{\stepcounter{Nbequa}}{\ifboolKV[ClesEquation]{FlecheDiv}{\stepcounter{Nbequa}}{}}
              \end{align*}
              \ifboolKV[ClesEquation]{Solution}{L'équation $\xintifboolexpr{#2=1}{}{\num{#2}}\useKV[ClesEquation]{Lettre}\xintifboolexpr{#3>0}{+\num{#3}}{-\num{\fpeval{0-#3}}}=\xintifboolexpr{#4=1}{}{\num{#4}}\useKV[ClesEquation]{Lettre}\xintifboolexpr{#5>0}{+\num{#5}}{-\num{\fpeval{0-#5}}}$ a une unique solution : \opdiv*{\Coeffb}{\Coeffa}{solution}{resteequa}\opcmp{resteequa}{0}$\ifboolKV[ClesEquation]{LettreSol}{\useKV[ClesEquation]{Lettre}=}{}\displaystyle\ifopeq\opexport{solution}{\solution}\num{\solution}\else\ifboolKV[ClesEquation]{Entier}{\SSimplifie{\Coeffb}{\Coeffa}}{\frac{\num{\Coeffb}}{\num{\Coeffa}}}\fi$.%
              }{}
            }{%ax+b=cx+d avec a<c              % Autre cas délicat
              \begin{align*}%
                \tikzmark{A-\theNbequa}\xintifboolexpr{#2=1}{}{\num{#2}}\useKV[ClesEquation]{Lettre}\xintifboolexpr{#3>0}{+\num{#3}}{-\num{\fpeval{0-#3}}}&=\xintifboolexpr{#4=1}{}{\num{#4}}\useKV[ClesEquation]{Lettre}\xintifboolexpr{#5>0}{+\num{#5}}{-\num{\fpeval{0-#5}}}\tikzmark{E-\theNbequa}\\
                \ifboolKV[ClesEquation]{Decomposition}{%
                \xintifboolexpr{#2=1}{}{\num{#2}}\useKV[ClesEquation]{Lettre}\mathcolor{Cdecomp}{\xintifboolexpr{#2>0}{-\num{#2}\useKV[ClesEquation]{Lettre}}{+\num{\fpeval{0-#2}}\useKV[ClesEquation]{Lettre}}}\xintifboolexpr{#3>0}{+\num{#3}}{-\num{\fpeval{0-#3}}}&=\xintifboolexpr{#4=1}{}{\num{#4}}\useKV[ClesEquation]{Lettre}\mathcolor{Cdecomp}{\xintifboolexpr{#2>0}{-\num{#2}\useKV[ClesEquation]{Lettre}}{+\num{\fpeval{0-#2}}\useKV[ClesEquation]{Lettre}}}\xintifboolexpr{#5>0}{+\num{#5}}{-\num{\fpeval{0-#5}}}\\
                }{}
                \tikzmark{B-\theNbequa}\xdef\Coeffa{\fpeval{#4-#2}}\xintifboolexpr{#3>0}{\num{#3}}{-\num{\fpeval{0-#3}}}&=\xintifboolexpr{\Coeffa=1}{}{\num{\Coeffa}}\useKV[ClesEquation]{Lettre}\xintifboolexpr{#5>0}{+\num{#5}}{-\num{\fpeval{0-#5}}}\tikzmark{F-\theNbequa}\\
                \ifboolKV[ClesEquation]{Decomposition}{%
                \num{#3}\mathcolor{Cdecomp}{\xintifboolexpr{#5>0}{-\num{#5}}{+\num{\fpeval{0-#5}}}}&=\xintifboolexpr{\Coeffa=1}{}{\num{\Coeffa}}\useKV[ClesEquation]{Lettre}\xintifboolexpr{#5>0}{+\num{#5}}{-\num{\fpeval{0-#5}}}\mathcolor{Cdecomp}{\xintifboolexpr{#5>0}{-\num{#5}}{+\num{\fpeval{0-#5}}}}\\
                }{}%
                \tikzmark{C-\theNbequa}\xdef\Coeffb{\fpeval{#3-#5}}\num{\Coeffb}&=\xintifboolexpr{\Coeffa=1}{}{\num{\Coeffa}}\useKV[ClesEquation]{Lettre}\tikzmark{G-\theNbequa}%\\
                % eric
                \ifboolKV[ClesEquation]{Decomposition}{\\\xintifboolexpr{\Coeffa=1}{}{\frac{\num{\Coeffb}}{\mathcolor{Cdecomp}{\num{\Coeffa}}}&=\frac{\num{\Coeffa}}{\mathcolor{Cdecomp}{\num{\Coeffa}}}\useKV[ClesEquation]{Lettre}}}{}
                % eric
                \xintifboolexpr{\Coeffa=1}{}{\\}
                \ifboolKV[ClesEquation]{Fleches}{%
                \leftcomment{A-\theNbequa}{B-\theNbequa}{A-\theNbequa}{$\xintifboolexpr{#2>0}{-\num{#2}}{+\num{\fpeval{0-#2}}}\useKV[ClesEquation]{Lettre}$}
                \rightcomment{E-\theNbequa}{F-\theNbequa}{E-\theNbequa}{$\xintifboolexpr{#2>0}{-\num{#2}}{+\num{\fpeval{0-#2}}}\useKV[ClesEquation]{Lettre}$}
                \leftcomment{B-\theNbequa}{C-\theNbequa}{A-\theNbequa}{$\xintifboolexpr{#5>0}{-\num{#5}}{+\num{\fpeval{0-#5}}}$}%
                \rightcomment{F-\theNbequa}{G-\theNbequa}{E-\theNbequa}{$\xintifboolexpr{#5>0}{-\num{#5}}{+\num{\fpeval{0-#5}}}$}%
                }{}
                \xintifboolexpr{\Coeffa=1}{}{%\ifnum\cmtd>1
                \tikzmark{D-\theNbequa}\frac{\num{\Coeffb}}{\num{\Coeffa}}&=\useKV[ClesEquation]{Lettre}\tikzmark{H-\theNbequa}%\\
                \ifboolKV[ClesEquation]{Fleches}{%
                \leftcomment{C-\theNbequa}{D-\theNbequa}{A-\theNbequa}{$\div\xintifboolexpr{\Coeffa<0}{(\num{\Coeffa})}{\num{\Coeffa}}$}%
                \rightcomment{G-\theNbequa}{H-\theNbequa}{E-\theNbequa}{$\div\xintifboolexpr{\Coeffa<0}{(\num{\Coeffa})}{\num{\Coeffa}}$}%            
                }{
                \ifboolKV[ClesEquation]{FlecheDiv}{%
                \leftcomment{C-\theNbequa}{D-\theNbequa}{A-\theNbequa}{$\div\xintifboolexpr{\Coeffa<0}{(\num{\Coeffa})}{\num{\Coeffa}}$}%
                \rightcomment{G-\theNbequa}{H-\theNbequa}{E-\theNbequa}{$\div\xintifboolexpr{\Coeffa<0}{(\num{\Coeffa})}{\num{\Coeffa}}$}%                 
                }{}
                }
                \ifboolKV[ClesEquation]{Entier}{%
                \SSimpliTest{\Coeffb}{\Coeffa}%
                \ifboolKV[ClesEquation]{Simplification}{%
                \ifthenelse{\boolean{Simplification}}{\\\SSimplifie{\Coeffb}{\Coeffa}&=\useKV[ClesEquation]{Lettre}}{}%\\
                }{}
                }{}
                }
                \ifboolKV[ClesEquation]{Fleches}{\stepcounter{Nbequa}}{\ifboolKV[ClesEquation]{FlecheDiv}{\stepcounter{Nbequa}}{}}
              \end{align*}
              \ifboolKV[ClesEquation]{Solution}{L'équation $\xintifboolexpr{#2=1}{}{\num{#2}}\useKV[ClesEquation]{Lettre}\xintifboolexpr{#3>0}{+\num{#3}}{-\num{\fpeval{0-#3}}}=\xintifboolexpr{#4=1}{}{\num{#4}}\useKV[ClesEquation]{Lettre}\xintifboolexpr{#5>0}{+\num{#5}}{-\num{\fpeval{0-#5}}}$ a une unique solution : \opdiv*{\Coeffb}{\Coeffa}{solution}{resteequa}\opcmp{resteequa}{0}$\ifboolKV[ClesEquation]{LettreSol}{\useKV[ClesEquation]{Lettre}=}{}\displaystyle\ifopeq\opexport{solution}{\solution}\num{\solution}\else\ifboolKV[ClesEquation]{Entier}{\SSimplifie{\Coeffb}{\Coeffa}}{\frac{\num{\Coeffb}}{\num{\Coeffa}}}\fi$.%
              }{}%
            }%
          }%
        }%
      }%
    }%
  }%
}%


