% Standardstil für normale Aufgaben (und Zusatzaufgaben) mit Punkten neben der Subsection

\xsimstyle{schule-default}

\DeclareExerciseEnvironmentTemplate{schule-default} {%
    \addpenalty{-30}
    \smallskip\noindent\textbf{%
            % Falls Zusatzaufgabe:
            \ifthenelse{\equal{\ExerciseType}{zusatzaufgabe}}{\llap{\GetExerciseProperty{symbol}$\bigstar$}~}{\llap{\GetExerciseProperty{symbol}~}}%
            \XSIMmixedcase{\GetExerciseName}\nobreakspace
            \GetExerciseProperty{counter}%
            \IfInsideSolutionF{%
                \IfExercisePropertySetT{subtitle}{
                    {\nobreakspace\GetExercisePropertyT{subtitle}{\normalfont \itshape \PropertyValue}}}% Subtitle setzten
                }%
        % Stil für Punkteanzeige
        \GetExercisePropertyTF{points}{%
            \normalfont% Punkte in normal font/Shape aber auf Ebene der Subsection
            \nobreakspace(\PropertyValue
            \GetExercisePropertyT{bonus-points}
            {\nobreakspace\small(+\PropertyValue)}% Kleinere Bonuspunkte?
            \nobreakspace%
                \IfExerciseGoalSingularTF{points}
                    {\XSIMtranslate{point}}
                    {\XSIMtranslate{points}}%
            )
        }{%Keine Punkte
            \GetExercisePropertyT{bonus-points}{
                \normalfont% Punkte in normal font/Shape aber auf Ebene der Subsection
                \nobreakspace(0\nobreakspace\small(+\PropertyValue)% Kleinere Bonuspunkte?
                \nobreakspace%
                \IfExerciseGoalSingularTF{bonus-points}
                    {\XSIMtranslate{point}}
                    {\XSIMtranslate{points}}%
                )
            }
        }
    }\par\smallskip\nopagebreak
    %
    %
}
{\IfInsideSolutionT{\par}}%
