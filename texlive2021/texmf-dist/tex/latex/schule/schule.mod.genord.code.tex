% Grosse Ueberschrift mit Name, Klasse, Datum zu Beginn
%\ifthenelse{\boolean{schule@erwartungshorizontAnzeigen}}{
	%\AtBeginDocument{
	%	\begin{center}
	%		\large
	%		\textbf{Name:} \luecke{12em}\hspace{1em}
	%		\textbf{Klasse:}\luecke{4em}\hspace{1em}
	%		\textbf{Datum:}\luecke{8em}\hspace{1em}
	%	\end{center}
	%	\vspace{1em}
	%}
%}{}
%\renewcommand{\schule@kopfAussen}{}
% Schöne Achtungsbox
\newcolumntype{C}[1]{>{\centering\arraybackslash}p{#1}}
\newtcolorbox{achtungbox}[1][]{enhanced,
  before skip=0.75em,after skip=1.25em,
  boxrule=0.4pt,left=5mm,right=2mm,top=1mm,bottom=1mm,
  colback=yellow!50,
  colframe=yellow!20!black,
  sharp corners,rounded corners=southeast,arc is angular,arc=3mm,
  underlay={%
    \path[fill=tcbcol@back!80!black] ([yshift=3mm]interior.south east)--++(-0.4,-0.1)--++(0.1,-0.2);
    \path[draw=tcbcol@frame,shorten <=-0.05mm,shorten >=-0.05mm] ([yshift=3mm]interior.south east)--++(-0.4,-0.1)--++(0.1,-0.2);
    \path[fill=yellow!50!black,draw=none] (interior.south west) rectangle node[white]{\Huge\bfseries !} ([xshift=4mm]interior.north west);
    },
  drop fuzzy shadow,#1}
% Versionssetter für Modulplan
\newcommand{\version}[3][nope]{%
	\ifthenelse{\equal{#1}{nope}}{%
		\ifoot{\normalfont Version: #3\\\footnotesize\today\ \DTMcurrenttime}
	}{%
		\ifoot{\normalfont Version: #3\\\footnotesize\today\ \DTMcurrenttime\ (\textit{#1})}
	}
	\ofoot{\qrcode[nolink,height=4em]{#2}}
}
% Box für ein neues Modulthema
\newenvironment{thema}[2][]{%
	\refstepcounter{schule@thema}
	\begin{tcolorbox}[title={\textbf{Thema \theschule@thema:} #2},%
		colback=white!98!black,%
		enhanced jigsaw, breakable,%
	]
}{%
	\end{tcolorbox}
	\vspace{0.75em}
}

% Merksatz innerhalb eines Modulthemas
\newenvironment{merke}[2][]{%
	\begin{tcolorbox}[title={\bfseries Merkheft \quad \symHeft},
		enhanced,attach boxed title to top left=
			{xshift=2mm,yshift=-2mm},
		boxrule=0.5pt,
		colback=white!93!black,
		halign lower=flush right,
		skin=bicolor,colbacklower=white!85!black,
	]
		\textbf{Überschrift:} #2
}{%
		\tcblower
		Erledigt: \chb
	\end{tcolorbox}
}

% Aufgaben innerhalb eines Modulthemas
\newenvironment{rechne}[1][]{%
	\begin{tcolorbox}[title={\bfseries Aufgaben \quad \symKlemmbrett},
		boxrule=0.5pt,
		colback=white!93!black,
		%underlay={\begin{tcbclipinterior}
		%	\draw[help lines,step=5mm,white!60!black,shift={(interior.north west)}]
		%	(interior.south west) grid (interior.north east);
		%	\end{tcbclipinterior}},
		enhanced,attach boxed title to top left=
			{xshift=2mm,yshift=-2mm},
	]
}{%
	\end{tcolorbox}
}

% Items für Aufgaben mit optionalem Symbol
\newcommand{\modafg}[1][\symHeft]{%
	\item[#1] 
}
\newcommand{\afg}[1][\symBleistift]{%
	\item[#1] Aufgabe\
}
\newcommand{\hinw}[2][\symZeigefinger]{%
	\\ {\raggedleft\color{gray} #1\ \prettyref{hinw:#2}}
}
% Basis-Aufgaben innerhalb von Rechenaufgaben innerhalb eines Modulthemas
%  optional: Schwieritgkeit (Basis default)
\ProvideDocumentEnvironment{buchafg}{O{Basis} m O{Erledigt} O{none}}{%
	\ifthenelse{\equal{#4}{none}}{\begin{minipage}{0.48\textwidth}}{}
		\begin{tcolorbox}[title={\textbf{#1:} #2 \quad \symBuch},
			boxrule=0.5pt,
			colback=white!91!black,
			%underlay={\begin{tcbclipinterior}
			%	\draw[help lines,step=5mm,white!60!black,shift={(interior.north west)}]
			%	(interior.south west) grid (interior.north east);
			%	\end{tcbclipinterior}},
			enhanced,attach boxed title to top center=
				{yshift=-2mm},
			halign lower=flush right,
			skin=bicolor,colbacklower=white!85!black,
		]
			\vspace{2mm}
			\begin{smallitemize}
}{%
			\end{smallitemize}
			\tcblower
			#3: \chb
		\end{tcolorbox}
		\ifthenelse{\equal{#4}{none}}{\end{minipage}}{}
}
\newenvironment{nochfit}{%
	\begin{center}
	\begin{minipage}{0.8\textwidth}
		\begin{tcolorbox}[title={\textbf{Noch fit?}},
			boxrule=0.5pt,
			colback=white!91!black,
			%underlay={\begin{tcbclipinterior}
			%	\draw[help lines,step=5mm,white!60!black,shift={(interior.north west)}]
			%	(interior.south west) grid (interior.north east);
			%	\end{tcbclipinterior}},
			enhanced,attach boxed title to top center=
				{yshift=-2mm},
			halign lower=flush right,
			skin=bicolor,colbacklower=white!85!black,
		]
			\vspace{2mm}
			%\begin{smallitemize}
}{%
			%\end{smallitemize}
			\tcblower
			Optional: \chb
		\end{tcolorbox}
	\end{minipage}
	\end{center}
}

% Zusammenfassung innerhalb eines Modulplans
\newenvironment{allesklar}[1][]{%
	\begin{tcolorbox}[title={\textbf{Zusammenfassung}},
		colback=white!98!black
	]
}{%
		\begin{smallitemize}
			\item[\symDaumenHoch] Nun kannst du den KT schreiben.
			\item[\symHeft]	Bitte gib dein Merkheft ab.
		\end{smallitemize}
	\end{tcolorbox}
}
% Alles Klar Aufgaben innerhalb der Zusammenfassung
\newenvironment{alkr}[2][]{%
		\begin{tcolorbox}[title={\textbf{Alles klar?} #2 \quad \symBuch},
			boxrule=0.5pt,
			colback=white!91!black,
			%underlay={\begin{tcbclipinterior}
			%	\draw[help lines,step=5mm,white!60!black,shift={(interior.north west)}]
			%	(interior.south west) grid (interior.north east);
			%	\end{tcbclipinterior}},
			enhanced,attach boxed title to top center=
				{yshift=-2mm},
			halign lower=flush right,
			skin=bicolor,colbacklower=white!85!black,
		]
			\vspace{2mm}
			\begin{smallitemize}
}{%
			\end{smallitemize}
			\tcblower
			Korrigiert: \chb\\
		\end{tcolorbox}
}
% SuS-Feedback
\newcommand{\frage}[2][]{%
		#2 & \chb & \chb & \chb & \chb\tabularnewline
}
\newcommand{\wunsch}[2][15em]{%
	#2 & \multicolumn{4}{l}{\luecke[]{#1}}\tabularnewline
		 & \multicolumn{4}{l}{\luecke[]{#1}}\tabularnewline
}
\newenvironment{feedback}[1][\clearpage]{%
	#1
	\begin{tcolorbox}[title={\textbf{Feedback}},
		colback=white!98!black
	]
		Bitte fülle die untenstehenden Fragen aus und gib sie ausgeschnitten ab.
		
		\begin{tcolorbox}[%
			boxrule=0.5pt,
			colback=white!91!black,
			%underlay={\begin{tcbclipinterior}
			%	\draw[help lines,step=5mm,white!60!black,shift={(interior.north west)}]
			%	(interior.south west) grid (interior.north east);
			%	\end{tcbclipinterior}},
			%enhanced,attach boxed title to top center=
			%	{yshift=-2mm},
			%halign lower=flush right,
			%skin=bicolor,
			%colbacklower=white!85!black,
		]
			\renewcommand{\arraystretch}{1.3}
			\begin{tabular}{P{.5\linewidth}cccc}
				\rowcolor{black!20}
				\textbf{Ich stimme\ldots} &
				\multicolumn{1}{m{3em}}{\centering \textit{gar nicht zu}} &
				\multicolumn{1}{m{3em}}{\centering \textit{nicht zu}} &
				\multicolumn{1}{m{3em}}{\centering \textit{zu}} &
				\multicolumn{1}{m{3em}}{\centering \textit{sehr zu}}\tabularnewline\hline
}{%
			\end{tabular}
			\renewcommand{\arraystretch}{1}
			\label{feedbackPage}
		\end{tcolorbox}
	\end{tcolorbox}
}
% Modulaufgaben, zusätzlich
\newtcolorbox{modAufgabeBox}[2][]{%
			title={\textbf{\symHeft\hspace{0.75em} #2}},
			enhanced,attach boxed title to top left={xshift=2mm,yshift=-2mm},%
			boxrule=0.5pt,%
			colback=white!93!black,%
			halign lower=flush right,%
			skin=bicolor,colbacklower=white!85!black,%
			#1
}
%\DeclareInstance{exsheets-heading}{genord-boxed}{default}{
%	title-pre-code={%
%		\ifthenelse{\equal{%
%			\GetQuestionProperty{symbol}{\CurrentQuestionID}}{??}}{%
%		}{%
%		\llap{\GetQuestionProperty{symbol}{\CurrentQuestionID}%
%		\space}}%
%	},
%	title-post-code={},
%	points-pre-code = {},
%	points-post-code = {},
%	join={%
%		%title[r,B]number[l,B](.333em,0pt);
%		%title[r,B]points[l,B](.333em,0pt)
%		title[r,B]subtitle[l,B](1em,0pt)
%	},
%	attach={%
%		main[l,vc]title[l,vc](0pt,0pt)
%	},
%	%subtitle-format={block},
%}
%\NewQuSolPair % Normale Aufgaben
%	{modAufgabe}[%
%		type=exam,
%		name={},	
%		headings=genord-boxed,
%		pre-hook={%
%			\begin{modAufgabeBox}{%
%				Modulaufgabe \GetQuestionProperty{counter}{\CurrentQuestionID}%
%			}%
%		},
%		post-hook={\end{modAufgabeBox}\vspace{0.75em}},
%	]
%	{modLoesung}[]%
\DeclareExerciseType{modAufgabe}{
	exercise-env = modAufgabe,
	solution-env = modLoesung,
	exercise-name = \XSIMtranslate{question},
	solution-name = \XSIMtranslate{solution},
	exercise-template = schule-default,
	solution-template = schule-default,
}
\ProvideDocumentEnvironment{modulafg}{O{Novize} O{Erledigt} O{}}{%
	\begin{tcolorbox}[title={\textbf{#1}},
			boxrule=0.5pt,
			colback=white!91!black,
			%underlay={\begin{tcbclipinterior}
			%	\draw[help lines,step=5mm,white!60!black,shift={(interior.north west)}]
			%	(interior.south west) grid (interior.north east);
			%	\end{tcbclipinterior}},
			enhanced,attach boxed title to top center=
				{yshift=-2mm},
			halign lower=flush right,
			skin=bicolor,colbacklower=white!85!black,
			#3
		]
			\vspace{2mm}
}{%
			\tcblower
			#2: \chb
		\end{tcolorbox}
}
% Kommandos um Modulplan zu strukturieren
\newcommand{\starteModulplan}{%
	\cfoot{\thepage\ von \pageref{LetzteModulplanseite}}
	%\ofoot{}
	%\frontmatter
	%\setcounter{page}{1}
	\pagenumbering{arabic}
	\section*{\Title\ -- \Subtitle}
	\vspace{0.75em}
}
% Zusatzmaterialien
\newcommand{\zusatzMaterial}{%
	%\mainmatter
	%\setcounter{page}{1}
	\label{LetzteModulplanseite}
	\clearpage
	\appendix
	\pagenumbering{Roman}
	\cfoot{\thepage\ von \pageref{LetzteInhaltsseite}}
}
\newcommand{\starteModulaufgaben}{%
	\clearpage
	\section*{Modulaufgaben}
	\vspace{0.75em}
}
\newcommand{\starteHinweiskarten}{%
	\clearpage
	\section*{Hinweiskarten}
	\vspace{0.75em}
}
% Hinweiskarten
\newtcolorbox{hinwKarteBox}[2][]{%
			title={\textbf{\symZeigefinger\hspace{0.75em} #2}},
			enhanced,attach boxed title to top left={xshift=2mm,yshift=-2mm},%
			boxrule=0.5pt,%
			colback=white!93!black,%
			halign lower=flush right,%
			skin=bicolor,colbacklower=white!85!black,%
}
% hinweiskarten setzen
\ProvideDocumentEnvironment{hinwKarte}{omo}{%
	\refstepcounter{schule@hinweiskarte}
	\begin{hinwKarteBox}{Hinweiskarte~\theschule@hinweiskarte}
		\textbf{#2\IfNoValueF{#3}{:}} \IfNoValueF{#3}{\textit{#3}}

}{%
		\IfNoValueF{#1}{\label{hinw:#1}}
	\end{hinwKarteBox}
	\vspace{1.75em}
}
% schöne Links als qrcode setzen
%  Optional: Breite der minipage
%  1.: URL für QRcode
%  2.: URL für Text
%  3.: Erklärungstext
\newcommand{\link}[4][0.5\textwidth]{%
	\begin{minipage}{#1}
		\begin{center}
			\begin{singlespace}
				\qrcode[height=1.5cm]{#2}\\[0.75em]	
					\textit{#4}\\
				\footnotesize{\url{#3}}
			\end{singlespace}
		\end{center}
	\end{minipage}
}
%\newenvironment{modAufgabe}[2][]{%
%		\textbf{Überschrift:} #2
%}{%
%		\tcblower
%		Erledigt: \chb
%	\end{tcolorbox}
%}
% Erwartungshorizont (GENORD:Eine Tabelle für alles, keine Punkte, Smilies)
% --------------------------------------------------------------------
\newcommand{\schule@erwartungshorizontGENord}{
	\renewcommand{\arraystretch}{1.1}
	% Zwischenspeicher für Tabelle
	\def\schule@EH{}
		
	%% Tabellarische Übersicht aus Einzelerwartungen zusammenbauen.
	\ForEachUsedExerciseByID{%
		% Tabelle zurücksetzen
		\def\schule@aktuelleAufgabeEH{}%
		%
		% Erwartungen laden und vollständig expandieren
		\edef\schule@aktuelleErwartungExpandiert{%
			\GetQuestionProperty{erwartungen}{##1}%
		}%
		%
		% Zeilenfarbe zurücksetzen
		\setboolean{schuleEHZeileGerade}{false}%
		%
		% Einzelne Erwartungen in Tabellenform sammeln
		\ForEachX{;}{%
			\ifthenelse{\thislevelcount=1}{%
				% Erste Erwartung ist leer => weglassen 
				% Speicherung wäre sonst kompliziert...
			}{%
				% Aufteilen in Inhalt und Punkte
				\schule@splitErwartung{\thislevelitem}%
				%
				% Zeilenfarbe festlegen
				\ifthenelse{\boolean{schuleEHZeileGerade}}{%
					\edef\schule@aktuelleAufgabeEH{%
						\schule@aktuelleAufgabeEH %
						\detokenize{%
							\rowcolor{black!10}%
						}%
					}%
				}{%
					\edef\schule@aktuelleAufgabeEH{%
						\schule@aktuelleAufgabeEH %
						\detokenize{%
							\rowcolor{white}%
						}%
					}%
				}%
				%
				% Zeilenfarbe umschalten
				\schule@EHZeilenfarbeUmschalten%
					%
				% Anschließend muss die Tabellenzeile aufgebaut und 
				% wieder serialisiert werden, da Tabellen sich nicht 
				% dynamisch bauen lassen. \tabularnewline, Rules und
				\edef\schule@aktuelleAufgabeEH{%
					\schule@aktuelleAufgabeEH%
					%
					\schule@aktuelleErwartungInhalt%
					\detokenize{&}%
					\schule@aktuelleErwartungPunkte%
					\detokenize{& \chb & \chb & \chb & \chb \\}%
						%\specialrule{.05em}{0em}{0em}}
					}%
			}%
		}{\schule@aktuelleErwartungExpandiert} %
		%
		% Gesamtpunkte mit Bonuspunkten kombinieren
		\def\schule@aktuelleAufgabePunkte{%
			\IfQuestionPropertyT{points}{##1}{%
				% Punkte angegeben
				\GetQuestionProperty{points}{##1}%
			}%
			\IfQuestionPropertyT{bonus-points}{##1}{%
				% Zusatzspunkte angegeben
				\space(+\,\GetQuestionProperty{bonus-points}{##1})%
			}%
		}%
		%
		% Titelzeile für die Aufgabe an die Tabelle anfügen.
		\edef\schule@EH{%
			\schule@EH%
			\detokenize{\specialrule{.1em}{0em}{0em}}%
			%\detokenize{\hrule}
			\detokenize{\rowcolor{black!20}\bfseries Aufgabe}%
			\space\GetQuestionProperty{counter}{##1}%
			\detokenize{& \bfseries}%
			%\schule@aktuelleAufgabePunkte
%			\detokenize{& & & \\}%
			\detokenize{& -- & \usym{1F641} & \usym{1F610} & \usym{1F642}\\}%
			%\detokenize{\hrule}
			\detokenize{\specialrule{.1em}{0em}{0em}}	%
		}%
		%
		% Erwartungshorizont der Aufgabe an globale Tabelle anhängen.
		\edef\schule@EH{\schule@EH\schule@aktuelleAufgabeEH}%	
	}%

				
		% Zusammengebaute Zeilen wieder deserialisieren
		\tokenize{%
			\schule@EHCode%
		}{%
			\schule@EH%
		}
		
		% Tabelle setzen
		\thispagestyle{empty}
		\begin{center}
			\large
			\textbf{Name:} \luecke{12em}\hspace{1em}
			\textbf{Klasse:}\luecke{4em}\hspace{1em}
			\textbf{Datum:}\luecke{8em}\hspace{1em}
		\end{center}
	
		\vspace*{-1ex}
		% Tabelle setzen
		\begin{longtable}{P{.5\linewidth}P{.2\linewidth}P{1em}P{1em}P{1em}P{1em}}
			% Kopfzeile
			\rowcolor{black!20}
			\textbf{Du\dots} &
			\textbf{Übung} &
			\multicolumn{4}{l}{\textbf{Rückmeldung}}%   %%
			\tabularnewline%\specialrule{.05em}{0em}{0em}  %
			% Erwartungen
			\schule@EHCode
			% Fusszeile
			\specialrule{.1em}{0em}{0em}
			\rowcolor{black!20}\textbf{Gesamt} &	
			\multicolumn{5}{r}{\textbf{\totalpoints*} Punkte}
		\end{longtable}	
		

		\clearpage
	\section*{Punkteverteilung}
	\begin{center}
		\punktuebersicht
	\end{center}
		
	\begin{center}
		\notenverteilung
	\end{center}
	
	\vspace{1em}	
		\begin{center}
			\large
			\textbf{Note:} \luecke{10em}\hspace{1em}
			\textbf{Datum:}\luecke{6em}\hspace{1em}
			\textbf{Unterschrift:}\luecke{6em}\hspace{1em}
		\end{center}

		\vspace{0.4em}

		\begin{minipage}{0.36\textwidth}
			\footnotesize
			\textbf{Erklärungen der Symbole:}
			\begin{description}
				\item[\usym{1F642}] Fehlerfrei
				\item[\usym{1F610}] Ohne grobe Fehler
				\item[\usym{1F641}] Fehler sind vorhanden, stehen aber einem Grundverständnis nicht im Wege
				\item[--] Durch die Häufigkeit von Ungenauigkeiten und Fehlern: Kompetenz nicht erreicht
			\end{description}
		\end{minipage}
		\begin{minipage}{0.63\textwidth}
			\textbf{Bemerkungen:}
			\vspace{10em}
		\end{minipage}

		
	\renewcommand{\arraystretch}{1}
	\clearpage
}
% AUTOMATISCHE BEPUNKTUNG EINLESEN
\newcommand{\schule@erwartungshorizontGENordPunkte}{
	\renewcommand{\arraystretch}{1.1}
	% Zwischenspeicher für Tabelle
	\def\schule@EH{}
		
	%% Tabellarische Übersicht aus Einzelerwartungen zusammenbauen.
	\ForEachQuestion{%
		% Tabelle zurücksetzen
		\def\schule@aktuelleAufgabeEH{}%
		%
		% Erwartungen laden und vollständig expandieren
		\edef\schule@aktuelleErwartungExpandiert{%
			\GetQuestionProperty{erwartungen}{##1}%
		}%
		%
		% Zeilenfarbe zurücksetzen
		\setboolean{schuleEHZeileGerade}{false}%
		%
		% Einzelne Erwartungen in Tabellenform sammeln
		\ForEachX{;}{%
			\ifthenelse{\thislevelcount=1}{%
				% Erste Erwartung ist leer => weglassen 
				% Speicherung wäre sonst kompliziert...
			}{%
				% Aufteilen in Inhalt und Punkte
				\schule@splitErwartung{\thislevelitem}%
				%
				% Zeilenfarbe festlegen
				\ifthenelse{\boolean{schuleEHZeileGerade}}{%
					\edef\schule@aktuelleAufgabeEH{%
						\schule@aktuelleAufgabeEH %
						\detokenize{%
							\rowcolor{black!10}%
						}%
					}%
				}{%
					\edef\schule@aktuelleAufgabeEH{%
						\schule@aktuelleAufgabeEH %
						\detokenize{%
							\rowcolor{white}%
						}%
					}%
				}%
				%
				% Zeilenfarbe umschalten
				\schule@EHZeilenfarbeUmschalten%
					%
				% Anschließend muss die Tabellenzeile aufgebaut und 
				% wieder serialisiert werden, da Tabellen sich nicht 
				% dynamisch bauen lassen. \tabularnewline, Rules und
				\edef\schule@aktuelleAufgabeEH{%
					\schule@aktuelleAufgabeEH%
					%
					\schule@aktuelleErwartungInhalt%
					\detokenize{&}%
					\schule@aktuelleErwartungPunkte%
					\detokenize{& \chb & \chb & \chb & \chb \\}%
						%\specialrule{.05em}{0em}{0em}}
					}%
			}%
		}{\schule@aktuelleErwartungExpandiert} %
		%
		% Gesamtpunkte mit Bonuspunkten kombinieren
		\def\schule@aktuelleAufgabePunkte{%
			\IfQuestionPropertyT{points}{##1}{%
				% Punkte angegeben
				\GetQuestionProperty{points}{##1}%
			}%
			\IfQuestionPropertyT{bonus-points}{##1}{%
				% Zusatzspunkte angegeben
				\space(+\,\GetQuestionProperty{bonus-points}{##1})%
			}%
		}%
		%
		% Titelzeile für die Aufgabe an die Tabelle anfügen.
		\edef\schule@EH{%
			\schule@EH%
			\detokenize{\specialrule{.1em}{0em}{0em}}%
			%\detokenize{\hrule}
			\detokenize{\rowcolor{black!20}\bfseries Aufgabe}%
			\space\GetQuestionProperty{counter}{##1}%
			\detokenize{& \bfseries}%
			%\schule@aktuelleAufgabePunkte
%			\detokenize{& & & \\}%
			\detokenize{& -- & \usym{1F641} & \usym{1F610} & \usym{1F642}\\}%
			%\detokenize{\hrule}
			\detokenize{\specialrule{.1em}{0em}{0em}}	%
		}%
		%
		% Erwartungshorizont der Aufgabe an globale Tabelle anhängen.
		\edef\schule@EH{\schule@EH\schule@aktuelleAufgabeEH}%	
	}%

				
		% Zusammengebaute Zeilen wieder deserialisieren
		\tokenize{%
			\schule@EHCode%
		}{%
			\schule@EH%
		}
	
		\DTLforeach{namen}{%
			\vorname=Vorname,\nachname=Nachname,\klasse=Klasse,\krzl=Krzl,\kommentar=Kommentar%
		}{%
				% Tabelle setzen
				\thispagestyle{empty}
				\begin{center}
					\large
					\textbf{Name:} \vorname\ \nachname\hspace{1em}
					\textbf{Klasse:} \klasse\hspace{1em}
					\textbf{Datum:} \today\hspace{1em}
				\end{center}
			
				\vspace*{-1ex}
				% Tabelle setzen
				\begin{longtable}{P{.5\linewidth}P{.2\linewidth}P{1em}P{1em}P{1em}P{1em}}
					% Kopfzeile
					\rowcolor{black!20}
					\textbf{Du\dots} &
					\textbf{Übung} &
					\multicolumn{4}{l}{\textbf{Rückmeldung}}%   %%
					\tabularnewline%\specialrule{.05em}{0em}{0em}  %
					% Erwartungen
					\schule@EHCode
					% Fusszeile
					\specialrule{.1em}{0em}{0em}
					\rowcolor{black!20}\textbf{Gesamt} &	
					\multicolumn{5}{r}{\textbf{\totalpoints*} Punkte}
				\end{longtable}	
				

				\clearpage
	
				\tikzmath{%
					real \summe;
					\summe = 0;
				}
			\section*{Punkteverteilung}
			\begin{center}
				\punktuebersicht[punkte]
			\end{center}
			
			%\section*{Notenverteilung}
			\begin{center}
				\notenverteilung
			\end{center}
			
			\vspace{1em}	
				\begin{center}
					\large
					\textbf{Note:} \getNote \hspace{1em}
					\textbf{Datum:} \today\hspace{1em}
					\textbf{Unterschrift:}\luecke{6em}\hspace{1em}
				\end{center}

				\vspace{0.4em}

				\begin{minipage}{0.36\textwidth}
					\footnotesize
					\textbf{Erklärungen der Symbole:}
					\begin{description}
						\item[\usym{1F642}] Fehlerfrei
						\item[\usym{1F610}] Ohne grobe Fehler
						\item[\usym{1F641}] Fehler sind vorhanden, stehen aber einem Grundverständnis nicht im Wege
						\item[--] Durch die Häufigkeit von Ungenauigkeiten und Fehlern: Kompetenz nicht erreicht
					\end{description}
				\end{minipage}
				\begin{minipage}{0.63\textwidth}
					\textbf{Bemerkungen:}
					\kommentar
				\end{minipage}

				
			\renewcommand{\arraystretch}{1}
			\clearpage
		}
	\getNotenAuswertung
}
\renewcommand{\erwartungshorizont}{
        % Stil
        \IfEqCase{\schule@erwartungshorizontStil}{
            % Einzeltabellen
            {einzeltabellen}{
                \schule@erwartungshorizontEinzeltabellen
            }%
            % Ohne Punkte, mit Smilies
            {simpel}{
                \schule@erwartungshorizontSimpel
            }%
						{genord}{
							\schule@erwartungshorizontGENord
						}%
        }[%
            % Standard => Alles in eine Tabelle
            \schule@erwartungshorizontStandard
        ]
}
% Notenübersicht nur mit ganzen Punkten
% ********************************************************************
%\SetupExSheets{%
%	grades/half=false, % NICHT Auf halbe Punkte runden
%	grades/round=0, % Eine Dezimalstelle
%}
\newcommand{\getNote}{%
	\ifthenelse{\boolean{schule@kmkPunkte}}{}{%
		\tikzmath{
			real \summe;
			\summe=0;
			real \gesamt;
			\gesamt = 0;
		}
		\ForEachQuestion{%
			\IfQuestionPropertyT{points}{##1}{%
				\tikzmath{\gesamt = \gesamt + \GetQuestionProperty{points}{##1};}%
			}%
			%\tikzmath{\gesamt = 32;}
			\tikzmath{\summe = \summe + \GetQuestionProperty{\krzl}{##1};}%
		}
		\tikzmath{
			real \eins;
			real \zwei;
			real \drei;
			real \vier;
			real \fuenf;
		}
		\notenschemaTikz
		\ifthenelse{\equal{\GetQuestionProperty{half}{1}}{true}}{%
			\tikzmath{%
				\eins = \eins*\gesamt;
				\zwei = \zwei*\gesamt;
				\drei = \drei*\gesamt;
				\vier = \vier*\gesamt;
				\fuenf = \fuenf*\gesamt;
			}
		}{%
			\tikzmath{%
				\eins = round(\eins*\gesamt);
				\zwei = round(\zwei*\gesamt);
				\drei = round(\drei*\gesamt);
				\vier = round(\vier*\gesamt);
				\fuenf = round(\fuenf*\gesamt);
			}
		}
		\tikzmath{
			if \summe < \fuenf then { let \note=ungenügend; }	else { 
				if \summe < \vier then { let \note=mangelhaft; } else {
					if \summe < \drei then { let \note=ausreichend; } else {
						if \summe < \zwei then { let \note=befriedigend; } else {
							if \summe < \eins then { let \note=gut;} else {
								let \note=sehr gut;
							};
						};
					};
				};
			};
		}
		\note
	}
}
\newcommand{\getNotenAuswertung}{%
	\ifthenelse{\boolean{schule@kmkPunkte}}{}{%
		\tikzmath{
			integer \einsen;
			integer \zweien;
			integer \dreien;
			integer \vieren;
			integer \fuenfen;
			integer \sechsen;
			\einsen = 0;
			\zweien = 0;
			\dreien = 0;
			\vieren = 0;
			\fuenfen = 0;
			\sechsen = 0;
		}
	}
	\clearpage
	\section*{Auswertung}
	\subsection*{Noten}
	\DTLforeach{namen}{%
		\vorname=Vorname,\nachname=Nachname,\klasse=Klasse,\krzl=Krzl,\kommentar=Kommentar%
	}{%
		\ifthenelse{\boolean{schule@kmkPunkte}}{}{%
			\tikzmath{
				real \summe;
				\summe=0;
				real \gesamt;
				\gesamt = 0;
			}
			\ForEachQuestion{%
				\IfQuestionPropertyT{points}{##1}{%
					\tikzmath{\gesamt = \gesamt + \GetQuestionProperty{points}{##1};}%
				}%
				%\tikzmath{\gesamt = 32;}
				\tikzmath{\summe = \summe + \GetQuestionProperty{\krzl}{##1};}%
			}
			\tikzmath{
				real \eins;
				real \zwei;
				real \drei;
				real \vier;
				real \fuenf;
			}
			\notenschemaTikz
			\ifthenelse{\equal{\GetQuestionProperty{half}{1}}{true}}{%
				\tikzmath{%
					\eins = \eins*\gesamt;
					\zwei = \zwei*\gesamt;
					\drei = \drei*\gesamt;
					\vier = \vier*\gesamt;
					\fuenf = \fuenf*\gesamt;
				}
			}{%
				\tikzmath{%
					\eins = round(\eins*\gesamt);
					\zwei = round(\zwei*\gesamt);
					\drei = round(\drei*\gesamt);
					\vier = round(\vier*\gesamt);
					\fuenf = round(\fuenf*\gesamt);
				}
			}
			\tikzmath{
				if \summe < \fuenf then { let \note=ungenügend; }	else { 
					if \summe < \vier then { let \note=mangelhaft; } else {
						if \summe < \drei then { let \note=ausreichend; } else {
							if \summe < \zwei then { let \note=befriedigend; } else {
								if \summe < \eins then { let \note=gut;} else {
									let \note=sehr gut;
								};
							};
						};
					};
				};
			}
			\tikzmath{
				if \summe < \fuenf then { \sechsen = \sechsen +1; }	else { 
					if \summe < \vier then { \fuenfen = \fuenfen + 1; } else {
						if \summe < \drei then { \vieren = \vieren + 1; } else {
							if \summe < \zwei then { \dreien = \dreien + 1; } else {
								if \summe < \eins then { \zweien = \zweien + 1;} else {
									\einsen = \einsen + 1;
								};
							};
						};
					};
				};
			}
			\begin{tabular}{p{2.5cm}p{2.5cm}p{3cm}}
				\vorname & \nachname & \note \\
			\end{tabular}

		}
	}
	\DTLsavelastrowcount{\n}
	\subsection*{Notenverteilung}
	\ifthenelse{\boolean{schule@kmkPunkte}}{}{%
		\tikzmath{
			real \schnitt;
			\schnitt=(\einsen+\zweien*2+\dreien*3+\vieren*4+\fuenfen*5+\sechsen*6)/\n;
		}
	
		\begin{center}
			\begin{tabular}{l|c|c|c|c|c|c}
				\textbf{Note:} & 1 & 2 & 3 & 4 & 5 & 6\\\hline
				\textbf{Anzahl:} & \einsen & \zweien & \dreien & \vieren & \fuenfen & \sechsen
			\end{tabular}
		\end{center}
		Es haben insgesamt \n\ Schülerinnen und Schüler mit einem \textbf{Schnitt von
		\pgfmathprintnumber[precision=2,zerofill]{\schnitt}} geschrieben.
	}
}
\renewcommand{\notenverteilung}{%
  \ifthenelse{\boolean{schule@kmkPunkte}}{
			kln
	}{%
	% Ohne Notenpunkte
	% ----------------------------------------------------------------
		\section*{Notenverteilung}
		\parbox{.45\linewidth}{
	%		\tiny
			\begin{tabular}{p{0.6\linewidth}rr}
				\rowcolor{black!20}
				\textbf{Note} & \textbf{$\ge$ P.}\\
				sehr gut  & $\schule@punkteZuNote{1}$\\
				\rowcolor{black!10}
				gut & $\schule@punkteZuNote{2}$ \\
				befriedigend & $\schule@punkteZuNote{3}$
			\end{tabular}
		}
		\parbox{.45\linewidth}{
		%	\tiny
			\begin{tabular}{p{0.6\linewidth}rr}
				\rowcolor{black!20}
				\textbf{Note} & \textbf{$\ge$ P.}\\
				ausreichend & $\schule@punkteZuNote{4}$\\
				\rowcolor{black!10}
				mangelhaft  & $\schule@punkteZuNote{5}$\\
				ungenügend & $0$ 
			\end{tabular}
		}
	}
}
% Punktübersicht
% ********************************************************************
\renewcommand{\punktuebersicht}[1][none]{
	\renewcommand{\arraystretch}{1.4}
	\ifthenelse{\equal{#1}{none}}{%
		\begin{tabular}{l|*{\numberofquestions}{c|}c}
			%\hline
			% Auflistung der Aufgaben
			\rowcolor{black!20}
			\textbf{Aufgabe} &
			\ForEachQuestion{
				\textbf{\QuestionNumber{##1}}
				\iflastquestion{}{&}
			} &
			\textbf{Gesamt} \\ \hline
			% Punkte auflisten
			Punkte
			&
			\ForEachQuestion{
				\IfQuestionPropertyTF{points}{##1}{
					% Punkte angegeben
					\GetQuestionProperty{points}{##1}
				}{
					% Keine Punkte angegeben
					0
				}
				\iflastquestion{}{&}
			} &
			\pointssum* \\ \hline
			% Zusatzpunkte auflisten
			\rowcolor{black!10}
			Zusatzpunkte
			&
			\ForEachQuestion{
				\IfQuestionPropertyTF{bonus-points}{##1}{
					% Zusatzspunkte angegeben
					\GetQuestionProperty{bonus-points}{##1}
				}{
					% Keine Zusatzpunkte angegeben
					0
				}
				\iflastquestion{}{&}
			} &
			\bonussum* \\ \hline
			% Erreichte Punkte
			Erreicht & 
			\ForEachQuestion{%
				\iflastquestion{}{&}%
			} & %\\ \hline
		\end{tabular}
	}{%
		\ForEachQuestion{%
			\tikzmath{\summe = \summe + \GetQuestionProperty{\krzl}{##1};}%
		}
		\begin{tabular}{l|*{\numberofquestions}{c|}c}
			%\hline
			% Auflistung der Aufgaben
			\rowcolor{black!20}
			\textbf{Aufgabe} &
			\ForEachQuestion{
				\textbf{\QuestionNumber{##1}}
				\iflastquestion{}{&}
			} &
			\textbf{Gesamt} \\ \hline
			% Punkte auflisten
			Punkte
			&
			\ForEachQuestion{
				\IfQuestionPropertyTF{points}{##1}{
					% Punkte angegeben
					\GetQuestionProperty{points}{##1}
				}{
					% Keine Punkte angegeben
					0
				}
				\iflastquestion{}{&}
			} &
			\pointssum* \\ \hline
			% Zusatzpunkte auflisten
			\rowcolor{black!10}
			Zusatzpunkte
			&
			\ForEachQuestion{
				\IfQuestionPropertyTF{bonus-points}{##1}{
					% Zusatzspunkte angegeben
					\GetQuestionProperty{bonus-points}{##1}
				}{
					% Keine Zusatzpunkte angegeben
					0
				}
				\iflastquestion{}{&}
			} &
			\bonussum* \\ \hline
			% Erreichte Punkte
			Erreicht & 
			\ifthenelse{\equal{#1}{punkte}}{%
				\ForEachQuestion{%
					\GetQuestionProperty{\krzl}{##1}&%
				} \pgfmathprintnumber{\summe} %\\ \hline
			}{%
				\ForEachQuestion{%
					\iflastquestion{}{&}%
				} & %\\ \hline
			}
		\end{tabular}
	}
	\renewcommand{\arraystretch}{1}
}
% Zusätze für Mathe
\usetikzlibrary{shapes.misc}
\tikzset{cross/.style={cross out, draw, 
         minimum size=2*(#1-\pgflinewidth), 
         inner sep=0pt, outer sep=0pt}}
\newcommand*\strecke[2][]{%
	\mkern 1.5mu\overline{\mkern-1.5mu#2\mkern-1.5mu}\mkern 1.5mu
}
