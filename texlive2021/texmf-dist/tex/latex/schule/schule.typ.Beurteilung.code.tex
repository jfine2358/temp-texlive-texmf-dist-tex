% allgemeine Voreinstellungen
% ****************************************************************
\pagestyle{scrheadings}
\KOMAoptions{headsepline=0.5pt}

\newcommand{\schule@dokumentTypBezeichnung}{Beurteilungsbeitrag}

% Tabelle zu den Beurteilungsgrundlagen füttern
\newcommand{\schule@grundlagen}{%
	01.05.-09.06.2019 & 9b & blabla\\
}

\newcommand{\setzeGrundlagen}[1]{%
	\renewcommand{\schule@grundlagen}{#1}
}


\newcommand{\schule@schuladresse}{%
	\ifthenelse{\equal{\schule@schulanschrift}{null}}{}{%
		\schule@schulanschrift\\[.2em]
	}
	\ifthenelse{\equal{\schule@schulstr}{null}}{}{%
		\schule@schulstr\\[.2em]
	}
	\ifthenelse{\equal{\schule@schulort}{null}}{}{%
		\schule@schulort\\[.2em]
	}
}
% Logo (und Adresse) der Schule im Kopf
\newcommand{\schule@schullogoprint}[1][\schule@schuladresse]{%
	\hfill
	\begin{minipage}{4cm}
		\centering
		\includegraphics[width=\textwidth]{logo.pdf}

		{\footnotesize \schule@schuladresse}
	\end{minipage}
	\vspace{1em}
}

% Kopfzeile festlegen
% ****************************************************************

% Kopf, Innenseite
\ihead{\today}

% Kopf, Mitte
\chead{\normalfont Beurteilungsbeitrag \schule@fach}

% Kopf, Außenseite
\ohead{\schule@beurteilung}

% Fußzeile festlegen
% ****************************************************************

% Seitenzahlen ==> Modul Format
\ofoot{\Seitenzahlen}

\cfoot{}

% Dokumentenname
\ifoot{\normalfont Beurteilungsbeitrag der \textit{\schule@schulname} für \textbf{\schule@ref}}


% Titelseite festlegen
% ****************************************************************
\renewcommand\maketitle{%
	\thispagestyle{empty}
	\ifthenelse{\boolean{schule@schullogo}}{%
		\schule@schullogoprint
	}{}
	\begin{center}
		{\Large\bfseries\sffamily Beurteilungsbeitrag gemäß §16 (2) OVP vom 10.\,April 2011,}\\[0.5em]

		geändert durch Verordnung vom 25.\,April 2016

		\vspace{2em}
		
		\renewcommand{\arraystretch}{1.5}
		\begin{tabular}{p{7cm}l}
			\toprule
			Lehramtsanwärterin/Lehramtsan\-wärter bzw. Lehrkraft in Ausbildung (OBAS): &
				\schule@ref \\
			Lehramt: & \schule@lehramt \\
			Ausbildungsschule: & \schule@schulname \\
			Fach: & \schule@fach \\
			Beurteilungszeitraum: & \schule@zeitraum \\
			Beurteilerin/Beurteiler: & \schule@beurteilung\\
			\bottomrule
		\end{tabular}
		\renewcommand{\arraystretch}{1}
	\end{center}
	\vspace{2em}
	\begin{center}
	\begin{minipage}{0.75\textwidth}
		\centering
		{\large\bfseries Beurteilungsgrundlagen}
		\begin{itemize}
			\item Ordnung des Vorbereitungsdienstes und der Staatsprüfung für Lehrämter an Schulen -- OVP
				--	vom 10.\,April 2011, geändert durch Verordnung vom 25.\,April 2016
			\item Kerncurriculum (RdErl. des MSW vom 02.\,September 2016)
			\item Beobachtungen im Unterricht und in sonstigen Ausbildungszusammenhängen
			\item Unterricht und Hospitation der Lehramtsanwärterin/des Lehramtsanwärters:

				\renewcommand{\arraystretch}{1.5}
				\begin{tabular}{lp{0.2\textwidth}p{0.4\textwidth}}
					\toprule
					\textbf{Zeitraum} & \textbf{Klasse/Lern\-gruppe} & \textbf{Themenschwerpunkte}\\
					\midrule
					\schule@grundlagen
					\bottomrule
				\end{tabular}
				\renewcommand{\arraystretch}{1}
		\end{itemize}
	\end{minipage}
	\end{center}
	\vspace{2em}

	\section*{%
		Verlauf und Erfolg des Vorbereitungsdienstes in den Handlungsfeldern bezogen auf die Kompetenzen
		und Standards der Anlage 1 zur OVP 2016
	}
}

\AtBeginDocument{\maketitle}

% Handlungsfelder formatieren
% ****************************************************************
\tcbset{%
	enhanced jigsaw,
	oversize,
	breakable,
	segmentation style=solid,
}
\newcommand{\schule@handlungsfeldStart}[4][]{%
	\begin{tcolorbox}[title={\bfseries\sffamily #3}, #1]
		\footnotesize
		\begin{enumerate}[label={Kompetenz \arabic*:},left=0pt,start=#2]
			#4
		\end{enumerate}
		\tcblower
		\normalfont
}
\newcommand{\schule@handlungsfeldEnd}[1][]{%
	\end{tcolorbox}

	\vspace{2em}
}
\newenvironment{handlungsfeld1}[1][]{%
	\schule@handlungsfeldStart[#1]{4}{Vielfalt als Herausforderung annehmen und als Chance nutzen}{%
		\item Lehrerinnen und Lehrer kennen die sozialen und kulturellen Lebensbedingungen, etwaige
				Benachteiligungen, Beeinträchtigungen -- auch gesundheitliche -- und Barrieren der Entwicklung
				des Lernens von Schülerinnen und Schülern und für Schülerinnen und Schüler und  nehmen im
				Rahmen der Schule Einfluss auf deren individuelle Entwicklung.
	}
}{%
	\schule@handlungsfeldEnd
}
\newenvironment{handlungsfeld2}[1][]{%
	\schule@handlungsfeldStart[#1]{1}{Unterricht für heterogene Lerngruppen gestalten und Lernprozesse
	nachhaltig anlegen}{%
		\item Lehrerinnen und Lehrer planen Unterricht unter Berücksichtigung unterschiedlicher
			Lernvoraussetzungen und Entwicklungsprozesse fach- und sachgerecht und führen ihn sachlich und
			fachlich korrekt durch.
		\item Lehrerinnen und Lehrer unterstützen durch die Gestaltung von Lernsituationen das Lernen
			von Schülerinnen und Schülern. Sie motivieren Schülerinnen und Schüler und befähigen sie,
			Zusammenhänge herzustellen und Gelerntes zu nutzen.
		\item Lehrerinnen und Lehrer fördern die Fähigkeiten von Schülerinnen und Schülern zum
			selbstbestimmten Lernen und Arbeiten.
	}
}{%
	\schule@handlungsfeldEnd
}
\newenvironment{handlungsfeld3}[1][]{%
	\schule@handlungsfeldStart[#1]{4}{%
		Den Erziehungsauftrag in Schule und Unterricht wahrnehmen
	}{%
		\item Lehrerinnen und Lehrer kennen die sozialen und kulturellen Lebensbedingungen, etwaige
				Benachteiligungen, Beeinträchtigungen -- auch gesundheitliche -- und Barrieren der Entwicklung
				des Lernens von Schülerinnen und Schülern und für Schülerinnen und Schüler und  nehmen im
				Rahmen der Schule Einfluss auf deren individuelle Entwicklung.
			\item Lehrerinnen und Lehrer vermitteln Werte und Normen, eine Haltung der Wertschätzung und
				Anerkennung von Diversität und unterstützen selbstbestimmtes Urteilen und Handeln von
				Schülerinnen und Schülern.
			\item Lehrerinnen und Lehrer finden Lösungsansätze für Schwierigkeiten und Konflikte in Schule
				und Unterricht.
	}
}{%
	\schule@handlungsfeldEnd
}
\newenvironment{handlungsfeld4}[1][]{%
	\schule@handlungsfeldStart[#1]{7}{%
		Lernen und Leisten herausfordern, dokumentieren, rückmelden und beurteilen
	}{%
		\item Lehrerinnen und Lehrer diagnostizieren Lernvoraussetzungen und Lernprozesse von
			Schülerinnen und Schülern; sie fördern Schülerinnen und Schüler gezielt und beraten Lernende
			und deren Eltern.
		\item Lehrerinnen und Lehrer erfassen die Leistungsentwicklung von Schülerinnen und Schülern und
			beurteilen Lernen und Leistung auf der Grundlage transparenter Beurteilungsmaßstäbe
	}
}{%
	\schule@handlungsfeldEnd
}
\newenvironment{handlungsfeld5}[1][]{%
	\schule@handlungsfeldStart[#1]{7}{%
		Schülerinnen und Schüler und Eltern beraten
	}{%
		\item Lehrerinnen und Lehrer diagnostizieren Lernvoraussetzungen und Lernprozesse von
			Schülerinnen und Schülern; sie fördern Schülerinnen und Schüler gezielt und beraten Lernende
			und deren Eltern.
	}
}{%
	\schule@handlungsfeldEnd
}
\newenvironment{handlungsfeld6}[1][]{%
	\schule@handlungsfeldStart[#1]{9}{%
		Im System Schule mit allen Beteiligten entwicklungsorientiert zusammenarbeiten
	}{%
		\item Lehrerinnen und Lehrer sind sich der besonderen Anforderungen des Lehrerberufs bewusst.
			Sie verstehen ihren Beruf als ein öffentliches Amt mit besonderer Verantwortung und
			Verpflichtung.
		\item Lehrerinnen und Lehrer verstehen ihren Beruf als ständige Lernaufgabe.
		\item Lehrerinnen und Lehrer beteiligen sich an der Planung und Umsetzung schulischer Projekte
			und Vorhaben.
	}
}{%
	\schule@handlungsfeldEnd
}


% Unterschriften formatieren
% ****************************************************************
\AtEndDocument{%
	\begin{center}
		%Unterschrift Beurteilung
		\begin{minipage}{0.7\textwidth}
			\begin{tcolorbox}[]
				\centering
				\begin{minipage}[t]{0.45\textwidth}
					\centering
					\vspace{3em}
					\hrule
					\vspace{0.5em}
					Ort, Datum
				\end{minipage}
				\hfil
				\begin{minipage}[t]{0.45\textwidth}
					\centering
					\vspace{3em}
					\hrule
					\vspace{0.5em}
					\schule@beurteilung\\ {\footnotesize(Beurteilerin/Beurteiler)}
				\end{minipage}
			\end{tcolorbox}
		\end{minipage}
		
		%Unterschrift Ref
		\vspace{3em}
		Von dem Beurteilungsbeitrag habe ich Kenntnis genommen und eine Durchschrift erhalten.

		\vspace{1em}
		\begin{minipage}{0.7\textwidth}
			\begin{tcolorbox}[]
				\centering
				\begin{minipage}[t]{0.45\textwidth}
					\centering
					\vspace{3em}
					\hrule
					\vspace{0.5em}
					Ort, Datum
				\end{minipage}
				\hfil
				\begin{minipage}[t]{0.45\textwidth}
					\centering
					\vspace{3em}
					\hrule
					\vspace{0.5em}
					\schule@ref\\ {\footnotesize(Anwärterin/Anwärter)}
				\end{minipage}
			\end{tcolorbox}
		\end{minipage}
	\end{center}
}
