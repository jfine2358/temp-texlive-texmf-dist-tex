%%%%%%%%%%%%%%%%%%%%%%%%%%%%%%%%%%%%%%%%%%%%%%%%%%%%%%%%%%%%%%%%%%%%%%%
%
% MYPHYX for TEX Version 2.9   -- created 12/15/88
%
%%%%%%%%%%%%%%%%%%%%%%%%%%%%%%%%%%%%%%%%%%%%%%%%%%%%%%%%%%%%%%%%%%%%%%%
%
\def\MYHEAD{\setbox0=\vtop{\baselineskip=10pt
     \ialign{\eightrm ##\hfil\cr
        \slacbin\cr
        P.~O.~Box 4349\cr
        Stanford, CA 94309\cropen{1\jot}
        (415) 926--4423\cr
        SHARON at SLACVM\cr}}%
   \setbox2=\hbox{\caps Stanford Linear Accelerator Center}%
   \hrule height 0pt \kern -0.5in
   \vbox to 0pt{\vss\centerline{\seventeenrm STANFORD UNIVERSITY}}
   \vbox{} \medskip
   \line{\hbox to 0.7\hsize{\hss \lower 10pt \box2 \hfill }\hfil
         \hbox to 0.25\hsize{\box0 \hfil }}\medskip }
 
  \def\myletter{\let\letterhead=\MYHEAD \letter}
  \def\binno{81}
  \def\slacext{2266}
  \def\half{\hbox{${1\over 2}\,$}}
  \def\etal{{\it et al.}}
 
  \def\bar#1{\overline{#1}}
  \def\kbar{{\mathchar'26\mskip-9muk}}
  \def\pbar{{\rlap/p}}
  \def\ubar{\bar{u}}
  \def\dbar{\bar{d}}
 
  \def\str{\penalty-10000\hfilneg\ }    % line break with right adjust
  \def\Str{\penalty-10000\hfilneg\ }    % line break with right adjust
  \def\nostr{\hfill\penalty-10000\ }    % line break with ragged right
  \let\brk=\nextline
  \def\|{\vrule height 16pt depth 6 pt}
 
\def\ZZ{\hbox{$\not\kern0.15em\not\kern-0.21em\lower0.2em
        \vbox{\hrule width 0.52em height 0.06em depth 0pt}
        \kern-0.50em\raise0.7em
        \vbox{\hrule width 0.52em height 0.06em depth 0pt}$}}
 
\def\frac#1/#2{\leavevmode\kern.1em\raise.5ex\hbox{\the\scriptfont0
         #1}\kern-.1em/\kern-.15em\lower.25ex\hbox{\the\scriptfont0 #2}}
 
%   \frac3/4 gives you a very nice 3/4 - slanted.
 
  \def\IR{{$\rm I\!R$}}
  \def\lozenge{\boxit{\hbox to 1.5pt{
               \vrule height 1pt width 0pt \hfill}}}
 
  \def\Buildrel#1\under#2{\mathrel{\mathop{#2}\limits_{#1}}}
  \def\simZ{{\Buildrel \sim \under Z}}
 
%%%%%%%%%%%%%%%%%%%%%%%%%%%%%%%%%%%%%%%%%%%%%%%%%%%%%%%%%%%%%%%%%%%%%%%%
%
% This macro inserts pictures (and, more generally, \vbox es) neatly
%   into the MIDDLE of a paragraph.
%        --- #1 should be a \vbox ---
%
%%%%%%%%%%%%%%%%%%%%%%%%%%%%%%%%%%%%%%%%%%%%%%%%%%%%%%%%%%%%%%%%%%%%%%%%
 
  \def\pix#1{\vadjust{\vbox{\kern 1.0cm #1 \kern 1.0cm}}}
 
%%%%%%%%%%%%%%%%%%%%%%%%%%%%%%%%%%%%%%%%%%%%%%%%%%%%%%%%%%%%%%%%%%%%%%%%
%
% These macros insert (\vbox es) using respectively \topinsert and
%     \pageinsert.  Since these macros put you into vertical mode,
%     they should only be called BETWEEN paragraphs.
%
%%%%%%%%%%%%%%%%%%%%%%%%%%%%%%%%%%%%%%%%%%%%%%%%%%%%%%%%%%%%%%%%%%%%%%%%
 
  \def\tpix#1{\topinsert #1 \endinsert}
  \def\ppix#1{\pageinsert #1 \endinsert}
 
%%%%%%%%%%%%%%%%%%%%%%%%%%%%%%%%%%%%%%%%%%%%%%%%%%%%%%%%%%%%%%%%%%%%%%%%%
%
%  To define a \vbox for an inserted figure, use the following command:
%
%%%%%%%%%%%%%%%%%%%%%%%%%%%%%%%%%%%%%%%%%%%%%%%%%%%%%%%%%%%%%%%%%%%%%%%%%
%
% \Picture\CompEP\height=4.125in \width=\hsize\caption{\narrower
%         The caption goes here.}
% \savepicture\FigureOne
%        (\Picture  calls \FIG and assigns the figure the caption
%                indicated and the figure number \CompEP.  It also
%                creates a \vbox of the right size for a picture
%                of the indicated height, and \savepicture assigns this
%                to the macro \FigureOne.)
%
 
%%%%%%%%%%%%%%%%%%%%%%%%%%%%%%%%%%%%%%%%%%%%%%%%%%%%%%%%%%%%%%%%%%%%%%%%
%
% TWO LINE MACRO (used in tables)
%
%%%%%%%%%%%%%%%%%%%%%%%%%%%%%%%%%%%%%%%%%%%%%%%%%%%%%%%%%%%%%%%%%%%%%%%%
 
\def\twoline#1:#2#3{\vtop{\hbox to #1{\hss \strut #2 \hss}
                          \hbox to #1{\hss \strut #3 \hss}}}
%
%
%%%%%%%%%%%%%%%%%%%%%%%%%%%%%%%%%%%%%%%%%%%%%%%%%%%%%%%%%%%%%%%%%%%%%%%%%
%
% Three nice dots left to right moving upward.
%
%%%%%%%%%%%%%%%%%%%%%%%%%%%%%%%%%%%%%%%%%%%%%%%%%%%%%%%%%%%%%%%%%%%%%%%%%
%
\def\mddots{\mathinner{\mskip1mu\raise1pt\hbox{.}\mskip2mu
   \raise4pt\hbox{.}\mskip2mu\raise7pt\vbox{\kern7pt\hbox{.}}\mskip1mu}}
%
 
%%%%%%%%%%%%%%%%%%%%%%%%%%%%%%%%%%%%%%%%%%%%%%%%%%%%%%%%%%%%%%%%%%%%%%%%%
%
% \clock returns time in hours:minutes on a 24 hour basis
% check \fullclock too
%
%%%%%%%%%%%%%%%%%%%%%%%%%%%%%%%%%%%%%%%%%%%%%%%%%%%%%%%%%%%%%%%%%%%%%%%%%
%
\newcount\timecount
\newcount\hours \newcount\minutes  \newcount\temp \newcount\pmhours
 
\hours = \time
\divide\hours by 60
\temp = \hours
\multiply\temp by 60
\minutes = \time
\advance\minutes by -\temp
\def\hour{\the\hours}
\def\minute{\ifnum\minutes<10 0\the\minutes
            \else\the\minutes\fi}
\def\clock{
\ifnum\hours=0 12:\minute\ AM
\else\ifnum\hours<12 \hour:\minute\ AM
      \else\ifnum\hours=12 12:\minute\ PM
            \else\ifnum\hours>12
                 \pmhours=\hours
                 \advance\pmhours by -12
                 \the\pmhours:\minute\ PM
                 \fi \fi \fi \fi }
 
\def\fullclock{\hour:\minute}
 
%%%%%%%%%%%%%%%%%%%%%%%%%%%%%%%%%%%%%%%%%%%%%%%%%%%%%%%%%%%%%%%%%%%%%%%%
%
% \draft causes the symbolic names of equations to be printed
%        alongside the equation numbers
%
%%%%%%%%%%%%%%%%%%%%%%%%%%%%%%%%%%%%%%%%%%%%%%%%%%%%%%%%%%%%%%%%%%%%%%%%
 
\let\eqnameold=\eqname
\def\draft{\def\eqname##1{\eqnameold##1:{\tt\string##1}}}
 
%%%%%%%%%%%%%%%%%%%%%%%%%%%%%%%%%%%%%%%%%%%%%%%%%%%%%%%%%%%%%%%%%%%%%%%
%
% Letter with no letterhead
% insert \blankheads after \letters
%
%%%%%%%%%%%%%%%%%%%%%%%%%%%%%%%%%%%%%%%%%%%%%%%%%%%%%%%%%%%%%%%%%%%%%%%
 
\def\blankheads{\let\letterhead=\blankletterhead
    \lettertopskip=7pt plus 3pt}
\def\blankletterhead{\hrule height 0pt \vskip 1.3in}
 
%%%%%%%%%%%%%%%%%%%%%%%%%%%%%%%%%%%%%%%%%%%%%%%%%%%%%%%%%%%%%%%%%%%%%%%%
%
%  this gives you a boldface character in math mode.
%     $\bold\beta$
%
%%%%%%%%%%%%%%%%%%%%%%%%%%%%%%%%%%%%%%%%%%%%%%%%%%%%%%%%%%%%%%%%%%%%%%%%
 
\def\bold#1{\setbox0=\hbox{$#1$}%
     \kern-.025em\copy0\kern-\wd0
     \kern.05em\copy0\kern-\wd0
     \kern-.025em\raise.0433em\box0 }
 
%%%%%%%%%%%%%%%%%%%%%%%%%%%%%%%%%%%%%%%%%%%%%%%%%%%%%%%%%%%%%%%%%%%%%%%%
%
%  Definition for a slash over a letter
%
%%%%%%%%%%%%%%%%%%%%%%%%%%%%%%%%%%%%%%%%%%%%%%%%%%%%%%%%%%%%%%%%%%%%%%%%
 
% from M. Peskin
\def\dslash{\not{\hbox{\kern-2pt $\partial$}}}
\def\Dslash{\not{\hbox{\kern-4pt $D$}}}
\def\Qslash{\not{\hbox{\kern-4pt $Q$}}}
\def\pslash{\not{\hbox{\kern-2.3pt $p$}}}
\def\kslash{\not{\hbox{\kern-2.3pt $k$}}}
\def\qslash{\not{\hbox{\kern-2.3pt $q$}}}
%
%
 \newtoks\slashfraction
 \slashfraction={.13}
 \def\slash#1{\setbox0\hbox{$ #1 $}
 \setbox0\hbox to \the\slashfraction\wd0{\hss \box0}/\box0 }
 
% EXAMPLE OF HOW TO USE IT
% $\slash D$
% {\slashfraction={.075} $\slash{\cal A}$}
% $\slash B$
% $\slash a$
% {\slashfraction={.09} $\slash p$}
% $\slash q$
 
%%%%%%%%%%%%%%%%%%%%%%%%%%%%%%%%%%%%%%%%%%%%%%%%%%%%%%%%%%%%%%%%%%%%%%%%%
%
% this gives you \leftrightarrow over \slash\partial
%
%%%%%%%%%%%%%%%%%%%%%%%%%%%%%%%%%%%%%%%%%%%%%%%%%%%%%%%%%%%%%%%%%%%%%%%%%
 
 \def\leftrightarrowfill{$\mathord-\mkern-6mu%
   \cleaders\hbox{$\mkern-2mu\mathord-\mkern-2mu$}\hfill
   \mkern-6mu\mathord\leftrightarrow$}
 \def\overlrarrow#1{\vbox{\ialign{##\crcr
       \leftrightarrowfill\crcr\noalign{\kern-1pt\nointerlineskip}
       $\hfil\displaystyle{#1}\hfil$\crcr}}}
 
%  $$ \overlrarrow{\slash\partial} $$
 
%%%%%%%%%%%%%%%%%%%%%%%%%%%%%%%%%%%%%%%%%%%%%%%%%%%%%%%%%%%%%%%%%%%%%%%%
%
%  This is a macro written by Marvin which makes Young tableaux:
%
%%%%%%%%%%%%%%%%%%%%%%%%%%%%%%%%%%%%%%%%%%%%%%%%%%%%%%%%%%%%%%%%%%%%%%%%%
 
  \def\yboxit#1#2{\vbox{\hrule height #1 \hbox{\vrule width #1
                  \vbox{#2}\vrule width #1 }\hrule height #1 }}
 
  \def\fillbox#1{\hbox to #1{\vbox to #1{\vfil}\hfil}}
  \def\ybox{\yboxit{0.4pt}{\fillbox{8pt}}\hskip-0.4pt}
  \def\nbox{\yboxit{0pt}{\fillbox{8pt}}}
 
  \def\tableaux#1{\vcenter{\offinterlineskip
                  \halign{&\tabskip 0pt##\cr #1}}}
  \def\cry{\cropen{-0.4pt}}
 
%%%%%%%%%%%%%%%%%%%%%%%%%%%%%%%%%%%%%%%%%%%%%%%%%%%%%%%%%%%%%%%%%%%%%%%%%
 
% An example of the use of \tableaux is:
% $$ X = \tableaux{\ybox & \ybox & \ybox \cry
%                  \ybox & \ybox & \nbox \cry
%                  \ybox & \nbox & \nbox \cry} $$
%
 
%%%%%%%%%%%%%%%%%%%%%%%%%%%%%%%%%%%%%%%%%%%%%%%%%%%%%%%%%%%%%%%%%%%%%%%%%
%
%       by M. Peskin
%       These macros make 4- and 5-pt skeleton diagrams
%            with labelled endpoints and internal lines.
%
%%%%%%%%%%%%%%%%%%%%%%%%%%%%%%%%%%%%%%%%%%%%%%%%%%%%%%%%%%%%%%%%%%%%%%%%%
 
  \def\fourptfcn#1#2#3#4#5{ \matrix{#1\cr\cr#2\cr}
      \hbox{$  \Bigg\rangle \kern -2.5pt
         {\phantom{whichcan}\over {\textstyle \downabit #5} }
           \kern-2.5pt  \Bigg\langle$} \matrix{#3\cr\cr#4\cr} }
  \def\fiveptfcn#1#2#3#4#5#6#7{ \matrix{#1\cr\cr#2\cr}
      \hbox{$  \Bigg\rangle \kern -2.5pt
        {{\pile} \lower  6pt\hbox{$\spire#5$} {\pile}\over
           \phantom{can}{\textstyle \downabit#6}\phantom{which}
             {\textstyle \downabit #7}\phantom{can}}
               \kern-2.5pt \Bigg\langle$} \matrix{#3\cr\cr#4\cr} }
  \def\spire#1{\matrix{#1\cr\Bigm|\cr}}
  \def\pile{\phantom{\matrix{1\cr2\cr3\cr}}}
  \def\downabit{\vphantom{\bigm|}}
 
%%%%%%%%%%%%%%%%%%%%%%%%%%%%%%%%%%%%%%%%%%%%%%%%%%%%%%%%%%%%%%%%%%%%%%%%%
 
\def\unlock{\catcode`@=11}   % This allows us to modify PLAIN macros.
%
\def\lock{\catcode`@=12}     % at signs are no longer letters
%
%%%%%%%%%%%%%%%%%%%%%%%%%%%%%%%%%%%%%%%%%%%%%%%%%%%%%%%%%%%%%%%%%%%%%%%%%
%
% These changes give you references on the line in brackets [3]
% and in the reference list.  [1]  A. Einstein ...
%
%
%\unlock
%\def\step@ver#1{\rel@x \ifmmode #1\else \ifhmode
%        \roll@ver{$\ {} #1$ }\else {\setbox0=\hbox{${} #1$ }}\fi\fi }
%\def\attach#1{\step@ver{\strut {\mkern 4mu #1} }}
%\def\normalrefmark#1{\attach{ [ #1 ] }}
%\def\NPrefmark#1{\step@ver{{\ [#1]}}}
%\def\refitem#1{\r@fitem{[#1]}}
%\lock
%%%%%%%%%%%%%%%%%%%%%%%%%%%%%%%%%%%%%%%%%%%%%%%%%%%%%%%%%%%%%%%%%%%%%%%%%
%
