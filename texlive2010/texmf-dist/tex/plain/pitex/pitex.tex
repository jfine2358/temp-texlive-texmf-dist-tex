% This is piTeX, a set of macros I (Paul Isambert) use to
% typeset documentations for my packages (that's why it is
% archived on CTAN).
%
% Perhaps in the future, when this achieves some kind of
% format-like completude, there'll be a documentation and
% it'll be publicly announced.
%
% You can of course use those macros, but you are on your
% own, and the files will probably be modified without announcement.
% The file is supposed to be \input on plain TeX with LuaTeX, at least v.0.6.
%
%
% The files needed are:
%
% texapi.tex (an independent package \input in the following file)
% yax.tex (an independent package)
% fonts.ptx (piTeX font "management")
% luaotfload.sty (an independent package (not by me))
% sections.ptx (piTeX sectionning commands)
%
% Date: July 2010.
%
%
% User interface
\input yax % which itself \input's texapi

\setcatcodes{\@\_=11}
\def\ptx@error{\senderror{PiTeX}}

% This to write in latin1.
\directlua{%
 function convert(buf)
   return string.gsub(buf,"(.)", function (ch)
       return unicode.utf8.char(string.byte(ch))
     end)
 end
 callback.register('process_input_buffer',convert)
 }

\newdimen\urvsize
\urvsize\vsize

\def\inputpitexfile#1 {\input #1.ptx }

\inputpitexfile fonts
\inputpitexfile sections
% No footnotes in documentations.
%\input footnotes.tex
% I redefine the output routine on every job.
%\input output.tex

\defactiveparameter document{%
  \pdfinfo{%
    /Author (\usevalueor #1 : author {Unknown author})
    /Title  (\usevalueor #1 : pdftitle  {\usevalueor #1 : title {No title}})
    \ifattribute #1 : subject {/Subject (\usevalue #1 : subject )}{}
    \ifattribute #1 : keywords {/Keywords (\usevalue #1 : keywords )}{}
    }%
  \pdfcatalog{%
    /PageMode
      \ifcasevalue #1 : display
        \val outlines   /UseOutlines
        \val signets    /UseOutlines % French!
        \val thumbs     /UseThumbs
        \val thumbnails /UseThumbs
        \val full       /FullScreen
        \val fullscreen /FullScreen
        \elseval        /UseNone
      \endval
    }%
  }

\restrictparameter document : author title pdftitle date subject keywords display version\par
\restrictattribute document:display outlines signets thumbs thumbnails full fullscreen\par


\defactiveparameter page{%
  \settovalue\pdfpagewidth  #1 : pagewidth
  \settovalue\pdfpageheight #1 : pageheight
  \settovalue\baselineskip  #1 : baselineskip
  \settovalueor\topskip     #1 : topskip {\topskip=\baselineskip}%
  \settovalue\pdfhorigin    #1 : left
  \ifattribute #1 : right
              {\hsize = \dimexpr(\pdfpagewidth-\pdfhorigin-\usevalue #1 : right )\relax}
              {\settovalue\hsize #1 : hsize }%
  \settovalue\pdfvorigin    #1 : top
  \settovalue\parindent     #1 : parindent
  \settovalue\parskip       #1 : parskip
  \ifattribute #1 : lines {\vsize=\usevalue #1 : lines \baselineskip}{}%
  }
\restrictparameter page : pagewidth
                          pageheight
                          hsize
                          baselineskip
                          topskip
                          left
                          right
                          top
                          lines
                          parindent
                          parskip
                          \par

% Defaults... they don't produce anything beautiful.
% I redefine them on every job.
\setparameter page:
  pageheight   = 28cm
  pagewidth    = 21cm
  hsize        = 15cm
  baselineskip = 12pt
  lines        = 42
  left         = 1in
  top          = 1in
  parskip      = 0pt


% Should be used with YaX to set parameters.
\long\pdef\newblock{%
  \ifnext*{\gobbleoneand\ptx@newblock_group}
          {\ptx@newblock_nogroup}%
  }
\def\ptx@newblock_nogroup#1#2#3{%
  \def#1{\ifnext/{\gobbleoneand{#3}}{#2}}%
  }
\def\ptx@newblock_group#1#2#3{%
  \def#1{\ifnext/{\gobbleoneand{#3\egroup}}{\bgroup#2}}%
  }
\def\antigobblespace{%
  \ifcatnext a{ }{\iffnext({ }}%
  }


% PDF links
\def\dest#1#2{%
  \ifcs{ptx@link_name=#1}
       {\ptx@error{Link `#1' already exists}#2}
       {\letcs{ptx@link_name=#1}\relax
        \bgroup
          \ifvmode
            \pdfdest name {#1} xyz #2%
          \else
            \setbox0=\hbox{#2}%
            \raise\ht0\hbox{\pdfdest name {#1} xyz}#2%
          \fi
        \egroup
        }%
  }
\def\link#1#2{%
  \pdfstartlink attr {/Border [0 0 0]} goto name {#1}#2\pdfendlink
  }

% Verbatim facilities.

\def\tcode#1{{\codefont#1}}
\long\def\com#1{%
  \bgroup
    \codefont
    \string#1%
  \egroup
  \antigobblespace
  }
\def\arg#1{{\codefont\it<#1>}\iffnext\spacechar{\kern.2ex }}
\def\barg#1{{\codefont\char"007B\relax{\it<#1>}\char"007D\relax}}
\def\oarg#1{{\codefont[{\it<#1>}]}}


\def\exampleskip{\vskip\baselineskip}
\def\example{%
  \exampleskip
  \bgroup
  \parindent0pt
  \setcatcodes{\\\{\}\$\&\#\^\_\ \~\%=12,\^^M=13}%
  \codefont
  \readexample
  }%
\bgroup
\setcatcodes{\\=12,\|=0,\^^M=13}%
|gdef|readexample#1^^M#2\example/{#2|egroup|exampleskip}%
|restorecatcodes%
\setcatcodes{\^^M=13}%
\gdef^^M{\quitvmode\endgraf}%
\egroup%
\def\verb#1{%
  \def\ptx@verb##1#1{##1\egroup}%
  \bgroup
  \setcatcodes{\\\{\}\$\&\#\^\_\ \~\%=12}%
  \codefont
  \ptx@verb
  }

%
% Read and process examples.
%
\newwrite\ptx@examplewrite

\bgroup
\setcatcodes{\%=12,\/=14}
\gdef\Example{/
  \immediate\openout\ptx@examplewrite=\jobname.pex
  \immediate\write\ptx@examplewrite{% This is a scratch file used to typeset examples in \jobname.tex.}/
  \immediate\write\ptx@examplewrite/
  \bgroup
  \setcatcodes{\\\#\%\^^M\ =12}/
  \writeexample
  }/
\egroup

\bgroup
\setcatcodes{\^^M=12}%
\gdef\writeexample#1^^M{%
  \immediate\write\ptx@examplewrite{\noexpand\example}%  
  \ptx@writeexample}%
\gdef\ptx@writeexample#1^^M{%
  \passexpanded{\ifstring{#1}}{\examplestring}%
               {\immediate\write\ptx@examplewrite{\string\example/}%
                \immediate\closeout\ptx@examplewrite\egroup%
                \processexample}%
               {\immediate\write\ptx@examplewrite{\primunexpanded{#1}}%
                \ptx@writeexample}%
  }%
{\setcatcodes{\\=12,\|=0}|gdef|examplestring{\Example/}}%
\egroup

\def\typesetexample{{\input\jobname.pex }}
\def\doexample{{\let\example\ptx@Example\input\jobname.pex }}
\def\ptx@Example{%
  \iffnext/{\gobbleoneand\endinput}%
  }
% To be redefined for different layout.
\def\processexample{\typesetexample\doexample}

\def\verb#1{%
  \def\ptx@verb##1#1{##1\egroup}%
  \bgroup
  \setcatcodes{\\\{\}\$\&\#\^\_\ \~\%=12}%
  \codefont
  \ptx@verb
  }


\newdimen\extraboxspace
\newdimen\ptx@extraboxspace_top
\newdimen\ptx@extraboxspace_right
\newdimen\ptx@extraboxspace_bottom
\newdimen\ptx@extraboxspace_left

\newfornoempty\ptx@colorbox_loop{1}#2,{%
  \ifcase#1
    \ptx@extraboxspace_top    =#2
    \ptx@extraboxspace_right  =#2
    \ptx@extraboxspace_bottom =#2
    \ptx@extraboxspace_left   =#2
  \or
    \ptx@extraboxspace_right  =#2
    \ptx@extraboxspace_left   =#2
  \or
    \ptx@extraboxspace_bottom =#2
  \or
    \ptx@extraboxspace_left   =#2
  \fi
  \passarguments{\numexpr(#1+1)}%
  }
\def\colorbox{%
  \ifnextnospace[\ptx@colorbox_setborders
          {\ptx@extraboxspace_top    =\extraboxspace
           \ptx@extraboxspace_right  =\extraboxspace
           \ptx@extraboxspace_bottom =\extraboxspace
           \ptx@extraboxspace_left   =\extraboxspace
           \ptx@colorbox_do}%
  }
\def\ptx@colorbox_setborders[#1]{%
  \ptx@colorbox_loop{0}{#1,}%
  \ptx@colorbox_do
  }
{\setcatcodes{pt=12}
\gdef\noPT#1pt{#1 }}
\def\ptx@colorbox_do#1#2{%
  \bgroup
  \setbox0=\hbox{#2}%
  \hbox{%
    \pdfliteral{
      q #1 rg #1 RG
      -\expandafter\noPT\the\ptx@extraboxspace_left
       \expandafter\noPT\the\dimexpr(\ht0+\ptx@extraboxspace_top)\relax
       \expandafter\noPT\the\dimexpr(\wd0+\ptx@extraboxspace_left+\ptx@extraboxspace_right)\relax
      -\expandafter\noPT\the\dimexpr(\ht0+\ptx@extraboxspace_top+\dp0+\ptx@extraboxspace_bottom)\relax
      re f Q}%
    #2}%
  \egroup
  }

\def\trace{\tracingcommands4 \tracingmacros4 }
\def\untrace{\tracingcommands0 \tracingmacros0 }

\restorecatcodes