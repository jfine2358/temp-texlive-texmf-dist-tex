	\def\IDENT{01-016}
%%%%%%%%%%%%%%%%%%%%%%%%%%%%%%%%%%%%%%%%%%%%%%%%%%%%%%%%%%%%%%%%%%%%%%%%%%%%
% Copyright 1989--2004 by Hunter Goatley.
%
% Permission is granted to anyone to use this software for any purpose
% on any computer system, and to redistribute it freely, subject to the
% following restrictions:
%
% 1. This software is distributed in the hope that it will be useful,
%    but WITHOUT ANY WARRANTY; without even the implied warranty of
%    MERCHANTABILITY or FITNESS FOR A PARTICULAR PURPOSE.
%
% 2. Altered versions must be plainly marked as such, and must not be
%    misrepresented as being the original software.
%
% End
%%%%%%%%%%%%%%%%%%%%%%%%%%%%%%%%%%%%%%%%%%%%%%%%%%%%%%%%%%%%%%%%%%%%%%%%%%%%
%
%  File:	NEWSLETTER_FORMAT.TEX
%
%  Abstract:
%
%	This file defines TeX control sequences required to produce a
%	newsletter.  It assumes plain.tex has been loaded.
%
%  Author:	Hunter Goatley
%		E-mail: goathunter@goatley.com
%
%		Partially based on examples from _The TeXbook_, by
%		Donald E. Knuth, and various other sources.  Virtually
%		all of the macros from other sources have been rewritten
%		or at least modified.
%
%  Date:	June 3, 1989
%
%  Modified by:
%
%	01-016		Hunter Goatley		22-JAN-2004 09:23
%		Changed the license for the code to make it truly free.
%
%	01-015		Hunter Goatley		28-JAN-2001 15:45
%		I'm back!  Eliminate undesired whitespace in header and
%		footer macros.
%
%	01-014		Hunter Goatley		21-AUG-1991 10:47
%		Rewrote double-quote macros (they work now!).  Cleaned up
%		a little bit.  Added \slant and \ital.
%
%	01-013		Hunter Goatley		25-JAN-1991 08:23
%		Added \newpage.  Added \tt definitions to \*point macros.
%		Fixed problem with \quotee (quotee name was getting split).
%		Added setting of \hyphenpenalty and \emergencystretch to
%		\newspage macro.
%
%	01-012		Hunter Goatley		15-JAN-1991 07:03
%		Added macro \round to help with keeping page & figure heights
%		even.  It helps some.  Re-worked \beginquote and \endquote.
%		Add \Quote and \quotee.  Added \listindent and dimens.
%		Modified \coltitle to include \noindent; changed amount of
%		\vglue.  Modified \beginlist and \endlist to check for
%		a \parskip of 0pt; if 0, skip .5\baselineskip.  Fixed
%		\centerbox (was \vbox, now \hbox).
%
%	01-011		Hunter Goatley		 5-JAN-1991 23:40
%		Changed \ednote macro so that \sl is redefined as \rm, not \ss.
%		Changed \say macro to look like LaTeX's \typeout macro.
%
%	01-010		Hunter Goatley		 6-OCT-1990 02:55
%		Commented out \ss calls at end of font macros.
%		Placed newsletter hsize commands in a macro (\newspage)
%		so that NEWTEX can be used like TeX normally is.
%
%	01-009		Hunter Goatley		10-MAR-1990 16:05
%		Added \par to the beginning of definition of \endlist.
%
%	01-008		Hunter Goatley		 7-DEC-1989 22:07
%		Added \farright (place hbox flush right or on next line if it
%		won't fit).  Modified definition of \eoa to call \farright.
%		Added \pmb -- "poor man's bold."  Added font cmssxb.  Added
%		\eldots and \edots.
%
%	01-007		Hunter Goatley		 1-OCT-1989 03:41
%		Added figure support for multiple column environment.
%		Added \centerbox.
%
%	01-006		Hunter Goatley		17-AUG-1989 10:11
%		Modified so that "@" is an active character throughout the
%		format file.  Added new header/footer commands that are more
%		flexible.  Improved appearance of shadow box created by
%		\leftshadowbox.
%
%	01-005		Hunter Goatley		10-AUG-1989 22:09
%		Modified \onepageout output routine so that it does not
%		disable interline skip (commented out \offinterlineskip).
%		This was causing the header and footer to appear flush
%		with the main body of text on the page.  Added routine
%		\checkdqbalsub.  Modified \include to automatically call
%		\checkdqbalsub after the file has been read in.
%
%	01-004		Hunter Goatley		30-JUL-1989 03:10
%		Added macros to handle font size changes.  Added code to
%		let " be used.
%
%	01-003		Hunter Goatley		29-JUL-1989 19:26
%		Modified double column routines so that the number of columns
%		can be specified.  The routines now work for 2 to 6 columns
%		of output.
%
%	01-002		Hunter Goatley		23-JUL-1989 22:21
%		Added more comments.  Fixed double column routines so that
%		page sizes are handled properly.
%
%	01-001		Hunter Goatley		 2-JUL-1989 21:28
%		Added macros to handle bibliographies.
%
%	01-000		Hunter Goatley		 3-JUN-1989 14:56
%		Original version.
%

\def\say#1{{\let\protect\string\immediate\write10{#1}}}

\say{TeX Input file for Newsletter format - version \IDENT}
\say{Copyright 1989-2004,  Hunter Goatley}

\everyjob{\say{TeX Newsletter version \IDENT. Copyright 1989-2004, Hunter Goatley}}

%
%  The \catcode command below lets us use "@" as a letter.  It can therefore
%  be used in command and variable names.  PLAIN TeX uses this to avoid
%  conflicts with user code, so we'll do it, too.
%
\catcode`@=11			% borrow the private macros of PLAIN (with care)

\say{Loading fonts...}
%
%  Load fonts and define commands to switch between fonts
%
\font\twelverm=cmr12  \font\tenrm=cmr10  \font\ninerm=cmr9  \font\eightrm=cmr8
\font\twelvei=cmmi12  \font\teni=cmmi10  \font\ninei=cmmi9  \font\eighti=cmmi8
%\font\twelvesy=cmsy12 \font\tensy=cmsy10 \font\ninesy=cmsy9 \font\eightsy=cmsy8
\font\twelvebf=cmbx12 \font\tenbf=cmbx10 \font\ninebf=cmbx9 \font\eightbf=cmbx8
\font\twelvett=cmtt12 \font\tentt=cmtt10 \font\ninett=cmtt9 \font\eighttt=cmtt8
\font\twelveit=cmti12 \font\tenit=cmti10 \font\nineit=cmti9 \font\eightit=cmti8
\font\twelvesl=cmsl12 \font\tensl=cmsl10 \font\ninesl=cmsl9 \font\eightsl=cmsl8
\font\twelvess=cmss12 \font\tenss=cmss10 \font\niness=cmss9 \font\eightss=cmss8
\font\twelvessi=cmssi12 \font\tenssi=cmssi10 \font\ninessi=cmssi9
\font\eightssi=cmssi8  \font\tenssb=cmssbx10
%
%  11-point font is scaled 10pt
%
\font\elevenrm=cmr10 scaled\magstephalf
\font\eleveni=cmmi10 scaled\magstephalf
\font\elevenbf=cmbx10 scaled\magstephalf
\font\eleventt=cmtt10 scaled\magstephalf
\font\elevenit=cmti10 scaled\magstephalf
\font\elevensl=cmsl10 scaled\magstephalf
\font\elevenss=cmss10 scaled\magstephalf
\font\elevenssi=cmssi10 scaled\magstephalf

\font\seventeenrm=cmr17  \font\seventeenss=cmss17  \font\seventeenssi=cmssi17

\def\seventeenpoint{%
	\def\sl{\seventeenssi}\def\it{\seventeenssi}\def\bf{\seventeenss}%
	\def\rm{\seventeenrm}\def\ss{\seventeenss}\def\ssi{\seventeenssi}%
	\baselineskip=19pt%			% Change baselineskip
	\rm%					% By default, use sans serif
}
\def\twelvepoint{%
	\def\sl{\twelvesl}\def\it{\twelveit}\def\bf{\twelvebf}%
	\def\rm{\twelverm\let\sl=\twelvesl}\def\ss{\twelvess\let\sl=\ssi}%
	\def\ssi{\twelvessi}\def\tt{\twelvett}%
	\baselineskip=14pt%			% Change baselineskip
	\rm%					% By default, use sans serif
}
\def\elevenpoint{%
	\def\sl{\elevensl}\def\it{\elevenit}\def\bf{\elevenbf}%
	\def\rm{\elevenrm\let\sl=\elevensl}\def\ss{\elevenss\let\sl=\ssi}%
	\def\ssi{\elevenssi}%
	\def\tt{\eleventt}%
	\baselineskip=13pt%			% Change baselineskip
	\rm%					% By default, use sans serif
}
\def\tenpoint{%
	\def\sl{\tensl}\def\it{\tenit}\def\bf{\tenbf}%
	\def\ssb{\tenssb}%
	\def\rm{\tenrm\let\sl=\tensl}\def\ss{\tenss\let\sl=\ssi}%
	\def\ssi{\tenssi}%
	\def\tt{\tentt}%
	\baselineskip=12pt%			% Change baselineskip
	\rm%					% By default, use sans serif
}
\def\ninepoint{%
	\def\sl{\ninesl}\def\it{\nineit}\def\bf{\ninebf}%
	\def\rm{\ninerm\let\sl=\ninesl}\def\ss{\niness\let\sl=\ssi}%
	\def\ssi{\ninessi}%
	\def\tt{\ninett}%
	\baselineskip=11pt%			% Change baselineskip
	\rm%					% By default, use sans serif
}
\def\eightpoint{%
	\def\sl{\eightsl}\def\it{\eightit}\def\bf{\eightbf}%
	\def\rm{\eightrm\let\sl=\eightsl}\def\ss{\eightss\let\sl=\ssi}%
	\def\ssi{\eightssi}%
	\def\tt{\eighttt}%
	\baselineskip=10pt%			% Change baselineskip
	\rm%					% By default, use sans serif
}

\font\HUGE=cmr17 %scaled\magstep2
\font\verysmallrm=cmr5			% Used to make small copyright "C"
\font\smallsy=cmsy7			% Used to make small copyright circle
\def\tiny{\eightpoint\ss}		% Equate \tiny to \eightpoint
\let\em=\eightssi
\font\quotefont=cmss12 at 14.4truept	% Quotation font
\font\quoteefont=cmcsc10		% Font for quotee

\say{Defining macros...}

\newdimen\normalhsize			% Create a new dimension
\newdimen\normalvsize			% Create a new dimension

\normalhsize=\hsize

%
%  \newspage
%
%	Set up the page for a newsletter (7" by 9" of text, higher tolerance).
%
\def\newspage{%
	\global\topskip=0pt		% Set 1 inch top margin
	\global\hoffset=-.25 true in	% Move output .25 in to the left
	\global\pretolerance=1000	% Set tolerance up (before hyphenation)
	\global\tolerance=1500		% ...  (after hyphenation)
	\global\hyphenpenalty=500	% Discourage hyphenation
	\global\emergencystretch=30pt
	\global\normalhsize=7in
	\global\hsize=\normalhsize	% Pages are 7 inches wide
	\global\vsize=9in		% ...  and 9 inches tall
	\global\abovedisplayskip=\baselineskip
	\global\belowdisplayskip=\baselineskip
	\global\pagewidth=\hsize
	\global\pageheight=\vsize
}

\clubpenalty=500 %-1000			% Set penalties for club and widow lines
\widowpenalty=1000			% ...

%
%=-=-=-=-=-=-=-=-=-=-=-=-=-=-=-=-=-=-=-=-=-=-=-=-=-=-=-=-=-=-=-=-=-=-=-=-=-=-=-=
%                           Multiple Column Output
%=-=-=-=-=-=-=-=-=-=-=-=-=-=-=-=-=-=-=-=-=-=-=-=-=-=-=-=-=-=-=-=-=-=-=-=-=-=-=-=
%
%	Define macros to handle multiple column output
%
%	These macros support figures, but the figures must be defined inside
%	the multiple-column environment.  Each column can have up to 3 figures
%	of varying sizes; the figures are referred to as "top", "middle", and
%	"bottom".  Figures are identified via page number, column number, and
%	the position.  In addition, two special figures that span all columns
%	can be specified using column 0 with "top" and "bottom".
%
%	For example, {3}{2}{middle}  refers to the middle figure in column 2
%	of output page 3.
%
%	The figures are stored as temporary inserts (the inserts are freed
%	after \endcolumns).  The inserts are treated as an array whose first
%	entry is \c@lfigstart.
%
%	Note: column figures are not handled on the last page of column output.
%	\balancecolumns does not understand how to deal with them.
%
%	While in X column mode, the text is arranged as one column
%	\columnwidth wide and \pageheight*X tall.  This narrow column is
%	then split into X boxes of equal height.
%
%***************************************
%
%     Allocate a bunch of dimens, boxes, and counts for use in multiple-column
%     macros.
%
\newdimen\columnwidth			% Width of a column
\newdimen\columnseprule			% Width of rule between columns
\newdimen\columnsep			% Width of whitespace between columns
\newdimen\pagewidth			% Total width of page
\newdimen\pageheight			% Total height of page
\newdimen\pageheightkeep		% Total height of page
\newdimen\ruleht			% Height of \hrules
\newif\ifcolfigs			% Create a new \if (true if figures)
\newbox\partialpage			% Box to hold partial page before cols.
\newbox\partialpagetop			% Box to hold figure for top of page
\newbox\partialpagebottom		% Box to hold figure for bottom of page
\newcount\mulc@lbegin			% Allocate a counter
\newcount\mulc@lpart 			% Allocate a counter
\newcount\numberofcols			% Number of columns
\newcount\c@lpageno			% Current page # for column environment
\newcount\maxcolfigs			% Maximum # of figures
\newcount\c@lfigstart			% Starting array slot # for inserts
\newcount\c@lslot			% Work counter to hold current slot #
\newcount\maxfigp@ges			% Maximum # of pages of column output
\newcount\figsperp@ge			% Number of figures on each page
%
%  Allocate multi-column work counters and dimens.
%
\newdimen\tmp@			% Dimen used to calculate pageheights, etc.
\newdimen\tmp@one		% Dimen used to calculate pageheights, etc.
\newcount\tmp@two		% Usually # of cols * 2
\newcount\tmp@three
\newcount\tmp@four
\newcount\tmp@five
\newcount\tmp@six
\newcount\tmp@seven
\newdimen\rtmp@			% Dimen used in rounding
%
%  Initialize the \column* dimens.  If \columnwidth is still 0pt when
%  \begincolumns is called, the correct \columnwidth will be
%  calculated from the current \hsize.  This lets the user set a column
%  width, without requiring that it be done.
%
\columnsep=20pt				% Space between columns
\columnseprule=.4pt			% Width of rule between columns
\columnwidth=0pt			% Initialize columnwidth to 0pt
%
% Initialize pagewidth, pageheight, and ruleht
%
\pagewidth=\hsize \pageheight=\vsize \ruleht=.5pt
%
%  Define constant values used to indicate type of figure stored in
%  figure boxes.  Probably inefficient to store them as macros, but it's
%  easier that way (and they're not hardcoded in the macros below).
%
\def\TopFig@{500}				% Figure is at top of page
\def\BotFig@{600}				% Figure is at bottom of page
\def\RegFig@{321}				% Figure is a corner figure
\def\topc@lpos{1}				% The top figure position (1)
\def\middlec@lpos{2}				% The middle figure position (2)
\def\bottomc@lpos{3}				% The bottom figure position (3)

%
%  Macro \onepageout
%
%	Output routine used to actually shipout pages to the DVI file.
%
\def\onepageout#1{\shipout\vbox{ % here we define one page of output
    {\hsize=\pagewidth\makeheadline}		% Do normal width headline
    %\offinterlineskip				% butt the boxes together
    \vbox to \pageheightkeep{			% Create a vbox big enough
      \boxmaxdepth=\maxdepth			% Set boxmaxdepth
      #1					% Now insert the information
      }						% \vbox to \pageheight
    {\hsize=\pagewidth\makefootline}		% Do normal width footline
    }						% End of \vbox
  \advancepageno				% Advance current page number
  }						% End of \def\onepageout

%\output{\onepageout{\unvbox255}}		% Send out any current output

%
%  When \begincolumns is called:
%
%  \output is set to perform the following functions:
%	Store the current vbox in the partialpage box (with some vskip)
%	Eject the page (executing \output - page is not really ejected)
%  Set new definition of output:
%	Call \multiplecolumnout to output the columns
%  Set \hsize = \columnwidth
%  Set \vsize = (\pageheight * X) - (\ht\partialpage * X) --- each column will
%	now be just as tall as the space below \partialpage
%
%   Sometimes the first \output was executed twice in a row, which caused the
%   first \partialpage to be lost.  The code below includes a counter that is
%   used to determine if the output routine is called a second time.  If it
%   is, the routine ships the previously stored partial page and then creates
%   a new \partialpage.
%
%  Originally, \begincolumns set \vsize = X * \pageheight (it did not
%  take the height of the \partialpage into account).  If the complete text
%  would fit in the full \vsize, the balancing routine would have problems
%  balancing and you'd end up with lots of whitespace on a page and the
%  multiple columns would show up on the next page.  Make sense?  If not,
%  just trust me; it caused problems.
%
%  The solution to the problem was to set
%
%	\vsize = (\pageheight * X) - (\ht\partialpage * X)
%
%  and then set \vsize back to (X*\pageheight) after the first page had been
%  ejected.  This worked fine and dandy except for the fact that grouping was
%  causing the changes to not hold.  The solution (and it is a little
%  dangerous) was to use \global\vsize so that the \vsize changes were never
%  local to the group.  \endcolumns then resets \vsize = \pageheight.
%  Again, trust me; it seems to work OK.
%
\def\begincolumns#1{
  \ifnum#1<2 \errmessage{Invalid number of columns -- #1; must be 1 < X < 7}\fi
  \ifnum#1>6 \errmessage{Invalid number of columns -- #1; must be 1 < X < 7}\fi
  \begingroup				% Begin a new group
  \global\mulc@lbegin=1			% Set "first time" output counter
  \global\mulc@lpart=1			% Set "first time" outputting dbl cols.
  \global\pageheightkeep=\vsize		% Initialize \pageheight
  \global\pageheight=\vsize		% Initialize \pageheight
  \round{\pageheight}{\baselineskip}{<}	% Round the pageheight down
  \global\pagewidth=\hsize		% Initialize \pagewidth
  \global\tmp@two=#1  \global\advance\tmp@two by-1
  \global\numberofcols=#1		% Initialize number of columns
  %  Calculate column width unless user set a value
  \ifdim\columnwidth=0pt		% If user did not set \columnwidth
     \columnwidth=\hsize		% Set columnwidth = normal page width
     \tmp@=-\columnsep			% Copy neg. amount of columnsep to tmp@
     \multiply\tmp@ by\tmp@two		% Multiply by X (total amount of colsep)
     \advance\columnwidth by\tmp@	% Subtract width space between columns
     \divide\columnwidth by\numberofcols  % Divide by X to get X column output
  \fi					% End of \columnwidth calculation
  \output={
        \ifnum\mulc@lbegin=1				% 1st time through?
	  \global\setbox\partialpage=\vbox{\unvbox255\bigskip}	% Store partial
	  \global\mulc@lbegin=2				% Increment counter
	\else						% 2nd time through...
	  \onepageout{\vbox{\unvbox\partialpage}}	% Ship previous partial
	  \global\setbox\partialpage=\vbox{\unvbox255\bigskip}	% Make a new
	\fi							% ... partial
	}\eject				% Force the output routine to execute
  \output={\multiplecolumnout}				% Output X columns
  \hsize=\columnwidth					% Set hsize = col. width
  % Set \vsize = (\pageheight * X) - (\ht\partialpage * X)
  \global\vsize=\pageheight				% Set \vsize=X*\pageht
  \global\multiply\vsize by\numberofcols		% Multiply by X
  \tmp@=-\ht\partialpage \multiply\tmp@ by\numberofcols	% Subtract (height of
  \global\advance\vsize by\tmp@				% ... partial page * X)
  \round{\vsize}{\baselineskip}{>}			% Round it up
  }							% End of \begincolumns

%
%  \endcolumns performs the following functions:
%
%    Sets \output to call \balancecolumns
%    \vfills page and ejects it
%    Terminates the group
%    Resets vsize to original size
%    Resets pagegoal = to original \vsize
%    Skips parskip vertical space
%    Signals that this is a good place for a break, if necessary
%
\def\endcolumns{\output={\balancecolumns} \eject
  \endgroup \global\vsize=\pageheightkeep \pagegoal=\vsize \bigskip \goodbreak}
%
%  \multiplecolumnout
%
%    Sets splittopskip = topskip
%    Sets splitmaxdepth = maxdepth
%    Sets TeX register \dimen@ = to height of the page
%    Subtracts the height of the partial page from \dimen@
%    Subtracts the heights of the top & bottom figures from \dimen@
%    Splits up the current output box into X boxes of size \dimen@, handling
%	figures, if present
%    Calls \onepageout to dump the new boxes
%    Resets \vsize = (X * \pageheight) if first time through macro
%    Frees up current output vbox (255)
%
\def\multiplecolumnout{\splittopskip=\topskip \splitmaxdepth=\maxdepth
  \dimen@=\pageheight \advance\dimen@ by-\ht\partialpage
  \advance\dimen@ by-\ht\partialpagetop		% Subtract height of top figure
  \advance\dimen@ by-\ht\partialpagebottom	% Subtract height of bottom fig.
  \tmp@four=\numberofcols   \multiply\tmp@four by2	% Calculate upper box #
  \tmp@five=0					% Start boxes with box 0
  \ifcolfigs					% Are there figures defined?
     \splitfigc@ls				% Yes - go handle text & figures
  \else						% No - split \box255 into X cols
    %
    %  Split box255 into a box of dimen@ height.  Loop until X boxes created.
    %
    {\loop
	\global\setbox\tmp@five=\vsplit255 to\dimen@  \advance\tmp@five by2
	\ifnum\tmp@five<\tmp@four \repeat}	% Loop if not done
  \fi						% \ifcolfigs
  \onepageout\pagesofar				% Send this page to DVI file
  \ifnum\mulc@lpart=1				% Does vsize need to be reset?
	\global\mulc@lpart=2			% Yes - change counter
	\global\vsize=\pageheight		% Set \vsize = X * \pageheight
	\global\multiply\vsize by\numberofcols	% ...
  \fi						% ...
  \ifcolfigs					% If figures defined...
     \global\advance\c@lpageno by\@ne		% Bump column output page no.
     \handlefigures				% Handle next page's figures
  \fi						% \ifcolfigs
  \unvbox255 \penalty\outputpenalty		% Free current output vbox
}						% End of \multiplcolumnout

%
%  \splitfigc@ls
%
%  This macro splits \box255 into X columns, handling the figures.
%
%    Get figure array slot number for figure 1, column 1 on current page
%    Loop for each column (starting with column in \tmp@five)
%	Make a copy of \dimen@ -> \tmp@
%	If a figure is defined for top of column
%	   Decrease \tmp@ by the height of the figure
%	   Set box 1 = the figure
%	 Else
%	   Set box 1 = null vbox
%	Bump figure array slot number - now points to slot for middle figure
%	If a figure is defined for middle of column
%	   Decrease \tmp@ by the height of the figure
%	   Set box 5 = the figure
%	 Else
%	   Set box 5 = null vbox
%	Bump figure array slot number - now points to slot for bottom figure
%	If a figure is defined for bottom of column
%	   Decrease \tmp@ by the height of the figure
%	   Set box 9 = the figure
%	 Else
%	   Set box 9 = null vbox
%	Bump figure array slot number - now points to slot for next top figure
%	Split \box255 into 1 or 2 pieces (2 if there is a middle figure)
%	Set \box\tmp@five = \box1 + \box3 + \box5 + \box7 + \box9
%	Advance \tmp@five by 2 and loop until X "columns" created
%
\def\splitfigc@ls{
  %
  %  Split box255 into a box of dimen@ height.  Loop until X boxes created.
  %
  \tmp@two=1				% Figure counter
  \calcc@lslot{\c@lpageno}{1}{\tmp@two}	% Get slot for figure 2 on current page
  {\loop				% Start loop
	\tmp@=\dimen@			% Make a working copy of dimen@
	%
	%  Handle top figure
	%
	\ifnum\count\c@lslot=\RegFig@	% If this slot holds a single-column fig
	   \advance\tmp@ by-\ht\c@lslot	% Subtract figure's height from \tmp@
	   \setbox1=\vbox{\unvbox\c@lslot}	% Copy figure to \box1
\say{Figure \the\tmp@two\space on page \the\c@lpageno, document page \folio}
	\else	\setbox1=\vbox{}	% Else set box 1 = null page
	\fi				% \ifnum
	%
	%  Handle middle figure
	%
	\advance\c@lslot by\@ne			% Point to next figure
	\advance\tmp@two by\@ne			% Advance figure counter
	\ifnum\count\c@lslot=\RegFig@	% If this slot holds a single-column fig
	   \advance\tmp@ by-\ht\c@lslot	% Subtract figure's height from \tmp@
	   \setbox5=\vbox{\unvbox\c@lslot}	% Copy figure to \box3
\say{Figure \the\tmp@two\space on page \the\c@lpageno, document page \folio}
	\else	\setbox5=\vbox{}	% Else set box 3 = null page
	\fi				% \ifnum
	%
	%  Handle bottom figure
	%
	\advance\c@lslot by\@ne		% Point to next figure
	\advance\tmp@two by\@ne		% Advance figure counter
	\ifnum\count\c@lslot=\RegFig@	% If this slot holds a single-column fig
	   \advance\tmp@ by-\ht\c@lslot	% Subtract figure's height from \tmp@
	   \setbox9=\vbox{\unvbox\c@lslot}	% Copy figure to \box3
\say{Figure \the\tmp@two\space on page \the\c@lpageno, document page \folio}
	\else	\setbox9=\vbox{}	% Else set box 3 = null page
	\fi				% \ifnum
	\advance\c@lslot by\@ne		% Point to next figure in array
	\advance\tmp@two by\@ne		% Advance figure counter
	%
	%  Here, top figure is in \box1, middle figure is in \box5, and bottom
	%	figure is in \box9.
	%  Split \box255 to fill the remaining column space, splitting in
	%	half if there is a middle figure.
	%  Special case: if height of \box1 = \pageheight, \box\tmp@five=null
	%
	\ifdim\ht1=\pageheight			% Is top = whole column?
	   \global\setbox3=\vbox{}		% Yes - set \box3 to null
	   \global\setbox7=\vbox{}		% Yes - set \box7 to null
	\else					% No - go ahead and do split
	   \ifdim\ht5>0pt			% Is there a middle figure?
	     \divide\tmp@ by2			% Yes - break text to 2 pieces
	     \global\setbox3=\vsplit255 to\tmp@	% Set \box3 = 1st piece
	     \global\setbox7=\vsplit255 to\tmp@	% Set \box7 = 2nd piece
	   \else				% No middle column...
	     \global\setbox3=\vsplit255 to\tmp@	% Put all text in \box3
	     \global\setbox7=\vbox{}		% Set \box7 to null
	   \fi					% \ifdim\ht5...
	\fi					% \ifdim
	%
	%  Now put the 5 pieces together as one vbox in \box\tmp@five
	%
	\global\setbox\tmp@five=\vbox to\dimen@{\offinterlineskip%
		\ifdim\ht1=0pt\else\vbox{\unvbox1}\fi
		\ifdim\ht3=0pt\else\vbox{\unvbox3}\fi
		\ifdim\ht5=0pt\else\vbox{\unvbox5}\fi
		\ifdim\ht7=0pt\else\vbox{\unvbox7}\fi
		\ifdim\ht9=0pt\else\vbox{\unvbox9}\fi
		\vfil\vfilneg		% Cancels spurious vglue ???? (it works)
		}
	\advance\tmp@five by2			% Bump column box #
	\ifnum\tmp@five<\tmp@four \repeat}	% Loop if not done
}						% End of \splitc@lfigs

%
%  \pagesofar
%
%   Releases \partialpage box
%   Sets width of X boxes (from box 0 to box X) = to \columnwidth
%   Creates an hbox = pagewidth that consists of box0 + separator rule + box 2
%	+ separator rule + box X
%   This new hbox is \box255 and can be used to \shipout
%
\def\pagesofar{\unvbox\partialpage
  \unvbox\partialpagetop				% Include top figure
  \tmp@four=\numberofcols \multiply\tmp@four by2	% Loop boundary number
  \tmp@five=0						% Start with box 0
  %
  %  For each box, set the width equal to the column width
  %
  {\loop \wd\tmp@five=\columnwidth  \advance\tmp@five by2
		\ifnum\tmp@five<\tmp@four \repeat}	% Loop until X done
  %
  %  Now put all the boxes together like this:
  %
  %		box  |  box  | ... |  box
  %
  \tmp@five=0						% Start with box 0 again
  \hbox to\pagewidth{\box\tmp@five\advance\tmp@five by2%  Do the left column
	\loop \hfil\vrule width\columnseprule \hfil \box\tmp@five%
		\advance\tmp@five by2			% Bump box counter
		\ifnum\tmp@five<\tmp@four \repeat	% Loop if not done
	}						% End of hbox
  \unvbox\partialpagebottom				% Include bottom figure
}							% End of \pagesofar

%
%  \balancecolumns - Balance both columns at end of X column page
%
%   Sets box 0 = current page
%   Sets dimen@ = height of the vbox in box 0
%   Adds topskip value to dimen@
%   Subtracts X * \baselineskip from dimen@
%   Divides dimen@ by X  -- dimen@ now has target height of each column
%   Sets splittopskip = to topskip so it can be added to all columns
%   Loops 
%	Copies box 0 to box 3
%	Splits box 3 to dimen@ and stores in box 1 & 3
%	Increments dimen@ by 1pt and loops if column in box 3 exceeds dimen@
%	Note: Splitting of box is actually implemented as a loop that creates
%	      X boxes
%   Moves columns in odd-numbered boxes to corresponding even-numbered boxes
%   Calls pagesofar to make the hbox for it
%
\def\balancecolumns{\setbox0=\vbox{\unvbox255} \dimen@=\ht0
  %
  % Subtract X * \baselineskip from dimen@ and divide dimen@ by X
  %
  \tmp@two=\numberofcols  \advance\tmp@two by-1
  \tmp@=-\baselineskip  \multiply\tmp@ by\tmp@two  \advance\dimen@ by\tmp@
  \divide\dimen@ by\numberofcols \splittopskip=\topskip		% Dimen@/X
  %
  %  Split the column X times so we have X columns of equal height.  If our
  %  last column is > dimen@, bump dimen@ by one and loop.
  %
  \tmp@four=\numberofcols   \multiply\tmp@four by2	% X * 2
  \tmp@seven=\tmp@four  \advance\tmp@seven by-1		% Work box:  (X*2)-1
  {\vbadness=10000 \loop \global\setbox\tmp@seven=\copy0
    \tmp@five=0  \tmp@six=1				% Work box starts with 1
    {\loop						% Loop for X boxes
	\global\setbox\tmp@six=\vsplit\tmp@seven to\dimen@	% Vsplit box
	\advance\tmp@six by2				% Bump box counter
	\ifnum\tmp@six<\tmp@seven \repeat}		% Loop if not done
    \ifdim\ht\tmp@seven>\dimen@ \global\advance\dimen@ by1pt \repeat}
  %
  %  Here we have X columns of equal height.  Note that the last column may
  %  not be equal to the others.
  %
  %  Copy the columns to the even-numbered boxes in preparation for \pagesofar
  %
  \tmp@five=0   \tmp@six=1
  {\loop \global\setbox\tmp@five=\vbox to\dimen@{\unvbox\tmp@six}
	\advance\tmp@five by2  \advance\tmp@six by2
	\ifnum\tmp@five<\tmp@four \repeat}
  \pagesofar}					% Call \pagesofar to build page
%
%  \c@lnewinsert
%
%  This macro is used to perform a \newinsert for temporary usage (inside a grp)
%
\def\c@lnewinsert{\advance\insc@unt by\m@ne	% Decrement insert counter
  \ch@ck0\insc@unt\count			% Make sure count is available
  \ch@ck1\insc@unt\dimen			% Make sure dimen is available
  \ch@ck2\insc@unt\skip				% Make sure skip is available
  \ch@ck4\insc@unt\box				% Make sure box is available
% NEED TO EMPTY BOX!!!
  \count\insc@unt=0				% Set the count to 0
  \allocationnumber=\insc@unt			% Set the allocation number
}						% End of \c@lnewinsert

%
%  \definefigs
%
%  This macro is called to establish the figure environment inside of the
%  multiple-column environment.  It allocates (until \endcolumns) an array
%  of inserts (boxes, counts, dimens, and skips) to handle all of the
%  figures per page for the given number of pages.  When the figures
%  are defined using \definefig, the proper box is filled with the figure.
%
%  Inputs:
%
%	#1	- Number of pages of multiple-column output (should be as large
%		  as the total number of pages between \begincolumns and
%		  \endcolumns)
%
\def\definefigs#1{
	\colfigstrue			% Set column figure flag
	\global\c@lpageno=1		% Get current page number
	\maxfigp@ges=#1			% Set maximum # of pages with figures
	\figsperp@ge=\numberofcols	% Calculate # of figures per page
	\multiply\figsperp@ge by3	%  Figs/Page = 2 + (3 * number of cols)
	\advance\figsperp@ge by2	% ...
	\maxcolfigs=\maxfigp@ges	% Figure out how many inserts are needed
	\multiply\maxcolfigs by\figsperp@ge	% ... for all of the figures
	\tmp@two=\maxcolfigs		% Start there and work down to first box
	\loop	\c@lnewinsert  \advance\tmp@two by\m@ne
		\ifnum\tmp@two>0 \repeat
	\c@lfigstart=\insc@unt		% Save starting array slot #
	% We've allocated all boxes now
	}				% End of \definefigs

%
%  \definefig
%
%  This macro stores figure information in the appropriate slot in the insert
%  array.  After calculating the proper slot number for the figure, it stores
%  the figure in the corresponding box and sets the corresponding \count to
%  a code identifying the box as holding a figure.
%
%  Inputs:
%
%	#1	Output page number
%	#2	Column Number
%	#3	Position (top, middle, bottom)
%	#4	The vbox for the figure
%
\def\definefig#1#2#3#4{
	\ifcolfigs				% Has \definefigs been called?
	\else
   \errmessage{Illegal use of \string\definefig\space before \string\definefigs}
	\fi					% \ifnum
	\ifnum#2>\numberofcols			% Illegal column number?
	   \errmessage{Column number #2 exceeds number of columns}
	\fi					% \ifnum
	\calcc@lslot{#1}{#2}{\csname #3c@lpos\endcsname}   % Calculate slot #
	\ifcase\csname #3c@lpos\endcsname
	   \or\setbox\c@lslot=\vbox{#4\vskip\belowdisplayskip}
	   \or\setbox\c@lslot=\vbox{\vskip\abovedisplayskip#4%
			\vskip\belowdisplayskip}
	   \or\setbox\c@lslot=\vbox{\vskip\abovedisplayskip#4}
		\tmp@=\ht\c@lslot			% Round up the size
		\round{\tmp@}{\baselineskip}{>}		% ...
		\tmp@one=\tmp@  \advance\tmp@one by-\ht\c@lslot
		\setbox\c@lslot=\vbox to\tmp@{\box\c@lslot}
	\fi
	\ifnum#2=0				% If column # is 0, special fig.
	   \ifnum\csname #3c@lpos\endcsname=1\count\c@lslot=\TopFig@
	     \else\count\c@lslot=\BotFig@\fi	% Top or Bottom figure that
	\else					% ... spans all columns
	      \count\c@lslot=\RegFig@		% Otherwise, identify the box
	\fi					% ... as containing a figure
\say{Processed #3 figure for column #2\space on page #1\space
	- slot \the\c@lslot}
	\say{The height is \the\ht\c@lslot}
}

%
%  \calcc@lslot
%
%  This macro is called to calculate the array slot number for a figure.
%  The formula for normal figures is:
%
%	((column# - 1) * 3figs/column) + Position
%
%  This macro assumes there can be 3 figures per column (top, middle, & bottom)
%
%  Inputs:
%
%	#1	- Page number
%	#2	- Column number (0 = special figure that spans all columns)
%	#3	- Figure number (1 = top, 2 = middle, 3 = bottom)
%
%  Returns:
%
%	\c@lslot	- Slot number for given figure.  This slot number
%			  identifies the allocated insert for the figure.
%
\def\calcc@lslot#1#2#3{
	\tmp@seven=#1				% Start with page #
	\advance\tmp@seven by\m@ne		% (Page - 1)
	\multiply\tmp@seven by\figsperp@ge	% (Page - 1) * # of figs
	\ifnum#2=0				% If column is 0, special one
	   \ifnum#3=1\tmp@six=1\else\tmp@six=\figsperp@ge\fi
	   \advance\tmp@six by\m@ne		% Decrement for slot #
	\else
	   \tmp@six=#2				% Column #
	   \advance\tmp@six by\m@ne		% (Column# - 1)
	   \multiply\tmp@six by3		% (Column# - 1) * 3
	   \advance\tmp@six by#3		% (Column# - 1) * 3 + Position
	\fi					% Really + 1, but -1 negates it
	\advance\tmp@seven by\tmp@six		% Add figure # to page #
	\advance\tmp@seven by\c@lfigstart	% Figure slot number
	\global\c@lslot=\tmp@seven		% Set the slot number
	}					% End of \calcc@lslot
%
%  \handlefigures
%
%  This macro is called to step through all of the figures for the current
%  page and subtract the height of each from the total \vsize.
%
%  The two special figures (top (1) and bottom (\figsperp@ge)) are handled
%  differently; because each spans all of the columns on the page, the height
%  of each is multiplied by the number of columns before subtracting it from
%  \vsize.  The top figure is then placed in \partialpagetop and the bottom
%  figure is placed in \partialpagebottom.
%
%  Returns:	Adjusted \vsize.
%
\def\handlefigures{
	\global\vsize=\pageheight		% \vsize to pageheight
	\global\multiply\vsize by\numberofcols	% Multiply by # of columns
	\tmp@three=\figsperp@ge			% Start with figure X
	\calcc@lslot{\c@lpageno}{0}{3}		% Start with last figure box
	{\loop
	    \ifnum\count\c@lslot=\TopFig@	% If box is top section of page
		\global\setbox\partialpagetop=\vbox{\unvbox\c@lslot} % Copy it
		\tmp@=\ht\partialpagetop	% Get the height
		\multiply\tmp@ by\numberofcols	% Multiply by number of columns
		\global\advance\vsize by-\tmp@	% (vsize - figure size)
	    \else				% Else
		\ifnum\count\c@lslot=\BotFig@	% If box is bottom section
		   \global\setbox\partialpagebottom=\vbox{\unvbox\c@lslot}
		   \tmp@=\ht\partialpagebottom
		   \multiply\tmp@ by\numberofcols
		   \global\advance\vsize by-\tmp@ % (vsize - figure size)
		\else				% Else, see if regular figure
		   \ifdim\ht\c@lslot>0pt\all@wfigure{\ht\c@lslot}\fi	% \vsize
		\fi				% End \ifnum
	     \fi				% End \ifnum
	     \advance\tmp@three by\m@ne	% Point to next box
	     \advance\c@lslot by\m@ne	% Point to next box
	     \ifnum\tmp@three>0 \repeat}	% Loop until done
	}					% End of \handlefigures

\def\all@wfigure#1{
	\tmp@=#1		% Height of figure
	\global\advance\vsize by-\tmp@	% Subtract figure size from vsize
	}

%%%%%%%%%%%%%%%%%%%%%%%%%%%%%%%%%%%%%%%%%%%%%%%%%%%%%%%%%%%%%%%%%%%%%%%%%%%%%%%%
%  Macro:	\round#1#2#3
%
%  Purpose:	Round a dimen parameter (#1) to the nearest even multiple of
%		parameter #2.  Primarily used to ensure that the page height
%		or a figure height is an even multiple of the baselineskip.
%
%		NOTE: this macro assumes there is no stretch or shrink to #2.
%
%  Inputs:
%		#1	Dimen variable to change (e.g., \pageheight)
%		#2	Dimen variable to use as multiple (e.g., \baselineskip)
%		#3	Symbol indicating round up (>) or round down (<)
%		\tmp@	Work dimen (saved and restored)
%
%  Example:
%	\pageheight=598.213pt   \baselineskip=12pt
%	\round{\pageheight}{\baselineskip}{<}	%yields \pageheight=588.0pt
%	\round{\pageheight}{\baselineskip}{>}	%yields \pageheight=600.0pt
%		
\def\round#1#2#3{\bgroup%			%Keep \tmp@ changes local
	\rtmp@=0pt				%Initialize \tmp@
	\loop					%Begin a loop
	\advance\rtmp@ by #2			%Bump \tmp@ by #2
	\ifdim#1>\rtmp@ \repeat			%Loop until \tmp@ > #1
	\ifx<#3					%If #3 = "<" then
	   \advance\rtmp@ by-#2			%... subtract #2 from \tmp@
	\fi					%...
	\global#1=\rtmp@			%Reset parameter #1
	\egroup					%End the group
}						%End of macro
%%%%%%%%%%%%%%%%%%%%%%%%%%%%%%%%%%%%%%%%%%%%%%%%%%%%%%%%%%%%%%%%%%%%%%%%%%%%%%%%
%
%  Define list macros
%
%	Dimens:
%
%		\llistindent	- Amount of left indent  (0pt by default)
%		\rlistindent	- Amount of right indent (0pt by default)
%
%	Macros:
%
%		\beginlist	- Begin list (skips space, sets indent)
%		\endlist	- Terminates a list
%		\beginlistt	- Begin list with glue (used for list headers)
%		\endlistt	- Terminates a \beginlistt
%		\dotitem	- Itemize with a dot "o"
%
\newdimen\llistindent   \llistindent=0pt
\newdimen\rlistindent   \rlistindent=0pt
\def\listindent#1{\llistindent=#1\rlistindent=#1}

\def\beginlist{\begingroup%
	\ifdim\parskip=0pt \vskip.5\baselineskip	% Skip 1/2 line
	\else \vskip\parskip				%   or the \parskip
	\fi						%
	\parindent=10pt\parskip=0pt%			% Reset parindent
	\leftskip=\llistindent \rightskip=\rlistindent}	% Indent margins

\def\endlist{\par\endgroup%
	\ifdim\parskip=0pt \vskip.5\baselineskip
	\else \vskip\parskip\fi}

\def\dotitem#1\par{\item{$\bullet$} #1 \par}

\def\beginlistt#1{#1\vglue0pt\begingroup%
	\divide\parskip by2\vskip\parskip\parindent=10pt\parskip=0pt%
	\leftskip=\llistindent \rightskip=\rlistindent}
\def\endlistt{\par\endgroup} %\vskip\parskip}

%
%  Define a small copyright (for use with 8-point type).
%
\def\smallcir{\smallsy\char13}
\def\smallcopyright{\leavevmode\raise.25ex\hbox{
	\ooalign{\hfil\raise.03ex\hbox{\kern .16em\verysmallrm C}
	\hfil\crcr\smallcir}}}

%%%%%%%%%%%%%%%%%%%%%%%%%%%%%%%%%%%%%%%%%%%%%%%%%%%%%%%%%%%%%%%%%%%%%%%%%%%%%%%%
%
%  Define macros to manipulate boxes
%
%	Dimens:
%
%		\boxitrule=Xpt     - Width of rules used to draw boxes
%		\boxitspace=Xpt    - Space between box rules and box contents
%		\boxshadowsize=Xpt - Width of shadow boxes
%
%	Macros:
%
%	\articletitle{Title}{byline}	- Do article title in double box
%	\coltitle{Title}	- Do a title box in a column
%	\shadowbox{some_box}    - Draw a shadow box around an hbox or vbox
%	\leftshadowbox{somebox} - Draw a left-hand shadow box around a box
%	\centerbox{somebox}     - Center a \vbox on a page
%	\boxit{some_box}        - Draw a box around an hbox or vbox
%	\ednote                 - Do editor's note in a box
%
\newdimen\boxitspace  \newdimen\boxitrule  \newdimen\boxitwidth
\boxitspace=3pt   \boxitrule=1.2pt
\boxitwidth=\boxitspace  \advance\boxitwidth by\boxitspace
\advance\boxitwidth by\boxitrule  \advance\boxitwidth by\boxitrule

%
%  Define macro to write an article title inside a double box
%
%  Parameters:
%
%	#1	- Title of article
%	#2	- Byline
%
%  For both \articletitle and \coltitle, the width of the box(es) must be
%  subtracted from the current hsize in order for centering and justification
%  to work right (otherwise the letters will run into the lines of the box).
%
\def\articletitle#1#2{
	\bestbreak{				% Say that this is best break
	\advance\hsize by -\boxitwidth		% Bring margin in before \center
	\advance\hsize by -\boxitwidth		% Do for both boxes!
	\vskip 10pt plus 5pt			% Skip some space
	\boxit{					% Box the box
	\divide\boxitrule by 2			% Make inside box lines thinner
	\boxit{					% Box the text
	\vbox{\noindent				% Create a box
	\centerline{\seventeenpoint\ss #1}	% Print title of article
	\centerline{\ninepoint\ss #2}		% Print article byline
	}}}}}					% Create the text box

\def\coltitle#1\par{{%				% Swallow next paragraph
	\advance\hsize by -\boxitwidth		% Bring margins in
	\boxit{%				% Draw a box around the text
	\vbox{\ss\noindent #1}}}%		% Create a vbox that contains
	\vglue.5\baselineskip%			% Skip some non-breakable space
	\noindent}				% Don't indent next paragraph

\def\ednote#1{{
	\sl					% Switch to slanted font
	\def\sl{\/\rm}				% Redefine \sl
	\advance\hsize by -\boxitwidth		% Bring margins in
	\boxit{					% Draw a box around the text
	\vbox{\noindent	Editor's note: #1}}}	% Create a vbox that contains
	\vglue 0pt}				% Finish it up

\def\boxit#1{\vbox{\tithrule\hbox{\titvrule\kern\boxitspace%
	\vbox{\kern\boxitspace #1 \kern\boxitspace}%
	\kern\boxitspace\titvrule}\tithrule}}

\def\tithrule{\hrule height\boxitrule}
\def\titvrule{\vrule width\boxitrule}

%
%  \centerbox
%
%  Create a \vbox that contains a centered \hbox.  The centering is relative
%  to the current \hsize.
%
%  Inputs:
%
%	#1	- \vbox to center   ->   \centerbox{\shadowbox{...}}
%
\def\centerbox#1{\hbox{\hfil#1\hfil}}	% Create a \vbox containing a centered


\newdimen\oboxht  \newdimen\oboxwd  \newdimen\boxshadowsize
\boxshadowsize=4pt				% Shadow box size is 4pt
%
%  Draw a righthand shadow box.  This is accomplished by building a vbox
%  containing an hbox that is the boxed text and an hbox that is the right
%  hand shadow.  This vbox is then joined with a vbox that forms
%  the bottom shadow.
%
\def\shadowbox#1{{
	\setbox0=\vbox{\boxit{#1}}		% Set box after \boxit
	\oboxht=\ht0  \oboxwd=\wd0		% Store the dimensions
	\advance\oboxwd by-\boxshadowsize	% Subtract shadow size from wd
	\vbox{					% Put it all in one vbox
	\offinterlineskip			% Butt \vboxes together
	\vbox{					% Create a vbox of whole thing
	\hbox{\vbox{\unvbox0}			% Create box with text box +
	\hskip-\fontdimen2\font
	\lower\boxshadowsize		% Draw the right-hand boxshadowsize
	\hbox{\vrule width\boxshadowsize height\oboxht}} % ...  Finish off \hbox
	}					% End of the \vbox
	\advance\boxshadowsize by\boxitrule
	\vskip-\boxshadowsize			% Back up to bottom of \vbox
	\vbox{					% Start a new \vbox
	\hbox{\kern\boxshadowsize\vbox{		% Create \hbox that is shadow
	\hrule height\boxshadowsize width\oboxwd}}	% ...
	}					% End of the \vbox
	}					% End of \vbox
	}}					% End of \shadowbox
%
%  Draw a lefthand shadow box.  This is accomplished by building a lowered
%  vbox containing an hbox that is the left hand shadow and an hbox that
%  contains the boxed text.  This vbox is then joined with a vbox that forms
%  the bottom shadow.
%
\def\leftshadowbox#1{{
	\setbox0=\vbox{\boxit{#1}}		% Set box after \boxit
	\oboxht=\ht0  \oboxwd=\wd0		% Store the dimensions
	\advance\oboxwd by-\boxshadowsize	% Subtract shadow size from wd
	\vbox{
	\offinterlineskip			% Butt \vboxes together
	\vbox{					% Create a vbox of whole thing
	\hbox{					% Create an hbox
	\hskip-\fontdimen2\font			% Move left one character width
	\hskip-\boxshadowsize			% Move left = size of shadow box
	\advance\boxitrule by\boxshadowsize	% Make shadow a tad bit wider
	\lower\boxshadowsize			% Move down the same amount
	\hbox{\vrule width\boxitrule height\oboxht}	% Draw the left box
	}					% ...  Finish off \hbox
	\vskip-\oboxht\vskip-\boxshadowsize	% Move back up to top of box
	\hbox{\vbox{\unvbox0}			% Create box with text inside it
	}}					% End of vboxes
	\advance\boxshadowsize by\boxitrule	% Make box a little thicker
						% ... so it overlaps bottom line
	\vskip-\boxitrule			% Move up height of bottom line
	\vbox{					% Start a new \vbox
	\hbox{\vbox{				% Create \hbox that is shadow
	\hrule height\boxshadowsize width\oboxwd}}	% ...
	}					% End of the \vbox
	}					% End of \vbox
	}}					% End of \shadowbox

\def\bestbreak{\par\penalty-1000}

%
%  Begin a quotation.  The quote is separated from the main text by two
%  hrules and is indented from the normal text.
%
\def\beginquote{
	\begingroup				% Define beginning of a group
	\quotefont\baselineskip=16.8pt
	\hrule height2pt			% Draw a line
	\parindent 5pt				% Indent paragraphs by 5pt
	\vglue\medskipamount			% Use some glue so rule sticks
	\narrower\narrower			% Bring margins in (10pt)
	\noindent				% Don't indent
	}					% End of macro
%
%  End a quotation.
%
\def\endquote{
	\vglue\medskipamount			% Use some glue so rule sticks
	\hrule height2pt			% Draw a line
	\endgroup				% End the quote group
	}

\def\quotee#1{{\hfill\break\hbox{}\nobreak\hfill\hbox{\quoteefont #1}
	\finalhyphendemerits=0}}
\def\Quote#1#2{\vbox{\vskip1.2pt\beginquote #1%
		\quotee{#2}\endquote\vskip1.20003pt}}

%
%  Insert current month and year
%
\def\DATE{\ifcase\month\or January\or February\or March\or April\or May
	\or June\or July\or August\or September\or October\or November
	\or December\fi\space\number\year}
%
%  Include a TeX file.
%
\def\include#1{\immediate\write10{Including TeX file #1}
	\input #1				% Read the file in
	% Things to do after formatting the file
	}
%
%  Separate articles with some vskip and an \hrule
%
\def\articlesep{				% Rule to separate articles
	\vglue 10pt plus2pt minus4pt		% Use vglue
	\hrule %height.4pt			% Draw a rule equal to \hsize
	\vskip 10pt plus2pt minus2pt		% Skip some vertical space
	}

% Put release flush right.  If it won't fit, put it on the next line.
% From TeXbook, Chapter 14.
\def\farright#1{{\unskip\nobreak\hfill\penalty50\hskip2em
  \hbox{}\nobreak\hfill \hbox{#1}\finalhyphendemerits=0}}

%
%  Define end-of-article marker.
%
\def\eoa{\farright{\vrule height1.5ex width1.5ex depth0pt}}
%
%  Generate a blank page and a blank line.
%
\def\nullpage{\eject\line{}\vfil\eject}		% Define an empty page
\def\nullline{\break\hbox{}\hfil\break}		% Define an empty line
%
%  Start on a new page.
%
\def\newpage{\vfill\eject}
%
%  Get rid of underfill errors
%
\def\ignoreunderfill{\vbadness=10000\hbadness=10000\tolerance=2000}

%=-=-=-=-=-=-=-=-=-=-=-=-=-=-=-=-=-=-=-=-=-=-=-=-=-=-=-=-=-=-=-=-=-=-=-=-=-=-=-=
% 			Macros for bibliography entries
%=-=-=-=-=-=-=-=-=-=-=-=-=-=-=-=-=-=-=-=-=-=-=-=-=-=-=-=-=-=-=-=-=-=-=-=-=-=-=-=
%
%  Sample usage:
%
%	\beginbibliography
%	\bibbook{The Wolf's Hour}
%	\ENUM	New York: Pocket Books, March 1989  (paperback)
%	\endbibliography
%
\newcount\enumno				% New counter - item #
%
%  \beginbibliography - Begin a bibliography
%
\def\beginbibliography{\begingroup\global\enumno=1
	\tiny					% Use 8pt font
	\parskip=1pt plus 1pt			% Skip up to 2 points
	}					% End of \beginbibliography
\def\endbibliography{\par\endbiblist\endgroup}	% End of bibliography
%
%  \beginbiblist - Begin a list of bibliographic references
%
\def\beginbiblist{\begingroup
	\vglue0pt\parindent=30pt\parskip=0pt}
%
%  \beginanotherlist - Begin a list inside a list of bibliographic references
%
\def\beginanotherlist{\begingroup
	\divide\parskip by 2
	\vglue\parskip\advance\parindent by10pt\parskip=0pt}
\def\endbiblist{\par\endgroup\vskip4pt}

%  ENUM   - Number items in a list
%  ENum   - No number, but spaced as if number was present
%  NoENUM - Only one reference is present.  Start reference where number would
%	    normally start (hanging into left column).
%
%  Examples:
%
%	\ENUM	First one			1.  First one
%	\ENUM	Second one	Yields		2.  Second one
%	\ENum	Third one			    Third one
%	\NoENUM	Fourth one			Fourth one
%
\def\ENUM#1\par{\item{\the\enumno.}\advance\enumno by 1 #1 \par }
\def\ENum#1\par{\item{}\advance\enumno by 1 #1 \par }
\def\NoENUM#1\par{\advance\enumno by 1\par\hang\hskip-10pt #1 \par }
%
%  \bibshort, \bibbook, \bibview
%
%  Macros to begin a new bibliography entry for a short story, a book, and
%  and interviews.  These macros will terminate the previous bibliography
%  entry (if there is one) and begin a new entry.
%
\def\bibshort#1{\ifnum\enumno>1 \bestbreak\endbiblist\fi	% Short story
	\noindent{\story{#1}}\beginbiblist
	}
\def\bibbook#1{\ifnum\enumno>1 \endbiblist\fi			% Book
	\noindent{\sl #1}\beginbiblist
	}
\def\bibview#1{\ifnum\enumno>1 \endbiblist\fi			% Interview
	\noindent{#1}\beginbiblist
	}
\def\subbib#1{\hskip-20pt #1\hfill}		%Subheading for a bib entry
\def\bibsectitle#1{				%Title of bib section (BOOKS...)
	\vskip 8pt plus1pt minus1pt		% Skip some space
	\hrule					% Draw an hrule
	{\tenss #1}				% Add text in 10pt font
	\vglue 10pt plus1pt minus1pt		% Skip some more space
	}
%=-=-=-=-=-=-=-=-=-=-=-=-=-=-=-=-=-=-=-=-=-=-=-=-=-=-=-=-=-=-=-=-=-=-=-=-=-=-=-=
%
%  File:	QUOTE.TEX
%
%  Author:	Hunter Goatley
%
%  Date:	August 14, 1991
%
%  Abstract:
%
%	This file defines the macros \begindoublequotes and \enddoublequotes,
%	which let TeX replace the double-quote character (") with TeX's
%	left double-quote and right double-quote.  For example:
%
%		"This is a test."     --->    ``This is a test.''
%
%	The double-quote character is still available via \dq.  (\" is still
%	treated as the umlaut accent.)
%
%	This macro makes a couple of assumptions about the double-quotes:
%
%	1.  Double-quotes are assumed to come in pairs.  When replacing
%	    double-quotes, the macro alternates between `` and ''.  The only
%	    exception to this is noted in (2) below.
%	2.  A double-quote at the beginning of a paragraph is always treated
%	    as ``.  This correctly handles the case where a quotation is
%	    continued into a second paragraph:
%
%		"This is the first paragraph.\par
%		"This is the second paragraph of the same quote."
%
%	Normal TeX spacing after `` and '' is maintained by saving and
%	restoring the \spacefactor.
%
%%%%%%%%%%%%%%%%%%%%%%%%%%%%%%%%%%%%%%%%%%%%%%%%%%%%%%%%%%%%%%%%%%%%%%%%%%%%%%%%
%
%  HOW IT WORKS:
%
%	The double-quote character (") is made active by \begindoublequotes.
%	The " macro keeps track of left-quote/right-quote pairs and inserts
%	the appropriate `` and '' in its place.
%
%	Each character has a \spacefactor associated with it, which specifies
%	the amount of stretch or shrink that a space following the character
%	can have.  Most characters have a factor of 1000, but some punctuation
%	marks have higher spacefactors, most notably the period, which has a
%	\spacefactor of 3000.  This means the space following a period can
%	stretch up to 3 times more than the space after a regular character,
%	accounting for the increased space at the end of sentences.
%
%	The `` and '' ligatures are assigned \spacefactor's of 0, so that the
%	\spacefactor that is applied to the next character is the same as that
%	of the character preceding the quotes.  Because " has been redefined as
%	a macro, any spaces following " are swallowed by TeX.  It was necessary
%	to have this macro re-insert any needed space so that the following
%	cases worked correctly:
%
%		"This is a test," she said. --> ``This is a test,'' she said.
%		"This is in a list"; etc.   --> ``This is in a list''; etc.
%
%	Without the added space, the first example becomes:
%
%		``This is a test,''she said.
%
%	The solution was to save the current \spacefactor before inserting a
%	right double-quote, then resetting the \spacefactor after the
%	insertion.  The net effect was that the " macro has a \spacefactor
%	of 0, which matches the way TeX treats `` and ''.
%
%%%%%%%%%%%%%%%%%%%%%%%%%%%%%%%%%%%%%%%%%%%%%%%%%%%%%%%%%%%%%%%%%%%%%%%%%%%%%%%%
{%					% Begin a group for which " is active
\catcode`\"=\active			% Make " an active character
\catcode`\@=11				% Make @ an active character
%
%  \begindoublequotes
%
%	This macro makes " an active character, resets the control sequence
%	\dblqu@te to L (left), and defines \dq as a replacement for ".
%
\gdef\begindoublequotes{%		% \begindoublequotes enables "
    \global\catcode`\"=\active		% Make " an active character
    \global\chardef\dq=`\"		% Double-quote char. via \dq
    \global\let\dblqu@te=L		% Always start with a left double-quote
    }					% End of macro
%
%  Define the macro that will be executed whenever " is encountered.
%
\gdef"{%
	%  If the double-quote is the first character in a new paragraph,
	%  make sure it becomes a left double-quote.  This case can be
	%  detected by checking to see if TeX is currently in vertical mode.
	%  If so, the double-quote is at the beginning of the paragraph
	%  (since " hasn't actually generated any horizontal mode tokens
	%  yet, TeX is still in vertical mode).  If the mode is vertical,
	%  set \dblqu@te equal to L.
	%
	\ifinner\else\ifvmode\let\dblqu@te=L\fi\fi
	%
	%  Now insert the appropriate left or right double-quote.
	%
	%  If \dblqu@te is L, insert a `` and set \dblqu@te to R.
	%
	\if L\dblqu@te``\global\let\dblqu@te=R%
	%
	%  Otherwise, save the current \spacefactor, insert '', set \dblqu@te
	%  to L, and reset the original \spacefactor.
	%
	\else
	   \let\xxx=\spacefactor		% Save the \spacefactor
	   ''\global\let\dblqu@te=L%		% Insert '' and reset \dblqu@te
	   \spacefactor\xxx			% Reset the \spacefactor
	\fi					% End of \if L\dblqu@te...
	}					% End of " macro
}						% End of group

\gdef\enddoublequotes{%
	\catcode`\"=12				%Set " back to other
	}
%
%%%%%%%%%%%%%%%%%%%%%%%%%%%%%%%%%%%%%%%%%%%%%%%%%%%%%%%%%%%%%%%%%%%%%%%%%%%%%%%%
%                           Header & Footer Macros
%%%%%%%%%%%%%%%%%%%%%%%%%%%%%%%%%%%%%%%%%%%%%%%%%%%%%%%%%%%%%%%%%%%%%%%%%%%%%%%%
%
%  These macros implement the headers and footers for the newsletter format.
%  The macros accept three parameters: text that is to appear flush-left on
%  the line, text that should be centered, and text that should be flush-right
%  on the line.  Parameters can be omitted by specifying empty braces ({}).
%
%  The following macros are defined for headers and footers:
%
%	\evenpageheader{LEFT}{CENTER}{RIGHT}
%	\oddpageheader{LEFT}{CENTER}{RIGHT}
%	\evenpagefooter{LEFT}{CENTER}{RIGHT}
%	\oddpagefooter{LEFT}{CENTER}{RIGHT}
%
%  If the headers/footers are the same for even & odd pages, the following
%  macros can be used instead of the four above:
%
%	\pageheader{LEFT}{CENTER}{RIGHT}
%	\pagefooter{LEFT}{CENTER}{RIGHT}
%
%  Additional header/footer definitions:
%
%	\pageheaderlinetrue		- A line should extend below header text
%	\pageheaderlinefalse		- Header does NOT have a line
%	\pagefooterlinetrue		- A line should extend above footer text
%	\pagefooterlinefalse		- Footer does NOT have a line
%	\headfootrule=Xpt		- Thickness of header/footer lines
%	\pageheaderskip=Xpt		- \vskip between header and line
%	\pagefooterskip=Xpt		- \vskip between footer and line
%	\headfont=\fontname		- Font to use for header text
%	\footfont=\fontname		- Font to use for footer text
%
%  Example:
%
%	\pageheader{}{My Newsletter}{}
%	\pagefooter{October 1989}{}{\pageno}
%
\newif\ifpageheaderline  \pageheaderlinefalse	% By default, no header line
\newif\ifpagefooterline  \pagefooterlinefalse	% By default, no footer line
\newdimen\headfootrule   \headfootrule=0.50pt	% Height of header & footer rule
\newdimen\pageheaderskip \pageheaderskip=4pt	% Space between header and rule
\newdimen\pagefooterskip \pagefooterskip=4pt	% Space between rule and footer

\let\headfont=\twelverm \let\footfont=\twelverm	% Assign fonts for head/foot

\def\@pageheader#1#2#3{%
	\ifpageheaderline			% If headerline
	\vbox{\hbox to\normalhsize{{\headfont\rlap{#1}\hss{#2}\hss\llap{#3}}}%
	\vskip\pageheaderskip\hrule height\headfootrule}% Do hbox and hrule
	\else
	{\headfont\rlap{#1}\hss{#2}\hss\llap{#3}}%
	\fi
	}

\def\@pagefooter#1#2#3{%
	\ifpagefooterline
	\vbox{\hrule height\headfootrule\vskip\pagefooterskip
	\hbox to\normalhsize{\footfont\rlap{#1}\hss{#2}\hss\llap{#3}}}%
	\else
	{\footfont\rlap{#1}\hss{#2}\hss\llap{#3}}%
	\fi
	}
%
% Define default headers and footers - null lines of text
%
\def\@oddhead{\nullline}  \def\@evenhead{\nullline}
\def\@oddfoot{\nullline}  \def\@evenfoot{\nullline}

\def\@newhead{\headline{\ifodd\pageno\@oddhead\else\@evenhead\fi}}
\def\@newfoot{\footline{\ifodd\pageno\@oddfoot\else\@evenfoot\fi}}

\def\oddpageheader#1#2#3{\@newhead\def\@oddhead{\@pageheader{#1}{#2}{#3}}}
\def\evenpageheader#1#2#3{\@newhead\def\@evenhead{\@pageheader{#1}{#2}{#3}}}
\def\oddpagefooter#1#2#3{\@newfoot\def\@oddfoot{\@pagefooter{#1}{#2}{#3}}}
\def\evenpagefooter#1#2#3{\@newfoot\def\@evenfoot{\@pagefooter{#1}{#2}{#3}}}
%
%  If no difference between even and odd pages, just define both to be the same.
%
\def\pageheader#1#2#3{\evenpageheader{#1}{#2}{#3}\oddpageheader{#1}{#2}{#3}}
\def\pagefooter#1#2#3{\evenpagefooter{#1}{#2}{#3}\oddpagefooter{#1}{#2}{#3}}
%%%%%%%%%%%%%%%%%%%%%%%%%%%%%%%%%%%%%%%%%%%%%%%%%%%%%%%%%%%%%%%%%%%%%%%%%%%%%%%%
%
%  Command for "poor man's bold":  \pmb   (use sparingly)
%
\def\pmb#1{\setbox0=\hbox{#1}%			% Copy box to box0
    \leavevmode\hbox{%				% Make an hbox that holds
    \kern-.025em\copy0\kern-\wd0%		% Move left 1/4 em and copy box0
    \kern.05em\copy0\kern-\wd0%			% Move right 1/4 em and copy it
    \kern-.025em\raise.0433em\box0 }}		% Raise a little and copy again

%
%  Define dots for ending sentences (4 dots instead of 3)
%
\def\eldots{\mathinner{\ldotp\ldotp\ldotp\ldotp}}
\def\edots{\relax\ifmmode\eldots\else$\m@th\eldots\,$\fi}
\def\ellip{\hskip.2em\ifmmode\ldots\else$\ldots$\fi\hskip.25em}
%
%  Define macros \ital and \slant to switch to italic (\it) and slanted (\sl)
%  respectively.  These macros automatically insert the italic correction
%  unless the next character is a period or a comma.  Based on the
%  \predict macro presented in _TeX for the Impatient_, p. 233.
%
%  These macros use \toks0 as a temporary.
%
%  The \futurelet\it@next in \ital and \slant defines \it@next to be whatever
%  the character following the parameter is.  \d@slant checks to see if
%  \it@next is a comma or period; if it is neither, the italic correction
%  (\/) is included.
%
\def\ital#1{\toks0={#1}\let\slf@nt=\it\futurelet\it@next\d@slant}
\def\slant#1{\toks0={#1}\let\slf@nt=\sl\futurelet\it@next\d@slant}
\def\d@slant{{\slf@nt\the\toks0}%
	\ifx\it@next,%			% If \it@next not a comma
	\else\ifx\it@next.%		% ... and is not a period
	\else\/%			% ... insert the correction (\/)
	\fi\fi%				% ...
	\let\it@next=\relax%		% "Undefine" \it@next
	}

\def\book#1{\ital{#1}}				%For ease, define \book too
\def\story#1{``#1''}				%Short story title (add quotes)
%
%  Important - make "@" a valid alphanumeric character again
%
\catcode`\@=12				% Follow TeX's lead on variable names

\tenpoint					% Default point size is 10pt
