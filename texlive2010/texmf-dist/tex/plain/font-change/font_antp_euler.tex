% The author of this work is Amit Raj Dhawan.
% This work has been released under 
% Creative Commons Attribution-Share Alike 3.0 Unported License 
% on August 17, 2009. For details visit: 
% http://creativecommons.org/licenses/by-sa/3.0/.

% roman text (Family 0)
\font\twentyrm=     antpr at20pt
\font\eighteenrm=   antpr at18pt
\font\sixteenrm=    antpr at16pt
\font\fourteenrm=   antpr at14pt
\font\twelverm=     antpr at12pt
\font\tenrm=        antpr
\font\ninerm=       antpr at9pt
\font\eightrm=      antpr at8pt
\font\sevenrm=      antpr at7pt
\font\sixrm=        antpr at6pt
\font\fiverm=       antpr at5pt

% math italic (Family 1)
\font\twentyi=      zeurm10 at20pt
\font\eighteeni=    zeurm10 at18pt
\font\sixteeni=     zeurm10 at16pt
\font\fourteeni=    zeurm10 at14pt
\font\twelvei=      zeurm10 at12pt
\font\teni=         zeurm10
\font\ninei=        zeurm10 at9pt
\font\eighti=       zeurm10 at8pt
\font\seveni=       zeurm7
\font\sixi=         zeurm10 at6pt
\font\fivei=        zeurm5

% math symbols (Family 2)
\font\twentysy=     zeusm10 at20pt
\font\eighteensy=   zeusm10 at18pt
\font\sixteensy=    zeusm10 at16pt
\font\fourteensy=   zeusm10 at14pt
\font\twelvesy=     zeusm10 at12pt
\font\tensy=        zeusm10 
\font\ninesy=       zeusm10 at9pt
\font\eightsy=      zeusm10 at8pt
\font\sevensy=      zeusm7
\font\sixsy=        zeusm10 at6pt
\font\fivesy=       zeusm5

% math extension (Family 3)
\font\twentyex=     zeuex10 at20pt
\font\eighteenex=   zeuex10 at18pt
\font\sixteenex=    zeuex10 at16pt
\font\fourteenex=   zeuex10 at14pt
\font\twelveex=     zeuex10 at12pt
\font\tenex=        zeuex10
\font\nineex=       zeuex10 at9pt
\font\eightex=      zeuex10 at8pt
\font\sevenex=      zeuex10 at7pt
\font\sixex=        zeuex10 at6pt
\font\fiveex=       zeuex10 at5pt

% italic text (Family 4)
\font\twentyit=     antpri at20pt
\font\eighteenit=   antpri at18pt
\font\sixteenit=    antpri at16pt
\font\fourteenit=   antpri at14pt
\font\twelveit=     antpri at12pt
\font\tenit=        antpri
\font\nineit=       antpri at9pt
\font\eightit=      antpri at8pt
\font\sevenit=      antpri at7pt
\font\sixit=        antpri at6pt
\font\fiveit=       antpri at5pt

% slanted text (Family 5)
\font\twentysl=     antpri at20pt
\font\eighteensl=   antpri at18pt
\font\sixteensl=    antpri at16pt
\font\fourteensl=   antpri at14pt
\font\twelvesl=     antpri at12pt
\font\tensl=        antpri
\font\ninesl=       antpri at9pt
\font\eightsl=      antpri at8pt
\font\sevensl=      antpri at7pt
\font\sixsl=        antpri at6pt
\font\fivesl=       antpri at5pt

% boldface text (Family 6)
\font\twentybf=     antpb at20pt
\font\eighteenbf=   antpb at18pt
\font\sixteenbf=    antpb at16pt
\font\fourteenbf=   antpb at14pt
\font\twelvebf=     antpb at12pt
\font\tenbf=        antpb
\font\ninebf=       antpb at9pt
\font\eightbf=      antpb at8pt
\font\sevenbf=      antpb at7pt
\font\sixbf=        antpb at6pt
\font\fivebf=       antpb at5pt

% typewriter text (Family 7)
\font\twentytt=     cmtt10 at20pt
\font\eighteentt=   cmtt10 at18pt
\font\sixteentt=    cmtt10 at16pt
\font\fourteentt=   cmtt10 at14pt
\font\twelvett=     cmtt10 at12pt
\font\tentt=        cmtt10
\font\ninett=       cmtt9
\font\eighttt=      cmtt8
\font\seventt=      cmtt10 at7pt
\font\sixtt=        cmtt10 at6pt
\font\fivett=       cmtt10 at5pt




% Family 0 (roman text)
\textfont0=\tenrm
\scriptfont0=\sevenrm
\scriptscriptfont0=\fiverm
\def\rm{\fam=0 \tenrm}

% Family 1 (math italic)
\textfont1=\teni
\scriptfont1=\seveni
\scriptscriptfont1=\fivei
\def\mit{\fam=1}

% Family 2 (math symbol)
\textfont2=\tensy
\scriptfont2=\sevensy
\scriptscriptfont2=\fivesy
\def\cal{\fam=2}

% Family 3 (math extension)
\textfont3=\tenex
\scriptfont3=\sevenex
\scriptscriptfont3=\fiveex


% Family 4 (italic text)
\def\it{\fam=\itfam \tenit}
\textfont\itfam=\tenit
\scriptfont\itfam=\sevenit
\scriptscriptfont\itfam=\fiveit

% Family 5 (slanted text)
\def\sl{\fam=\slfam \tensl}
\textfont\slfam=\tensl
\scriptfont\slfam=\sevensl
\scriptscriptfont\slfam=\fivesl

% Family 6 (boldface text)
\def\bf{\fam=\bffam \tenbf}
\textfont\bffam=\tenbf
\scriptfont\bffam=\sevenbf
\scriptscriptfont\bffam=\fivebf

% Family 7 (typewriter text)
\def\tt{\fam=\ttfam \tentt}
\textfont\ttfam=\tentt
\scriptfont\ttfam=\seventt
\scriptscriptfont\ttfam=\fivett

% italic boldface
\font\twentyitbf=      antpbi at20pt
\font\eighteenitbf=    antpbi at18pt
\font\sixteenitbf=     antpbi at16pt
\font\fourteenitbf=    antpbi at14pt
\font\twelveitbf=      antpbi at12pt
\font\itbf=            antpbi
\font\nineitbf=        antpbi at9pt
\font\eightitbf=       antpbi at8pt
\font\sevenitbf=       antpbi at7pt
\font\sixitbf=         antpbi at6pt
\font\fiveitbf=        antpbi at5pt

% slanted boldface
\font\twentyslbf=      antpbi at20pt
\font\eighteenslbf=    antpbi at18pt
\font\sixteenslbf=     antpbi at16pt
\font\fourteenslbf=    antpbi at14pt
\font\twelveslbf=      antpbi at12pt
\font\slbf=            antpbi
\font\nineslbf=        antpbi at9pt
\font\eightslbf=       antpbi at8pt
\font\sevenslbf=       antpbi at7pt
\font\sixslbf=         antpbi at6pt
\font\fiveslbf=        antpbi at5pt

% caps % no caps or bold caps
%
%
\mathchardef\Gamma="2100
\mathchardef\Delta="2101
\mathchardef\Theta="2102
\mathchardef\Lambda="2103
\mathchardef\Xi="2104
\mathchardef\Pi="2105
\mathchardef\Sigma="2106
\mathchardef\Upsilon="2107

\mathchardef\Phi="2108
\mathchardef\Psi="2109
\mathchardef\Omega="210A

\mathchardef\varrho="211A

%
\rm 