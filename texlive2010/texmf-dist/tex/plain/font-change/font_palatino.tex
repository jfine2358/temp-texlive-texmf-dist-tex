% The author of this work is Amit Raj Dhawan.
% This work has been released under 
% Creative Commons Attribution-Share Alike 3.0 Unported License 
% on August 17, 2009. For details visit: 
% http://creativecommons.org/licenses/by-sa/3.0/.

% roman text (Family 0)
\font\twentyrm=     pxr at20pt
\font\eighteenrm=   pxr at18pt
\font\sixteenrm=    pxr at16pt
\font\fourteenrm=   pxr at14pt
\font\twelverm=     pxr at12pt
\font\tenrm=        pxr  % cs-qplr (TeX encoding) and adobe's "pplr" offer better rendering of letters e.g. D
\font\ninerm=       pxr at9pt
\font\eightrm=      pxr at8pt
\font\sevenrm=      pxr at7pt
\font\sixrm=        pxr at6pt
\font\fiverm=       pxr at5pt

% math italic (Family 1)
\font\twentyi=      pxmi at20pt
\font\eighteeni=    pxmi at18pt
\font\sixteeni=     pxmi at16pt
\font\fourteeni=    pxmi at14pt
\font\twelvei=      pxmi at12pt
\font\teni=         pxmi
\font\ninei=        pxmi at9pt
\font\eighti=       pxmi at8pt
\font\seveni=       pxmi at7pt
\font\sixi=         pxmi at6pt
\font\fivei=        pxmi at5pt

% math symbols (Family 2)
\font\twentysy=     pxsy at20pt
\font\eighteensy=   pxsy at18pt
\font\sixteensy=    pxsy at16pt
\font\fourteensy=   pxsy at14pt
\font\twelvesy=     pxsy at12pt
\font\tensy=        pxsy
\font\ninesy=       pxsy at9pt
\font\eightsy=      pxsy at8pt
\font\sevensy=      pxsy at7pt
\font\sixsy=        pxsy at6pt
\font\fivesy=       pxsy at5pt

% math extension (Family 3)
\font\twentyex=     pxex at20pt
\font\eighteenex=   pxex at18pt
\font\sixteenex=    pxex at16pt
\font\fourteenex=   pxex at14pt
\font\twelveex=     pxex at12pt
\font\tenex=        pxex
\font\nineex=       pxex at9pt
\font\eightex=      pxex at8pt
\font\sevenex=      pxex at7pt
\font\sixex=        pxex at6pt
\font\fiveex=       pxex at5pt

% italic text (Family 4)
\font\twentyit=     pxi at20pt
\font\eighteenit=   pxi at18pt
\font\sixteenit=    pxi at16pt
\font\fourteenit=   pxi at14pt
\font\twelveit=     pxi at12pt
\font\tenit=        pxi
\font\nineit=       pxi at9pt
\font\eightit=      pxi at8pt
\font\sevenit=      pxi at7pt
\font\sixit=        pxi at6pt
\font\fiveit=       pxi at5pt

% slanted text (Family 5)
\font\twentysl=     pxsl at20pt
\font\eighteensl=   pxsl at18pt
\font\sixteensl=    pxsl at16pt
\font\fourteensl=   pxsl at14pt
\font\twelvesl=     pxsl at12pt
\font\tensl=        pxsl
\font\ninesl=       pxsl at9pt
\font\eightsl=      pxsl at8pt
\font\sevensl=      pxsl at7pt
\font\sixsl=        pxsl at6pt
\font\fivesl=       pxsl at5pt

% boldface text (Family 6)
\font\twentybf=     pxb at20pt
\font\eighteenbf=   pxb at18pt
\font\sixteenbf=    pxb at16pt
\font\fourteenbf=   pxb at14pt
\font\twelvebf=     pxb at12pt
\font\tenbf=        pxb
\font\ninebf=       pxb at9pt
\font\eightbf=      pxb at8pt
\font\sevenbf=      pxb at7pt
\font\sixbf=        pxb at6pt
\font\fivebf=       pxb at5pt

% typewriter text (Family 7)
\font\twentytt=     cmtt10 at20pt
\font\eighteentt=   cmtt10 at18pt
\font\sixteentt=    cmtt10 at16pt
\font\fourteentt=   cmtt10 at14pt
\font\twelvett=     cmtt10 at12pt
\font\tentt=        cmtt10
\font\ninett=       cmtt9
\font\eighttt=      cmtt8
\font\seventt=      cmtt10 at7pt
\font\sixtt=        cmtt10 at6pt
\font\fivett=       cmtt10 at5pt




% Family 0 (roman text)
\textfont0=\tenrm
\scriptfont0=\sevenrm
\scriptscriptfont0=\fiverm
\def\rm{\fam=0 \tenrm}

% Family 1 (math italic)
\textfont1=\teni
\scriptfont1=\seveni
\scriptscriptfont1=\fivei
\def\mit{\fam=1}

% Family 2 (math symbol)
\textfont2=\tensy
\scriptfont2=\sevensy
\scriptscriptfont2=\fivesy
\def\cal{\fam=2}

% Family 3 (math extension)
\textfont3=\tenex
\scriptfont3=\sevenex
\scriptscriptfont3=\fiveex


% Family 4 (italic text)
\def\it{\fam=\itfam \tenit}
\textfont\itfam=\tenit
\scriptfont\itfam=\sevenit
\scriptscriptfont\itfam=\fiveit

% Family 5 (slanted text)
\def\sl{\fam=\slfam \tensl}
\textfont\slfam=\tensl
\scriptfont\slfam=\sevensl
\scriptscriptfont\slfam=\fivesl

% Family 6 (boldface text)
\def\bf{\fam=\bffam \tenbf}
\textfont\bffam=\tenbf
\scriptfont\bffam=\sevenbf
\scriptscriptfont\bffam=\fivebf

% Family 7 (typewriter text)
\def\tt{\fam=\ttfam \tentt}
\textfont\ttfam=\tentt
\scriptfont\ttfam=\seventt
\scriptscriptfont\ttfam=\fivett

% italic boldface
\font\twentyitbf=      pxbi at20pt
\font\eighteenitbf=    pxbi at18pt
\font\sixteenitbf=     pxbi at16pt
\font\fourteenitbf=    pxbi at14pt
\font\twelveitbf=      pxbi at12pt
\font\itbf=            pxbi
\font\nineitbf=        pxbi at9pt
\font\eightitbf=       pxbi at8pt
\font\sevenitbf=       pxbi at7pt
\font\sixitbf=         pxbi at6pt
\font\fiveitbf=        pxbi at5pt

% slanted boldface
\font\twentyslbf=      pxbsl at20pt
\font\eighteenslbf=    pxbsl at18pt
\font\sixteenslbf=     pxbsl at16pt
\font\fourteenslbf=    pxbsl at14pt
\font\twelveslbf=      pxbsl at12pt
\font\slbf=            pxbsl
\font\nineslbf=        pxbsl at9pt
\font\eightslbf=       pxbsl at8pt
\font\sevenslbf=       pxbsl at7pt
\font\sixslbf=         pxbsl at6pt
\font\fiveslbf=        pxbsl at5pt


% caps
\font\twentycaps=      pxsc at20pt
\font\eighteencaps=    pxsc at18pt
\font\sixteencaps=     pxsc at16pt
\font\fourteencaps=    pxsc at14pt
\font\twelvecaps=      pxsc at12pt
\font\caps=            pxsc  % cs-qplr-sc of TeX Gyre or pplrc7t of Adobe can also be used.
\font\ninecaps=        pxsc at9pt
\font\eightcaps=       pxsc at8pt
\font\sevencaps=       pxsc at7pt
\font\sixcaps=         pxsc at6pt
\font\fivecaps=        pxsc at5pt

% caps boldface
\font\twentycapsbf=      pxbsc at20pt
\font\eighteencapsbf=    pxbsc at18pt
\font\sixteencapsbf=     pxbsc at16pt
\font\fourteencapsbf=    pxbsc at14pt
\font\twelvecapsbf=      pxbsc at12pt
\font\capsbf=            pxbsc % cs-qplb-sc of TeX Gyre or pplrc7t of Adobe can also be used.
\font\ninecapsbf=        pxbsc at9pt
\font\eightcapsbf=       pxbsc at8pt
\font\sevencapsbf=       pxbsc at7pt
\font\sixcapsbf=         pxbsc at6pt
\font\fivecapsbf=        pxbsc at5pt

\font\handbf=pplbu

\rm 