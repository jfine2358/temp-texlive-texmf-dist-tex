% The author of this work is Amit Raj Dhawan.
% This work has been released under 
% Creative Commons Attribution-Share Alike 3.0 Unported License 
% on August 17, 2009. For details visit: 
% http://creativecommons.org/licenses/by-sa/3.0/.

% roman text (Family 0)
\font\twentyrm=     txr at20pt
\font\eighteenrm=   txr at18pt
\font\sixteenrm=    txr at16pt
\font\fourteenrm=   txr at14pt
\font\twelverm=     txr at12pt
\font\tenrm=        txr
\font\ninerm=       txr at9pt
\font\eightrm=      txr at8pt
\font\sevenrm=      txr at7pt
\font\sixrm=        txr at6pt
\font\fiverm=       txr at5pt

% math italic (Family 1)
\font\twentyi=      txmi at20pt
\font\eighteeni=    txmi at18pt
\font\sixteeni=     txmi at16pt
\font\fourteeni=    txmi at14pt
\font\twelvei=      txmi at12pt
\font\teni=         txmi
\font\ninei=        txmi at9pt
\font\eighti=       txmi at8pt
\font\seveni=       txmi at7pt
\font\sixi=         txmi at6pt
\font\fivei=        txmi at5pt

% math symbols (Family 2)
\font\twentysy=     txsy at20pt
\font\eighteensy=   txsy at18pt
\font\sixteensy=    txsy at16pt
\font\fourteensy=   txsy at14pt
\font\twelvesy=     txsy at12pt
\font\tensy=        txsy
\font\ninesy=       txsy at9pt
\font\eightsy=      txsy at8pt
\font\sevensy=      txsy at7pt
\font\sixsy=        txsy at6pt
\font\fivesy=       txsy at5pt

% math extension (Family 3)
\font\twentyex=     txex at20pt
\font\eighteenex=   txex at18pt
\font\sixteenex=    txex at16pt
\font\fourteenex=   txex at14pt
\font\twelveex=     txex at12pt
\font\tenex=        txex
\font\nineex=       txex at9pt
\font\eightex=      txex at8pt
\font\sevenex=      txex at7pt
\font\sixex=        txex at6pt
\font\fiveex=       txex at5pt

% italic text (Family 4)
\font\twentyit=     txi at20pt
\font\eighteenit=   txi at18pt
\font\sixteenit=    txi at16pt
\font\fourteenit=   txi at14pt
\font\twelveit=     txi at12pt
\font\tenit=        txi
\font\nineit=       txi at9pt
\font\eightit=      txi at8pt
\font\sevenit=      txi at7pt
\font\sixit=        txi at6pt
\font\fiveit=       txi at5pt

% slanted text (Family 5)
\font\twentysl=     txsl at20pt
\font\eighteensl=   txsl at18pt
\font\sixteensl=    txsl at16pt
\font\fourteensl=   txsl at14pt
\font\twelvesl=     txsl at12pt
\font\tensl=        txsl
\font\ninesl=       txsl at9pt
\font\eightsl=      txsl at8pt
\font\sevensl=      txsl at7pt
\font\sixsl=        txsl at6pt
\font\fivesl=       txsl at5pt

% boldface text (Family 6)
\font\twentybf=     txb at20pt
\font\eighteenbf=   txb at18pt
\font\sixteenbf=    txb at16pt
\font\fourteenbf=   txb at14pt
\font\twelvebf=     txb at12pt
\font\tenbf=        txb
\font\ninebf=       txb at9pt
\font\eightbf=      txb at8pt
\font\sevenbf=      txb at7pt
\font\sixbf=        txb at6pt
\font\fivebf=       txb at5pt

% typewriter text (Family 7)
\font\twentytt=     cmtt10 at20pt
\font\eighteentt=   cmtt10 at18pt
\font\sixteentt=    cmtt10 at16pt
\font\fourteentt=   cmtt10 at14pt
\font\twelvett=     cmtt10 at12pt
\font\tentt=        cmtt10
\font\ninett=       cmtt9
\font\eighttt=      cmtt8
\font\seventt=      cmtt10 at7pt
\font\sixtt=        cmtt10 at6pt
\font\fivett=       cmtt10 at5pt




% Family 0 (roman text)
\textfont0=\tenrm
\scriptfont0=\sevenrm
\scriptscriptfont0=\fiverm
\def\rm{\fam=0 \tenrm}

% Family 1 (math italic)
\textfont1=\teni
\scriptfont1=\seveni
\scriptscriptfont1=\fivei
\def\mit{\fam=1}

% Family 2 (math symbol)
\textfont2=\tensy
\scriptfont2=\sevensy
\scriptscriptfont2=\fivesy
\def\cal{\fam=2}

% Family 3 (math extension)
\textfont3=\tenex
\scriptfont3=\sevenex
\scriptscriptfont3=\fiveex


% Family 4 (italic text)
\def\it{\fam=\itfam \tenit}
\textfont\itfam=\tenit
\scriptfont\itfam=\sevenit
\scriptscriptfont\itfam=\fiveit

% Family 5 (slanted text)
\def\sl{\fam=\slfam \tensl}
\textfont\slfam=\tensl
\scriptfont\slfam=\sevensl
\scriptscriptfont\slfam=\fivesl

% Family 6 (boldface text)
\def\bf{\fam=\bffam \tenbf}
\textfont\bffam=\tenbf
\scriptfont\bffam=\sevenbf
\scriptscriptfont\bffam=\fivebf

% Family 7 (typewriter text)
\def\tt{\fam=\ttfam \tentt}
\textfont\ttfam=\tentt
\scriptfont\ttfam=\seventt
\scriptscriptfont\ttfam=\fivett

% italic boldface
\font\twentyitbf=      txbi at20pt
\font\eighteenitbf=    txbi at18pt
\font\sixteenitbf=     txbi at16pt
\font\fourteenitbf=    txbi at14pt
\font\twelveitbf=      txbi at12pt
\font\itbf=            txbi
\font\nineitbf=        txbi at9pt
\font\eightitbf=       txbi at8pt
\font\sevenitbf=       txbi at7pt
\font\sixitbf=         txbi at6pt
\font\fiveitbf=        txbi at5pt

% slanted boldface
\font\twentyslbf=      txbsl at20pt
\font\eighteenslbf=    txbsl at18pt
\font\sixteenslbf=     txbsl at16pt
\font\fourteenslbf=    txbsl at14pt
\font\twelveslbf=      txbsl at12pt
\font\slbf=            txbsl
\font\nineslbf=        txbsl at9pt
\font\eightslbf=       txbsl at8pt
\font\sevenslbf=       txbsl at7pt
\font\sixslbf=         txbsl at6pt
\font\fiveslbf=        txbsl at5pt

% caps
\font\twentycaps=      txsc at20pt
\font\eighteencaps=    txsc at18pt
\font\sixteencaps=     txsc at16pt
\font\fourteencaps=    txsc at14pt
\font\twelvecaps=      txsc at12pt
\font\caps=            txsc  % cs-qtmr-sc of TeX Gyre or ptmrc7t of Adobe can also be used.
\font\ninecaps=        txsc at9pt
\font\eightcaps=       txsc at8pt
\font\sevencaps=       txsc at7pt
\font\sixcaps=         txsc at6pt
\font\fivecaps=        txsc at5pt

% caps boldface
\font\twentycapsbf=      txbsc at20pt
\font\eighteencapsbf=    txbsc at18pt
\font\sixteencapsbf=     txbsc at16pt
\font\fourteencapsbf=    txbsc at14pt
\font\twelvecapsbf=      txbsc at12pt
\font\capsbf=            txbsc % cs-qtmb-sc of TeX Gyre or ptmbc7t of Adobe can also be used.
\font\ninecapsbf=        txbsc at9pt
\font\eightcapsbf=       txbsc at8pt
\font\sevencapsbf=       txbsc at7pt
\font\sixcapsbf=         txbsc at6pt
\font\fivecapsbf=        txbsc at5pt
\rm 