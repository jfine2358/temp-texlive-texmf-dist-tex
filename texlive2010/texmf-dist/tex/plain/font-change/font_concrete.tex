% The author of this work is Amit Raj Dhawan.
% This work has been released under 
% Creative Commons Attribution-Share Alike 3.0 Unported License 
% on August 17, 2009. For details visit: 
% http://creativecommons.org/licenses/by-sa/3.0/.

% Does not work with pdftex

% roman text (Family 0)
\font\twentyrm=     pcr10 at20pt
\font\eighteenrm=   pcr10 at18pt
\font\sixteenrm=    pcr10 at16pt
\font\fourteenrm=   pcr10 at14pt
\font\twelverm=     pcr10 at12pt
\font\tenrm=        pcr10 % ccr10 the original but bitmap % vnccr10, eoti10 misses \Delta etc.
\font\ninerm=       pcr10 at9pt
\font\eightrm=      pcr10 at8pt
\font\sevenrm=      pcr10 at7pt % there is pcr7 but we use this for consistency
\font\sixrm=        pcr10 at6pt
\font\fiverm=       pcr10 at5pt % there is pcr5 but we use this for consistency

% math italic (Family 1)
\font\twentyi=      pcmi10 at20pt
\font\eighteeni=    pcmi10 at18pt
\font\sixteeni=     pcmi10 at16pt
\font\fourteeni=    pcmi10 at14pt
\font\twelvei=      pcmi10 at12pt
\font\teni=         pcmi10 % xccmi10 original but bitmap. xccmi10 also fine
\font\ninei=        pcmi10 at9pt
\font\eighti=       pcmi10 at8pt
\font\seveni=       pcmi10 at7pt % xccmi7 original but bitmap
\font\sixi=         pcmi10 at6pt
\font\fivei=        pcmi10 at5pt % xccmi5 original but bitmap

% math symbols (Family 2)
\font\twentysy=     cmsy10 at20pt
\font\eighteensy=   cmsy10 at18pt
\font\sixteensy=    cmsy10 at16pt
\font\fourteensy=   cmsy10 at14pt
\font\twelvesy=     cmsy10 at12pt
\font\tensy=        cmsy10 % xccsy10 original but bitmap, and no big difference
\font\ninesy=       cmsy10 at9pt
\font\eightsy=      cmsy10 at8pt
\font\sevensy=      cmsy10 at7pt % xccsy7 original but bitmap, and no big difference
\font\sixsy=        cmsy10 at6pt
\font\fivesy=       cmsy10 at5pt % xccsy5 original but bitmap, and no big difference

% math extension (Family 3)
\font\twentyex=     cmex10 at20pt
\font\eighteenex=   cmex10 at18pt
\font\sixteenex=    cmex10 at16pt
\font\fourteenex=   cmex10 at14pt
\font\twelveex=     cmex10 at12pt
\font\tenex=        cmex10 % xccex10 original but bitmap, and no big difference
\font\nineex=       cmex9
\font\eightex=      cmex8
\font\sevenex=      cmex7 % xccex7 original but bitmap, and no big difference
\font\sixex=        cmex10 at6pt
\font\fiveex=       cmex10 at5pt % xccex5 original but bitmap, and no big difference

% italic text (Family 4)
\font\twentyit=     pcti10 at20pt
\font\eighteenit=   pcti10 at18pt
\font\sixteenit=    pcti10 at16pt
\font\fourteenit=   pcti10 at14pt
\font\twelveit=     pcti10 at12pt
\font\tenit=        pcti10 % ccti10 original but bitmap
\font\nineit=       pcti10 at9pt
\font\eightit=      pcti10 at8pt
\font\sevenit=      pcti10 at7pt
\font\sixit=        pcti10 at6pt
\font\fiveit=       pcti10 at5pt

% slanted text (Family 5)
\font\twentysl=     pcsl10 at20pt
\font\eighteensl=   pcsl10 at18pt
\font\sixteensl=    pcsl10 at16pt
\font\fourteensl=   pcsl10 at14pt
\font\twelvesl=     pcsl10 at12pt
\font\tensl=        pcsl10 % ccsl10
\font\ninesl=       pcsl10 at9pt
\font\eightsl=      pcsl10 at8pt
\font\sevensl=      pcsl10 at7pt
\font\sixsl=        pcsl10 at6pt
\font\fivesl=       pcsl10 at5pt

% boldface text (Family 6)

% typewriter text (Family 7)
\font\twentytt=     cmtt10 at20pt
\font\eighteentt=   cmtt10 at18pt
\font\sixteentt=    cmtt10 at16pt
\font\fourteentt=   cmtt10 at14pt
\font\twelvett=     cmtt10 at12pt
\font\tentt=        cmtt10
\font\ninett=       cmtt9
\font\eighttt=      cmtt8
\font\seventt=      cmtt10 at7pt
\font\sixtt=        cmtt10 at6pt
\font\fivett=       cmtt10 at5pt




% Family 0 (roman text)
\textfont0=\tenrm
\scriptfont0=\sevenrm
\scriptscriptfont0=\fiverm
\def\rm{\fam=0 \tenrm}

% Family 1 (math italic)
\textfont1=\teni
\scriptfont1=\seveni
\scriptscriptfont1=\fivei
\def\mit{\fam=1}

% Family 2 (math symbol)
\textfont2=\tensy
\scriptfont2=\sevensy
\scriptscriptfont2=\fivesy
\def\cal{\fam=2}

% Family 3 (math extension)
\textfont3=\tenex
\scriptfont3=\sevenex
\scriptscriptfont3=\fiveex


% Family 4 (italic text)
\def\it{\fam=\itfam \tenit}
\textfont\itfam=\tenit
\scriptfont\itfam=\sevenit
\scriptscriptfont\itfam=\fiveit

% Family 5 (slanted text)
\def\sl{\fam=\slfam \tensl}
\textfont\slfam=\tensl
\scriptfont\slfam=\sevensl
\scriptscriptfont\slfam=\fivesl

% Family 6 (boldface text)
\def\bf{\fam=\bffam \tenbf}
\textfont\bffam=\tenbf
\scriptfont\bffam=\sevenbf
\scriptscriptfont\bffam=\fivebf

% Family 7 (typewriter text)
\def\tt{\fam=\ttfam \tentt}
\textfont\ttfam=\tentt
\scriptfont\ttfam=\seventt
\scriptscriptfont\ttfam=\fivett

% italic boldface

% slanted boldface

% caps
\font\twentycaps=      pccsc10 at20pt
\font\eighteencaps=    pccsc10 at18pt
\font\sixteencaps=     pccsc10 at16pt
\font\fourteencaps=    pccsc10 at14pt
\font\twelvecaps=      pccsc10 at12pt
\font\caps=            pccsc10 % vncccsc10
\font\ninecaps=        pccsc10 at9pt
\font\eightcaps=       pccsc10 at8pt
\font\sevencaps=       pccsc10 at7pt
\font\sixcaps=         pccsc10 at6pt
\font\fivecaps=        pccsc10 at5pt

% caps bold missing!

\rm 