% The author of this work is Amit Raj Dhawan.
% This work has been released under 
% Creative Commons Attribution-Share Alike 3.0 Unported License 
% on August 17, 2009. For details visit: 
% http://creativecommons.org/licenses/by-sa/3.0/.

% The "kurier" family has other versions like "condensed", "light", etc. Stick to cs-???????

% roman text (Family 0)
\font\twentyrm=     cs-kurierr at20pt
\font\eighteenrm=   cs-kurierr at18pt
\font\sixteenrm=    cs-kurierr at16pt
\font\fourteenrm=   cs-kurierr at14pt
\font\twelverm=     cs-kurierr at12pt
\font\tenrm=        cs-kurierr
\font\ninerm=       cs-kurierr at9pt
\font\eightrm=      cs-kurierr at8pt
\font\sevenrm=      cs-kurierr at7pt
\font\sixrm=        cs-kurierr at6pt
\font\fiverm=       cs-kurierr at5pt

% math italic (Family 1)
\font\twentyi=      mi-kurierri at20pt
\font\eighteeni=    mi-kurierri at18pt
\font\sixteeni=     mi-kurierri at16pt
\font\fourteeni=    mi-kurierri at14pt
\font\twelvei=      mi-kurierri at12pt
\font\teni=         mi-kurierri
\font\ninei=        mi-kurierri at9pt
\font\eighti=       mi-kurierri at8pt
\font\seveni=       mi-kurierri at7pt
\font\sixi=         mi-kurierri at6pt
\font\fivei=        mi-kurierri at5pt

% math symbols (Family 2)
\font\twentysy=     sy-kurierrz at20pt
\font\eighteensy=   sy-kurierrz at18pt
\font\sixteensy=    sy-kurierrz at16pt
\font\fourteensy=   sy-kurierrz at14pt
\font\twelvesy=     sy-kurierrz at12pt
\font\tensy=        sy-kurierrz
\font\ninesy=       sy-kurierrz at9pt
\font\eightsy=      sy-kurierrz at8pt
\font\sevensy=      sy-kurierrz at7pt
\font\sixsy=        sy-kurierrz at6pt
\font\fivesy=       sy-kurierrz at5pt

% math extension (Family 3)
\font\twentyex=     ex-kurierr at20pt
\font\eighteenex=   ex-kurierr at18pt
\font\sixteenex=    ex-kurierr at16pt
\font\fourteenex=   ex-kurierr at14pt
\font\twelveex=     ex-kurierr at12pt
\font\tenex=        ex-kurierr
\font\nineex=       ex-kurierr at9pt
\font\eightex=      ex-kurierr at8pt
\font\sevenex=      ex-kurierr at7pt
\font\sixex=        ex-kurierr at6pt
\font\fiveex=       ex-kurierr at5pt

% italic text (Family 4)
\font\twentyit=     cs-kurierri at20pt
\font\eighteenit=   cs-kurierri at18pt
\font\sixteenit=    cs-kurierri at16pt
\font\fourteenit=   cs-kurierri at14pt
\font\twelveit=     cs-kurierri at12pt
\font\tenit=        cs-kurierri
\font\nineit=       cs-kurierri at9pt
\font\eightit=      cs-kurierri at8pt
\font\sevenit=      cs-kurierri at7pt
\font\sixit=        cs-kurierri at6pt
\font\fiveit=       cs-kurierri at5pt

% slanted text (Family 5)
\font\twentysl=     cs-kurierri at20pt
\font\eighteensl=   cs-kurierri at18pt
\font\sixteensl=    cs-kurierri at16pt
\font\fourteensl=   cs-kurierri at14pt
\font\twelvesl=     cs-kurierri at12pt
\font\tensl=        cs-kurierri
\font\ninesl=       cs-kurierri at9pt
\font\eightsl=      cs-kurierri at8pt
\font\sevensl=      cs-kurierri at7pt
\font\sixsl=        cs-kurierri at6pt
\font\fivesl=       cs-kurierri at5pt

% boldface text (Family 6)
\font\twentybf=     cs-kurierb at20pt
\font\eighteenbf=   cs-kurierb at18pt
\font\sixteenbf=    cs-kurierb at16pt
\font\fourteenbf=   cs-kurierb at14pt
\font\twelvebf=     cs-kurierb at12pt
\font\tenbf=        cs-kurierb
\font\ninebf=       cs-kurierb at9pt
\font\eightbf=      cs-kurierb at8pt
\font\sevenbf=      cs-kurierb at7pt
\font\sixbf=        cs-kurierb at6pt
\font\fivebf=       cs-kurierb at5pt

% typewriter text (Family 7)
\font\twentytt=     cmtt10 at20pt
\font\eighteentt=   cmtt10 at18pt
\font\sixteentt=    cmtt10 at16pt
\font\fourteentt=   cmtt10 at14pt
\font\twelvett=     cmtt10 at12pt
\font\tentt=        cmtt10
\font\ninett=       cmtt9
\font\eighttt=      cmtt8
\font\seventt=      cmtt10 at7pt
\font\sixtt=        cmtt10 at6pt
\font\fivett=       cmtt10 at5pt




% Family 0 (roman text)
\textfont0=\tenrm
\scriptfont0=\sevenrm
\scriptscriptfont0=\fiverm
\def\rm{\fam=0 \tenrm}

% Family 1 (math italic)
\textfont1=\teni
\scriptfont1=\seveni
\scriptscriptfont1=\fivei
\def\mit{\fam=1}

% Family 2 (math symbol)
\textfont2=\tensy
\scriptfont2=\sevensy
\scriptscriptfont2=\fivesy
\def\cal{\fam=2}

% Family 3 (math extension)
\textfont3=\tenex
\scriptfont3=\sevenex
\scriptscriptfont3=\fiveex


% Family 4 (italic text)
\def\it{\fam=\itfam \tenit}
\textfont\itfam=\tenit
\scriptfont\itfam=\sevenit
\scriptscriptfont\itfam=\fiveit

% Family 5 (slanted text)
\def\sl{\fam=\slfam \tensl}
\textfont\slfam=\tensl
\scriptfont\slfam=\sevensl
\scriptscriptfont\slfam=\fivesl

% Family 6 (boldface text)
\def\bf{\fam=\bffam \tenbf}
\textfont\bffam=\tenbf
\scriptfont\bffam=\sevenbf
\scriptscriptfont\bffam=\fivebf

% Family 7 (typewriter text)
\def\tt{\fam=\ttfam \tentt}
\textfont\ttfam=\tentt
\scriptfont\ttfam=\seventt
\scriptscriptfont\ttfam=\fivett

% italic boldface
\font\twentyitbf=      cs-kurierbi at20pt
\font\eighteenitbf=    cs-kurierbi at18pt
\font\sixteenitbf=     cs-kurierbi at16pt
\font\fourteenitbf=    cs-kurierbi at14pt
\font\twelveitbf=      cs-kurierbi at12pt
\font\itbf=            cs-kurierbi
\font\nineitbf=        cs-kurierbi at9pt
\font\eightitbf=       cs-kurierbi at8pt
\font\sevenitbf=       cs-kurierbi at7pt
\font\sixitbf=         cs-kurierbi at6pt
\font\fiveitbf=        cs-kurierbi at5pt

% slanted boldface
\font\twentyslbf=      cs-kurierbi at20pt
\font\eighteenslbf=    cs-kurierbi at18pt
\font\sixteenslbf=     cs-kurierbi at16pt
\font\fourteenslbf=    cs-kurierbi at14pt
\font\twelveslbf=      cs-kurierbi at12pt
\font\slbf=            cs-kurierbi
\font\nineslbf=        cs-kurierbi at9pt
\font\eightslbf=       cs-kurierbi at8pt
\font\sevenslbf=       cs-kurierbi at7pt
\font\sixslbf=         cs-kurierbi at6pt
\font\fiveslbf=        cs-kurierbi at5pt

% caps
\font\twentycaps=      cs-kurierrcap at20pt
\font\eighteencaps=    cs-kurierrcap at18pt
\font\sixteencaps=     cs-kurierrcap at16pt
\font\fourteencaps=    cs-kurierrcap at14pt
\font\twelvecaps=      cs-kurierrcap at12pt
\font\caps=            cs-kurierrcap
\font\ninecaps=        cs-kurierrcap at9pt
\font\eightcaps=       cs-kurierrcap at8pt
\font\sevencaps=       cs-kurierrcap at7pt
\font\sixcaps=         cs-kurierrcap at6pt
\font\fivecaps=        cs-kurierrcap at5pt

% caps boldface
\font\twentycapsbf=      cs-kurierbcap at20pt
\font\eighteencapsbf=    cs-kurierbcap at18pt
\font\sixteencapsbf=     cs-kurierbcap at16pt
\font\fourteencapsbf=    cs-kurierbcap at14pt
\font\twelvcsapsbf=      cs-kurierbcap at12pt
\font\capsbf=            cs-kurierbcap
\font\nincsapsbf=        cs-kurierbcap at9pt
\font\eightcapsbf=       cs-kurierbcap at8pt
\font\sevencapsbf=       cs-kurierbcap at7pt
\font\sixcapsbf=         cs-kurierbcap at6pt
\font\fivcsapsbf=        cs-kurierbcap at5pt

\rm 