% knitting.tex
%
% Provides commands useful for writing knitting patterns in plain TeX
%
% author: Ariel Barton
%
% Copyright Ariel Barton, 2010
%
% This work may be distributed and/or modified under the
% conditions of the LaTeX Project Public License, either
% version 1.3 of this license or (at your option) any
% later version.
% The latest version of the license is in
%    http://www.latex-project.org/lppl.txt
% and version 1.3 or later is part of all distributions of
% LaTeX version 2003/06/01 or later.
%
% This work has the LPPL maintenance status "author-maintained".
%
% The complete list of files considered part of this work is in
% the file `knitting-doc.pdf' and its source code `knitting-doc.tex'.
%

% Date: 2010/01/17

%\pdfmapfile{+knitfont.map}

\ifnum\catcode`\@=\catcode`A \else
  \chardef\catcountknit=\catcode`@
  \catcode`@=11
\fi

\newif \ifchartsonly \chartsonlyfalse

\newif \ifpdf@knit
\pdf@knittrue
\ifx \pdfoutput \undefined \pdf@knitfalse \fi
\ifx \pdfoutput \relax     \pdf@knitfalse \fi

% This defines pdfTeX-friendly or dvips-friendly grays.
% Change them to use your driver's syntax if you insist on
% not using pdfTeX.

% For black lines, just say \let\knitlinecolor\relax
\ifpdf
	\ifnum \pdftexversion < 140
	\def\purlgray{\pdfliteral {0.65 g 0.65 G}\aftergroup\makenormal@knit}
	\def\gridcolor{\pdfliteral {0.3 g 0.3 G}\aftergroup\makenormalkp@knit} 
	\def\knitlinecolor{\pdfliteral {0.7 0 0 rg 0.7 0 0 RG}\aftergroup\makenormalkp@knit} 
	\def\makenormal@knit{\pdfliteral {0 g 0 G}}
	\def\makenormalkp@knit{\ifnum \passnum@knit = 0 \pdfliteral {0.65 g 0.65 G}\else\pdfliteral {0 g 0 G}\fi}
	\else
	\def\purlgray{\pdfcolorstack 0 push {0.65 g 0.65 G}\aftergroup\makenormal@knit}
	\def\gridcolor{\pdfcolorstack 0 push {0.3 g 0.3 G}\aftergroup\makenormal@knit}
	\def\knitlinecolor{\pdfcolorstack 0 push {0.7 0 0 rg 0.7 0 0 RG}\aftergroup\makenormal@knit}
	\def\makenormal@knit{\pdfcolorstack 0 pop}
	\fi
\else
	\def\purlgray{\special{color push gray 0.65}\aftergroup\makenormal@knit}
	\def\gridcolor{\special{color push gray 0.3}\aftergroup\makenormal@knit}
	\def\knitlinecolor{\special{color push rgb 0.7 0 0}\aftergroup\makenormal@knit}
	\def\makenormal@knit{\special{color pop}}
\fi

% If grayscale really just isn't working at all,
% go down and redefine \changeknitsize

\newif \ifgrid
\newif \ifresetrn \resetrntrue
\newif \ifleftrn@knit

\newdimen\leftgap@knit
\newdimen\bgshift@knit
\newdimen\chartwidth@knit

\newdimen\fontsize@knit

\newbox\bgbox@knit
\newbox\fgbox@knit
\newbox\ggbox@knit

\newcount\rownumber
\newcount\tempcount@knit
\newcount\rownumberskip \rownumberskip = 1
\def\passnum@knit{2} 

\def\stitchwd{\fontdimen6\ff@knit}
\def\stitchht{\fontdimen5\ff@knit}
\def\stitchdp{\fontdimen8\ff@knit}
% fontdimen9 is the LaTeX stitch height
\def\gridwidth{\fontdimen10\ff@knit}
\def\knitlinewd{\fontdimen11\ff@knit}
\def\narrowincraise@knit{\fontdimen12\the\font}
\def\fontvoffset@knit{\fontdimen13\the\font}
\def\purlextend@knit{\fontdimen14\ff@knit}
\def\rownumberwd{\fontdimen6\ff@knit}

\def\changeknitsize#1{\fontsize@knit = #1
	\font\knitsf = cmss10 at \fontsize@knit
	\font\rmten@knit = cmr10 at \fontsize@knit
	\font\knitsfsmall@knit = cmss8 at 0.8\fontsize@knit
	\font\cablesf@knit = cmss8 at 0.8\fontsize@knit
	\font\cablesfsmall@knit = ecss0600 at 0.6\fontsize@knit
	% I would use cmss as the sans serif font here, 
	% but cmss6 doesn't exist.
	%
	\font\gridff@knit = knitgn at \fontsize@knit
	\font\wideff@knit = knitwn at \fontsize@knit
	\font\nogridff@knit = knitnn at \fontsize@knit
	%
	\font\gridgf@knit = knitgg at \fontsize@knit
	\font\widegf@knit = knitwg at \fontsize@knit
	\font\raiseff@knit = knitnr at \fontsize@knit
	\font\lowerff@knit = knitnl at \fontsize@knit
	%
	\font\gridpf@knit = knitgp at \fontsize@knit
	\font\widepf@knit = knitwp at \fontsize@knit
	\font\nogridpf@knit = knitnp at \fontsize@knit
	%
	%%% Use if you can't get the grays to work.
	%%% In this case, don't use \Purl or \purlbox.
	% \let \gridpf@knit \nullfont
	% \let \widepf@knit \nullfont
	% \let \nogridpf@knit \nullfont
	% \let \purlgray \relax
}

\def\knitnogrid{\gridfalse\def\ff@knit{\nogridff@knit}\def\pf@knit{\nogridpf@knit}\def\gf@knit{\nullfont   }}
\def\knitgrid{\gridtrue   \def\ff@knit{\gridff@knit  }\def\pf@knit{\gridpf@knit  }\def\gf@knit{\gridgf@knit}}
\def\knitwide{\gridtrue   \def\ff@knit{\wideff@knit  }\def\pf@knit{\widepf@knit  }\def\gf@knit{\widegf@knit}}

\def\purlpass{\futurelet\next@knit\purlpass@@knit}
\def\gridpass{\futurelet\next@knit\gridpass@@knit}
\def\mainpass{\futurelet\next@knit\mainpass@@knit}
\def\purlpass@@knit{\ifx\next@knit[\def\nextstep@knit{\purlpass@knit}\else\def\nextstep@knit{\purlpass@knit[]}\fi\nextstep@knit}
\def\gridpass@@knit{\ifx\next@knit[\def\nextstep@knit{\gridpass@knit}\else\def\nextstep@knit{\gridpass@knit[]}\fi\nextstep@knit}
\def\mainpass@@knit{\ifx\next@knit[\def\nextstep@knit{\mainpass@knit}\else\def\nextstep@knit{\mainpass@knit[]}\fi\nextstep@knit}
\def\purlpass@knit[#1]#2{\ifnum\passnum@knit = 0 #2\else #1\fi}
\def\gridpass@knit[#1]#2{\ifnum\passnum@knit = 1 #2\else #1\fi}
\def\mainpass@knit[#1]#2{\ifnum\passnum@knit = 2 #2\else #1\fi}

\changeknitsize{10pt}

\knitgrid

\def\textknit#1{\ifvmode\leavevmode\fi\hbox{%
	\rlap{\pf@knit\purlgray\def\passnum@knit{0}#1}%
	\ifgrid\rlap{\gf@knit\gridcolor\def\passnum@knit{1}#1}\fi
	\ff@knit\def\passnum@knit{2}#1}}

\def\do@endofpar@knit#1{\def\par{#1\endgraf\let\par\endgraf}}

% Special symbols

\def \narrowdecrease#1{\char1\ifnum \passnum@knit = 2 \llap{\vbox to\stitchht{\vss\vskip\stitchdp\vskip\fontvoffset@knit\hbox to \stitchwd{\hfil\knitsfsmall@knit #1\hfil}\vskip-\narrowincraise@knit\vss}}\fi}
\def\pnarrowdecrease#1{\char2\ifnum \passnum@knit = 2 \llap{\vbox to\stitchht{\vss\vskip\stitchdp\vskip\fontvoffset@knit\hbox to \stitchwd{\hfil\knitsfsmall@knit #1\hfil}\vskip-\narrowincraise@knit\vss}}\fi}
\def \narrowincrease#1{\char3\ifnum \passnum@knit = 2 \llap{\vbox to\stitchht{\vss\vskip\stitchdp\vskip\fontvoffset@knit\hbox to \stitchwd{\hfil\knitsfsmall@knit #1\hfil}\vskip \narrowincraise@knit\vss}}\fi}
\def\pnarrowincrease#1{\char4\ifnum \passnum@knit = 2 \llap{\vbox to\stitchht{\vss\vskip\stitchdp\vskip\fontvoffset@knit\hbox to \stitchwd{\hfil\knitsfsmall@knit #1\hfil}\vskip \narrowincraise@knit\vss}}\fi}
\def\bobble#1{\char0         \ifnum \passnum@knit = 2 \llap{\vbox to\stitchht{\vss\vskip\stitchdp\vskip\fontvoffset@knit\hbox to \stitchwd{\hfil\knitsfsmall@knit #1\hfil}\vss}}\fi}

\def\wideincrease#1{\strut@knit
	\hbox to #1\stitchwd{\char25\leaders\hbox{\char22}\hfil\char29\leaders\hbox{\char22}\hfil\char26}}
\def\widedecrease#1{\strut@knit\ifnum #1 = 5
	\char31\else
	\hbox to #1\stitchwd{\char27\leaders\hbox{\char22}\hfil\char30\leaders\hbox{\char22}\hfil\char28}\fi}

% Standard chart commands

% \obeylines is normally defined with \let^^M\par, not \def^^M{\par};
% doing it this way makes it cooperate with \do@endofpar.

{\catcode`\^^M = \active \catcode`\| = \active
\global\def\commands@knit{%
	\let\[\pnarrowincrease
	\let\]\pnarrowdecrease
	\let\<\narrowincrease
	\let\>\narrowdecrease
	\let\@\bobble
	\let \! \barthin@knit
	\let \| \bar@knit
	\let | \bar@knit
	\let\par\endgraf %Just in case someone's redefined it
	\leftgap@knit=0pt
	\def~{\ifvmode \advance \leftgap@knit by \stitchwd
		\else \kern \stitchwd \fi}%
	\def^^M{\par}%
	\def\\{\par}%
	\let\_\horizline@knit
	\let\=\horizlinewide@knit
	\let\-\horizlinenarrow@knit
	\let\overline\overline@knit
	\let\underline\underline@knit
	\let\rn\rn@knit
	\let\rnbox\rnbox@knit
	\let\rnboxleft\rnboxleft@knit
	\let\rnboxright\rnboxright@knit
	\let\rnleft\rnleft@knit
	\let\rnright\rnright@knit
	\let\rncore@save@knit \rncore@knit
	\let\nonumber\relax
	}%
}

\def\strut@knit{\vrule width 0pt height \stitchht depth \stitchdp}

% Drawing lines on the chart

\def        \horizline@knit{\ifvmode\nonumber\leavevmode\fi \dimen0 =   \stitchwd \advance \dimen0 \knitlinewd 
	\ifnum\passnum@knit = 2
	\hskip -0.5\knitlinewd\smash{{\knitlinecolor\vrule width \dimen0 height 0.5\knitlinewd depth 0.5\knitlinewd}}\hskip -0.5\knitlinewd
	\else\hskip \stitchwd \fi}
\def\horizlinenarrow@knit#1{\ifvmode\nonumber\leavevmode\fi \dimen0 = #1\stitchwd \advance \dimen0 \gridwidth  
	\ifnum\passnum@knit = 2
	\hskip -0.5\gridwidth \smash{{\knitlinecolor\vrule width \dimen0 height 0.5\knitlinewd depth 0.5\knitlinewd}}\hskip -0.5\gridwidth
	\else\hskip #1\stitchwd \fi}
\def  \horizlinewide@knit#1{\ifvmode\nonumber\leavevmode\fi \dimen0 = #1\stitchwd \advance \dimen0 \knitlinewd 
 	\ifnum\passnum@knit = 2
                         \smash{{\knitlinecolor\vrule width \dimen0 height 0.5\knitlinewd depth 0.5\knitlinewd}}
	\else\hskip \dimen0 \fi}
	
\def\bar@knit{\ifvmode\leavevmode\fi\ifgrid
		\ifnum \passnum@knit = 0
		\hbox{\knitlinecolor\vrule width \knitlinewd depth \stitchdp height \stitchht}%
		\else
		\kern \knitlinewd
		\fi
	\else
		\ifnum \passnum@knit = 2
		\hbox{\knitlinecolor\vrule width \knitlinewd depth \stitchdp height \stitchht}%
		\else
		\kern \knitlinewd
		\fi
	\fi}

\def\barthin@knit{\ifvmode\leavevmode\fi
	\ifnum \passnum@knit = 2
	\hbox to 0pt{\hss\knitlinecolor\vrule width \knitlinewd depth \stitchdp height \stitchht\hss}%
	\else \strut@knit\fi}

\def\overline@knit#1{\setbox0 = \hbox{#1}%
	\leavevmode
	\ifgrid
	\ifnum \passnum@knit = 0
		\dimen1=\wd0 \advance\dimen1 by \gridwidth
		{\rlap{\raise\ht0\hbox{\hskip -0.5\gridwidth\knitlinecolor\vrule width \dimen1 height \knitlinewd depth 0pt}}}%
	\else
		\raise\ht0\hbox{\vrule width 0pt height \knitlinewd depth 0pt}%
	\fi
	\else
	\ifnum \passnum@knit = 2
		\dimen1=\wd0 \advance\dimen1 by \gridwidth
		{\rlap{\raise\ht0\hbox{\hskip -0.5\gridwidth\knitlinecolor\vrule width \dimen1 height \knitlinewd depth 0pt}}}%
	\else
		\dimen0=\ht0 \advance \dimen0 \knitlinewd
    	\vrule width 0pt height \dimen0 depth 0pt
	\fi
	\fi
	#1}

\def\underline@knit#1{\setbox0 = \hbox{#1}%
	\leavevmode
	\dimen1=\wd0 \advance\dimen1 by \gridwidth
	\ifgrid
	\ifnum \passnum@knit = 0
	\rlap{\raise-\dp0\hbox{\hskip -0.5\gridwidth\knitlinecolor\vrule width \dimen1 height 0pt depth \knitlinewd \hskip -0.5\gridwidth}}%
	\else
	\rlap{\raise-\dp0\hbox{\vrule width 0pt height 0pt depth \knitlinewd}}%
	\fi
	\else
	\ifnum \passnum@knit = 2
	\rlap{\raise-\dp0\hbox{\hskip -0.5\gridwidth\knitlinecolor\vrule width \dimen1 height 0pt depth \knitlinewd \hskip -0.5\gridwidth}}%
	\else
	\rlap{\raise-\dp0\hbox{\vrule width 0pt height 0pt depth \knitlinewd}}%
	\fi
	\fi
	\copy0
	}

% Fancy cabling

\def\overcableleft@knit#1{%
	\setbox0=\hbox{#1}%
	\rlap{#1}%
	\hbox to \wd0{\leaders\hbox to \stitchwd{\hfil\char5}\hfil\hskip\stitchwd\char10}}
\def\undercableleft@knit#1{%
	\setbox0=\hbox{#1}%
	\rlap{#1}%
	\hbox to \wd0{\leaders\hbox to \stitchwd{\hfil\char6}\hfil\hskip\stitchwd\char9}}
\def\undercableright@knit#1{%
	\setbox0=\hbox{#1}%
	\rlap{#1}%
	\hbox to \wd0{\char13\char12\hskip\stitchwd\leaders\hbox to \stitchwd{\char8\hfil}\hfil}}
\def\overcableright@knit#1{%
	\setbox0=\hbox{#1}%
	\rlap{#1}%
	\hbox to \wd0{\char14\char11\hskip\stitchwd\leaders\hbox to \stitchwd{\char7\hfil}\hfil}}

\def\cableleft#1#2{\ifvmode\leavevmode\fi 
	\ifcase\passnum@knit 
	\hbox{\gridfalse \nogridpf@knit #1#2}
	\or
	\setbox0=\hbox{\gridfalse \nogridpf@knit #1#2} \vrule width \wd0 height 0pt depth 0pt \vrule width 0pt height \ht0 depth \dp0
	\else
	\hbox{\def\ff@knit{\nogridff@knit}\gridfalse \let\knitsf \cablesf@knit \let \knitsfsmall@knit \cablesfsmall@knit {\lowerff@knit\overcableleft@knit{#1}}{\raiseff@knit\undercableright@knit{#2}}}\fi}
\def\cableright#1#2{\ifvmode\leavevmode\fi 
	\ifcase\passnum@knit 
	\hbox{\gridfalse \nogridpf@knit #1#2}
	\or
	\setbox0=\hbox{\gridfalse \nogridpf@knit #1#2} \vrule width \wd0 height 0pt depth 0pt \vrule width 0pt height \ht0 depth \dp0
	\else
	\hbox{\def\ff@knit{\nogridff@knit}\gridfalse \let\knitsf \cablesf@knit \let \knitsfsmall@knit \cablesfsmall@knit {\raiseff@knit\undercableleft@knit{#1}}{\lowerff@knit\overcableright@knit{#2}}}\fi}

% Knit, purl, knitboxes

\def\knit#1{\strut@knit\hbox to #1\stitchwd{\leaders\hbox{-}\hfil}}
\def\purl#1{\strut@knit\hbox to #1\stitchwd{\leaders\hbox{=}\hfil}}

\def\Knit#1#2{\strut@knit
	\setbox0 = \vbox to\stitchht{\vss\hbox{\knitsf #1}\vss\vskip\stitchdp}
	\ifcase \passnum@knit 
	\vrule width #2\stitchwd height 0pt depth 0pt
	\or
	\rlap{\hbox to #2\stitchwd{\leaders\hbox{\char5}\hfil}}%
	\hbox to #2\stitchwd{\leaders\hbox to \stitchwd{\char6\hfil\char6}\hfil\hskip\wd0\leaders\hbox to \stitchwd{\char6\hfil\char6}\hfil}%
	\or
	\hbox to #2\stitchwd{\leaders\hbox{-}\hfil \lower\stitchdp\box0\leaders\hbox{-}\hfil}
	\fi}
\def\Purl#1#2{\strut@knit
	\setbox0 = \vbox to\stitchht{\vss\hbox{\knitsf #1}\vss\vskip\stitchdp}%
	\ifcase \passnum@knit 
	\purlbackground{\vrule width #2\stitchwd depth \stitchdp height \stitchht}%
	\or
	\rlap{\hbox to #2\stitchwd{\leaders\hbox{\char5}\hfil}}%
	\hbox to #2\stitchwd{\leaders\hbox to \stitchwd{\char6\hfil\char6}\hfil\hskip\wd0\leaders\hbox to \stitchwd{\char6\hfil\char6}\hfil}%
	\or
	\hbox to #2\stitchwd{\leaders\hbox{=}\hfil \lower\stitchdp\box0\leaders\hbox{=}\hfil}
	\fi}

\def\knitbox#1#2{\strut@knit
	\ifcase \passnum@knit 
	\vrule width #2\stitchwd height 0pt depth 0pt
	\or
	\hbox to #2\stitchwd{\char6\leaders\hbox{\char5}\hfil\char6}%
	\or
	\hbox to #2\stitchwd{\hfil \lower\stitchdp\vbox to\stitchht{\vss\vskip\fontvoffset@knit\hbox{\knitsf #1}\vss\vskip\stitchdp}\hfil}
	\fi}
\def\purlbox#1#2{\strut@knit
	\ifcase \passnum@knit 
	\purlbackground{\vrule width #2\stitchwd height \stitchht depth \stitchdp}
	\or
	\hbox to #2\stitchwd{\char6\leaders\hbox{\char5}\hfil\char6}%
	\or
	\hbox to #2\stitchwd{\hfil \lower\stitchdp\vbox to\stitchht{\vss\vskip\fontvoffset@knit\hbox{\knitsf #1}\vss\vskip\stitchdp}\hfil}
	\fi}

\def\purlbackground#1{\ifvmode\leavevmode\fi\setbox0=\hbox{#1}%
	\dimen0 = \wd0 \advance\dimen0 2\purlextend@knit
	\dimen1 = \stitchht \advance\dimen1 \purlextend@knit
	\dimen2 = \stitchdp \advance\dimen2 \purlextend@knit
	\ifnum \passnum@knit = 0  \kern-\purlextend@knit
		\vrule width 0pt height \ht0 depth \dp0
		\smash{\vrule width \dimen0 height \dimen1 depth \dimen2}%
		\kern-\purlextend@knit
		\else\box0\fi}

% Row number commands

\def\numberrow#1#2#3{\ifvmode\nonumber\leavevmode\fi
	\strut@knit
	\count255=#1
	\hbox to \stitchwd{\hss\ifnum \passnum@knit = 2 \rmten@knit#1\fi\hss}%
	\advance\count255 -1
	\loop \ifnum \count255>#3
	\tempcount@knit = \count255
	\divide\tempcount@knit #2
	\multiply \tempcount@knit #2
	\ifnum\tempcount@knit = \count255
	\hbox to \stitchwd{\hss\ifnum \passnum@knit = 2 \rmten@knit\the\count255 \fi \hss}%
	\else
	\kern\stitchwd
	\fi
	\advance\count255 -1
	\repeat
	\hbox to \stitchwd{\hss\ifnum \passnum@knit = 2 \rmten@knit#3\fi \hss}%
	}

\def\rnoddonly{\def\rncore@knit{\ifnum \passnum@knit = 2  \ifodd\rownumber\rmten@knit\the\rownumber\fi\fi \rnstep@knit}}
\def\rnevenonly{\def\rncore@knit{\ifnum \passnum@knit = 2  \ifodd\rownumber\else\rmten@knit\the\rownumber\fi\fi \rnstep@knit}}
\def\rnnormal{\def\rncore@knit{\ifnum \passnum@knit = 2  \rmten@knit\the\rownumber\fi \rnstep@knit}}

\def\rncore@knit{\ifnum \passnum@knit = 2 \rmten@knit\the\rownumber\fi \rnstep@knit}
\def\rnstep@knit{\ifnum \passnum@knit = 2 \global\advance\rownumber -\rownumberskip\relax\fi
				 \ifnum \passnum@knit = 0 \global\advance\rownumber -\rownumberskip\relax\fi}

\def\rn@knit{\ifvmode\leavevmode\fi
	\hbox to \rownumberwd{\hss\rncore@knit\hss}}
\def\rnleft@knit{\ifvmode\leavevmode\fi
	\hbox to \rownumberwd{\hss\rnmiddle@knit\llap{\rncore@knit}\hss}}
\def\rnright@knit{\ifvmode\leavevmode\fi
	\hbox to \rownumberwd{\hss\rlap{\rncore@knit}\rnmiddle@knit\hss}}
\def\rnbox@knit#1{\ifvmode\leavevmode\fi\hbox to \rownumberwd{\hss\mainpass{\rmten@knit #1}\hss}}
\def\rnboxleft@knit#1{\ifvmode\leavevmode\fi\hbox to \rownumberwd{\hss\rnmiddle@knit\llap{\mainpass{\rmten@knit #1}}\hss}}
\def\rnboxright@knit#1{\ifvmode\leavevmode\fi\hbox to \rownumberwd{\hss\rnmiddle@knit\llap{\mainpass{\rmten@knit #1}}\hss}}

% The chart commands

\def\chart{\futurelet\next@knit\chart@@knit}
\def\chart@@knit{\ifx\next@knit[%
	\def\nextstep@knit{\smallpage@knit\obeylines \catcode`\|=\active \chart@knit}\else
	\def\nextstep@knit{\smallpage@knit\obeylines \catcode`\|=\active \chart@knit[]}\fi\nextstep@knit}

% Some special stuff for chartsonly mode
\let\extracommands@knit\relax
\def\smallpage@knit{\ifvmode\noindent\fi\hbox\bgroup}
\let\endsmallpage@knit\egroup

% The chart command proper
\long\def\chart@knit[#1]#2{%
	\global \chartwidth@knit = 0pt
	\ifresetrn \rownumber = 0 \else \tempcount@knit = \rownumber \fi
	\setbox1=\hbox{\rmten@knit 5}%
	\edef\rnmiddle@knit{\hskip \the\wd1}%
	\setbox\bgbox@knit=\vbox{\def\passnum@knit{0}\hsize=\maxdimen
		\pf@knit
		\lineskip=0pt
		\parskip=0pt
		\baselineskip=0pt
		\parindent=0pt
		\emergencystretch = \stitchwd
		\leftskip=0pt
		\rightskip=0pt
		\parfillskip = 0pt plus 1fil
		\ifresetrn\else\let\rnstep@knit\relax\fi
		\commands@knit\extracommands@knit
		\global \bgshift@knit = \rownumberwd
		\csname auto#1@knit\endcsname #2\par}%
	\ifresetrn \rownumber = -\rownumber \else
	           \rownumber = \tempcount@knit \fi
	\csname setbgshift#1@knit\endcsname
	\ifgrid
	\setbox\ggbox@knit=\vbox{\def\passnum@knit{1}\hsize=\maxdimen
		\gf@knit
		\lineskip=0pt
		\parskip=0pt
		\baselineskip=0pt
		\parindent=0pt
		\emergencystretch = \stitchwd
		\leftskip=0pt
		\rightskip=0pt
		\parfillskip = 0pt plus 1fil
		\commands@knit\extracommands@knit
		\csname auto#1@knit\endcsname #2\par}%
	\fi
	\setbox\fgbox@knit=\vbox{\def\passnum@knit{2}\hsize=\maxdimen
		\ff@knit
		\lineskip=0pt
		\parskip=0pt
		\baselineskip=0pt
		\parindent=0pt
		\emergencystretch = \stitchwd
		\leftskip=0pt
		\rightskip=0pt
		\parfillskip = 0pt plus 1fil
		\commands@knit\extracommands@knit
		\csname auto#1@knit\endcsname #2\par}%
	\hbox to \chartwidth@knit{%
	\rlap{\purlgray\hskip -\bgshift@knit \box\bgbox@knit}%
	\ifgrid\rlap{\gridcolor\box\ggbox@knit}\fi
	\box\fgbox@knit%
	\hss}%
	\ifchartsonly \vskip 0.5\gridwidth \fi \endsmallpage@knit
	}

% Special charts only macros

\def\chartsonly{\chartsonlytrue
	\ifpdf@knit\else\errmessage{\chartsonly should only be used with pdfTeX.}\fi%
	\hoffset=-1in
	\voffset=-1in
	\vsize = 120in
	\def\smallpage{\vfil\break
		\global\chartwidth@knit=0pt \setbox0 = \vbox\bgroup}
	\def\endsmallpage{\egroup%
		%
		\ifdim\chartwidth@knit>0pt \pdfpagewidth = \chartwidth@knit \else
		\pdfpagewidth=\wd0 \fi
		%
		\dimen0=\ht0 \advance \dimen0 by \dp0
		\pdfpageheight=\dimen0
		%
		\ifdim \pdfpageheight > \vsize
		{\newlinechar=`|
		\message{||You need to increase \string\vsize.}
		\message{What do you want such a big chart for, anyway?0||}}
		\fi
		%
		\box0
		\vfil\break
		}
	\let\smallpage@knit\smallpage
	\let\endsmallpage@knit\endsmallpage
	\def\extracommands@knit{\hsize=\maxdimen
		\leftskip = 0.5\gridwidth
		\rightskip = 0.5\gridwidth
		\vskip 0.5\gridwidth
		}
}
\let\smallpage\begingroup
\let\endsmallpage\endgroup

% Autonumbering macros

\def\everypar@knit{\hskip\leftgap@knit \leftgap@knit=0pt \relax}

\def        \auto@knit{\let\nonumber\relax \everypar={\everypar@knit\do@endofpar@knit{\adjustchartwidth@knit}}}
\def    \autoleft@knit{\def\nonumber{\def\rncore@knit{\global\let\rncore@knit\rncore@save@knit}}\everypar={\everypar@knit\rnleft@knit\do@endofpar@knit{\adjustchartwidth@knit}}}
\def   \autoright@knit{\def\nonumber{\def\rncore@knit{\global\let\rncore@knit\rncore@save@knit}}\everypar={\everypar@knit\do@endofpar@knit{\rnright@knit\adjustchartwidth@knit}}}
\def \autooddleft@knit{\def\nonumber{\def\rncore@knit{\global\let\rncore@knit\rncore@save@knit}}\everypar={\everypar@knit\ifodd \rownumber \rnleft@knit\do@endofpar@knit{\adjustchartwidth@knit}\else\do@endofpar@knit{\rnright@knit\adjustchartwidth@knit}\kern\rownumberwd\fi}}
\def\autooddright@knit{\def\nonumber{\def\rncore@knit{\global\let\rncore@knit\rncore@save@knit}}\everypar={\everypar@knit\ifodd \rownumber \do@endofpar@knit{\rnright@knit\adjustchartwidth@knit}\kern\rownumberwd\else\rnleft@knit\do@endofpar@knit{\adjustchartwidth@knit}\fi}}
\let\autoevenleft@knit\autooddright@knit
\let\autoevenright@knit\autooddleft@knit

\def         \setbgshift@knit{\leftrn@knitfalse \global \bgshift@knit = 0pt}
\def     \setbgshiftleft@knit{\leftrn@knittrue \resetrnwidth@knit \global \advance \bgshift@knit -\rownumberwd}
\def    \setbgshiftright@knit{\leftrn@knitfalse\resetrnwidth@knit \global \bgshift@knit = 0pt}
\def  \setbgshiftoddleft@knit{\leftrn@knittrue \resetrnwidth@knit \global \advance \bgshift@knit -\rownumberwd}
\def \setbgshiftoddright@knit{\leftrn@knittrue \resetrnwidth@knit \global \advance \bgshift@knit -\rownumberwd}
\def \setbgshiftevenleft@knit{\leftrn@knittrue \resetrnwidth@knit \global \advance \bgshift@knit -\rownumberwd}
\def\setbgshiftevenright@knit{\leftrn@knittrue \resetrnwidth@knit \global \advance \bgshift@knit -\rownumberwd}

% Make wider boxes if we've got autonumbered rows.
% We don't do this for \chart so that the rownumbers on the left
% can be aligned to with ~.
\def\resetrnwidth@knit{%
	\setbox1=\hbox{\rmten@knit 5}%
	\setbox0=\hbox{\hskip \bgshift@knit \hskip -\wd1 \rmten@knit \the\rownumber}%
	\edef\rownumberwd{\the\wd0}
	\setbox1=\hbox{\rmten@knit \the\rownumber}%
	\edef\rnmiddle@knit{\hskip\the\wd1}}

\def\adjustchartwidth@knit{\endgraf
	\ifnum \passnum@knit = 2 \setbox0=\lastbox
	\setbox1=\hbox{\unhcopy0\unskip}%
	\box0
	\ifdim\wd1 >\chartwidth@knit \global\chartwidth@knit=\wd1\fi\fi}

\catcode`\@=\catcountknit

