%% BEGIN pst-bspline.tex
%% Author: Michael Sharpe (msharpe at ucsd.edu)
\def\fileversion{1.2}
\def\filedate{2010/06/12}
%%
%\message{ v\fileversion, \filedate}
\csname PSTBsplineLoaded\endcsname
\let\PSTBsplineLoaded\endinput
\ProvidesFile{pst-bspline.tex}[\filedate\space v\fileversion\space Bspline routines for pstricks]
\ifx\PSTnodesLoaded\endinput\else\input pst-node.tex \fi\relax
\edef\TheAtCode{\the\catcode`\@}\catcode`\@=11
%
%\newcount\pst@args%used in several macros--now defined in pst-node.tex
\def\PST@root{}
\newcount\bsp@args
\newcount\bsp@cntA
\newcount\bsp@cntB
\newif\ifbsp@omitends
\newif\ifbsp@closed
%\newif\ifbsp@showpoints
%\newif\ifbsp@doubleline
\newif\ifbsp@seq
%
\def\bsp@init{%these items are used only within the plotting macros
\def\bsp@arrowA{}%
\def\bsp@arrowB{}%
}%
\def\refreshbspopts{%
\bsp@closedfalse%
\bsp@omitendsfalse%
\bsp@seqfalse%
}%
% showframe key and its default are global
\newif\ifshowframe%
\def\psset@showframe#1{\@nameuse{showframe#1}}%
\psset@showframe{false}%
%
%Bspline drawing macros
\def\psBsplineNodes{\def\pst@par{}\pst@object{psBsplineNodes}}%
\def\psBsplineNodesC{\def\pst@par{}\pst@object{psBsplineCNodes}}%
\def\psBsplineNodesE{\def\pst@par{}\pst@object{psBsplineENodes}}%
\def\psBsplineC{\def\pst@par{}\pst@object{psBsplineC}}%
\def\psBsplineE{\def\pst@par{}\pst@object{psBsplineE}}%
\def\psBspline{\def\pst@par{}\pst@object{psBspline}}%
%Bspline internal macros, level 1
\def\psBsplineC@i{\bsp@init\refreshbspopts\bsp@closedtrue\ps@bspline}%
\def\psBsplineE@i{\bsp@init\refreshbspopts\bsp@omitendstrue\ps@bspline}%
\def\psBspline@i{\bsp@init\refreshbspopts\ps@bspline}%
\def\psBsplineNodes@i{\bsp@init\refreshbspopts\bsp@seqtrue\ps@bsplineNodes}%
\def\psBsplineCNodes@i{\bsp@init\refreshbspopts\bsp@seqtrue\bsp@closedtrue%
\ps@bsplineNodes}%
\def\psBsplineENodes@i{\bsp@init\refreshbspopts\bsp@seqtrue\bsp@omitendstrue%
\ps@bsplineNodes}%
%Bspline internal macros, level 2
\def\ps@bsplineNodes#1#2{%noderoot, endindex
\def\bsp@root{#1}%
\bsp@args=#2\relax%
\next@Point%
}%
%
\def\ps@bspline{\bsp@args=0\relax%
\@ifnextchar({\psBspline@read{pn@de}}{\psBspline@read}}%
%   
\def\psBspline@read#1{%
\def\bsp@root{#1}%
\next@Point%
}%
%
\def\next@Point{\@ifnextchar({\next@Point@i}{\psBsplineMain}}%
\def\next@Point@i(#1){%
\pnode(#1){\bsp@root\the\bsp@args}%
\advance\bsp@args by \@ne%
\next@Point}%
% The main drawing routine
\def\psBsplineMain{%done reading nodes
\pst@killglue%
\begingroup%
\use@par%
\ifbsp@closed\begin@ClosedObj \else \begin@OpenObj \fi%
\let\bsp@linecolor\pslinecolor%
\let\bsp@arrowA\psk@arrowA%
\let\bsp@arrowB\psk@arrowB%
\let\bsp@linestyle\pslinestyle%
\ifbsp@seq\advance\bsp@args by \@ne\fi%
\ifbsp@closed%
  \psset{doubleline=false}%
  \pnode(\bsp@root0){\bsp@root\the\bsp@args}%
  \advance\bsp@args by \@ne%
  \pnode(\bsp@root1){\bsp@root\the\bsp@args}%
  \advance\bsp@args by \@ne%
\fi%
\advance\bsp@args by \m@ne%
% Define the left and right bezier control points in each interval. These are 
% denoted (root+) R0, L1, R1, L2, R2, etc
\multido{\ia=0+1,\ib=1+1}{\bsp@args}{%
\ifshowframe%
  \ncline[linestyle=dashed,linecolor=gray,arrows=*-*]{\bsp@root\ia}{\bsp@root\ib}%
\fi%
  \nodexn{.667(\bsp@root\ia)+.333(\bsp@root\ib)}{\bsp@root R\ia}%
  \nodexn{.333(\bsp@root\ia)+.667(\bsp@root\ib)}{\bsp@root L\ib}%
  }%
%Finally, define the bezier endpoints for each interval
\advance\bsp@args by \m@ne%
\mmultido{\ia=0+1}{\bsp@args}{%
\ifshowframe%
  \ncline[linestyle=solid,linecolor=red]{\bsp@root L\ia}{\bsp@root R\ia}%
\fi%
\midAB(\bsp@root L\ia)(\bsp@root R\ia){\bsp@root S\ia}%
\ifshowframe%
  \psdot[linecolor=red](\bsp@root S\ia)%
\fi%
}%
\ifbsp@closed%
  \pnode(\bsp@root S\the\bsp@args){\bsp@root S0}%
 \else%
   \advance\bsp@args by \@ne%
   \pnode(\bsp@root0){\bsp@root S0}%
   \pnode(\bsp@root\the\bsp@args){\bsp@root S\the\bsp@args}%
 \fi%
\advance\bsp@args by \m@ne%
%the pieces of bezier curve
\ifbsp@closed%
\pscustom[arrows=-]{%
\psbezier[showpoints=false](\bsp@root S0)(\bsp@root R0)(\bsp@root L1)(\bsp@root S1)%
\multido{\ia=1+1,\ib=2+1}{\bsp@args}{\psbezier(\bsp@root R\ia)(\bsp@root L\ib)%
(\bsp@root S\ib)}%
%within \pscustom, \psbezier takes 3 args, using last point of previous
}%
\else %open curve---make arrows, don't use \pscustom
 \advance\bsp@args by \m@ne%
 \ifbsp@omitends \bsp@cntA=2\relax \bsp@cntB=\@ne \advance\bsp@args by -2\relax%
 \else \bsp@cntA=\@ne\bsp@cntB=\z@ \fi%
 \psbezier[arrows=-\bsp@arrowA,showpoints=false](\bsp@root S\the\bsp@cntA)%
 (\bsp@root L\the\bsp@cntA)(\bsp@root R\the\bsp@cntB)(\bsp@root S\the\bsp@cntB)%
 \mmultido{\ia=\the\bsp@cntA+1,\ib=\the\bsp@cntB+1}{\bsp@args}%
 {\psbezier[arrows=-,showpoints=false](\bsp@root S\ib)(\bsp@root R\ib)%
 (\bsp@root L\ia)(\bsp@root S\ia)}%
 %
 \ifbsp@omitends \advance\bsp@args by \@ne \fi%
 \advance\bsp@args by \@ne%
 \bsp@cntA=\bsp@args%
 \bsp@cntB=\bsp@cntA%
 \advance\bsp@cntB by \@ne%
 \psbezier[arrows=-\bsp@arrowB,showpoints=false](\bsp@root S\the\bsp@cntA)%
 (\bsp@root R\the\bsp@cntA)(\bsp@root L\the\bsp@cntB)(\bsp@root S\the\bsp@cntB)%
 \ifshowframe \psdot[linecolor=red](\bsp@root L\the\bsp@cntB) \fi%
\fi% end \ifbsp@closed
\ifshowframe% 
  \psdot[linecolor=red](\bsp@root R0)%
  \psdot[linecolor=red](\bsp@root L\the\bsp@args)%
\fi%
\ifshowpoints%
  \multido{\i=0+1}{\bsp@args}{\psdot(\bsp@root S\i)}%
\fi%
 \ifbsp@closed%
   \end@ClosedObj%
 \else%
   \end@OpenObj%
 \fi%
\endgroup%
\ignorespaces%
}
%
\def\psbspline{\def\pst@par{}\pst@object{psbspline}}%
\def\psbspline@i{\pst@getarrows\psbspline@ii}%
\def\psbspline@ii{%
\pnodesX%make the following sequences into nodes named S@0,...
%last index=\pst@args
}%
\def\psbspline@iii{%
\pst@killglue%
\psBsplineInterp{S@}{\the\pst@args}%creates S@B0...
\begingroup%
\use@par%
\psBsplineNodes{S@B}{\the\pst@args}%draw curve
\endgroup\ignorespaces}%
%
\def\pnodesX{%version of \pnodes where the last step moves to \psbspline@iii
\pst@args=\z@%
\def\PST@root{S@}%
\next@NodeX}%
%
\def\next@NodeX{\@ifnextchar({\next@NodeX@i}{\advance\pst@args by \m@ne\typeout{Defined nodes \PST@root0 ..  \PST@root\the\pst@args.}\psbspline@iii}}%
%
\def\next@NodeX@i(#1){%
\pnode(#1){\PST@root\the\pst@args}%
\advance\pst@args\@ne%
\next@NodeX}%
%
\def\make@mcoeff#1{%
\pstVerb{ /mcoeff #1\space 1 add array def %
 mcoeff 1 0.25 put %
 1 1 #1\space 1 sub %
 {/j exch def %j is counter
 mcoeff j 1 add 1 4 mcoeff j get sub div put  } for }}%
 %
\def\psBsplineInterp#1#2{%
% #1=node name root to interpolate, #2=end index, 
% eg, if we want to interpolate S0,..., S100,  #1=S, #2=100,
\newcount\top@mone \top@mone=#2\relax%100
\newcount\top@ndx \top@ndx=\top@mone% 
\advance\top@ndx by -1\relax%99
\advance\top@mone by -2\relax%98
\newcount\top@mtwo \top@mtwo=\top@mone \advance\top@mtwo by \m@ne%97
\make@mcoeff{#2}%
\def\noderoot{#1}%
\pnode(\noderoot 0){\noderoot B0}%
\pnode(\noderoot #2){\noderoot B#2}%
\psLCNode(\noderoot 1){6}(\noderoot 0){-1}{\noderoot B1}% 6*S1-S0->SB1
\multido{\iA=2+1,\iAA=1+1}{\top@mone}{\psLCNodeVar(\noderoot\iA)(\noderoot B\iAA)(! 6 mcoeff \iAA\space get neg ){\noderoot B\iA}}%2 to n-2, 6*S_k-m_{k-1}*SB_{k-1}->SB_k
\psLCNodeVar(\noderoot \the\top@ndx)(\noderoot #2)(! mcoeff \the\top@ndx\space get dup 6 mul exch neg  ){tmp@node}%m_{99}*(6*S99-S100)->tmp@node
\psLCNodeVar(tmp@node)(\noderoot B\the\top@mone)(! 1 mcoeff \the\top@ndx\space get mcoeff \the\top@mone\space get mul neg ){\noderoot B\the\top@ndx}% temp@node -m_{98}*m_{99}*SB98->SB99
\multido{\iA=\the\top@mone+-1,\iAA=\the\top@ndx+-1}{\the\top@mone}{\psLCNodeVar(\noderoot B\iA)(\noderoot B\iAA)(! mcoeff \iA\space get dup neg ){\noderoot B\iA}% m_k*SB_k-m_k*SB_{k+1}->SB_k, k=98 downto 1
}% end multido
}% end def
\def\psBsplineInterpC#1#2{%
 % #1=node name root, #2=top index,
 % eg, if we want to interpolate S0,..., S100, #1=100, #2=S
 % interpolation in this case is periodic (closed curve)
\pst@killglue
\newcount\top@mone \top@mone=#2\relax%
\newcount\top@ndx \top@ndx=\top@mone% 
\advance\top@ndx by \@ne%number of nodes to interpolate
\newcount\top@mtwo \top@mtwo=\top@mone \advance\top@mtwo\m@ne%
% After copying S0 to S101, \top@ndx, \top@mone, \top@mtwo are top index, top index-1, top index-two---101, 100, 99 in the example
\make@mcoeff{\the\top@ndx}%
\def\noderoot{#1}%
\pstVerb{ /bcoeff #2\space 2 add array def  /ccoeff #2\space 2 add array def /xcoeff #2\space 2 add array def /ycoeff #2\space 2 add array def
%% initialize b, c, x ,y values
 1 1 \the\top@ndx\space %1 add
 {/j exch def %j is counter
 ccoeff j 0 put bcoeff j 0 put } for 
bcoeff 1 1 put bcoeff #2\space 1 put bcoeff #2\space 1 add  4 put 
ccoeff 1 0.25 put ccoeff #2\space 1 put }%
%end \pstVerb
\pnode(\noderoot 0){\noderoot\the\top@ndx}%copy node 0 to top end
\pstVerb{ xcoeff 1 1 put ycoeff 1 1.5 put }
\pnode(! \psGetNodeCenter{\noderoot 1} xcoeff  1 \noderoot 1.x 1.5 mul put ycoeff 1  \noderoot 1.y 1.5 mul put 1 1){temp@}% 1.5=6*0.25
\multido{\iA=2+1}{\the\top@mone}{\pnode(! ycoeff \iA\space xcoeff  \iA\space \psGetNodeCenter{#1\iA} #1\iA.x 6 mul put #1\iA.y 6 mul put 1 1){temp@}}% move x, y coords times 6 to arrays
\multido{\iA=2+1,\iB=1+1}{\the\top@mtwo}{\pstVerb{%
 ccoeff \iA\space ccoeff \iA\space get ccoeff \iB\space get sub mcoeff \iA\space get mul put
 xcoeff \iA\space xcoeff \iA\space get xcoeff \iB\space get sub mcoeff \iA\space get mul put
 ycoeff \iA\space ycoeff \iA\space get ycoeff \iB\space get sub mcoeff \iA\space get mul put
bcoeff \iA\space bcoeff \iA\space get bcoeff \iB\space get mcoeff \iB\space get mul sub put
bcoeff \the\top@ndx\space bcoeff \the\top@ndx\space get bcoeff \iB\space get ccoeff \iB\space get mul sub put
xcoeff \the\top@ndx\space xcoeff \the\top@ndx\space get xcoeff \iB\space get bcoeff \iB\space get mul sub put
ycoeff \the\top@ndx\space ycoeff \the\top@ndx\space get ycoeff \iB\space get bcoeff \iB\space get mul sub put 
}% end \pstVerb
}% end \multido
\pstVerb{%
 bcoeff \the\top@ndx\space bcoeff \the\top@ndx\space get bcoeff \the\top@mone\space get 
ccoeff \the\top@mone\space get mul sub put %b(n)=b(n)-b(n-1)*c(n-1)
xcoeff \the\top@ndx\space xcoeff \the\top@ndx\space get xcoeff \the\top@mone\space get bcoeff \the\top@mone\space get mul sub bcoeff \the\top@ndx\space get div put %x(n)=(x(n)-x(n-1)*b(n-1))/b(n)
ycoeff \the\top@ndx\space ycoeff \the\top@ndx\space get ycoeff \the\top@mone\space get bcoeff \the\top@mone\space get mul sub bcoeff \the\top@ndx\space get div put %y(n)=(y(n)-y(n-1)*b(n-1))/b(n)
xcoeff \the\top@mone\space xcoeff \the\top@mone\space get xcoeff \the\top@ndx\space get ccoeff \the\top@mone\space get mul sub put %x(n-1)=x(n-1)-c(n-1)*x(n)
ycoeff \the\top@mone\space ycoeff \the\top@mone\space get ycoeff \the\top@ndx\space get ccoeff \the\top@mone\space get mul sub put %y(n-1)=y(n-1)-c(n-1)*y(n)
}% end pstVerb
\mmultido{\iA=\the\top@mone+-1,\iB=\the\top@ndx+-1}{\the\top@mtwo}{%
\pstVerb{ xcoeff \iA\space xcoeff \iA\space get mcoeff \iA\space get xcoeff \iB\space get mul sub ccoeff \iA\space get xcoeff \the\top@ndx\space get mul sub put % x(k)=x(k)-m(k)*x(k+1)-c(k)*x(n)
ycoeff \iA\space ycoeff \iA\space get mcoeff \iA\space get ycoeff \iB\space get mul sub ccoeff \iA\space get ycoeff \the\top@ndx\space get mul sub put 
}}%y(k)=...
\multido{\iA=1+1}{\the\top@ndx}{\pnode(! xcoeff \iA\space get ycoeff \iA\space get )%
{\noderoot B\iA}}%
\pnode(\noderoot B\the\top@ndx){\noderoot B0}%
\ignorespaces}%
\catcode`\@=\TheAtCode\relax
\endinput