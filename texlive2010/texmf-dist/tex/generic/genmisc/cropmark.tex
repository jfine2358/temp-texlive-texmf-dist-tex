%% Save file as: CROPMARK.TEX                   Source: FILESERV@SHSU.BITNET
% Author: Peter Ungar <LAXA@ACFcluster.NYU.EDU>
% 
% CROPMARK MACROS FOR PLAIN TEX. Feb. 10, 1992 01:31 Peter Ungar 914 723 7187
% Cropmarks are markings on pages intended to be phototypeset. They mark the
% corners of the intended book page.
% THE DIMENSION SPECIFICATIONS (\paperheight,\paperwidth, \topmargin,
% \outsidemargin, \hsize, \vsize) SHOULD COME BEFORE THE MACROS. OTHERWISE
% THE CROPMARKS GO INTO ODD LOCATIONS. THESE QUANTITIES ARE ASSIGNED VALUES
% IN THIS FILE TO MAKE THE EXAMPLE WORK. REPLACE THE GIVEN VALUES BY THE
% ONES YOU NEED BEFORE INSTALLING THE MACROS. HSIZE IS THE LENGTH OF THE LINES
% AND VSIZE IS THE HEIGHT OF THE PRINTED PAGE NOT COUNTING HEADER OR FOOTER.
% \lefttopcropmark should be inserted at the beginning of the
% headline. It is a \llap and should not take space away from any other item
% in the headline. \righttopcropmark should be at the end of the headline.
% If there is no \hfil in the headline, use one to put the cropmark to
% the end of the headline. It is a \rlap and in theory it should not take
% space away
% from other items. Do not to put a space in front of \righttopcropmark be-
% cause the space will be typeset unless it is preceded by a control sequence.
% The bottom cropmarks should be inserted in the beginning and end of the
% \footline. Be sure to fill space in the middle with \hfil.
% Remember that the \footline definition has to come BEFORE the end of the
% first page.
% The values of \voffset and \hoffset can be adjusted to try to push the
% entire page including the cutmarks into the area your printer can print.
% The quantities you have to input are
% \paperheight, \paperwidth, \topmargin (measured from the top of the headline
% which is assumed to be 10-point size, and \outsidemargin which will be the
% right margin of odd-numbered pages and the left margin of even numbered
% pages. The other quantities which are needed are computed by the macros.
% The computations two empirical corrections.
% If the dimensions on your printout are slightly off, try over or under-
% stating the true dimensions of the above quantities.
% The macros ought to work for all values of \hsize and \vsize but if these
% are too large the cropmarks may not fit the area most printers can print.
% \nocropmarks replaces the cropmark macros by \relax (blanks). The cropmarks
% are needed only for phototypesetting and we may not want to see them on
% printout prepared for other purposes.
% TEX WILL NOT LET YOU KNOW IF YOU DID NOT ASSIGN A VALUE TO A DIMENSION.
% THAT MAY BE THE REASON IF YOU GET BIZARRE RESULTS. THE VALUES MUST COME
% BEFORE THE MACROS.

%\magnification 1042 %% Needed to get exact dimensions on the MacPlus screen.

\voffset=20mm\hoffset=15mm % Needed to bring cropmarks within printable area.
\overfullrule=0mm  % To prevent overfullrules being added to cropmarks
\newdimen\paperheight
\newdimen\paperwidth
\newdimen\topmargin
\newdimen\outsidemargin
\newdimen\insidemargin
\newdimen\leftmargin
\newdimen\rightmargin
\newdimen\topliftadjustment
\newdimen\bottomliftadjustment
\newdimen\topcropmarklift
\newdimen\bottomcropmarklift
\newdimen\leftsideskip
\newdimen\rightsideskip
\newdimen\bclifta
\newdimen\bcliftb

\vsize=182mm
\hsize=112mm
\paperheight=227.4mm     % These six quantities specify the margins.
\paperwidth=152.3mm      % The ones introduced below are either
\topmargin=14mm          % computed from these by the macros or they
\outsidemargin=25mm      % are auxiliary.

\topliftadjustment=2.6mm
\bottomliftadjustment=15.5mm
\insidemargin=\paperwidth
\advance\insidemargin by -\outsidemargin
\advance\insidemargin by -\hsize

\def\leftorright{\ifodd\pageno
\leftmargin=\insidemargin\rightmargin=\outsidemargin\else
\leftmargin=\outsidemargin\rightmargin=\insidemargin\fi
\leftsideskip=\leftmargin
\advance\leftsideskip by -4.05mm
\rightsideskip=\rightmargin
\advance\rightsideskip by -4.05mm}

\def\lefttopcropmark{{\parindent=0mm\llap{\vtop{
\hsize=\leftmargin
\vskip-\topliftadjustment
\vskip-\topmargin
\vrule height 0.05mm depth 4mm width 0.05mm
\vrule height 0.05mm depth0mm width 4mm
\hfill}}}}
\def\righttopcropmark{{\parindent=0mm\rlap{\vtop{\hsize=\rightmargin
\vskip-\topliftadjustment
\vskip-\topmargin
\hfill
\vrule height 0.05mm depth0mm width 4mm
\vrule height 0.05mm depth 4mm width 0.05mm
}}}}

\bottomcropmarklift=-\paperheight
\advance\bottomcropmarklift by\topmargin
\advance\bottomcropmarklift by\vsize
\advance\bottomcropmarklift by\bottomliftadjustment

\bclifta=\bottomcropmarklift
\advance\bclifta by 4mm
\bcliftb=\bottomcropmarklift
\advance\bcliftb by -0.05mm

\def\leftbottomcropmark{{\parindent=0mm\llap{
\hbox{\vrule height\bclifta depth -\bottomcropmarklift width 0.05mm
\vrule height \bottomcropmarklift depth -\bcliftb width 4mm
\hskip\leftsideskip}}}}

\def\rightbottomcropmark{{\parindent=0mm\rlap{\hbox{\hskip\rightsideskip
\vrule height \bottomcropmarklift depth -\bcliftb width 4mm
\vrule height\bclifta depth -\bottomcropmarklift width 0.05mm}}}}


\def\nocropmarks{\def\lefttopcropmark{\relax}  % THIS TURNS OFF CROPMARKS
\def\righttopcropmark{\relax}
\def\leftbottomcropmark{\relax}
\def\rightbottomcropmark{\relax}}
\endinput

% END OF THE MACROS.  AN EXAMPLE FOLLOWS. COMMENT OUT \endinput TO
% TYPESET IT.

\headline={\leftorright\lefttopcropmark\ifodd\pageno
\hfil HEADLINE\hfil\folio\else
\folio\hfil HEADLINE\hfil \fi
\righttopcropmark}
\footline={\leftorright\leftbottomcropmark
\hfil
\rightbottomcropmark}

\noindent Topliftadjustment=2.6mm bottomliftadjustment=15.5mm
\hfill H

\noindent paperheight=227.4mm  paperwidth=152.3mm

\noindent topmargin=14mm outsidemargin=25mm

\noindent vsize=182mm hsize=112mm

NOTE THAT THE DIMENSION SPECIFICATIONS MUST COME BEFORE THE MACROS.
\vfill
\noindent\line{Here is the bottom left corner. \hfil Bottom right is H}
\eject
\noindent\line{Here is the top left corner.\hfil Top right is H}
\vfill
\noindent\line{Here is the bottom left corner. \hfil Bottom right is H}
\eject
\end
