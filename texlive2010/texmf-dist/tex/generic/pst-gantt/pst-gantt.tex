%%
%% This is file `pst-gantt.tex',
%% 
%% IMPORTANT NOTICE:
%% 
%% Package `pst-gantt.tex'
%%
%% Denis Girou, Herbert Voss
%%
%% This program can be redistributed and/or modified under the terms
%% of the LaTeX Project Public License Distributed from CTAN archives
%% in directory macros/latex/base/lppl.txt.
%%
%% DESCRIPTION:
%%   `pst-gantt' is a PSTricks package to draw GANTT diagrams and timelines
%%
\csname PSTganttLoaded\endcsname
\let\PSTganttLoaded\endinput
% Require PSTricks, pst-node, pstcol, pst-plot, pst-3d packages
\ifx\PSTricksLoaded\endinput\else\input pstricks.tex\fi
\ifx\PSTnodesLoaded\endinput\else\input pst-node.tex\fi
\ifx\PSTplotLoaded\endinput\else \input pst-grad.tex\fi
\ifx\PSTXKeyLoaded\endinput\else \input pst-xkey.tex\fi
\ifx\MultidoLoaded\endinput\else \input multido.tex \fi
%
\edef\PstAtCode{\the\catcode`\@} 
\catcode`\@=11\relax
%
% interface to the `xkeyval' package
\pst@addfams{pst-gantt}
%
\def\fileversion{0.21}
\def\filedate{2010/02/23}
\message{`pst-gantt' v\fileversion, \filedate\space (Denis Girou and Herbert Voss)}
%
% "pspicture" environment or not?
\define@boolkey[psset]{pst-gantt}[Pst@]{PstPicture}[true]{}
\psset[pst-gantt]{PstPicture=true}
% Intervals to show?
\define@boolkey[psset]{pst-gantt}[Pst@]{ChartShowIntervals}[true]{}
\psset[pst-gantt]{ChartShowIntervals=false}
\define@key[psset]{pst-gantt}{ChartStartInterval}[1]{\def\psk@GanttStartInterval{#1}}
\psset[pst-gantt]{ChartStartInterval=1}
%
\define@boolkey[psset]{pst-gantt}[Pst@]{ChartModulo}[true]{}
\psset[pst-gantt]{ChartModulo=false}
\define@key[psset]{pst-gantt}{ChartModuloValue}[52]{\def\psk@ChartModuloValue{#1}}
\psset[pst-gantt]{ChartModuloValue=52}
%
% Style for the tasks
\define@key[psset]{pst-gantt}{TaskStyle}[TaskStyleDefault]{\def\psk@GanttTaskStyle{#1}}
\psset[pst-gantt]{TaskStyle=TaskStyleDefault}
% Name for unit interval
\define@key[psset]{pst-gantt}{ChartUnitIntervalName}[Week]{\def\psk@GanttChartUnitIntervalName{#1}}
% Name for basic unit
\define@key[psset]{pst-gantt}{ChartUnitBasicIntervalName}[Day]{%
  \def\psk@GanttChartUnitBasicIntervalName{#1}}
% Unit interval for the tasks (7 for a week, 30 for a month, etc.)
% Warning: define it before "TaskUnitType"!
\define@key[psset]{pst-gantt}{TaskUnitIntervalValue}[7]{%
  \pst@cntg=#1\relax
  \edef\psk@GanttTaskUnitIntervalValue{\the\pst@cntg}}
% Unit type for the tasks (UnitIntervalName or UnitBasicIntervalName)
\define@key[psset]{pst-gantt}{TaskUnitType}[Week]{%
  \def\psk@GanttTaskUnitValue{#1}%
% Validation of the parameter
  \ifx\psk@GanttTaskUnitValue\psk@GanttChartUnitIntervalName
    \edef\psk@GanttTaskUnitValue{\psk@GanttTaskUnitIntervalValue}%
  \else
    \ifx\psk@GanttTaskUnitValue\psk@GanttChartUnitBasicIntervalName
      \def\psk@GanttTaskUnitValue{1}%
    \else
      {\@pstrickserr{GanttTaskUnitType must be `\psk@GanttChartUnitIntervalName'
                   or `\psk@GanttChartUnitBasicIntervalName'
                  (and not `\psk@GanttTaskUnitValue')}\@eha}%
    \fi
  \fi}
% Outside label for the tasks
\define@key[psset]{pst-gantt}{TaskOutsideLabel}[]{\def\psk@GanttTaskOutsideLabel{#1}}
% Inside label for the tasks
\define@key[psset]{pst-gantt}{TaskInsideLabel}[]{\def\psk@GanttTaskInsideLabel{#1}}
% Maximum outside size label for the tasks (in unit TaskUnitType !)
\define@key[psset]{pst-gantt}{TaskOutsideLabelMaxSize}[0]{%
  \pst@cntg=#1\relax
  \edef\psk@GanttTaskOutsideLabelMaxSize{\the\pst@cntg}}
% Default values
% --------------
% pspicture environment, don't show intervals, default task style,
% unit for tasks is a week (so 7 days), no outside and inside labels
\psset[pst-gantt]{ChartUnitIntervalName=Week,ChartUnitBasicIntervalName=Day,
  TaskUnitIntervalValue=7,TaskUnitType=Week,
  TaskOutsideLabel=,TaskInsideLabel=,TaskOutsideLabelMaxSize=0}
\newpsstyle{TaskStyleDefault}{fillstyle=solid,fillcolor=yellow}
% The environment \PstGanttChart
% ------------------------------
\SpecialCoor
%
% Syntax: \PstGanttChart[parameters]{Number of tasks}{Number of days}
\def\PstGanttChart{\@ifnextchar[\PstGanttChart@i{\PstGanttChart@i[]}}
\def\PstGanttChart@i[#1]#2#3{%
\begingroup
\psset{unit=0.1,#1}    % Affectation of local parameters
%
\ifPst@PstPicture               % "pspicture" environment
  \pst@cnta=\psk@GanttTaskOutsideLabelMaxSize
  \multiply\pst@cnta\psk@GanttTaskUnitValue
  %
  \pst@cntb=#2
  \multiply\pst@cntb by 5
  \advance\pst@cntb\@ne
  %
  \pst@cntc=#3
  \multiply\pst@cntc\psk@GanttTaskUnitValue
  \advance\pst@cntc\tw@
  %
  \ifPst@ChartShowIntervals
    \pspicture(-\pst@cnta,-\pst@cntb)(\pst@cntc,2)
  \else
    \pspicture(-\pst@cnta,-\pst@cntb)(\pst@cntc,0)
  \fi
\fi
\psframe(0,-\pst@cntb)(\pst@cntc,0)
%
\ifPst@ChartShowIntervals
  % We will show the intervals
  \pst@cnta=#3
  \multiply\pst@cnta\psk@GanttTaskUnitValue
  \divide\pst@cnta\psk@GanttTaskUnitIntervalValue
  \advance\pst@cnta\@ne
  %
  \pst@cntb=#2
  \multiply\pst@cntb by 5
  \advance\pst@cntb\@ne
  %
  \pst@dima=\psk@GanttTaskUnitIntervalValue pt
  \divide\pst@dima\tw@
  \advance\pst@dima\@ne pt
  \pst@dimtonum{\pst@dima}{\pst@tempa}
  %
  \multido{\iStartInterval=\psk@GanttStartInterval+1,\iInterval=1+1,
           \iIntervalPos=1+\psk@GanttTaskUnitIntervalValue,
           \nIntervalPos=\pst@tempa+\psk@GanttTaskUnitIntervalValue.0}{\pst@cnta}{%
    \ifnum\iInterval=\pst@cnta
      \psline(\iIntervalPos,0)(\iIntervalPos,1.5)
      \psline[linestyle=dotted](\iIntervalPos,-\pst@cntb)(\iIntervalPos,0)
    \else
      \ifPst@ChartModulo
        \pst@mod{\iStartInterval}{\psk@ChartModuloValue}\pst@iInterval%
        \pst@cntm=\pst@iInterval \advance\pst@cntm by \@ne
        \rput(\nIntervalPos,1){\psk@GanttChartUnitIntervalName\space\the\pst@cntm}
      \else
        \rput(\nIntervalPos,1){\psk@GanttChartUnitIntervalName\space\iStartInterval}
      \fi
      \psline(\iIntervalPos,0)(\iIntervalPos,1.5)
      \psline[linestyle=dotted](\iIntervalPos,-\pst@cntb)(\iIntervalPos,0)
    \fi}
\fi}
\def\endPstGanttChart{%
\ifPst@PstPicture
  \endpspicture                 % End of "pspicture" environment
\fi
\endgroup}
% The macro \PstGanttTask
% -----------------------
\newcount\pst@GanttTaskCnt
\pst@GanttTaskCnt=\z@
% Syntax: \PstGanttTask[parameters]{Start}{Length}
\def\PstGanttTask{\@ifnextchar[\PstGanttTask@i{\PstGanttTask@i[]}}
\def\PstGanttTask@i[#1]#2#3{
\advance\pst@GanttTaskCnt\m@ne
\begingroup
\psset{#1}%            % Affectation of local parameters
% Frame
\pst@cnta=\psk@GanttTaskUnitValue
\multiply\pst@cnta by #2
\advance\pst@cnta\@ne
%
\pst@cntb=\psk@GanttTaskUnitValue
\multiply\pst@cntb by #3
\advance\pst@cntb\pst@cnta
%
\pst@cntc=\pst@GanttTaskCnt
\multiply\pst@cntc by 5
%
\pst@cntd=\pst@cntc
\advance\pst@cntd by 4
%
\psframe[style=\psk@GanttTaskStyle](\pst@cnta,\pst@cntc)(\pst@cntb,\pst@cntd)
% Inside label
\ifx\psk@GanttTaskInsideLabel\empty
\else
  \pst@dima=\pst@cnta pt
  \advance\pst@dima\pst@cntb pt
  \divide\pst@dima\tw@
  \pst@dimtonum{\pst@dima}{\pst@tempa}
  %
  \pst@dimb=\pst@cntc pt
  \advance\pst@dimb\pst@cntd pt
  \divide\pst@dimb\tw@
  \pst@dimtonum{\pst@dimb}{\pst@tempb}
  %
  \rput(\pst@tempa,\pst@tempb){\psk@GanttTaskInsideLabel}
\fi
% Outside label
\ifx\psk@GanttTaskOutsideLabel\empty
\else
  \pst@dimb=\pst@cntc pt
  \advance\pst@dimb\pst@cntd pt
  \divide\pst@dimb\tw@
  \pst@dimtonum{\pst@dimb}{\pst@tempb}
  \rput[r](-1.5,\pst@tempb){\psk@GanttTaskOutsideLabel}
\fi
\endgroup}
%
\catcode`\@=\PstAtCode\relax
%% END: pst-gantt.tex
\endinput

