%% $Id: pst-func.tex 918 2014-05-19 12:32:37Z herbert $
%%
%% This is file `pst-func.tex',
%%
%% IMPORTANT NOTICE:
%%
%% Package `pst-func.tex'
%%
%% Herbert Voss <hvoss@tug.org>
%%
%% This program can be redistributed and/or modified under the terms
%% of the LaTeX Project Public License Distributed from CTAN archives
%% in directory macros/latex/base/lppl.txt.
%%
%% DESCRIPTION:
%%   `pst-func' is a PSTricks package to plot special functions
%%
%% For a ChangeLog go the the end
%%
\csname PSTfuncLoaded\endcsname
\let\PSTfuncLoaded\endinput
% Requires some PSTricks packages 
\ifx\PSTricksLoaded\endinput\else   \input pstricks.tex\fi
\ifx\PSTnodesLoaded\endinput\else   \input pst-plot.tex\fi
\ifx\PSTmathLoaded\endinput \else   \input pst-math.tex\fi
\ifx\PSTtoolsLoaded\endinput \else  \input pst-tools.tex\fi
\ifx\PSTricksAddLoaded\endinput\else\input pstricks-add.tex\fi
\ifx\PSTXKeyLoaded\endinput\else    \input pst-xkey.tex \fi
%
\edef\PstAtCode{\the\catcode`\@} \catcode`\@=11\relax
% interface to the `xkeyval' package
\pst@addfams{pst-func}
%
\def\fileversion{0.80}
\def\filedate{2015/05/19}
\message{`PST-func' v\fileversion, \filedate\space (hv)}
%
\pstheader{pst-func.pro}
%\pstheader{pst-math.pro}% for GAMMALN
%
% Shortcuts ....

\def\ChebyshevT{ tx@FuncDict begin ChebyshevT end }
\def\ChebyshevU{ tx@FuncDict begin ChebyshevU end }

%
\define@key[psset]{pst-func}{epsilon}[1.e-08]{\def\psk@epsilon{#1 }}
\psset[pst-func]{epsilon=1.e-08}
\define@key[psset]{pst-func}{xShift}[0]{\def\psk@xShift{#1}}
\psset[pst-func]{xShift=0}
%
\define@key[psset]{pst-func}{cosCoeff}[0]{\def\psk@cosCoeff{#1}}
\define@key[psset]{pst-func}{sinCoeff}[1]{\def\psk@sinCoeff{#1}}
\psset[pst-func]{cosCoeff=0,sinCoeff=1} % coeff=a0 a1 a2 a3 ...
%
\def\psFourier{\@ifnextchar[{\psFourier@i}{\psFourier@i[]}}
\def\psFourier@i[#1]#2#3{{%
  \pst@killglue
  \psset{#1}
  \psplot[algebraic=false]{#2}{#3}{%
      /type (cos) def
      /Fourier {
        aload length /n exch def
        n -1 roll 2 div n 1 roll % a0/2
        n 1 sub -1 0 {
          /i exch def
          i x mul 180 mul 3.141592 div
          type (sin) eq {sin}{cos} ifelse
          mul n 1 roll
        } for
        n 1 sub -1 1 { pop add } for
      } def
      [\psk@cosCoeff] Fourier
      /type (sin) def
      [0 \psk@sinCoeff] Fourier add
    }%
}\ignorespaces}
%
\define@key[psset]{pst-func}{coeff}[0 1]{\def\psk@coeff{#1}}
\define@key[psset]{pst-func}{Derivation}[0]{\def\psk@Derivation{#1}}
\define@boolkey[psset]{pst-func}[Pst@]{markZeros}[true]{}
\define@key[psset]{pst-func}{epsZero}[0.1]{\def\psk@epsZero{#1}}
\define@key[psset]{pst-func}{dZero}[0.1]{\def\psk@dZero{#1}}
\define@key[psset]{pst-func}{zeroLineTo}[-1]{\def\psk@zeroLineTo{#1}}
\define@key[psset]{pst-func}{zeroLineColor}[black]{\pst@getcolor{#1}\psk@zeroLineColor}
\newdimen\psk@zeroLineWidth
\define@key[psset]{pst-func}{zeroLineWidth}[0.5\pslinewidth]{\pssetlength\psk@zeroLineWidth{#1}}
\define@key[psset]{pst-func}{zeroLineStyle}[dashed]{%
  \@ifundefined{psls@#1}%
    {\@pstrickserr{Line style `#1' not defined}\@eha}%
    {\edef\psk@zeroLineStyle{#1}}%
}
\psset[pst-func]{%
       coeff=0 1,      % coeff=a0 a1 a2 a3 ...
       Derivation=0, % 0 is the original function
       markZeros=false,% no dots for the zeros
       epsZero=0.1,    % the distance between two zero points   
       dZero=0.1,      % the distance of the x value for scanning the function
       zeroLineTo=-1,  % a line to the value of the lineTo's Derivation (-1= none)
       zeroLineStyle=dashed,%
       zeroLineWidth=0.5\pslinewidth,%
       zeroLineColor=black}%
%
\def\psGetZeros{\pst@object{psGetZeros}}
\def\psGetZeros@i(#1,#2)#3{%
  \pst@killglue
  \begin@SpecialObj
  \use@par
  \typeout{>>> GetZeros}%
  \addto@pscode{
    tx@FuncDict begin
    /xStart #1 def /xEnd #2 def
    /x xStart def
    /func { /x exch def \ifPst@algebraic (#3) AlgParser cvx exec \else #3 \fi } def
    #1 Steffensen
    \ifPst@markZeros \fi
    end
  }%
  \end@SpecialObj
\ignorespaces
}
\iffalse
/Steffensen {% the start value and the function must be on top of the stack
  /y0 exch def % the start value
  /Iter 0 def /MaxIter 200 def
  {
    y0 Function /F exch def
    F abs eps2 lt { exit } if
    y0 F sub /Phi exch def
    Phi Function /F2 exch def
    F2 abs eps2 le { exit }{
      Phi y0 sub dup mul Phi F2 sub 2 Phi mul sub y0 add Div /Diff exch def
      y0 Diff sub /y0 exch def
      Diff abs eps1 le { exit } if
    } ifelse
    /Iter Iter 1 add def
    Iter MaxIter gt { exit } if
  } loop
  y0 % the returned value ist the zero point
  0 \tx@ScreenCoor pstack
  3 0 360 arc gsave 0 0 1 setrgbcolor fill grestore 1 setlinewidth stroke
} def 
\fi
%
\def\psPolynomial{\pst@object{psPolynomial}}
\def\psPolynomial@i#1#2{%
  \pst@killglue%
  \begingroup%
  \use@par%
  \@nameuse{beginplot@\psplotstyle}%
  \gdef\psplot@init{}%
  \@nameuse{testqp@\psplotstyle}%
  \addto@pscode{%
    tx@FuncDict begin
    /coeff [ \psk@coeff ] def
    /x0 #1 def /x1 #2 def
    /dx x1 x0 sub \psk@plotpoints\space div def
    /Derivation \psk@Derivation\space def
    /x x0 def
    \ifPst@markZeros 
      5 dict begin % hold all local!
      gsave
      \pst@number\psk@zeroLineWidth SLW
      \pst@usecolor\psk@zeroLineColor
      \psk@epsZero\space \psk@dZero\space FindZeros
        aload length {  % zero array is on stack
        /xZero exch def
        xZero \pst@number\psxunit mul /xPixel exch def
        \psk@dotsize
        \@nameuse{psds@\psk@dotstyle}%
        xPixel 0 Dot 
        \psk@zeroLineTo\space 0 ge { % line to function \psk@lineTo
          xPixel 0 moveto
          xZero coeff \psk@zeroLineTo\space FuncValue 
          \pst@number\psyunit mul xPixel exch L
          \@nameuse{psls@\psk@zeroLineStyle}
        } if
      } repeat
      grestore
      end
    \fi
    /xy {
      x \psk@xShift\space sub coeff Derivation FuncValue \pst@number\psyunit mul 
      x \pst@number\psxunit mul exch
    } def
    xy moveto
  }%
  \if@pst%	lines and dots
    \psPolynomial@iii%
  \else%	curves
    \psPolynomial@ii%
  \fi%
  \endgroup
\ignorespaces}
%
\def\psPolynomial@ii{%
  \addto@pscode{%
     xy \@nameuse{beginqp@\psplotstyle}
    \psk@plotpoints {
      xy \@nameuse{doqp@\psplotstyle}
      /x x dx add def
    } repeat
    xy \@nameuse{doqp@\psplotstyle}
    end
  }%
  \@nameuse{endqp@\psplotstyle}%
}
\def\psPolynomial@iii{%    curves
  \addto@pscode{%
    mark
    /n 2 def
    \psk@plotpoints {
      xy 
      n 2 roll
      /n n 2 add def
      /x x dx add def
    } repeat
    /x x1 def
    xy
    n 2 roll
    end
  }%
  \@nameuse{endplot@\psplotstyle}%
}
%
% Bessel 2004-06-08
% Manuel Luque, Herbert Voss
% Look at the end for some more documentation about the algorithm
%
\define@key[psset]{pst-func}{constI}[1]{\def\psk@constI{#1 }}
\define@key[psset]{pst-func}{constII}[0]{\def\psk@constII{#1 }}
\psset{constI=1,constII=0}
%
\def\psBessel{\@ifnextchar[{\psBessel@i}{\psBessel@i[]}}
\def\psBessel@i[#1]#2#3#4{{%%%  #2 = n
  \pst@killglue
  \psset{plotpoints=500}%
  \psset{#1}%
  \parametricplot{#3}{#4}{%
    /J1 0 def
    /k { 57.29577951 mul } def
    /xBessel t k def
    0 0.1 180 {
      /tB exch k def
      /J1 J1 0.1 xBessel
        tB sin mul tB #2\space mul sub cos mul add def
    } for
    t J1 180 div \psk@constI mul \psk@constII add
  }%
}\ignorespaces}
%
%
\def\psModBessel{\@ifnextchar[{\psModBessel@i}{\psModBessel@i[]}}%% hv 20111021
\def\psModBessel@i[#1]#2#3{{%%%  #2 = n
  \pst@killglue%
  \psset{nue=0,#1}%
  \psplot{#2}{#3}[ /nue \psk@nue def /epsilon 1e-20 def ]{%
    /Sum 0 def
    /Iter 0 def
    {/Sum_Iter  
       x dup mul 4 div Iter exp        % nominator
       nue Iter add 1 add GAMMA Iter tx@AddMathFunc begin ! end mul  % denominator
       Div def
     Sum_Iter abs epsilon lt { exit } if
     /Sum Sum Sum_Iter add def
     /Iter Iter 1 add  def 
    } loop
    x 0.5 mul nue exp Sum mul
  }%
}\ignorespaces}
%
\define@key[psset]{pst-func}{sigma}[0.5]{\def\psk@sigma{#1 }}
\define@key[psset]{pst-func}{mue}[0]{\def\psk@mue{#1 }}
\define@key[psset]{pst-func}{nue}[1]{\def\psk@nue{#1 }}
\psset[pst-func]{sigma=0.5,mue=0,nue=1}
%
\def\psGauss{\@ifnextchar[{\psGauss@i}{\psGauss@i[]}}
\def\psGauss@i[#1]#2#3{{%
    \pst@killglue%
    \psset{plotpoints=200}%
    \psset{#1}%
    \psplot[algebraic=false]{#2}{#3}{%
        Euler x \psk@mue sub dup mul 2 div \psk@sigma dup mul div neg exp 
	1.0 \psk@sigma div TwoPi sqrt div mul%
    }%
}\ignorespaces}
%
\define@key[psset]{pst-func}{Simpson}[5]{\def\psk@Simpson{#1 }}
\psset[pst-func]{Simpson=5}
%
\def\psGaussI{\pst@object{psGaussI}}
\def\psGaussI@i#1#2{%
  \addbefore@par{plotpoints=200,plotstyle=line}
  \begin@OpenObj%
  \addto@pscode{
    /a #1  def
    /dx #2 #1 sub \psk@plotpoints\space div def
    /b a dx add def
    /scx { \pst@number\psxunit mul } def 
    /scy { \pst@number\psyunit mul } def
    tx@FuncDict begin 
    /C 1 \psk@sigma div TwoPi sqrt div def
    /SFunc {% x on Stack
      Euler exch \psk@mue\space sub dup mul 2 div \psk@sigma\space dup mul div neg exp C mul 
    } def 
    end
%    a scx 0 moveto
    a scx 0 \@nameuse{beginqp@\psplotstyle}
    \psk@plotpoints 1 sub {
      a b \psk@Simpson % a b M on Stack
      tx@FuncDict begin Simpson I end % y value on stack
      scy b scx exch \@nameuse{doqp@\psplotstyle} %lineto 
      /b b dx add def
    } repeat
%    stroke
  }%
  \end@OpenObj%
}
%
\def\psSi{\pst@object{psSi}}
\def\psSi@i#1#2{%
  \begin@OpenObj%
  \addto@pscode{
    /x #1  def
    /dx #2 #1 sub \psk@plotpoints\space div def
    /scx { \pst@number\psxunit mul } def
    /scy { \pst@number\psyunit mul } def
    x scx x tx@FuncDict begin Si end scy moveto
    \psk@plotpoints 1 sub {
      x dup scx exch tx@FuncDict begin Si end scy lineto
      /x x dx add def
    } repeat
    stroke
  }%
  \end@OpenObj%
}
\def\pssi{\pst@object{pssi}}
\def\pssi@i#1#2{%
  \begin@OpenObj%
  \addto@pscode{
    /x #1  def
    /dx #2 #1 sub \psk@plotpoints\space div def
    /scx { \pst@number\psxunit mul } def
    /scy { \pst@number\psyunit mul } def
    x scx x tx@FuncDict begin si end scy moveto
    \psk@plotpoints 1 sub {
      x dup scx exch tx@FuncDict begin si end scy lineto
      /x x dx add def
    } repeat
    stroke
  }%
  \end@OpenObj%
}
%
\def\psCi{\pst@object{psCi}}
\def\psCi@i#1#2{%
  \begin@OpenObj%
  \addto@pscode{
    /x #1  def
    /dx #2 #1 sub \psk@plotpoints\space div def
    /scx { \pst@number\psxunit mul } def
    /scy { \pst@number\psyunit mul } def
    x scx x tx@FuncDict begin Ci end scy moveto
    \psk@plotpoints 1 sub {
      x dup scx exch tx@FuncDict begin Ci end scy lineto
      /x x dx add def
    } repeat
    stroke
  }%
  \end@OpenObj%
}
\def\psci{\pst@object{psci}}
\def\psci@i#1#2{%
  \begin@OpenObj%
  \addto@pscode{
    /x #1  def
    /dx #2 #1 sub \psk@plotpoints\space div def
    /scx { \pst@number\psxunit mul } def
    /scy { \pst@number\psyunit mul } def
    x scx x tx@FuncDict begin ci end scy moveto
    \psk@plotpoints 1 sub {
      x dup scx exch tx@FuncDict begin ci end scy lineto
      /x x dx add def
    } repeat
    stroke
  }%
  \end@OpenObj%
}
%
\define@key[psset]{pst-func}{PSfont}[Times-Roman]{\def\psk@PSfont{/#1 }}
\define@key[psset]{pst-func}{valuewidth}[10]{\pst@getint{#1}\psk@valuewidth }
\define@key[psset]{pst-func}{fontscale}[10]{\pst@checknum{#1}\psk@fontscale }
\define@key[psset]{pst-func}{decimals}[-1]{\pst@getint{#1}\psk@decimals }
\psset[pst-func]{PSfont=Times-Roman,fontscale=10,valuewidth=10,decimals=-1}
%
\def\psPrintValue{\pst@object{psPrintValue}}
\def\psPrintValue@i#1{\expandafter\psPrintValue@ii#1,,\@nil}
\def\psPrintValue@ii#1,#2,#3\@nil{%  #1,#2 only for algebraic code
  \begin@SpecialObj
  \addto@pscode{  
     gsave \psk@PSfont findfont \psk@fontscale scalefont setfont 
     \ifPst@algebraic 
       /x #1 def 
       /Func (#2) tx@AlgToPs begin AlgToPs end cvx def 
       Func 
     \else #1 \fi
     \psk@decimals -1 gt { 10 \psk@decimals exp dup 3 1 roll mul cvi exch div } if
     \psk@valuewidth string cvs %/Output exch def % save output
     \ifPst@comma dot2comma \fi        % do we have to change dot to comma
     \psk@xShift\space 0 moveto  %Output 
     show grestore
  }%
  \end@SpecialObj%
}

\define@boolkey[psset]{pst-func}[Pst@]{round}[true]{}%
\define@boolkey[psset]{pst-func}[Pst@]{science}[true]{%
  \ifPst@science\def\psk@Scin{true }\else\def\psk@Scin{false }\fi}
\psset[pst-func]{science=false,round=false}
\def\psPrintValueNew{\pst@object{psPrintValueNew}}
\def\psPrintValueNew@i#1{\expandafter\psPrintValueNew@ii#1,,\@nil}
\def\psPrintValueNew@ii#1,#2,#3\@nil{%  #1,#2 only for algebraic code
  \begin@SpecialObj
  \addto@pscode{  %		thanks to Buddy Ledger
     /mfont { \psk@PSfont findfont \psk@fontscale scalefont setfont } bind def
     /mfontexp { \psk@PSfont findfont \psk@fontscale 1.2 div scalefont setfont } bind def
     /s1 { /Symbol findfont \psk@fontscale scalefont setfont } bind def
     \ifPst@algebraic 
        /x #1 def
        /Func (#2) tx@AlgToPs begin AlgToPs end cvx def 
        Func  
     \else #1 \fi
     /value ED
     \psk@Scin {
       value 0 ne { value log floor cvi /expon ED }{ /expon 0 def } ifelse
       value 10 expon exp div 
       \psk@decimals -1 gt { 10  \psk@decimals exp dup 3 1 roll mul 
         \ifPst@round round \else cvi \fi  exch div } if
       \psk@decimals 0 eq { cvi } if /numb ED
       expon \psk@valuewidth string cvs /expon exch def
       numb \psk@valuewidth string cvs 
       \ifPst@comma dot2comma \fi  % do we have to change dot to comma
       /Output exch def
       /txspc \psk@fontscale 4 div def
       \psk@xShift\space 0 moveto mfont Output show
       txspc 0 rmoveto s1 (\string\264) show
       txspc 0 rmoveto mfont (10) show
       txspc 2 div txspc 1.5 mul rmoveto mfontexp expon show }
     { value
       \psk@decimals -1 gt { 10 \psk@decimals exp dup 3 1 roll mul 
         \ifPst@round round \else cvi \fi exch div } if
       \psk@decimals 0 eq { cvi } if %inserted to handle decimals=0
       \psk@valuewidth string cvs 
       \ifPst@comma dot2comma \fi         % do we have to change dot to comma
       \psk@xShift\space 0 moveto mfont %Output 
       show
     } ifelse
  }%
  \end@SpecialObj%
}
%
% Integrals 2006-01-16
% Jose-Emilio Vila-Forcen, Herbert Voss
%
\def\psCumIntegral{\pst@object{psCumIntegral}}
\def\psCumIntegral@i#1#2#3{%
  \begin@OpenObj%
  \addto@pscode{
    /a #1  def
    /dx #2 #1 sub \psk@plotpoints\space div def
    /b a dx add def
    /scx { \pst@number\psxunit mul } def
    /scy { \pst@number\psyunit mul } def
    tx@FuncDict begin /SFunc { #3 } def end
    a scx 0 moveto
    \psk@plotpoints 1 sub {
      a b \psk@Simpson % a b M on Styack
      tx@FuncDict begin Simpson I end % y value on stack
      scy b scx exch lineto
      /b b dx add def
    } repeat
%    stroke
  }%
%  \psk@fillstyle%
%  \pst@stroke%
  \end@OpenObj%
}
%
\def\psIntegral{\pst@object{psIntegral}}
\def\psIntegral@i#1#2(#3,#4)#5{%
  \begin@OpenObj%
  \addto@pscode{
    /a #3  def
    /dx #4 #3 sub \psk@plotpoints\space div def
    /b #4 def
    /aa #1 def
    /dd #2 #1 sub \psk@plotpoints\space div def
    /t aa dd add def
    /scx { \pst@number\psxunit mul } def
    /scy { \pst@number\psyunit mul } def
    tx@FuncDict begin /SFunc { t #5 } def end
      a b \psk@Simpson % a b M on Stack
      tx@FuncDict begin Simpson I end % y value on stack
      scy t scx exch moveto
      /t t dd add def
    \psk@plotpoints 1 sub {
      a b \psk@Simpson % a b M on Stack
      tx@FuncDict begin Simpson I end % y value on stack
      scy t scx exch lineto
      /t t dd add def
    } repeat
%    stroke
  }%
%  \psk@fillstyle%
%  \pst@stroke%
  \end@OpenObj%
}
%
\def\psConv{\@ifnextchar[{\psConv@i}{\psConv@i[]}}
\def\psConv@i[#1]#2#3(#4,#5)#6#7{%
  \psIntegral[#1]{#2}{#3}(#4,#5){pop pop x #6\space x t neg add #7\space mul}%
}%
%
\define@boolkey[psset]{pst-func}[Pst@]{printValue}[true]{}
\define@key[psset]{pst-func}{barwidth}[1]{\def\psFunc@barwidth{#1 }}% a factor, not a dimen
\psset[pst-func]{printValue=false,barwidth=1}
%
\def\psBinomial{\pst@object{psBinomial}}
\def\psBinomial@i#1#2{\psBinomial@ii#1,,,\@nil{#2}}%
\def\psBinomial@ii#1,#2,#3,#4\@nil#5{%
  \def\pst@tempA{#2}%
  \ifx\pst@tempA\@empty
    \psBinomial@iii{0}{#1}{#1}{#5}%
  \else
    \def\pst@tempA{#3}%
    \ifx\pst@tempA\@empty\psBinomial@iii{#1}{#2}{#2}{#5}%
    \else\psBinomial@iii{#1}{#2}{#3}{#5}\fi
  \fi}%
\def\psBinomial@iii#1#2#3#4{%
  \begin@OpenObj%
  \addto@pscode{
    /scx { \pst@number\psxunit mul } def
    /scy { \pst@number\psyunit mul } def
    /m #1 def
    /n #2 def
    /N #3 def
    /p #4 def
    /dx \psFunc@barwidth 2 div def
    /q 1 p sub def
    /kOld dx neg m add def
    kOld scx 0 moveto   % starting point
    0 1 m 1 sub {
      /k exch def       % save loop variable
      k 0 eq
        { /Y q N exp def }
        { /Y Y N k sub 1 add mul k div p mul q div def }
      ifelse
    } for
    m 1 n {             % n-m+1 times
      /k exch def       % save loop variable
      k 0 eq
        { /Y q N exp def }
        { /Y Y N k sub 1 add mul k div p mul q div def }
      ifelse % recursive definition
      kOld scx Y scy L k dx add scx Y scy L
      \ifPst@markZeros k dx add scx 0 L kOld 1 add scx 0 L \fi
      \ifPst@printValue
        gsave \psk@PSfont findfont \psk@fontscale scalefont setfont
        Y \psk@valuewidth string cvs 
        \ifPst@comma dot2comma \fi
        k scx \psk@fontscale 2 div add
        Y scy \pst@number\pslabelsep add moveto
        90 rotate show grestore
      \fi
      /kOld kOld 1 add def
    } for
    \ifPst@markZeros\else k dx add scx 0 L \fi % last line down to x-axis
  }%
%  \psk@fillstyle%
%  \pst@stroke%
  \end@OpenObj%
}%
%
\def\psBinomialN{\pst@object{psBinomialN}}
\def\psBinomialN@i#1#2{%
  \leavevmode
  \pst@killglue
  \begingroup
  \use@par
  \init@pscode
  \def\cplotstyle{curve}%
  \ifx\psplotstyle\cplotstyle \@nameuse{beginplot@\psplotstyle} \fi%
  \addto@pscode{
    \ifx\psplotstyle\cplotstyle /Curve true def \else /Curve false def \fi
    /scx { \pst@number\psxunit mul } def 
    /scy { \pst@number\psyunit mul } def
    /N #1 def 
    /p #2 def				% probability
    /q 1 p sub def
    /E N p mul def
    /sigma E q mul sqrt def 		% variant
    /dx 1.0 sigma div 2 div def
    /xOld dx neg E sub sigma div def
    /xEnd xOld neg dx add scx def
    Curve 
      { /Coors [xOld dx sub scx 0] def }% saves the coordinates for curve
      { xOld scx 0 moveto }	% starting point
    ifelse 
   0 1 N {				% N times
      /k exch def			% save loop variable
      k 0 eq 
        { /Y q N exp def }
        { /Y Y N k sub 1 add mul k div p mul q div def }
      ifelse % recursive definition
      /x k E sub sigma div dx add def
      /y Y sigma mul def		% normalize
      Curve 
        { x dx sub scx y scy Coors aload length 2 add array astore /Coors exch def}
        { xOld scx y scy L x scx y scy L
          \ifPst@markZeros x scx 0 L \fi % 
        } ifelse
      \ifPst@printValue 
        gsave \psk@PSfont findfont \psk@fontscale scalefont setfont 
        y \psk@valuewidth string cvs %/Output exch def
        \ifPst@comma dot2comma \fi      % do we have to change dot to comma
        x dx sub scx \psk@fontscale 2 div add 
        y scy \pst@number\pslabelsep add moveto 
        90 rotate show grestore
      \fi
      /xOld x def
    } for
    Curve { [ xEnd 0 Coors aload pop } if % showpoints on top of the stack
  }%
  \ifx\psplotstyle\cplotstyle\@nameuse{endplot@\psplotstyle}\else%
    \psk@fillstyle%
    \pst@stroke%
  \fi%
  \use@pscode%
  \endgroup%
  \ignorespaces%
}
%
\def\psPoisson{\pst@object{psPoisson}}% with contributions from Gerry Coombes
\def\psPoisson@i#1#2{\psPoisson@ii#1,,\@nil{#2}}%
\def\psPoisson@ii#1,#2,#3\@nil#4{%
  \def\pst@tempA{#2}%
  \ifx\pst@tempA\@empty\psPoisson@iii{0}{#1}{#4}\else
    \psPoisson@iii{#1}{#2}{#4}\fi}%
\def\psPoisson@iii#1#2#3{%  M N lambda
  \begin@OpenObj%
  \addto@pscode{
    /scx { \pst@number\psxunit mul } def
    /scy { \pst@number\psyunit mul } def
    /M #1 def
    /N #2 def
    /lambda #3 def
    /elambda Euler #3 neg exp def  % e^-lambda
    /dx \psFunc@barwidth 2 div def
    /kOld dx neg M add def         % addition of M here
    kOld scx 0 moveto              % starting point
    /Y elambda def                 % start value
    0 1 M 1 sub {                  % skip over first M-1 rectangles
      /k exch def                  % whilst recursing probabilities
      k 0 eq { /Y elambda def }{ /Y Y lambda mul k div def } ifelse
    } for                          % nothing happens if M=0
    M 1 N {                        % N-M+1 times
      /k exch def                  % save loop variable
      k 0 eq { /Y elambda def }{ /Y Y lambda mul k div def } ifelse
      kOld scx Y scy L k dx add scx Y scy L
      \ifPst@markZeros k dx add scx 0 L \fi
      \ifPst@printValue
        gsave \psk@PSfont findfont \psk@fontscale scalefont setfont
        Y \psk@valuewidth string cvs %/Output exch def
        \ifPst@comma dot2comma \fi         % do we have to change dot to comma
        k scx \psk@fontscale 2 div add
        Y scy \pst@number\pslabelsep add moveto
        90 rotate show grestore
      \fi
      /kOld kOld 1 add def
      \ifPst@markZeros kOld scx 0 moveto \fi
    } for
    \ifPst@markZeros \else k dx add scx 0 L \fi % last line down to x-axis
  }%
% \psk@fillstyle
% \pst@stroke
  \end@OpenObj%
}
%
\define@key[psset]{pst-func}{alpha}[0.5]{\pst@checknum{#1}\psk@alpha }
\define@key[psset]{pst-func}{beta}[0.5]{\pst@checknum{#1}\psk@beta }
\psset[pst-func]{alpha=0.5,beta=0.5}
%
\def\psGammaDist{\pst@object{psGammaDist}}
\def\psGammaDist@i#1#2{%
  \ifdim#1pt<\z@ \psframebox*{\color{red}!!!\#1 must be greater than 0!!!}
  \else
    \addbefore@par{plotpoints=500,alpha=0.5,beta=0.5}%
    \begin@OpenObj
    \psplot[algebraic=false]{#1}{#2}{
      \psk@beta x mul \psk@alpha exp x div Euler \psk@beta neg x mul \psk@alpha GAMMALN sub exp mul}
    \end@OpenObj%
  \fi%
  \ignorespaces%
}
%
\def\psBetaDist{\pst@object{psBetaDist}}
\def\psBetaDist@i#1#2{%
  \ifdim#1pt<\z@ \psframebox*{\color{red}!!!\#1 must be greater than 0!!!}
  \else
    \addbefore@par{plotpoints=200,alpha=1,beta=1}%
    \begin@OpenObj
    \psplot[algebraic=false]{#1}{#2}{ 
      \psk@beta \psk@alpha add GAMMA
      \psk@beta GAMMA \psk@alpha GAMMA mul div 
      1 x sub \psk@beta 1.0 sub exp mul
      x \psk@alpha 1.0 sub exp mul }
    \end@OpenObj%
  \fi%
  \ignorespaces%
}
%
\def\psChiIIDist{\pst@object{psChiIIDist}}
\def\psChiIIDist@i#1#2{%
  \addbefore@par{plotpoints=500,nue=1}%
  \begin@OpenObj
%  \ifdim\psk@nue pt<\z@ \psframebox*{\color{red}!!!nue must be greater than 0!!!}
%  \else
     \psplot[algebraic=false]{#1}{#2}{%
      x 2 div \psk@nue 2 div exp x div Euler -0.5 x mul \psk@nue 2 div GAMMALN sub exp mul }%
%  \fi%
  \end@OpenObj%
  \ignorespaces%
}
%
\def\psTDist{\pst@object{psTDist}}
\def\psTDist@i#1#2{%
  \leavevmode
  \pst@killglue
  \begingroup
  \addbefore@par{plotpoints=500}%
  \use@par
  \ifdim\psk@nue pt<\z@ \psframebox*{\color{red}!!!nue must be greater than 0!!!}
  \else
     \psplot[algebraic=false]{#1}{#2}{
      1 x 2 exp \psk@nue div 1 add \psk@nue 1 add 2 div exp div
      \psk@nue Pi mul sqrt div
      Euler \psk@nue 1 add 2 div GAMMALN \psk@nue 2 div GAMMALN sub exp mul
     }%
  \fi%
  \endgroup%
  \ignorespaces%
}
%
\def\psFDist{\pst@object{psFDist}}
\def\psFDist@i#1#2{%
  \ifdim#1pt<\z@ \psframebox*{\color{red}!!!\#1 must be greater than 0!!!}
  \else
    \leavevmode
    \pst@killglue
    \begingroup
    \addbefore@par{plotpoints=500,mue=1}%
    \use@par
     \psplot[algebraic=false]{#1}{#2}{
      x \psk@mue mul \psk@nue div dup \psk@mue 2 div exp x div
      exch 1 add \psk@mue \psk@nue add 2 div exp div
      Euler \psk@mue \psk@nue add 2 div GAMMALN
      \psk@mue 2 div GAMMALN sub \psk@nue 2 div GAMMALN sub exp mul
     }%
    \endgroup%
  \fi%
  \ignorespaces%
}
%
\define@key[psset]{pst-func}{m}[0]{\def\psk@cauchy@m{#1 }}
\define@key[psset]{pst-func}{b}[1]{\def\psk@cauchy@b{#1 }}
\psset[pst-func]{m=0,b=1}
%
\def\psCauchy{\pst@object{psCauchy}}
\def\psCauchy@i#1#2{{%
    \pst@killglue%
    \addbefore@par{plotpoints=200}%
    \use@par%
    \psplot[algebraic=false]{#1}{#2}{
      \psk@cauchy@b dup dup mul x \psk@cauchy@m sub dup mul add div Pi div
    }%
}\ignorespaces}
%
\def\psCauchyI{\pst@object{psCauchyI}}
\def\psCauchyI@i#1#2{{%
    \pst@killglue%
    \addbefore@par{plotpoints=200}%
    \use@par%
    \psplot[algebraic=false]{#1}{#2}{
      x \psk@cauchy@m sub \psk@cauchy@b div ATAN1 DegtoRad Pi div 0.5 add
    }%
}\ignorespaces}
%
\def\psWeibull{\pst@object{psWeibull}}
\def\psWeibull@i#1#2{%
  \addbefore@par{plotpoints=500,alpha=1,beta=1}%
  \begin@OpenObj
  \def\pst@tempA{#1}%
  \ifdim#1pt<\z@ \psline(#1,0)(0,0)\def\pst@tempA{0}\fi
  \psplot[algebraic=false]{\pst@tempA}{#2}{
    \psk@alpha \psk@beta \psk@alpha neg exp mul % alpha*beta^(-alpha)
    x \psk@alpha 1 sub exp                      % x^(alpha-1) 
    mul
    Euler x \psk@beta div \psk@alpha exp neg exp % e^(-(x/beta)^alpha))
    mul }
  \end@OpenObj%
  \ignorespaces%
}
\def\psWeibullI{\pst@object{psWeibullI}}
\def\psWeibullI@i#1#2{%
  \addbefore@par{plotpoints=500,alpha=1,beta=1}%
  \begin@OpenObj
  \def\pst@tempA{#1}%
  \ifdim#1pt<\z@ \psline(#1,0)(0,0)\def\pst@tempA{0}\fi
  \psplot[algebraic=false]{\pst@tempA}{#2}{
    1
    Euler x \psk@beta div \psk@alpha exp neg exp % e^(-(x/beta)^alpha))
    sub
  }%
  \end@OpenObj%
  \ignorespaces%
}
%
\define@key[psset]{pst-func}{pd}[0.22]{\pst@checknum{#1}\psk@probability }
\define@key[psset]{pst-func}{R2}[0.11]{\pst@checknum{#1}\psk@portfolio }
\psset[pst-func]{pd=0.22,R2=0.11}
%
\def\psVasicek{\pst@object{psVasicek}}
\def\psVasicek@i#1#2{%
  \addbefore@par{plotpoints=500}%
  \begin@OpenObj
  \psplot{#1}{#2}[/pd \psk@probability\space def /R2 \psk@portfolio\space def ]{x tx@FuncDict begin vasicek end}
  \end@OpenObj%
  \ignorespaces%
}
\define@boolkey[psset]{pst-func}[Pst@]{Gini}[true]{}
\psset[pst-func]{Gini=false}
%
\def\psLorenz{\pst@object{psLorenz}}
\def\psLorenz@i#1{{%
%  \readdata{\L@Data}{#1}%
  \if@star\addto@par{fillstyle=solid,fillcolor=\pslinecolor}\fi
  \use@par%
\iffalse
  \def\Lorenz@code{
  /D {} def
  [ \L@Data\space counttomark dup 
    1 sub /m ED 2 div cvi /n ED % m=0..n-1    n=number of pairs
  ] /xyValues ED
  /Xval [] def  /Yval [] def /Xmax 0 def 
  /Xsum 0 def /Ysum 0 def /XYsum 0 def
  xyValues aload pop             	% [ x y x y x y ... ]
  n { 2 copy mul XYsum add /XYsum ED
      dup 
      Yval aload length 1 add array astore /Yval ED 
      Ysum add /Ysum ED
      dup
      Xval aload length 1 add array astore /Xval ED 
      dup Xsum add /Xsum ED
      dup Xmax gt { /Xmax ED }{ pop } ifelse
  } repeat
  Xval bubblesort /Xval ED
  Yval bubblesort /Yval ED
  Xval { Xmax div } forall n array astore /XvalRelMax ED 
  Xval { Xsum div } forall n array astore /XvalRel ED 
  Yval { Ysum div } forall n array astore /YvalRel ED 
  0 1 n 1 sub { 
    cvi /Index ED
    Xval Index get 
    Yval Index get
    mul } for
  n array astore /XmulY ED
  XmulY aload length 1 sub { add } repeat 
  /XmulYsum ED
  XmulY { XmulYsum div } forall 
  n array astore /XmulYdivXmulYsum ED
  /X [0] def
  /Y [0] def
  /Xsum 0 def /Ysum 0 def
  0 1 n 1 sub {
    /Index ED 
%    XvalRel Index get Xsum add /Xsum ED
%    X aload length 1 add Xsum exch array astore /X ED
    X aload length 1 add XvalRelMax Index get exch array astore /X ED %%
    XmulYdivXmulYsum Index get Ysum add /Ysum ED
    Y aload length 1 add Ysum exch array astore /Y ED
  } for
  \ifPst@Gini
  0 % start value for Gini
  0 1 X length 2 sub { 
    /Index ED
    Y Index get Y Index 1 add get add 2 div % yHeight=(y0+y1)/2
    X Index 1 add get X Index get sub abs   % xWidth=x1-x0 
    mul % x*y
    add
  } for
  2 mul 1 sub neg % triangle area divided by the area under the polygon
  \psk@PSfont findfont \psk@fontscale scalefont setfont 
  \psk@decimals -1 gt { 10 \psk@decimals exp dup 3 1 roll mul cvi exch div } if
  \psk@valuewidth string cvs %/Output exch def % save output
  \ifPst@comma dot2comma \fi      % do we have to change dot to comma
  /Output ED
  \psk@xShift\space -30 moveto (Gini: ) show 
  Output show 
  \fi
  0 1 n { dup X exch get exch Y exch get } for 
  \if@star 1 0 0 0 \fi                   % add values for the closed curve
  }% filling the area under the curve.
\fi
%%%%%%%%%%%%%%%%%%%%%%5
  \def\Lorenz@code{
  [ #1 ] dup length /n ED
  bubblesort /Yval ED 
  [ 1 1 n { } for ] /Xval ED
  /Xsum n dup 1 add mul 2 div cvi def 
  /Ysum 0 def /XYsum 0 def
  0 Yval { add } forall /Ysum ED
  Xval { n div } forall n array astore /XvalRelMax ED 
  Xval { Xsum div } forall n array astore /XvalRel ED 
  Yval { Ysum div } forall n array astore /YvalRel ED 
  0 1 n 1 sub { 
    /Index ED
    Xval Index get 
    Yval Index get
    mul } for
  n array astore /XmulY ED
  XmulY aload length 1 sub { add } repeat 
  /XmulYsum ED
  XmulY { XmulYsum div } forall 
  n array astore /XmulYdivXmulYsum ED
  /X [0] def
  /Y [0] def
  /Xsum 0 def /Ysum 0 def
  0 1 n 1 sub {
    /Index ED 
%    XvalRel Index get Xsum add /Xsum ED
%    X aload length 1 add Xsum exch array astore /X ED
    X aload length 1 add XvalRelMax Index get exch array astore /X ED %%
    XmulYdivXmulYsum Index get Ysum add /Ysum ED
    Y aload length 1 add Ysum exch array astore /Y ED
  } for
  \ifPst@Gini
  0 % start value for Gini
  0 1 X length 2 sub { 
    /Index ED
    Y Index get Y Index 1 add get add 2 div % yHeight=(y0+y1)/2
    X Index 1 add get X Index get sub abs   % xWidth=x1-x0 
    mul % x*y
    add
  } for
  2 mul 1 sub neg % triangle area divided by the area under the polygon
  \psk@PSfont findfont \psk@fontscale scalefont setfont 
  \psk@decimals -1 gt { 10 \psk@decimals exp dup 3 1 roll mul cvi exch div } if
  \psk@valuewidth string cvs %/Output exch def % save output
  \ifPst@comma dot2comma \fi      % do we have to change dot to comma
  /Output ED
  \psk@xShift\space -30 moveto (Gini: ) show 
  Output show 
  \fi
  0 1 n { dup X exch get exch Y exch get } for 
  \if@star 1 0 0 0 \fi                   % add values for the closed curve
  }% filling the area under the curve.
  \if@star\listplot*{\Lorenz@code}\else\listplot{\Lorenz@code}%
%                                       \listplot[plotstyle=bezier,linecolor=red]{\Lorenz@code}
  \fi%
}\ignorespaces}
%
% Superellipese / Lamefunction
\define@key[psset]{pst-func}{radiusA}[1]{\pst@getlength{#1}\pst@radiusA}
\define@key[psset]{pst-func}{radiusB}[1]{\pst@getlength{#1}\pst@radiusB}
\psset[pst-func]{radiusA=1,radiusB=1}
%
\def\psLame{\pst@object{psLame}}
\def\psLame@i#1{%
  \leavevmode
  \pst@killglue
  \begingroup
  \addbefore@par{plotpoints=200}%
  \use@par
  \parametricplot{0}{360}{%
     t cos dup mul 1 #1\space div exp \pst@radiusA \pst@number\psxunit div mul 
     t 90 gt { t 270 lt { neg } if } if
     t sin dup mul 1 #1\space div exp \pst@radiusB \pst@number\psyunit div mul 
     t 180 gt { neg } if }
  \endgroup\ignorespaces}
%
\def\psWeierstrass{\pst@object{psWeierstrass}}
\def\psWeierstrass@i(#1,#2){\@ifnextchar[{\psWeierstrass@ii(#1,#2)}{\psWeierstrass@iii(#1,#2)}}
\def\psWeierstrass@ii(#1,#2)[#3]#4{% #1 xStart; #2 xEnd; #3 a; #4 b  %% the original definition
  \addbefore@par{plotpoints=500}%
  \begin@SpecialObj
  \psplot{#1}{#2}[/ps@a #3 def /ps@b #4 def]{
    /ps@n 0 def
    /ps@WS 0 def
    { 
      ps@a ps@n exp ps@b ps@n exp Pi mul x mul RadtoDeg cos mul 
      dup abs \psk@epsilon gt { ps@WS add /ps@WS ED }{ ps@WS add exit } ifelse   
      ps@n 1 add /ps@n ED
    } loop
  }%
  \end@SpecialObj
}
\def\psWeierstrass@iii(#1,#2)#3{% #1 xStart; #2 xEnd; #3 a
  \addbefore@par{plotpoints=500}%
  \begin@SpecialObj
  \psplot{#1}{#2}[/ps@a #3 def]{
    /ps@k 1 def
    /ps@WS 0 def
    { 
      ps@k ps@a exp Pi mul dup x mul RadtoDeg sin exch div 
      dup abs \psk@epsilon gt { ps@WS add /ps@WS ED }{ ps@WS add exit } ifelse   
      ps@k 1 add /ps@k ED
    } loop
  }%
  \end@SpecialObj
}
%
% For polar plots
%\define@boolkey[psset]{pst-func}[PstAdd@]{polarplot}[true]{}
%\psset[pst-func]{polarplot=false}
%
%\define@boolkey[psset]{pstricks-add}[Pst@]{GetFinalState}[true]{}
%\define@key[psset]{pstricks-add}{filename}{\def\psk@filename{#1}}%
%\define@boolkey[psset]{pstricks-add}[Pst@]{saveData}[true]{} % \ifPst@saveData
%\psset[pstricks-add]{GetFinalState=false,saveData=false,filename=PSTdata}
%
\define@key[psset]{pst-func}{stepFactor}[0.67]{\pst@checknum{#1}\psk@stepFactor }
\psset[pst-func]{stepFactor=0.67}
%
\def\psplotImp{\pst@object{psplotImp}}% 20060420
\def\psplotImp@i(#1,#2)(#3,#4){%
  \@ifnextchar[{\psplotImp@ii(#1,#2)(#3,#4)}{\psplotImp@ii(#1,#2)(#3,#4)[]}}
\def\psplotImp@ii(#1,#2)(#3,#4)[#5]#6{%
  \addbefore@par{filename=\jobname.data}%
  \begin@OpenObj%
  \addto@pscode{
    \ifPst@saveData /Pst@data (\psk@filename) (w) file def \fi
    /xMin #1 def
    /xMax #3 def
    /yMin #2 def
    /yMax #4 def
    #5 % additional PS code
    \ifPst@polarplot 
      /@PolarAlgPlot (#6) tx@addDict begin AlgParser end cvx def
      /Func {
        /phi y x atan def
        /r x y Pyth def 
        \ifPst@algebraic @PolarAlgPlot \else #6 \fi } def 
    \else
      /Func \ifPst@algebraic (#6) tx@addDict begin AlgParser end cvx \else { #6 } \fi  def
    \fi
    /xPixel xMax xMin sub \pst@number\psxunit mul round cvi def
    /yPixel yMax yMin sub \pst@number\psyunit mul round cvi def
    /dx xMax xMin sub xPixel div def
    /dy yMax yMin sub yPixel div def
    /setpixel { 
      dy div exch 
      dx div exch 
      \ifPst@saveData 
        2 copy
        \pst@number\psyunit div exch  \pst@number\psxunit div 
        20 string cvs Pst@data exch writestring 
        Pst@data (\space) writestring 
        20 string cvs Pst@data exch writestring 
%        Pst@data (\string\]) writestring
        Pst@data (\string\n) writestring 
      \fi
      \pst@number\pslinewidth 2 div 0 360 arc fill } bind def
%
    /VZ true def % suppose that F(x,y)>=0
    /x xMin def /y yMin def Func 0.0 lt { /VZ false def } if % erster Wert
    xMin dx \psk@stepFactor\space mul xMax {
      /x exch def
      \ifPst@saveData Pst@data ([\string\n) writestring \fi
      yMin dy \psk@stepFactor\space mul yMax {
	/y exch def
	Func 0 lt 
	    { VZ { x y setpixel /VZ false def} if }
	    { VZ {}{ x y setpixel /VZ true def } ifelse } ifelse
      } for
      \ifPst@saveData Pst@data (]\string\n) writestring \fi
    } for
%% the same for the other way round without saving the data
    /VZ true def % suppose that F(x,y)>=0
    /x xMin def /y yMin def Func 0.0 lt { /VZ false def } if % erster Wert
    yMin dy \psk@stepFactor\space mul yMax {
      /y exch def
      \ifPst@saveData Pst@data ([\string\n) writestring \fi
      xMin dx \psk@stepFactor\space mul xMax {
	/x exch def
	Func 0 lt 
	    { VZ { x y setpixel /VZ false def} if }
	    { VZ {}{ x y setpixel /VZ true def } ifelse } ifelse
      } for
      \ifPst@saveData Pst@data (]\string\n) writestring \fi
    } for
%
\iffalse
    /x xMin def /y yMin def Func 0.0 lt { /VZ false def } if % erster Wert
    yMin dy \psk@stepFactor\space mul yMax {
      /y exch def
      xMin dx \psk@stepFactor\space mul xMax {
	/x exch def
	Func 0 lt 
	    { VZ { x y setpixel /VZ false def} if }
	    { VZ {}{ x y setpixel /VZ true def } ifelse } ifelse
      } for
    } for
\fi
    \ifPst@saveData Pst@data closefile \fi
  }%
  \end@OpenObj%
}
%
\def\psVolume{\pst@object{psVolume}}% 2007-06-23
\def\psVolume@i(#1,#2)#3#4{%
  \leavevmode
  \pst@killglue
  \begingroup
  \use@par
  \psplot[fillstyle=none]{#1}{#2}{#4}%     original function
  \ifPst@algebraic
    \psplot[fillstyle=none]{#1}{#2}{-(#4)}
  \else 
    \psplot[fillstyle=none]{#1}{#2}{#4 neg}% mirrored at the x-axis
  \fi
  \multido{\iA=1+1}{#3}{%	 run it #3 times with increment \A
    \pscustom{%                  to get a closed filled ellipse
      \code{ %					 the PS code
        /dX #2 #1 sub #3 div def % delta x, the step
        /Start dX \iA\space 1 sub mul #1 add def % xStart
        /End Start dX add def                    % xEnd=xStart+dX
        /Height End Start add 2 div /x ED 
          \ifPst@algebraic (#4) AlgParser cvx exec \else #4 \fi def } % height=f(x)
         % x is the mean between Start+End
      \psellipticarc(!Start 0)(! Height 8 div Height){90}{270}
      % draw the first falf of the ellipse
      \rlineto(! dX 0)% draw a line in x-direction
      \psellipticarc(!End 0)(! Height 8 div Height){270}{90}
      % draw the other half of the ellipse
      \rlineto(!dX neg 0)}}% draw a line in negative x-direction
  \psset{fillstyle=none}
%  \psellipse(#2,0)(!#2 dup #1 sub #3 div 2 div sub /x ED #4 dup 
%	 8 div exch)% draw again the ellipse to get the borderline.
  \psellipse(#2,0)(!{ #2 dup #1 sub #3 div 2 div sub /x ED 
         \ifPst@algebraic (#4) AlgParser cvx exec \else #4 \fi dup
	 8 div exch})  % draw again the ellipse to get the borderline.
  \psset{plotstyle=line,linestyle=dashed,
	 plotpoints=40,dotstyle=*,dotsize=0.5pt}
  \psplot[fillstyle=none]{#1}{#2}{#4}
  \ifPst@algebraic
    \psplot[fillstyle=none]{#1}{#2}{-(#4)}
  \else 
    \psplot[fillstyle=none]{#1}{#2}{#4 neg}% mirrored at the x-axis
  \fi
  % draw again the curves to get the borderline
  \endgroup%
  \ignorespaces%
}
\def\txFunc@BezierCurve{ tx@FuncDict begin BezierCurve Points end }
\def\txFunc@BezierShowPoints{ tx@Dict begin /Points ED BezierShowPoints end }
\def\pst@BezierType{2 }	% the default
%
\def\psBezier#1{%		% allowed order is 1 ... 9
  \ifnum#1>0 \ifnum#1<10 \def\pst@BezierType{#1 }\fi\fi%
  \pst@object{psBezier}}
\def\psBezier@i{%
  \pst@getarrows{%
    \addbefore@par{plotpoints=200}%
    \begin@OpenObj
    \pst@getcoors[\psBezier@ii%
}}
\def\psBezier@ii{%
  \addto@pscode{%
    \psk@plotpoints	    	% step for Bezier T=0,0+epsilon,0+i*epsilon,...,1
    \pst@BezierType 		% type of the Bezier curve 2,3,4,... 
    \txFunc@BezierCurve 
    \ifshowpoints \txFunc@BezierShowPoints \else pop \fi 
  }%
  \end@OpenObj}
%
\def\tx@Bernstein{ tx@FuncDict begin Bernstein end }
\define@boolkey[psset]{pst-func}[Pst@]{envelope}[true]{}
\psset[pst-func]{envelope=false}
%
\def\psBernstein{\pst@object{psBernstein}}%  \psBernstein[options](t1,t2)(i,n)
\def\psBernstein@i(#1,#2){%
  \@ifnextchar({\psBernstein@ii(#1,#2)}{\psBernstein@ii(0,1)(#1,#2)}}
%  
\def\psBernstein@ii(#1,#2)(#3,#4){%	(tStart,tEnd)(i,n)
  \addbefore@par{plotpoints=200}%
  \begin@OpenObj
  \addto@pscode{%
    /ScreenCoor { \tx@ScreenCoor } def
    #1\space #2\space 
    1.0 \psk@plotpoints\space div	    	% step=1/plotpoints
    #3\space #4\space
%   on stack we have tStart tEnd epsilon i n
    \ifPst@envelope true \else false \fi
    \tx@Bernstein
  }%
  \end@OpenObj}
%
\def\psThomae{\pst@object{psThomae}}
\def\psThomae@i(#1,#2)#3{%
  \addbefore@par{dotsize=1pt}
  \begin@ClosedObj
  \addto@pscode{
    1 1 #3 {
      dup 
      /ipSave ED	% save loop value
      /ip ED		% dito
      1 1 #3 {		
        dup		
        /iqSave ED	% sabve loop value
        /iq ED		% dito
        { 
          iq 0 le { exit } if
          ip iq mod 
          /ip iq def
          /iq ED 
        } loop
        ip 1 eq { 
          /xVal ipSave iqSave div def
          xVal #1 ge { xVal #2 le {
            \psk@dotsize
            \@nameuse{psds@\psk@dotstyle}
            \pst@usecolor\pslinecolor xVal 1 iqSave div \tx@ScreenCoor            
            2 copy moveto Dot } if } if
        } if
      } for 
    } for
  }%
  \end@ClosedObj%
}
%
\def\psCplot{\def\pst@par{}\pst@object{psCplot}}
\def\psCplot@i#1#2#3#4{% start | end | complex variables | function
  \pst@killglue
  \begingroup
    \use@par
    \@nameuse{beginplot@\psplotstyle}%
    \addto@pscode{%
      \psplot@init
      /x #1 def
      /x1 #2 def
      /dx x1 x sub \psk@plotpoints div def
      #3
      /xy {
%        x 
        tx@FuncDict begin
        #4 aload pop \pst@number\psyunit mul exch \pst@number\psxunit mul exch
        end
      } def}%
    \gdef\psplot@init{}%
    \@pstfalse
    \@nameuse{testqp@\psplotstyle}%
    \if@pst
      \psplot@ii
    \else
      \psplot@iii
    \fi
  \endgroup
  \ignorespaces}
%
\define@boolkey[psset]{pst-func}[Pst@]{Newton}[true]{}
\define@boolkey[psset]{pst-func}[Pst@]{PrintCoord}[true]{}
\define@boolkey[psset]{pst-func}[Pst@]{onlyNode}[true]{}
\define@boolkey[psset]{pst-func}[Pst@]{onlyYVal}[true]{}
\define@boolkey[psset]{pst-func}[Pst@]{originV}[true]{}
\define@key[psset]{pst-func}{PointName}[]{\def\psk@PointName{#1}}
\define@key[psset]{pst-func}{ydecimals}[2]{\pst@getint{#1}\psk@ydecimals }
\psset[pst-func]{originV=false,onlyNode=false,ydecimals=2,
  PrintCoord=false,onlyYVal=false,Newton=false,PointName=I,
}
%--------------------------------------------------------------------
%------------- calculate the value of an intersectionpoint -----------
%---------------------------------------------------------------------
\def\psZero{\def\pst@par{}\pst@object{psZero}}
\def\psZero@i(#1,#2)#3{\@ifnextchar[{\psZero@ii(#1,#2)#3}{\psZero@ii(#1,#2){#3}[0]}}
\def\psZero@ii(#1,#2)#3[#4]#5{% 
% (#1,#2) Intervall für die Nullstelle, bzw #1 Startwert für Newton, #3 1. Funktion, #4 2. Funktion, #5 Knotenname
  \begingroup
  \pst@killglue
%  \addbefore@par{fontscale=40,PSfont=Times-Roman}%
  \use@par
  \pstVerb{
    /FunctionA 
      \ifPst@algebraic (#3) tx@AlgToPs begin AlgToPs end cvx \else {#3} \fi def 
    /FunctionB 
      \ifPst@algebraic (#4) tx@AlgToPs begin AlgToPs end cvx \else {#4} \fi def 
  \ifPst@Newton %   ---------------------------        Newton-Verfahren --- TS 05.2014
   /eps 1e-6 def  %                                         eps f\"{u}r (f(x+eps)-f(x))/eps
   /x0 #1 def     %                                         Startwert wird vorgegeben
   /Xend x0 def
    {
   /xe x0 eps add def %                                      xe=x+eps
   /x x0 def          %                                      x nun x0 setzen, damit f(x0) berechnet werden kann
   /func0 FunctionA FunctionB sub def %                       func0=F(x0)-FB(x0)
   func0 0 eq {exit} if %                                    wenn func0 = 0 -> exit
   /x xe def  %                                              x xe setzen
   /func1 FunctionA FunctionB sub def %                       func1=F(xe)-FB(xe)
   /Xend x0 func0 eps mul func1 func0 sub div sub def %      Xend=x0-(func0*eps)/(func1-func0)
   /x Xend def %                                             x wird nun Xend
   /funcend FunctionA FunctionB sub def %                     funcend=F(Xend)-FB(Xend)
   funcend abs 1e-6 le {exit} if %                           Wenn funcend < 1e-6 -> exit
   /x0 Xend def %                                            sonst x0 = Xend und
   } loop %                                                  die Schleife beginnt mit x0 = Xend
   /xM Xend def
  \else  % --------------------   Intervallhalbierungsverfahren von Manuel L.
   /Xinf #1 def %                                            linker Wert des Intervalls, in dem die Nullstelle gesucht wird
   /Xsup #2 def %                                            linker Wert des Intervalls, in dem die Nullstelle gesucht wird
   { /xM Xinf Xsup add 2 div def %                           Mittelwert -> Mitte des Intervalls ist xM2
   /x Xinf def  %                                            Variable x wird als linker Wert festgelegt
   /F_1 FunctionA FunctionB sub def %                         Definition der Differenzfunktion F_1 an der Stelle x=Xinf2
   /x xM def  %                                              Variable x wird als Mittelwert festgelegt
   /F_M FunctionA FunctionB sub def %                         Definition der Differenzfunktion F_M an der Stelle x=xM2
   F_M 0 eq {exit} if %                                      Ist F_M = 0 => fertig
   F_1 F_M mul 0 ge {/Xinf xM def} {/Xsup xM def} ifelse %   F_1 * F_M > 0 (F_1 und F_M haben gleiches VZ) => neuer linker x-Wert = xM2
   Xinf Xsup sub abs 1e-6 le {exit} if } loop %              Die Schleife endet, wenn die Differenz von linkem und rechtem Wert < 10^-6
  \fi
   /x xM def FunctionA /yM exch def %                               x wird Mittelwert und an dieser Stelle die Funktion berechnet und als yM2 definiert
  }%
  \pnode(! xM yM){#5}%
  \addto@pscode{
   \ifPst@onlyNode \else
   /dec \psk@decimals\space def   %                                                         Anzahl der Dezimahlen f\"{u}r x
   /ydec \psk@ydecimals\space def %                                                         Anzahl der Dezimalen f\"{u}r y
   /symb { /Symbol findfont \psk@fontscale\space scalefont setfont } bind def             % Symbolschrift f\"{u}r Klammer und senkrechten Strich
   /schrift { \psk@PSfont findfont \psk@fontscale scalefont setfont } bind def            % Schrift f\"{u}r die Ergebnisse
   /TiefSchrift { /Symbol findfont \psk@fontscale 1.6 div scalefont setfont } bind def    % Schrift f\"{u}r tiefergestellte Buchstaben
   /spdx \psk@fontscale 4 div def %                                                         Abstand, der mit der Schriftg\"{o}{\ss}e skaliert ist
   /spdy \psk@fontscale 15 div def %                                                        Abstand, der mit der Schriftg\"{o}{\ss}e skaliert ist
   /Wert { 10 dec exp mul round 10 dec exp div dec 0 eq {cvi 15 string cvs} {15 string cvs } ifelse % x-Wert runden und als string ablegen
   \ifPst@comma dot2comma \fi show } def %                                                     gegebenenfalls Komma statt Punkt und Ergebnis anzeigen
   /yWert { 10 ydec exp mul round 10 ydec exp div ydec 0 eq {cvi 15 string cvs} {15 string cvs } ifelse % y-Wert runden und als string ablegen
      \ifPst@comma dot2comma \fi show } def %                                                     gegebenenfalls Komma statt Punkt und Ergebnis anzeigen
   /FunctionA \ifPst@algebraic (#3) tx@AlgToPs begin AlgToPs end cvx \else  {#3} \fi def 
   /FunctionB \ifPst@algebraic (#4) tx@AlgToPs begin AlgToPs end cvx \else {#4} \fi def 
  \ifPst@Newton
   /eps 1e-6 def
   /x0 #1 def
   /Xend x0 def
   { /xe x0 eps add def
     /x x0 def
     /func0 FunctionA FunctionB sub def
     func0 0 eq {exit} if
     /x xe def
     /func1 FunctionA FunctionB sub def
     /Xend x0 func0 eps mul func1 func0 sub div sub def
     /x Xend def
     /funcend FunctionA FunctionB sub def
     funcend abs 1e-6 le {exit} if
     /x0 Xend def
   } loop
   /xM Xend def
  \else
   /Xinf #1 def   /Xsup #2 def
   { /xM Xinf Xsup add 2 div def
   /x Xinf def  /F_1 FunctionA FunctionB sub def
   /x Xsup def  /F_2 FunctionA FunctionB sub def
   /x xM def    /F_M FunctionA FunctionB sub def
   F_M 0 eq {exit} if
   F_1 F_M mul 0 ge {/Xinf xM def} {/Xsup xM def} ifelse
   Xinf Xsup sub abs 1e-6 le {exit} if } loop
  \fi
   /x xM def FunctionA /yM exch def
   \ifPst@originV 0 0 \else
     xM \psk@xShift\space add \pst@number\psxunit mul
     yM \psk@yShift\space add \pst@number\psyunit mul 
   \fi
   \ifPst@PrintCoord
     moveto schrift (\psk@PointName) show
     spdx neg spdy 2 mul neg rmoveto TiefSchrift (\psk@postString) show
     0 spdy 3 mul rmoveto symb (\string\050) show
     0 spdy neg rmoveto schrift xM Wert
     spdx 3 div spdy rmoveto symb (\string\174) show
     spdx 3 div spdy neg rmoveto yM yWert 0 spdy rmoveto symb (\string\051) show
   \else
     \ifPst@onlyYVal moveto schrift yM yWert \else moveto schrift xM Wert \fi
   \fi\fi
  }%
  \ifPst@markZeros \psdot(#5)\fi
\endgroup\use@pscode\ignorespaces
}%
%

\catcode`\@=\PstAtCode\relax
%
%% END: pst-func.tex
\endinput
%
