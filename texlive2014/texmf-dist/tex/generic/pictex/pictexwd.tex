% G"andertes PiCTeX.TEX mit weniger Dimensionsregistern %
% von Andreas Schrell, ohne Gew"ahr!!!!!!!
% Version 0.0 vom 31.03.1994
% FEHLERMELDUNGEN an Andreas.Schrell@RS.maus.de

% This is PiCTeXWD, Version 1.1   9/21/87
 
% CAVEAT: The PiCTeX manual often has a more lucid explanation
%   of any given topic than you will find in the internal documentation
%   of the macros.
 
% PiCTeX's commands can be classified into two groups: (1) public (or
%   external), and (2) private (or internal). The public macros are
%   discussed at length in the manual. The only discussion of the private
%   macros is the internal documentation. The private macros all have
%   names beginning with an exclamation point (!) of category code 11. 
%   Since in normal usage "!" has category code 12, these macros can't
%   be accessed or modified by the general user.
 
% The macros are organized into thematically related groups. For example,
%   the macros dealing with dots & dashes are all in the DASHPATTERN group.
%   The table below shows which macros are in which groups. The table
%   covers all public macros, and many (but not all) of PiCTeX's upper level
%   private macros. Following the table, the various groups are listed
%   in the order in which they appear in the table.
 
% *********************** TABLE OF GROUPS OF MACROS **********************
 
% HACKS:  Utility macros
%    \PiC
%    \PiCTeX
%    \placevalueinpts 
%    \!!loop 
%    \!cfor
%    \!copylist
%    \!ecfor
%    \!etfor
%    \!getnext
%    \!getnextvalueof
%    \!ifempty
%    \!ifnextchar
%    \!leftappend
%    \!listaddon 
%    \!loop
%    \!lop
%    \!mlap
%    \!not
%    \!removept
%    \!rightappend
%    \!tfor 
%    \!vmlap
%    \!wlet
 
% ALLOCATION:  Allocates registers
 
% AREAS: Deals with plot areas 
%    \axis
%    \grid 
%    \invisibleaxes
%    \normalgraphs 
%    \plotheading 
%    \setplotarea
%    \visibleaxes
 
% ARROWS:  Draws arrows
%    \arrow
%    \betweenarrows
 
% BARS:  Draws bars
%    \putbar 
%    \setbars
 
% BOXES:  Draws rectangles
%    \frame
%    \putrectangle 
%    \rectangle
%    \shaderectangleson
%    \shaderectanglesoff
 
% CURVES:  Upper level plot commands
%    \hshade 
%    \plot 
%    \sethistograms
%    \setlinear
%    \setquadratic
%    \vshade
 
% DASHPATTERNS:  Sets up dash patterns
%    \findlength 
%    \setdashes 
%    \setdashesnear
%    \setdashpattern
%    \setdots 
%    \setdotsnear 
%    \setsolid
%    \!dashingoff
%    \!dashingon
 
% DIVISION:  Does long division of dimension registers
%    \Divide 
%    \!divide
 
% ELLIPSES:  Draws ellipses and circles
%    \circulararc 
%    \ellipticalarc 
 
% RULES:  Draws rules, i.e., horizontal & vertical lines
%    \putrule 
%    \!putdashedhline
%    \!putdashedvline
%    \!puthline 
%    \!putsolidhline  
%    \!putsolidvline  
%    \!putvline
 
% LINEAR ARC:  Draws straight lines -- solid and dashed
%    \inboundscheckoff
%    \inboundscheckon
%    \!advancedashing 
%    \!drawlinearsegment
%    \!initinboundscheck
%    \!linearsolid
%    \!lineardashed
%    \!ljoin
%    \!plotifinbounds     
%    \!start 
 
% LOGTEN:  Log_10 function
%    \!logten
 
% PICTURES:  Basic setups for PiCtures; \put commands
%    \accountingoff
%    \accountingon
%    \beginpicture
%    \endpicture    
%    \endpicturesave 
%    \lines
%    \multiput
%    \put 
%    \setcoordinatemode
%    \setcoordinatesystem
%    \setdimensionmode
%    \stack 
%    \Lines
%    \Xdistance
%    \Ydistance
%    \!dimenput
%    \!ifcoordmode
%    \!ifdimenmode
%    \!setcoordmode
%    \!setdimenmode
%    \!setputobject
   
% PLOTTING:  Things to do with plotting
%    \dontsavelinesandcurves
%    \replot
%    \savelinesandcurves 
%    \setplotsymbol
%    \writesavefile 
%    \!plot
 
% PYTHAGORAS:  Euclidean distance function
%    \placehypotenuse 
%    \!Pythag
 
% QUADRATIC ARC:  Draws a quadratic arc
%    \!qjoin 
 
% ROTATIONS:  Handles rotations
%    \startrotation 
%    \stoprotation
%    \!rotateaboutpivot
%    \!rotateonly
 
% SHADING:  Handles shading
%    \setshadegrid 
%    \setshadesymbol
%    \!lshade
%    \!qshade
%    \!starthshade
%    \!startvshade  
 
% TICKS:  Draws ticks on graphs
%    \gridlines
%    \loggedticks
%    \nogridlines
%    \ticksin
%    \ticksout
%    \unloggesticks
 
% ***************** END OF TABLE OF GROUPS OF MACROS ********************


\catcode`!=11 %  ***** THIS MUST NEVER BE OMITTED
% *******************************
% *** HACKS  (Utility macros) ***
% *******************************
 
% ** User commands
% **   \PiC{P\kern-.12em\lower.5ex\hbox{I}\kern-.075emC}
% **   \PiCTeX{\PiC\kern-.11em\TeX}
% **   \placevalueinpts of <DIMENSION REGISTER> in {CONTROL SEQUENCE}
  
% ** Internal commands
% **   \!ifnextchar{CHARACTER}{TRUE ACTION}{FALSE ACTION}
% **   \!tfor NAME := LIST \do {BODY}
% **   \!etfor NAME:= LIST \do {BODY}
% **   \!cfor NAME := LIST \do {BODY}
% **   \!ecfor NAME:= LIST \do {BODY}
% **   \!ifempty{MACRO}{TRUE ACTION}{FALSE ACTION}
% **   \!getnext\\ITEMfrom\LIST
% **   \!getnextvalueof\DIMEN\from\LIST
% **   \!copylist\LISTMACRO_A\to\LISTMACRO_B
% **   \!wlet\CONTROL_SEQUENCE_A=\CONTROL_SEQUENCE_B
% **   \!listaddon ITEM LIST
% **   \!rightappendITEM\withCS\to\LISTMACRO
% **   \!leftappendITEM\withCS\to\LISTMACRO
% **   \!lop\LISTMACRO\to\ITEM
% **   \!loop ... repeat
% **   \!!loop ... repeat
% **   \!removept{DIMENSION REGISTER}{CONTROL SEQUENCE}
% **   \!mlap{...}  
% **   \!vmlap{...}
% **   \!not{TEK if-CONDITION}

% ** First, here are the the PiCTeX logo, and the syllable PiC:
\def\PiC{P\kern-.12em\lower.5ex\hbox{I}\kern-.075emC}
\def\PiCTeX{\PiC\kern-.11em\TeX}

% ** The following macro expands to parameter #2 or parameter #3 according to
% ** whether the next non-blank character following the macro is or is not #1. 
% ** Blanks following the macro are gobbled.
\def\!ifnextchar#1#2#3{%
  \let\!testchar=#1%
  \def\!first{#2}%
  \def\!second{#3}%
  \futurelet\!nextchar\!testnext}
\def\!testnext{%
  \ifx \!nextchar \!spacetoken 
    \let\!next=\!skipspacetestagain
  \else
    \ifx \!nextchar \!testchar
      \let\!next=\!first
    \else 
      \let\!next=\!second 
    \fi 
  \fi
  \!next}
\def\\{\!skipspacetestagain} 
  \expandafter\def\\ {\futurelet\!nextchar\!testnext} 
\def\\{\let\!spacetoken= } \\  %  ** set \spacetoken to a space token
 
 
% ** Borrow the "tfor" macro from Latex:
% **   \!tfor NAME := LIST \do {BODY}
% **   if, before expansion, LIST = T1 ... Tn,  where each  Ti  is a token
% **   or  {...},  then executes  BODY  n  times, with  NAME = Ti  on the
% **   i-th iteration.  Works for  n=0.
\def\!tfor#1:=#2\do#3{%
  \edef\!fortemp{#2}%
  \ifx\!fortemp\!empty 
    \else
    \!tforloop#2\!nil\!nil\!!#1{#3}%
  \fi}
\def\!tforloop#1#2\!!#3#4{%
  \def#3{#1}%
  \ifx #3\!nnil
    \let\!nextwhile=\!fornoop
  \else
    #4\relax
    \let\!nextwhile=\!tforloop
  \fi 
  \!nextwhile#2\!!#3{#4}}
 
 
% **   \!etfor NAME:= LIST\do {BODY}
% **   This is like \!cfor, but LIST is any balanced token list whose complete
% **     expansion has the form  T1 ... Tn
\def\!etfor#1:=#2\do#3{%
  \def\!!tfor{\!tfor#1:=}%
  \edef\!!!tfor{#2}%
  \expandafter\!!tfor\!!!tfor\do{#3}}
 
 
% **   modify the Latex \tfor (token-for) loop to a \cfor (comma-for) loop.
% **   \!cfor NAME := LIST \do {BODY}
% **     if, before expansion, LIST = a1,a2,...an, then executes  BODY n times,
% **     with  NAME = ai  on the i-th iteration.  Works for  n=0.
\def\!cfor#1:=#2\do#3{%
  \edef\!fortemp{#2}%
  \ifx\!fortemp\!empty 
  \else
    \!cforloop#2,\!nil,\!nil\!!#1{#3}%
  \fi}
\def\!cforloop#1,#2\!!#3#4{%
  \def#3{#1}%
  \ifx #3\!nnil
    \let\!nextwhile=\!fornoop 
  \else
    #4\relax
    \let\!nextwhile=\!cforloop
  \fi
  \!nextwhile#2\!!#3{#4}}
 
 
% **   \!ecfor NAME:= LIST\do {BODY}
% **   This is like \!cfor, but LIST is any balanced token list whose complete
% **     expansion has the form  a1,a2,...,an.
\def\!ecfor#1:=#2\do#3{%
  \def\!!cfor{\!cfor#1:=}%
  \edef\!!!cfor{#2}%
  \expandafter\!!cfor\!!!cfor\do{#3}}
 
 
\def\!empty{}
\def\!nnil{\!nil}
\def\!fornoop#1\!!#2#3{}
 
 
% **  \!ifempty{ARG}{TRUE ACTION}{FALSE ACTION}
\def\!ifempty#1#2#3{%
  \edef\!emptyarg{#1}%
  \ifx\!emptyarg\!empty
    #2%
  \else
    #3%
  \fi}
 
% **  \!getnext\\ITEMfrom\LIST
% **    \LIST has the form \\{item1}\\{item2}\\{item3}...\\{itemk}
% **    This routine sets \ITEM to item1, and cycles \LIST to
% **    \\{item2}\\{item3}...\\{itemk}\\{item1}
\def\!getnext#1\from#2{%
  \expandafter\!gnext#2\!#1#2}%
\def\!gnext\\#1#2\!#3#4{%
  \def#3{#1}%
  \def#4{#2\\{#1}}%
  \ignorespaces}
 
 
% ** \!getnextvalueof\DIMEN\from\LIST
% **   Similar to !getnext.  
% **   \LIST has the form \\{dimen1}\\{dimen2}\\{dimen3} ... 
% **   \DIMEN is a dimension register
% **   Works also for counts
%
\def\!getnextvalueof#1\from#2{%
  \expandafter\!gnextv#2\!#1#2}%
\def\!gnextv\\#1#2\!#3#4{%
  #3=#1%
  \def#4{#2\\{#1}}%
  \ignorespaces}
 
 
% ** \!copylist\LISTMACROA\to\LISTMACROB
% **   makes the replacement text of LISTMACRO B identical to that of
% **   list macro A.
\def\!copylist#1\to#2{%
  \expandafter\!!copylist#1\!#2}
\def\!!copylist#1\!#2{%
  \def#2{#1}\ignorespaces}
 
 
% **  \!wlet\CSA=\CSB
% **  lets control sequence \CSB = control sequence \CSA, and writes a
% **    message to that effect in the log file using plain TEK's \wlog
\def\!wlet#1=#2{%
  \let#1=#2 
  \wlog{\string#1=\string#2}}
 
% ** \!listaddon ITEM LIST
% ** LIST <-- LIST \\ ITEM
\def\!listaddon#1#2{%
  \expandafter\!!listaddon#2\!{#1}#2}
\def\!!listaddon#1\!#2#3{%
  \def#3{#1\\#2}}
 
% **  \!rightappendITEM\to\LISTMACRO
% **    \LISTMACRO --> \LISTMACRO\\{ITEM}
%\def\!rightappend#1\to#2{\expandafter\!!rightappend#2\!{#1}#2}
%\def\!!rightappend#1\!#2#3{\def#3{#1\\{#2}}}
 
 
% **  \!rightappendITEM\withCS\to\LISTMACRO
% **    \LISTMACRO --> \LISTMACRO||CS||{ITEM}
\def\!rightappend#1\withCS#2\to#3{\expandafter\!!rightappend#3\!#2{#1}#3}
\def\!!rightappend#1\!#2#3#4{\def#4{#1#2{#3}}}
 
 
% **  \!leftappendITEM\withCS\to\LISTMACRO
% **    \LISTMACRO --> CS||{ITEM}||\LISTMACRO
\def\!leftappend#1\withCS#2\to#3{\expandafter\!!leftappend#3\!#2{#1}#3}
\def\!!leftappend#1\!#2#3#4{\def#4{#2{#3}#1}}
 
 
% **  \!lop\LISTMACRO\to\ITEM
% **    \\{item1}\\{item2}\\{item3} ... --> \\{item2}\\{item3} ...
% **    item1 --> \ITEM
\def\!lop#1\to#2{\expandafter\!!lop#1\!#1#2}
\def\!!lop\\#1#2\!#3#4{\def#4{#1}\def#3{#2}}
 
 
% **  \!placeNUMBER\of\LISTMACRO\in\ITEM
% **    the NUMBERth item of \LISTMACRO --> replacement text of \ITEM
%\def\!place#1\of#2\in#3{\def#3{\outofrange}%
%{\count0=#1\def\\##1{\advance\count0-1 \ifnum\count0=0 \gdef#3{##1}\fi}#2}}
 
 
% **  Following code converts a commalist to a list macro, with all items 
% **    fully expanded.
%\!ecfor\item:=\commalist\do{\expandafter\!rightappend\item\to\list}
 
 
% ** \!loop ... repeat
% ** This is exactly like TEX's \loop ... repeat.  It can be used in nesting
% ** two loops, without puting the inner one inside a group.
\def\!loop#1\repeat{\def\!body{#1}\!iterate}
\def\!iterate{\!body\let\!next=\!iterate\else\let\!next=\relax\fi\!next}
 
% ** \!!loop ... repeat
% ** This is exactly like TEX's \loop ... repeat.  It can be used in nesting
% ** two loops, without puting the inner one inside a group.
\def\!!loop#1\repeat{\def\!!body{#1}\!!iterate}
\def\!!iterate{\!!body\let\!!next=\!!iterate\else\let\!!next=\relax\fi\!!next}
%  (\multiput uses \!!loop)
 
% ** \!removept{DIMENREG}{\CS}
% ** Defines the control sequence CS to be the value (in points) in the
% ** dimension register DIMENREG (but without the "pt" TEK usually adds)
% ** E.g., after  \dimen0=12.3pt \!removept\dimen0\A, \A expands to 12.3
\def\!removept#1#2{\edef#2{\expandafter\!!removePT\the#1}}
{\catcode`p=12 \catcode`t=12 \gdef\!!removePT#1pt{#1}}

% ** \pladevalueinpts of <DIMENSION REGISTER> in {CONTROL SEQUENCE}
\def\placevalueinpts of <#1> in #2 {%
  \!removept{#1}{#2}}
 
% ** \!mlap{...}  \!vmlap{...}
% ** Center  ...  in a box of width 0.
\def\!mlap#1{\hbox to 0pt{\hss#1\hss}}
\def\!vmlap#1{\vbox to 0pt{\vss#1\vss}}
 
% ** \!not{TEK if-CONDITION}
% ** By a TEK if-CONDITION is meant something like 
% **     \ifnum\N<0,   or   \ifdim\A>\B
% ** \!not produces an if-condition which is false if the original condition
% ** is true, and true if the original condition is false.
\def\!not#1{%
  #1\relax
    \!switchfalse
  \else
    \!switchtrue
  \fi
  \if!switch
  \ignorespaces}
 

% *******************
% *** ALLOCATIONS ***
% *******************

% This section allocates all the registers PiCTeX uses. Following
% each allocation is a string of the form  ....N.D...L......... ;
% the various letters show which sections of PiCTeX make explicit
% reference to that register, according to the following code:
 
%   H Hacks
%   A Areas
%   W arroWs
%   B Bars
%   X boXes
%   C Curves
%   D Dashpattterns
%   V diVision
%   E Ellipses
%   U rUles
%   L Linear arc
%   G loGten
%   P Pictures
%   O plOtting
%   Y pYthagoras
%   Q Quadratic arc
%   R Rotations
%   S Shading
%   T Ticks

% Turn off messages from TeX's allocation macros
\let\!!!wlog=\wlog              % "\wlog" is defined in plain TeX
\def\wlog#1{}    

\newskip\headingtoplotskip      %.A.................	%-as-%
\newskip\linethickness         %.A..X....U........T
\newskip\longticklength         %.A................T	%-as-%
\newskip\plotsymbolspacing     %......D...L....Q...
\newskip\shortticklength        %.A................T	%-as-%
\newskip\stackleading           %.A..........P......	%-as-%
\newskip\tickstovaluesleading   %.A................T	%-as-%
\newskip\totalarclength        %......D...L....Q...
\newskip\valuestolabelleading   %.A.................	%-as-%

\newbox\!boxA                   %.AW...............T
\newbox\!boxB                   %..W................
\newbox\!picbox                 %............P......
\newbox\!plotsymbol             %..........L..O.....
\newbox\!putobject              %............PO...S.
\newbox\!shadesymbol            %.................S.

\newcount\!countA               %.A....D..UL....Q.ST
\newcount\!countB               %......D..U.....Q.ST
\newcount\!countC               %...............Q..T
\newcount\!countD               %...................
\newcount\!countE               %.............O....T
\newcount\!countF               %.............O....T
\newcount\!countG               %..................T
\newcount\!fiftypt              %.........U.........
\newcount\!intervalno           %..........L....Q...
\newcount\!npoints              %..........L........
\newcount\!nsegments            %.........U.........
\newcount\!ntemp                %............P......
\newcount\!parity               %.................S.
\newcount\!scalefactor          %..................T
\newcount\!tickcase             %..................T

\newskip\!Xleft                %............P......
\newskip\!Xright               %............P......
\newskip\!Xsave                %.A................T
\newskip\!Ybot                 %............P......
\newskip\!Ysave                %.A................T
\newskip\!Ytop                 %............P......
\newskip\!angle                %........E..........
\newskip\!arclength            %..W......UL....Q...
\newskip\!areabloc             %.A........L........
\newskip\!arealloc             %.A........L........
\newskip\!arearloc             %.A........L........
\newskip\!areatloc             %.A........L........
\newskip\!bshrinkage           %.................S.
\newskip\!checkbot             %..........L........
\newskip\!checkleft            %..........L........
\newskip\!checkright           %..........L........
\newskip\!checktop             %..........L........
\newdimen\!dimenA               %.AW.X.DVEUL..OYQRST
\newdimen\!dimenB               %....X.DVEU...O.QRS.
\newdimen\!dimenC               %..W.X.DVEU......RS.
\newdimen\!dimenD               %..W.X.DVEU....Y.RS.
\newdimen\!dimenE               %..W........G..YQ.S.
\newdimen\!dimenF               %...........G..YQ.S.
\newdimen\!dimenG               %...........G..YQ.S.
\newdimen\!dimenH               %...........G..Y..S.
\newdimen\!dimenI               %...BX.........Y....
\newdimen\!distacross           %..........L....Q...
\newdimen\!downlength           %..........L........
\newdimen\!dp                   %.A..X.......P....S.
\newdimen\!dshade               %.................S.
\newdimen\!dxpos                %..W......U..P....S.
\newdimen\!dxprime              %...............Q...
\newdimen\!dypos                %..WB.....U..P......
\newdimen\!dyprime              %...............Q...
\newdimen\!ht                   %.A..X.......P....S.
\newdimen\!leaderlength         %......D..U.........
\newdimen\!lshrinkage           %.................S.
\newdimen\!midarclength         %...............Q...
\newdimen\!offset               %.A................T
\newdimen\!plotheadingoffset    %.A.................
\newdimen\!plotsymbolxshift     %..........L..O.....
\newdimen\!plotsymbolyshift     %..........L..O.....
\newdimen\!plotxorigin          %..........L..O.....
\newdimen\!plotyorigin          %..........L..O.....
\newdimen\!rshrinkage           %.................S.
\newdimen\!shadesymbolxshift    %.................S.
\newdimen\!shadesymbolyshift    %.................S.
\newdimen\!tshrinkage           %.................S.
\newdimen\!uplength             %..........L........
\newdimen\!wd                   %....X.......P....S.
\newdimen\!xB                   %...............Q...
\newdimen\!xC                   %...............Q...
\newdimen\!xE                   %..W.....E.L....Q.S.
\newdimen\!xM                   %..W.....E......Q.S.
\newdimen\!xS                   %..W.....E.L....Q.S.
\newdimen\!xaxislength          %.A................T
\newdimen\!xdiff                %..........L........
\newdimen\!xleft                %............P......
\newdimen\!xloc                 %..WB.....U.......S.
\newdimen\!xorigin              %.A........L.P....S.
\newdimen\!xpivot               %................R..
\newdimen\!xpos                 %..........L.P..Q.ST
\newdimen\!xprime               %...............Q...
\newdimen\!xright               %............P......
\newdimen\!xshade               %.................S.
\newdimen\!xshift               %..W.........PO...S.
\newdimen\!xtemp                %............P......
\newdimen\!xunit                %.AWBX...EUL.P..QRS.
\newdimen\!xxE                  %........E..........
\newdimen\!xxM                  %........E..........
\newdimen\!xxS                  %........E..........
\newdimen\!xxloc                %..WB....EU.........
\newdimen\!yB                   %...............Q...
\newdimen\!yC                   %...............Q...
\newdimen\!yE                   %..W.....E.L....Q...
\newdimen\!yM                   %..W.....E......Q...
\newdimen\!yS                   %..W.....E.L....Q...
\newdimen\!yaxislength          %.A................T
\newdimen\!ybot                 %............P......
\newdimen\!ydiff                %..........L........
\newdimen\!yloc                 %..WB.....U.......S.
\newdimen\!yorigin              %.A........L.P....S.
\newdimen\!ypivot               %................R..
\newdimen\!ypos                 %..........L.P..Q.ST
\newdimen\!yprime               %...............Q...
\newdimen\!yshade               %.................S.
\newdimen\!yshift               %..W.........PO...S.
\newdimen\!ytemp                %............P......
\newdimen\!ytop                 %............P......
\newdimen\!yunit                %.AWBX...EUL.P..QRS.
\newdimen\!yyE                  %........E..........
\newdimen\!yyM                  %........E..........
\newdimen\!yyS                  %........E..........
\newdimen\!yyloc                %..WB....EU.........

\newif\if!axisvisible           %.A.................
\newif\if!gridlinestoo          %..................T
\newif\if!keepPO                %...................
\newif\if!placeaxislabel        %.A.................
\newif\if!switch                %H..................
\newif\if!xswitch               %.A................T

\newtoks\!axisLaBeL             %.A.................
\newtoks\!keywordtoks           %.A.................

\newwrite\!replotfile           %.............O.....

\newhelp\!keywordhelp{The keyword mentioned in the error message in unknown. 
Replace NEW KEYWORD in the indicated response by the keyword that 
should have been specified.}    %.A.................

% The following commands assign alternate names to some of the 
% above registers.  "\!wlet"  is defined in  Hacks.
\!wlet\!!origin=\!xM                   %.A................T
\!wlet\!!unit=\!uplength               %.A................T
\!wlet\!Lresiduallength=\!dimenG       %.........U.........
\!wlet\!Rresiduallength=\!dimenF       %.........U.........
\!wlet\!axisLength=\!distacross        %.A................T
\!wlet\!axisend=\!ydiff                %.A................T
\!wlet\!axisstart=\!xdiff              %.A................T
\!wlet\!axisxlevel=\!arclength         %.A................T
\!wlet\!axisylevel=\!downlength        %.A................T
\!wlet\!beta=\!dimenE                  %...............Q...
\!wlet\!gamma=\!dimenF                 %...............Q...
\!wlet\!shadexorigin=\!plotxorigin     %.................S.
\!wlet\!shadeyorigin=\!plotyorigin     %.................S.
\!wlet\!ticklength=\!xS                %..................T
\!wlet\!ticklocation=\!xE              %..................T
\!wlet\!ticklocationincr=\!yE          %..................T
\!wlet\!tickwidth=\!yS                 %..................T
\!wlet\!totalleaderlength=\!dimenE     %.........U.........
\!wlet\!xone=\!xprime                  %....X..............
\!wlet\!xtwo=\!dxprime                 %....X..............
\!wlet\!ySsave=\!yM                    %...................
\!wlet\!ybB=\!yB                       %.................S.
\!wlet\!ybC=\!yC                       %.................S.
\!wlet\!ybE=\!yE                       %.................S.
\!wlet\!ybM=\!yM                       %.................S.
\!wlet\!ybS=\!yS                       %.................S.
\!wlet\!ybpos=\!yyloc                  %.................S.
\!wlet\!yone=\!yprime                  %....X..............
\!wlet\!ytB=\!xB                       %.................S.
\!wlet\!ytC=\!xC                       %.................S.
\!wlet\!ytE=\!downlength               %.................S.
\!wlet\!ytM=\!arclength                %.................S.
\!wlet\!ytS=\!distacross               %.................S.
\!wlet\!ytpos=\!xxloc                  %.................S.
\!wlet\!ytwo=\!dyprime                 %....X..............


% Initial values for registers
\def\!zpt{0pt}                              % static
\!xunit=1pt
\!yunit=1pt
\!arearloc=\!xunit
\!areatloc=\!yunit
\!dshade=5pt
\!leaderlength=24in
\def\!tfs{256}                              % static
\def\!wmax{5.3pt}                           % static
\def\!wmin{2.7pt}                           % static
\!xaxislength=\!xunit
\!xpivot=\!zpt
\!yaxislength=\!yunit 
\!ypivot=\!zpt
\plotsymbolspacing=.4pt
  \!dimenA=50pt \def\!fiftypt{\the\!dimenA}     % static

\def\!rootten{3.162278pt}                   % static
\def\!tenAa{8.690286pt}                     % static  (A5)
\def\!tenAc{2.773839pt}                     % static  (A3)
\def\!tenAe{2.543275pt}                     % static  (A1)

% Initial values for control sequences
\def\!cosrotationangle{1}      %................R..
\def\!sinrotationangle{0}      %................R..
\def\!xpivotcoord{0}           %................R..
\def\!xref{0}                  %............P......
\def\!xshadesave{0}            %.................S.
\def\!ypivotcoord{0}           %................R..
\def\!yref{0}                  %............P......
\def\!yshadesave{0}            %.................S.
\def\!zero{0}                  %..................T

% Reset TeX to report allocations
\let\wlog=\!!!wlog
%  *************************************
%  ***  AREAS: Deals with plot areas ***
%  *************************************
%
%  ** User commands
%  **   \setplotarea x from LEFT XCOORD to RIGTH XCOORD, y from BOTTOM YCOORD
%  **      to TOP YCOORD
%  **   \axis BOTTOM-LEFT-TOP-RIGHT  [SHIFTEDTO xy=COORD] [VISIBLE-INVISIBLE]
%  **      [LABEL {label}] [TICKS] /
%  **   \visibleaxes
%  **   \invisibleaxes
%  **   \plotheading {HEADING}
%  **   \grid {# of columns} {# of rows}
%  **   \normalgraphs 
  
%  **  \normalgraphs
%  **    Sets defaults for graph setup. See Subsection 3.4 of manual.
\def\normalgraphs{%
  \longticklength=.4\baselineskip
  \shortticklength=.25\baselineskip
  \tickstovaluesleading=.25\baselineskip
  \valuestolabelleading=.8\baselineskip
  \linethickness=.4pt
  \stackleading=.17\baselineskip
  \headingtoplotskip=1.5\baselineskip
  \visibleaxes
  \ticksout
  \nogridlines
  \unloggedticks}
%
% **  \setplotarea x from LEFT XCOORD to RIGTH XCOORD, y from BOTTOM YCOORD
% **    to TOP YCOORD
% **  Reserves space in PICBOX for a rectangular box with the indicated
% **   coordinates.  Must be specified before calls to  \axis, 
% **   \grid, \plotheading.
% **  See Subsection 3.1 of the manual.
\def\setplotarea x from #1 to #2, y from #3 to #4 {%
  \!arealloc=\!M{#1}\!xunit \advance \!arealloc -\!xorigin
  \!areabloc=\!M{#3}\!yunit \advance \!areabloc -\!yorigin
  \!arearloc=\!M{#2}\!xunit \advance \!arearloc -\!xorigin
  \!areatloc=\!M{#4}\!yunit \advance \!areatloc -\!yorigin
  \!initinboundscheck
  \!xaxislength=\!arearloc  \advance\!xaxislength -\!arealloc
  \!yaxislength=\!areatloc  \advance\!yaxislength -\!areabloc
  \!plotheadingoffset=\!zpt
  \!dimenput {{\setbox0=\hbox{}\wd0=\!xaxislength\ht0=\!yaxislength\box0}}
     [bl] (\!arealloc,\!areabloc)}
%
% ** \visibleaxes, \invisibleaxes 
% ** Switches for setting visibility of subsequent axes.
% ** See Subsection 3.2 of the manual.
\def\visibleaxes{%
  \def\!axisvisibility{\!axisvisibletrue}}
\def\invisibleaxes{%
  \def\!axisvisibility{\!axisvisiblefalse}}
%
% ** The next few macros enable the user to fix up an erroneous keyword
% **   in the \axis command.
%  \newhelp is in ALLOCATIONS
%  \newhelp\!keywordhelp{The keyword mentioned in the error message in unknown. 
%  Replace NEW KEYWORD in the indicated response by the keyword that 
%  should have been specified.}

\def\!fixkeyword#1{%
  \errhelp=\!keywordhelp
  \errmessage{Unrecognized keyword `#1': \the\!keywordtoks{NEW KEYWORD}'}}

%  \newtoks\!keywordtoks    In ALLOCATIONS.
\!keywordtoks={enter `i\fixkeyword}

\def\fixkeyword#1{%
  \!nextkeyword#1 }

% ** \axis BOTTOM-LEFT-TOP-RIGHT  [SHIFTEDTO xy=COORD] [VISIBLE-INVISIBLE]
% **   [LABEL {label}] [TICKS] /
% ** Exactly one of the keywords BOTTOM, LEFT, TOP, RIGHT must be
% ** specified. Axis is drawn along the indicated edge of the current
% ** plot area, shifted if the SHIFTEDTO option is used, visible or
% ** invisible according the selected option, with an optional LABEL,
% ** and optional TICKS (see ticks.tex for the options avialabel with
% ** TICKS). The TICKS option must be the last one specified. The \axis
% ** MUST be terminated with a / followed by a space.
% ** See Subsection 3.2 of the manual for more information.

% ** The various options of the \axis command are processed by the
% ** \!nextkeyword macro defined below. For example, 
% ** `\!nextkeyword shiftedto ' expands to `\!axisshiftedto'.
\def\axis {%
  \def\!nextkeyword##1 {%
    \expandafter\ifx\csname !axis##1\endcsname \relax
      \def\!next{\!fixkeyword{##1}}%
    \else
      \def\!next{\csname !axis##1\endcsname}%
    \fi
    \!next}%
  \!offset=\!zpt
  \!axisvisibility
  \!placeaxislabelfalse
  \!nextkeyword}

% ** This and the various macros that follow handle the keyword
% ** specifications on the \axis command
% ** See Subsection 3.2 of the manual.
\def\!axisbottom{%
  \!axisylevel=\!areabloc
  \def\!tickxsign{0}%
  \def\!tickysign{-}%
  \def\!axissetup{\!axisxsetup}%
  \def\!axislabeltbrl{t}%
  \!nextkeyword}

\def\!axistop{%
  \!axisylevel=\!areatloc
  \def\!tickxsign{0}%
  \def\!tickysign{+}%
  \def\!axissetup{\!axisxsetup}%
  \def\!axislabeltbrl{b}%
  \!nextkeyword}

\def\!axisleft{%
  \!axisxlevel=\!arealloc
  \def\!tickxsign{-}%
  \def\!tickysign{0}%
  \def\!axissetup{\!axisysetup}%
  \def\!axislabeltbrl{r}%
  \!nextkeyword}

\def\!axisright{%
  \!axisxlevel=\!arearloc
  \def\!tickxsign{+}%
  \def\!tickysign{0}%
  \def\!axissetup{\!axisysetup}%
  \def\!axislabeltbrl{l}%
  \!nextkeyword}

\def\!axisshiftedto#1=#2 {%
  \if 0\!tickxsign
    \!axisylevel=\!M{#2}\!yunit
    \advance\!axisylevel -\!yorigin
  \else
    \!axisxlevel=\!M{#2}\!xunit
    \advance\!axisxlevel -\!xorigin
  \fi
  \!nextkeyword}

\def\!axisvisible{%
  \!axisvisibletrue  
  \!nextkeyword}

\def\!axisinvisible{%
  \!axisvisiblefalse
  \!nextkeyword}

\def\!axislabel#1 {%
  \!axisLaBeL={#1}%
  \!placeaxislabeltrue
  \!nextkeyword}

\expandafter\def\csname !axis/\endcsname{%
  \!axissetup % This could done already by "ticks"; if so, now \relax
  \if!placeaxislabel
    \!placeaxislabel
  \fi
  \if +\!tickysign %                 ** (A "top" axis)
    \!dimenA=\!axisylevel
    \advance\!dimenA \!offset %      ** dimA = top of the axis structure
    \advance\!dimenA -\!areatloc %   ** dimA = excess over the plot area
    \ifdim \!dimenA>\!plotheadingoffset
      \!plotheadingoffset=\!dimenA % ** Greatest excess over the plot area
    \fi
  \fi}

% ** \grid {c} {r} 
% ** Partitions the plot area into c columns and r rows; see Subsection 3.3
% ** of the manual.
% ** (Other grid patterns can be drawn with the TICKS option of the \axis 
% ** command.
\def\grid #1 #2 {%
  \!countA=#1\advance\!countA 1
  \axis bottom invisible ticks length <\!zpt> andacross quantity {\!countA} /
  \!countA=#2\advance\!countA 1
  \axis left   invisible ticks length <\!zpt> andacross quantity {\!countA} / }

% ** \plotheading{HEADING}
% ** Places HEADING centered above the top of the plotarea (and above
% ** any top axis ticks marks, tick labels, and axis label); see
% ** Subsection 3.3 of the manual.
\def\plotheading#1 {%
  \advance\!plotheadingoffset \headingtoplotskip
  \!dimenput {#1} [B] <.5\!xaxislength,\!plotheadingoffset>
    (\!arealloc,\!areatloc)}

% ** From here on, the routines are internal.
\def\!axisxsetup{%
  \!axisxlevel=\!arealloc
  \!axisstart=\!arealloc
  \!axisend=\!arearloc
  \!axisLength=\!xaxislength
  \!!origin=\!xorigin
  \!!unit=\!xunit
  \!xswitchtrue
  \if!axisvisible 
    \!makeaxis
  \fi}

\def\!axisysetup{%
  \!axisylevel=\!areabloc
  \!axisstart=\!areabloc
  \!axisend=\!areatloc
  \!axisLength=\!yaxislength
  \!!origin=\!yorigin
  \!!unit=\!yunit
  \!xswitchfalse
  \if!axisvisible
    \!makeaxis
  \fi}

\def\!makeaxis{%
  \setbox\!boxA=\hbox{% (Make a pseudo-y[x] tick for an x[y]-axis)
    \beginpicture
      \!setdimenmode
      \setcoordinatesystem point at {\!zpt} {\!zpt}   
      \putrule from {\!zpt} {\!zpt} to
        {\!tickysign\!tickysign\!axisLength} 
        {\!tickxsign\!tickxsign\!axisLength}
    \endpicturesave <\!Xsave,\!Ysave>}%
    \wd\!boxA=\!zpt
    \!placetick\!axisstart}

\def\!placeaxislabel{%
  \advance\!offset \valuestolabelleading
  \if!xswitch
    \!dimenput {\the\!axisLaBeL} [\!axislabeltbrl]
      <.5\!axisLength,\!tickysign\!offset> (\!axisxlevel,\!axisylevel)
    \advance\!offset \!dp  % ** advance offset by the "tallness"
    \advance\!offset \!ht  % ** of the label
  \else
    \!dimenput {\the\!axisLaBeL} [\!axislabeltbrl]
      <\!tickxsign\!offset,.5\!axisLength> (\!axisxlevel,\!axisylevel)
  \fi
  \!axisLaBeL={}}


% *******************************
% *** ARROWS  (Draws arrows)  ***
% *******************************
%
% ** User commands
% **  \arrow <ARROW HEAD LENGTH> [MID FRACTION, BASE FRACTION]
% **    [<XSHIFT,YSHIFT>] from XFROM YFROM to XTO YTO
% **  \betweenarrows {TEXT} [orientation & shift] from XFROM YFROM to XTO YTO

% ** \arrow <ARROW HEAD LENGTH> [MID FRACTION, BASE FRACTION]
% **    [<XSHIFT,YSHIFT>] from XFROM YFROM to XTO YTO
% ** Draws an arrow from (XFROM,YFROM) to (XTO,YTO).  The arrow head
% ** is constructed two quadratic arcs, which extend back a distance
% ** ARROW HEAD LENGTH (a dimension) on both sides of the arrow shaft.
% ** All the way back the arcs are a distance BASE FRACTION*ARROW HEAD
% ** LENGTH apart, while half-way back they are a distance MID FRACTION*
% ** ARROW HEAD LENGTH apart. <XSHIFT,YSHIFT> is optional, and has
% ** its usual interpreation. See Subsection 5.4 of the manual.

\def\arrow <#1> [#2,#3]{%
  \!ifnextchar<{\!arrow{#1}{#2}{#3}}{\!arrow{#1}{#2}{#3}<\!zpt,\!zpt> }}

\def\!arrow#1#2#3<#4,#5> from #6 #7 to #8 #9 {%
%
% ** convert to dimensions
  \!xloc=\!M{#8}\!xunit   
  \!yloc=\!M{#9}\!yunit
  \!dxpos=\!xloc  \!dimenA=\!M{#6}\!xunit  \advance \!dxpos -\!dimenA
  \!dypos=\!yloc  \!dimenA=\!M{#7}\!yunit  \advance \!dypos -\!dimenA
  \let\!MAH=\!M%                         ** save current c/d mode
  \!setdimenmode%                        ** go into dimension mode
%
  \!xshift=#4\relax  \!yshift=#5\relax%  ** pick up shift
  \!reverserotateonly\!xshift\!yshift%   ** back rotate shift
  \advance\!xshift\!xloc  \advance\!yshift\!yloc
%
% **  draw shaft of arrow
  \!xS=-\!dxpos  \advance\!xS\!xshift
  \!yS=-\!dypos  \advance\!yS\!yshift
  \!start (\!xS,\!yS)
  \!ljoin (\!xshift,\!yshift)
%
% ** find 32*cosine and 32*sine of angle of rotation
  \!Pythag\!dxpos\!dypos\!arclength
  \!divide\!dxpos\!arclength\!dxpos  
  \!dxpos=32\!dxpos  \!removept\!dxpos\!!cos
  \!divide\!dypos\!arclength\!dypos  
  \!dypos=32\!dypos  \!removept\!dypos\!!sin
% 
% ** construct arrowhead
  \!halfhead{#1}{#2}{#3}%                ** draw half of arrow head
  \!halfhead{#1}{-#2}{-#3}%              ** draw other half
%
  \let\!M=\!MAH%                         ** restore old c/d mode
  \ignorespaces}
%
% ** draw half of arrow head
  \def\!halfhead#1#2#3{%
    \!dimenC=-#1%                
    \divide \!dimenC 2 %                 ** half way back
    \!dimenD=#2\!dimenC%                 ** half the mid width
    \!rotate(\!dimenC,\!dimenD)by(\!!cos,\!!sin)to(\!xM,\!yM)
    \!dimenC=-#1%                        ** all the way back
    \!dimenD=#3\!dimenC
    \!dimenD=.5\!dimenD%                 ** half the full width
    \!rotate(\!dimenC,\!dimenD)by(\!!cos,\!!sin)to(\!xE,\!yE)
    \!start (\!xshift,\!yshift)
    \advance\!xM\!xshift  \advance\!yM\!yshift
    \advance\!xE\!xshift  \advance\!yE\!yshift
    \!qjoin (\!xM,\!yM) (\!xE,\!yE) 
    \ignorespaces}


% ** \betweenarrows {TEXT} [orientation & shift] from XFROM YFROM to XTO YTO
% **   Makes things like <--- text --->, using arrow heads from TeX's fonts.
% **   See Subsection 5.4 of the manual.
\def\betweenarrows #1#2 from #3 #4 to #5 #6 {%
  \!xloc=\!M{#3}\!xunit  \!xxloc=\!M{#5}\!xunit%   
  \!yloc=\!M{#4}\!yunit  \!yyloc=\!M{#6}\!yunit%           
  \!dxpos=\!xxloc  \advance\!dxpos by -\!xloc
  \!dypos=\!yyloc  \advance\!dypos by -\!yloc
  \advance\!xloc .5\!dxpos
  \advance\!yloc .5\!dypos
%
  \let\!MBA=\!M%           ** save current coord\dimen mode
  \!setdimenmode%          ** express locations in dimens
  \ifdim\!dypos=\!zpt
    \ifdim\!dxpos<\!zpt \!dxpos=-\!dxpos \fi
    \put {\!lrarrows{\!dxpos}{#1}}#2{} at {\!xloc} {\!yloc}
  \else
    \ifdim\!dxpos=\!zpt
      \ifdim\!dypos<\!zpt \!dypos=-\!zpt \fi
      \put {\!udarrows{\!dypos}{#1}}#2{} at {\!xloc} {\!yloc}
    \fi
  \fi
  \let\!M=\!MBA%           ** restore previous c/d mode
  \ignorespaces}

% ** Subroutine for left-right between arrows 
\def\!lrarrows#1#2{% #1=width, #2=text
  {\setbox\!boxA=\hbox{$\mkern-2mu\mathord-\mkern-2mu$}%
   \setbox\!boxB=\hbox{$\leftarrow$}\!dimenE=\ht\!boxB
   \setbox\!boxB=\hbox{}\ht\!boxB=2\!dimenE
   \hbox to #1{$\mathord\leftarrow\mkern-6mu
     \cleaders\copy\!boxA\hfil
     \mkern-6mu\mathord-$%
     \kern.4em $\vcenter{\box\!boxB}$$\vcenter{\hbox{#2}}$\kern.4em
     $\mathord-\mkern-6mu
     \cleaders\copy\!boxA\hfil
     \mkern-6mu\mathord\rightarrow$}}}

% ** Subroutine for up-down between arrows 
\def\!udarrows#1#2{% #1=width, #2=text
  {\setbox\!boxB=\hbox{#2}%
   \setbox\!boxA=\hbox to \wd\!boxB{\hss$\vert$\hss}%
   \!dimenE=\ht\!boxA \advance\!dimenE \dp\!boxA \divide\!dimenE 2
   \vbox to #1{\offinterlineskip
      \vskip .05556\!dimenE
      \hbox to \wd\!boxB{\hss$\mkern.4mu\uparrow$\hss}\vskip-\!dimenE
      \cleaders\copy\!boxA\vfil
      \vskip-\!dimenE\copy\!boxA
      \vskip\!dimenE\copy\!boxB\vskip.4em
      \copy\!boxA\vskip-\!dimenE
      \cleaders\copy\!boxA\vfil
      \vskip-\!dimenE \hbox to \wd\!boxB{\hss$\mkern.4mu\downarrow$\hss}
      \vskip .05556\!dimenE}}}


% ***************************
% *** BARS  (Draws bars)  ***
% ***************************
%
% ** User commands:
% ** \putbar [<XSHIFT,YSHIFT>] breadth <BREADTH> from XSTART YSTART
% **   to XEND YEND
% ** \setbars [<XSHIFT,YSHIFT>] breadth <BREADTH> baseline at XY = COORD
% **   [baselabels ([B_ORIENTATION_x,B_ORIENTATION_y] <B_XSHIFT,B_YSHIFT>)]
% **   [endlabels  ([E_ORIENTATION_x,E_ORIENTATION_y] <E_XSHIFT,E_YSHIFT>)]


% ** \putbar [<XSHIFT,YSHIFT>] breadth <BREADTH> from XSTART YSTART
% **   to XEND YEND
% ** Either XSTART=XEND or YSTART=YEND. Draws a rectangle between
% **   (XSTART,YSTART) & (XEND,YEND). The "depth" of the rectangle
% **   is determined by those two plot positions; its other
% **   dimension "breadth" is specified by the dimension BREADTH.
% ** See Subsection 4.2 of the manual.
\def\putbar#1breadth <#2> from #3 #4 to #5 #6 {%
  \!xloc=\!M{#3}\!xunit  \!xxloc=\!M{#5}\!xunit%   
  \!yloc=\!M{#4}\!yunit  \!yyloc=\!M{#6}\!yunit%           
  \!dypos=\!yyloc  \advance\!dypos by -\!yloc
  \!dimenI=#2  
% 
  \ifdim \!dimenI=\!zpt %            ** If 0 breadth
    \putrule#1from {#3} {#4} to {#5} {#6} % ** Then draw line
  \else %                            ** Else, put in a rectangle
    \let\!MBar=\!M%                  ** save current c/d mode
    \!setdimenmode %                 ** go into dimension mode
    \divide\!dimenI 2
    \ifdim \!dypos=\!zpt             
      \advance \!yloc -\!dimenI %    ** Equal y coordinates
      \advance \!yyloc \!dimenI
    \else
      \advance \!xloc -\!dimenI %    ** Equal x coordinates
      \advance \!xxloc \!dimenI
    \fi
    \putrectangle#1corners at {\!xloc} {\!yloc} and {\!xxloc} {\!yyloc}
    \let\!M=\!MBar %                 ** restore c/d mode
  \fi
  \ignorespaces}


% ** \setbars [<XSHIFT,YSHIFT>] breadth <BREADTH> baseline at XY = COORD
% **   [baselabels ([B_ORIENTATION_x,B_ORIENTATION_y] <B_XSHIFT,B_YSHIFT>)]
% **   [endlabels  ([E_ORIENTATION_x,E_ORIENTATION_y] <E_XSHIFT,E_YSHIFT>)]
% ** This command puts PiCTeX into the bar graph drawing mode described
% **   in Subsection 4.4 of the manual.
\def\setbars#1breadth <#2> baseline at #3 = #4 {%
  \edef\!barshift{#1}%
  \edef\!barbreadth{#2}%
  \edef\!barorientation{#3}%
  \edef\!barbaseline{#4}%
  \def\!bardobaselabel{\!bardoendlabel}%
  \def\!bardoendlabel{\!barfinish}%
  \let\!drawcurve=\!barcurve
  \!setbars}
\def\!setbars{%
  \futurelet\!nextchar\!!setbars}
\def\!!setbars{%
  \if b\!nextchar
    \def\!!!setbars{\!setbarsbget}%
  \else 
    \if e\!nextchar
      \def\!!!setbars{\!setbarseget}%
    \else
      \def\!!!setbars{\relax}%
    \fi
  \fi
  \!!!setbars}
\def\!setbarsbget baselabels (#1) {%
  \def\!barbaselabelorientation{#1}%
  \def\!bardobaselabel{\!!bardobaselabel}%
  \!setbars}
\def\!setbarseget endlabels (#1) {%
  \edef\!barendlabelorientation{#1}%
  \def\!bardoendlabel{\!!bardoendlabel}%
  \!setbars}

% ** \!barcurve
% ** Draws a bargraph with preset values of barshift, barbreadth,
% ** barorientation (x or y) and barbaseline (coordinate)
\def\!barcurve #1 #2 {%
  \if y\!barorientation
    \def\!basexarg{#1}%
    \def\!baseyarg{\!barbaseline}%
  \else
    \def\!basexarg{\!barbaseline}%
    \def\!baseyarg{#2}%
  \fi
  \expandafter\putbar\!barshift breadth <\!barbreadth> from {\!basexarg}
    {\!baseyarg} to {#1} {#2}
  \def\!endxarg{#1}%
  \def\!endyarg{#2}%
  \!bardobaselabel}

\def\!!bardobaselabel "#1" {%
  \put {#1}\!barbaselabelorientation{} at {\!basexarg} {\!baseyarg}
  \!bardoendlabel}
 
\def\!!bardoendlabel "#1" {%
  \put {#1}\!barendlabelorientation{} at {\!endxarg} {\!endyarg}
  \!barfinish}

\def\!barfinish{%
  \!ifnextchar/{\!finish}{\!barcurve}}
 

% ********************************
% *** BOXES (Draws rectangles) ***
% ********************************
%
% ** User commands:
% **   \putrectangle [<XSHIFT,YSHIFT>] corners at  XCOORD1 YCOORD1
% **     and  XCOORD2 YCOORD2 
% **   \shaderectangleson
% **   \shaderectanglesoff
% **   \frame [<SEPARATION>] {TEXT}
% **   \rectangle <WIDTH> <HEIGHT>
%
%
% **  \putrectangle [<XSHIFT,YSHIFT>] corners at XCOORD1 YCOORD1 
% **    and  XCOORD2 YCOORD2 
% **  Draws a rectangle with corners at (X1,Y1), (X2,Y1), (X1,Y2), (X2,Y2)
% **  Lines have thickness \linethickness, and overlap at the corners.
% **  The optional field  <XSHIFT,YSHIFT>  functions as with a \put command.
% **  See Subsection 4.2 of the manual.
\def\putrectangle{%
  \!ifnextchar<{\!putrectangle}{\!putrectangle<\!zpt,\!zpt> }}
\def\!putrectangle<#1,#2> corners at #3 #4 and #5 #6 {%
%
% ** get locations
  \!xone=\!M{#3}\!xunit  \!xtwo=\!M{#5}\!xunit%   
  \!yone=\!M{#4}\!yunit  \!ytwo=\!M{#6}\!yunit%           
  \ifdim \!xtwo<\!xone
    \!dimenI=\!xone  \!xone=\!xtwo  \!xtwo=\!dimenI
  \fi
  \ifdim \!ytwo<\!yone
    \!dimenI=\!yone  \!yone=\!ytwo  \!ytwo=\!dimenI
  \fi
  \!dimenI=#1\relax  \advance\!xone\!dimenI  \advance\!xtwo\!dimenI
  \!dimenI=#2\relax  \advance\!yone\!dimenI  \advance\!ytwo\!dimenI
  \let\!MRect=\!M%                  ** save current coord/dimen mode
  \!setdimenmode
%
% ** shade rectangle if appropriate
  \!shaderectangle
%
% ** draw horizontal edges
  \!dimenI=.5\linethickness
  \advance \!xone  -\!dimenI%       ** adjust x-location to overlap corners
  \advance \!xtwo   \!dimenI%       ** ditto
  \putrule from {\!xone} {\!yone} to {\!xtwo} {\!yone} 
  \putrule from {\!xone} {\!ytwo} to {\!xtwo} {\!ytwo} 
%
% ** draw vertical edges
  \advance \!xone   \!dimenI%       ** restore original x-values
  \advance \!xtwo  -\!dimenI% 
  \advance \!yone  -\!dimenI%       ** adjust y-location to overlap corners
  \advance \!ytwo   \!dimenI%       ** ditto
  \putrule from {\!xone} {\!yone} to {\!xone} {\!ytwo} 
  \putrule from {\!xtwo} {\!yone} to {\!xtwo} {\!ytwo} 
%
  \let\!M=\!MRect%                  ** restore coord/dimen mode
  \ignorespaces}
 
% ** \shaderectangleson 
% **   Subsequent rectangles will be shaded according to 
% **   the current shading pattern.  Affects \putrectangle, \putbar,
% **   \frame, \sethistograms, and \setbars. See Subsection 7.5 of the manual.
\def\shaderectangleson{%     
  \def\!shaderectangle{\!!shaderectangle}%
  \ignorespaces}
% ** \shaderectanglesoff 
% **    Suppresses  \shaderectangleson.  The default.
\def\shaderectanglesoff{%
  \def\!shaderectangle{}%
  \ignorespaces}

\shaderectanglesoff
 
% ** The following internal routine shades the current rectangle, when
% **   \!shaderectangle = \!!shaderectangle . 
\def\!!shaderectangle{%
  \!dimenA=\!xtwo  \advance \!dimenA -\!xone
  \!dimenB=\!ytwo  \advance \!dimenB -\!yone
  \ifdim \!dimenA<\!dimenB
    \!startvshade (\!xone,\!yone,\!ytwo)
    \!lshade      (\!xtwo,\!yone,\!ytwo)
  \else
    \!starthshade (\!yone,\!xone,\!xtwo)
    \!lshade      (\!ytwo,\!xone,\!xtwo)
  \fi
  \ignorespaces}
  
% ** \frame [<SEPARATION>] {TEXT}
% ** Draws a frame of thickness linethickness about the box enclosing
% **   TEXT; the frame is separated from the box by a distance of
% **   SEPARATION.  The result is an hbox with the same baseline as TEXT.
% **   If <SEPARATION> is omitted, you get the effect of <0pt>.
% ** See Subsection 4.2 of the manual.
\def\frame{%
  \!ifnextchar<{\!frame}{\!frame<\!zpt> }}
\long\def\!frame<#1> #2{%
  \beginpicture
    \setcoordinatesystem units <1pt,1pt> point at 0 0 
    \put {#2} [Bl] at 0 0 
    \!dimenA=#1\relax
    \!dimenB=\!wd \advance \!dimenB \!dimenA
    \!dimenC=\!ht \advance \!dimenC \!dimenA
    \!dimenD=\!dp \advance \!dimenD \!dimenA
    \let\!MFr=\!M
    \!setdimenmode
    \putrectangle corners at {-\!dimenA} {-\!dimenD} and {\!dimenB} {\!dimenC}
    \!setcoordmode
    \let\!M=\!MFr
  \endpicture
  \ignorespaces}
 
% ** \rectangle <WIDTH> <HEIGHT>
% ** Constructs a rectangle of width WIDTH and heigth HEIGHT. 
% ** See Subsection 4.2 of the manual.
\def\rectangle <#1> <#2> {%
  \setbox0=\hbox{}\wd0=#1\ht0=#2\frame {\box0}}


% *********************************************
% ***  CURVES  (Upper level \plot commands) ***
% *********************************************
%
% ** User commands
% **   \plot  DATA  /
% **   \plot  "FILE NAME"
% **   \setquadratic
% **   \setlinear
% **   \sethistograms
% **   \vshade  ...
% **   \hshade  ...

% \plot: multi-purpose command. Draws histograms, bar graphs, piecewise-linear
% or piecewise quadratic curves, depending on the setting of \!drawcurve.
% See Subsections 4.3-4.5, 5.1, 5.2 of the manual.
\def\plot{%
  \!ifnextchar"{\!plotfromfile}{\!drawcurve}}
\def\!plotfromfile"#1"{%
  \expandafter\!drawcurve \input #1 /}

% Command to set piecewise quadratic mode
% See Subsections 5.1, 7.3, and 7.4 of the manual.
\def\setquadratic{%
  \let\!drawcurve=\!qcurve
  \let\!!Shade=\!!qShade
  \let\!!!Shade=\!!!qShade}

% Command to set piecewise linear mode
% See Subsections 5.1, 7.3, and 7.4 of the manual.
\def\setlinear{%
  \let\!drawcurve=\!lcurve
  \let\!!Shade=\!!lShade
  \let\!!!Shade=\!!!lShade}

% Command to set histogram mode
% See Subsection 4.3 of the manual.
\def\sethistograms{%
  \let\!drawcurve=\!hcurve}

% Commands to cycle through list of coordinates in piecewise quadratic 
% interpolation mode
\def\!qcurve #1 #2 {%
  \!start (#1,#2)
  \!Qjoin}
\def\!Qjoin#1 #2 #3 #4 {%
  \!qjoin (#1,#2) (#3,#4)             % \!qjoin  is defined in QUADRATIC
  \!ifnextchar/{\!finish}{\!Qjoin}}

% Commands to cycle through list of coordinates in piecewise linear 
% interpolation mode
\def\!lcurve #1 #2 {%
  \!start (#1,#2)
  \!Ljoin}
\def\!Ljoin#1 #2 {%
  \!ljoin (#1,#2)                    % \!ljoin  is defined in LINEAR
  \!ifnextchar/{\!finish}{\!Ljoin}}

\def\!finish/{\ignorespaces}

% Command to cycle through list of coordinates in histogram mode
\def\!hcurve #1 #2 {%
  \edef\!hxS{#1}%
  \edef\!hyS{#2}%
  \!hjoin}
\def\!hjoin#1 #2 {%
  \putrectangle corners at {\!hxS} {\!hyS} and {#1} {#2}
  \edef\!hxS{#1}%
  \!ifnextchar/{\!finish}{\!hjoin}}


% \vshade: See Subsection 7.3 of the manual.
\def\vshade #1 #2 #3 {%
  \!startvshade (#1,#2,#3)
  \!Shadewhat}

% \hshade: See Subsection 7.4 of the manual.
\def\hshade #1 #2 #3 {%
  \!starthshade (#1,#2,#3)
  \!Shadewhat}

% Commands to cycle through coordinates and optional "edge effect"
% fields while shading.
\def\!Shadewhat{%
  \futurelet\!nextchar\!Shade}
\def\!Shade{%
  \if <\!nextchar
    \def\!nextShade{\!!Shade}%
  \else
    \if /\!nextchar
      \def\!nextShade{\!finish}%
    \else
      \def\!nextShade{\!!!Shade}%
    \fi
  \fi
  \!nextShade}
\def\!!lShade<#1> #2 #3 #4 {%
  \!lshade <#1> (#2,#3,#4)                 % \!lshade is defined in SHADING
  \!Shadewhat}
\def\!!!lShade#1 #2 #3 {%
  \!lshade (#1,#2,#3)
  \!Shadewhat} 
\def\!!qShade<#1> #2 #3 #4 #5 #6 #7 {%
  \!qshade <#1> (#2,#3,#4) (#5,#6,#7)      % \!qshade is defined in SHADING
  \!Shadewhat}
\def\!!!qShade#1 #2 #3 #4 #5 #6 {%
  \!qshade (#1,#2,#3) (#4,#5,#6)
  \!Shadewhat} 

% ** Set default interpolation mode
\setlinear


%  ********************************************
%  *** DASHPATTERNS (Sets up dash patterns) ***
%  ********************************************

%  **  User commands:
%  **    \setdashpattern <DIMEN1,DIMEN2,DIMEN3,...>
%  **    \setdots <INTRADOT_DISTANCE>
%  **    \setdotsnear <INTRADOT_DISTANCE> for <ARC LENGTH>
%  **    \setdashes <DASH/SKIP_DISTANCE>
%  **    \setdashesnear <DASH/SKIP_DISTANCE> for <ARC LENGTH>
%  **    \setsolid
%  **    \findlength {CURVE CMDS}
 
%  **  Internal commands:
%  **    \!dashingon
%  **    \!dashingoff
 
%  **  Dash patterns are specified by a balanced token list whose complete
%  **    expansion has the form: DIMEN1,DIMEN2,DIMEN3,DIMEN4,... ; this produces
%  **    an arc of length DIMEN1, a skip of length DIMEN2, an arc of length
%  **    DIMEN3, a skip of length DIMEN4, ... .  Any number of DIMEN values may
%  **    be given. The pattern is repeated as many times (perhaps fractional)
%  **    as necessary to draw the curve. 
%  **  A dash pattern remains in effect until it is overridden by a call to
%  **    \setdashpattern, or to \setdots, \setdotsnear ... , \setdashes, 
%  **    \setdashesnear ... , or \setsolid.
%  **  Solid lines are the default.
 
 
%  **  \def\setdashpattern <DIMEN1,DIMEN2,DIMEN3,...>
%  **  The following routine converts a balanced list of tokens whose
%  **  complete expansion has the form  DIMEN1,DIMEN2, ... , DIMENk  into 
%  **  three list macros that are used in drawing dashed rules and curves:
%  **    !Flist:   \!Rule{DIMEN1}\!Skip{DIMEN2}\!Rule{DIMEN3}\!Skip{DIMEN4} ...
%  **    !Blist:   ...\!Skip{DIMEN4}\!Rule{DIMEN3}\!Skip{DIMEN2}\!Rule{DIMEN1}
%  **    !UDlist:  \\{DIMEN1}\\{DIMEN2}\\{DIMEN3}\\{DIMEN4} ...;
%  **  calculates \!leaderlength := DIMEN1 + ... + DIMENk; and
%  **  sets the curve drawing routines to dash mode.
%  **  Those lists are used by the curve drawing routines.
%  **  Dimenj ... may be given as an explicit dimension (e.g., 5pt), or
%  **  as an expression involving a dimension register (e.g., -2.5\dimen0).
%  **  See Subsection 6.2 of the manual
\def\setdashpattern <#1>{%
  \def\!Flist{}\def\!Blist{}\def\!UDlist{}%
  \!countA=0
  \!ecfor\!item:=#1\do{%
    \!dimenA=\!item\relax
    \expandafter\!rightappend\the\!dimenA\withCS{\\}\to\!UDlist%
    \advance\!countA  1
    \ifodd\!countA
      \expandafter\!rightappend\the\!dimenA\withCS{\!Rule}\to\!Flist%
      \expandafter\!leftappend\the\!dimenA\withCS{\!Rule}\to\!Blist%
    \else 
      \expandafter\!rightappend\the\!dimenA\withCS{\!Skip}\to\!Flist%
      \expandafter\!leftappend\the\!dimenA\withCS{\!Skip}\to\!Blist%
    \fi}%
  \!leaderlength=\!zpt
  \def\!Rule##1{\advance\!leaderlength  ##1}%
  \def\!Skip##1{\advance\!leaderlength  ##1}%
  \!Flist%
  \ifdim\!leaderlength>\!zpt 
  \else
    \def\!Flist{\!Skip{24in}}\def\!Blist{\!Skip{24in}}\ignorespaces
    \def\!UDlist{\\{\!zpt}\\{24in}}\ignorespaces
    \!leaderlength=24in
  \fi
  \!dashingon}   
 
 
%  **  \!dashingon  -- puts the curve drawing routines into dash mode
%  **  \!dashingoff -- puts the curve drawing routines into solid mode
%  **  These are internal commands, invoked by \setdashpattern and \setsolid
\def\!dashingon{%
  \def\!advancedashing{\!!advancedashing}%
  \def\!drawlinearsegment{\!lineardashed}%
  \def\!puthline{\!putdashedhline}%
  \def\!putvline{\!putdashedvline}%
%  \def\!putsline{\!putdashedsline}%
  \ignorespaces}% 
\def\!dashingoff{%
  \def\!advancedashing{\relax}%
  \def\!drawlinearsegment{\!linearsolid}%
  \def\!puthline{\!putsolidhline}%
  \def\!putvline{\!putsolidvline}%
%  \def\!putsline{\!putsolidsline}%
  \ignorespaces}
 
 
%  **  \setdots <LENGTH>  --  sets up a dot/skip pattern where dot (actually
%  **    the current plotsymbol) is plunked down once for every LENGTH 
%  **    traveled along the curve.  LENGTH defaults to 5pt.
%  **    See Subsection 6.1 of the manual.
\def\setdots{%
  \!ifnextchar<{\!setdots}{\!setdots<5pt>}}
\def\!setdots<#1>{%
  \!dimenB=#1\advance\!dimenB -\plotsymbolspacing
  \ifdim\!dimenB<\!zpt
    \!dimenB=\!zpt
  \fi
\setdashpattern <\plotsymbolspacing,\!dimenB>}
 
% ** \setdotsnear <LENGTH> for <ARC LENGTH>
% ** sets up a dot pattern where the dots are approximately LENGTH apart,
% ** the total length of the pattern is ARC LENGTH, and the pattern
% ** begins and ends with a dot. See Subsection 6.3 of the manual.
\def\setdotsnear <#1> for <#2>{%
  \!dimenB=#2\relax  \advance\!dimenB -.05pt  
  \!dimenC=#1\relax  \!countA=\!dimenC 
  \!dimenD=\!dimenB  \advance\!dimenD .5\!dimenC  \!countB=\!dimenD
  \divide \!countB  \!countA
  \ifnum 1>\!countB 
    \!countB=1
  \fi
  \divide\!dimenB  \!countB
  \setdots <\!dimenB>}
 
%  **  \setdashes <LENGTH>  --  sets up a dash/skip pattern where the dash
%  **    and the skip are each of length LENGTH (the dash is formed by
%  **    plunking down the current plotsymbol over an arc of length LENGTH
%  **    and so may actually be longer than LENGTH.  LENGTH defaults to 5pt.
%  **    See Subsection 6.1 of the manual.
\def\setdashes{%
  \!ifnextchar<{\!setdashes}{\!setdashes<5pt>}}
\def\!setdashes<#1>{\setdashpattern <#1,#1>}
 
% ** \setdashesnear ...
% ** Like \setdotsnear; the pattern begins and ends with a dash.
% ** See Subsection 6.3 of the manual.
\def\setdashesnear <#1> for <#2>{%
  \!dimenB=#2\relax  
  \!dimenC=#1\relax  \!countA=\!dimenC 
  \!dimenD=\!dimenB  \advance\!dimenD .5\!dimenC  \!countB=\!dimenD
  \divide \!countB  \!countA
  \ifodd \!countB 
  \else 
    \advance \!countB  1
  \fi
  \divide\!dimenB  \!countB
  \setdashes <\!dimenB>}
 
%  **  \setsolid  --  puts the curve drawing routines in "solid line" mode,
%  **    the default mode.  See Subsection 6.1 of the manual.
\def\setsolid{%
  \def\!Flist{\!Rule{24in}}\def\!Blist{\!Rule{24in}}%  
  \def\!UDlist{\\{24in}\\{\!zpt}}%
  \!dashingoff}  
\setsolid

%  **  \findlength {CURVE CMDS}
%  **  PiCTeX executes the \start, \ljoin, and \qjoin cmds comprising
%  **  CURVE CMDS without plotting anything, but stashes the length
%  **  of the phantom curve away in \totalarclength.
%  **  See Subsection 6.3 of the manual.
\def\findlength#1{%
  \begingroup
    \setdashpattern <0pt, \maxdimen>
    \setplotsymbol ({})  
    \dontsavelinesandcurves
    #1%
  \endgroup
  \ignorespaces}


% *************************************************************
% *** DIVISION  (Does long division of dimension registers) ***
% *************************************************************
 
% ** User command:
% **   \Divide {DIVIDEND} by {DIVISOR} forming {RESULT}
  
% ** Internal command
% **   \!divide{DIVIDEND}{DIVISOR}{RESULT}
 
% **  \!divide DIVIDEND [by] DIVISOR [to get] ANSWER
% **  Divides the dimension DIVIDEND by the dimension DIVISOR, placing the 
% **  quotient in the dimension register ANSWER.  Values are understood to 
% **  be in points.  E.g.  12.5pt/1.4pt=8.92857pt.
% **  Quotient is accurate to 1/65536pt=2**[-16]pt
% **  |DIVISOR| should be < 2048pt (about 28 inches).
\def\!divide#1#2#3{%
  \!dimenB=#1%                      **  dimB  holds current remainder (r)
  \!dimenC=#2%                      **  dimC  holds divisor (d)
  \!dimenD=\!dimenB%                **  dimD  holds quotient q=r/d for this 
  \divide \!dimenD \!dimenC%        **    step, in units of scaled pts
  \!dimenA=\!dimenD%                **  dimA  eventually holds answer (a)
  \multiply\!dimenD \!dimenC%       **  r <-- r - dq
  \advance\!dimenB -\!dimenD%       **  First step complete. Have integer part
%                                   **  of a, and corresponding remainder.
  \!dimenD=\!dimenC%                **  Temporarily use dimD to hold |d|
    \ifdim\!dimenD<\!zpt \!dimenD=-\!dimenD 
  \fi
  \ifdim\!dimenD<64pt%              **  Branch on the magnitude of |d|
    \!divstep[\!tfs]\!divstep[\!tfs]%
  \else 
    \!!divide
  \fi
  #3=\!dimenA\ignorespaces}

% **  The following code handles divisors  d  with 
% **    (1)  .88in =  64pt <= d <  256pt =  3.54in
% **    (2) 3.54in = 256pt <= d < 2048pt = 28.34in
% **  Anything bigger than that may result in an overflow condition.
% **  For our purposes, we should never even see case (2).
\def\!!divide{%
  \ifdim\!dimenD<256pt
    \!divstep[64]\!divstep[32]\!divstep[32]%
  \else 
    \!divstep[8]\!divstep[8]\!divstep[8]\!divstep[8]\!divstep[8]%
    \!dimenA=2\!dimenA
  \fi}
 
 
% **  The following macro does the real long division work.
\def\!divstep[#1]{%                 **  #1 = "B"
  \!dimenB=#1\!dimenB%              **  r <-- B*r
  \!dimenD=\!dimenB%                **  dimD  holds quotient q=r/d for this 
    \divide \!dimenD by \!dimenC%   **    step, in units of scaled pts
  \!dimenA=#1\!dimenA%              **  a <-- B*a + q
    \advance\!dimenA by \!dimenD%
  \multiply\!dimenD by \!dimenC%    **  r <-- r - dq
    \advance\!dimenB by -\!dimenD}
 
% **  \Divide:  See Subsection 9.3 of the manual.
\def\Divide <#1> by <#2> forming <#3> {%
  \!divide{#1}{#2}{#3}}


% *********************************************
% *** ELLIPSES (Draws ellipses and circles) ***
% *********************************************
 
% ** User commands
% **   \ellipticalarc  axes ratio A:B  DEGREES degrees from XSTART YSTART 
% **      center at XCENTER YCENTER 
% **   \circulararc DEGREES degrees from XSTART YSTART 
% **      center at XCENTER YCENTER 
 
% ** Internal command
% **   \!sinandcos{32*ANGLE in radians}{32*SIN}{32*COS}
 
 
% **   \ellipticalarc  axes ratio A:B  DEGREES degrees from XSTART YSTART 
% **      center at XCENTER YCENTER 
% **    Draws a elliptical arc starting at the coordinate point (XSTART,YSTART).
% **    The center of the ellipse of which the arc is a segment is at 
% **      (XCENTER,YCENTER).
% **    The arc extends through an angle of DEGREES degrees (may be + or -).
% **    A:B is the ratio of the length of the xaxis to the length of
% **      the yaxis of the ellipse
% **    Sqrt{[(XSTART-XCENTER)/A]**2 + [(YSTART-YCENTER)/B]**2}
% **      must be < 512pt (about 7in).
% **    Doesn't modify the dimensions (ht, dp, wd) of the PiCture under
% **      construction.
 
% ** \circulararc  --  See Subsection 5.3 of the manual.
\def\circulararc{%
  \ellipticalarc axes ratio 1:1 }

% ** \ellipticalarc  --  See Subsection 5.3 of the manual.
\def\ellipticalarc axes ratio #1:#2 #3 degrees from #4 #5 center at #6 #7 {%
  \!angle=#3pt\relax%                    ** get angle
  \ifdim\!angle>\!zpt 
    \def\!sign{}%                        ** counterclockwise
  \else 
    \def\!sign{-}\!angle=-\!angle%       ** clockwise
  \fi
  \!xxloc=\!M{#6}\!xunit%                ** convert CENTER to dimension
  \!yyloc=\!M{#7}\!yunit     
  \!xxS=\!M{#4}\!xunit%                  ** get STARTing point on rim of ellipse
  \!yyS=\!M{#5}\!yunit
  \advance\!xxS -\!xxloc%                ** make center of ellipse (0,0)
  \advance\!yyS -\!yyloc
  \!divide\!xxS{#1pt}\!xxS %             ** scale point on ellipse to point on 
  \!divide\!yyS{#2pt}\!yyS %                 corresponding circle
%
  \let\!MC=\!M%                          ** save current c/d mode
  \!setdimenmode%                        ** go into dimension mode
%
  \!xS=#1\!xxS  \advance\!xS\!xxloc
  \!yS=#2\!yyS  \advance\!yS\!yyloc
  \!start (\!xS,\!yS)%
  \!loop\ifdim\!angle>14.9999pt%         ** draw in major portion of ellipse 
    \!rotate(\!xxS,\!yyS)by(\!cos,\!sign\!sin)to(\!xxM,\!yyM) 
    \!rotate(\!xxM,\!yyM)by(\!cos,\!sign\!sin)to(\!xxE,\!yyE)
    \!xM=#1\!xxM  \advance\!xM\!xxloc  \!yM=#2\!yyM  \advance\!yM\!yyloc
    \!xE=#1\!xxE  \advance\!xE\!xxloc  \!yE=#2\!yyE  \advance\!yE\!yyloc
    \!qjoin (\!xM,\!yM) (\!xE,\!yE)
    \!xxS=\!xxE  \!yyS=\!yyE 
    \advance \!angle -15pt
  \repeat
  \ifdim\!angle>\!zpt%                   ** complete remaining arc, if any
    \!angle=100.53096\!angle%            ** convert angle to radians, divide
    \divide \!angle 360 %                **   by 2, and multiply by 32
    \!sinandcos\!angle\!!sin\!!cos%      ** get 32*sin & 32*cos
    \!rotate(\!xxS,\!yyS)by(\!!cos,\!sign\!!sin)to(\!xxM,\!yyM) 
    \!rotate(\!xxM,\!yyM)by(\!!cos,\!sign\!!sin)to(\!xxE,\!yyE)
    \!xM=#1\!xxM  \advance\!xM\!xxloc  \!yM=#2\!yyM  \advance\!yM\!yyloc
    \!xE=#1\!xxE  \advance\!xE\!xxloc  \!yE=#2\!yyE  \advance\!yE\!yyloc
    \!qjoin (\!xM,\!yM) (\!xE,\!yE)
  \fi
%
  \let\!M=\!MC%                          ** restore c/d mode
  \ignorespaces}%                        **   if appropriate
 
 
%  ** \!rotate(XREG,YREG)by(32cos,32sin)to(XXREG,YYREG)
%  ** rotates (XREG,YREG) by angle with specfied scaled cos & sin to
%  ** (XXREG,YYREG).  Uses \!dimenA & \!dimenB as scratch registers.
\def\!rotate(#1,#2)by(#3,#4)to(#5,#6){% 
  \!dimenA=#3#1\advance \!dimenA -#4#2%   ** Rcos(x+t)=Rcosx*cost - Rsinx*sint
  \!dimenB=#3#2\advance \!dimenB  #4#1%   ** Rsin(x+t)=Rsinx*cost + Rcosx*sint
  \divide \!dimenA 32  \divide \!dimenB 32 
  #5=\!dimenA  #6=\!dimenB
  \ignorespaces}
\def\!sin{4.17684}%                       ** 32*sin(pi/24) (pi/24=7.5deg)
\def\!cos{31.72624}%                      ** 32*cos(pi/24)
 
 
%  ** \!sinandcos{32*ANGLE in radians}{\SINCS}{\COSCS}
%  **   Computes the 32*sine and 32*cosine of a small ANGLE expressed in 
%  **   radians/32 and puts these values in the replacement texts of 
%  **   \SINCS and \COSCS
\def\!sinandcos#1#2#3{%
 \!dimenD=#1%                **  angle is expressed in radians/32: 1pt = 1/32rad
 \!dimenA=\!dimenD%          **  dimA will eventually contain 32sin(angle)in pts
 \!dimenB=32pt%              **  dimB will eventually contain 32cos(angle)in pts
 \!removept\!dimenD\!value%  **  get value of 32*angle, without "pt"
 \!dimenC=\!dimenD%          **  holds 32*angle**i/i! in pts
 \!dimenC=\!value\!dimenC \divide\!dimenC by 64 %   ** now 32*angle**2/2
 \advance\!dimenB by -\!dimenC%                     ** 32-32*angle**2/2
 \!dimenC=\!value\!dimenC \divide\!dimenC by 96 %   ** now 32*angle**3/3!
 \advance\!dimenA by -\!dimenC%                     ** now 32*(angle-angle**3/6)
 \!dimenC=\!value\!dimenC \divide\!dimenC by 128 %  ** now 32*angle**4/4!
 \advance\!dimenB by \!dimenC%
 \!removept\!dimenA#2%                              ** set 32*sin(angle)
 \!removept\!dimenB#3%                              ** set 32*cos(angle)
 \ignorespaces}


% *****************************************************************
% ***  RULES  (Draws rules, i.e., horizontal & vertical lines)  ***
% *****************************************************************

% **  User command:
% **    \putrule [<XDIMEN,YDIMEN>] from  XCOORD1 YCOORD1 
% **      to  XCOORD2 YCOORD2 

% **  Internal commands:
% **    \!puthline [<XDIMEN,YDIMEN>]    (h = horizontal)
% **      Set by dashpat to either: \!putsolidhline  or \!putdashedhline
% **    \!putvline [<XDIMEN,YDIMEN>]    (v = vertical)
% **      Either:  \!putsolidvline  or  \!putdashedvline


% **  \putrule [<XDIMEN,YDIMEN>] from XCOORD1 YCOORD1
% **    to XCOORD2 YCOORD2
% **  Draws a rule -- dashed or solid depending on the current dash pattern --
% **    from (X1,Y1) to (X2,Y2).  Uses TEK's  \hrule & \vrule & \leaders  
% **    constructions to handle horizontal & vertical lines efficiently both
% **    in terms of execution time and space in the DVI file.  
% **  See Subsection 4.1 of the manual.
\def\putrule#1from #2 #3 to #4 #5 {%
  \!xloc=\!M{#2}\!xunit  \!xxloc=\!M{#4}\!xunit%   
  \!yloc=\!M{#3}\!yunit  \!yyloc=\!M{#5}\!yunit%           
  \!dxpos=\!xxloc  \advance\!dxpos by -\!xloc
  \!dypos=\!yyloc  \advance\!dypos by -\!yloc
%
  \ifdim\!dypos=\!zpt
    \def\!!Line{\!puthline{#1}}\ignorespaces
  \else
    \ifdim\!dxpos=\!zpt
      \def\!!Line{\!putvline{#1}}\ignorespaces
    \else 
       \def\!!Line{}
    \fi
  \fi
  \let\!ML=\!M%           ** save current coord\dimen mode
  \!setdimenmode%         ** express locations in dimens
  \!!Line%
  \let\!M=\!ML%           ** restore previous c/d mode
  \ignorespaces}
 
 
% **  \!putsolidhline [<XDIMEN,YDIMEN>]
% **  Place horizontal solid line
\def\!putsolidhline#1{%
  \ifdim\!dxpos>\!zpt 
    \put{\!hline\!dxpos}#1[l] at {\!xloc} {\!yloc}
  \else 
    \put{\!hline{-\!dxpos}}#1[l] at {\!xxloc} {\!yyloc}
  \fi
  \ignorespaces}
 
% **  \!putsolidvline [shifted <XDIMEN,YDIMEN>]
% **  Place vertical solid line
\def\!putsolidvline#1{%
  \ifdim\!dypos>\!zpt 
    \put{\!vline\!dypos}#1[b] at {\!xloc} {\!yloc}
  \else 
    \put{\!vline{-\!dypos}}#1[b] at {\!xxloc} {\!yyloc}
  \fi
  \ignorespaces}
 
\def\!hline#1{\hbox to #1{\leaders \hrule height\linethickness\hfill}}
\def\!vline#1{\vbox to #1{\leaders \vrule width\linethickness\vfill}}
 
 
% **  \!putdashedhline [<XDIMEN,YDIMEN>]
% **  Place dashed horizontal line
\def\!putdashedhline#1{%
  \ifdim\!dxpos>\!zpt 
    \!DLsetup\!Flist\!dxpos
    \put{\hbox to \!totalleaderlength{\!hleaders}\!hpartialpattern\!Rtrunc}
      #1[l] at {\!xloc} {\!yloc} 
  \else 
    \!DLsetup\!Blist{-\!dxpos}
    \put{\!hpartialpattern\!Ltrunc\hbox to \!totalleaderlength{\!hleaders}}
      #1[r] at {\!xloc} {\!yloc} 
  \fi
  \ignorespaces}
 
% **  \!putdashedhline [<XDIMEN,YDIMEN>]
% **  Place dashed vertical line
\def\!putdashedvline#1{%
  \!dypos=-\!dypos%            ** vertical leaders go from top to bottom
  \ifdim\!dypos>\!zpt 
    \!DLsetup\!Flist\!dypos 
    \put{\vbox{\vbox to \!totalleaderlength{\!vleaders}
      \!vpartialpattern\!Rtrunc}}#1[t] at {\!xloc} {\!yloc} 
  \else 
    \!DLsetup\!Blist{-\!dypos}
    \put{\vbox{\!vpartialpattern\!Ltrunc
      \vbox to \!totalleaderlength{\!vleaders}}}#1[b] at {\!xloc} {\!yloc} 
  \fi
  \ignorespaces}
 

% **  The rest of the macros in this section are subroutines used by 
% **  \!putdashedhline and \!putdashedvline.
\def\!DLsetup#1#2{%            ** Dashed-Line set up
  \let\!RSlist=#1%             ** set !Rule-Skip list
  \!countB=#2%                 ** convert rule length to integer (number of sps)
  \!countA=\!leaderlength%     ** ditto, leaderlength
  \divide\!countB by \!countA% ** number of complete leader units
  \!totalleaderlength=\!countB\!leaderlength
  \!Rresiduallength=#2%
  \advance \!Rresiduallength by -\!totalleaderlength%  \** excess length
  \!Lresiduallength=\!leaderlength
  \advance \!Lresiduallength by -\!Rresiduallength
  \ignorespaces}
 
\def\!hleaders{%
  \def\!Rule##1{\vrule height\linethickness width##1}%
  \def\!Skip##1{\hskip##1}%
  \leaders\hbox{\!RSlist}\hfill}
 
\def\!hpartialpattern#1{%
  \!dimenA=\!zpt \!dimenB=\!zpt 
  \def\!Rule##1{#1{##1}\vrule height\linethickness width\!dimenD}%
  \def\!Skip##1{#1{##1}\hskip\!dimenD}%
  \!RSlist}
 
\def\!vleaders{%
  \def\!Rule##1{\hrule width\linethickness height##1}%
  \def\!Skip##1{\vskip##1}%
  \leaders\vbox{\!RSlist}\vfill}
 
\def\!vpartialpattern#1{%
  \!dimenA=\!zpt \!dimenB=\!zpt 
  \def\!Rule##1{#1{##1}\hrule width\linethickness height\!dimenD}%
  \def\!Skip##1{#1{##1}\vskip\!dimenD}%
  \!RSlist}
 
\def\!Rtrunc#1{\!trunc{#1}>\!Rresiduallength}
\def\!Ltrunc#1{\!trunc{#1}<\!Lresiduallength}
 
\def\!trunc#1#2#3{%          
  \!dimenA=\!dimenB         
  \advance\!dimenB by #1%
  \!dimenD=\!dimenB  \ifdim\!dimenD#2#3\!dimenD=#3\fi
  \!dimenC=\!dimenA  \ifdim\!dimenC#2#3\!dimenC=#3\fi
  \advance \!dimenD by -\!dimenC}
 

%  ****************************************************************
%  ***  LINEAR ARC  (Draws straight lines -- solid and dashed)  ***
%  ****************************************************************
 
%  **  User commands
%  **    \inboundscheckoff
%  **    \inboundscheckon
   
%  **  Internal commands 
%  **    \!start (XCOORD,YCOORD)
%  **    \!ljoin (XCOORD,YCOORD)
%  **    \!drawlinearsegment  --  set by \dashpat to either
%  **      \!linearsolid  or  \!lineardashed
%  **    \!advancedashing     --  set by \dashpat to either
%  **       \relax  or  \!!advancedashing
%  **    \!plotifinbounds     --  set by \inboundscheck off/on to either
%  **       \!plot  or  \!!plotifinbounds
%  **    \!initinboundscheck  --  set by \inboundscheck off/on to either
%  **       \relax  or  \!!initinboundscheck
 
 
%  \plotsymbolspacing  ** distance between consecutive plot positions
%  \!xS                ** starting x
%  \!yS                ** starting y
%  \!xE                ** ending   x
%  \!yE                ** ending   y
%  \!xdiff             ** x_end - x_start
%  \!ydiff             ** y_end - y_start
%  \!distacross        ** how far along curve next point to be plotted is
%  \!arclength         ** approximate length of arc for current interval
%  \!downlength        ** remaining length for "pen" to be down    
%  \!uplength          ** length for "pen" to be down    
%  \!intervalno        ** counts segments to curve
%  \totalarclength     ** cumulative distance along curve
%  \!npoints           ** approximately  (arc length / plotsymbolspacing)
 
%  **  Calls -- \!Pythag, \!divide, \!plot
 
 
%  **  \!start (XCOORD,YCOORD)
%  **  Sets initial point for linearly (or quadratically) interpolated curve
\def\!start (#1,#2){%
  \!plotxorigin=\!xorigin  \advance \!plotxorigin by \!plotsymbolxshift
  \!plotyorigin=\!yorigin  \advance \!plotyorigin by \!plotsymbolyshift
  \!xS=\!M{#1}\!xunit \!yS=\!M{#2}\!yunit
  \!rotateaboutpivot\!xS\!yS
  \!copylist\!UDlist\to\!!UDlist% **\!UDlist has the form \\{dimen1}\\{dimen2}..
%                                 ** Routine will draw dashed line with pen
%                                 ** down for dimen1, up for dimen2, ...
  \!getnextvalueof\!downlength\from\!!UDlist
  \!distacross=\!zpt%             ** 1st point goes at start of curve
  \!intervalno=0 %                ** initialize interval counter
  \global\totalarclength=\!zpt%   ** initialize distance traveled along curve
  \ignorespaces}
 

%  **  \!ljoin (XCOORD,YCOORD) 
%  **  Draws a straight line starting at the last point specified
%  **    by the most recent \!start, \!ljoin, or \!qjoin, and
%  **    ending at (XCOORD,YCOORD).
\def\!ljoin (#1,#2){%
  \advance\!intervalno by 1
  \!xE=\!M{#1}\!xunit \!yE=\!M{#2}\!yunit
  \!rotateaboutpivot\!xE\!yE
  \!xdiff=\!xE \advance \!xdiff by -\!xS%**  xdiff = xE - xS
  \!ydiff=\!yE \advance \!ydiff by -\!yS%**  ydiff = yE - yS
  \!Pythag\!xdiff\!ydiff\!arclength%     **  arclength = sqrt(xdiff**2+ydiff**2) 
  \global\advance \totalarclength by \!arclength%
  \!drawlinearsegment%   ** set by dashpat to \!linearsolid or \!lineardashed
  \!xS=\!xE \!yS=\!yE%   ** shift ending points to starting points
  \ignorespaces}
 
 
% **  The following routine is used to draw a "solid" line between (xS,yS)
% **  and (xE,yE).  Points are spaced nearly every  \plotsymbolspacing length
% **  along the line.  
\def\!linearsolid{%
  \!npoints=\!arclength
  \!countA=\plotsymbolspacing
  \divide\!npoints by \!countA%      ** now #pts =. arclength/plotsymbolspacing
  \ifnum \!npoints<1 
    \!npoints=1 
  \fi
  \divide\!xdiff by \!npoints
  \divide\!ydiff by \!npoints
  \!xpos=\!xS \!ypos=\!yS
%
  \loop\ifnum\!npoints>-1
    \!plotifinbounds
    \advance \!xpos by \!xdiff
    \advance \!ypos by \!ydiff
    \advance \!npoints by -1
  \repeat
  \ignorespaces}
 
 
% ** The following routine is used to draw a dashed line between (xS,yS)
% ** and (xE,yE). The dash pattern continues from the previous segment.
\def\!lineardashed{%
% **
  \ifdim\!distacross>\!arclength
    \advance \!distacross by -\!arclength  %nothing to plot in this interval
%
  \else
%
    \loop\ifdim\!distacross<\!arclength
%     ** plot point, interpolating linearly in x and y
      \!divide\!distacross\!arclength\!dimenA%  ** dimA = across/arclength
      \!removept\!dimenA\!t%  ** \!t holds value in dimA, without the "pt"
      \!xpos=\!t\!xdiff \advance \!xpos by \!xS
      \!ypos=\!t\!ydiff \advance \!ypos by \!yS
      \!plotifinbounds
      \advance\!distacross by \plotsymbolspacing
      \!advancedashing
    \repeat  
%
    \advance \!distacross by -\!arclength%    ** prepare for next interval 
  \fi
  \ignorespaces}
 
 
\def\!!advancedashing{%
  \advance\!downlength by -\plotsymbolspacing
  \ifdim \!downlength>\!zpt
  \else
    \advance\!distacross by \!downlength
    \!getnextvalueof\!uplength\from\!!UDlist
    \advance\!distacross by \!uplength
    \!getnextvalueof\!downlength\from\!!UDlist
  \fi}
 
 
% ** \inboundscheckoff & \inboundscheckon: See Subsection 5.5 of the manual.
\def\inboundscheckoff{%
  \def\!plotifinbounds{\!plot(\!xpos,\!ypos)}%
  \def\!initinboundscheck{\relax}\ignorespaces}
\def\inboundscheckon{%
  \def\!plotifinbounds{\!!plotifinbounds}%
  \def\!initinboundscheck{\!!initinboundscheck}%
  \!initinboundscheck\ignorespaces} 
\inboundscheckoff
 
% ** The following code plots the current point only if it falls in the
% ** current plotarea.  It doesn't matter if the coordinate system has
% ** changed since the plotarea was set up.  However, shifts of the plot
% ** are ignored (how the plotsymbol stands relative to its plot position is
% ** unknown anyway).
\def\!!plotifinbounds{%
  \ifdim \!xpos<\!checkleft
  \else
    \ifdim \!xpos>\!checkright
    \else
      \ifdim \!ypos<\!checkbot
      \else
         \ifdim \!ypos>\!checktop
         \else
           \!plot(\!xpos,\!ypos)
         \fi 
      \fi
    \fi
  \fi}
 
 
\def\!!initinboundscheck{%
  \!checkleft=\!arealloc     \advance\!checkleft by \!xorigin
  \!checkright=\!arearloc    \advance\!checkright by \!xorigin
  \!checkbot=\!areabloc      \advance\!checkbot by \!yorigin
  \!checktop=\!areatloc      \advance\!checktop by \!yorigin}
 

% *********************************
% *** LOGTEN  (Log_10 function) ***
% *********************************
%
% ** \!logten{X}
% ** Calculates log_10 of X.  X and LOG10(X) are in fixed point notation.
% **  X must be positive; it may have an optional `+' sign; any number
% **  of digits may be specified for X.  The absolute error in LOG10(X) is
% **  less than .0001 (probably < .00006).  That's about as good as you
% **  hope for, since TEX only operates to 5 figures after the decimal
% **  point anyway.

%  \!rootten=3.162278pt       **** These are values are set in ALLOCATIONS
%  \!tenAe=2.543275pt  (=A5)
%  \!tenAc=2.773839pt  (=A3)
%  \!tenAa=8.690286pt  (=A1)

\def\!logten#1#2{%
  \expandafter\!!logten#1\!nil
  \!removept\!dimenF#2%
  \ignorespaces}

\def\!!logten#1#2\!nil{%
  \if -#1%
    \!dimenF=\!zpt
    \def\!next{\ignorespaces}%
  \else
    \if +#1%
      \def\!next{\!!logten#2\!nil}%
    \else
      \if .#1%
        \def\!next{\!!logten0.#2\!nil}%
      \else
        \def\!next{\!!!logten#1#2..\!nil}%
      \fi
    \fi
  \fi
  \!next}

\def\!!!logten#1#2.#3.#4\!nil{%
  \!dimenF=1pt %                 ** DimF holds log10 original argument
  \if 0#1%                      
    \!!logshift#3pt %            ** Argument < 1
  \else %                        ** Argument >= 1
    \!logshift#2/%               ** Shift decimal pt as many places
    \!dimenE=#1.#2#3pt %         **   as there are figures in #2
  \fi %                          ** Now dimE holds revised X want log10 of
  \ifdim \!dimenE<\!rootten%          ** Transform X to XX between sqrt(10) 
    \multiply \!dimenE 10 %           **   and 10*sqrt(10)
    \advance  \!dimenF -1pt
  \fi
  \!dimenG=\!dimenE%                  ** dimG <- (XX + 10)
    \advance\!dimenG 10pt
  \advance\!dimenE -10pt %            ** dimE <- (XX - 10)
  \multiply\!dimenE 10 %              ** dimE = 10*(XX-10)
  \!divide\!dimenE\!dimenG\!dimenE%   ** Now dimE=10t==10*(XX-10)/(XX+10)
  \!removept\!dimenE\!t%              ** !t=10t, with "pt" removed
  \!dimenG=\!t\!dimenE%               ** dimG=100t**2
  \!removept\!dimenG\!tt%             ** !tt=100t**2, with "pt" removed
  \!dimenH=\!tt\!tenAe%               ** dimH=10*a5*(10t)**2 /100
    \divide\!dimenH 100
  \advance\!dimenH \!tenAc%           ** ditto + 10*a3
  \!dimenH=\!tt\!dimenH%              ** ditto * (10t)**2 /100
    \divide\!dimenH 100   
  \advance\!dimenH \!tenAa%           ** ditto + 10*a1
  \!dimenH=\!t\!dimenH%               ** ditto * 10t / 100
    \divide\!dimenH 100 %             ** Now dimH = log10(XX) - 1
  \advance\!dimenF \!dimenH}%         ** dimF = log10(X)

\def\!logshift#1{%
  \if #1/%
    \def\!next{\ignorespaces}%
  \else
    \advance\!dimenF 1pt 
    \def\!next{\!logshift}%
  \fi 
  \!next}
 
 \def\!!logshift#1{%
   \advance\!dimenF -1pt
   \if 0#1%
     \def\!next{\!!logshift}%
   \else
     \if p#1%
       \!dimenF=1pt
       \def\!next{\!dimenE=1p}%
     \else
       \def\!next{\!dimenE=#1.}%
     \fi
   \fi
   \!next}


% ***********************************************************
% *** PICTURES (Basic setups for PiCtures; \put commands) ***
% ***********************************************************
 
% **  User Commands:
% **    \beginpicture
% **    \endpicture    
% **    \endpicturesave <XREG,YREG>
% **    \setcoordinatesystem units <XUNIT,YUNIT> point at XREF YREF
% **    \put {OBJECT} [ORIENTATION] <XSHIFT,YSHIFT> at XCOORD YCOORD
% **    \multiput {OJBECT} [ORIENTATION] <XSHIFT,YSHIFT>) at
% **      XCOORD YCOORD
% **      *NUMBER_OF_TIMES DXCOORD DYCOORD  /
% **    \accountingon
% **    \accountingoff
% **    \stack [ORIENTATION] <LEADING> {LIST OF ITEMS}
% **    \lines [ORIENTATION] {LINES}
% **    \Lines [ORIENTATION] {LINES}
% **    \setdimensionmode
% **    \setcoordinatemode
% **    \Xdistance
% **    \Ydistance
  
% **  Internal commands:
% **    \!setputobject{OBJECT}{[ORIENTATION]<XSHIFT,YSHIFT>}
% **    \!dimenput{OBJECT}[ORIENTATION]<XSHIFT,YSHIFT>(XDIMEN,YDIMEN)
% **    \!setdimenmode
% **    \!setcoordmode
% **    \!ifdimenmode
% **    \!ifcoordmode
  
 
% **  \beginpicture
% **  \endpicture
% **  \endpicturesave <XREG,YREG>
% **    \beginpicture ... \endpicture  creates an hbox.  Objects are
% **    placed in this box using the \put command and the like (see below).
% **    The location of an object is specified in terms of coordinate system(s)
% **    established by \setcoordinatesystem.  Each coordinate system (there
% **    might be just one) specifies the length of 1 horizontal unit, the length
% **    of 1 vertical unit, and the coordinates of a "reference point".  The
% **    reference points of various coordinate systems will be in the same
% **    physical location.  The macros keep track of the size of the objects
% **    and their locations. The resulting hbox is the smallest hbox which
% **    encloses all the objects, and whose TEK reference point is the point
% **    on the left edge of the box closest vertically to the PICTEX reference
% **    point. Using \endpicturesave, you can (globally) save the distance TEK's
% **    reference point is to the right (respectively, up from) PICTEX's 
% **    reference point in the dimension register \XREG (respectively \YREG). 
% **    You can then \put the picture OBJECT into a larger picture so that its
% **    reference point is at (XCOORD,YCOORD) with the command
% **      \put {picture OBJECT} [Bl] <\XREG, \YREG> at  XCOORD YCOORD 

% **  \beginpicture : See Subsection 1.1 of the manual.
\def\beginpicture{%
  \setbox\!picbox=\hbox\bgroup%
  \!xleft=\maxdimen  
  \!xright=-\maxdimen
  \!ybot=\maxdimen
  \!ytop=-\maxdimen}
 
% **  \endpicture : See Subsection 1.1 of the manual.
\def\endpicture{%
  \ifdim\!xleft=\maxdimen%  ** check if nothing was put in picbox
    \!xleft=\!zpt \!xright=\!zpt \!ybot=\!zpt \!ytop=\!zpt 
  \fi
  \global\!Xleft=\!xleft \global\!Xright=\!xright
  \global\!Ybot=\!ybot \global\!Ytop=\!ytop
  \egroup%
  \ht\!picbox=\!Ytop  \dp\!picbox=-\!Ybot
  \ifdim\!Ybot>\!zpt
  \else 
    \ifdim\!Ytop<\!zpt
      \!Ybot=\!Ytop
    \else
      \!Ybot=\!zpt
    \fi
  \fi
  \hbox{\kern-\!Xleft\lower\!Ybot\box\!picbox\kern\!Xright}}
 
% **  \endpicturesave : See Subsection 8.4 of the manual.
\def\endpicturesave <#1,#2>{%
  \endpicture \global #1=\!Xleft \global #2=\!Ybot \ignorespaces}
 
 
% **   \setcoordinatesystem units <XUNIT,YUNIT> 
% **     point at XREF YREF  
% **   Each of `units <XUNIT,YUNIT>' and `point at XREF YREF' 
% **     are optional.
% **   Unit lengths must be given in dimensions (e.g., <10pt,1in>).
% **     Default unit lengths are 1pt, 1pt, or previous unit lengths.
% **   Reference point is specified in current units (e.g., 3 5 ). 
% **     Default reference point is 0 0 , or previous reference point.
% **   Unit lengths and reference points obey TEX's scoping rules.
% **   See Subsection 1.2 of the manual.
\def\setcoordinatesystem{%
  \!ifnextchar{u}{\!getlengths }
    {\!getlengths units <\!xunit,\!yunit>}}
\def\!getlengths units <#1,#2>{%
  \!xunit=#1\relax
  \!yunit=#2\relax
  \!ifcoordmode 
    \let\!SCnext=\!SCccheckforRP
  \else
    \let\!SCnext=\!SCdcheckforRP
  \fi
  \!SCnext}
\def\!SCccheckforRP{%
  \!ifnextchar{p}{\!cgetreference }
    {\!cgetreference point at {\!xref} {\!yref} }}
\def\!cgetreference point at #1 #2 {%
  \edef\!xref{#1}\edef\!yref{#2}%
  \!xorigin=\!xref\!xunit  \!yorigin=\!yref\!yunit  
  \!initinboundscheck % ** See linear.tex
  \ignorespaces}
\def\!SCdcheckforRP{%
  \!ifnextchar{p}{\!dgetreference}%
    {\ignorespaces}}
\def\!dgetreference point at #1 #2 {%
  \!xorigin=#1\relax  \!yorigin=#2\relax
  \ignorespaces}

 
%  ** \put {OBJECT} [XY] <XDIMEN,YDIMEN> at (XCOORD,YCOORD)
%  **   `[XY]' and `<XDIMEN,YDIMEN>' are optional.
%  **   First OBJECT is placed in an hbox (the "objectbox") and then a
%  **     "reference point" is assigned to the objectbox as follows:
%  **     [1] first, the reference point is taken to be the center of the box;
%  **     [2] next, centering is overridden by the specifications
%  **           X=l -- reference point along the left edge of the objectbox
%  **           X=r -- reference point along the right edge of the objectbox
%  **           Y=b -- reference point along the bottom edge of the objectbox
%  **           Y=B -- reference point along the Baseline of the objectbox
%  **           Y=t -- reference point along the top edge of the objectbox;
%  **     [3] finally the reference point is shifted left by XDIMEN, down
%  **           by YDIMEN  (both default to 0pt).
%  **   The objectbox is placed within PICBOX with its reference point at  
%  **     (XCOORD,YCOORD). 
%  **   If OBJECT is a saved box, say  box0, you have to write
%  **     \put{\box0}...   or  \put{\copy0}...
%  **   The objectbox is void after the put.
%  **   See Subsection 2.1 of the manual.
\long\def\put#1#2 at #3 #4 {%
  \!setputobject{#1}{#2}%
  \!xpos=\!M{#3}\!xunit  \!ypos=\!M{#4}\!yunit  
  \!rotateaboutpivot\!xpos\!ypos%
  \advance\!xpos -\!xorigin  \advance\!xpos -\!xshift
  \advance\!ypos -\!yorigin  \advance\!ypos -\!yshift
  \kern\!xpos\raise\!ypos\box\!putobject\kern-\!xpos%
  \!doaccounting\ignorespaces}
 
%  **   \multiput etc.  Like  \put.  The objectbox is not voided until the
%  **     termininating /, and is placed repeatedly with:
%  **     XCOORD YCOORD -- the objectbox is put down with its reference point
%  **       at (XCOORD,YCOORD);
%  **     *N DXCOORD DYCOORD -- each of N times the current
%  **       (xcoord,ycoord) is incremented by (DXCOORD,DYCOORD), and the
%  **       objectbox is put down with its reference point at (xcoord,ycoord)
%  **       (This specification has to follow an XCOORD YCOORD pair)
%  **     See Subsection 2.2 of the manual.
\long\def\multiput #1#2 at {%
  \!setputobject{#1}{#2}%
  \!ifnextchar"{\!putfromfile}{\!multiput}}
\def\!putfromfile"#1"{%
  \expandafter\!multiput \input #1 /}
\def\!multiput{%
  \futurelet\!nextchar\!!multiput}
\def\!!multiput{%
  \if *\!nextchar
    \def\!nextput{\!alsoby}%
  \else
    \if /\!nextchar
      \def\!nextput{\!finishmultiput}%
    \else
      \def\!nextput{\!alsoat}%
    \fi
  \fi
  \!nextput}
\def\!finishmultiput/{%
  \setbox\!putobject=\hbox{}%
  \ignorespaces}
 
%  **   \!alsoat XCOORD YCOORD 
%  **     The objectbox is put down with reference point at XCOORD,YCOORD
\def\!alsoat#1 #2 {%
  \!xpos=\!M{#1}\!xunit  \!ypos=\!M{#2}\!yunit  
  \!rotateaboutpivot\!xpos\!ypos%
  \advance\!xpos -\!xorigin  \advance\!xpos -\!xshift
  \advance\!ypos -\!yorigin  \advance\!ypos -\!yshift
  \kern\!xpos\raise\!ypos\copy\!putobject\kern-\!xpos%
  \!doaccounting
  \!multiput}
 
% **   \!alsoby*N DXCOORD DYCOORD
% **     N times, the current (XCOORD,YCOORD) is advanced by (DXCOORD,DYCOORD),
% **     and the current (shifted, oriented) OBJECT is put down.
\def\!alsoby*#1 #2 #3 {%
  \!dxpos=\!M{#2}\!xunit \!dypos=\!M{#3}\!yunit 
  \!rotateonly\!dxpos\!dypos
  \!ntemp=#1%
  \!!loop\ifnum\!ntemp>0
    \advance\!xpos by \!dxpos  \advance\!ypos by \!dypos
    \kern\!xpos\raise\!ypos\copy\!putobject\kern-\!xpos%
    \advance\!ntemp by -1
  \repeat
  \!doaccounting 
  \!multiput}
 
% **  \accountingoff : Suspends PiCTeX's accounting of the aggregate
% **    size of the picture box.
% **  \accounting on : Reinstates accounting.
% **  See Subsection 8.2 of the manual.
\def\accountingon{\def\!doaccounting{\!!doaccounting}\ignorespaces}
\def\accountingoff{\def\!doaccounting{}\ignorespaces}
\accountingon
\def\!!doaccounting{%
  \!xtemp=\!xpos  
  \!ytemp=\!ypos
  \ifdim\!xtemp<\!xleft 
     \!xleft=\!xtemp 
  \fi
  \advance\!xtemp by  \!wd 
  \ifdim\!xright<\!xtemp 
    \!xright=\!xtemp
  \fi
  \advance\!ytemp by -\!dp
  \ifdim\!ytemp<\!ybot  
    \!ybot=\!ytemp
  \fi
  \advance\!ytemp by  \!dp
  \advance\!ytemp by  \!ht 
  \ifdim\!ytemp>\!ytop  
    \!ytop=\!ytemp  
  \fi}
 
\long\def\!setputobject#1#2{%
  \setbox\!putobject=\hbox{#1}%
  \!ht=\ht\!putobject  \!dp=\dp\!putobject  \!wd=\wd\!putobject
  \wd\!putobject=\!zpt
  \!xshift=.5\!wd   \!yshift=.5\!ht   \advance\!yshift by -.5\!dp
  \edef\!putorientation{#2}%
  \expandafter\!SPOreadA\!putorientation[]\!nil%
  \expandafter\!SPOreadB\!putorientation<\!zpt,\!zpt>\!nil\ignorespaces}
 
\def\!SPOreadA#1[#2]#3\!nil{\!etfor\!orientation:=#2\do\!SPOreviseshift}
 
\def\!SPOreadB#1<#2,#3>#4\!nil{\advance\!xshift by -#2\advance\!yshift by -#3}
 
\def\!SPOreviseshift{%
  \if l\!orientation 
    \!xshift=\!zpt
  \else 
    \if r\!orientation 
      \!xshift=\!wd
    \else 
      \if b\!orientation
        \!yshift=-\!dp
      \else 
        \if B\!orientation 
          \!yshift=\!zpt
        \else 
          \if t\!orientation 
            \!yshift=\!ht
          \fi 
        \fi
      \fi
    \fi
  \fi}
 
 
%  **  \!dimenput{OBJECT} <XDIMEN,YDIMEN> [XY] (XLOC,YLOC)
%  **    This is an internal put routine, similar to \put, except that
%  **    XLOC=distance right from reference point, YLOC=distance up from
%  **    reference point. XLOC and YLOC are dimensions, so this routine
%  **    is completely independent of the current coordinate system. 
%  **    This routine does NOT do ROTATIONS.
\long\def\!dimenput#1#2(#3,#4){%
  \!setputobject{#1}{#2}%
  \!xpos=#3\advance\!xpos by -\!xshift
  \!ypos=#4\advance\!ypos by -\!yshift
  \kern\!xpos\raise\!ypos\box\!putobject\kern-\!xpos%
  \!doaccounting\ignorespaces}
 
 
%  ** The following macros permit the picture drawing routines to be used 
%  ** either in the default "coordinate mode", or in "dimension mode".
%  **   In coordinate mode  \!M(1.5,\!xunit)    expands to  1.5\!xunit
%  **   In dimension  mode  \!M(1.5pt,\!xunit)  expands to  1.5pt
%  ** Dimension mode is useful in coding macros.
%  ** Any special purpose picture macro that sets dimension mode should 
%  ** reset coordinate mode before completion.
%  ** See Subsection 9.2 of the manual.
\def\!setdimenmode{%
  \let\!M=\!M!!\ignorespaces}
\def\!setcoordmode{%
  \let\!M=\!M!\ignorespaces}
\def\!ifcoordmode{%
  \ifx \!M \!M!}
\def\!ifdimenmode{%
  \ifx \!M \!M!!}
\def\!M!#1#2{#1#2} 
\def\!M!!#1#2{#1}
\!setcoordmode
\let\setdimensionmode=\!setdimenmode
\let\setcoordinatemode=\!setcoordmode

%  ** \Xdistance{XCOORD}, \Ydistance{YCOORD}  are the horizontal and
%  **   vertical distances from the origin (0,0) to the point
%  **   (XCOORD,YCOORD)  in the current coordinate system.
%  ** See Subsection 9.2 of the manual.
\def\Xdistance#1{%
  \!M{#1}\!xunit
  \ignorespaces}
\def\Ydistance#1{%
  \!M{#1}\!yunit
  \ignorespaces}

% ** The following macros -- \stack, \line, and \Lines -- are useful for
% **   annotating PiCtures. They can be used outside the \beginpicture ...
% **   \endpicture environment.

% ** \stack [POSITIONING] <LEADING> {VALUESLIST}
% ** Builds a vertical stack of the values in VALUESLIST. Values in
% ** VALUESLIST are separated by commas.  In the resulting stack, values are
% ** centered by default, and positioned flush left (right) if 
% ** POSITIONING = l (r).  Values are separated vertically by LEADING,
% ** which defaults to \stackleading.
% ** See Subsection 2.3 of the manual.
\def\stack{%
  \!ifnextchar[{\!stack}{\!stack[c]}}
\def\!stack[#1]{%
  \let\!lglue=\hfill \let\!rglue=\hfill
  \expandafter\let\csname !#1glue\endcsname=\relax
  \!ifnextchar<{\!!stack}{\!!stack<\stackleading>}}
\def\!!stack<#1>#2{%
  \vbox{\def\!valueslist{}\!ecfor\!value:=#2\do{%
    \expandafter\!rightappend\!value\withCS{\\}\to\!valueslist}%
    \!lop\!valueslist\to\!value
    \let\\=\cr\lineskiplimit=\maxdimen\lineskip=#1%
    \baselineskip=-1000pt\halign{\!lglue##\!rglue\cr \!value\!valueslist\cr}}%
  \ignorespaces}

% ** \lines [POSITIONING] {LINES}
% ** Builds a vertical array of the lines in LINES. Each line in LINES
% ** is terminated by a \cr.  In the resulting array, lines are
% ** centered by default, and positioned flush left (right) if 
% ** POSITIONING = l (r).  The lines in the array are subject to TeX's
% ** usual spacing rules: in particular the baselines are ordinarily an equal
% ** distance apart. The baseline of the array is the baseline of the
% ** the bottom line.
% ** See Subsection 2.3 of the manual.
\def\lines{%
  \!ifnextchar[{\!lines}{\!lines[c]}}
\def\!lines[#1]#2{%
  \let\!lglue=\hfill \let\!rglue=\hfill
  \expandafter\let\csname !#1glue\endcsname=\relax
  \vbox{\halign{\!lglue##\!rglue\cr #2\crcr}}%
  \ignorespaces}

% ** \Lines [POSITIONING] {LINES}
% ** Like \lines, but the baseline of the array is the baseline of the
% ** top line.  See Subsection 2.3 of the manual.
\def\Lines{%
  \!ifnextchar[{\!Lines}{\!Lines[c]}}
\def\!Lines[#1]#2{%
  \let\!lglue=\hfill \let\!rglue=\hfill
  \expandafter\let\csname !#1glue\endcsname=\relax
  \vtop{\halign{\!lglue##\!rglue\cr #2\crcr}}%
  \ignorespaces}


% *********************************************
% *** PLOTTING (Things to do with plotting) ***
% *********************************************
 
% **  User commands
% **    \setplotsymbol ({PLOTSYMBOL} [ORIENTATION] <XSHIFT,YSHIFT>)
% **    \savelinesandcurves on "FILE_NAME"
% **    \dontsavelinesandcurves
% **    \writesavefile {MESSAGE}
% **    \replot {FILE_NAME}
 
% **  Internal command
% **    \!plot(XDIMEN,YDIMEN)
 
% **  \setplotsymbol ({PLOTSYMBOL} [ ] < , >)
% **  Save PLOTSYMBOL away in an hbox for use with curve plotting routines
% **  See Subsection 5.2 of the manual.
\def\setplotsymbol(#1#2){%
  \!setputobject{#1}{#2}
  \setbox\!plotsymbol=\box\!putobject%
  \!plotsymbolxshift=\!xshift 
  \!plotsymbolyshift=\!yshift 
  \ignorespaces}
 
\font\fiverm=cmr5
\setplotsymbol({\fiverm .})%       ** initialize plotsymbol
 
 
% **  \!plot is either \!!plot (when no lines and curves are being saved) or
% **                   \!!!plot (when   lines and curves are being saved)
 
% **  \!!plot(XDIMEN,YDIMEN)
% **  Places the current plotsymbol a horizontal distance=XDIMEN-xorigin 
% **    and a vertical distance=YDIMEN-yorigin from the current
% **    reference point.  
\def\!!plot(#1,#2){%
  \!dimenA=-\!plotxorigin \advance \!dimenA by #1%    ** over
  \!dimenB=-\!plotyorigin \advance \!dimenB by #2%    ** up
  \kern\!dimenA\raise\!dimenB\copy\!plotsymbol\kern-\!dimenA%
  \ignorespaces}
 
% **  \!!!plot(XDIMEN,YDIMEN)
% **  Like \!!plot, but also saves the plot location in units of 
% **    scaled point, on file `replotfile'
\def\!!!plot(#1,#2){%
  \!dimenA=-\!plotxorigin \advance \!dimenA by #1%    ** over
  \!dimenB=-\!plotyorigin \advance \!dimenB by #2%    ** up
  \kern\!dimenA\raise\!dimenB\copy\!plotsymbol\kern-\!dimenA%
  \!countE=\!dimenA
  \!countF=\!dimenB
  \immediate\write\!replotfile{\the\!countE,\the\!countF.}%
  \ignorespaces}


% ** \savelinesandcurves on "FILE_NAME"
% **   Switch to save locations used for plotting lines and curves
% **   (No advantage in saving locations for solid lines; however
% **   replotting curve locations speeds things up by a factor of about 4. 
% ** \dontsavelinesandcurves
% **   Terminates \savelinesandcurves. The default.
% ** See Subsection 5.6 of the manual.
\def\savelinesandcurves on "#1" {%
  \immediate\closeout\!replotfile
  \immediate\openout\!replotfile=#1%
  \let\!plot=\!!!plot}

\def\dontsavelinesandcurves {%
  \let\!plot=\!!plot}
\dontsavelinesandcurves

% ** \writesavefile {MESSAGE}
% ** The message is preceded by a "%", so that it won't interfere
% ** with replotting.
% ** See Subsection 5.6 of the manual.
{\catcode`\%=11\xdef\!Commentsignal{%}}
\def\writesavefile#1 {%
  \immediate\write\!replotfile{\!Commentsignal #1}%
  \ignorespaces}

% ** \replot "FILE_NAME"
% **   Replots the locations saved earlier under \savelinesandcurves
% **   on "FILE_NAME"
% ** See Subsection 5.6 of the manual.
\def\replot"#1" {%
  \expandafter\!replot\input #1 /}
\def\!replot#1,#2. {%
  \!dimenA=#1sp
  \kern\!dimenA\raise#2sp\copy\!plotsymbol\kern-\!dimenA
  \futurelet\!nextchar\!!replot}
\def\!!replot{%
  \if /\!nextchar 
    \def\!next{\!finish}%
  \else
    \def\!next{\!replot}%
  \fi
  \!next}
% **************************************************
% ***  PYTHAGORAS  (Euclidean distance function) ***
% **************************************************

% ** User command:
% **   \placehypotenuse for <dimension1> and <dimension2> in <register> 

% ** Internal command:
% **   \!Pythag{X}{Y}{Z}
% **     Input X,Y are dimensions, or dimension registers.
% **     Output Z == sqrt(X**2+Y**2) must be a dimension register.
% **     Assumes that |X|+|Y| < 2048pt (about 28in).
 
% ** Without loss of generality, suppose  x>0, y>0.  Put s = x+y,
% **   z = sqrt(x**2+y**2). Then  z = s*f,  where  f = sqrt(t**2 + (1-t)**2)
% **   = sqrt((1+tau**2)/2), where  t = x/s  and  tau = 2(t-1/2) .
 
% ** Uses the \!divide macro (which uses registers \!dimenA--\!dimenD.
% ** Uses the \!removept macro   (e.g., 123.45pt --> 123.45)
% ** Uses registers \!dimenE--\!dimenI.
\def\!Pythag#1#2#3{%
  \!dimenE=#1\relax                                     
  \ifdim\!dimenE<\!zpt 
    \!dimenE=-\!dimenE 
  \fi%                                            ** dimE = |x|
  \!dimenF=#2\relax
  \ifdim\!dimenF<\!zpt 
    \!dimenF=-\!dimenF 
  \fi%                                            ** dimF = |y|
  \advance \!dimenF by \!dimenE%                  ** dimF = s = |x|+|y|
  \ifdim\!dimenF=\!zpt 
    \!dimenG=\!zpt%                               ** dimG = z = sqrt(x**2+y**2)
  \else 
    \!divide{8\!dimenE}\!dimenF\!dimenE%          ** now dimE = 8t = (8|x|)/s
    \advance\!dimenE by -4pt%                     ** 8tau = (8t-4)*2
      \!dimenE=2\!dimenE%                         **   (tau = 2*t - 1)
    \!removept\!dimenE\!!t%                       ** 8tau, without "pt"
    \!dimenE=\!!t\!dimenE%                        ** (8tau)**2, in pts
    \advance\!dimenE by 64pt%                     ** u = [64 + (8tau)**2]/2
    \divide \!dimenE by 2%                        **   [u = (8f)**2]
    \!dimenH=7pt%                                 ** initial guess g at sqrt(u)
    \!!Pythag\!!Pythag\!!Pythag%                  ** 3 iterations give sqrt(u)
    \!removept\!dimenH\!!t%                       ** 8f=sqrt(u), without "pt"
    \!dimenG=\!!t\!dimenF%                        ** z = (8f)*s/8
    \divide\!dimenG by 8
  \fi
  #3=\!dimenG
  \ignorespaces}

\def\!!Pythag{%                                   ** Newton-Raphson for sqrt
  \!divide\!dimenE\!dimenH\!dimenI%               ** v = u/g
  \advance\!dimenH by \!dimenI%                   ** g <-- (g + u/g)/2
    \divide\!dimenH by 2}

% **  \placehypotenuse for <XI> and <ETA> in <ZETA>
% **  See Subsection 9.3 of the manual.
\def\placehypotenuse for <#1> and <#2> in <#3> {%
  \!Pythag{#1}{#2}{#3}}


% **********************************************
% *** QUADRATIC ARC  (Draws a quadratic arc) ***
% **********************************************
 
% **  Internal command
% **    \!qjoin (XCOORD1,YCOORD1) (XCOORD2,YCOORD2)
 
% **  \!qjoin (XCOORD1,YCOORD1) (XCOORD2,YCOORD2)
% **  Draws an arc starting at the (last) point specified by the most recent
% **  \!qjoin, or \!ljoin, or \!start  and passing through (X_1,Y_1), (X_2,Y_2).
% **  Uses quadratic interpolation in both  x  and  y:  
% **    x(t), 0 <= t <= 1, interpolates  x_0, x_1, x_2  at  t=0, .5, 1
% **    y(t), 0 <= t <= 1, interpolates  y_0, y_1, y_2  at  t=0, .5, 1
 
\def\!qjoin (#1,#2) (#3,#4){%
  \advance\!intervalno by 1
  \!ifcoordmode
    \edef\!xmidpt{#1}\edef\!ymidpt{#2}%
  \else
    \!dimenA=#1\relax \edef\!xmidpt{\the\!dimenA}%
    \!dimenA=#2\relax \edef\!ymidpt{\the\!dimenA}%
  \fi
  \!xM=\!M{#1}\!xunit  \!yM=\!M{#2}\!yunit   \!rotateaboutpivot\!xM\!yM
  \!xE=\!M{#3}\!xunit  \!yE=\!M{#4}\!yunit   \!rotateaboutpivot\!xE\!yE
%
% ** Find coefficients for x(t)=a_x + b_x*t + c_x*t**2
  \!dimenA=\!xM  \advance \!dimenA by -\!xS%   ** dimA = I = xM - xS
  \!dimenB=\!xE  \advance \!dimenB by -\!xM%   ** dimB = II = xE-xM
  \!xB=3\!dimenA \advance \!xB by -\!dimenB%   ** b=3I-II
  \!xC=2\!dimenB \advance \!xC by -2\!dimenA%  ** c=2(II-I)
%
% ** Find coefficients for y(t)=y_x + b_y*t + c_y*t**2
  \!dimenA=\!yM  \advance \!dimenA by -\!yS%   
  \!dimenB=\!yE  \advance \!dimenB by -\!yM%  
  \!yB=3\!dimenA \advance \!yB by -\!dimenB%  
  \!yC=2\!dimenB \advance \!yC by -2\!dimenA% 
%
% ** Use Simpson's rule to calculate arc length over [0,1/2]:
% **   arc length = 1/2[1/6 f(0) + 4/6 f(1/4) + 1/6 f(1/2)]
% ** with f(t) = sqrt(x'(t)**2 + y'(t)**2).
  \!xprime=\!xB  \!yprime=\!yB%          ** x'(t) = b + 2ct
  \!dxprime=.5\!xC  \!dyprime=.5\!yC%    ** dt=1/4 ==> dx'(t) = c/2
  \!getf \!midarclength=\!dimenA
  \!getf \advance \!midarclength by 4\!dimenA
  \!getf \advance \!midarclength by \!dimenA
  \divide \!midarclength by 12
%
% ** Get arc length over [0,1].
  \!arclength=\!dimenA
  \!getf \advance \!arclength by 4\!dimenA
  \!getf \advance \!arclength by \!dimenA
  \divide \!arclength by 12%             ** Now have arc length over [1/2,1]
  \advance \!arclength by \!midarclength
  \global\advance \totalarclength by \!arclength
%
%
% ** Check to see if there's anything to plot in this interval
  \ifdim\!distacross>\!arclength 
    \advance \!distacross by -\!arclength%   ** nothing 
%
  \else
    \!initinverseinterp%  ** initialize for inverse interpolation on arc length
    \loop\ifdim\!distacross<\!arclength%     ** loop over points on arc 
      \!inverseinterp%    ** find  t  such that arc length[0,t] = distacross,
%                         **   using inverse quadratic interpolation
%                         ** now evaluate x(t)=(c*t + b)*t + a
      \!xpos=\!t\!xC \advance\!xpos by \!xB
        \!xpos=\!t\!xpos \advance \!xpos by \!xS
%                                             ** evaluate y(t)
      \!ypos=\!t\!yC \advance\!ypos by \!yB
        \!ypos=\!t\!ypos \advance \!ypos by \!yS
      \!plotifinbounds%                       ** plot point if in bounds
      \advance\!distacross \plotsymbolspacing%** advance arc length for next pt
      \!advancedashing%                       ** see "linear"
    \repeat  
%
    \advance \!distacross by -\!arclength%    ** prepare for next interval 
  \fi
%
  \!xS=\!xE%              ** shift ending points to starting points
  \!yS=\!yE
  \ignorespaces}
 
 
% ** \!getf -- Calculates sqrt(x'(t)**2 + y'(t)**2) and advances
% **   x'(t) and y'(t)
\def\!getf{\!Pythag\!xprime\!yprime\!dimenA%
  \advance\!xprime by \!dxprime
  \advance\!yprime by \!dyprime}
 
 
% ** \!initinverseinterp -- initializes for inverse quadratic interpolation
% ** of arc length provided  1/3 < midarclength/arclength < 2/3; otherwise
% ** initializes for inverse linear interpolation.
\def\!initinverseinterp{%
  \ifdim\!arclength>\!zpt
    \!divide{8\!midarclength}\!arclength\!dimenE% ** dimE=8w=8r/s, where  r 
%                                               **  = midarclength, s=arclength
% **  Test for  w  out of range:  w<1/3  or w>2/3
    \ifdim\!dimenE<\!wmin \!setinverselinear
    \else 
      \ifdim\!dimenE>\!wmax \!setinverselinear
      \else%                                    ** w  in range: initialize
        \def\!inverseinterp{\!inversequad}\ignorespaces
%
% **     Calculate the coefficients  \!beta  and  \!gamma  of the quadratic
% **                    t = \!beta*v + \!gamma*v**2
% **     taking the values  t=0, 1/2, 1  at  v=0, w==r/s, 1  respectively:
% **        \!beta = (1/2 - w**2)/[w(1-w)] 
% **        \!gamma = 1 - beta.
%
         \!removept\!dimenE\!Ew%           **  8w, without "pt"
         \!dimenF=-\!Ew\!dimenE%           **  -(8w)**2
         \advance\!dimenF by 32pt%         **  32 - (8w)**2
         \!dimenG=8pt 
         \advance\!dimenG by -\!dimenE%    **  8 - 8w
         \!dimenG=\!Ew\!dimenG%            **  (8w)*(8-8w)
         \!divide\!dimenF\!dimenG\!beta%   **  beta = (32-(8w)**2)/(8w(8-8w))
%                                          **       = (1/2 - w**2)/(w(1-w))
         \!gamma=1pt
         \advance \!gamma by -\!beta%      **  gamma = 1-beta
      \fi%       ** end of the \ifdim\!dimenE>\!wmax
    \fi%         ** end of the \ifdim\!dimenE<\!wmin
  \fi%           ** end of the \ifdim\!arclength>\!zpt
  \ignorespaces}
 
 
% ** For 0 <= t <= 1, let AL(t) = arclength[0,t]/arclength[0,1]; note
% ** AL(0)=0, AL(1/2)=midarclength/arclength, AL(1)=1.  This routine
% ** calculates an approximation to AL^{-1}(distance across/arclength),
% ** using the assumption that AL^{-1} is quadratic.  Specifically, 
% ** it finds  t  such that
% **    AL^{-1}(v) =. t = v*(\!beta + \!gamma*v)
% ** where  \!beta  and  \!gamma  are set by \!initinv, and where
% ** v=distance across/arclength
\def\!inversequad{%
  \!divide\!distacross\!arclength\!dimenG%   ** dimG = v = distacross/arclength
  \!removept\!dimenG\!v%                     ** v, without "pt"
  \!dimenG=\!v\!gamma%                       ** gamma*v
  \advance\!dimenG by \!beta%                ** beta + gamma*v
  \!dimenG=\!v\!dimenG%                      ** t = v*(beta + gamma*v)
  \!removept\!dimenG\!t}%                    ** t, without "pt"
 
 
% ** When  w <= 1/3  or  w >= 2/3, the following routine writes (using
% ** plain TEK's \wlog command) a warning message on the user's log file,
% ** and initializes for inverse linear interpolation on arc length.
\def\!setinverselinear{%
  \def\!inverseinterp{\!inverselinear}%
  \divide\!dimenE by 8 \!removept\!dimenE\!t
  \!countC=\!intervalno \multiply \!countC 2
  \!countB=\!countC     \advance \!countB -1
  \!countA=\!countB     \advance \!countA -1
  \wlog{\the\!countB th point (\!xmidpt,\!ymidpt) being plotted 
    doesn't lie in the}%
  \wlog{ middle third of the arc between the \the\!countA th 
    and \the\!countC th points:}%
  \wlog{ [arc length \the\!countA\space to \the\!countB]/[arc length 
    \the \!countA\space to \the\!countC]=\!t.}%
  \ignorespaces}
 
% **  Inverse linear interpolation
\def\!inverselinear{% 
  \!divide\!distacross\!arclength\!dimenG
  \!removept\!dimenG\!t}


% **************************************
% **  ROTATIONS  (Handles rotations) ***
% **************************************
 
% ** User commands
% **   \startrotation [by COS_OF_ANGLE SIN_OF_ANGLE] [about XPIVOT YPIVOT]
% **   \stoprotation

% **   \startrotation [by COS_OF_ANGLE SIN_OF_ANGLE] [about XPIVOT YPIVOT]
% ** Future (XCOORD,YCOORD)'s will be rotated about (XPIVOT,YPIVOT) 
% ** by the angle with the give COS and SIN. Both fields are optional.
% ** [COS,SIN] defaults to previous value, or (1,0).
% ** (XPIVOT,YPIVOT) defaults to previous value, or (0,0)
% ** You can't change the coordinate system in the scope of a rotation.
% ** See Subsection 9.1 of the manual.
\def\startrotation{%
  \let\!rotateaboutpivot=\!!rotateaboutpivot
  \let\!rotateonly=\!!rotateonly
  \!ifnextchar{b}{\!getsincos }%
    {\!getsincos by {\!cosrotationangle} {\!sinrotationangle} }}
\def\!getsincos by #1 #2 {%
  \edef\!cosrotationangle{#1}%
  \edef\!sinrotationangle{#2}%
  \!ifcoordmode 
    \let\!ROnext=\!ccheckforpivot
  \else
    \let\!ROnext=\!dcheckforpivot
  \fi
  \!ROnext}
\def\!ccheckforpivot{%
  \!ifnextchar{a}{\!cgetpivot}%
    {\!cgetpivot about {\!xpivotcoord} {\!ypivotcoord} }}
\def\!cgetpivot about #1 #2 {%
  \edef\!xpivotcoord{#1}%
  \edef\!ypivotcoord{#2}%
  \!xpivot=#1\!xunit  \!ypivot=#2\!yunit
  \ignorespaces}
\def\!dcheckforpivot{%
  \!ifnextchar{a}{\!dgetpivot}{\ignorespaces}}
\def\!dgetpivot about #1 #2 {%
  \!xpivot=#1\relax  \!ypivot=#2\relax
  \ignorespaces}
  

% ** Following terminates rotation.
% ** See Subsection 9.1 of the manual.
\def\stoprotation{%
  \let\!rotateaboutpivot=\!!!rotateaboutpivot
  \let\!rotateonly=\!!!rotateonly
  \ignorespaces}
 
% ** !!rotateaboutpivot{XREG}{YREG}
% ** XREG <-- xpvt + cos(angle)*(XREG-xpvt) - sin(angle)*(YREG-ypvt)
% ** YREG <-- ypvt + cos(angle)*(YREG-ypvt) + sin(angle)*(XREG-xpvt)
% ** XREG,YREG are dimension registers. Can't be \!dimenA to \!dimenD
\def\!!rotateaboutpivot#1#2{%
  \!dimenA=#1\relax  \advance\!dimenA -\!xpivot
  \!dimenB=#2\relax  \advance\!dimenB -\!ypivot
  \!dimenC=\!cosrotationangle\!dimenA
    \advance \!dimenC -\!sinrotationangle\!dimenB
  \!dimenD=\!cosrotationangle\!dimenB
    \advance \!dimenD  \!sinrotationangle\!dimenA
  \advance\!dimenC \!xpivot  \advance\!dimenD \!ypivot
  #1=\!dimenC  #2=\!dimenD
  \ignorespaces}

% ** \!!rotateonly{XREG}{YREG}
% ** Like \!!rotateaboutpivot, but with a pivot of  (0,0)
\def\!!rotateonly#1#2{%
  \!dimenA=#1\relax  \!dimenB=#2\relax 
  \!dimenC=\!cosrotationangle\!dimenA
    \advance \!dimenC -\!rotsign\!sinrotationangle\!dimenB
  \!dimenD=\!cosrotationangle\!dimenB
    \advance \!dimenD  \!rotsign\!sinrotationangle\!dimenA
  #1=\!dimenC  #2=\!dimenD
  \ignorespaces}
\def\!rotsign{}
\def\!!!rotateaboutpivot#1#2{\relax}
\def\!!!rotateonly#1#2{\relax}
\stoprotation

\def\!reverserotateonly#1#2{%
  \def\!rotsign{-}%
  \!rotateonly{#1}{#2}%
  \def\!rotsign{}%
  \ignorespaces}


% **********************************
% *** SHADING  (Handles shading) ***
% **********************************

% **  User commands
% **    \setshadegrid [span <SPAN>] [point at XSHADE YSHADE] 
% **    \setshadesymbol [<LS, RS, BS, TS>] ({SHADESYMBOL}
% **      <XDIMEN,YDIMEN> [ORIENTATION])

% **  Internal commands:
% **    \!startvshade  (xS,ybS,ytS)
% **    \!starthshade  (yS,xlS,xrS)
% **    \!lshade [<LS,RS,BS,TS>]
% **       ** when shading vertically:
% **       [the region from (xS,ybS,ytS) to] (xE,ybE,ytE)
% **       ** when shading horizontally:
% **       [the region from (yS,xlS,xrS) to] (yE,xlE,xrE)
% **    \!qshade [<LS,RS,BS,TS>]
% **       ** when shading vertically:
% **       [the region from (xS,ybS,ytS) to] (xM,ybM,ytM)  (xE,ybE,ytE)
% **       ** when shading horizontally:
% **       [the region from (yS,xlS,xrS) to] (yM,xlM,xrM)  (yE,xlE,xrE)
% **    \!lattice{ANCHOR}{SPAN}{LOCATION}{INDEX}{LATTICE LOCATION}
% **    \!override{NOMINAL DIMEN}{REPLACEMENT DIMEN}{DIMEN}


% **  The shading routine can operate either in a "vertical mode" or a
% **  "horizontal mode".  In vertical mode, the region to be shaded is specified
% **  in the form
% **                 {(x,y): xl <= x <= xr  &  yb(x) <= y <= yt(x)}
% **  where  yb  and  yt  are functions of  x.  In horizontal mode, the region
% **  is specified in the form
% **                 {(x,y): yb <= y <= yt  &  xl(y) <= x <= xr(y)}.
% **  The functions  yb  and  yt  may be either both linear or both quadratic;
% **  similarly for  xl  and  xr.  A region with say, piecewise quadratic bottom
% **  and top boundaries, can be shaded by consecutive (vertical) \!qshades,
% **  proceeding from left to right.  Similarly, a region with piecewise     
% **  quadratic left and right boundaries can be shaded by consecutive
% **  (horizontal) \!qshades, proceeding from bottom to top.  More complex
% **  regions can be shaded by partitioning them into appropriate subregions,
% **  and shading those.
    
% **  Shading is accomplished by placing a user-selected shading symbol at
% **  those points of a regular grid which fall within the region to be
% **  shaded.  This region can be "shrunk" so that a largish shading symbol
% **  will not extend outside it.  Shrinking is accomplished by specifying
% **  shrinkages for the left, right, bottom, and top boundaries, in a manner
% **  discussed further below.

% **  \shades and \!joins MUST NOT be intermingled.  Finish drawing a curve
% **  before starting to shade a region, and finish shading a region before
% **  starting to draw a curve.


% **  \setshadegrid [span <SPAN>] [point at XSHADE YSHADE] 
% **  The shading symbol is placed down on the points of a grid centered
% **  at the coordinate point (XSHADE,YSHADE).  The grid points are of the
% **  form (j*SPAN,k*SPAN), with  j+k  even.  SPAN is specified
% **  as a dimension.
% **  (XSHADE,YSHADE) defaults to previous (XSHADE,YSHADE) (or (0,0) if none)
% **  SPAN defaults to previous span (or 5pt if none)
% **  See Subsection 7.2 of the manual.
\def\setshadegrid{%
  \!ifnextchar{s}{\!getspan }
    {\!getspan span <\!dshade>}}
\def\!getspan span <#1>{%
  \!dshade=#1\relax
  \!ifcoordmode 
    \let\!GRnext=\!GRccheckforAP
  \else
    \let\!GRnext=\!GRdcheckforAP
  \fi
  \!GRnext}
\def\!GRccheckforAP{%
  \!ifnextchar{p}{\!cgetanchor }
    {\!cgetanchor point at {\!xshadesave} {\!yshadesave} }}
\def\!cgetanchor point at #1 #2 {%
  \edef\!xshadesave{#1}\edef\!yshadesave{#2}%
  \!xshade=\!xshadesave\!xunit  \!yshade=\!yshadesave\!yunit
  \ignorespaces}
\def\!GRdcheckforAP{%
  \!ifnextchar{p}{\!dgetanchor}%
    {\ignorespaces}}
\def\!dgetanchor point at #1 #2 {%
  \!xshade=#1\relax  \!yshade=#2\relax
  \ignorespaces}

% **  \setshadesymbol  [<LS, RS, BS, TS>] ({SHADESYMBOL}
% **    <XDIMEN,YDIMEN> [ORIENTATION])
% **  Saves SHADESYMBOL away in an hbox for use with shading routines.
% **  A shade symbol will not be plotted if its plot position comes within
% **    distance LS of the left boundary,  RS of the right boundary,  TS of the
% **    top boundary,  BS of the bottom boundary.  These parameters have 
% **    default values that should work in most cases (see below).
% **    To override a default value, specify the replacement value
% **    in the appropriate subfield of the shrinkages field.
% **    0pt may be coded as  "z" (without the quotes).  To accept a
% **    default value, leave the field empty.  Thus
% **      [,z,,5pt]  sets  LS=default, RS=0pt, BS=default, TS=5pt .
% **    Skipping the shrinkages field accepts all the defaults.
% **  See Subsection 7.1 of the manual.
\def\setshadesymbol{%
  \!ifnextchar<{\!setshadesymbol}{\!setshadesymbol<,,,> }}

\def\!setshadesymbol <#1,#2,#3,#4> (#5#6){%
% **  set the shadesymbol
  \!setputobject{#5}{#6}%                        
  \setbox\!shadesymbol=\box\!putobject%
  \!shadesymbolxshift=\!xshift \!shadesymbolyshift=\!yshift
%
% **  set the shrinkages
  \!dimenA=\!xshift \advance\!dimenA \!smidge% ** default LS = xshift - smidge
  \!override\!dimenA{#1}\!lshrinkage%         
  \!dimenA=\!wd \advance \!dimenA -\!xshift%   ** default RS = width - xshift
    \advance\!dimenA \!smidge%                                  - smidge
    \!override\!dimenA{#2}\!rshrinkage
  \!dimenA=\!dp \advance \!dimenA \!yshift%    ** default BS = depth + yshift
    \advance\!dimenA \!smidge%                                  - smidge
    \!override\!dimenA{#3}\!bshrinkage
  \!dimenA=\!ht \advance \!dimenA -\!yshift%   ** default TS = height - yshift
    \advance\!dimenA \!smidge%                                  - smidge
    \!override\!dimenA{#4}\!tshrinkage
  \ignorespaces}
\def\!smidge{-.2pt}%

% ** \!override{NOMINAL DIMEN}{REPLACEMENT DIMEN}{DIMEN}
% ** Overrides the NOMINAL DIMEN by the REPLACEMENT DIMEN to produce DIMEN,
% ** according to the following rules:
% **   REPLACEMENT DIMEN empty: DIMEN <-- NOMINAL DIMEN
% **   REPLACEMENT DIMEN z:     DIMEN <-- 0pt
% **   otherwise:               DIMEN <-- REPLACEMENT DIMEN
% ** DIMEN must be a dimension register
\def\!override#1#2#3{%
  \edef\!!override{#2}% 
  \ifx \!!override\empty
    #3=#1\relax
  \else
    \if z\!!override
      #3=\!zpt
    \else
      \ifx \!!override\!blankz
        #3=\!zpt
      \else
        #3=#2\relax
      \fi
    \fi
  \fi
  \ignorespaces}
\def\!blankz{ z}

\setshadesymbol ({\fiverm .})%       ** initialize plotsymbol
%                                    ** \fivesy ^^B  is a small cross


% ** \!startvshade [at] (xS,ybS,ytS)
% ** Initiates vertical shading mode
\def\!startvshade#1(#2,#3,#4){%
  \let\!!xunit=\!xunit%
  \let\!!yunit=\!yunit%
  \let\!!xshade=\!xshade%
  \let\!!yshade=\!yshade%
  \def\!getshrinkages{\!vgetshrinkages}%
  \let\!setshadelocation=\!vsetshadelocation%
  \!xS=\!M{#2}\!!xunit
  \!ybS=\!M{#3}\!!yunit
  \!ytS=\!M{#4}\!!yunit
  \!shadexorigin=\!xorigin  \advance \!shadexorigin \!shadesymbolxshift
  \!shadeyorigin=\!yorigin  \advance \!shadeyorigin \!shadesymbolyshift
  \ignorespaces}
 
% ** \!starthshade [at] (yS,xlS,xrS)
% ** Initiates horizontal shading mode
\def\!starthshade#1(#2,#3,#4){%
  \let\!!xunit=\!yunit%
  \let\!!yunit=\!xunit%
  \let\!!xshade=\!yshade%
  \let\!!yshade=\!xshade%
  \def\!getshrinkages{\!hgetshrinkages}%
  \let\!setshadelocation=\!hsetshadelocation%
  \!xS=\!M{#2}\!!xunit
  \!ybS=\!M{#3}\!!yunit
  \!ytS=\!M{#4}\!!yunit
  \!shadexorigin=\!xorigin  \advance \!shadexorigin \!shadesymbolxshift
  \!shadeyorigin=\!yorigin  \advance \!shadeyorigin \!shadesymbolyshift
  \ignorespaces}


% **  \!lattice{ANCHOR}{SPAN}{LOCATION}{INDEX}{LATTICE LOCATION}
% **  Consider the lattice with points  ANCHOR + j*SPAN. This routine determines
% **  the index  k  of the smallest lattice point >= LOCATION, and sets
% **  LATTICE LOCATION = ANCHOR + k*SPAN.
% **  INDEX is assumed to be a count register, LATTICE LOCATION a dimen reg.
\def\!lattice#1#2#3#4#5{%
  \!dimenA=#1%                        ** dimA = ANCHOR
  \!dimenB=#2%                        ** dimB = SPAN  (assumed > 0pt)
  \!countB=\!dimenB%                  ** ctB  = SPAN, as a count
%
% ** Determine index of smallest lattice point >= LOCATION
  \!dimenC=#3%                        ** dimC = LOCATION
  \advance\!dimenC -\!dimenA%         ** now dimC = LOCATION-ANCHOR
  \!countA=\!dimenC%                  ** ctA = above, as a count
  \divide\!countA \!countB%           ** now ctA = desired index, if dimC <= 0
  \ifdim\!dimenC>\!zpt
    \!dimenD=\!countA\!dimenB%        ** (tentative k)*span
    \ifdim\!dimenD<\!dimenC%          ** if this is false, ctA = desired index
      \advance\!countA 1 %            ** if true, have to add 1
    \fi
  \fi
%
  \!dimenC=\!countA\!dimenB%          ** lattice location = anchor + ctA*span
    \advance\!dimenC \!dimenA
  #4=\!countA%                        ** the desired index
  #5=\!dimenC%                        ** corresponding lattice location
  \ignorespaces}


% ** \!qshade [with shrinkages] [[LS,RS,BS,TS]]
% ***** during vertical shading:
% **    [the region from (xS,ybS,ytS) to] (xM,ybM,ytM) [and] (xE,ybE,ytE)
% ** Shades the region {(x,y): xS <= x <= xE, yb(x) <= y <= yt(x)}, where 
% **   yb is the quadratic thru (xS,ybS) & (xM,ybM) & (xE,ybE)
% **   yt is the quadratic thru (xS,ytS) & (xM,ybM) & (xE,ytE)
% ** xS,ybS,ytS are either given by \!startvshade or carried over
% **   as the ending values of the immediately preceding \!qshade.
% ** For the interpretation of LS, RS, BS, & TS, see \setshadesymbol. The
% **   values set there can be overridden, for the course of this \!qshade
% **   only, in the same manner as overrides are specified for
% **   \setshadesymbol.
% ***** during horizontal shading:
% **    [the region from (yS,xlS,xrS) to] (yM,xlM,xrM) [and] (yE,xlE,xrE)
\def\!qshade#1(#2,#3,#4)#5(#6,#7,#8){%
  \!xM=\!M{#2}\!!xunit
  \!ybM=\!M{#3}\!!yunit
  \!ytM=\!M{#4}\!!yunit
  \!xE=\!M{#6}\!!xunit
  \!ybE=\!M{#7}\!!yunit
  \!ytE=\!M{#8}\!!yunit
  \!getcoeffs\!xS\!ybS\!xM\!ybM\!xE\!ybE\!ybB\!ybC%**Get coefficients B & C for
  \!getcoeffs\!xS\!ytS\!xM\!ytM\!xE\!ytE\!ytB\!ytC%**y=y0 + B(x-X0) + C(x-X0)**2
  \def\!getylimits{\!qgetylimits}%
  \!shade{#1}\ignorespaces}
 
% ** \!lshade ... (xE,ybE,ytE)
% ** This is like \!qshade, but the top and bottom boundaries are linear,
% ** rather than quadratic.
\def\!lshade#1(#2,#3,#4){%
  \!xE=\!M{#2}\!!xunit
  \!ybE=\!M{#3}\!!yunit
  \!ytE=\!M{#4}\!!yunit
  \!dimenE=\!xE  \advance \!dimenE -\!xS%   ** xE-xS
  \!dimenC=\!ytE \advance \!dimenC -\!ytS%  ** ytE-ytS
  \!divide\!dimenC\!dimenE\!ytB%            ** ytB = (ytE-ytS)/(xE-xS)
  \!dimenC=\!ybE \advance \!dimenC -\!ybS%  ** ybE-ybS
  \!divide\!dimenC\!dimenE\!ybB%            ** ybB = (ybE-ybS)/(xE-xS)
  \def\!getylimits{\!lgetylimits}%
  \!shade{#1}\ignorespaces}
 
% **  \!getcoeffs{X0}{Y0}{X1}{Y1}{X2}{Y2}{B}{C}
% **  Finds  B  and  C  such that the quadratic  y = Y0 + B(x-X0) + C(x-X0)**2
% **  passes through (X1,Y1) and (X2,Y2):  when X0=0=Y0, the formulas are:
% **                   B = S1 - X1*C,   C = (S2-S1)/X2
% **  with
% **                 S1 = Y1/X1,   S2 = (Y2-Y1)/(X2-X1).
\def\!getcoeffs#1#2#3#4#5#6#7#8{% 
  \!dimenC=#4\advance \!dimenC -#2%            ** dimC=Y1-Y0
  \!dimenE=#3\advance \!dimenE -#1%            ** dimE=X1-X0
  \!divide\!dimenC\!dimenE\!dimenF%            ** dimF=S1
  \!dimenC=#6\advance \!dimenC -#4%            ** dimC=Y2-Y1
  \!dimenH=#5\advance \!dimenH -#3%            ** dimH=X2-X1
  \!divide\!dimenC\!dimenH\!dimenG%            ** dimG=S2
  \advance\!dimenG -\!dimenF%                  ** dimG=S2-S1
  \advance \!dimenH \!dimenE%                  ** dimH=X2-X0
  \!divide\!dimenG\!dimenH#8%                  ** C=(S2-S1)/(X2-X0)
  \!removept#8\!t%                             ** C, without "pt"
  #7=-\!t\!dimenE%                             ** -C*(X1-X0)
  \advance #7\!dimenF%                         ** B=S1-C*(X1-X0)
  \ignorespaces}
 
 
\def\!shade#1{%
% ** Get LS,RS,BS,TS for this panel
  \!getshrinkages#1<,,,>\!nil% %       ** now effective LS=dimE, RS=dimF,
%                                      **   BS=dimG, TS=dimH
  \advance \!dimenE \!xS%              ** now dimE=xS+LS
  \!lattice\!!xshade\!dshade\!dimenE%  ** set parity=index of left-mst x-lattice
    \!parity\!xpos%                    **   point >= xS+LS, xpos=its location
  \!dimenF=-\!dimenF%                  ** set dimF=xE-RS
    \advance\!dimenF \!xE
%               
  \!loop\!not{\ifdim\!xpos>\!dimenF}%  ** loop over x-lattice points <= xE-RS
    \!shadecolumn%                 
    \advance\!xpos \!dshade%           ** move over to next column
    \advance\!parity 1%                ** increase index of x-point
  \repeat
%
  \!xS=\!xE%                           ** shift ending values to starting values
  \!ybS=\!ybE
  \!ytS=\!ytE
  \ignorespaces}
 
 
\def\!vgetshrinkages#1<#2,#3,#4,#5>#6\!nil{%
  \!override\!lshrinkage{#2}\!dimenE
  \!override\!rshrinkage{#3}\!dimenF
  \!override\!bshrinkage{#4}\!dimenG
  \!override\!tshrinkage{#5}\!dimenH
  \ignorespaces}
\def\!hgetshrinkages#1<#2,#3,#4,#5>#6\!nil{%
  \!override\!lshrinkage{#2}\!dimenG
  \!override\!rshrinkage{#3}\!dimenH
  \!override\!bshrinkage{#4}\!dimenE
  \!override\!tshrinkage{#5}\!dimenF
  \ignorespaces}
 
 
\def\!shadecolumn{%
  \!dxpos=\!xpos
  \advance\!dxpos -\!xS%            ** dx = x - xS
  \!removept\!dxpos\!dx%            ** ditto, without "pt"
  \!getylimits%                     ** get top and bottom y-values
  \advance\!ytpos -\!dimenH%        ** less TS
  \advance\!ybpos \!dimenG%         ** plus BS
  \!yloc=\!!yshade%                 ** get anchor point for this column
  \ifodd\!parity 
     \advance\!yloc \!dshade
  \fi
  \!lattice\!yloc{2\!dshade}\!ybpos%
    \!countA\!ypos%                 ** ypos=smallest y point for this column
  \!dimenA=-\!shadexorigin \advance \!dimenA \!xpos%      ** over
  \loop\!not{\ifdim\!ypos>\!ytpos}% ** loop over ypos <= yt(t)
    \!setshadelocation%             ** vmode: xloc=xpos, yloc=ypos 
%                                   ** hmode: xloc=ypos, yloc=xpos 
    \!rotateaboutpivot\!xloc\!yloc%
    \!dimenA=-\!shadexorigin \advance \!dimenA \!xloc%    ** over
    \!dimenB=-\!shadeyorigin \advance \!dimenB \!yloc%    ** up
    \kern\!dimenA \raise\!dimenB\copy\!shadesymbol \kern-\!dimenA
    \advance\!ypos 2\!dshade
  \repeat
  \ignorespaces}
 
\def\!qgetylimits{%
  \!dimenA=\!dx\!ytC              
  \advance\!dimenA \!ytB%         ** yt(t)=ytS + dx*(Bt + dx*Ct)
  \!ytpos=\!dx\!dimenA
  \advance\!ytpos \!ytS
  \!dimenA=\!dx\!ybC              
  \advance\!dimenA \!ybB%         ** yb(t)=ybS + dx*(Bb + dx*Cb)
  \!ybpos=\!dx\!dimenA
  \advance\!ybpos \!ybS}
 
\def\!lgetylimits{%
  \!ytpos=\!dx\!ytB%              ** yt(t)=ytS + dx*Bt
  \advance\!ytpos \!ytS
  \!ybpos=\!dx\!ybB%              ** yb(t)=ybS + dx*Bb
  \advance\!ybpos \!ybS}
 
\def\!vsetshadelocation{%         ** vmode: xloc=xpos, yloc=ypos 
  \!xloc=\!xpos
  \!yloc=\!ypos}
\def\!hsetshadelocation{%         ** hmode: xloc=ypos, yloc=xpos 
  \!xloc=\!ypos
  \!yloc=\!xpos}


% **************************************
% *** TICKS  (Draws ticks on graphs) ***
% **************************************

% ** User commands
% **   \ticksout
% **   \ticksin
% **   \gridlines
% **   \nogridlines
% **   \loggedticks
% **   \unloggesticks
% ** See Subsection 3.4 of the manual

% ** The following is an option of the \axis command
% **   ticks 
% **     [in] [out] 
% **     [long] [short] [length <LENGTH>] 
% **     [width <WIDTH>]
% **     [andacross] [butnotacross] 
% **     [logged] [unlogged] 
% **     [unlabeled] [numbered] [withvalues VALUE1 VALUE2 ... VALUEk / ]
% **     [quantity Q] [at LOC1 LOC2 ... LOCk / ] [from LOC1 to LOC2 by
% **       LOC_INCREMENT]
% ** See Subsection 3.2 of the manual for the rules.

% ** The various options of the  tick  field are processed by the
% ** \!nextkeyword  command defined below.
% ** For example, `\!nextkeyword short '  expands to  `\!ticksshort',
% ** while `\!nextkeyword withvalues' expands to `\!tickswithvalues'.

\def\!axisticks {%
  \def\!nextkeyword##1 {%
    \expandafter\ifx\csname !ticks##1\endcsname \relax
      \def\!next{\!fixkeyword{##1}}%
    \else
      \def\!next{\csname !ticks##1\endcsname}%
    \fi
    \!next}%
  \!axissetup
    \def\!axissetup{\relax}%
  \edef\!ticksinoutsign{\!ticksinoutSign}%
  \!ticklength=\longticklength
  \!tickwidth=\linethickness
  \!gridlinestatus
  \!setticktransform
  \!maketick
  \!tickcase=0
  \def\!LTlist{}%
  \!nextkeyword}

\def\ticksout{%
  \def\!ticksinoutSign{+}}
\def\ticksin{%
  \def\!ticksinoutSign{-}}
\ticksout

\def\gridlines{%
  \def\!gridlinestatus{\!gridlinestootrue}}
\def\nogridlines{%
  \def\!gridlinestatus{\!gridlinestoofalse}}
\nogridlines

\def\loggedticks{%
  \def\!setticktransform{\let\!ticktransform=\!logten}}
\def\unloggedticks{%
  \def\!setticktransform{\let\!ticktransform=\!donothing}}
\def\!donothing#1#2{\def#2{#1}}
\unloggedticks

% ** \!ticks/ : terminates read of tick options
\expandafter\def\csname !ticks/\endcsname{%
  \!not {\ifx \!LTlist\empty}
    \!placetickvalues
  \fi
  \def\!tickvalueslist{}%
  \def\!LTlist{}%
  \expandafter\csname !axis/\endcsname}

\def\!maketick{%
  \setbox\!boxA=\hbox{%
    \beginpicture
      \!setdimenmode
      \setcoordinatesystem point at {\!zpt} {\!zpt}   
      \linethickness=\!tickwidth
      \ifdim\!ticklength>\!zpt
        \putrule from {\!zpt} {\!zpt} to
          {\!ticksinoutsign\!tickxsign\!ticklength}
          {\!ticksinoutsign\!tickysign\!ticklength}
      \fi
      \if!gridlinestoo
        \putrule from {\!zpt} {\!zpt} to
          {-\!tickxsign\!xaxislength} {-\!tickysign\!yaxislength}
      \fi
    \endpicturesave <\!Xsave,\!Ysave>}%
    \wd\!boxA=\!zpt}
  
\def\!ticksin{%
  \def\!ticksinoutsign{-}%
  \!maketick
  \!nextkeyword}

\def\!ticksout{%
  \def\!ticksinoutsign{+}%
  \!maketick
  \!nextkeyword}

\def\!tickslength<#1> {%
  \!ticklength=#1\relax
  \!maketick
  \!nextkeyword}

\def\!tickslong{%
  \!tickslength<\longticklength> }

\def\!ticksshort{%
  \!tickslength<\shortticklength> }

\def\!tickswidth<#1> {%
  \!tickwidth=#1\relax
  \!maketick
  \!nextkeyword}

\def\!ticksandacross{%
  \!gridlinestootrue
  \!maketick
  \!nextkeyword}

\def\!ticksbutnotacross{%
  \!gridlinestoofalse
  \!maketick
  \!nextkeyword}

\def\!tickslogged{%
  \let\!ticktransform=\!logten
  \!nextkeyword}

\def\!ticksunlogged{%
  \let\!ticktransform=\!donothing
  \!nextkeyword}

\def\!ticksunlabeled{%
  \!tickcase=0
  \!nextkeyword}

\def\!ticksnumbered{%
  \!tickcase=1
  \!nextkeyword}

\def\!tickswithvalues#1/ {%
  \edef\!tickvalueslist{#1! /}%
  \!tickcase=2
  \!nextkeyword}

\def\!ticksquantity#1 {%
  \ifnum #1>1
    \!updatetickoffset
    \!countA=#1\relax
    \advance \!countA -1
    \!ticklocationincr=\!axisLength
      \divide \!ticklocationincr \!countA
    \!ticklocation=\!axisstart
    \loop \!not{\ifdim \!ticklocation>\!axisend}
      \!placetick\!ticklocation
      \ifcase\!tickcase
          \relax %  Case 0: no labels
        \or
          \relax %  Case 1: numbered -- not available here
        \or
          \expandafter\!gettickvaluefrom\!tickvalueslist
          \edef\!tickfield{{\the\!ticklocation}{\!value}}%
          \expandafter\!listaddon\expandafter{\!tickfield}\!LTlist%
      \fi
      \advance \!ticklocation \!ticklocationincr
    \repeat
  \fi
  \!nextkeyword}

\def\!ticksat#1 {%
  \!updatetickoffset
  \edef\!Loc{#1}%
  \if /\!Loc
    \def\next{\!nextkeyword}%
  \else
    \!ticksincommon
    \def\next{\!ticksat}%
  \fi
  \next}    
      
\def\!ticksfrom#1 to #2 by #3 {%
  \!updatetickoffset
  \edef\!arg{#3}%
  \expandafter\!separate\!arg\!nil
  \!scalefactor=1
  \expandafter\!countfigures\!arg/
  \edef\!arg{#1}%
  \!scaleup\!arg by\!scalefactor to\!countE
  \edef\!arg{#2}%
  \!scaleup\!arg by\!scalefactor to\!countF
  \edef\!arg{#3}%
  \!scaleup\!arg by\!scalefactor to\!countG
  \loop \!not{\ifnum\!countE>\!countF}
    \ifnum\!scalefactor=1
      \edef\!Loc{\the\!countE}%
    \else
      \!scaledown\!countE by\!scalefactor to\!Loc
    \fi
    \!ticksincommon
    \advance \!countE \!countG
  \repeat
  \!nextkeyword}

\def\!updatetickoffset{%
  \!dimenA=\!ticksinoutsign\!ticklength
  \ifdim \!dimenA>\!offset
    \!offset=\!dimenA
  \fi}

\def\!placetick#1{%
  \if!xswitch
    \!xpos=#1\relax
    \!ypos=\!axisylevel
  \else
    \!xpos=\!axisxlevel
    \!ypos=#1\relax
  \fi
  \advance\!xpos \!Xsave
  \advance\!ypos \!Ysave
  \kern\!xpos\raise\!ypos\copy\!boxA\kern-\!xpos
  \ignorespaces}

\def\!gettickvaluefrom#1 #2 /{%
  \edef\!value{#1}%
  \edef\!tickvalueslist{#2 /}%
  \ifx \!tickvalueslist\!endtickvaluelist
    \!tickcase=0
  \fi}
\def\!endtickvaluelist{! /}

\def\!ticksincommon{%
  \!ticktransform\!Loc\!t
  \!ticklocation=\!t\!!unit
  \advance\!ticklocation -\!!origin
  \!placetick\!ticklocation
  \ifcase\!tickcase
    \relax % Case 0: no labels
  \or %      Case 1: numbered
    \ifdim\!ticklocation<-\!!origin
      \edef\!Loc{$\!Loc$}%
    \fi
    \edef\!tickfield{{\the\!ticklocation}{\!Loc}}%
    \expandafter\!listaddon\expandafter{\!tickfield}\!LTlist%
  \or %      Case 2: labeled
    \expandafter\!gettickvaluefrom\!tickvalueslist
    \edef\!tickfield{{\the\!ticklocation}{\!value}}%
    \expandafter\!listaddon\expandafter{\!tickfield}\!LTlist%
  \fi}

\def\!separate#1\!nil{%
  \!ifnextchar{-}{\!!separate}{\!!!separate}#1\!nil}
\def\!!separate-#1\!nil{%
  \def\!sign{-}%
  \!!!!separate#1..\!nil}
\def\!!!separate#1\!nil{%
  \def\!sign{+}%
  \!!!!separate#1..\!nil}
\def\!!!!separate#1.#2.#3\!nil{%
  \def\!arg{#1}%
  \ifx\!arg\!empty
    \!countA=0
  \else
    \!countA=\!arg
  \fi
  \def\!arg{#2}%
  \ifx\!arg\!empty
    \!countB=0
  \else
    \!countB=\!arg
  \fi}
 
\def\!countfigures#1{%
  \if #1/%
    \def\!next{\ignorespaces}%
  \else
    \multiply\!scalefactor 10
    \def\!next{\!countfigures}%
  \fi
  \!next}

\def\!scaleup#1by#2to#3{%
  \expandafter\!separate#1\!nil
  \multiply\!countA #2\relax
  \advance\!countA \!countB
  \if -\!sign
    \!countA=-\!countA
  \fi
  #3=\!countA
  \ignorespaces}

\def\!scaledown#1by#2to#3{%
  \!countA=#1\relax%                          ** get original #
  \ifnum \!countA<0 %                         ** take abs value,
    \def\!sign{-}%                            **   remember sign
    \!countA=-\!countA
  \else
    \def\!sign{}%
  \fi
  \!countB=\!countA%                          ** copy |#|
  \divide\!countB #2\relax%                   ** integer part (|#|/sf)
  \!countC=\!countB%                          ** get sf * (|#|/sf)
    \multiply\!countC #2\relax
  \advance \!countA -\!countC%                ** ctA is now remainder
  \edef#3{\!sign\the\!countB.}%               ** +- integerpart.
  \!countC=\!countA %                         ** Tack on proper number
  \ifnum\!countC=0 %                          **   of zeros after .
    \!countC=1
  \fi
  \multiply\!countC 10
  \!loop \ifnum #2>\!countC
    \edef#3{#3\!zero}%
    \multiply\!countC 10
  \repeat
  \edef#3{#3\the\!countA}%                    ** Add on rest of remainder
  \ignorespaces}

\def\!placetickvalues{%
  \advance\!offset \tickstovaluesleading
  \if!xswitch
    \setbox\!boxA=\hbox{%
      \def\\##1##2{%
        \!dimenput {##2} [B] (##1,\!axisylevel)}%
      \beginpicture 
        \!LTlist
      \endpicturesave <\!Xsave,\!Ysave>}%
    \!dimenA=\!axisylevel
      \advance\!dimenA -\!Ysave
      \advance\!dimenA \!tickysign\!offset
      \if -\!tickysign
        \advance\!dimenA -\ht\!boxA
      \else
        \advance\!dimenA  \dp\!boxA
      \fi
    \advance\!offset \ht\!boxA 
      \advance\!offset \dp\!boxA
    \!dimenput {\box\!boxA} [Bl] <\!Xsave,\!Ysave> (\!zpt,\!dimenA)
  \else
    \setbox\!boxA=\hbox{%
      \def\\##1##2{%
        \!dimenput {##2} [r] (\!axisxlevel,##1)}%
      \beginpicture 
        \!LTlist
      \endpicturesave <\!Xsave,\!Ysave>}%
    \!dimenA=\!axisxlevel
      \advance\!dimenA -\!Xsave
      \advance\!dimenA \!tickxsign\!offset
      \if -\!tickxsign
        \advance\!dimenA -\wd\!boxA
      \fi
    \advance\!offset \wd\!boxA
    \!dimenput {\box\!boxA} [Bl] <\!Xsave,\!Ysave> (\!dimenA,\!zpt)
  \fi}


\normalgraphs
\catcode`!=12 %  *****  THIS MUST NEVER BE OMITTED
