%% Package `pst-labo.tex'
%%
%% This program can be redistributed and/or modified under
%% the terms of the LaTeX Project Public License Distributed
%% from CTAN archives in directory macros/latex/base/lppl.txt.
%%
%% DESCRIPTION:
%%   `pst-labo' is a PSTricks package for 
%%   chemical objects.
%%
%%
%% Authors  :   Denis Girou <Denis.Girou@idris.fr>
%%              Christophe Jorsen <Christophe.Jorssen@free.fr>
%%              Manuel Luque <Mluque5130@aol.com>
%%              Herbert Voss <hvoss@tug.org>
%%
\csname PSTLaboLoaded\endcsname
\let\PSTLaboLoaded\endinput
%
% Require PSTricks, pstcol,pstplot, and multido packages
\ifx\PSTricksLoaded\endinput\else\input pstricks.tex\fi
\ifx\PSTplotLoaded\endinput\else\input pst-plot.tex\fi
\ifx\MultidoLoaded\endinput\else\input multido.tex\fi
\ifx\PSTGradLoaded\endinput\else\input pst-grad.tex\fi
\ifx\PSTXKeyLoaded\endinput\else\input pst-xkey.tex\fi
%
\def\fileversion{2.03}
\def\filedate{2006/08/16}
%
\message{`PST-labo8' v\fileversion, \filedate\space (ML/CJ/DG/HV)}
\edef\PstAtCode{\the\catcode`\@} \catcode`\@=11\relax
%
\pst@addfams{pst-labo}
%
\input random
%%%%%%%% Les Diff�rents appareils de chimie %%%%%%%%%%%%

%%%%%%%%% Dosage %%%%%%%%%%%%%%%%%%%%%%%%%%%%%%%%%%%%%%
\def\pst@burette@corps{%
\psline(-0.5,11.5)(-0.25,11)(-0.25,3)(0,2.75)(0,0.5)
\psline(0.2,0.7)(0.2,2.75)(0.5,3)(0.5,11)(0.75,11.5) }
\def\pst@burette@robinet{%
\psframe[fillstyle=solid](-.1,1.6)(.3,2.1)
\psframe(-.1,1.5)(-.2,2.2)
\psframe(.3,1.5)(.4,2.2)
\pspolygon(.4,1.7)(.5,1.7)(.6,1.4)(.8,1.4)(.8,2.3)(.6,2.3)(.5,2)(.4,2)
{\psset{fillstyle=solid,fillcolor=LightBlue}
\pspolygon(-.5,1.5)(-.75,1.7)(-.75,2)(-.5,2.2)(-.3,2.2)(-.3,1.5)
\psframe(-.5,1.5)(-.2,2.2)}}

\def\pst@dosage@pHmetre{%
% boitier
\psframe[framearc=0.25](-1,0)(1,3)%
\psframe[fillcolor=GrisClair,fillstyle=solid](-0.75,2)(0.75,2.75)%
\rput(0,2.375){\white PH}%
\pscircle(0,1.25){0.3}%
\psline{->}(0,1.25)(0,1.5)%
\rput(0.65,1.5){$^\circ$C}%
\pscircle(-0.5,0.5){0.2}%
\pscircle(0.5,0.5){0.2}%
% �lectrode
\psframe[framearc=0.5](2.25,2.5)(2.75,4.5)%
\psframe[fillstyle=solid,fillcolor=GrisClair](2.25,4)(2.75,4.5)%
\pscircle(2.5,2.25){0.25}
% fil
\psline[linearc=0.4,linewidth=0.1,linecolor=LightBlue]%
(2.5,4.5)(2.5,5)(1.25,4.75)(0,5)(0,3)}

\def\pst@dosage@support{%
\psframe(-1.5,0)(1.5,1.5)
{\psset{doubleline=true}%
\pscircle*[linecolor=red](0,0.75){0.2}%
\psframe(-1.25,0.3)(-0.75,1.2)%
\pscircle(1,0.75){0.25}}}

\def\pst@dosage@aimant{%
{\psset{fillstyle=solid,framearc=0.5}%
\psframe(-0.5,0.1)(0.5,0.4)%
\psframe(-0.05,0)(0.05,0.5)}}%

\def\pst@burette@graduation{%
\multido{\n=3.0+0.32}{25}{\psline(-0.25,\n)(0,\n)}}


%%%%%%% d�but pipette%%%%%%%%%%%%%%%%%%%%%%%%%%%%%%%

\def\pst@pipette@corps{%
\begingroup%
\psset{linewidth=0.053,linearc=0.05}%
\pspolygon(-0.05,0)(-0.15,0.2)(-0.15,6)%
(0.15,6)(0.15,0.2)(0.05,0)%
\psellipse[fillstyle=solid](0,6)(0.15,0.05)
\multido{\Niveau=0.5+0.5}{10}{%
\psline[linewidth=0.02](0,\Niveau)(0.1,\Niveau)}
\endgroup
}
\def\pst@goutte{%
\psbezier[linewidth=0.01](0,0)(-0.04,-0.083)(-0.083,-0.22)%
(0,-0.25)(0.083,-0.22)(0.04,-0.083)(0,0)}

%%%%%% Eprouvette  %%%%%%%%%%%%%%%%%%%%%%
\def\pst@eprouvette@corps{%
\psline(-0.5,5)(-0.5,0)(0.5,0)(0.5,4.9)(0.6,4.93)}

\def\pst@eprouvette@rebord{%
\pscurve(0.6,-0.07)
(0.498,0.008)
(0.483,0.026)
(0.433,0.05)%
(0.353,0.0707)
(0.25,0.0866)
(0.13,0.0966)(0,0.1)%
(-0.13,0.0966)
(-0.25,0.0866)(-0.353,0.0707)
(-0.433,0.05)%
(-0.483,0.026)
(-0.5,0)
(-0.483,-0.026)
(-0.353,-0.0707)%
(-0.25,-0.0866)
(-0.13,-0.096)
(0,-0.1)%
(0.13,-0.096)
(0.25,-0.0866)
(0.353,-0.0707)
(0.433,-0.05)%
(0.498,-0.026)
(0.6,-0.07)}

\def\pst@eprouvette@pied{%
\pscustom{\gsave%
\psarcn(-0.95,0.5){0.45}{0}{-90}
\psarc(-0.95,0.025){0.025}{90}{270}
\psline(-0.95,0)(0.95,0)
\psarc(0.95,0.025){0.025}{-90}{90}
\psarcn(0.95,0.5){0.45}{-90}{180}
\pscurve(0.5,0.4)(0,0.25)(-0.5,0.4)
\fill[fillstyle=solid]\grestore}
\psarcn(-0.95,0.5){0.45}{0}{-90}
\psarc(-0.95,0.025){0.025}{90}{270}
\psline(-0.95,0)(0.95,0)
\psarc(0.95,0.025){0.025}{-90}{90}
\psarcn(0.95,0.5){0.45}{-90}{180}
\pscurve(0.5,0.4)(0,0.25)(-0.5,0.4)
}

\def\pst@eprouvette@graduation{%
\multido{\nyD=0.5+0.5}{9}{%
\psline(-0.4,\nyD)(-0.1,\nyD)}
\multido{\nyd=0.5+0.1}{40}{%
\psline(-0.4,\nyd)(-0.25,\nyd)}}

%%%%%%%%%%% Distillation fractionn�e %%%%%%%%%%%%%%%%

\def\pst@DistillationFractionnee{
\def\pst@goutte{\psset{unit=0.6}\psbezier[linewidth=0.017](0,0)(0.25,-.4)(-.25,-.4)(0,0)}%
\def\colon{\psline[linearc=0.3](-.25,10.5)(-.25,8.5)(-.75,8)(-.75,2.5)(-.25,2)(-.25,0)%
\psline[linearc=0.3](0.25,0)(0.25,2)(0.75,2.5)(.75,8)(0.25,8.5)(0.25,9.36)%
\psline(0.25,8.5)(0.25,9.30)\psline(0.25,9.5)(0.25,10.5)%
\def\pointeG{\psline[linewidth=0.4\pslinewidth](0,0)(0.5,-.5)(0,-.1)}%
\multido{\i=3+1}{6}{\rput(-.75,\i){\pointeG}}%
\def\pointeD{\psline[linewidth=0.4\pslinewidth](0,0)(-.5,-.5)(0,-.1)}%
\multido{\i=3+1}{6}{\rput(0.75,\i){\pointeD}}%
{\psset{linestyle=none,fillstyle=hlines,hatchwidth=0.01,hatchsep=0.03}\psframe(-.25,10)(-.05,10.5)%
\psframe(.25,10)(0.05,10.5)}%
\multido{\n=2.5+1.0}{6}{\rput(-.25,\n){\pst@goutte}}%
\multido{\n=2.5+1.0}{6}{\rput(.25,\n){\pst@goutte}}
\rput(-0.25,-.1){\pst@goutte}\rput(0.25,-.1){\pst@goutte}}%
%r�frig�rant
\def\refri{\psarc(3,0){0.5}{90}{270}\psarc(7,0){0.5}{270}{90}%
\psframe[linestyle=none,fillstyle=solid](0,-.05)(9,0.05)%
\begingroup
\psset{linewidth=0.5\pslinewidth}
\psline(-.075,.05)(9,0.05)
\psline(0.02,-.05)(9,-.05)
%\psline(0,0.05)(9,0.05)%
\endgroup
\psline(3,0.5)(3.5,0.5)
\psline(3,-.5)(6.5,-.5)
\psline(3.6,0.5)(7,0.5)%
\begingroup
\psset{linewidth=0.5\pslinewidth}
\psline(3.5,0.5)(3.5,1.5)
\psline(3.6,0.5)(3.6,1.5)%
\psline(6.6,-.5)(6.6,-1.5)
\psline(6.5,-.5)(6.5,-1.5)
\endgroup
\psline(6.6,-.5)(7,-.5)%
\psset{arrowscale=2}\psline{->}(6.55,-2.5)(6.55,-1.5)\psline{->}(3.55,1.5)(3.55,2.5)}%
%ballon
\def\ballon{{\psset{linestyle=none,fillstyle=hlines,hatchwidth=0.007,hatchsep=0.03}%
\psframe(-.5,2.25)(-.25,3)\psframe(0.5,2.25)(0.25,3)}%
\psline(0.5,1.5)(0.5,3)%
\psline(-.5,1.5)(-.5,3)%
\psarc(0,0){1.581}{108.435}{71.565}%
\pscustom[linestyle=none]{%
\pscurve(-1.581,0)(-1.25,0.1)(-.75,-.1)(0,.1)(0.75,-.1)(1.25,.1)(1.581,0)%
\psarcn(0,0){1.581}{0}{180} \psset{style=\psk@Distillation@AspectMelange}
\fill%
}%
\psarc[linewidth=0.03](0,0){1.581}{180}{0}
}%
%thermom�tre
\def\thermo{\psline[linearc=0.05](-.05,.25)(-.05,2.75)(0.05,2.75)(0.05,0.25)%
\psframe[framearc=0.8,fillstyle=solid,fillcolor=red](-.15,-.25)(0.15,0.25)
\psline[linecolor=red](0,2)
}%
%chauffe ballon
\def\chauffe{\psline(1.17,2.5)(0,2.5)(0,0)(5.5,0)(5.5,2.5)(4.33,2.5)}
%�prouvette
\def\pst@eprouv{%
\begingroup
\psset{linewidth=1.5\pslinewidth}
\psline(-0.5,6)(-0.5,.5)
\psarcn(-1,0.5){.5}{0}{-90}\psline(-1,0)(1,0)
\psarcn(1,0.5){0.5}{270}{180}\psline(0.5,0.5)(0.5,6)
\endgroup}%

%%%%%%%%%%%%%% fin distillation fractionn�e %%%%%%%%%%%%%%

%%%%%%%%%%%%%%% �prouvette %%%%%%%%%%
\def\pst@eprouvette{%
\psclip{\pscustom[linestyle=none]{\pst@eprouv}}
\psframe*[linecolor=\psk@Distillation@CouleurDistillat](-1.5,-1)(1.5,1)
\pst@eprouv
\endpsclip}
\rput(0,-3){\rput(2.75,13.5){\thermo}
\rput(2.75,2.5){\ballon}
\rput(2.75,4){\colon}%
\rput(3,13.4){\rput{-45}(0,0){\refri}}
\rput(0,0){\chauffe}
\rput(9.4,0){\pst@eprouvette%
\begingroup
\psset{linewidth=0.5\pslinewidth}
\multido{\Niveau=1+1}{5}{%
\psline(-.4,\Niveau)(-.1,\Niveau)}
\multido{\Niveau=0.2+0.2}{24}{%
\psline(-.4,\Niveau)(-.2,\Niveau)}\endgroup}
\rput(9.37,7){\pst@goutte}\rput(9.37,6){\pst@goutte}\rput(9.37,4)
{\pst@goutte}\rput(9.37,1.2){\pst@goutte}
\rput(0.75,.75){\pscircle(0,0){.5}\multido{\i=0+30}{7}{\psline(0.55;\i)(0.65;\i)}%
\psline[linewidth=0.035]{->}(0,.45)}}}
%%%%%%%%%%%%%%%%%%% Fin �prouvette %%%%%%%%%%%%%%%%%%%%%%%%%%%%%%%%%%%%%%%%%%

%%%%%%%%%% Bec Bunsen sans grille %%%%%%%%%%%%%%%%

\def\pst@BecBunsen{%
\psset{dimen=middle,linewidth=0.053}%
\psframe(-1.25,0)(1.25,0.25)%
\psframe(-.5,1.25)(0.5,2.25)%
\multido{\n=-0.3+0.3}{3}{%
\pscircle(\n,1.75){0.1}}%
\psframe(-.25,2.25)(0.25,4.25)%
\psline(0.25,1.25)(0.25,0.5)(1.25,0.25)%
\psline(-1.25,0.25)(-.25,0.5)(-0.25,0.75)%
\psline(-2.25,0.75)(-.25,0.75)%
\psline(-2.25,1)(-.25,1)%
\psellipse(-.25,0.875)(0.1,0.125)%
\psframe[fillstyle=solid,linestyle=none](-2.25,0.75)(-0.25,1)%
\psline(-2.25,0.75)(-0.25,0.75)%
\psline(-2.25,1)(-0.25,1)(-.25,1.25)%
\pscurve(-0.25,0.5)(0,0.4)(0.25,0.5)
%flamme
\rput(0,4.25){%
\psclip{\psbezier[linestyle=none,fillstyle=gradient,gradmidpoint=0,%
gradbegin=OrangePale,gradend=yellow]%
(-0.25,0)(-0.35,0.5)(-0.4,0.75)%
(-0.35,1)(-0.25,1.5)(0.5,2)%
(0.25,1.5)(0.35,1)(0.4,0.75)%
(0.35,0.5)(0.25,0)(0,0)}%
\pspolygon[linestyle=none,fillstyle=gradient,gradmidpoint=0,gradbegin=cyan,gradend=white]%
(-0.25,0)(0.25,0)(0,1)%
\endpsclip}}

%%%%BecBunsen avec grille%%%%%%%%%%%%

\def\pst@FlammeGrille{%
\pscustom[linestyle=none]{%
\psarc(0,0.75){0.75}{270}{0}
\psarcn(1.5,0.75){0.75}{180}{90}
\psline(1.5,1.5)(-1.5,1.5)
\psarcn(-1.5,0.75){0.75}{90}{0}
\psarc(0,0.75){0.75}{180}{270}
\fill[style=flammeEtGrille]}}

\def\pst@BecBunsenGrille{%
\psframe(-1.25,0)(1.25,0.25)%
\psframe(-.5,1.25)(0.5,2.25)%
\multido{\n=-0.3+0.3}{3}{%
\pscircle(\n,1.75){0.1}}%
\psframe(-.25,2.25)(0.25,4.25)%
\psline(0.25,1.25)(0.25,0.5)(1.25,0.25)%
\psline(-1.25,0.25)(-.25,0.5)(-0.25,0.75)%
\psline(-2.25,0.75)(-.25,0.75)%
\psline(-2.25,1)(-.25,1)%
\psellipse(-.25,0.875)(0.1,0.125)%
\psframe[fillstyle=solid,linestyle=none](-2.25,0.75)(-0.25,1)%
\psline(-2.25,0.75)(-0.25,0.75)%
\psline(-2.25,1)(-0.25,1)(-.25,1.25)%
\pscurve(-0.25,0.5)(0,0.4)(0.25,0.5)
%flamme
\rput(0,4.25){%
\psclip{\pst@FlammeGrille}%
\pspolygon[linestyle=none,fillstyle=gradient,gradmidpoint=0,gradbegin=cyan,gradend=white]%
(-0.25,0)(0.25,0)(0,1)%
\endpsclip}
\psline[linestyle=dashed,linewidth=0.08](-2,5.75)(2,5.75)}

%%%%%%%%% fin bec Bunsen avec grille %%%%%%%%%%%%%%%%%%%%%%%%%%%

\def\pst@ballon{
{\psset{linewidth=0.053}
\psarcn(-0.7,3.4){0.2}{90}{0}
\psline(-0.5,3.4)(-0.5,2.4)
\psarc(0,1.25){1.25}{113}{67}
\psline(0.5,2.4)(0.5,3.4)
\psarcn(0.7,3.4){0.2}{180}{90}
\psellipse(0,3.6)(0.7,0.1)}}

\def\pst@ballonGrille{
{\psset{linewidth=0.053}
\psarcn(-0.7,2.15){0.2}{90}{0}
\psline(-0.5,2.15)(-0.5,1.4)
\psarc(0,0){1.25}{113}{67}
\psline(0.5,1.15)(0.5,3.4)
\psarcn(0.7,2.15){0.2}{180}{90}
\psellipse(0,2.35)(0.7,0.1)}}

\def\pst@TubeEssais@Corps{{%
\psset{linewidth=0.053}
\psline(-0.5,3.5)(-0.5,0.5)
\psarc(0,0.5){0.5}{180}{0}
\psline(0.5,0.5)(0.5,3.5)
\psellipse[linewidth=0.08](0,3.5)(0.5,0.1)}}

\def\pst@TubeEssais@Droit{%
\psline[linewidth=0.5\pslinewidth](0.05,-1)(0.05,5)%
\psline[linewidth=0.5\pslinewidth](-0.05,-1)(-0.05,5)%
\pst@TubeEssais@Bouchon
\psframe[framearc=0.5,fillstyle=solid,dimen=middle,linewidth=0.5\pslinewidth](-0.05,5)(0.05,0.5)%
\psline[linecolor=white,linewidth=0.14](-0.1,5)(0.1,5)}

\def\pst@TubeEssais@Coude{%
\psline[linewidth=0.5\pslinewidth](0.05,-1)(0.05,0)%
\psline[linewidth=0.5\pslinewidth](-0.05,-1)(-0.05,0)%
\pst@TubeEssais@Bouchon
\pspolygon[fillstyle=solid,linearc=0.05,linewidth=0.5\pslinewidth](-0.05,0.5)(-0.05,2)(8,2)(8,1.9)
(0.05,1.9)(0.05,0.5)%
\psline[linecolor=white,linewidth=0.14](7.95,1.9)(8.05,2)}

\def\pst@TubeEssais@CoudeU{%
\psline[linewidth=0.5\pslinewidth](0.05,-1)(0.05,0)%
\psline[linewidth=0.5\pslinewidth](-0.05,-1)(-0.05,0)%
\pst@TubeEssais@Bouchon
\pspolygon[fillstyle=solid,linearc=0.05,linewidth=0.5\pslinewidth](-0.05,0.5)(-0.05,2)(4.5,2)(4.5,-3)
(4.4,-3)(4.4,1.9)(0.05,1.9)(0.05,0.5)%
\psline[linecolor=white,linewidth=0.14](4.3,-3)(4.6,-3)}

\def\pst@TubeEssais@CoudeUB{%
\psline[linewidth=0.5\pslinewidth](0.05,-1)(0.05,0)%
\psline[linewidth=0.5\pslinewidth](-0.05,-1)(-0.05,0)%
\pst@TubeEssais@Bouchon
\pspolygon[fillstyle=solid,linearc=0.05,linewidth=0.5\pslinewidth](-0.05,0.5)(-0.05,2)(4.5,2)(4.5,-9)
(4.4,-9)(4.4,1.9)(0.05,1.9)(0.05,0.5)%
\psline[linecolor=white,linewidth=0.14](4.3,-9)(4.6,-9)}

\def\pst@TubeRecourbe{%
  \psline[linewidth=0.5\pslinewidth](0.05,-1)(0.05,0)%
  \psline[linewidth=0.5\pslinewidth](-0.05,-1)(-0.05,0)%
  \pst@TubeEssais@Bouchon
  \begingroup
  \psset{linewidth=0.5\pslinewidth}
  \psline(-0.05,0.5)(-0.05,1)
  \psarcn(0.5,1){0.55}{180}{90}
  \psline(0.5,1.55)(5.5,1.55)
  \psarcn(5.5,1){0.55}{90}{0}
  \psline(6.05,1)(6.05,-8.8)
  \psarc(6.5,-8.8){0.45}{180}{0}
  \psline(6.95,-8.8)(6.95,-8.3)
  \psline(7.05,-8.3)(7.05,-8.8)
  \psarcn(6.5,-8.8){0.55}{0}{180}
  \psline(5.95,-8.8)(5.95,1)
  \psarc(5.5,1){0.45}{0}{90}
  \psline(5.5,1.45)(0.5,1.45)
  \psarc(0.5,1){.45}{90}{180}
  \psline(0.05,1)(0.05,0.5)
  \psline(-0.05,.5)(0.05,0.5)\endgroup
  \psframe*[linecolor=white](-0.025,0.52)(0.020,0.6)
}

\def\pst@TubeRecourbe@Court{%
  \psline[linewidth=0.5\pslinewidth](0.05,-1)(0.05,0)%
  \psline[linewidth=0.5\pslinewidth](-0.05,-1)(-0.05,0)%
  \pst@TubeEssais@Bouchon \psline(-0.05,0.5)(-0.05,1)
  \begingroup
  \psset{linewidth=0.5\pslinewidth}
  \psarcn(0.5,1){0.55}{180}{90}
  \psline(0.5,1.55)(3.5,1.55)
  \psarcn(3.5,1){0.55}{90}{0}
  \psline(4.05,1)(4.05,-2.8)
  \psarc(4.5,-2.8){0.45}{180}{0}
  \psline(4.95,-2.8)(4.95,-2.3)
  \psline(5.05,-2.3)(5.05,-2.8)
  \psarcn(4.5,-2.8){0.55}{0}{180}
  \psline(3.95,-2.8)(3.95,1)
  \psarc(3.5,1){0.45}{0}{90}
  \psline(3.5,1.45)(0.5,1.45)
  \psarc(0.5,1){.45}{90}{180}
  \psline(0.05,1)(0.05,0.5)
  \psline(-0.05,.5)(0.05,0.5)
  \endgroup
  \psframe*[linecolor=white](-0.025,0.52)(0.020,0.6)}

\def\pst@TubeEssais@Bouchon{%
  \begingroup
  \psset{fillstyle=solid,fillcolor=GrisClair}
  \psellipse(0,-0.3)(0.4,0.1)
  \pspolygon(-0.4,-0.3)(-0.6,0.5)(0.6,0.5)(0.4,-0.3)%
  \psellipse[linestyle=none,dimen=outer,linewidth=0.07](0,-0.3)(0.4,0.1)%
  \psellipse(0,0.5)(0.6,0.1)
  \endgroup
  \pscurve[linewidth=0.07](-0.5,0)(0,-0.07)(0.5,0)%
}
%
\def\pst@TubeEssais@DoubleTube{%
\begingroup
\psset{linewidth=0.5\pslinewidth}
\psline(-0.3,-3)(-0.3,0)
\psline(-.2,-3)(-0.2,0)
\psline(0.25,-1)(0.25,0)%
\psline(.15,-1)(.15,0)%
\endgroup
\pst@TubeEssais@Bouchon
\rput(0.2,0){\pst@TubeRecourbe@Court@DoubleTube}
\rput(0,0.5){%
\begingroup
\psset{linewidth=0.5\pslinewidth}
\pspolygon[fillstyle=solid,linearc=0.05]%
(-.3,0)(-0.3,2)(-0.5,2.2)(-0.5,3)(0,3)%
(0,2.2)(-0.2,2)(-0.2,0)
\pspolygon(0.1,1.55)(-.5,1.55)(-0.5,1.7)(-.55,1.7)%
(-.55,1.3)(-0.5,1.3)(-0.5,1.55)(-.5,1.45)(.1,1.45)
\psframe[fillstyle=solid,framearc=0.2](-0.4,1.3)(-0.1,1.7)
\endgroup\psellipse[fillstyle=solid](-0.25,3)(0.3,0.1)}}


\def\pst@TubeRecourbe@Court@DoubleTube{
\begingroup
\psset{linewidth=0.5\pslinewidth}
\psline(-0.05,0.5)(-0.05,1)
\psarcn(0.5,1){0.55}{180}{90}
\psline(0.5,1.55)(3.5,1.55)
\psarcn(3.5,1){0.55}{90}{0}
\psline(4.05,1)(4.05,-2.8)
\psarc(4.5,-2.8){0.45}{180}{0}
\psline(4.95,-2.8)(4.95,-2.3)
\psline(5.05,-2.3)(5.05,-2.8)
\psarcn(4.5,-2.8){0.55}{0}{180}
\psline(3.95,-2.8)(3.95,1)
\psarc(3.5,1){0.45}{0}{90}
\psline(3.5,1.45)(0.5,1.45)
\psarc(0.5,1){.45}{90}{180}
\psline(0.05,1)(0.05,0.5)
\psline(-0.05,.5)(0.05,0.5)
\endgroup
\psframe*[linecolor=white](-0.025,0.52)(0.018,0.8)}

\def\pst@TubeEssais@Pince{
\pscustom{%
\pscurve(0.5,0)(0,-0.1)(-0.5,0)
\psline(-0.5,0)(-0.8,0)
\psarc(-0.5,0){0.3}{180}{270}
\psline(-.5,-.3)(2.5,-.3)(2.4,-.2)(1,0)(0.5,0)
\fill[fillstyle=solid,fillcolor=OrangePale]}
\pscustom{%
\psline(-0.8,0)(-0.8,-0.3)
\psarc(-0.5,-0.3){0.3}{180}{270}
\psline(-0.5,-0.6)(2.5,-.6)(2.5,-.3)(-.5,-.3)
\psarcn(-.5,0){0.3}{-90}{180}
\fill[fillstyle=solid,fillcolor=OrangePale]}
\pscustom{%
\psline(-.5,0.05)(-.8,0.05)
\psarcn(-0.5,0.05){0.25}{180}{90}
\psline(-.5,0.3)(-0.5,0.05)
\fill[fillstyle=solid,fillcolor=OrangePale]}
\pscustom{%
\psline(0.5,0.3)(4.9,0.3)(5,0.2)(1,0)(0.5,0)(0.5,0.3)
\fill[fillstyle=solid,fillcolor=OrangePale]}
\pscustom{%
\psline(0.5,0)(5,0.2)(5,0)(2,-0.2)(1,0)
\fill[fillstyle=solid,fillcolor=Marron]}
\psellipse[linewidth=1.5\pslinewidth](1,0)(0.3,0.1)
\psline[linewidth=1.5\pslinewidth](0.7,0)(0.4,-.3)(0.4,-.6)}


\def\pst@FIOLEJAUGE{%
\psline[linearc=0.1,linewidth=0.053](-0.21,3.5)(-0.1,3.5)
(-0.1,1.5)(-0.75,0)(0.75,0)(0.1,1.5)(0.1,3.5)(0.21,3.5)
\psline[linewidth=0.02](-0.1,2.5)(0.1,2.5)}



\def\Cristallisoir{%
\psline(-1.9,2)(-2,2)
\psarc(-2,1.95){0.05}{90}{270}
\psline(-2,1.9)(-2,1)
\psarc(-1,1){1}{180}{270}
\psline(-1,0)(1,0)
\psarc(1,1){1}{270}{0}
\psline(2,1)(2,1.9)
\psarc(2,1.95){0.05}{270}{90}
\psline(2,2)(1.9,2)
\psline(1.9,2)(1.9,1)
\psarcn(1,1){0.9}{0}{-90}
\psline(1,0.1)(-1,0.1)
\psarcn(-1,1){0.9}{270}{180}
\psline(-1.9,1)(-1.9,2)}

\def\InterieurCristallisoir{\pscustom{\psline(1.9,2)(1.9,1)
\psarcn(1,1){0.9}{0}{-90}
\psline(1,0.1)(-1,0.1)
\psarcn(-1,1){0.9}{270}{180}
\psline(-1.9,1)(-1.9,2)}}

\def\pst@Cristallisoir{%
\psclip{\InterieurCristallisoir}
\psframe*[linecolor=BleuClair](-3,-1)(3,1.5)
\endpsclip\Cristallisoir}

\def\pst@TubeRenverse{%
  \begingroup
  \psset{linewidth=0.07}
  \psframe[linestyle=none,fillstyle=solid,fillcolor=BleuClair](-0.5,0)(0.5,5.5)
  \psline(-0.5,0)(-0.5,6)
  \psarcn(0,6){0.5}{180}{0}
  \psline(0.5,6)(0.5,0)
\endgroup}

\def\pst@ReposeTube{%
\begingroup
\psset{linewidth=0.1,linecolor=lightgray}
\psarc(0,-0.25){1}{100}{140}
\psarc(0,-.25){1}{150}{160}
\psarc(0,-0.25){1}{20}{80}
\endgroup}

\def\pst@RebordBecher{%
\psellipse[linewidth=0.053](0,2.55)(1.1,.1)
\multido{\nDiv=0.5+0.5}{4}{
\psline[linewidth=0.03](-.75,\nDiv)(-.4,\nDiv)}
\multido{\nSubDiv=0.5+0.1}{16}{
\psline[linewidth=0.02](-0.75,\nSubDiv)(-.5,\nSubDiv)}}

\def\pst@Becher@Corps{%
\psline[linewidth=0.053,linearc=0.5](-1,2.5)(-1,0)(1,0)(1,2.5)}

\def\pst@Erlen@Corps{%
\psline[linearc=0.3,linewidth=0.053](-0.5,3.5)(-0.5,2.5)%
(-1.5,0)(1.5,0)(0.5,2.5)(0.5,3.5)}

\def\pst@Flacon@Corps{%
{\psset{linewidth=0.053}
\psarcn(-0.6,3.4){0.1}{90}{0}
\psline(-0.5,3.4)(-0.5,3.1)
\psarcn(-0.6,3.1){0.1}{0}{-90}
\psarc(-0.6,2.6){0.4}{90}{180}
\psline(-1,2.6)(-1,0.2)
\psarc(-.8,0.2){0.2}{180}{270}
\psline(-.8,0)(.8,0)
\psarc(.8,0.2){0.2}{-90}{0}
\psline(1,0.2)(1,2.6)
\psarc(0.6,2.6){0.4}{0}{90}
\psarcn(0.6,3.1){0.1}{-90}{180}
\psline(0.5,3.1)(0.5,3.4)
\psarcn(0.6,3.4){0.1}{180}{90}}}

%%%%%%%%%%% d�but  entonnoir %%%%%%%%%%%%%%%%%%%%%%%

\def\pst@Entonnoir@Contour{%
\psline(-0.07,0)(-0.07,1)(-1,2.5)
\psline(1,2.5)(0.07,1)(0.07,0.1)}


\def\pst@Entonnoir@Corps{%
\psclip{\pscustom[linestyle=none]{\pst@Entonnoir@Contour}}
\psframe[fillstyle=solid](-2,-1)(2,3)
\endpsclip
{\psset{linearc=0.25}
\pst@Entonnoir@Contour}
\psellipse(0,2.5)(1,0.1)}
%%%%%%%%%%%%%%%%%%%%%%%%%%%%%%%%%%%%%%%%%%%%%%%%%%%%
%%%%%%%%%%% fin Entonnoir %%%%%%%%%%%%%%%%%%%%%%%%%
%%%%%%%%% R�frig�rant � boulles %%%%%%%%%%%%%%%%%%

\def\pst@TigeBoulles{%
\pscustom[linestyle=none]{%
\psarcn(-0.35,7.5){.1}{90}{0}
\psline(-.25,7.5)(-.25,6.6)
\psarc(-.15,6.6){0.1}{180}{270}
\psarcn(-.15,6.4){0.1}{90}{0}
\psline(-.05,6.4)(-0.05,6)
\multido{\nB=5.75+-0.75,\Nt=5.50+-0.75,\nt=5.25+-.75}{7}{%
\psarc(-0.05,\nB){0.25}{90}{270}
\psline(-0.05,\Nt)(-0.05,\nt)}
\psline(-0.05,0.75)(-0.05,-1.5)
\psline(0.05,-1.5)(0.05,1)
\multido{\nB=1.25+0.75,\nt=1.50+0.75,\Nt=1.75+0.75}{7}{
\psarc(0.05,\nB){0.25}{270}{90}
\psline(0.05,\nt)(0.05,\Nt)}
\psline(0.05,6)(0.05,6.4)
\psarcn(0.15,6.4){0.1}{180}{90}
\psarc(0.15,6.6){0.1}{-90}{0}
\psline(0.25,6.6)(0.25,7.5)
\psarcn(0.35,7.5){0.1}{180}{90}
\fill[fillstyle=solid]}
\psarcn(-0.35,7.5){.1}{90}{0}
\psline(-.25,7.5)(-.25,6.6)
\psarc(-.15,6.6){0.1}{180}{270}
\psarcn(-.15,6.4){0.1}{90}{0}
\psline(-.05,6.4)(-0.05,6)
\multido{\nB=5.75+-0.75,\Nt=5.50+-0.75,\nt=5.25+-.75}{7}{%
\psarc(-0.05,\nB){0.25}{90}{270}
\psline(-0.05,\Nt)(-0.05,\nt)}
\psline(-0.05,0.75)(-0.05,-1.5)
\psline(0.05,-1.5)(0.05,1)
\multido{\nB=1.25+0.75,\nt=1.50+0.75,\Nt=1.75+0.75}{7}{
\psarc(0.05,\nB){0.25}{270}{90}
\psline(0.05,\nt)(0.05,\Nt)}
\psline(0.05,6)(0.05,6.4)
\psarcn(0.15,6.4){0.1}{180}{90}
\psarc(0.15,6.6){0.1}{-90}{0}
\psline(0.25,6.6)(0.25,7.5)
\psarcn(0.35,7.5){0.1}{180}{90}
\rput(0,-1.6){\pst@goutte}}

\def\pst@CorpsRefri{
\pscustom[linestyle=none]{
\psarc(0,5.75){0.5}{0}{180}
\psline(-0.5,5.75)(-0.5,5.5)(-1.5,5.5)
\psline(-1.5,5.3)(-.5,5.3)(-.5,1.7)(-1.5,1.7)
\psline(-1.5,1.5)(-.5,1.5)(-0.5,1)
\psarc(0,1){0.5}{180}{0}
\psline(0.5,1)(0.5,5.75)
%\fill[fillstyle=solid,fillcolor=LightBlue]
\fill[fillstyle=gradient,gradbegin=cyan,gradend=OrangePale,gradmidpoint=0]
}
{\psset{linewidth=1.1\pslinewidth}
\psarc(0,5.75){0.5}{0}{180}
\psline(-0.5,5.75)(-0.5,5.5)(-1.5,5.5)
\psline(-0.5,5.75)(-0.5,5.5)(-1.5,5.5)
\psline(-1.5,5.3)(-.5,5.3)(-.5,1.7)(-1.5,1.7)
\psline(-1.5,1.5)(-.5,1.5)(-0.5,1)
\psarc(0,1){0.5}{180}{0}
\psline(0.5,1)(0.5,5.75)}
\psline[linecolor=cyan]{->}(-2.5,1.6)(-1.6,1.6)
\psline[linecolor=OrangePale]{<-}(-2.5,5.4)(-1.6,5.4)}
\def\pst@Refrigerant@Boulles{
\rput(0,0.2){\pst@CorpsRefri\pst@TigeBoulles}
\pst@TubeEssais@Bouchon
\pscustom{
\psline[linearc=0.025](-0.05,0.6)(-0.05,0.5)(0.05,0.5)(0.05,0.6)
\fill[fillstyle=solid]}}

%%%%%%%%%%%%%%%%%%%fin r�frig�rant � boulles %%%%%%%%%%%

%%%%%%%%%%%%% Chauffe-Ballon %%%%%%%%%%%%%%%%%%%%%%%%%%%
\def\pst@ChauffeBallon{
\psframe[fillstyle=solid](-2,-1)(2,1)
\pscircle(-1.25,0){0.4}
\rput(-1.25,0){
\multido{\iA=0+30}{7}{
\psline(0.45;\iA)(0.6;\iA)}
\psline[arrowinset=0]{->}(0,0)(0.3;60)}
\psframe[doubleline=true,fillstyle=solid,fillcolor=gray](1,-.5)(1.5,0.5)
\pscircle*[linecolor=red](1.25,0.7){0.1}}

%%%%%%%%%%%% Fin de Chauffe-Ballon %%%%%%%%%%%%%%%%%%%%%%

\def\pst@Ballon@Reflux{%
\psclip{%
\pscustom[linestyle=none]{\pst@ballon}}
\pscustom[fillstyle=gradient,gradbegin=cyan,gradend=white,gradmidpoint=0,linecolor=cyan]
{\pscurve(-1.5,1.5)(-1.375,1.55)%
(-1.25,1.5)(-1.125,1.45)%
(-1,1.5)(-.875,1.55)%
(-.75,1.5)(-.625,1.45)%
(-.5,1.5)(-.375,1.55)%
(-.25,1.5)(-0.125,1.45)%
(0,1.5)(0.125,1.55)%
(.25,1.5)(0.375,1.45)%
(0.5,1.5)(0.625,1.55)%
(0.75,1.5)(0.875,1.45)%
(1,1.5)(1.125,1.55)%
(1.25,1.5)(1.375,1.45)(1.5,1.5)%
\psline(1.5,1.5)(1.5,0)(-1.5,0)(-1.5,1.5)}
\endpsclip%
\pst@ballon
}

%
\definecolor{Beige} {rgb}{0.96,0.96,0.86}
\definecolor{GrisClair} {rgb}{0.8,0.8,0.8}
\definecolor{GrisTresClair} {rgb}{0.9,0.9,0.9}
\definecolor{OrangeTresPale}{cmyk}{0,0.1,0.3,0}
\definecolor{OrangePale}{cmyk}{0,0.2,0.4,0}
\definecolor{BleuClair}{cmyk}{0.2,0,0,0}
\definecolor{LightBlue}{rgb}{.68,.85,.9}
\definecolor{Copper}{cmyk}{0,0.9,0.9,0.2}
\definecolor{Marron}{cmyk}{0,0.3,0.5,.3}
\SpecialCoor

\newdimen{\BulleX}
\newdimen{\BulleY}
\newdimen\GrenailleX
\newdimen\GrenailleY
\newdimen\TournureX
\newdimen\TournureY
\newdimen\RAYONBULLE
\newdimen\ClouX
\newdimen\ClouY
\newdimen{\pst@FilamentWidth}
\newdimen\pst@dimA
\newcount\lab@AngleRotation


\newif\ifPst@burette
\define@key[psset]{pst-labo}{burette}[true]{\@nameuse{Pst@burette#1}}%
\define@key[psset]{pst-labo}{couleurReactifBurette}{\edef\psk@Burette@CouleurReactif{#1}}%
\define@key[psset]{pst-labo}{niveauReactifBurette}{%
  \pst@cnta=#1\relax 
  \ifnum\pst@cnta>25 
    \typeout{Niveau must be 25 and not `\the\pst@cnta'. Value 25 forced.} \pst@cnta=25 
  \fi%
  \def\psk@burette@niveauReactif{#1}%
}%
\define@key[psset]{pst-labo}{reactifBurette}{\def\psk@Reactif@Burette{#1}}%
\define@key[psset]{pst-labo}{reactifBecher}{\def\psk@Reactif@Becher{#1}}%

\newif\ifpst@dosage@PHmetre
\define@key[psset]{pst-labo}{phmetre}[true]{\@nameuse{pst@dosage@PHmetre#1}}%
%
\define@key[psset]{pst-labo}{couleurReactif}{\edef\psk@pipette@couleurReactif{#1}}%
\define@key[psset]{pst-labo}{niveauReactif}{%
  \pst@cntg=#1\relax
  % tubePenche must be between -65 and 65
  \ifnum\pst@cntg>10 
    \typeout{Niveau must be 10 and not `\the\pst@cntg'. Value 10 forced.} \pst@cntg=10
  \fi
  \def\psk@pipette@niveauReactif{#1}%
}%

\define@key[psset]{pst-labo}{CouleurDistillat}{\def\psk@Distillation@CouleurDistillat{#1}}
\define@key[psset]{pst-labo}{AspectMelange}{\def\psk@Distillation@AspectMelange{#1}}

% Parameters of \TubeEssais
% -------------------------
\newif\ifPst@TubeEssais@bouchon
\define@key[psset]{pst-labo}{bouchon}[true]{\@nameuse{Pst@TubeEssais@bouchon#1}}%
%
\newif\ifPst@TubeEssais@tubeDroit
\define@key[psset]{pst-labo}{tubeDroit}[true]{\@nameuse{Pst@TubeEssais@tubeDroit#1}}

\newif\ifPst@TubeEssais@tubeCoude
\define@key[psset]{pst-labo}{tubeCoude}[true]{\@nameuse{Pst@TubeEssais@tubeCoude#1}}

\newif\ifPst@TubeEssais@tubeCoudeU
\define@key[psset]{pst-labo}{tubeCoudeU}[true]{\@nameuse{Pst@TubeEssais@tubeCoudeU#1}}

\newif\ifPst@TubeEssais@tubeCoudeUB
\define@key[psset]{pst-labo}{tubeCoudeUB}[true]{\@nameuse{Pst@TubeEssais@tubeCoudeUB#1}}

\newif\ifPst@TubeEssais@tubeRecourbe
\define@key[psset]{pst-labo}{tubeRecourbe}[true]{\@nameuse{Pst@TubeEssais@tubeRecourbe#1}}

\newif\ifPst@TubeEssais@tubeRecourbe@Court
\define@key[psset]{pst-labo}{tubeRecourbeCourt}[true]{\@nameuse{Pst@TubeEssais@tubeRecourbe@Court#1}}

\newif\ifPst@TubeEssais@DoubleTube
\define@key[psset]{pst-labo}{doubletube}[true]{\@nameuse{Pst@TubeEssais@DoubleTube#1}}

\newif\ifPst@TubeEssais@Refrigerant@Boulles
\define@key[psset]{pst-labo}{refrigerantBoulles}[true]{\@nameuse{Pst@TubeEssais@Refrigerant@Boulles#1}}

\newif\ifPst@TubeEssais@Pince
\define@key[psset]{pst-labo}{pince}[true]{\@nameuse{Pst@TubeEssais@Pince#1}}%

\newif\ifpst@barbotage
\define@key[psset]{pst-labo}{barbotage}[true]{\@nameuse{pst@barbotage#1}}%

\newif\ifpst@recuperationGaz
\define@key[psset]{pst-labo}{recuperationGaz}[true]{\@nameuse{pst@recuperationGaz#1}}%

\newif\ifpst@tubeSeul
\define@key[psset]{pst-labo}{tubeSeul}[true]{\@nameuse{pst@tubeSeul#1}}%

\newif\ifpst@tubeDegagementDroit
\define@key[psset]{pst-labo}{tubeDegagementDroit}[true]{\@nameuse{pst@tubeDegagementDroit#1}}%

\newif\ifpst@becBunsen
\define@key[psset]{pst-labo}{becBunsen}[true]{\@nameuse{pst@becBunsen#1}}%

\define@key[psset]{pst-labo}{Numero}[true]{\@namedef{pst@Numero}{#1}}%

\newif\ifPst@Etiquette
\define@key[psset]{pst-labo}{etiquette}[true]{\@nameuse{Pst@Etiquette#1}}%

\newif\ifPst@Agitateur@Magnetique
\define@key[psset]{pst-labo}{agitateurMagnetique}[true]{\@nameuse{Pst@Agitateur@Magnetique#1}}%

% Straight or oblique tube
\define@key[psset]{pst-labo}{tubePenche}{%
  \pst@cntg=#1\relax
  % tubePenche must be between -65 and 65
  \ifnum\pst@cntg<-65
    \typeout{Angle must be >=-65 and not `\the\pst@cntg'. Value -65 forced.}%
    \pst@cntg=65
  \fi%
  \ifnum\pst@cntg>65 
    \typeout{Angle must be <=65 and not `\the\pst@cntg'. Value 65 forced.}%
    \pst@cntg=65
  \fi%
  \edef\psk@TubeEssais@tubePenche{\the\pst@cntg}}
% Level of liquids
\define@key[psset]{pst-labo}{niveauLiquide1}{\@namedef{psk@TubeEssais@niveauLiquide1}{#1}}
\define@key[psset]{pst-labo}{niveauLiquide2}{\@namedef{psk@TubeEssais@niveauLiquide2}{#1}}
\define@key[psset]{pst-labo}{niveauLiquide3}{\@namedef{psk@TubeEssais@niveauLiquide3}{#1}}
% Aspect of liquids
\define@key[psset]{pst-labo}{aspectLiquide1}{\@namedef{psk@TubeEssais@aspectLiquide1}{#1}}
\define@key[psset]{pst-labo}{aspectLiquide2}{\@namedef{psk@TubeEssais@aspectLiquide2}{#1}}
\define@key[psset]{pst-labo}{aspectLiquide3}{\@namedef{psk@TubeEssais@aspectLiquide3}{#1}}
% Substance
\define@key[psset]{pst-labo}{substance}{\def\@tempa{#1}\let\psk@TubeEssais@Subtance\@tempa}
%Solides
\define@key[psset]{pst-labo}{solide}{\def\@tempa{#1}\let\psk@TubeEssais@Solide\@tempa}
%
%  0->TUBE  1->BALLON   2->BECHER  3->ERLEN  4->FLACON  5->fioleJauge  6->Verre
\def\pst@@TUBE{tube}
\def\pst@@BALLON{ballon}
\def\pst@@BECHER{becher}
\def\pst@@ERLEN{erlen}
\def\pst@@FLACON{flacon}
\def\pst@@fioleJauge{fioleJauge}
\def\pst@@Verre{verre}
%
\define@key[psset]{pst-labo}{glassType}{%
  \def\pst@tempA{#1}
  \edef\psk@glassType{%
    \ifx\pst@@TUBE\pst@tempA 0 \else                     % TUBE
    \ifx\pst@@BALLON\pst@tempA 1 \else                   % BALLON
    \ifx\pst@@BECHER\pst@tempA 2 \else                   % BECHER
    \ifx\pst@@ERLEN\pst@tempA 3  \else                   % ERLEN
    \ifx\pst@@FLACON\pst@tempA 4 \else                   % FLACON
    \ifx\pst@@fioleJauge\pst@tempA 5 \else               % fioleJauge
    \ifx\pst@@Verre\pst@tempA 6 \else 0                  % Verre
      \typeout{pst-labo: unknown glassType -> #1}
      \typeout{          using type "tube" instead.}
    \fi\fi\fi\fi\fi\fi\fi%                               % default is tube
}}
%
% Default values
\newpsstyle{aspectLiquide1}{linestyle=none,fillstyle=solid,fillcolor=cyan}
\newpsstyle{aspectLiquide2}{linestyle=none,fillstyle=solid,fillcolor=yellow}
\newpsstyle{aspectLiquide3}{linestyle=none,fillstyle=solid,fillcolor=magenta}
\newpsstyle{BilleThreeD}{linestyle=none,fillstyle=gradient,gradmidpoint=0,gradend=white,GradientCircle=true}
\newpsstyle{Champagne}{linestyle=none,fillstyle=solid,fillcolor=Beige}
\newpsstyle{Sang}{linestyle=none,fillstyle=solid,fillcolor=red}
\newpsstyle{Cobalt}{linewidth=0.2,fillstyle=solid,fillcolor=blue}
\newpsstyle{Huile}{linestyle=none,fillstyle=solid,fillcolor=yellow}
\newpsstyle{Vinaigre}{linestyle=none,fillstyle=solid,fillcolor=magenta}
\newpsstyle{Diffusion}{linestyle=none,fillstyle=gradient,gradmidpoint=0}
\newpsstyle{DiffusionMelange2}{fillstyle=gradient,gradbegin=white,gradend=red,gradmidpoint=0,linecolor=red}
\newpsstyle{flammeEtGrille}{linestyle=none,fillstyle=gradient,gradmidpoint=0,gradbegin=OrangePale,gradend=yellow}
\newpsstyle{rayuresJaunes}{fillstyle=hlines,linecolor=yellow,hatchcolor=yellow}
\newpsstyle{DiffusionBleue}{fillstyle=gradient,gradmidpoint=0,linestyle=none,gradbegin=green,gradend=cyan}


\psset[pst-labo]{glassType=tube,bouchon=false,tubeDroit=false,tubeCoude=false,tubePenche=0,substance=\relax,%
  niveauLiquide1=50,aspectLiquide1=aspectLiquide1,%
  niveauLiquide2=0,aspectLiquide2=aspectLiquide2,%
  niveauLiquide3=0,aspectLiquide3=aspectLiquide3,barbotage=false,tubeSeul=false,%
  tubeDegagementDroit=false,becBunsen=true,tubeCoudeU=false,tubeCoudeUB=false,%
  tubeRecourbe=false,tubeRecourbeCourt=false,%
  pince=false,refrigerantBoulles=false,%
  doubletube=false,recuperationGaz=false,Numero={},etiquette=false,%
  agitateurMagnetique=true,solide=\relax,%
  couleurReactif=OrangePale,niveauReactif=2,burette=true,%
  couleurReactifBurette=OrangeTresPale,niveauReactifBurette=20,%
  reactifBurette={},reactifBecher={},phmetre=false,%
  CouleurDistillat=yellow,AspectMelange=DiffusionBleue}
  
%
\def\pstDosage{\pst@object{pst@Dosage}}
\def\pst@Dosage@i{%
  \addbefore@par{dimen=middle}%
  \begin@SpecialObj
  \ifPst@burette
    \ifpst@dosage@PHmetre\pspicture(-5,0)(3,16)\else\pspicture(-2,0)(2,16)\fi
  \else
    \ifpst@dosage@PHmetre\pspicture(-5,0)(3,5)\else\pspicture(-2,0)(2,5)\fi% without any additional object
  \fi
  \rput(0,3.5){\pstBallon}%
  \ifPst@burette\rput(0,4.5){%
    \psclip{\pscustom[linestyle=none]{\pst@burette@corps}}
      \psframe*[linecolor=\psk@Burette@CouleurReactif](-1,-1)(! 1 \the\pst@cnta\space 0.32 mul 2.68 add)
    \endpsclip}
  \fi%
  \psset{linewidth=0.053}%
  \ifPst@burette\rput(0,4.5){\pst@burette@corps\pst@burette@robinet\pst@burette@graduation}\fi%
  \ifpst@dosage@PHmetre\rput(-3,0){\pst@dosage@pHmetre}\fi%
  \ifPst@Agitateur@Magnetique
    \rput(0,1.5){\pst@dosage@aimant}%
    \rput(0,0){\pst@dosage@support}\else\psframe(-1.5,0)(1.5,1.5)%
  \fi%
  \ifPst@burette\uput[0](2,11){\psk@Reactif@Burette}\fi%
  \uput[0](2,3.5){\psk@Reactif@Becher}%
  \endpspicture%
  \end@SpecialObj%
}
%
\def\pstpipette{\pst@object{pst@pipette}}
\def\pst@pipette@i{%
  \addbefore@par{dimen=middle}%
  \begin@SpecialObj
  \pspicture(-1,-1)(1,6)%\psgrid[subgriddiv=0,gridcolor=red]%
  \psclip{%
    \pscustom[linestyle=none]{\pst@pipette@corps\pst@goutte}}
    \psframe*[linecolor=\psk@pipette@couleurReactif](-1,-2)(! 1 \the\pst@cntg\space 2 div)%
  \endpsclip%
  \pst@pipette@corps%
  \pst@goutte%
  \endpspicture
  \end@SpecialObj%
}%

% To show liquids in the eprouvette
\def\pst@Eprouvette@Liquide#1#2#3{%
  \pspolygon[style=#2]%
  (! -4 4.5 #1 mul 100 div #3 cos mul)(-4,-1)(5,-1)(! 5 4.5 #1 mul 100 div #3 cos mul)}
% To show liquids in the tube
\def\pst@TubeEssais@Liquide#1#2#3{%
  \pspolygon[style=#2](! -4 3 #1 mul 100 div #3 cos mul)(-4,-1)(5,-1)(! 5 3 #1 mul 100 div #3 cos mul)}

\def\pstEprouvette{\pst@object{pst@Eprouvette}}
\def\pst@Eprouvette@i{{%
  \addbefore@par{dimen=middle}%
  \begin@SpecialObj
  \pspicture(-2,0)(2,6)
  \psclip{\rput{\psk@TubeEssais@tubePenche}{%
      \pscustom[linestyle=none]{\pst@eprouvette@corps}}}
    \multido{\iLiquide=\@ne+\@ne}{\thr@@}{%
      \ifnum\csname psk@TubeEssais@niveauLiquide\iLiquide\endcsname>\z@
        \pst@Eprouvette@Liquide%
        {\csname psk@TubeEssais@niveauLiquide\iLiquide\endcsname}%
        {\csname psk@TubeEssais@aspectLiquide\iLiquide\endcsname}
        {\psk@TubeEssais@tubePenche}%
      \fi%
    }%
    \psk@TubeEssais@Subtance
  \endpsclip
  \rput{\psk@TubeEssais@tubePenche}{
    \pst@eprouvette@corps\pst@eprouvette@pied{\psset{linewidth=0.02}\pst@eprouvette@graduation}
    \rput(0,5){\pst@eprouvette@rebord}}
  \endpspicture
  \end@SpecialObj%
}}

\def\pstDistillation{\pst@object{pst@Distillation}}
\def\pst@Distillation@i{\@ifnextchar({\pst@Distillation@ii}{\pst@Distillation@ii(-4,-10)(8,7)}}%
\def\pst@Distillation@ii(#1,#2)(#3,#4){%
  \begin@SpecialObj%
  \pspicture(#1,#2)(#3,#4)
    \rput(-3,-7){\pst@DistillationFractionnee}
  \endpspicture%
  \end@SpecialObj%
}
%
\def\pstEntonnoir{\pst@object{pst@Entonnoir}}
\def\pst@Entonnoir@i{{%
  \addbefore@par{dimen=middle}%
  \begin@SpecialObj%
    \pspicture(-2,-2)(2,5)%\psgrid[subgriddiv=0,gridcolor=red]%
    \rput(0,0){\pstTubeEssais} \rput(0,1.5){\pst@Entonnoir@Corps}%
    \endpspicture%
  \end@SpecialObj%
}}
%
%
\def\pstTubeEssais{\pst@object{pst@TubeEssais}}
\def\pst@TubeEssais@i{{%
  \addbefore@par{dimen=middle}
  \begin@SpecialObj
  \ifPst@TubeEssais@tubeCoude\pspicture(-1,0)(6,4)\else
    \ifnum\psk@TubeEssais@tubePenche=0 \pspicture(-1.5,0)(1.5,4)\else
      \ifnum\psk@TubeEssais@tubePenche>0 \pspicture(-3,0)(1,4)\else
        \ifnum\psk@TubeEssais@tubePenche<0 \pspicture(-1,0)(3,4)\else
          \pspicture(-2,0)(2,4)%
  \fi\fi\fi\fi
%  0->TUBE  1->BALLON   2->BECHER  3->ERLEN  4->FLACON  5->fioleJauge  6->Verre
  \psclip{%
    \rput{\psk@TubeEssais@tubePenche}{%
      \pscustom[linestyle=none]{%
        \ifcase\psk@glassType\pst@TubeEssais@Corps\or  %0
        \pst@ballon\or\pst@Becher@Corps\or\pst@Erlen@Corps\or\pst@Flacon@Corps\or
          \pst@FIOLEJAUGE\or\pst@Verre@Corps\fi}}%
    \psframe[linestyle=none](-3,-1)(! 
      5 3 \@nameuse{psk@TubeEssais@niveauLiquide1}\space mul 100 div
      \psk@TubeEssais@tubePenche\space cos mul)%
  }%
    \multido{\iLiquide=\@ne+\@ne}{\thr@@}{%
      \ifnum\csname psk@TubeEssais@niveauLiquide\iLiquide\endcsname>\z@%
        \pst@TubeEssais@Liquide%
        {\csname psk@TubeEssais@niveauLiquide\iLiquide\endcsname}%
        {\csname psk@TubeEssais@aspectLiquide\iLiquide\endcsname}%
        {\psk@TubeEssais@tubePenche} 
     \fi%
    }%
    \psk@TubeEssais@Subtance%
    \psk@TubeEssais@Solide%
  \endpsclip%
%  \psk@TubeEssais@Solide%
  \rput{\psk@TubeEssais@tubePenche}(0,0){%
    \ifPst@Etiquette\rput(0,2){\psframebox[linestyle=none,%
      fillstyle=solid,fillcolor=GrisTresClair]{\pst@Numero}}\fi%
    \ifcase\psk@glassType\pst@TubeEssais@Corps\or\pst@ballon\or
      \pst@Becher@Corps\pst@RebordBecher\or
      \pst@Erlen@Corps\psellipse[linewidth=0.07](0,3.5)(0.6,0.1)\or
      \pst@Flacon@Corps\psellipse[linewidth=0.07](0,3.5)(0.6,0.1)\or
      \pst@FIOLEJAUGE\or\pst@Verre@Corps\pst@Verre@Pied%
    \fi
    \ifPst@TubeEssais@tubeDroit\rput(0,3.5){\pst@TubeEssais@Droit}\else
      \ifPst@TubeEssais@tubeCoude\rput(0,3.5){\pst@TubeEssais@Coude}\else
        \ifPst@TubeEssais@tubeCoudeU\rput(0,3.5){\pst@TubeEssais@CoudeU}\else
          \ifPst@TubeEssais@bouchon\rput(0,3.5){\pst@TubeEssais@Bouchon}
    \fi\fi\fi\fi
    \ifPst@TubeEssais@Pince\rput(0,3.2){\pst@TubeEssais@Pince}\fi%
  }%
  \endpspicture%
  \end@SpecialObj%
}}
%
\def\pstChauffageTube{\pst@object{pst@ChauffageTube}}
\def\pst@ChauffageTube@i{%
  \begin@SpecialObj%
  \ifpst@tubeSeul\pspicture(-2,-5)(5,4)\else\pspicture(-3,-5)(10,4)\fi%
  \ifpst@barbotage%
    \rput(2.5,2.5){%
      \pstTubeEssais[tubeCoude=true,tubePenche=-60]}%
      \rput(8.2,-3){\pstTubeEssais[niveauLiquide1=50,tubePenche=30,pince=false,%
        niveauLiquide1=70,glassType=tube,tubeCoude=false,%
        aspectLiquide1=Champagne,substance={\pstBullesChampagne[20]}]}%
  \else%
    \ifcase\psk@glassType
      \ifpst@tubeSeul\rput(1.5,2.5){\psset{tubePenche=-60}\pstTubeEssais}\else%
        \ifPst@TubeEssais@tubeCoude\rput(2,2.5){\psset{tubePenche=-60}\pstTubeEssais}\else%
	  \rput(1,2.5){\psset{tubePenche=-60}\pstTubeEssais}%
      \fi\fi%
      \or\rput(2,2.5){\psset{tubePenche=-60}\pstTubeEssais}% ballon
    \else\rput(1,2.5){\psset{tubePenche=-60}\pstTubeEssais}%
    \fi%
  \fi%
  \ifpst@becBunsen\rput(1,-5){\pst@BecBunsen}\fi%
  \endpspicture%
  \end@SpecialObj%
}%
%%%%%%%%%%%%%%%� fin ChauffageTube %%%%%%%%%%%%%%%%%%%%%%%%%%%%%%%%%%
%  0->TUBE  1->BALLON   2->BECHER  3->ERLEN  4->FLACON  5->fioleJauge  6->Verre
\def\pstChauffageBallon{\pst@object{pst@ChauffageBallon}}
%\def\pst@ChauffageBallon@i{\@ifnextchar(\pst@ChauffageBallon@ii{\pst@ChauffageBallon@iii(\@empty)(\@empty)}}%
%\def\pst@ChauffageBallon@ii(#1){\@ifnextchar({\pst@ChauffageBallon@iii(#1)}{\pst@ChauffageBallon@iii(0,0)(#1)}}
%\def\pst@ChauffageBallon@iii(#1)(#2){%
%  \def\pst@tempA{#1}
\def\pst@ChauffageBallon@i{%
  \begin@SpecialObj
%  \ifx\pst@tempA\@empty% no special coors
    \ifpst@becBunsen\pst@cnth=-5 \else\pst@cnth=1 \fi 
    \ifnum\psk@glassType=2 \pspicture(-2,\pst@cnth)(3,5)\else % becher
      \ifnum\psk@glassType=3 \pspicture(-2,\pst@cnth)(4,6)\else % erlen
        \ifpst@barbotage\pspicture(-2,\pst@cnth)(6,7)\else
          \ifPst@TubeEssais@tubeRecourbe@Court\pspicture(-2,0)(9,8)\else
            \ifPst@TubeEssais@DoubleTube\pspicture(-2,0)(9,8)\else
            \pspicture(-2,\pst@cnth)(4,6)
            \fi  
          \fi
        \fi
      \fi
    \fi
 % \else\pspicture(#1)(#2)\fi
  %\psgrid[subgriddiv=0,gridcolor=red]%
  %\psset{aspectLiquide1=Champagne,substance={\pstBullesChampagne[20]}}%
  \ifpst@recuperationGaz%
    \ifPst@TubeEssais@tubeRecourbe%
        \rput(8,-5){\pst@Cristallisoir\pst@ReposeTube%
           \rput(0,0.7){\pst@TubeRenverse\pstbulles}}\else%
    \ifPst@TubeEssais@tubeRecourbe@Court\pst@becBunsenfalse%
        \rput(6,1){\pst@Cristallisoir\pst@ReposeTube%
          \rput(0,0.7){\pst@TubeRenverse\pstbulles}}\else%
    \ifPst@TubeEssais@DoubleTube\pst@becBunsenfalse%
        \rput(6.2,1){\pst@Cristallisoir\pst@ReposeTube%
          \rput(0,0.7){\pst@TubeRenverse\pstbulles}}%
  \fi\fi\fi\fi%
  \rput(1,3){\pstBallon}%
  \ifpst@barbotage%
    \rput(5.3,-3){\pstTubeEssais[niveauLiquide1=50,niveauLiquide1=70,glassType=tube]}%
  \fi %
  \ifpst@becBunsen\rput(1,-5){\psset{dimen=middle,linewidth=0.053}\pst@BecBunsenGrille}\fi%
  \endpspicture%
  \end@SpecialObj%
}
%
%%%%%%%%%%%%%%%%%%%%%%%%%%%%%%%%%%%%%%%%%%%%%%%%%%%%%%%%%%%%%%%%%%%%%%%%%%
%  0->TUBE  1->BALLON   2->BECHER  3->ERLEN  4->FLACON  5->fioleJauge  6->Verre
\def\pstBallon{\pst@object{pst@Ballon}}
\def\pst@Ballon@i{%
  \begin@SpecialObj
  \ifPst@TubeEssais@Refrigerant@Boulles\pspicture(-2,0)(2,12)\else\pspicture(-2,0)(2,4)\fi
  \psclip{\rput{\psk@TubeEssais@tubePenche}{%
    \pscustom[linestyle=none]{%
      \ifcase\psk@glassType\pst@ballon\or\pst@ballon\or\pst@Becher@Corps\or\pst@Erlen@Corps\or
        \pst@Flacon@Corps\else\pst@ballon\fi%  ballon is default
    }%
  }%
  \psframe[linestyle=none](-3,-1)(! 
    4 3
    \@nameuse{psk@TubeEssais@niveauLiquide1}\space mul 100 div
    \psk@TubeEssais@tubePenche\space cos mul)}
  \multido{\iLiquide=\@ne+\@ne}{\thr@@}{%
    \ifnum\csname psk@TubeEssais@niveauLiquide\iLiquide\endcsname>\z@
      \pst@TubeEssais@Liquide{\csname psk@TubeEssais@niveauLiquide\iLiquide\endcsname}%
         {\csname psk@TubeEssais@aspectLiquide\iLiquide\endcsname}%
         {\psk@TubeEssais@tubePenche}%
    \fi%
  }%
  \psk@TubeEssais@Subtance
  \psk@TubeEssais@Solide%
  \endpsclip%
  \rput{\psk@TubeEssais@tubePenche}{%
    \ifPst@Etiquette\rput(0,1.5){\psframebox[linestyle=none,fillstyle=solid,%
      fillcolor=GrisTresClair]{\pst@Numero}}%
    \fi%
    \ifcase\psk@glassType\pst@ballon\or\pst@ballon\or\pst@Becher@Corps\pst@RebordBecher\or
      \pst@Erlen@Corps\psellipse[linewidth=0.07](0,3.5)(0.6,0.1)\or
      \pst@Flacon@Corps\psellipse[linewidth=0.07](0,3.5)(0.6,0.1)\else\pst@ballon
    \fi
    \ifPst@TubeEssais@tubeDroit\rput(0,3.5){\pst@TubeEssais@Droit}\else
    \ifPst@TubeEssais@tubeCoude\rput(0,3.5){\pst@TubeEssais@Coude}\else
    \ifPst@TubeEssais@tubeCoudeU\rput(0,3.5){\pst@TubeEssais@CoudeU}\else
    \ifPst@TubeEssais@tubeCoudeUB\rput(0,3.5){\pst@TubeEssais@CoudeUB}\else
    \ifPst@TubeEssais@tubeRecourbe\rput(0,3.5){\pst@TubeRecourbe}\else
    \ifPst@TubeEssais@tubeRecourbe@Court\rput(0,3.5){\pst@TubeRecourbe@Court}\else
    \ifPst@TubeEssais@bouchon\rput(0,3.5){\pst@TubeEssais@Bouchon}\else
    \ifPst@TubeEssais@DoubleTube\rput(0,3.5){\pst@TubeEssais@DoubleTube}\else
    \ifPst@TubeEssais@Refrigerant@Boulles
      \pst@Ballon@Reflux\rput(0,3.5){\pst@Refrigerant@Boulles}\pst@ChauffeBallon
    \fi\fi\fi\fi\fi\fi\fi\fi\fi%
  }
  \endpspicture%
  \end@SpecialObj%
}
%
\def\pstBullesChampagne{\@ifnextchar[{\pst@BullesChampagne@i}{\pst@BullesChampagne@i[25]}}%
\def\pst@BullesChampagne@i[#1]{{%
  \multido{\iBulle=1+1}{#1}{%
    \setrandim{\BulleX}{-1pt}{1pt} 
    \setrandim{\BulleY}{0pt}{2pt}
    \pst@dimA=\BulleY
    \ifdim\BulleY<1pt 
      \multiply\pst@dimA by 3
      \else\ifdim\BulleY<2pt 
        \multiply\pst@dimA by 2
        \else\multiply\pst@dimA by 1
      \fi\fi%
    \rput{\psk@TubeEssais@tubePenche}{%
%      \psdot[dotscale=\pointless\pst@dimh,dotstyle=o](! 
      \pscircle(!
        \pointless\BulleX\space \pointless\BulleY\space 
        \csname psk@TubeEssais@niveauLiquide1\endcsname\space mul 100 div){\BulleY}%
    }%
  }
}}
%  
%%%%%%%%% bulles pour tube renvers� %%%%%%%%%%%%%
\def\pstbulles{\@ifnextchar[{\pst@bulles@i}{\pst@bulles@i[25]}}%
\def\pst@bulles@i[#1]{{%  
  \multido{\iBulle=1+1}{#1}{%
    \setrandim{\BulleX}{-0.1pt}{0.1pt} \setrandim{\BulleY}{0.15pt}{3pt}
    \pst@dimh=\BulleY 
    \ifdim\BulleY<1pt \multiply\pst@dimh by 3
      \else
    \ifdim\BulleY<2pt \multiply\pst@dimh by 2
      \else \multiply\pst@dimh by 1
    \fi\fi
  \rput{\psk@TubeEssais@tubePenche}{%
    \psdot[dotscale=\pointless\pst@dimh,dotstyle=o](! \pointless\BulleX\space
  \pointless\BulleY\space 1.5 mul)}}
}}
%
%
\def\pstFilaments{\@ifnextchar[{\pst@Filaments@i}{\pst@Filaments@i[5]}}%
\def\pst@Filaments@i[#1]#2{{%
  \multido{\iFilament=1+1}{#1}{%
    \setrandim{\pst@FilamentWidth}{0pt}{0.05pt}
    \rput{\psk@TubeEssais@tubePenche}{%
    \pscustom[linewidth=\pointless\pst@FilamentWidth,linecolor=#2]{%
      \code{/MyRand rand 2 31 exp div def} 
      \pscurve(! MyRand 1 sub MyRand 2 mul 1 sub) 
              (! MyRand 0.2 add MyRand 2 add)(! MyRand MyRand 1 sub)(! MyRand 0.5 add MyRand 4 add)}}}%
}}
%
\def\pst@Clous@{{%
  \psarc[unit=0.5,linewidth=0.1,linecolor=blue](0,0.5){.4}{190}{260}
  \psline[unit=0.5,linecolor=blue](-0.28,0.2)(0.5,1)}}
%
%
\def\pstClouFer{\@ifnextchar[{\pst@ClouFer@i}{\pst@ClouFer@i[60]}}%
\def\pst@ClouFer@i[#1]{{%
  \multido{\iClou=1+1}{#1}{%
    \setrandim{\ClouX}{-1cm}{1cm} \setrandim{\ClouY}{-.2cm}{0.4cm}
    \setrannum{\lab@AngleRotation}{-90}{90}
    \rput{\psk@TubeEssais@tubePenche}{%
      \rput{\the\lab@AngleRotation}(\ClouX,\ClouY){\pst@Clous@}}}%
}}
%
\def\pst@Cuivre{{
  \psbezier[unit=0.5,linewidth=0.1,linecolor=Copper]%
    (.25,0.25)(-0,0)(-0.25,.25)(0.25,0.75)(0,.5)(-0.25,0.75)(0.25,1.25)(0,1)(-0.25,1.25)(0.25,1.75)}}
%
\def\pstTournureCuivre{\@ifnextchar[{\pst@TournureCuivre@i}{\pst@TournureCuivre@i[30]}}%
\def\pst@TournureCuivre@i[#1]{{%
  \multido{\iTournure=1+1}{#1}{%
    \setrandim{\TournureX}{-1cm}{1cm} \setrandim{\TournureY}{-.2cm}{0.3cm}
    \setrannum{\lab@AngleRotation}{-180}{180}
    \rput{\psk@TubeEssais@tubePenche}{%
      \rput{\the\lab@AngleRotation}(\TournureX,\TournureY){\pst@Cuivre}}}%
}}

%%%%%%%%%%%% grenaille de zinc %%%%%%%%%%%%%%%%%%
\def\pst@Zinc{{%
  \pscurve[unit=0.5,fillstyle=solid,fillcolor=GrisClair](0,0)(-0.25,0.25)(0,.5)(0.25,0.75)(0.5,0.5)(0.15,0.4)(0.,0)}}
%
\def\pstGrenailleZinc{\@ifnextchar[{\pst@GrenailleZinc@i}{\pst@GrenailleZinc@i[25]}}%
\def\pst@GrenailleZinc@i[#1]{{%
  \multido{\iGrenaille=1+1}{#1}{%
    \setrandim{\GrenailleX}{-1cm}{1cm} \setrandim{\GrenailleY}{-.2cm}{0.3cm}
    \setrannum{\lab@AngleRotation}{-180}{180}
    \rput{\psk@TubeEssais@tubePenche}{%
      \rput{\the\lab@AngleRotation}(\GrenailleX,\GrenailleY){\pst@Zinc}}}%
}}

%%%%%%%%%%%%%% Vapeurs %%%%%%%%%%%%%%%%%%%%%%%%%%%
%%%%%%%%%%%%%%%%%% Bulles3D%%%%%%%%%%%%%%%%%%%%%%%%%%%
%
\def\pstBilles{\@ifnextchar[{\pst@Billes@i}{\pst@Billes@i[50]}}%
\def\pst@Billes@i[#1]{{%
  \multido{\IBULLE=1+1}{#1}{%
    \setrandim{\BulleX}{-1.5\psunit}{1.5\psunit}
    \setrandim{\BulleY}{0\psunit}{3\psunit}
    \pst@dimh=\BulleY
    \setrandim{\RAYONBULLE}{0.01\psunit}{0.1\psunit}
    \ifdim\BulleY>2\psunit \multiply \RAYONBULLE by 3
      \else\ifdim\BulleY < 2\psunit \ifdim\BulleY >1\psunit \multiply \RAYONBULLE by 2\fi
        \else\multiply \RAYONBULLE by 1
    \fi\fi%
    \rput{\psk@TubeEssais@tubePenche}{%
      %\psset{linestyle=none,fillstyle=gradient,gradmidpoint=0,gradend=white,GradientCircle=true}
      \pscircle[style=BilleThreeD](\BulleX,\BulleY){\RAYONBULLE}}}%
}}
%%%%%%%%%%%%%%%%%%% Bulles2D%%%%%%%%%%%%%%%%%%%%%%%%%%%
%
\def\pstBULLES{\@ifnextchar[{\pst@BULLES@i}{\pst@BULLES@i[50]}}%
\def\pst@BULLES@i[#1]#2{{%
  \multido{\IBULLE=1+1}{#1}{%
    \setrandim{\BulleX}{-1.5\psunit}{1.5\psunit}
    \setrandim{\BulleY}{0\psunit}{3\psunit}
    \pst@dimh=\BulleY
    \setrandim{\RAYONBULLE}{0.01\psunit}{0.1\psunit}
    \ifdim\BulleY>2\psunit \multiply \RAYONBULLE by 3
    \else\ifdim\BulleY < 2\psunit \ifdim\BulleY >1\psunit \multiply \RAYONBULLE by 2\fi
      \else\multiply \RAYONBULLE by 1
    \fi\fi
    \rput{\psk@TubeEssais@tubePenche}{%
      \pscircle[fillstyle=solid,linewidth=0.2\pslinewidth,linecolor=#2](\BulleX,\BulleY){\RAYONBULLE}}}%
}}
\iffalse
%%%%%%%%%% solides %%%%%%%%%%%%%%%%%%%%%%%%%%%
\def\pstTournureCuivre{%
  \rput{\psk@TubeEssais@tubePenche}{\psbezier[linewidth=0.1,linecolor=Copper](.25,0.25)(0,0)
    (-0.25,.25)(0.25,0.75)(0,.5)(-0.25,0.75)(0.25,1.25)(0,1)(-0.25,1.25)(0.25,1.75)}}
\def\pstClouFer{%
  \rput{\psk@TubeEssais@tubePenche}{{\psset{linecolor=blue}
  \psarc[linewidth=0.1](0,0.5){.4}{190}{260} \psline(-0.28,0.2)(0.5,1)}}}
\def\pstGrenailleZinc{%
  \rput{\psk@TubeEssais@tubePenche}{\pscurve[fillstyle=solid,fillcolor=GrisClair]%
  (0,0)(-0.25,0.25)(0,.5)(0.25,0.75)(0.5,0.5)(0.15,0.4)(0.,0)}}
\fi

\catcode`\@=\PstAtCode\relax
\endinput
