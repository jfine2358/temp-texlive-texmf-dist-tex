%%%%%%%%%%%%%%%%%%%%%%%
%%                   %%
%% pmx.tex           %%
%%                   %%
%%%%%%%%%%%%%%%%%%%%%%%
\def\pmxversion{2.7}\def\pmxdate{3 April 13}

% 2.619 At movement break, directly set \nbinstruments in \newmovement;
%   probably don't need \newnoi any more.
%   (2.619a) Added \resetlyrics in \newmovement.
% 2.614 Comment out definitions of some dotted rests, since either in 
%         musixtex.tex or not needed.
% Modified for two figured bass lines 081115
% 2.502 (Olivier Vogel) change all the commands of the form
%   \font\...=\fontid sy1000 to \font\...=cmsy10
% 2.408 comment out extra definition of \mdot
%        tweak \hpausep, replace \liftPAuse, add \liftPAusep
% 2.406  redefine \starteq to put down strut for first system inside \znotes
% 2.354a add \zcharnote{##2}{~} to \tieforsl
% 2.354
%   Macros \tieforisu, etc, to replace slurs with ties, for use with musixps
% 2.353
%   Define \pmxversion
%   Insert Dirk's 2nd set of (LaTeX-aware) redefinitions for \centerline etc.
%   Add 5th option for \endset to \newmovement, for Rz.
% 12/24/01 add \setzalaligne
% 12/23/01 Remove \parskip re-definition.
% 12/16/01 Define \psforts (postscript slur endings to replace \midslur)
% 10/20/01 added defn's of \bigcna, etc
% 10/20/01 define \resetsize, redefine \gaft, \grace, and \shlft so resizing
%   is conditional on current staff line spacing. 
% 11 September Redefine liftpausc to include ledger line.
% 26 Aug 01 Add \pcaesura, \pbreath
% 10/21/00  Added PMXbarnotrue/false and stopped PMX zapping writezbarno
% 15 October added vertical equalization macros
% 4 July added \PAusep, \brevep, dynamic mark placement \pmxdyn, \sfz
% 1 July inserted stuff for hairpins 
% 31 May 99 Redefined \smno
%
\ifx\pmxnewclefs\undefined\else\endinput\fi
\immediate\write10%
{PMX, a Preprocessor for MusiXTeX, Version\space\pmxversion a\space<\pmxdate>}%
%
\edef\catcodeat{\the\catcode`\@}\catcode`\@=11
%
%  The next 4 lines are for Don's home use only
%
\font\specfnt=cmssqi8%
\def\mtr#1{\zcharnote{#1}%
{\specfnt\kern 1.5\internote\raise 0.3\internote\hbox to 0pt{/\hss}%
\kern -0.7\internote\raise 1.0\internote\hbox to 0pt{/\hss}}}
%
\newif\ifolder%
%
%  Older fonts had clefs at different heights.  If you have the older 
%  version you should uncomment the following line
%\oldertrue%
%
%  The next 3 lines should be in musixmad.  They were omitted in 
%  some early versions!
%
%\newcount\s@Nx
%\newcount\s@Nxi
%\newcount\s@Nxii
%
%  Special clef change stuff
%
\newcount\reflect
\def\pmxnewclefs{\m@loop\s@l@ctclefs\o@c\a@c\repeat}%
\newtoks\pmxclefsym
\def\pmxclef#1#2{\ifnum#1=0\def\pmxclefsym{\smalltrebleclef}\else\ifnum#1<5%
\def\pmxclefsym{\smallaltoclef}\else\ifnum#1=7\def\pmxclefsym{\smalltrebleclef}%
\else\def\pmxclefsym{\smallbassclef}\fi\fi\fi%
% Adjust height here for older clef default heights.
\reflect#2\ifolder\ifnum#1=0\advance\reflect-2\else\ifnum#1<5\advance\reflect-4%
\else\advance\reflect-6\fi\fi\fi%
\lcharnote{0}{\raise\reflect\internote\hbox{\pmxclefsym}}}%
%
%  Grace note stuff, incl. macro to reset size after going tiny.
%  This may assume that \musicsize is 20pt but some staves may have
%    \setsize#1\smallvalue 
%
\def\resetsize{\ifdim\internote<.95\Internote%
\let\musictinyfont\musicthirteen\smallnotesize\else\normalnotesize\fi}%
\def\settiny{\ifdim\internote<.95\Internote\let\musictinyfont\musiceleven\fi%
\tinynotesize}%
\def\grace#1#2#3{\off{-#1pt}\settiny\stdstemfalse#2\zcu{#3}%
\resetsize\off{#1pt}}%
\def\shlft#1#2{\off{-#1pt}\settiny\stdstemfalse#2\resetsize\off{#1pt}}%
\def\gaft#1#2{\bsk\roffset{#1}{{\settiny\stdstemfalse#2\resetsize}}\sk}%
%
%  The following keep octaviation out of brackets.
%  10/20/01: Are these used anywhere???
%
%\def\Gshl#1{\off{-#1pt}\tinynotesize}%
%\def\Gshr#1{\normalnotesize\off{#1pt}}%
\def\segnoo#1#2{\zcharnote{#2}{\kern#1pt\musicsmallfont\char"56}}
%
% Some ornaments...
%
% Font for x-trill symbol.  Could make this more general.
%
\def\xtr#1{\zcharnote{#1}{\xtrfont\char"02}}
%
% Plus-trill
%
\def\ptr#1{\ptrx{#1}{1.8}{.35}}
\def\ptrx#1#2#3{\zcharnote{#1}{\raise.9\internote\hbox{%
\pcil\h@lf\qn@width\kern\pcil%
\razclef#2\internote\pcil#3\internote\kern-\h@lf\razclef\vrule height \h@lf\pcil 
depth \h@lf\pcil width \razclef\kern-\h@lf\razclef\kern-\h@lf\pcil\kern-.1pt
\vrule height \h@lf\razclef depth \h@lf\razclef width \pcil}}%
}%
%
%  Put in a number for an xtuplet.
%
\def\xnum#1#2#3{\off{#1\elemskip}\zcharnote{#2}{\smalltype\it #3}%
\off{-#1\elemskip}}
%
%  accidental skips
%
\def\ast#1{\off{#1\elemskip}}
%
%  Check for and read a .mod file; open the .ask file
%
\newread\pmxmod
\def\readmod#1{\openin\pmxmod #1.mod\relax
\ifeof\pmxmod\else\input #1.mod\fi\closein\pmxmod} 
%
%  Macros for optional small notes, usually w/ down stems.  Offset to center
%  them below a large note.  I used these in Matteis.
%
%\def\smno#1{\roffset{.17}{\tinynotesize#1\normalnotesize}}%
%
%  Redefined, due to some mysterious problem with \roffset
%
\def\smno#1{\off{.17\qn@width}\tinynotesize#1\normalnotesize\off{-.17\qn@width}}%
\def\pmno#1{\roffset{.08}{\smallnotesize#1\normalnotesize}}%
\def\smq#1{\smno{\zql{#1}}}\def\smc#1{\smno{\zcl{#1}}}%
\def\smh#1{\smno{\zhl{#1}}}%
%
%  Single-digit meter symbol 
%
\newbox\workbox%
\def\meterN#1{\setbox\workbox=\vbox{\hbox{\ \meterfont #1}}%
\vbox to 8\internote{\offinterlineskip\vss\hbox to \wd\workbox{\hss
\meterfont #1\hss}\vss }}%
%
%  Meter symbol '3' with vertical slash
%
\def\meterIIIS{\kern\internote\raise\Interligne\hbox{\meterfont 3}%
\kern -2.0\internote
\vrule height 7\internote depth -\internote width0.3\internote
\kern 1.7\internote}%
%
% Fractional meter with a slash
%
\def\meterfracS#1#2{%
\kern\internote\raise2\Interligne\rlap{\meterfont #1}%\lower2\Interligne%
\hbox{\meterfont #2}%
\kern -1.85\internote
\vrule height 8\internote width0.3\internote
\kern 1.55\internote}%
%
%  *Symbols* for use in figures
%
\def\sharpfig{\musixchar92}
\def\flatfig{\musixchar90}
\def\natfig{\musixchar94}
\def\fsmsh{\llap{\musixchar92}}
\def\fsmfl{\llap{\musixchar90}}
\def\fsmna{\llap{\musixchar94}}
%
%  Macros for editorial accidentals
%
\def\qedit#1{\roffset{.2}{\zcharnote{#1}{\normtype\rm?}}}
\def\qsharp#1{%
\loffset{.2}{\zcharnote{#1}{\sharpfig\lower1.3\internote\hbox{\normtype\rm?}}}}
\def\qflat#1{%
\loffset{.2}{\zcharnote{#1}{\flatfig\lower.8\internote\hbox{\normtype\rm?}}}}
\def\qnat#1{%
\loffset{.2}{\zcharnote{#1}{\natfig\lower1.2\internote\hbox{\normtype\rm?}}}}
\def\esharp#1{\roffset{.3}{\zcharnote{#1}{\sharpfig}}}
\def\eflat#1{\roffset{.3}{\zcharnote{#1}{\flatfig}}}
\def\enat#1{\roffset{.3}{\zcharnote{#1}{\natfig}}}
%
%  Cautionary big accidentals
%
\def\bigcna{\cna} \def\bigcfl{\cfl} \def\bigcsh{\csh}
%
%  Some fonts...for some reason my system can't do cmbx12 scaled \magstep4
%  9/17/97 Must redo the following 3 lines to permit EC/DC fonts
%
%\font\BIGfont=cmbx10 scaled\magstep4\font\Bigfont=cmbx10 scaled\magstep2
%\font\tempo=\fontid bx12% 
%\font\dyn=\fontid bx10% 
\iflongDCfontnames
\font\xtrfont=cmsy10
\font\tempo=\fontid bx1200 
\font\dyn=\fontid bx1000 
\else\iflongECfontnames
\font\xtrfont=cmsy10
\font\tempo=\fontid bx1200 
\font\dyn=\fontid bx1000 
\else
%\font\xtrfont=cmsy10
%\font\BIGfont=cmbx10 scaled\magstep4
%\font\Bigfont=cmbx10 scaled\magstep2
\font\xtrfont=cmsy10
\font\BIGfont=\fontid bx10 scaled\magstep4
\font\Bigfont=\fontid bx10 scaled\magstep2
\font\tempo=\fontid bx12% 
\font\dyn=\fontid bx10% 
\fi\fi
%
%  Continuation figure
%
\def\Cont#1#2{\advance\figdrop by #1\lower\figdrop\internote%
\hbox to \z@{\kern -0.3\noteskip%
\vbox{\hrule height 1.4\lthick width #2\noteskip}\hss}%
\advance\figdrop by -#1}%
\def\Cott#1#2{\advance\figdtwo by #1\lower\figdtwo\internote%
\hbox to \z@{\kern -0.3\noteskip%
\vbox{\hrule height 1.4\lthick width #2\noteskip}\hss}%
\advance\figdtwo by -#1}%

%
%  Tiny C clef on line #1:  Will need to mod with new fonts 11-4-96
%
\newdimen\razclef\newdimen\symwid%
\newdimen\pcil %\ifnum\musicsize=20\pcil4pt\else\pcil3.25pt\fi%
\def\preclef#1#2{%
\ifnum#2=20\pcil4pt\else\pcil3.25pt\fi%
\symwid3.692\pcil%
\hbox{\vbox{\hrule height \lthick width \symwid}%
\kern-\symwid\raise\pcil\vbox{\hrule height \lthick width \symwid}%
\kern-\symwid\raise2\pcil\vbox{\hrule height \lthick width \symwid}%
\kern-\symwid\raise3\pcil\vbox{\hrule height \lthick width \symwid}%
\kern-\symwid\raise4\pcil\vbox{\hrule height \lthick width \symwid}%
\razclef-2.43\pcil\advance\razclef#1\pcil%
\kern-\symwid\raise\razclef\hbox to \symwid{\hss\smallaltoclef\hss}%
}}%
\def\namewpc#1#2#3#4#5{\raise#4pt\hbox to \parindent{\raise#5pt\hbox{#1}%
\hfill\preclef{#2}{#3}}}
%
%  Stuff for figure placements
%
%  9/17/97 redo font names
%
%\font\figfont=cmr10
\def\figfont{\normtype\rm}%
%
\newcount\figdrop
\newcount\figdtwo
%\figdrop=6
\newcount\sysno
\global\sysno=0\relax%
\def\Figu#1#2{\off{.9pt}\advance\figdrop by #1\lower\figdrop\internote%
\hbox to \z@{\figfont#2\hss}\off{-.9pt}\advance\figdrop by -#1}%
%
%  Special Figure macro for 2-bass parts
%
%\def\Figt#1#2{\zcharnote{#1}{\kern.9pt\figfont#2}}%
\def\Figt#1#2{\off{.9pt}\advance\figdtwo by #1\lower\figdtwo\internote%
\hbox to \z@{\figfont#2\hss}\off{-.9pt}\advance\figdtwo by -#1}%
%
%  Bar numbering
%
\systemnumbers%
\def\raisebarno{3.5\internote}%
\def\shiftbarno{3.5\internote}%
\newdimen\rbnbl\newdimen\sbnbl\newdimen\xrbn\newdimen\xsbn%
\newdimen\rbndim\newdimen\sbndim%
\global\rbndim\raisebarno\global\sbndim\shiftbarno%
\global\def\raisebarno{\rbndim}\global\def\shiftbarno{\sbndim}%
\global\rbnbl\rbndim\global\sbnbl\sbndim\global\xrbn0pt\global\xsbn0pt%
%
%  The following 2 macros are messy, but they retain \raisebarno as a macro and
%  retain original definition in musixtex for the end-of-line utility.
% 
\def\bnrs#1#2{%
%
% If here, baseline values will not change.  On exit, \rbndim will
% be the temporary value, \xrbn will be the increment (must save to check later;
% if <>0 then baseline hasn't changed!), and \rbnbl is still the baseline value.
%
 \global\xrbn#1\internote\global\advance\rbndim\xrbn%
 \global\xsbn#2\internote\global\advance\sbndim\xsbn%
}%
\def\writebarno{%
  \ifnum\barno>1%
    \boxit{\smalltype\bf\the\barno\barnoadd}%
    \ifdim\raisebarno=\rbnbl%
      \empty%
    \else%
      \ifdim\xrbn=0pt%
        \global\rbnbl\raisebarno%
        \global\def\raisebarno{\rbndim}%
      \else%
        \global\xrbn0pt%
      \fi%
      \global\rbndim\rbnbl%
    \fi%
    \ifdim\shiftbarno=\sbnbl%
      \empty%
    \else%
      \ifdim\xsbn=0pt%
        \global\sbnbl\shiftbarno%
        \global\def\shiftbarno{\sbndim}%
      \else%
        \global\xsbn0pt%
      \fi%
      \global\sbndim\sbnbl%
    \fi%
  \fi%
}%
%
%  Make small accidentals the default
%
\smallaccid%
%
%  Put in a new time signature
%
\def\newtimes#1%
{\ifnum#1=0%
  \n@wbar\writ@newclefs\advance\barsinlin@\@ne%
  \wbarno@x\Writ@newsigns%
  \advance\n@skip0.3\afterruleskip\widthtyp@\@ne\t@rmskip%
  \Writ@meters\addspace\afterruleskip%
\else\ifnum#1=1%
  \alaligne 
\else
%
% Start new page
%
  \wbarno@x%
%
% -0.7 gave too much space before meter.  
%
%  \advance\n@skip-0.7\afterruleskip\widthtyp@\@ne\t@rmskip%
  \advance\n@skip-\afterruleskip\widthtyp@\@ne\t@rmskip%
  \Writ@meters\addspace\afterruleskip%
\fi\fi}%
%
%  changecontext with no line break
%
\def\xchangecontext{\n@wbar\writ@newclefs
  \wbarno@x\Writ@newsigns\Writ@meters\addspace\afterruleskip}%
%
%  changecontext with forced line break and no barline
%
\def\zachangecontext{\advance\n@skip\beforeruleskip \widthtyp@\@ne \t@rmskip
  \zn@wbar \writ@newclefs \advance\barsinlin@\@ne 
  \Writ@newsigns\z@suspend\contpiece }
%
%  Dotted rests.
%
%\def\hsp{\pt7\hs} 
%\def\hspp{\ppt7\hs} 
%\def\qsp{\pt5\qs}\def\dsp{\pt5\ds}\def\qpp{\pt5\qp}
%\def\qspp{\ppt5\qs}\def\dspp{\ppt5\ds}\def\qppp{\ppt5\qp}
%\def\pausep{\off{.33\qn@width}\pt5\off{-.33\qn@width}\pause}
%\def\pausep{\wholeshift{\pt5}\pause}
\def\hpausepp{\wholeshift{\ppt5}\hpause}
\def\pausepp{\wholeshift{\ppt5}\pause}
%
%  Redefine headline to lower pagenumber.  The original defn:
%  \def\makeheadline{\vbox to\z@{\vskip-22.5\p@
%  \line{\vbox to8.5\p@{}\the\headline}\vss}\nointerlineskip}
%
\def\makeheadline{\vbox to\z@{\vskip-8\p@
  \line{\vbox to8.5\p@{}\the\headline}\vss}\nointerlineskip}
%
% Dot at arbitrary height above a top line of staff
%
%\def\mdot#1{\zcharnote8{\kern-5.3pt\raise{#1}\internote\hbox{\musixchar0}}}
%
% Titles
%
%\def\titles#1#2#3#4#5#6#7{\kern-\lin@pos%
%\kern-3.2\Interligne\kern-\parindent\kern-\afterruleskip%
%\kern-\sign@skip%
%\vbox{\vskip#1\Interligne
%\ifx\relax#2\relax\else\leftline{\Bigtype #2}\fi\vskip#3\Interligne%
%\ifx\relax#4\relax\else\centerline{\BIGtype #4}\fi\vskip#5\Interligne%
%\ifx\relax#6\relax\else\rightline{\Bigtype #6}\fi\vskip#7\Interligne}}%
%
% New def'n per Dirk Laurie to allow multiple lines.
%
\ifx\documentclass\undefined
\def\centerlines#1{{\def\\{\hss\egroup\medskip\par\line\bgroup\hss}%
  \line\bgroup\hss#1\hss\egroup}}
\def\leftlines#1{{\def\\{\hss\egroup\smallskip\par\line\bgroup}%
  \line\bgroup#1\hss\egroup}}
\def\rightlines#1{{\def\\{\egroup\smallskip\par\line\bgroup\hss}%
  \line\bgroup\hss#1\egroup}}
\else
\def\centerlines#1{{\centering#1\\}}
\def\leftlines#1{{\raggedright#1\\}}
\def\rightlines#1{{\raggedleft#1\\}}
\fi
%\def\centerline#1{{\def\\{\hss\egroup\medskip\par\line\bgroup\hss}%
%  \line\bgroup\hss#1\hss\egroup}}
%\def\leftline#1{{\def\\{\hss\egroup\smallskip\par\line\bgroup}%
%  \line\bgroup#1\hss\egroup}}
%\def\rightline#1{{\def\\{\egroup\smallskip\par\line\bgroup\hss}%
%  \line\bgroup\hss#1\egroup}}
\def\titles#1#2#3#4#5#6#7{\kern-\lin@pos%
\kern-3.2\Interligne\kern-\parindent\kern-\afterruleskip%
\kern-\sign@skip%
\vbox{\vskip#1\Interligne
\ifx\relax#2\relax\else{\Bigtype\leftlines{#2}}\fi\vskip#3\Interligne%
\ifx\relax#4\relax\else{\BIGtype\centerlines{#4}}\fi\vskip#5\Interligne%
\ifx\relax#6\relax\else{\Bigtype\rightlines{#6}}\fi\vskip#7\Interligne}}%
%
% Slashes on stems
%
\def\usoff#1{%
% Set \pcil to (stem length)-9\Internote
  \ifnum#1>10
    \pcil-3.8\Internote
  \else
    \pcil#1\Internote
    \ifnum#1>3
      \pcil-.38\pcil\advance\pcil-.5\Internote
      \ifnum#1>7
        \advance\pcil.38\Internote
      \fi
    \else
      \ifnum#1<-2
        \pcil-\pcil\advance\pcil-5\Internote
      \else
        \pcil-2\Internote
      \fi
    \fi
  \fi
  \advance\pcil#1\Internote
}%
%
\def\us#1{\usoff{#1}%
  \raise\pcil\hbox{\loffset{.5}{\ibu009}\roffset{.5}{\tbu0}}}%
\def\ls#1{\reflect-#1\advance\reflect8 %
\usoff{\reflect}\advance\pcil-7\Internote%
  \lower\pcil\hbox{\loffset{.5}{\ibl009}\roffset{.5}{\tbl0}}}%
%
%  Thinner slashes, better coding
%
\newdimen\pmxtop\def\aslash#1#2#3{%
%
% #1 = note level relative to bottom line.  #2 = 0/1 for down/up stem
% #3 = 0/1 for down/up slash,  For downstem, reflect, do as up, then unreflect.
%
\ifcase#3\def\slchar{\char248}\or\def\slchar{\char184}\fi%
\ifcase#2\pmxtop-#1\internote\advance\pmxtop8\internote\or\pmxtop#1\internote%
\fi\advance\pmxtop4.66\interbeam\ifdim\pmxtop>11\internote\uptop{11}\uptop{12}%
\uptop{13}\uptop{14}\uptop{15}\uptop{16}\fi%
\ifdim\pmxtop<4\internote\pmxtop4\internote\fi%
%
% \pmxtop now top of stem for upstem.  Unreflect if downstem
%
\ifcase#2\pmxtop-\pmxtop\advance\pmxtop8\internote\ifcase#3%
\advance\pmxtop2.5\internote\or\advance\pmxtop.9\internote\fi%
\ccharnote0{\raise\pmxtop\hbox{\musictinyfont\slchar}}%
\or\ifcase#3\advance\pmxtop-.9\internote\or\advance\pmxtop-2.5\internote\fi%
\roff{\ccharnote0{\raise\pmxtop\hbox{\musictinyfont\slchar}}}\fi}%
\def\uptop#1{\ifdim\pmxtop>#1\internote\advance\pmxtop-.25\interbeam\fi}%
%
% Signature change at end of line
%
\def\sigatend#1#2{\setdoublebar\xbar\hardspace{-#2pt}\generalsignature{#1}%
\zchangecontext\hardspace{-#2pt}\advance\barno-1\let\barrul@\empty}%
%
%  Macros for beams
%
\def\rbbu#1{\roff{\tbbu{#1}}}
\def\rbbbu#1{\roff{\tbbbu{#1}}}
\def\rbbbbu#1{\roff{\tbbbbu{#1}}}
\def\rbbl#1{\roff{\tbbl{#1}}}
\def\rbbbl#1{\roff{\tbbbl{#1}}}
\def\rbbbbl#1{\roff{\tbbbbl{#1}}}
%
%  Stuff for second voice per staff
%
\def\nextvoice{%
\@ndstaff\reflect\noport@@\advance\noport@@-1\beginstaff\noport@@\reflect}%
%
%  Macros for moving slur starts and stops
%
\def\isu#1#2#3{\roffset{#3}{\isluru{#1}{#2}}}%
\def\isd#1#2#3{\roffset{#3}{\islurd{#1}{#2}}}%
\def\ts#1#2#3{\roffset{#3}{\tslur{#1}{#2}}}%
%
%  Macro to replace old \tslur with special postscript slur endings 
%
\def\psforts#1{\let\tst\tslur\def\tslur##1##2{%
\ifnum#1=0\tst{##1}{##2}\else%
\ifnum#1=1\tfslur{##1}{##2}\else%
\ifnum#1<4\tst{##1}{##2}\else%
\ifnum#1=4\thslur{##1}{##2}\else%
\ifnum#1=5\tHslur{##1}{##2}\else%
\tHHslur{##1}{##2}\fi\fi\fi\fi\fi\let\tslur\tst}}%
%
%  Macros to replace slurs with ties
%
\def\tieforisu{\let\ist\isluru%
\def\isluru##1##2{\itieu{##1}{##2}\let\isluru\ist}}%
\def\tieforisd{\let\ist\islurd%
\def\islurd##1##2{\itied{##1}{##2}\let\islurd\ist}}%
\def\tieforts{\let\ist\tslur%
\def\tslur##1##2{\ttie{##1}\zcharnote{##2}{~}\let\tslur\ist}}%
%
%  Replacement trill macros, to avoid overfull boxes on 1st pass
%
\let\savtr\trille\let\savTr\Trille%
\def\trille#1#2{\ifeof\inmux\else\savtr{#1}{#2}\fi}%
\def\Trille#1#2{\ifeof\inmux\else\savTr{#1}{#2}\fi}%
%
% Mordent as \rpar.  If dotted, must move dot to right.
%
\def\lpn#1{\loffset{.3}{\lpar{#1}}}%   '(' 
\def\rpn#1{\roffset{.3}{\rpar{#1}}}%   Beam or not, no dot [ ')' only ]
\def\clm#1#2{\mdot{#1}{#2}\cl{#2}}%          Non-beam, dot [ '.' + ')' + note ]
\def\cum#1#2{\mdot{#1}{#2}\cu{#2}}%
\def\qlm#1#2{\mdot{#1}{#2}\ql{#2}}%                 
\def\qum#1#2{\mdot{#1}{#2}\qu{#2}}%
\def\hlm#1#2{\mdot{#1}{#2}\hl{#2}}%                 
\def\hum#1#2{\mdot{#1}{#2}\hu{#2}}%
\def\qbm#1#2#3{\mdot{#2}{#3}\qb{#1}{#3}}%    Beamed & dotted
\def\mdot#1#2{\roffset{.32}{\pt{#1}}\roffset{.24}{\rpar{#2}}}%
\def\lpnu#1{\smallnotesize\loffset{.3}{\zcharnote{#1}%
{\raise1pt\hbox{\musixchar3}}}\normalnotesize}% 
\def\lpnd#1{\smallnotesize\loffset{.3}{\zcharnote{#1}%
{\lower1pt\hbox{\musixchar3}}}\normalnotesize}% 
\def\rpnu#1{\smallnotesize\roffset{.6}{\zcharnote{#1}%
{\raise1pt\hbox{\musixchar4}}}\normalnotesize}%
\def\rpnd#1{\smallnotesize\roffset{.6}{\zcharnote{#1}%
{\lower1pt\hbox{\musixchar4}}}\normalnotesize}%
%
% For use with mid-bar signature changes, to permit using \ast machinery
%
\def\rdoff{\let\toff\off\let\off\addspace}%
%
% For raising arpeggios by .5/internote
%
\def\raisearp#1#2{\zcharnote{#1}{%
  \raise.5\internote\hbox{%
  \uplap{\leaders\hbox{\musixchar70}\vskip#2\Interligne}}}}
%
% New Movement Macro
%
%\def\newmovement#1#2{\let\holdstop\stoppiece\let\holdcont\contpiece%
\def\newmovement#1#2#3{\let\holdstop\stoppiece\let\holdcont\contpiece%
\ifcase#2\def\endset{\setdoubleBAR}\or\def\endset{\setdoublebar}\or%
%\def\endset{\setrightrepeat}\or\def\endset{\empty}\fi%
\def\endset{\setrightrepeat}\or\def\endset{\empty}%
 \or\def\endset{\empty}\fi%
 \def\stoppiece{\endset%
%
%+++
\ifnum#2=4\zstoppiece\else%
%+++
%
\holdstop%
%
%+++
\fi%
%+++
%
\vskip#1\internote%
\let\stoppiece\holdstop}%
% \def\contpiece{
 \def\contpiece{\def\nbinstruments{#3}% <-- assignment inserted here
%
% Added 120904 per Rainer's suggestion to fix problem with M-Tx at m-break.
%
\ifx\resetlyrics\undefined\else\resetlyrics\fi%
%
\startpiece\addspace\afterruleskip\let\contpiece\holdcont}%
}%
%
%  Redefinitions for moving dots vertically by x\interligne and
%  horizontally by y headwidths.  Should work for all kinds of dotted notes.
%  Usage: \def\C@Point#1#2{\PMXpt{.0}{-.2}} (x and y hardwired by PMX)
% 
\global\let\C@Psave\C@Point%
\def\PMXpt#1#2#3{\ifodd\n@i\else\raise\internote\fi%
\hbox{\raise#1\internote\hbox{\kern#2\qn@width\musixchar#3\kernm#2\qn@width}}%
\global\let\C@Point\C@Psave}%
%
%  Shifted, pointed chord notes (Missing in musixtex.tex ???) 
%
\def\lhp#1{\loff{\zhp{#1}}}
\def\rhp#1{\roff{\zhp{#1}}}
\def\lqp#1{\loff{\zqp{#1}}}
\def\rqp#1{\roff{\zqp{#1}}}
\def\rhpp#1{\roff{\zhpp{#1}}}
\def\lhpp#1{\loff{\zhpp{#1}}}
\def\rqpp#1{\roff{\zqpp{#1}}}
\def\lqpp#1{\loff{\zqpp{#1}}}
%
%  Redefine to include \sk !!!
%
\def\liftpause#1{\C@ps\@l@v@n\si@{#1}\sk}
%\def\liftPAuse#1{\C@ps{58}0{#1}\sk}
% 1/19/03 Replaced old def'n; added \liftPAusep
\def\liftPAuse#1{\reflect#1\multiply\reflect2%
\zcharnote{0}{\raise\reflect\internote\hbox{\kern.4\qn@width\musixchar58}}\sk}%
\def\liftPAusep#1{\reflect#1\multiply\reflect2%
\zcharnote{0}{\raise\reflect\internote\hbox{\kern.4\qn@width\musixchar58%
\kern-\qn@width\pt4}}\sk}%
\def\lifthpause#1{\C@ps\t@n\f@ur{#1}\sk}%
\def\liftpausep#1{\reflect#1\advance\reflect2%
\wholeshift{\raise\reflect\Interligne\hbox{\pt0}}\liftpause{#1}}%
\def\liftpausepp#1{\reflect#1\advance\reflect2%
\wholeshift{\raise\reflect\Interligne\hbox{\ppt0}}\liftpause{#1}}%
\def\lifthpausep#1{\reflect#1\advance\reflect2%
\wholeshift{\raise\reflect\Interligne\hbox{\pt0}}\lifthpause{#1}}%
\def\lifthpausepp#1{\reflect#1\advance\reflect2%
\wholeshift{\raise\reflect\Interligne\hbox{\ppt0}}\lifthpause{#1}}%
%
%  Text up to bar line.  Use before last note, assume 1 noteskip to bar.
% 
\def\bartext#1#2{\sk\loffset{.3}{\zcharnote{#1}{\llap{#2}}}\bsk}
%
%  Macro used before movement break to change # of instruments.
%  120818 Set \nbinstruments in \newmovement; probably don't need this any more.
% 
\def\newnoi#1{\let\atnb\atnextbar\def\atnextbar{\atnb\def\nbinstruments{#1}}}%
%
%  Set up top page numbers with optional centered heading
%
% #2=0 if odds on rt, else 1.  #1 = initial page no. #3=name
\def\toppageno#1#2#3{\pageno#1%
\headline{\reflect#2\advance\reflect\pageno%
\ifodd\reflect\rhead{#3}\else\lhead{#3}\fi}%
\def\rhead##1{\tempo\ifnum\pageno>1\hfil{##1}\fi\hfil\llap\folio}%
\def\lhead##1{\tempo\rlap\folio\hfil{##1}\hfil}}%
%
% Temporary date and file name. To use it:
%
% (1) Define a script to create a file tempdate.dat. The file will 
%     contain a single line of text to be centered at the bottom of
%     each page.  I use a 4DOS alias (makedate) that inserts current date
%     filename, and filedate as follows:
%
%  echo \smalltype\rm Printed %_date from file %1.pmx %@filedate[c:\pmx\%1.pmx],
%  %@filetime[c:\pmx\%1.pmx] >tempdate.dat
%
% (2) Include a call to makedate in the batch file you use to run pmx+tex.
%     Be sure tempdate.dat is written to a directory accessible to TeX.
%
% (3) Include in-line tex \\tempdate\ at the top of the pmx file.
%
\def\tempdate{\def\makefootline{\baselineskip2pt\line{\the\footline}}
\footline{\hss\input tempdate.dat\hss}}
%
\def\pnotes#1{\vnotes#1\elemskip}%
%
% Hairpins and other dynamic stuff
%
\newdimen\hpi\newdimen\hpii\newdimen\hpiii\newdimen\hpiv%
\newdimen\hpv\newdimen\hpvi\newdimen\hpvii\newdimen\hpviii%
\newdimen\hpix\newdimen\hpx\newdimen\hpxi\newdimen\hpxii\newdimen\hptmp%
%
\def\hpstrt#1#2{\getcurpos\advance\y@v#2\qn@width\advance\y@v\txt@ff%
\global\csname hp\romannumeral#1\endcsname=\y@v}%
%
\def\hpendall#1#2#3#4{\getcurpos%
\advance\y@v-\csname hp\romannumeral#1\endcsname%
\zcharnote{#2}{\kernm\y@v\advance\y@v\txt@ff\advance\y@v#3\qn@width#4{\y@v}}}%
%
\def\hpcend#1#2#3{\hpendall{#1}{#2}{#3}\crescendo}%
\def\hpdend#1#2#3{\hpendall{#1}{#2}{#3}\decrescendo}%
%
\def\pmxdyn#1#2#3{\ccharnote{#1}{\kern#2\qn@width#3}}
\def\txtdyn#1#2#3{\zcharnote{#1}{\kern#2\qn@width#3}}
%
\def\sfz{{\ppff s\f@kern\f@kern f\f@kern z}}%
%
% Dotted breve and rest
%
\def\brevep{\def\w@h{\musixchar32\roffset\qu@rt{\C@Point\z@\raise}}%
  \y@v\wn@width \g@w}
\def\PAusep{\loffset{.5}{\pt5}\PAuse}%
%
% Vertical equalization macros
%
%\newskip\pssav\pssav\parskip%
%
%  Remove this from here in 2.352, since it caused some incompatibilities.
%   (Let PMX write it into *.tex when needed).
%
%\parskip 0pt plus 12\Interligne minus 99\Interligne%
\def\upamt{27}\def\dnamt{-20}%
%\def\upstrut{\znotes\nextinstrument\nextinstrument\zcharnote{\upamt}{X}\en}%
%
%  Let PMX write the def'n of \upstrut since I couldn't get the loop to work
%
\def\dnstrut{\znotes\zcharnote{\dnamt}{~}\en}%
%
%  Call the following inline type1 anywhere in first line of equalization.
%
\def\starteq{\gdef\everystaff{\upstrut\dnstrut}\dnstrut}%
%
%  And put this as type 1 in the next-to-last line of equalization!
%
%\def\endeq{\gdef\everystaff{\upstrut\global\parskip\pssav%
\def\endeq{\gdef\everystaff{\upstrut%
\gdef\everystaff{\empty}}}%
%
\gdef\spread#1{\global\let\cont\contpiece%
\gdef\contpiece{\vskip#1\internote\cont\global\let\contpiece\cont}}%
%
% The following avoid zapping \writezbarno in several special situations
%
\def\PMXbarnotrue{\let\z@sw\empty}%
\def\PMXbarnofalse{\let\z@sw\@ne}%
%
% Centered rests
%
\def\pausc{\lrlap{\pause\off{\qn@width}}}%
\def\PAusc{\lrlap{\PAuse\off{\txt@ff}}}%
%\def\liftpausc#1{\raise#1\Interligne\pausc}%
\def\liftpausc#1{\raise#1\Interligne\lrlap{\
\raise6\internote\hbox{\musixchar11}\off{1.2\qn@width}}}%
\def\liftPAusc#1{\raise#1\Interligne\PAusc}%
%
% CenterBar and mbrest modified 2/01 to account for simick mods in
%   musixtex 1.01
%
\def\CenterBar#1#2#3{%
\y@ii\lin@pos\advance\y@ii-\lastbarpos%
\ifx\volta@startcor\undefined \else\advance\y@ii\cut@v\fi%
\advance\y@ii-#2pt\advance\y@ii-#3pt\kern-#3pt%
\kern-\h@lf\y@ii\lrlap{#1}\kern\h@lf\y@ii\kern#3pt}%
\def\mbrest#1#2#3{%
\CenterBar{%
\ifnum #1<10\raise 9\internote\lrlap{\meterfont #1}\fi%
\ifcase #1%
\relax%
%\or\lrlap{\pause\off{\qn@width}}%
\or\pausc%
%\or\lrlap{\PAuse\off{\txt@ff}}%
\or\PAusc%
\or\lrlap{\PAuse\qsk\pause\off{\qn@width}}%
\or\lrlap{\PAUSe\off{0.5\txt@ff}}%
\or\lrlap{\PAUSe\qsk\pause\off{\qn@width}}%
\or\lrlap{\PAUSe\qsk\PAuse\off{\txt@ff}}%
\or\lrlap{\PAUSe\qsk\PAuse\qsk\pause\off{\txt@ff}\off{\qn@width}}%
\or\lrlap{\PAUSe\qsk\PAUSe\off{\txt@ff}}%
\or\lrlap{\PAUSe\qsk\PAUSe\qsk\pause\off{\txt@ff}\off{\qn@width}}%
\else\def\vertpart{\hbox{\vrule width.6pt height1.5\internote depth1.5\internote}}%
\raise\f@ur\internote\hbox{%
\vertpart\hbox{%
\advance\y@ii-\cut@v\advance\y@ii-\volta@startcor\advance\y@ii-\volta@endcor%
\vrule width0.7\y@ii height.5\internote depth.5\internote%
\kern-0.35\y@ii\raise5\internote\lrlap{\meterfont #1}\kern0.35\y@ii}%
\vertpart}%
\fi}{#2}{#3}}%
%
\def\pcaesura#1#2{\raise#1\internote\hbox{%
\rlap{\kern.5\noteskip\kern#2\qn@width\musixchar79}}}%
\def\pbreath#1#2{\zcharnote6{\raise#1\internote\hbox{%
\rlap{\kern.5\noteskip\kern#2\qn@width\BIGfont'}}}}%
%
% Macro to set up for blank bar line
%
%\def\setzalaligne{\let\alat\alaligne%
%  \def\alaligne{\zalaligne\let\alaligne\alat}}%
\def\setzalaligne{\let\zalat\stoppiece%
  \def\stoppiece{\zstoppiece\let\stoppiece\zalat}}%
%
% Slanted line arpeggio or coule ornament
%
\def\arpg#1#2{%
\roffset{#2}{\zcharnote0{\raise#1\internote\hbox{\varline0{6pt}{20}}}}}%
\def\arpgu#1{\arpg{#1}{.8}}%
%
\catcode`\@=\catcodeat
