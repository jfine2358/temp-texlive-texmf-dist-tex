%% @texfile{
%%   author = "Karl Berry, Steven Smith, Oleg Katsitadze, and others",
%%   version = "3.6",
%%   date = "Mon Sep 30 11:39:28 PDT 2013",
%%   filename = "eplain.tex",
%%   email = "bug-eplain@tug.org",
%%   checksum = "4346   9755 139706",
%%   codetable = "ASCII",
%%   supported = "yes",
%%   docstring = "This file defines macros that extend and expand on
%%                plain TeX. eplain.tex is xeplain.tex and the other
%%                source files with comments stripped; see the original
%%                files for author credits, etc.  The original sources
%%                can be found in Eplain source zip archive in your TeX
%%                distribution, on CTAN or on Eplain's home page at
%%                http://tug.org/eplain.  And please base diffs or
%%                other contributions on xeplain.tex, not the
%%                stripped-down eplain.tex.",
%% }
\ifx\eplain\undefined
  \let\next\relax
\else
  \expandafter\let\expandafter\next\csname endinput\endcsname
\fi
\next
%%
%% This is file `ifpdf.sty',
%% generated with the docstrip utility.
%%
%% The original source files were:
%%
%% ifpdf.dtx  (with options: `package')
%% 
%% This is a generated file.
%% 
%% Project: ifpdf
%% Version: 2011/01/30 v2.3
%% 
%% Copyright (C) 2001, 2005-2011 by
%%    Heiko Oberdiek <heiko.oberdiek at googlemail.com>
%% 
%% This work may be distributed and/or modified under the
%% conditions of the LaTeX Project Public License, either
%% version 1.3c of this license or (at your option) any later
%% version. This version of this license is in
%%    http://www.latex-project.org/lppl/lppl-1-3c.txt
%% and the latest version of this license is in
%%    http://www.latex-project.org/lppl.txt
%% and version 1.3 or later is part of all distributions of
%% LaTeX version 2005/12/01 or later.
%% 
%% This work has the LPPL maintenance status "maintained".
%% 
%% This Current Maintainer of this work is Heiko Oberdiek.
%% 
%% The Base Interpreter refers to any `TeX-Format',
%% because some files are installed in TDS:tex/generic//.
%% 
%% This work consists of the main source file ifpdf.dtx
%% and the derived files
%%    ifpdf.sty, ifpdf.pdf, ifpdf.ins, ifpdf.drv, ifpdf-test1.tex.
%% 
\begingroup\catcode61\catcode48\catcode32=10\relax%
  \catcode13=5 % ^^M
  \endlinechar=13 %
  \catcode35=6 % #
  \catcode39=12 % '
  \catcode44=12 % ,
  \catcode45=12 % -
  \catcode46=12 % .
  \catcode58=12 % :
  \catcode64=11 % @
  \catcode123=1 % {
  \catcode125=2 % }
  \expandafter\let\expandafter\x\csname ver@ifpdf.sty\endcsname
  \ifx\x\relax % plain-TeX, first loading
  \else
    \def\empty{}%
    \ifx\x\empty % LaTeX, first loading,
    \else
      \expandafter\ifx\csname PackageInfo\endcsname\relax
        \def\x#1#2{%
          \immediate\write-1{Package #1 Info: #2.}%
        }%
      \else
        \def\x#1#2{\PackageInfo{#1}{#2, stopped}}%
      \fi
      \x{ifpdf}{The package is already loaded}%
    \fi
  \fi
\endgroup%
\begingroup\catcode61\catcode48\catcode32=10\relax%
  \catcode13=5 % ^^M
  \endlinechar=13 %
  \catcode35=6 % #
  \catcode39=12 % '
  \catcode40=12 % (
  \catcode41=12 % )
  \catcode44=12 % ,
  \catcode45=12 % -
  \catcode46=12 % .
  \catcode47=12 % /
  \catcode58=12 % :
  \catcode64=11 % @
  \catcode91=12 % [
  \catcode93=12 % ]
  \catcode123=1 % {
  \catcode125=2 % }
  \expandafter\ifx\csname ProvidesPackage\endcsname\relax
    \def\x#1#2#3[#4]{\endgroup
      \xdef#1{#4}%
    }%
  \else
    \def\x#1#2[#3]{\endgroup
      #2[{#3}]%
      \ifx#1\@undefined
        \xdef#1{#3}%
      \fi
      \ifx#1\relax
        \xdef#1{#3}%
      \fi
    }%
  \fi
\expandafter\x\csname ver@ifpdf.sty\endcsname
\ProvidesPackage{ifpdf}%
  [2011/01/30 v2.3 Provides the ifpdf switch (HO)]%
\begingroup\catcode61\catcode48\catcode32=10\relax%
  \catcode13=5 % ^^M
  \endlinechar=13 %
  \catcode123=1 % {
  \catcode125=2 % }
  \catcode64=11 % @
  \def\x{\endgroup
    \expandafter\edef\csname ifpdf@AtEnd\endcsname{%
      \endlinechar=\the\endlinechar\relax
      \catcode13=\the\catcode13\relax
      \catcode32=\the\catcode32\relax
      \catcode35=\the\catcode35\relax
      \catcode61=\the\catcode61\relax
      \catcode64=\the\catcode64\relax
      \catcode123=\the\catcode123\relax
      \catcode125=\the\catcode125\relax
    }%
  }%
\x\catcode61\catcode48\catcode32=10\relax%
\catcode13=5 % ^^M
\endlinechar=13 %
\catcode35=6 % #
\catcode64=11 % @
\catcode123=1 % {
\catcode125=2 % }
\def\TMP@EnsureCode#1#2{%
  \edef\ifpdf@AtEnd{%
    \ifpdf@AtEnd
    \catcode#1=\the\catcode#1\relax
  }%
  \catcode#1=#2\relax
}
\TMP@EnsureCode{10}{12}% ^^J
\TMP@EnsureCode{39}{12}% '
\TMP@EnsureCode{40}{12}% (
\TMP@EnsureCode{41}{12}% )
\TMP@EnsureCode{44}{12}% ,
\TMP@EnsureCode{45}{12}% -
\TMP@EnsureCode{46}{12}% .
\TMP@EnsureCode{47}{12}% /
\TMP@EnsureCode{58}{12}% :
\TMP@EnsureCode{60}{12}% <
\TMP@EnsureCode{91}{12}% [
\TMP@EnsureCode{93}{12}% ]
\TMP@EnsureCode{94}{7}% ^
\TMP@EnsureCode{96}{12}% `
\begingroup
  \expandafter\ifx\csname ifpdf\endcsname\relax
  \else
    \edef\i/{\expandafter\string\csname ifpdf\endcsname}%
    \expandafter\ifx\csname PackageError\endcsname\relax
      \def\x#1#2{%
        \edef\z{#2}%
        \expandafter\errhelp\expandafter{\z}%
        \errmessage{Package ifpdf Error: #1}%
      }%
      \def\y{^^J}%
      \newlinechar=10 %
    \else
      \def\x#1#2{%
        \PackageError{ifpdf}{#1}{#2}%
      }%
      \def\y{\MessageBreak}%
    \fi
    \x{Name clash, \i/ is already defined}{%
      Incompatible versions of \i/ can cause problems,\y
      therefore package loading is aborted.%
    }%
    \endgroup
    \expandafter\ifpdf@AtEnd
  \fi%
\endgroup
\begingroup
  \def\skip#1\relax\begingroup{}%
  \expandafter\ifx\csname pdfoutput\endcsname\relax
  \else
    \expandafter\skip
  \fi
  \expandafter\ifx\csname directlua\endcsname\relax
    \expandafter\skip
  \fi
\endgroup
\begingroup\expandafter\expandafter\expandafter\endgroup
\expandafter\ifx\csname RequirePackage\endcsname\relax
  \def\TMP@RequirePackage#1[#2]{%
    \begingroup\expandafter\expandafter\expandafter\endgroup
    \expandafter\ifx\csname ver@#1.sty\endcsname\relax
      \input #1.sty\relax
    \fi
  }%
  \TMP@RequirePackage{ifluatex}[2009/04/10]%
\else
  \RequirePackage{ifluatex}[2009/04/10]%
\fi
\ifluatex
  \ifnum\luatexversion<36 %
  \else
    \begingroup
      \directlua{tex.enableprimitives('ifpdf', {'pdfoutput'})}%
      \global\let\pdfoutput\ifpdfpdfoutput
    \endgroup
  \fi
\fi
\relax\begingroup\endgroup
\begingroup\expandafter\expandafter\expandafter\endgroup
\expandafter\ifx\csname newif\endcsname\relax
  \edef\pdffalse{%
    \let
    \expandafter\noexpand\csname ifpdf\endcsname
    \expandafter\noexpand\csname iffalse\endcsname
  }%
  \edef\pdftrue{%
    \let
    \expandafter\noexpand\csname ifpdf\endcsname
    \expandafter\noexpand\csname iftrue\endcsname
  }%
  \pdffalse
\else
  \csname newif\expandafter\endcsname\csname ifpdf\endcsname
\fi
\begingroup\expandafter\expandafter\expandafter\endgroup
\expandafter\ifx\csname pdfoutput\endcsname\relax
\else
  \ifnum\pdfoutput<1 %
  \else
    \pdftrue
  \fi
\fi
\begingroup
  \expandafter\ifx\csname pdfoutput\endcsname\relax
  \else
    \escapechar=92 %
    \edef\m{\meaning\pdfoutput}%
    \edef\p{\string\pdfoutput}%
    \ifx\m\p
    \else
      \expandafter\ifx\csname PackageWarningNoLine\endcsname\relax
        \def\PackageWarningNoLine#1#2{%
          \immediate\write16{%
            Package `#1' Warning: #2.%
          }%
        }%
      \fi
      \PackageWarningNoLine{ifpdf}{%
        Someone has redefined \string\pdfoutput%
      }%
    \fi
  \fi
\endgroup
\begingroup
  \expandafter\ifx\csname PackageInfo\endcsname\relax
    \def\x#1#2{%
    }%
  \else
    \let\x\PackageInfo
    \expandafter\let\csname on@line\endcsname\empty
  \fi
  \x{ifpdf}{pdfTeX in PDF mode is \ifpdf\else not \fi detected}%
\endgroup
\ifpdf@AtEnd%
%%
%% End of file `ifpdf.sty'.
\def\makeactive#1{\catcode`#1 = \active \ignorespaces}%
\chardef\letter = 11
\chardef\other = 12
\def\makeatletter{%
  \edef\resetatcatcode{\catcode`\noexpand\@\the\catcode`\@\relax}%
  \catcode`\@11\relax
}%
\def\makeatother{%
  \edef\resetatcatcode{\catcode`\noexpand\@\the\catcode`\@\relax}%
  \catcode`\@12\relax
}%
\edef\leftdisplays{\the\catcode`@}%
\catcode`@ = \letter
\let\@eplainoldatcode = \leftdisplays
\toksdef\toks@ii = 2
\def\uncatcodespecials{%
   \def\do##1{\catcode`##1 = \other}%
   \dospecials
}%
{%
   \makeactive\^^M %
   \long\gdef\letreturn#1{\let^^M = #1}%
}%
\let\@eattoken = \relax  % Define this, so \eattoken can be used in \edef.
\def\eattoken{\let\@eattoken = }%
\def\gobble#1{}%
\def\gobbletwo#1#2{}%
\def\gobblethree#1#2#3{}%
\def\identity#1{#1}%
\def\@emptymarkA{\@emptymarkB} 
\def\ifempty#1{\@@ifempty #1\@emptymarkA\@emptymarkB}%
\def\@@ifempty#1#2\@emptymarkB{\ifx #1\@emptymarkA}%
\def\@gobbleminus#1{\ifx-#1\else#1\fi}%
\def\ifinteger#1{\ifcat_\ifnum9<1\@gobbleminus#1 _\else A\fi}%
\def\isinteger{TT\fi\ifinteger}%
\def\@gobblemeaning#1:->{}%
\def\sanitize{\expandafter\@gobblemeaning\meaning}%
\def\ifundefined#1{\expandafter\ifx\csname#1\endcsname\relax}%
\def\csn#1{\csname#1\endcsname}%
\def\ece#1#2{\expandafter#1\csname#2\endcsname}%
\def\expandonce{\expandafter\noexpand}%
\let\@plainwlog = \wlog
\let\wlog = \gobble
\newlinechar = `^^J
\def\gloggingall{\begingroup \globaldefs = 1 \loggingall \endgroup}%
\def\loggingall{\tracingcommands\tw@\tracingstats\tw@
   \tracingpages\@ne\tracingoutput\@ne\tracinglostchars\@ne
   \tracingmacros\tw@\tracingparagraphs\@ne\tracingrestores\@ne
   \showboxbreadth\maxdimen\showboxdepth\maxdimen
}%
\def\tracingboxes{\showboxbreadth = \maxdimen \showboxdepth = \maxdimen}%
\def\gtracingoff{\begingroup \globaldefs = 1 \tracingoff \endgroup}%
\def\tracingoff{\tracingonline\z@\tracingcommands\z@\tracingstats\z@
  \tracingpages\z@\tracingoutput\z@\tracinglostchars\z@
  \tracingmacros\z@\tracingparagraphs\z@\tracingrestores\z@
  \showboxbreadth5 \showboxdepth3
}%
\begingroup
  \catcode`\{ = 12 \catcode`\} = 12
  \catcode`\[ = 1 \catcode`\] = 2
  \gdef\lbracechar[{]%
  \gdef\rbracechar[}]%
  \catcode`\% = \other
  \gdef\percentchar[%]\endgroup
\def\vpenalty{\ifhmode\par\fi \penalty}%
\def\hpenalty{\ifvmode\leavevmode\fi \penalty}%
\def\iterate{%
  \let\loop@next\relax
  \body
  \let\loop@next\iterate
  \fi
  \loop@next
}%
\def\edefappend#1#2{%
  \toks@ = \expandafter{#1}%
  \edef#1{\the\toks@ #2}%
}%
\def\allowhyphens{\nobreak\hskip\z@skip}%
\long\def\hookprepend{\@hookassign{\the\toks@ii \the\toks@}}%
\long\def\hookappend{\@hookassign{\the\toks@ \the\toks@ii}}%
\let\hookaction = \hookappend % either one should be ok
\long\def\@hookassign#1#2#3{%
  \expandafter\ifx\csname @#2hook\endcsname \relax
    \toks@ = {}%
  \else
    \expandafter\let\expandafter\temp \csname @#2hook\endcsname
    \toks@ = \expandafter{\temp}%
  \fi
  \toks2 = {#3}% Don't expand the argument all the way.
  \ece\edef{@#2hook}{#1}%
}%
\long\def\hookactiononce#1#2{%
  \edefappend#2{\global\let\noexpand#2\relax}
  \hookaction{#1}#2%
}%
\def\hookrun#1{%
  \expandafter\ifx\csname @#1hook\endcsname \relax \else
    \def\temp{\csname @#1hook\endcsname}%
    \expandafter\temp
  \fi
}%
\def\setproperty#1#2#3{\ece\edef{#1@p#2}{#3}}%
\def\setpropertyglobal#1#2#3{\ece\xdef{#1@p#2}{#3}}%
\def\getproperty#1#2{%
  \expandafter\ifx\csname#1@p#2\endcsname\relax
  \else \csname#1@p#2\endcsname
  \fi
}%
\ifx\@undefinedmessage\@undefined
  \def\@undefinedmessage
    {No .aux file; I won't warn you about undefined labels.}%
\fi
%% @texfile{
%%   author = "Karl Berry and Oren Patashnik",
%%   version = "0.99n",
%%   date = "30 September 2013",
%%   filename = "btxmac.tex",
%%   address = "tex-eplain@tug.org",
%%   supported = "yes",
%%   docstring = "Defines macros that make BibTeX work with plain TeX",
%% }
\edef\cite{\the\catcode`@}%
\catcode`@ = 11
\let\@oldatcatcode = \cite
\chardef\@letter = 11
\chardef\@other = 12
\def\@innerdef#1#2{\edef#1{\expandafter\noexpand\csname #2\endcsname}}%
\@innerdef\@innernewcount{newcount}%
\@innerdef\@innernewdimen{newdimen}%
\@innerdef\@innernewif{newif}%
\@innerdef\@innernewwrite{newwrite}%
\def\@gobble#1{}%
\ifx\inputlineno\@undefined
   \let\@linenumber = \empty % Pre-3.0.
\else
   \def\@linenumber{\the\inputlineno:\space}%
\fi
\long\def\@futurenonspacelet#1{\def\cs{#1}%
   \afterassignment\@stepone\let\@nexttoken=
}%
\begingroup % The grouping here avoids stepping on an outside use of `\\'.
\def\\{\global\let\@stoken= }%
\\ % now \@stoken is a space token (\\ is a control symbol, so that
\endgroup
\def\@stepone{\expandafter\futurelet\cs\@steptwo}%
\def\@steptwo{\expandafter\ifx\cs\@stoken\let\@@next=\@stepthree
   \else\let\@@next=\@nexttoken\fi \@@next}%
\def\@stepthree{\afterassignment\@stepone\let\@@next= }%
\def\@getoptionalarg#1{%
   \let\@optionalusercs = #1%
   \let\@optionalnext = \relax
   \@futurenonspacelet\@optionalnext\@bracketcheck
}%
\def\@bracketcheck{%
   \ifx [\@optionalnext
      \expandafter\@@getoptionalarg % we have an optional arg
   \else
      \let\@optionalarg = \empty    % no optional arg
      \expandafter\@optionalusercs
   \fi
}%
\def\@@getoptionalarg[#1]{%
   \def\@optionalarg{#1}%
   \let\@optdummy=\relax % just in case it has become \outer somehow
   \@futurenonspacelet\@optdummy\@optionalusercs
}%
\def\@nnil{\@nil}%
\def\@fornoop#1\@@#2#3{}%
\def\@for#1:=#2\do#3{%
   \edef\@fortmp{#2}%
   \ifx\@fortmp\empty \else
      \expandafter\@forloop#2,\@nil,\@nil\@@#1{#3}%
   \fi
}%
\def\@forloop#1,#2,#3\@@#4#5{\def#4{#1}\ifx #4\@nnil \else
       #5\def#4{#2}\ifx #4\@nnil \else#5\@iforloop #3\@@#4{#5}\fi\fi
}%
\def\@iforloop#1,#2\@@#3#4{\def#3{#1}\ifx #3\@nnil
       \let\@nextwhile=\@fornoop \else
      #4\relax\let\@nextwhile=\@iforloop\fi\@nextwhile#2\@@#3{#4}%
}%
\@innernewif\if@fileexists
\def\@testfileexistence{\@getoptionalarg\@finishtestfileexistence}%
\def\@finishtestfileexistence#1{%
   \begingroup
      \def\extension{#1}%
      \immediate\openin0 =
         \ifx\@optionalarg\empty\jobname\else\@optionalarg\fi
         \ifx\extension\empty \else .#1\fi
         \space
      \ifeof 0
         \global\@fileexistsfalse
      \else
         \global\@fileexiststrue
      \fi
      \immediate\closein0
   \endgroup
}%
\toks0 = {%
\def\bibliographystyle#1{%
   \@readauxfile
   \@writeaux{\string\bibstyle{#1}}%
}%
\let\bibstyle = \@gobble
\let\bblfilebasename = \jobname
\def\bibliography#1{%
   \@readauxfile
   \@writeaux{\string\bibdata{#1}}%
   \@testfileexistence[\bblfilebasename]{bbl}%
   \if@fileexists
      \nobreak
      \@readbblfile
   \fi
}%
\let\bibdata = \@gobble
\def\nocite#1{%
   \@readauxfile
   \@writeaux{\string\citation{#1}}%
}%
\@innernewif\if@notfirstcitation
\def\cite{\@getoptionalarg\@cite}%
\def\@cite#1{%
   \let\@citenotetext = \@optionalarg
   \printcitestart
   \nocite{#1}%
   \@notfirstcitationfalse
   \@for \@citation :=#1\do
   {%
      \expandafter\@onecitation\@citation\@@
   }%
   \ifx\empty\@citenotetext\else
      \printcitenote{\@citenotetext}%
   \fi
   \printcitefinish
}%
\def\@onecitation#1\@@{%
   \if@notfirstcitation
      \printbetweencitations
   \fi
   \expandafter \ifx \csname\@citelabel{#1}\endcsname \relax
      \if@citewarning
         \message{\@linenumber Undefined citation `#1'.}%
      \fi
      \expandafter\gdef\csname\@citelabel{#1}\endcsname{%
         {\tt
            \escapechar = -1
            \nobreak\hskip0pt
            \expandafter\string\csname#1\endcsname
            \nobreak\hskip0pt
         }%
      }%
   \fi
   \printcitepreitem{#1}%
   \csname\@citelabel{#1}\endcsname
   \printcitepostitem
   \@notfirstcitationtrue
}%
\def\@citelabel#1{b@#1}%
\def\@citedef#1#2{{\expandafter\gdef\csname\@citelabel{#1}\endcsname{#2}}}%
\def\@readbblfile{%
   \ifx\@itemnum\@undefined
      \@innernewcount\@itemnum
   \fi
   \begingroup
      \ifx\begin\@undefined
         \def\begin##1##2{%
            \setbox0 = \hbox{\biblabelcontents{##2}}%
            \biblabelwidth = \wd0
         }%
         \let\end = \@gobble % The arg is `thebibliography' again.
      \fi
      \@itemnum = 0
      \def\bibitem{\@getoptionalarg\@bibitem}%
      \def\@bibitem{%
         \ifx\@optionalarg\empty
            \expandafter\@numberedbibitem
         \else
            \expandafter\@alphabibitem
         \fi
      }%
      \def\@alphabibitem##1{%
         \expandafter \xdef\csname\@citelabel{##1}\endcsname {\@optionalarg}%
         \ifx\biblabelprecontents\@undefined
            \let\biblabelprecontents = \relax
         \fi
         \ifx\biblabelpostcontents\@undefined
            \let\biblabelpostcontents = \hss
         \fi
         \@finishbibitem{##1}%
      }%
      \def\@numberedbibitem##1{%
         \advance\@itemnum by 1
         \expandafter \xdef\csname\@citelabel{##1}\endcsname{\number\@itemnum}%
         \ifx\biblabelprecontents\@undefined
            \let\biblabelprecontents = \hss
         \fi
         \ifx\biblabelpostcontents\@undefined
            \let\biblabelpostcontents = \relax
         \fi
         \@finishbibitem{##1}%
      }%
      \def\@finishbibitem##1{%
         \bblitemhook{##1}%
         \biblabelprint{\csname\@citelabel{##1}\endcsname}%
         \@writeaux{\string\@citedef{##1}{\csname\@citelabel{##1}\endcsname}}%
         \ignorespaces
      }%
      \ifx\undefined\em \let\em=\bblem \fi
      \ifx\undefined\emph \let\emph=\bblemph \fi
      \ifx\undefined\mbox \let\mbox=\bblmbox \fi
      \ifx\undefined\newblock \let\newblock=\bblnewblock \fi
      \ifx\undefined\sc \let\sc=\bblsc \fi
      \ifx\undefined\textbf \let\textbf=\bbltextbf \fi
      \frenchspacing
      \clubpenalty = 4000 \widowpenalty = 4000
      \tolerance = 10000 \hfuzz = .5pt
      \everypar = {\hangindent = \biblabelwidth
                      \advance\hangindent by \biblabelextraspace}%
      \bblrm
      \parskip = 1.5ex plus .5ex minus .5ex
      \biblabelextraspace = .5em
      \bblhook
      \input \bblfilebasename.bbl
   \endgroup
}%
\@innernewdimen\biblabelwidth
\@innernewdimen\biblabelextraspace
\def\biblabelprint#1{%
   \noindent
   \hbox to \biblabelwidth{%
      \biblabelprecontents
      \biblabelcontents{#1}%
      \biblabelpostcontents
   }%
   \kern\biblabelextraspace
}%
\def\biblabelcontents#1{{\bblrm [#1]}}%
\def\bblrm{\rm}%
\def\bblem{\it}%
\def\bblemph#1{{\bblem #1\/}}
\def\bbltextbf#1{{\bf #1}}
\def\bblmbox{\leavevmode\hbox}
\def\bblsc{\ifx\@scfont\@undefined
              \font\@scfont = cmcsc10
           \fi
           \@scfont
}%
\def\bblnewblock{\hskip .11em plus .33em minus .07em }%
\let\bblhook = \empty
\let\bblitemhook = \@gobble
\def\printcitestart{[}%         left bracket
\def\printcitefinish{]}%        right bracket
\def\printbetweencitations{, }% comma, space
\let\printcitepreitem\@gobble % takes label
\let\printcitepostitem\empty
\def\printcitenote#1{, #1}%     comma, space, note (if it exists)
\let\citation = \@gobble
\@innernewcount\@numparams
\ifx\newcommand\undefined
\long\def\newcommand#1{%
   \def\@commandname{#1}%
   \@getoptionalarg\@continuenewcommand
}%
\fi
\ifx\renewcommand\undefined
  \let\renewcommand = \newcommand
\fi
\ifx\providecommand\undefined
\long\def\providecommand#1{%
   \def\@commandname{#1}%
   \expandafter\ifx\@commandname \@undefined
     \let\cs=\@continuenewcommand  % undefined, so we'll define it
   \else
     \let\cs=\@gobble              % already defined, so ignore it
   \fi
   \@getoptionalarg\cs
}%
\fi
\def\@continuenewcommand{%
   \@numparams = \ifx\@optionalarg\empty 0\else\@optionalarg \fi \relax
   \@newcommand
}%
\def\@newcommand#1{%
   \def\@startdef{\expandafter\def\@commandname}%
   \ifnum\@numparams=0
      \let\@paramdef = \empty
   \else
      \ifnum\@numparams>9
         \errmessage{\the\@numparams\space is too many parameters}%
      \else
         \ifnum\@numparams<0
            \errmessage{\the\@numparams\space is too few parameters}%
         \else
            \edef\@paramdef{%
               \ifcase\@numparams
                  \empty  No arguments.
               \or ####1%
               \or ####1####2%
               \or ####1####2####3%
               \or ####1####2####3####4%
               \or ####1####2####3####4####5%
               \or ####1####2####3####4####5####6%
               \or ####1####2####3####4####5####6####7%
               \or ####1####2####3####4####5####6####7####8%
               \or ####1####2####3####4####5####6####7####8####9%
               \fi
            }%
         \fi
      \fi
   \fi
   \expandafter\@startdef\@paramdef{#1}%
}%
}%
\ifx\nobibtex\@undefined \the\toks0 \fi
\def\@readauxfile{%
   \if@auxfiledone \else % remember: \@auxfiledonetrue if \noauxfile is defined
      \global\@auxfiledonetrue
      \@testfileexistence{aux}%
      \if@fileexists
         \begingroup
            \endlinechar = -1
            \catcode`@ = 11
            \input \jobname.aux
         \endgroup
      \else
         \message{\@undefinedmessage}%
         \global\@citewarningfalse
      \fi
      \immediate\openout\@auxfile = \jobname.aux
   \fi
}%
\newif\if@auxfiledone
\ifx\noauxfile\@undefined \else \@auxfiledonetrue\fi
\@innernewwrite\@auxfile
\def\@writeaux#1{\ifx\noauxfile\@undefined \write\@auxfile{#1}\fi}%
\ifx\@undefinedmessage\@undefined
   \def\@undefinedmessage{No .aux file; I won't give you warnings about
                          undefined citations.}%
\fi
\@innernewif\if@citewarning
\ifx\noauxfile\@undefined \@citewarningtrue\fi
\catcode`@ = \@oldatcatcode
\let\auxfile = \@auxfile
\let\for = \@for
\let\futurenonspacelet = \@futurenonspacelet
\def\iffileexists{\if@fileexists}%
\let\innerdef = \@innerdef
\let\innernewcount = \@innernewcount
\let\innernewdimen = \@innernewdimen
\let\innernewif = \@innernewif
\let\innernewwrite = \@innernewwrite
\let\linenumber = \@linenumber
\let\readauxfile = \@readauxfile
\let\spacesub = \@spacesub
\let\testfileexistence = \@testfileexistence
\let\writeaux = \@writeaux
\def\innerinnerdef#1{\expandafter\innerdef\csname inner#1\endcsname{#1}}%
\innerinnerdef{newbox}%
\innerinnerdef{newfam}%
\innerinnerdef{newhelp}%
\innerinnerdef{newinsert}%
\innerinnerdef{newlanguage}%
\innerinnerdef{newmuskip}%
\innerinnerdef{newread}%
\innerinnerdef{newskip}%
\innerinnerdef{newtoks}%
\def\immediatewriteaux#1{%
  \ifx\noauxfile\@undefined
    \immediate\write\@auxfile{#1}%
  \fi
}%
\def\bblitemhook#1{\gdef\@hlbblitemlabel{#1}}%
\def\biblabelprint#1{%
   \noindent
   \hbox to \biblabelwidth{%
      \hldest@impl{bib}{\@hlbblitemlabel}%
      \biblabelprecontents
      \biblabelcontents{#1}%
      \biblabelpostcontents
   }%
   \kern\biblabelextraspace
}%
\def\eplainprintcitepreitem#1{\hlstart@impl{cite}{#1}}%
\def\eplainprintcitepostitem{\hlend@impl{cite}}%
\def\printcitepreitem#1{%
  \testfileexistence[\bblfilebasename]{bbl}%
  \iffileexists
    \global\let\printcitepreitem\eplainprintcitepreitem
    \global\let\printcitepostitem\eplainprintcitepostitem
  \else
    \global\let\printcitepreitem\gobble
    \global\let\printcitepostitem\relax
  \fi
  \printcitepreitem{#1}%
}%
\def\@Nnil{\@Nil}%
\def\@Fornoop#1\@@#2#3{}%
\def\For#1:=#2\do#3{%
   \edef\@Fortmp{#2}%
   \ifx\@Fortmp\empty \else
      \expandafter\@Forloop#2,\@Nil,\@Nil\@@#1{#3}%
   \fi
}%
\def\@Forloop#1,#2,#3\@@#4#5{\@Fordef#1\@@#4\ifx #4\@Nnil \else
       #5\@Fordef#2\@@#4\ifx #4\@Nnil \else#5\@iForloop #3\@@#4{#5}\fi\fi
}%
\def\@iForloop#1,#2\@@#3#4{\@Fordef#1\@@#3\ifx #3\@Nnil
       \let\@Nextwhile=\@Fornoop \else
      #4\relax\let\@Nextwhile=\@iForloop\fi\@Nextwhile#2\@@#3{#4}%
}%
\def\@Forspc{ }%
\def\@Fordef{\futurelet\@Fortmp\@@Fordef}% Peep at the next token.
\def\@@Fordef{%
  \expandafter\ifx\@Forspc\@Fortmp % Next token is a space.
    \expandafter\@Fortrim
  \else
    \expandafter\@@@Fordef
  \fi
}%
\expandafter\def\expandafter\@Fortrim\@Forspc#1\@@{\@Fordef#1\@@}%
\def\@@@Fordef#1\@@#2{\def#2{#1}}%
\def\tmpfileextension{.tmp}%
\let\tmpfilebasename = \jobname
\ifx\eTeXversion\undefined
  \innernewwrite\eplain@tmpfile
  \def\scantokens#1{%
    \toks@={#1}%
    \immediate\openout\eplain@tmpfile=\tmpfilebasename\tmpfileextension
    \immediate\write\eplain@tmpfile{\the\toks@}%
    \immediate\closeout\eplain@tmpfile
    \input \tmpfilebasename\tmpfileextension\relax
  }%
\fi
\begingroup
   \makeactive\^^M \makeactive\ % No spaces or ^^M's from here on.
\gdef\obeywhitespace{%
\makeactive\^^M\def^^M{\par\futurelet\next\@finishobeyedreturn}%
\makeactive\ \let =\ %
\aftergroup\@removebox%
\futurelet\next\@finishobeywhitespace%
}%
\gdef\@finishobeywhitespace{{%
\ifx\next %
\aftergroup\@obeywhitespaceloop%
\else\ifx\next^^M%
\aftergroup\gobble%
\fi\fi}}%
\gdef\@finishobeyedreturn{%
\ifx\next^^M\vskip\blanklineskipamount\fi%
\indent%
}%
\endgroup
\def\@obeywhitespaceloop#1{\futurelet\next\@finishobeywhitespace}%
\def\@removebox{%
  \ifhmode
    \setbox0 = \lastbox
    \ifdim\wd0=\parindent
      \setbox2 = \hbox{\unhcopy0}% Preserve \box0 so we can put it back.
      \ifdim\wd2=0pt
        \ignorespaces
      \else
        \box0 % Put it back: it wasn't empty.
      \fi
    \else
       \box0 % Put it back: it wasn't the right width.
    \fi
  \fi
}%
\newskip\blanklineskipamount
\blanklineskipamount = 0pt
\def\frac#1/#2{\leavevmode
   \kern.1em \raise .5ex \hbox{\the\scriptfont0 #1}%
   \kern-.1em $/$%
   \kern-.15em \lower .25ex \hbox{\the\scriptfont0 #2}%
}%
\newdimen\hruledefaultheight  \hruledefaultheight = 0.4pt
\newdimen\hruledefaultdepth   \hruledefaultdepth = 0.0pt
\newdimen\vruledefaultwidth   \vruledefaultwidth = 0.4pt
\def\ehrule{\hrule height\hruledefaultheight depth\hruledefaultdepth}%
\def\evrule{\vrule width\vruledefaultwidth}%
%%% ====================================================================
%%%  @TeX-style-file{
%%%     author          = "Nelson H. F. Beebe",
%%%     version         = "1.10",
%%%     date            = "02 March 1998",
%%%     time            = "08:36:13 MST",
%%%     filename        = "texnames.sty",
%%%     address         = "Center for Scientific Computing
%%%                        Department of Mathematics
%%%                        South Physics Building
%%%                        University of Utah
%%%                        Salt Lake City, UT 84112
%%%                        USA
%%%                        Tel: (801) 581-5254
%%%                        FAX: (801) 581-4148",
%%%     checksum        = "27723 296 1385 12423",
%%%     email           = "beebe@magna.math.utah.edu (Internet)",
%%%     codetable       = "ISO/ASCII",
%%%     keywords        = "TeX names",
%%%     supported       = "yes",
%%%     docstring       = "This style file for AmSTeX, LaTeX, and TeX
%%%                        defines macros for the names of TeX
%%%                        and METAFONT programs, in several
%%%                        letter-case variants:
%%%
%%%                        \AMSTEX, \AMSTeX, \AmSTeX
%%%                        \BIBTEX, \BIBTeX, \BibTeX
%%%                        \LAMSTeX, \LAmSTeX
%%%                        \LaTeX, \LATEX
%%%                        \METAFONT, \MF
%%%                        \SLITEX, \SLITeX, \SLiTeX, \SliTeX
%%%
%%%                        It will NOT redefine any macro that
%%%                        already exists, so it can be included
%%%                        harmlessly after other style files.
%%%
%%%                        In AmSTeX or Plain TeX, just do
%%%
%%%                        \input texnames.sty
%%%
%%%                        In LaTeX, do
%%%
%%%                        \documentstyle[...,texnames]{...}
%%%
%%%                        This file grew out of original work by
%%%
%%%                        Richard Furuta
%%%                        Department of Computer Science
%%%                        University of Maryland
%%%                        College Park, MD  20742
%%%
%%%                        furuta@mimsy.umd.edu
%%%                        seismo!umcp-cs!furuta
%%%
%%%                        22 October 1986, first release (1.00)
%%%
%%%                        1 April 1987 (1.01): Modified by William
%%%                        LeFebvre, Rice University to include
%%%                        definitions for BibTeX and SLiTeX, as they
%%%                        appear in the LaTeX Local User's Guide
%%%                        template (the file latex/local.tex in
%%%                        standard distributions)
%%%
%%%                        26 October 1991 (1.02): Modified by
%%%                        Nelson H. F. Beebe <beebe@math.utah.edu> to
%%%                        add several new macro names, and adapt for
%%%                        use with Plain TeX and AmSTeX.
%%%
%%%                        26 October 1991 (1.03): Add \LaTeX and
%%%                        \LATEX
%%%
%%%                        25 November 1991 (1.04): Add \LamSTeX
%%%                        and \LAMSTeX
%%%
%%%                        27 January 1991 (1.05 and 1.06): Add slanted
%%%                        font support for \MF.  Make several comment
%%%                        changes.  Add a couple of missing % at end
%%%                        of line, and replace blank lines by empty
%%%                        comments.
%%%
%%%                        30 December 1992 (1.07): Use \TeX in
%%%                        definitions of \BibTeX and \LaTeX.  Remove
%%%                        occurrences of \rm.  Change \sc to use
%%%                        \scriptfont instead of hardwiring cmcsc10.
%%%                        Use \cal for \LAMSTeX.
%%%
%%% 			   1 March 1993 (1.08): Consolidate \ifx's onto
%%% 			   single lines for brevity.  Add
%%% 			   \spacefactor1000 to definitions for \TeX and \MF.
%%%
%%%                        16 March 1993 (1.09): Add \AmS, \AMS, \AmSLaTeX,
%%%                        and \AMSLaTeX.
%%%
%%%                        02 March 1998 (1.10): Add \LaTeXe.
%%%
%%%                        The checksum field above contains a CRC-16
%%%                        checksum as the first value, followed by the
%%%                        equivalent of the standard UNIX wc (word
%%%                        count) utility output of lines, words, and
%%%                        characters.  This is produced by Robert
%%%                        Solovay's checksum utility.",
%%%
%%%  }
%%% ====================================================================
\ifx\sc\undefined
    \def\sc{%
      \expandafter\ifx\the\scriptfont\fam\nullfont
        \font\temp = cmr7 \temp
      \else
        \the\scriptfont\fam
      \fi
      \def\uppercasesc{\char\uccode`}%
    }%
\fi
\ifx\uppercasesc\undefined
  \let\uppercasesc = \relax
\fi
\def\TeX{T\kern-.1667em\lower.5ex\hbox{E}\kern-.125emX\spacefactor1000 }%
\ifx\AmS\undefined
    \def\AmS{{\the\textfont2 A}\kern-.1667em\lower.5ex\hbox
        {\the\textfont2 M}\kern-.125em{\the\textfont2 S}}
\fi
\ifx\AMS\undefined \let\AMS=\AmS \fi
\ifx\AmSLaTeX\undefined
    \def\AmSLaTeX{\AmS-\LaTeX}
\fi
\ifx\AMSLaTeX\undefined \let\AMSLaTeX=\AmSLaTeX \fi
\ifx\AmSTeX\undefined
    \def\AmSTeX{$\cal A$\kern-.1667em\lower.5ex\hbox{$\cal M$}%
            \kern-.125em$\cal S$-\TeX}%
\fi
\ifx\AMSTEX\undefined \let\AMSTEX=\AmSTeX \fi
\ifx\AMSTeX\undefined \let\AMSTeX=\AmSTeX \fi
\ifx\BibTeX\undefined
    \def\BibTeX{B{\sc \uppercasesc i\kern-.025em \uppercasesc b}\kern-.08em
                \TeX}%
\fi
\ifx\BIBTeX\undefined \let\BIBTeX=\BibTeX \fi
\ifx\BIBTEX\undefined \let\BIBTEX=\BibTeX \fi
\ifx\LAMSTeX\undefined
    \def\LAMSTeX{L\raise.42ex\hbox{\kern-.3em\the\scriptfont2 A}%
                 \kern-.2em\lower.376ex\hbox{\the\textfont2 M}%
                 \kern-.125em {\the\textfont2 S}-\TeX}%
\fi
\ifx\LamSTeX\undefined \let\LamSTeX=\LAMSTeX \fi
\ifx\LAmSTeX\undefined \let\LAmSTeX=\LAMSTeX \fi
\ifx\LaTeX\undefined
    \def\LaTeX{L\kern-.36em\raise.3ex\hbox{\sc \uppercasesc a}\kern-.15em\TeX}%
\fi
\ifx\LATEX\undefined \let\LATEX=\LaTeX \fi
\ifx\LaTeXe\undefined
    \def\LaTeXe{\LaTeX{}\kern.05em2$_{\textstyle\varepsilon}$}
\fi
\ifx\MF\undefined
    \ifx\manfnt\undefined
            \font\manfnt=logo10
    \fi
    \ifx\manfntsl\undefined
            \font\manfntsl=logosl10
    \fi
    \def\MF{{\ifdim\fontdimen1\font>0pt \let\manfnt = \manfntsl \fi
      {\manfnt META}\-{\manfnt FONT}}\spacefactor1000 }%
\fi
\ifx\METAFONT\undefined \let\METAFONT=\MF \fi
\ifx\SLITEX\undefined
    \def\SLITEX{S\kern-.065em L\kern-.18em\raise.32ex\hbox{i}\kern-.03em\TeX}%
\fi
\ifx\SLiTeX\undefined \let\SLiTeX=\SLITEX \fi
\ifx\SliTeX\undefined \let\SliTeX=\SLITEX \fi
\ifx\SLITeX\undefined \let\SLITeX=\SLITEX \fi
%%%  @TeX-style-file
%%%  {
%%%     author          = "Philip Taylor",
%%%     version         = "3.05",
%%%     date            = "7 April 2011",
%%%     time            = "15:59:31 BST",
%%%     filename        = "path.sty",
%%%     license         = "Unlimited copying and redistribution of this
%%%                        file are permitted as long as this file is
%%%                        not modified. Modifications are permitted,
%%%                        but only if the resulting file is not named
%%%                        path.sty regardless of case",
%%%     address         = "c/o The Hellenic Institute
%%%                        Royal Holloway, University of London
%%%                        Egham Hull, Egham
%%%                        Surrey TW20 0EX
%%%                        United Kingdom",
%%%     telephone       = "Tel:   +44 1622 820 735",
%%%     email           = "P.Taylor@Rhul.Ac.Uk",
%%%     codetable       = "ISO/ASCII",
%%%     keywords        = "file name, filename, path name, pathname,
%%%                        discretionary, discretionaries",
%%%     supported       = "yes",
%%%     docstring       = {Computer filenames, host names, and e-mail
%%%                        addresses tend to be long strings that
%%%                        cause line breaking problems for TeX.
%%%                        Sometimes rather long strings are
%%%                        encountered, such as:
%%%
%%%                        mighty-mouse-gw.scrc.symbolics.com
%%%
%%%                        This file defines a macro, \path|...|,
%%%                        similar to LaTeX's \verb|...| macro, that
%%%                        sets the text in the typewriter font,
%%%                        allowing hyphen-less line breaks at
%%%                        punctuation characters.
%%%
%%%                        The default set of punctuation characters is
%%%                        defined as
%%%
%%%                        \discretionaries |~!@$%^&*()_+`-=#{"}[]:;'<>,.?\/|
%%%
%%%                        However, you can change it as needed, for
%%%                        example
%%%
%%%                        \discretionaries +@%!.+
%%%
%%%                        would assign to it the set @ % ! . which
%%%                        commonly occur in electronic mail addresses.
%%%
%%%                        The delimiter characters surrounding the
%%%                        arguments to \discretionaries and \path
%%%                        will normally be a punctuation character not
%%%                        found in the argument, and not otherwise
%%%                        significant to TeX.  In particular, backslash
%%%                        cannot be used as a delimiter.  In the rare
%%%                        event that this is required, set
%%%
%%%                        \specialpathdelimiterstrue
%%%
%%%                        This practice is not recommended, because TeX
%%%                        then runs about four times slower while it is
%%%                        typesetting \path\...\ requests.
%%%                        \specialpathdelimitersfalse is the normal
%%%                        setting.
%%%
%%%                        This file may be used in Plain TeX or AmSTeX
%%%                        by
%%%
%%%                        \input path.sty
%%%
%%%                        and in LaTeX by
%%%
%%%                        \usepackage {path}
%%%
%%%                        The checksum field above, if present, contains
%%%                        a CRC-16 checksum as the first value, followed
%%%                        by the equivalent of the standard UNIX wc (word
%%%                        count) utility output of lines, words, and
%%%                        characters.  This is produced by Robert
%%%                        Solovay's checksum utility.
%%%                       }
%%%  }
%%% ====================================================================
%%% The use of `\path' as a temporary control sequence is a kludge, but
%%% it's a reasonably simple way to accomplish making @ a letter (to
%%% provide ``concealed'' control sequences) without overwriting something
%%% (without an `@' in its name) that might already be defined.
\edef \path {\the \catcode `\@}%
\catcode `\@ = 11
\let \oldc@tcode = \path
%%% Make commercial-at a letter to provide concealed control sequences
\catcode `\@ = 11
%%% and then declare two \count variables with concealed names
\newcount \c@tcode
\newcount \c@unter
%%% and a boolean variable with an open name, to specify the nature of
%%% the delimiters which will be associated with the \path command
\newif \ifspecialpathdelimiters
%%% If desired, you can define \pathafterhook to take 
\let \pathafterhook = \relax
%%% We need to define control sequences which expand to both
%%% active and passive spaces ...
\begingroup
\catcode `\ = 10
\gdef \passivesp@ce { }
\catcode `\ = 13\relax%
\gdef\activesp@ce{ }%
\endgroup
%%% \discretionaries will define a macro \discr@ti@n@ri@s which will
%%% make every character between the first and final <delim> a
%%% discretionary breakpoint.
\def \discretionaries %%% <delim> <chars> <delim>
    {\begingroup
        \c@tcodes = 13
        \discr@tionaries
    }
%%% \discr@tionaries will receive as parameter the initial <delim>
%%% which will delimit the set of discretionaries; this <delim>
%%% will be active.
\def \discr@tionaries #1%%% <delim>
    {\def \discr@ti@naries ##1#1%%% <chars> <delim>
         {\endgroup
          \def \discr@ti@n@ries ####1%%% <char> or <delim>
              {\if   \noexpand ####1\noexpand #1%
                     \let \n@xt = \relax
               \else
                     \catcode `####1 = 13
                     \def ####1{\discretionary
                                  {\char `####1}{}{\char `####1}}%
                     \let \n@xt = \discr@ti@n@ries
               \fi
               \n@xt
              }%
          \def \discr@ti@n@ri@s {\discr@ti@n@ries ##1#1}%
         }%
     \discr@ti@naries
    }
%%% \path, which is the user interface to \p@th, first checks
%%% to see whether \specialpathdelimiters is <true> or <false>;
%%% if <true>, it needs to take special action to ensure that
%%% \stress {all} characters (apart from <space>) are acceptable
%%% as delimiters; this is very time-consuming, and should be
%%% avoided if at all possible.  It also defines \endp@th, to
%%% close the appropriate number of groups, and finally transfers
%%% control to \p@th
\def \path
    {\ifspecialpathdelimiters
        \begingroup
        \c@tcodes = 12
        \def \endp@th {\endgroup \endgroup \pathafterhook}%
     \else
        \def \endp@th {\endgroup \pathafterhook}%
     \fi
     \p@th
    }
%%% \p@th, which has essentially the same syntax as \discretionaries,
%%% expects to be followed by a <delim>, a <path>, and a second instance
%%% of <delim>; it will typeset <path> in the \tt font with hyphenation
%%% inhibited --- breaks (but not true hyphenation) are allowed at any
%%% or all of the special characters which have
%%% previously been declared as \discretionaries.
\def \p@th #1%%% <delim>
    {\begingroup
        \tt
        \c@tcode = \catcode `#1
        \discr@ti@n@ri@s
        \catcode `\ = \active
        \expandafter \edef \activesp@ce {\passivesp@ce \hbox {}}%
        \catcode `#1 = \c@tcode
        \def \p@@th ##1#1% <chars> <delim>
            {\leavevmode \hbox {}##1%
             \endp@th
            }%
     \p@@th
    }
%%% \c@tcodes expects to be followed by the r-h-s of a numeric
%%% assignment optionally including the assignment operator; it saves
%%% the value of the r-h-s in \c@tcode, and invokes \c@tc@des.
\def \c@tcodes {\afterassignment \c@tc@des \c@tcode}
%%% \c@tc@des uses the value saved in \c@tcode, and assigns it to the
%%% \catcode of all characters with the single exception of <space>,
%%% which retains its normal catcode of 10; on exit, every single
%%% character apart from <space> will have the catcode which followed
%%% \c@tcodes.  The code is optimised to avoid unnecessary usage of
%%% save-stack space.
\def \c@tc@des
    {\c@unter = 0
     \loop
            \ifnum \catcode \c@unter = \c@tcode
            \else
                \catcode \c@unter = \c@tcode
            \fi
     \ifnum \c@unter < 255
            \advance \c@unter by 1
     \repeat
     \catcode `\ = 10
    }
%%% We restore the original catcode of commercial-at
\catcode `\@ = \oldc@tcode
%%% Define a default set of discretionary break characters
%%% to include all punctuation characters except vertical
%%% bar
\discretionaries |~!@$%^&*()_+`-=#{}[]:";'<>,.?\/|
\ifx\eTeX\undefined
  \def\eTeX{\hbox{\mathsurround=0pt $\varepsilon$-\kern-.125em\TeX}}%
\fi
\ifx\ExTeX\undefined
  \def\ExTeX{\hbox{\mathsurround=0pt
    $\textstyle\varepsilon_{\kern-0.15em\cal{X}}$\kern-.2em\TeX}}%
\fi
\def\eplain@Xe@reflect#1{%
  \ifx\reflectbox\undefined
    \errmessage{A graphics package must be loaded for \string\XeTeX}%
  \else
    \ifdim \fontdimen1\font>0pt
      \raise 1.75ex \hbox{\kern.1em\rotatebox{180}{#1}}\kern-.1em
    \else
      \reflectbox{#1}%
    \fi
  \fi
}%
\def\eplain@Xe#1{\leavevmode
  \smash{\hbox{X%
    \setbox0=\hbox{\TeX}\setbox2=\hbox{E}%
    \lower\dp0\hbox{\raise\dp2\hbox{\kern-.125em\eplain@Xe@reflect{E}}}%
    \kern-.1667em #1}}}%
\ifx\XeTeX\undefined
  \def\XeTeX{\eplain@Xe\TeX}%
\fi
\ifx\XeLaTeX\undefined
  \def\XeLaTeX{\eplain@Xe{\thinspace\LaTeX}}%
\fi
\def\blackbox{\vrule height .8ex width .6ex depth -.2ex \relax}% square bullet
\def\makeblankbox#1#2{%
  \ifvoid0
    \errhelp = \@makeblankboxhelp
    \errmessage{Box 0 is void}%
  \fi
  \hbox{\lower\dp0
    \vbox{\hidehrule{#1}{#2}%
      \kern -#1% overlap rules
      \hbox to \wd0{\hidevrule{#1}{#2}%
        \raise\ht0\vbox to #1{}% vrule height
        \lower\dp0\vtop to #1{}% vrule depth
        \hfil\hidevrule{#2}{#1}%
      }%
      \kern-#1\hidehrule{#2}{#1}%
    }%
  }%
}%
\newhelp\@makeblankboxhelp{Assigning to the dimensions of a void^^J%
  box has no effect.  Do `\string\setbox0=\string\null' before you^^J%
  define its dimensions.}%
\def\hidehrule#1#2{\kern-#1\hrule height#1 depth#2 \kern-#2}%
\def\hidevrule#1#2{%
  \kern-#1%
  \dimen@=#1\advance\dimen@ by #2%
  \vrule width\dimen@
  \kern-#2%
}%
\newdimen\boxitspace \boxitspace = 3pt
\long\def\boxit#1{%
  \vbox{%
    \ehrule
    \hbox{%
      \evrule
      \kern\boxitspace
      \vbox{\kern\boxitspace \parindent = 0pt #1\kern\boxitspace}%
      \kern\boxitspace
      \evrule
    }%
    \ehrule
  }%
}%
\def\numbername#1{\ifcase#1%
   zero%
   \or one%
   \or two%
   \or three%
   \or four%
   \or five%
   \or six%
   \or seven%
   \or eight%
   \or nine%
   \or ten%
   \or #1%
   \fi
}%
\let\@plainnewif = \newif
\let\@plainnewdimen = \newdimen
\let\newif = \innernewif
\let\newdimen = \innernewdimen
\edef\@eplainoldandcode{\the\catcode`& }%
\catcode`& = 11
\toks0 = {%
\edef\thinlines{\the\catcode`@ }%
\catcode`@ = 11
\let\@oldatcatcode = \thinlines
\def\smash@@{\relax % \relax, in case this comes first in \halign
  \ifmmode\def\next{\mathpalette\mathsm@sh}\else\let\next\makesm@sh
  \fi\next}
\def\makesm@sh#1{\setbox\z@\hbox{#1}\finsm@sh}
\def\mathsm@sh#1#2{\setbox\z@\hbox{$\m@th#1{#2}$}\finsm@sh}
\def\finsm@sh{\ht\z@\z@ \dp\z@\z@ \box\z@}
\edef\@oldandcatcode{\the\catcode`& }%
\catcode`& = 11
\def\&whilenoop#1{}%
\def\&whiledim#1\do #2{\ifdim #1\relax#2\&iwhiledim{#1\relax#2}\fi}%
\def\&iwhiledim#1{\ifdim #1\let\&nextwhile=\&iwhiledim 
        \else\let\&nextwhile=\&whilenoop\fi\&nextwhile{#1}}%
\newif\if&negarg
\newdimen\&wholewidth
\newdimen\&halfwidth
\font\tenln=line10
\def\thinlines{\let\&linefnt\tenln \let\&circlefnt\tencirc
  \&wholewidth\fontdimen8\tenln \&halfwidth .5\&wholewidth}%
\def\thicklines{\let\&linefnt\tenlnw \let\&circlefnt\tencircw
  \&wholewidth\fontdimen8\tenlnw \&halfwidth .5\&wholewidth}%
\def\drawline(#1,#2)#3{\&xarg #1\relax \&yarg #2\relax \&linelen=#3\relax
  \ifnum\&xarg =0 \&vline \else \ifnum\&yarg =0 \&hline \else \&sline\fi\fi}%
\def\&sline{\leavevmode
  \ifnum\&xarg< 0 \&negargtrue \&xarg -\&xarg \&yyarg -\&yarg
  \else \&negargfalse \&yyarg \&yarg \fi
  \ifnum \&yyarg >0 \&tempcnta\&yyarg \else \&tempcnta -\&yyarg \fi
  \ifnum\&tempcnta>6 \&badlinearg \&yyarg0 \fi
  \ifnum\&xarg>6 \&badlinearg \&xarg1 \fi
  \setbox\&linechar\hbox{\&linefnt\&getlinechar(\&xarg,\&yyarg)}%
  \ifnum \&yyarg >0 \let\&upordown\raise \&clnht\z@
  \else\let\&upordown\lower \&clnht \ht\&linechar\fi
  \&clnwd=\wd\&linechar
  \&whiledim \&clnwd <\&linelen \do {%
    \&upordown\&clnht\copy\&linechar
    \advance\&clnht \ht\&linechar
    \advance\&clnwd \wd\&linechar
  }%
  \advance\&clnht -\ht\&linechar
  \advance\&clnwd -\wd\&linechar
  \&tempdima\&linelen\advance\&tempdima -\&clnwd
  \&tempdimb\&tempdima\advance\&tempdimb -\wd\&linechar
  \hskip\&tempdimb \multiply\&tempdima \@m
  \&tempcnta \&tempdima \&tempdima \wd\&linechar \divide\&tempcnta \&tempdima
  \&tempdima \ht\&linechar \multiply\&tempdima \&tempcnta
  \divide\&tempdima \@m
  \advance\&clnht \&tempdima
  \ifdim \&linelen <\wd\&linechar \hskip \wd\&linechar
  \else\&upordown\&clnht\copy\&linechar\fi}%
\def\&hline{\vrule height \&halfwidth depth \&halfwidth width \&linelen}%
\def\&getlinechar(#1,#2){\&tempcnta#1\relax\multiply\&tempcnta 8
  \advance\&tempcnta -9 \ifnum #2>0 \advance\&tempcnta #2\relax\else
  \advance\&tempcnta -#2\relax\advance\&tempcnta 64 \fi
  \char\&tempcnta}%
\def\drawvector(#1,#2)#3{\&xarg #1\relax \&yarg #2\relax
  \&tempcnta \ifnum\&xarg<0 -\&xarg\else\&xarg\fi
  \ifnum\&tempcnta<5\relax \&linelen=#3\relax
    \ifnum\&xarg =0 \&vvector \else \ifnum\&yarg =0 \&hvector
    \else \&svector\fi\fi\else\&badlinearg\fi}%
\def\&hvector{\ifnum\&xarg<0 \rlap{\&linefnt\&getlarrow(1,0)}\fi \&hline
  \ifnum\&xarg>0 \llap{\&linefnt\&getrarrow(1,0)}\fi}%
\def\&vvector{\ifnum \&yarg <0 \&downvector \else \&upvector \fi}%
\def\&svector{\&sline
  \&tempcnta\&yarg \ifnum\&tempcnta <0 \&tempcnta=-\&tempcnta\fi
  \ifnum\&tempcnta <5 
    \if&negarg\ifnum\&yarg>0                   % 3d quadrant; dp > 0
      \llap{\lower\ht\&linechar\hbox to\&linelen{\&linefnt
        \&getlarrow(\&xarg,\&yyarg)\hss}}\else % 4th quadrant; ht > 0
      \llap{\hbox to\&linelen{\&linefnt\&getlarrow(\&xarg,\&yyarg)\hss}}\fi
    \else\ifnum\&yarg>0                        % 1st quadrant; ht > 0
      \&tempdima\&linelen \multiply\&tempdima\&yarg
      \divide\&tempdima\&xarg \advance\&tempdima-\ht\&linechar
      \raise\&tempdima\llap{\&linefnt\&getrarrow(\&xarg,\&yyarg)}\else
      \&tempdima\&linelen \multiply\&tempdima-\&yarg % 2d quadrant; dp > 0
      \divide\&tempdima\&xarg
      \lower\&tempdima\llap{\&linefnt\&getrarrow(\&xarg,\&yyarg)}\fi\fi
  \else\&badlinearg\fi}%
\def\&getlarrow(#1,#2){\ifnum #2 =\z@ \&tempcnta='33\else
\&tempcnta=#1\relax\multiply\&tempcnta \sixt@@n \advance\&tempcnta
-9 \&tempcntb=#2\relax\multiply\&tempcntb \tw@
\ifnum \&tempcntb >0 \advance\&tempcnta \&tempcntb\relax
\else\advance\&tempcnta -\&tempcntb\advance\&tempcnta 64
\fi\fi\char\&tempcnta}%
\def\&getrarrow(#1,#2){\&tempcntb=#2\relax
\ifnum\&tempcntb < 0 \&tempcntb=-\&tempcntb\relax\fi
\ifcase \&tempcntb\relax \&tempcnta='55 \or 
\ifnum #1<3 \&tempcnta=#1\relax\multiply\&tempcnta
24 \advance\&tempcnta -6 \else \ifnum #1=3 \&tempcnta=49
\else\&tempcnta=58 \fi\fi\or 
\ifnum #1<3 \&tempcnta=#1\relax\multiply\&tempcnta
24 \advance\&tempcnta -3 \else \&tempcnta=51\fi\or 
\&tempcnta=#1\relax\multiply\&tempcnta
\sixt@@n \advance\&tempcnta -\tw@ \else
\&tempcnta=#1\relax\multiply\&tempcnta
\sixt@@n \advance\&tempcnta 7 \fi\ifnum #2<0 \advance\&tempcnta 64 \fi
\char\&tempcnta}%
\def\&vline{\ifnum \&yarg <0 \&downline \else \&upline\fi}%
\def\&upline{\hbox to \z@{\hskip -\&halfwidth \vrule width \&wholewidth
   height \&linelen depth \z@\hss}}%
\def\&downline{\hbox to \z@{\hskip -\&halfwidth \vrule width \&wholewidth
   height \z@ depth \&linelen \hss}}%
\def\&upvector{\&upline\setbox\&tempboxa\hbox{\&linefnt\char'66}\raise 
     \&linelen \hbox to\z@{\lower \ht\&tempboxa\box\&tempboxa\hss}}%
\def\&downvector{\&downline\lower \&linelen
      \hbox to \z@{\&linefnt\char'77\hss}}%
\def\&badlinearg{\errmessage{Bad \string\arrow\space argument.}}%
\thinlines
\countdef\&xarg     0
\countdef\&yarg     2
\countdef\&yyarg    4
\countdef\&tempcnta 6
\countdef\&tempcntb 8
\dimendef\&linelen  0
\dimendef\&clnwd    2
\dimendef\&clnht    4
\dimendef\&tempdima 6
\dimendef\&tempdimb 8
\chardef\@arrbox    0
\chardef\&linechar  2
\chardef\&tempboxa  2           % \&linechar and \&tempboxa don't interfere.
\let\lft^%
\let\rt_% distinguish between \rt and \lft
\newif\if@pslope % test for positive slope
\def\@findslope(#1,#2){\ifnum#1>0
  \ifnum#2>0 \@pslopetrue \else\@pslopefalse\fi \else
  \ifnum#2>0 \@pslopefalse \else\@pslopetrue\fi\fi}%
\def\generalsmap(#1,#2){\getm@rphposn(#1,#2)\plnmorph\futurelet\next\addm@rph}%
\def\sline(#1,#2){\setbox\@arrbox=\hbox{\drawline(#1,#2){\sarrowlength}}%
  \@findslope(#1,#2)\d@@blearrfalse\generalsmap(#1,#2)}%
\def\arrow(#1,#2){\setbox\@arrbox=\hbox{\drawvector(#1,#2){\sarrowlength}}%
  \@findslope(#1,#2)\d@@blearrfalse\generalsmap(#1,#2)}%
\newif\ifd@@blearr
\def\bisline(#1,#2){\@findslope(#1,#2)%
  \if@pslope \let\@upordown\raise \else \let\@upordown\lower\fi
  \getch@nnel(#1,#2)\setbox\@arrbox=\hbox{\@upordown\@vchannel
    \rlap{\drawline(#1,#2){\sarrowlength}}%
      \hskip\@hchannel\hbox{\drawline(#1,#2){\sarrowlength}}}%
  \d@@blearrtrue\generalsmap(#1,#2)}%
\def\biarrow(#1,#2){\@findslope(#1,#2)%
  \if@pslope \let\@upordown\raise \else \let\@upordown\lower\fi
  \getch@nnel(#1,#2)\setbox\@arrbox=\hbox{\@upordown\@vchannel
    \rlap{\drawvector(#1,#2){\sarrowlength}}%
      \hskip\@hchannel\hbox{\drawvector(#1,#2){\sarrowlength}}}%
  \d@@blearrtrue\generalsmap(#1,#2)}%
\def\adjarrow(#1,#2){\@findslope(#1,#2)%
  \if@pslope \let\@upordown\raise \else \let\@upordown\lower\fi
  \getch@nnel(#1,#2)\setbox\@arrbox=\hbox{\@upordown\@vchannel
    \rlap{\drawvector(#1,#2){\sarrowlength}}%
      \hskip\@hchannel\hbox{\drawvector(-#1,-#2){\sarrowlength}}}%
  \d@@blearrtrue\generalsmap(#1,#2)}%
\newif\ifrtm@rph
\def\@shiftmorph#1{\hbox{\setbox0=\hbox{$\scriptstyle#1$}%
  \setbox1=\hbox{\hskip\@hm@rphshift\raise\@vm@rphshift\copy0}%
  \wd1=\wd0 \ht1=\ht0 \dp1=\dp0 \box1}}%
\def\@hm@rphshift{\ifrtm@rph
  \ifdim\hmorphposnrt=\z@\hmorphposn\else\hmorphposnrt\fi \else
  \ifdim\hmorphposnlft=\z@\hmorphposn\else\hmorphposnlft\fi \fi}%
\def\@vm@rphshift{\ifrtm@rph
  \ifdim\vmorphposnrt=\z@\vmorphposn\else\vmorphposnrt\fi \else
  \ifdim\vmorphposnlft=\z@\vmorphposn\else\vmorphposnlft\fi \fi}%
\def\addm@rph{\ifx\next\lft\let\temp=\lftmorph\else
  \ifx\next\rt\let\temp=\rtmorph\else\let\temp\relax\fi\fi \temp}%
\def\plnmorph{\dimen1\wd\@arrbox \ifdim\dimen1<\z@ \dimen1-\dimen1\fi
  \vcenter{\box\@arrbox}}%
\def\lftmorph\lft#1{\rtm@rphfalse \setbox0=\@shiftmorph{#1}%
  \if@pslope \let\@upordown\raise \else \let\@upordown\lower\fi
  \llap{\@upordown\@vmorphdflt\hbox to\dimen1{\hss % \dimen1=\wd\@arrbox
    \llap{\box0}\hss}\hskip\@hmorphdflt}\futurelet\next\addm@rph}%
\def\rtmorph\rt#1{\rtm@rphtrue \setbox0=\@shiftmorph{#1}%
  \if@pslope \let\@upordown\lower \else \let\@upordown\raise\fi
  \llap{\@upordown\@vmorphdflt\hbox to\dimen1{\hss
    \rlap{\box0}\hss}\hskip-\@hmorphdflt}\futurelet\next\addm@rph}%
\def\getm@rphposn(#1,#2){\ifd@@blearr \dimen@\morphdist \advance\dimen@ by
  .5\channelwidth \@getshift(#1,#2){\@hmorphdflt}{\@vmorphdflt}{\dimen@}\else
  \@getshift(#1,#2){\@hmorphdflt}{\@vmorphdflt}{\morphdist}\fi}%
\def\getch@nnel(#1,#2){\ifdim\hchannel=\z@ \ifdim\vchannel=\z@
    \@getshift(#1,#2){\@hchannel}{\@vchannel}{\channelwidth}%
    \else \@hchannel\hchannel \@vchannel\vchannel \fi
  \else \@hchannel\hchannel \@vchannel\vchannel \fi}%
\def\@getshift(#1,#2)#3#4#5{\dimen@ #5\relax
  \&xarg #1\relax \&yarg #2\relax
  \ifnum\&xarg<0 \&xarg -\&xarg \fi
  \ifnum\&yarg<0 \&yarg -\&yarg \fi
  \ifnum\&xarg<\&yarg \&negargtrue \&yyarg\&xarg \&xarg\&yarg \&yarg\&yyarg\fi
  \ifcase\&xarg \or  % There is no case 0
    \ifcase\&yarg    % case 1
      \dimen@i \z@ \dimen@ii \dimen@ \or % case (1,0)
      \dimen@i .7071\dimen@ \dimen@ii .7071\dimen@ \fi \or
    \ifcase\&yarg    % case 2
      \or % case 0,2 wrong
      \dimen@i .4472\dimen@ \dimen@ii .8944\dimen@ \fi \or
    \ifcase\&yarg    % case 3
      \or % case 0,3 wrong
      \dimen@i .3162\dimen@ \dimen@ii .9486\dimen@ \or
      \dimen@i .5547\dimen@ \dimen@ii .8321\dimen@ \fi \or
    \ifcase\&yarg    % case 4
      \or % case 0,2,4 wrong
      \dimen@i .2425\dimen@ \dimen@ii .9701\dimen@ \or\or
      \dimen@i .6\dimen@ \dimen@ii .8\dimen@ \fi \or
    \ifcase\&yarg    % case 5
      \or % case 0,5 wrong
      \dimen@i .1961\dimen@ \dimen@ii .9801\dimen@ \or
      \dimen@i .3714\dimen@ \dimen@ii .9284\dimen@ \or
      \dimen@i .5144\dimen@ \dimen@ii .8575\dimen@ \or
      \dimen@i .6247\dimen@ \dimen@ii .7801\dimen@ \fi \or
    \ifcase\&yarg    % case 6
      \or % case 0,2,3,4,6 wrong
      \dimen@i .1645\dimen@ \dimen@ii .9864\dimen@ \or\or\or\or
      \dimen@i .6402\dimen@ \dimen@ii .7682\dimen@ \fi \fi
  \if&negarg \&tempdima\dimen@i \dimen@i\dimen@ii \dimen@ii\&tempdima\fi
  #3\dimen@i\relax #4\dimen@ii\relax }%
\catcode`\&=4  % Back to alignment tab
}%
\catcode`& = 4
\toks2 = {%
\catcode`\&=4  % Back to alignment tab
\def\generalhmap{\futurelet\next\@generalhmap}%
\def\@generalhmap{\ifx\next^ \let\temp\generalhm@rph\else
  \ifx\next_ \let\temp\generalhm@rph\else \let\temp\m@kehmap\fi\fi \temp}%
\def\generalhm@rph#1#2{\ifx#1^
    \toks@=\expandafter{\the\toks@#1{\rtm@rphtrue\@shiftmorph{#2}}}\else
    \toks@=\expandafter{\the\toks@#1{\rtm@rphfalse\@shiftmorph{#2}}}\fi
  \generalhmap}%
\def\m@kehmap{\mathrel{\smash@@{\the\toks@}}}%
\def\mapright{\toks@={\mathop{\vcenter{\smash@@{\drawrightarrow}}}\limits}%
  \generalhmap}%
\def\mapleft{\toks@={\mathop{\vcenter{\smash@@{\drawleftarrow}}}\limits}%
  \generalhmap}%
\def\bimapright{\toks@={\mathop{\vcenter{\smash@@{\drawbirightarrow}}}\limits}%
  \generalhmap}%
\def\bimapleft{\toks@={\mathop{\vcenter{\smash@@{\drawbileftarrow}}}\limits}%
  \generalhmap}%
\def\adjmapright{\toks@={\mathop{\vcenter{\smash@@{\drawadjrightarrow}}}\limits}%
  \generalhmap}%
\def\adjmapleft{\toks@={\mathop{\vcenter{\smash@@{\drawadjleftarrow}}}\limits}%
  \generalhmap}%
\def\hline{\toks@={\mathop{\vcenter{\smash@@{\drawhline}}}\limits}%
  \generalhmap}%
\def\bihline{\toks@={\mathop{\vcenter{\smash@@{\drawbihline}}}\limits}%
  \generalhmap}%
\def\drawrightarrow{\hbox{\drawvector(1,0){\harrowlength}}}%
\def\drawleftarrow{\hbox{\drawvector(-1,0){\harrowlength}}}%
\def\drawbirightarrow{\hbox{\raise.5\channelwidth
  \hbox{\drawvector(1,0){\harrowlength}}\lower.5\channelwidth
  \llap{\drawvector(1,0){\harrowlength}}}}%
\def\drawbileftarrow{\hbox{\raise.5\channelwidth
  \hbox{\drawvector(-1,0){\harrowlength}}\lower.5\channelwidth
  \llap{\drawvector(-1,0){\harrowlength}}}}%
\def\drawadjrightarrow{\hbox{\raise.5\channelwidth
  \hbox{\drawvector(-1,0){\harrowlength}}\lower.5\channelwidth
  \llap{\drawvector(1,0){\harrowlength}}}}%
\def\drawadjleftarrow{\hbox{\raise.5\channelwidth
  \hbox{\drawvector(1,0){\harrowlength}}\lower.5\channelwidth
  \llap{\drawvector(-1,0){\harrowlength}}}}%
\def\drawhline{\hbox{\drawline(1,0){\harrowlength}}}%
\def\drawbihline{\hbox{\raise.5\channelwidth
  \hbox{\drawline(1,0){\harrowlength}}\lower.5\channelwidth
  \llap{\drawline(1,0){\harrowlength}}}}%
\def\generalvmap{\futurelet\next\@generalvmap}%
\def\@generalvmap{\ifx\next\lft \let\temp\generalvm@rph\else
  \ifx\next\rt \let\temp\generalvm@rph\else \let\temp\m@kevmap\fi\fi \temp}%
\toksdef\toks@@=1
\def\generalvm@rph#1#2{\ifx#1\rt % append
    \toks@=\expandafter{\the\toks@
      \rlap{$\vcenter{\rtm@rphtrue\@shiftmorph{#2}}$}}\else % prepend
    \toks@@={\llap{$\vcenter{\rtm@rphfalse\@shiftmorph{#2}}$}}%
    \toks@=\expandafter\expandafter\expandafter{\expandafter\the\expandafter
      \toks@@ \the\toks@}\fi \generalvmap}%
\def\m@kevmap{\the\toks@}%
\def\mapdown{\toks@={\vcenter{\drawdownarrow}}\generalvmap}%
\def\mapup{\toks@={\vcenter{\drawuparrow}}\generalvmap}%
\def\bimapdown{\toks@={\vcenter{\drawbidownarrow}}\generalvmap}%
\def\bimapup{\toks@={\vcenter{\drawbiuparrow}}\generalvmap}%
\def\adjmapdown{\toks@={\vcenter{\drawadjdownarrow}}\generalvmap}%
\def\adjmapup{\toks@={\vcenter{\drawadjuparrow}}\generalvmap}%
\def\vline{\toks@={\vcenter{\drawvline}}\generalvmap}%
\def\bivline{\toks@={\vcenter{\drawbivline}}\generalvmap}%
\def\drawdownarrow{\hbox to5pt{\hss\drawvector(0,-1){\varrowlength}\hss}}%
\def\drawuparrow{\hbox to5pt{\hss\drawvector(0,1){\varrowlength}\hss}}%
\def\drawbidownarrow{\hbox to5pt{\hss\hbox{\drawvector(0,-1){\varrowlength}}%
  \hskip\channelwidth\hbox{\drawvector(0,-1){\varrowlength}}\hss}}%
\def\drawbiuparrow{\hbox to5pt{\hss\hbox{\drawvector(0,1){\varrowlength}}%
  \hskip\channelwidth\hbox{\drawvector(0,1){\varrowlength}}\hss}}%
\def\drawadjdownarrow{\hbox to5pt{\hss\hbox{\drawvector(0,-1){\varrowlength}}%
  \hskip\channelwidth\lower\varrowlength
  \hbox{\drawvector(0,1){\varrowlength}}\hss}}%
\def\drawadjuparrow{\hbox to5pt{\hss\hbox{\drawvector(0,1){\varrowlength}}%
  \hskip\channelwidth\raise\varrowlength
  \hbox{\drawvector(0,-1){\varrowlength}}\hss}}%
\def\drawvline{\hbox to5pt{\hss\drawline(0,1){\varrowlength}\hss}}%
\def\drawbivline{\hbox to5pt{\hss\hbox{\drawline(0,1){\varrowlength}}%
  \hskip\channelwidth\hbox{\drawline(0,1){\varrowlength}}\hss}}%
\def\commdiag#1{\null\,
  \vcenter{\commdiagbaselines
  \m@th\ialign{\hfil$##$\hfil&&\hfil$\mkern4mu ##$\hfil\crcr
      \mathstrut\crcr\noalign{\kern-\baselineskip}
      #1\crcr\mathstrut\crcr\noalign{\kern-\baselineskip}}}\,}%
\def\commdiagbaselines{\baselineskip15pt \lineskip3pt \lineskiplimit3pt }%
\def\gridcommdiag#1{\null\,
  \vcenter{\offinterlineskip
  \m@th\ialign{&\vbox to\vgrid{\vss
    \hbox to\hgrid{\hss\smash@@{$##$}\hss}}\crcr
      \mathstrut\crcr\noalign{\kern-\vgrid}
      #1\crcr\mathstrut\crcr\noalign{\kern-.5\vgrid}}}\,}%
\newdimen\harrowlength \harrowlength=60pt
\newdimen\varrowlength \varrowlength=.618\harrowlength
\newdimen\sarrowlength \sarrowlength=\harrowlength
\newdimen\hmorphposn \hmorphposn=\z@
\newdimen\vmorphposn \vmorphposn=\z@
\newdimen\morphdist  \morphdist=4pt
\dimendef\@hmorphdflt 0       % These two dimensions are
\dimendef\@vmorphdflt 2       % defined by \getm@rphposn
\newdimen\hmorphposnrt  \hmorphposnrt=\z@
\newdimen\hmorphposnlft \hmorphposnlft=\z@
\newdimen\vmorphposnrt  \vmorphposnrt=\z@
\newdimen\vmorphposnlft \vmorphposnlft=\z@
\let\hmorphposnup=\hmorphposnrt
\let\hmorphposndn=\hmorphposnlft
\let\vmorphposnup=\vmorphposnrt
\let\vmorphposndn=\vmorphposnlft
\newdimen\hgrid \hgrid=15pt
\newdimen\vgrid \vgrid=15pt
\newdimen\hchannel  \hchannel=0pt
\newdimen\vchannel  \vchannel=0pt
\newdimen\channelwidth \channelwidth=3pt
\dimendef\@hchannel 0         % Defined via the
\dimendef\@vchannel 2         % macro \getch@nnel
\catcode`& = \@oldandcatcode
\catcode`@ = \@oldatcatcode
}%
\let\newif = \@plainnewif
\let\newdimen = \@plainnewdimen
\ifx\noarrow\@undefined \the\toks0 \the\toks2 \fi
\catcode`& = \@eplainoldandcode
\def\environment#1{%
   \ifx\@groupname\@undefined\else
      \errhelp = \@unnamedendgrouphelp
      \errmessage{`\@groupname' was not closed by \string\endenvironment}%
   \fi
   \edef\@groupname{#1}%
   \begingroup
      \let\@groupname = \@undefined
}%
\def\endenvironment#1{%
   \endgroup
   \edef\@thearg{#1}%
   \ifx\@groupname\@thearg
   \else
      \ifx\@groupname\@undefined
         \errhelp = \@isolatedendenvironmenthelp
         \errmessage{Isolated \string\endenvironment\space for `#1'}%
      \else
         \errhelp = \@mismatchedenvironmenthelp
         \errmessage{Environment `#1' ended, but `\@groupname' started}%
         \endgroup % Probably a typo in the names.
      \fi
   \fi
   \let\@groupname = \@undefined
}%
\newhelp\@unnamedendgrouphelp{Most likely, you just forgot an^^J%
   \string\endenvironment.  Maybe you should try inserting another^^J%
   \string\endgroup to recover.}%
\newhelp\@isolatedendenvironmenthelp{You ended an environment X, but^^J%
   no \string\environment{X} to start it is anywhere in sight.^^J%
   You might also be at an \string\endenvironment\space that would match^^J%
   a \string\begingroup, i.e., you forgot an \string\endgroup.}%
\newhelp\@mismatchedenvironmenthelp{You started an environment named X, but^^J%
   you ended one named Y.  Maybe you made a typo in one^^J%
   or the other of the names?}%
\newif\ifenvironment
\def\checkenv{\ifenvironment \errhelp = \@interwovenenvhelp
   \errmessage{Interwoven environments}%
   \egroup \fi
}%
\newhelp\@interwovenenvhelp{Perhaps you forgot to end the previous^^J%
   environment? I'm finishing off the current group,^^J%
   hoping that will fix it.}%
\newtoks\previouseverydisplay
\let\@leftleftfill\relax % as it was
\newdimen\leftdisplayindent \leftdisplayindent=\parindent
\newif\if@leftdisplays
\def\leftdisplays{%
  \if@leftdisplays\else
    \previouseverydisplay = \everydisplay
    \everydisplay = {\the\previouseverydisplay \leftdisplaysetup}%
    \let\@save@maybedisableeqno = \@maybedisableeqno
    \let\@saveeqno = \eqno
    \let\@saveleqno = \leqno
    \let\@saveeqalignno = \eqalignno
    \let\@saveleqalignno = \leqalignno
    \let\@maybedisableeqno = \relax
    \def\eqno{\hfill\textstyle\enspace}%
    \def\leqno{%
      \hfill
      \hbox to0pt\bgroup
        \kern-\displaywidth
        \kern-\leftdisplayindent    % I'll use just \leftdisplayindent
        $\aftergroup\@leftleqnoend  % inserted after ending $
    }%
    \@redefinealignmentdisplays
    \@leftdisplaystrue
  \fi
}%
\newbox\@lignbox
\newdimen\disprevdepth
\def\centereddisplays{%
  \if@leftdisplays
    \everydisplay = \previouseverydisplay
    \let\@maybedisableeqno = \@save@maybedisableeqno
    \let\eqno = \@saveeqno
    \let\leqno = \@saveleqno
    \let\eqalignno = \@saveeqalignno
    \let\leqalignno = \@saveleqalignno
    \@leftdisplaysfalse
  \fi
}%
\def\leftdisplaysetup{%
   \dimen@ = \leftdisplayindent
   \advance\dimen@ by \leftskip
   \advance\displayindent by \dimen@
   \advance\displaywidth by -\dimen@
   \halign\bgroup##\cr \noalign\bgroup
      \disprevdepth = \prevdepth
      \setbox\z@ = \hbox to\displaywidth\bgroup
      $\displaystyle
      \aftergroup\@lefteqend % inserted after ending $
}
\def\@lefteqend{% gets inserted between the ending $$
   \hfil\egroup% end box 0
   \@putdisplay}
\def\@leftleqnoend{\hss \egroup $}% end the \hbox to 0pt for \leqno, restore $
\def\@putdisplay{%
   \ifvoid\@lignbox %  Ordinary display; use it.
     \moveright\displayindent\box\z@ 
   \else % alignment display; unwrap alignment
     \prevdepth = \dp\@lignbox % affects the skip *below*
     \unvbox\@lignbox
   \fi
   \egroup\egroup % end \noalign, end outer \halign
   $% restore first $ of trailing $$
}
\def\@redefinealignmentdisplays{%
  \def\displaylines##1{
    \global\setbox\@lignbox\vbox{%
      \prevdepth = \disprevdepth
      \displ@y
      \tabskip\displayindent
      \halign{\hbox to\displaywidth
        {$\@lign\displaystyle####\hfil$\hfil}\crcr
              ##1\crcr}}}%
  \def\eqalignno##1{%
    \def\eqno{&}%
    \def\leqno{&}%
    \global\setbox\@lignbox\vbox{%
      \prevdepth = \disprevdepth
      \displ@y
      \advance\displaywidth by \displayindent
      \tabskip\displayindent
      \halign to\displaywidth{%
         \hfil $\@lign\displaystyle{####}$\@leftleftfill\tabskip\z@skip
        &$\@lign\displaystyle{{}####}$\hfil\tabskip\centering
        &\llap{$\@lign####$}\tabskip\z@skip\crcr
        ##1\crcr}}}%
  \def\leqalignno##1{%
    \def\eqno{&}%
    \def\leqno{&}%
    \global\setbox\@lignbox\vbox{%
      \prevdepth = \disprevdepth
      \displ@y
      \advance\displaywidth by \displayindent
      \tabskip\displayindent
      \halign to\displaywidth{%
         \hfil $\@lign\displaystyle{####}$\@leftleftfill\tabskip\z@skip
        &$\@lign\displaystyle{{}####}$\hfil\tabskip\centering
        &\kern-\displaywidth 
         \rlap{\kern\displayindent \kern-\leftdisplayindent$\@lign####$}%
         \tabskip\displaywidth\crcr
        ##1\crcr}}}%
}%
\let\@primitivenoalign = \noalign
\newtoks\@everynoalign
\def\@lefteqalignonoalign#1{%
  \@primitivenoalign{%
    \advance\leftskip by -\parindent
    \advance\leftskip by -\leftdisplayindent
    \parskip = 0pt
    \parindent = 0pt
    \the\@everynoalign
    #1%
  }%
}%
\def\monthname{%
   \ifcase\month
      \or Jan\or Feb\or Mar\or Apr\or May\or Jun%
      \or Jul\or Aug\or Sep\or Oct\or Nov\or Dec%
   \fi
}%
\def\fullmonthname{%
   \ifcase\month
      \or January\or February\or March\or April\or May\or June%
      \or July\or August\or September\or October\or November\or December%
   \fi
}%
\def\timestring{\begingroup
   \count0 = \time
   \divide\count0 by 60
   \count2 = \count0   % The hour, from zero to 23.
   \count4 = \time
   \multiply\count0 by 60
   \advance\count4 by -\count0   % The minute, from zero to 59.
   \ifnum\count4<10
      \toks1 = {0}%
   \else
      \toks1 = {}%
   \fi
   \ifnum\count2<12
      \toks0 = {a.m.}%
   \else
      \toks0 = {p.m.}%
      \advance\count2 by -12
   \fi
   \ifnum\count2=0
      \count2 = 12
   \fi
   \number\count2:\the\toks1 \number\count4 \thinspace \the\toks0
\endgroup}%
\def\timestamp{\number\day\space\monthname\space\number\year\quad\timestring}%
\def\today{\the\day\ \fullmonthname\ \the\year}%
\newskip\abovelistskipamount      \abovelistskipamount = .5\baselineskip
  \newcount\abovelistpenalty      \abovelistpenalty    = 10000
  \def\abovelistskip{\vpenalty\abovelistpenalty \vskip\abovelistskipamount}%
\newskip\interitemskipamount      \interitemskipamount = 0pt
  \newcount\belowlistpenalty      \belowlistpenalty    = -50
  \def\belowlistskip{\vpenalty\belowlistpenalty \vskip\belowlistskipamount}%
\newskip\belowlistskipamount      \belowlistskipamount = .5\baselineskip
  \newcount\interitempenalty      \interitempenalty    = 0
  \def\interitemskip{\vpenalty\interitempenalty \vskip\interitemskipamount}%
\newdimen\listleftindent    \listleftindent = 0pt
\newdimen\listrightindent   \listrightindent = 0pt        
\let\listmarkerspace = \enspace
\newtoks\everylist
\def\listcompact{\interitemskipamount = 0pt \relax}%
\newdimen\@listindent
\def\beginlist{%
  \abovelistskip
  \@listindent = \parindent
  \advance\@listindent by \listleftindent
  \advance\leftskip by \@listindent
  \advance\rightskip by \listrightindent
  \itemnumber = 1
  \the\everylist
}%
\def\li{\@getoptionalarg\@finli}%
\def\@finli{%
  \let\@lioptarg\@optionalarg
  \ifx\@lioptarg\empty \else
    \begingroup
      \@@hldestoff
      \expandafter\writeitemxref\expandafter{\@lioptarg}%
    \endgroup
  \fi
  \ifnum\itemnumber=1 \else \interitemskip \fi
  \begingroup
    \ifx\@lioptarg\empty \else
      \toks@ = \expandafter{\@lioptarg}%
      \let\li@nohldest@marker\marker
      \edef\marker{\noexpand\hldest@impl{li}{\the\toks@}\noexpand\li@nohldest@marker}%
    \fi
    \printitem
  \endgroup
  \advance\itemnumber by 1
  \advance\itemletter by 1
  \advance\itemromannumeral by 1
  \ignorespaces
}%
\def\writeitemxref#1{\definexref{#1}\marker{item}}%
\def\printitem{%
  \par
  \nobreak
  \vskip-\parskip
  \noindent
  \printmarker\marker
}%
\def\printmarker#1{\llap{\marker \enspace}}%
\def\endlist{\belowlistskip}%
\newcount\numberedlistdepth
\newcount\itemnumber
\newcount\itemletter
\newcount\itemromannumeral
\def\numberedmarker{%
  \ifcase\numberedlistdepth
      (impossible)%
  \or \printitemnumber
  \or \printitemletter
  \or \printitemromannumeral
  \else *%
  \fi
}%
\def\printitemnumber{\number\itemnumber}%
\def\printitemletter{\char\the\itemletter}%
\def\printitemromannumeral{\romannumeral\itemromannumeral}%
\def\numberedprintmarker#1{\llap{#1) \listmarkerspace}}%
\def\numberedlist{\environment{@numbered-list}%
  \advance\numberedlistdepth by 1
  \itemletter = `a
  \itemromannumeral = 1
  \beginlist
  \let\marker = \numberedmarker
  \let\printmarker = \numberedprintmarker
}%
\def\endnumberedlist{%
  \par
  \endenvironment{@numbered-list}%
  \endlist
}%
\let\orderedlist = \numberedlist
\let\endorderedlist = \endnumberedlist
\newcount\unorderedlistdepth
\def\unorderedmarker{%
  \ifcase\unorderedlistdepth
      (impossible)%
  \or \blackbox
  \or ---%
  \else *%
  \fi
}%
\def\unorderedprintmarker#1{\llap{#1\listmarkerspace}}%
\def\unorderedlist{\environment{@unordered-list}%
  \advance\unorderedlistdepth by 1
  \beginlist
  \let\marker = \unorderedmarker
  \let\printmarker = \unorderedprintmarker
}%
\def\endunorderedlist{%
  \par
  \endenvironment{@unordered-list}%
  \endlist
}%
\def\listing#1{%
   \par \begingroup
   \@setuplisting
   \setuplistinghook
   \input #1
   \endgroup
}%
\let\setuplistinghook = \relax
\def\linenumberedlisting{%
  \ifx\lineno\undefined \innernewcount\lineno \fi
  \lineno = 0
  \everypar = {\advance\lineno by 1 \printlistinglineno}%
}%
\def\printlistinglineno{\llap{[\the\lineno]\quad}}%
\def\nolastlinelisting{\aftergroup\@removeboxes}%
\def\@removeboxes{%
  \setbox0 = \lastbox
  \ifvoid0
    \ignorespaces % Ignore spaces after the \obeywhitespace's group.
  \else
    \expandafter\@removeboxes
  \fi
}%
{%
  \makeactive\^^L
  \let^^L = \relax
  \gdef\@setuplisting{%
     \uncatcodespecials
     \obeywhitespace
     \makeactive\`
     \makeactive\^^I
     \makeactive\^^L
     \def^^L{\vfill\break}%
     \parskip = 0pt
     \listingfont
  }%
}%
\def\listingfont{\tt}%
{%
   \makeactive\`
   \gdef`{\relax\lq}% Defeat ligatures.
}%
{%
   \makeactive\^^I
   \gdef^^I{\hskip8\fontdimen2\font \relax}%
}%
\def\verbatimescapechar#1{%
  \gdef\@makeverbatimescapechar{%
    \@makeverbatimdoubleescape #1%
    \catcode`#1 = 0
  }%
}%
\def\@makeverbatimdoubleescape#1{%
  \catcode`#1 = \other
  \begingroup
    \lccode`\* = `#1%
    \lowercase{\endgroup \ece\def*{*}}%
}%
\verbatimescapechar\|  % initially escapechar is |
\def\verbatim{\begingroup
  \uncatcodespecials
  \makeactive\` % make space character a single space, not stretchable
  \@makeverbatimescapechar
  \tt\obeywhitespace}
\let\endverbatim = \endgroup
\def\definecontentsfile#1{%
  \ece\innernewwrite{#1file}%
  \ece\innernewif{if@#1fileopened}%
  \ece\let{#1filebasename} = \jobname
  \ece\def{open#1file}{\opencontentsfile{#1}}%
  \ece\def{write#1entry}{\writecontentsentry{#1}}%
  \ece\def{writenumbered#1entry}{\writenumberedcontentsentry{#1}}%
  \ece\def{writenumbered#1line}{\writenumberedcontentsline{#1}}%
  \ece\innernewif{ifrewrite#1file} \csname rewrite#1filetrue\endcsname
  \ece\def{read#1file}{\readcontentsfile{#1}}%
}%
\definecontentsfile{toc}%
\def\opencontentsfile#1{%
  \csname if@#1fileopened\endcsname \else
     \ece{\immediate\openout}{#1file} = \csname #1filebasename\endcsname.#1
     \ece\global{@#1fileopenedtrue}%
  \fi
}%
\def\writecontentsentry#1#2#3{\writenumberedcontentsentry{#1}{#2}{#3}{}}%
\def\writenumberedcontentsentry#1#2#3#4{%
  \csname ifrewrite#1file\endcsname
    \writenumberedcontents@cmdname{#1}{#2}%
    \def\temp{#3}% the text
    \toks2 = \expandafter{#4}%
    \edef\cs{\the\toks2}%
    \edef\@wr{%
      \write\csname #1file\endcsname{%
        \the\toks0 % the \toc...entry control sequence
        {\sanitize\temp}% the text
        \ifx\empty\cs\else {\sanitize\cs}\fi % a secondary number, or nothing
        {\noexpand\folio}% the page number
      }%
    }%
    \@wr
  \fi
  \ignorespaces
}%
\def\writenumberedcontentsline#1#2#3#4{%
  \csname ifrewrite#1file\endcsname
    \writenumberedcontents@cmdname{#1}{#2}%
    \def\temp{#4}% the text
    \toks2 = \expandafter{#3}%
    \edef\cs{\the\toks2}%
    \edef\@wr{%
      \write\csname #1file\endcsname{%
        \the\toks0 % the \toc...entry control sequence
        \ifx\empty\cs\else {\sanitize\cs}\fi % a secondary number, or nothing
        {\sanitize\temp}% the text
        {\noexpand\folio}% the page number
      }%
    }%
    \@wr
  \fi
  \ignorespaces
}%
\def\writenumberedcontents@cmdname#1#2{%
  \csname open#1file\endcsname
  \edef\temp{#2}% Expand PART fully and see if this produced an integer.
  \expandafter\if\expandafter\isinteger\expandafter{\temp}%
    \toks0 = {\expandafter\noexpand \csname #1entry\endcsname}%
    \edef\temp{\the\toks0{\temp}}%
    \toks0 = \expandafter{\temp}%
  \else
    \toks0 = {\expandafter\noexpand \csname #1#2entry\endcsname}%
  \fi
}%
\def\readcontentsfile#1{%
   \edef\temp{%
     \noexpand\testfileexistence[\csname #1filebasename\endcsname]{#1}%
   }\temp
   \if@fileexists
      \input \csname #1filebasename\endcsname.#1\relax
   \fi
}%
\def\tocchapterentry#1#2{\line{\bf #1 \dotfill\ #2}}%
\def\tocsectionentry#1#2{\line{\quad\sl #1 \dotfill\ \rm #2}}%
\def\tocsubsectionentry#1#2{\line{\qquad\rm #1 \dotfill\ #2}}%
\def\tocentry#1#2#3{\line{\rm\hskip#1em #2 \dotfill\ #3}}%
\let\ifxrefwarning = \iftrue
\def\xrefwarningtrue{\@citewarningtrue \let\ifxrefwarning = \iftrue}%
\def\xrefwarningfalse{\@citewarningfalse \let\ifxrefwarning = \iffalse}%
\begingroup
  \catcode`\_ = 8
  \gdef\xrlabel#1{#1_x}%
\endgroup
\def\xrdef#1{%
  \begingroup
    \hldest@impl{xrdef}{#1}%
    \begingroup
      \@@hldestoff
      \definexref{#1}{\noexpand\folio}{page}%
    \endgroup
  \endgroup
  \ignorespaces
}%
\def\definexref#1#2#3{%
  \hldest@impl{definexref}{#1}%
  \edef\temp{#1}%
  \readauxfile
  \edef\@wr{\noexpand\writeaux{\string\@definelabel{\temp}{#2}{#3}}}%
  \@wr
  \ignorespaces
}%
\def\@definelabel#1{% #2 and #3 will be read later.
  \begingroup % Will be ended in \@definelabel@nocheck.
    \expandafter\ifx\csname\xrlabel{#1}\endcsname \relax
      \expandafter\@definelabel@nocheck
    \else
      \expandafter\@definelabel@warn
    \fi
    {#1}%
}%
\def\@definelabel@nocheck#1#2#3{%
    \expandafter\gdef\csname\xrlabel{#1}\endcsname{#2}%
    \setpropertyglobal{\xrlabel{#1}}{class}{#3}%
  \endgroup % From \@definelabel.
}%
\def\@definelabel@warn#1#2#3{%
  \message{^^J\linenumber Label `#1' multiply defined,
           value `#2' of class `#3' overwriting value
           `\csname\xrlabel{#1}\endcsname' of class
           `\getproperty{\xrlabel{#1}}{class}'.}%
  \@definelabel@nocheck{#1}{#2}{#3}%
}%
\def\reftie{\penalty\@M \ }% Do not rely on `~' being defined as a tie.
\let\refspace\ 
\def\xrefn{\@getoptionalarg\@finxrefn}%
\def\@finxrefn#1{%
  \hlstart@impl{ref}{#1}%
  \ifx\@optionalarg\empty \else
    \let\@xrefnoptarg\@optionalarg
    \readauxfile
    {\@@hloff\@xrefnoptarg}\reftie
  \fi
  \plain@xrefn{#1}%
  \hlend@impl{ref}%
}%
\def\plain@xrefn#1{%
  \readauxfile
  \expandafter \ifx\csname\xrlabel{#1}\endcsname\relax
    \if@citewarning
       \message{\linenumber Undefined label `#1'.}%
    \fi
    \expandafter\def\csname\xrlabel{#1}\endcsname{%
      `{\tt
        \escapechar = -1
        \expandafter\string\csname#1\endcsname
      }'%
    }%
  \fi
  \csname\xrlabel{#1}\endcsname % Always produce something.
}%
\let\refn = \xrefn
\def\xrefpageword{p.\thinspace}%
\def\xref{\@getoptionalarg\@finxref}%
\def\@finxref#1{%
  \hlstart@impl{xref}{#1}%
  \ifx\@optionalarg\empty \else
    {\@@hloff\@optionalarg}\refspace
  \fi
  \xrefpageword\plain@xrefn{#1}%
  \hlend@impl{xref}%
}%
\def\@maybewarnref{%
  \ifundefined{amsppt.sty}%
  \else
    \message{Warning: amsppt.sty and Eplain both define \string\ref. See
             the Eplain manual.}%
    \let\amsref = \ref
  \fi
  \let\ref = \eplainref
  \ref
}
\let\ref = \@maybewarnref
\def\eplainref{\@getoptionalarg\@fineplainref}%
\def\@fineplainref{\@generalref{1}{}}%
\def\refs{\let\@optionalarg\empty \@generalref{0}s}%
\def\@generalref#1#2#3{%
  \let\@generalrefoptarg\@optionalarg
  \readauxfile
  \ifcase#1 \else \hlstart@impl{ref}{#3}\fi
  \edef\@generalref@class{\getproperty{\xrlabel{#3}}{class}}%
  \expandafter\ifx\csname \@generalref@class word\endcsname\relax
    \ifx\@generalrefoptarg\empty \else {\@@hloff\@generalrefoptarg\reftie}\fi
  \else
    \begingroup
      \@@hloff
      \ifx\@generalrefoptarg\empty \else \@generalrefoptarg \refspace \fi
      \csname \@generalref@class word\endcsname
      #2\reftie
    \endgroup
  \fi
  \ifcase#1 \hlstart@impl{ref}{#3}\fi
  \plain@xrefn{#3}%
  \hlend@impl{ref}%
}%
\newcount\eqnumber
\newcount\subeqnumber
\def\eqdefn{\@getoptionalarg\@fineqdefn}%
\def\@fineqdefn#1{%
  \ifx\@optionalarg\empty
    \global\advance\eqnumber by 1
    \def\temp{\eqconstruct{\number\eqnumber}}%
  \else
    \def\temp{\@optionalarg}% 
  \fi
  \global\subeqnumber = 0
  \gdef\@currenteqlabel{#1}%
  \toks0 = \expandafter{\@currenteqlabel}%
  \begingroup
    \def\eqrefn{\noexpand\plain@xrefn}%
    \def\xrefn{\noexpand\plain@xrefn}%
    \edef\temp{\noexpand\@eqdefn{\the\toks0}{\temp}}%
    \temp
  \endgroup
}%
\def\eqsubdefn#1{%
  \global\advance\subeqnumber by 1
  \toks0 = {#1}%
  \toks2 = \expandafter{\@currenteqlabel}%
  \begingroup
    \def\eqrefn{\noexpand\plain@xrefn}%
    \def\xrefn{\noexpand\plain@xrefn}%
    \def\eqsubreftext{\noexpand\eqsubreftext}%
    \edef\temp{%
      \noexpand\@eqdefn
        {\the\toks0}%
        {\eqsubreftext{\eqrefn{\the\toks2}}{\the\subeqnumber}}%
    }%
    \temp           
  \endgroup
}%
\newcount\phantomeqnumber
\def\phantomeqlabel{PHEQ\the\phantomeqnumber}%
\def\@eqdefn#1#2{%
  \ifempty{#1}%
    \global\advance\phantomeqnumber by 1
    \edef\hl@eqlabel{\phantomeqlabel}%
    \readauxfile
  \else
    \def\hl@eqlabel{#1}%
    {\@@hldestoff \definexref{#1}{#2}{eq}}%
  \fi
  \hldest@impl{eq}{\hl@eqlabel}%
  \begingroup % \@definelabel@nocheck will end this group.
    \@definelabel@nocheck{#1}{#2}{eq}%
}%
\def\eqdef{\@getoptionalarg\@fineqdef}%
\def\@fineqdef{%
  \toks0 = \expandafter{\@optionalarg}%
  \edef\temp{\noexpand\@eqdef{\noexpand\eqdefn[\the\toks0]}}%
  \temp
}%
\def\eqsubdef{\@eqdef\eqsubdefn}%
\def\@eqdef#1#2{%
  \@maybedisableeqno
  \eqnum #1{#2}% Define the label and hyperlink destination.
        \let\@optionalarg\empty % \@fineqref will examine \@optionalarg.
        {\@@hloff\@fineqref{#2}}% Print the text without a hyperlink.
  \@mayberestoreeqno
  \ignorespaces
}%
\let\@mayberestoreeqno = \relax
\def\@maybedisableeqno{%
  \ifinner
    \global\let\eqno = \relax
    \global\let\leqno = \relax
    \global\let\@mayberestoreeqno = \@restoreeqno
  \fi
}%
\let\@primitiveeqno = \eqno
\let\@primitiveleqno = \leqno
\def\@restoreeqno{%
  \global\let\eqno = \@primitiveeqno
  \global\let\leqno = \@primitiveleqno
  \global\let\@mayberestoreeqno = \empty
}%
\def\righteqnumbers{%
  \def\eqnum{\eqno}%
  \def\eqalignnum{\eqalignno}%
}%
\def\lefteqnumbers{%
  \def\eqnum{\leqno}%
  \def\eqalignnum{\leqalignno}%
}%
\righteqnumbers
\def\eqrefn{\@getoptionalarg\@fineqrefn}%
\def\@fineqrefn#1{%
  \eqref@start{#1}%
  \plain@xrefn{#1}%
  \hlend@impl{eq}%
}%
\def\eqref{\@getoptionalarg\@fineqref}%
\def\@fineqref#1{%
  \eqref@start{#1}%
  \eqprint{\plain@xrefn{#1}}%
  \hlend@impl{eq}%
}%
\def\eqref@start#1{%
  \let\@eqrefoptarg\@optionalarg
  \ifempty{#1}%
    \hlstart@impl{eq}{\phantomeqlabel}%
  \else
    \hlstart@impl{eq}{#1}%
  \fi
  \ifx\@eqrefoptarg\empty \else
    {\@@hloff\@eqrefoptarg\reftie}%
  \fi
}%
\let\eqconstruct = \identity
\def\eqprint#1{(#1)}%
\def\eqsubreftext#1#2{#1.#2}%
\let\extraidxcmdsuffixes = \empty
\def\defineindex#1{%
  \def\@idxprefix{#1}%
  \expandafter\innernewif\csname if\@idxprefix dx\endcsname
  \csname \@idxprefix dxtrue\endcsname
  \for\@idxcmd:=,marked,submarked,name%
                \extraidxcmdsuffixes\do
  {%
    \@defineindexcmd\@idxcmd
  }%
  \ece\innernewwrite{@#1indexfile}%
  \ece\innernewif{if@#1indexfileopened}%
}%
\newif\ifsilentindexentry
\def\@defineindexcmd#1{%
  \@defineoneindexcmd{s}{#1}\silentindexentrytrue
  \@defineoneindexcmd{}{#1}\silentindexentryfalse
}%
\def\@defineoneindexcmd#1#2#3{%
  \toks@ = {#3}%
  \edef\temp{%
    \def
      \expandonce\csname#1\@idxprefix dx#2\endcsname % e.g., \idx or \sidxname.
      {\def\noexpand\@idxprefix{\@idxprefix}% define \@idxprefix
       \expandonce\csname @@#1idx#2\endcsname
      }%
    \def
      \expandonce\csname @@#1idx#2\endcsname{% e.g., \@@idx
        \the\toks@
        \noexpand\@idxgetrange\expandonce\csname @#1idx#2\endcsname
      }%
  }%
  \temp
}%
\let\indexfilebasename = \jobname
\def\@idxwrite#1#2{%
  \csname if\@idxprefix dx\endcsname
    \@openidxfile
    \def\temp{#1}%
    \edef\@wr{%
      \expandafter\write\csname @\@idxprefix indexfile\endcsname{%
        \string\indexentry
        {\sanitize\temp}%
        {\noexpand#2}%
      }%
    }%
    \@wr
  \else
    \write-1{}%
  \fi
  \ifindexproofing
    \def\temp{#1}%
    \edef\temp{%
      \insert\@indexproof{\noexpand\indexproofterm{\sanitize\temp}}%
    }%
    \temp
    \ifhmode\allowhyphens\fi
  \fi
  \hookrun{afterindexterm}%
  \ifsilentindexentry \expandafter\ignorespaces\fi
}%
\def\@openidxfile{%
  \csname if@\@idxprefix indexfileopened\endcsname \else
    \expandafter\immediate\openout\csname @\@idxprefix indexfile\endcsname =
      \indexfilebasename.\@idxprefix dx
    \expandafter\global\csname @\@idxprefix indexfileopenedtrue\endcsname
  \fi
}%
\newif\ifindexproofing
\newinsert\@indexproof
\dimen\@indexproof = \maxdimen                  % No limit on number of terms.
\count\@indexproof = 0  \skip\@indexproof = 0pt % They take up no space.
\font\indexprooffont = cmtt8
\def\indexproofterm#1{\hbox{\strut \indexprooffont #1}}%
\let\@plainmakeheadline = \makeheadline
\def\makeheadline{%
  \expandafter\ifx\csname\idxpageanchor{\folio}\endcsname\relax \else
    {\@@hldeston \hldest@impl{idx}{\hlidxpagelabel{\folio}}}%
  \fi
  \indexproofunbox
  \@plainmakeheadline
}%
\def\indexsetmargins{%
  \ifx\undefined\outsidemargin
    \dimen@ = 1truein
    \advance\dimen@ by \hoffset
    \edef\outsidemargin{\the\dimen@}%
    \let\insidemargin = \outsidemargin
  \fi
}%
\def\indexproofunbox{%
  \ifvoid\@indexproof\else
    \indexsetmargins
    \rlap{%
      \kern\hsize
      \ifodd\pageno \kern\outsidemargin \else \kern\insidemargin \fi
      \vbox to 0pt{\unvbox\@indexproof\vss}%
    }\nointerlineskip
  \fi
}%
\def\idxrangebeginword{begin}%
\def\idxbeginrangemark{(}% the corresponding magic char to go in the idx file
\def\idxrangeendword{end}%
\def\idxendrangemark{)}%
\def\idxseecmdword{see}%
\def\idxseealsocmdword{seealso}%
\newif\if@idxsee
\newif\if@hlidxencap
\let\@idxseenterm = \relax
\def\idxpagemarkupcmdword{pagemarkup}%
\let\@idxpagemarkup = \relax
\def\@idxgetrange#1{%
  \let\@idxrangestr = \empty
  \let\@afteridxgetrange = #1%
  \begingroup
    \catcode\idxargopen=1
    \@getoptionalarg\@finidxgetopt
}%
\def\@finidxgetopt{%
    \global\let\@idxgetrange@arg\@optionalarg
  \endgroup
  \@hlidxencaptrue
  \for\@idxarg:=\@idxgetrange@arg\do{%
    \expandafter\@idxcheckpagemarkup\@idxarg=,%
    \ifx\@idxarg\idxrangebeginword
      \def\@idxrangestr{\idxencapoperator\idxbeginrangemark}%
    \else
      \ifx\@idxarg\idxrangeendword
        \def\@idxrangestr{\idxencapoperator\idxendrangemark}%
        \@hlidxencapfalse
      \else
        \ifx\@idxarg\idxseecmdword
          \def\@idxpagemarkup{indexsee}%
          \@idxseetrue
          \@hlidxencapfalse
        \else
          \ifx\@idxarg\idxseealsocmdword
            \def\@idxpagemarkup{indexseealso}%
            \@idxseetrue
            \@hlidxencapfalse
          \else
             \ifx\@idxpagemarkup\relax
               \errmessage{Unrecognized index option `\@idxarg'}%
             \fi
          \fi
        \fi
      \fi
    \fi
  }%
  \ifnum\hldest@place@idx < 0 \else
    \if@hlidxencap
      \ifx\@idxpagemarkup\relax
        \let\@idxpagemarkup\empty
      \fi
      \ifcase\hldest@place@idx \relax
        \edef\@idxpagemarkup{hlidxpage{\@idxpagemarkup}}%
        \definepageanchor{\noexpand\folio}%
      \else
        \global\advance\hlidxlabelnumber by 1
        \edef\@idxpagemarkup{hlidx{\hlidxlabel}{\@idxpagemarkup}}%
        \hldest@impl{idx}{\hlidxlabel}%
      \fi
    \fi
  \fi
  \@afteridxgetrange
}%
\def\@idxcheckpagemarkup#1=#2,{%
  \def\temp{#1}%
  \ifx\temp\idxpagemarkupcmdword
    \if ,#2, % If #2 is empty, complain.
      \errmessage{Missing markup command to `pagemarkup'}%
    \else
      \def\temp##1={##1}%
      \edef\@idxpagemarkup{\temp\string#2}%
    \fi
  \fi
}%
\def\hldest@place@idx{-1}%
\begingroup
  \catcode`\_ = 8
  \gdef\idxpageanchor#1{#1_p}%
\endgroup
\def\definepageanchor#1{%
  \readauxfile
  \edef\@wr{\noexpand\writeaux{\string\@definepageanchor{#1}}}%
  \@wr
  \ignorespaces
}%
\def\@definepageanchor#1{%
  \expandafter\gdef\csname\idxpageanchor{#1}\endcsname{}%
}%
\newcount\hlidxlabelnumber
\def\hlidxlabel{IDX\the\hlidxlabelnumber}%
\def\hlidxpagelabel#1{IDXPG#1}%
\def\hlidx#1#2#3{%
  \ifempty{#2}%
    \let\@idxpageencap\relax
  \else
    \expandafter\let\expandafter\@idxpageencap\csname #2\endcsname
  \fi
  \hlstart@impl{idx}{#1}%
  \@idxpageencap{#3}%
  \hlend@impl{idx}%
}%
\def\hlidxpage#1#2{%
  \@hlidxgetpages{#2}%
  \ifempty{#1}%
    \let\@idxpageencap\relax
  \else
    \expandafter\let\expandafter\@idxpageencap\csname #1\endcsname
  \fi
  \hlstart@impl{idx}{\hlidxpagelabel{\@idxpageiref}}%
  \expandafter\@idxpageencap\expandafter{\@idxpagei}%
  \hlend@impl{idx}%
  \ifx\@idxpageii\empty \else
    \@idxpagesep
    \hlstart@impl{idx}{\hlidxpagelabel{\@idxpageiiref}}%
    \expandafter\@idxpageencap\expandafter{\@idxpageii}%
    \hlend@impl{idx}%
  \fi
}%
\def\@hlidxgetpages#1{%
  \idxparselist{#1}%
  \ifx\idxpagei\empty
    \idxparserange{#1}%
    \ifx\idxpagei\empty
      \def\@idxpageiref{#1}% Label for \hlstart.
    \else
      \let\@idxpageiref\idxpagei % Label for \hlstart.
    \fi
    \def\@idxpagei{#1}%
    \let\@idxpageii\empty
  \else
    \let\@idxpagei\idxpagei
    \let\@idxpageii\idxpageii
    \let\@idxpageiref\idxpagei % Label for \hlstart.
    \let\@idxpageiiref\idxpageii % Label for \hlstart.
    \let\@idxpagesep\idxpagelistdelimiter
  \fi
}%
\def\setidxpagelistdelimiter#1{%
  \gdef\idxpagelistdelimiter{#1}%
  \gdef\@removepagelistdelimiter##1#1{##1}%
  \gdef\@idxparselist##1#1##2\@@{%
    \ifempty{##2}%
      \let\idxpagei\empty
    \else
      \def\idxpagei{##1}%
      \edef\idxpageii{\@removepagelistdelimiter##2}%
    \fi
  }%
  \gdef\idxparselist##1{\@idxparselist##1#1\@@}%
}%
\def\setidxpagerangedelimiter#1{%
  \gdef\idxpagerangedelimiter{#1}%
  \gdef\@removepagerangedelimiter##1#1{##1}%
  \gdef\@idxparserange##1#1##2\@@{%
    \ifempty{##2}%
      \let\idxpagei\empty
    \else
      \def\idxpagei{##1}%
      \edef\idxpageii{\@removepagerangedelimiter##2}%
    \fi
  }%
  \gdef\idxparserange##1{\@idxparserange##1#1\@@}%
}%
\setidxpagelistdelimiter{, }%
\setidxpagerangedelimiter{--}%
\def\idxsubentryseparator{!}%
\def\idxencapoperator{|}%
\def\idxmaxpagenum{99999}%
\newtoks\@idxmaintoks
\newtoks\@idxsubtoks
\def\@idxtokscollect{%
  \edef\temp{\the\@idxsubtoks}%
  \edef\@indexentry{%
    \the\@idxmaintoks
    \ifx\temp\empty\else \idxsubentryseparator\the\@idxsubtoks \fi
    \@idxrangestr
  }%
  \if@idxsee
    \@idxseefalse % Reset so the next term won't be a `see'.
    \edef\temp{\noexpand\idx@getverbatimarg
      {\noexpand\@finidxtokscollect{\idxmaxpagenum}}}%
  \else
    \def\temp{\@finfinidxtokscollect\folio}%
  \fi
  \temp
}%
\def\@finidxtokscollect#1#2{%
  \def\@idxseenterm{#2}%
  \@finfinidxtokscollect{#1}%
}%
\def\@finfinidxtokscollect#1{%
  \ifx\@idxpagemarkup\relax \else
    \toks@ = \expandafter{\@indexentry}%
    \edef\@indexentry{%
      \the\toks@
      \ifx\@idxrangestr\empty \idxencapoperator \fi
      \@idxpagemarkup
    }%
    \let\@idxpagemarkup = \relax
  \fi
  \ifx\@idxseenterm\relax \else
    \toks@ = \expandafter{\@indexentry}%
    \edef\@indexentry{\the\toks@{\sanitize\@idxseenterm}}%
    \let\@idxseenterm = \relax
  \fi
  \expandafter\@idxwrite\expandafter{\@indexentry}{#1}%
}%
\def\@idxcollect#1#2{%
  \@idxmaintoks = {#1}%
  \@idxsubtoks = {#2}%
  \@idxtokscollect
}%
\def\idxargopen{`\{}%
\def\idxargclose{`\}}%
\def\idx@getverbatimarg#1{%
  \def\idx@getverbatimarg@cmd{#1}% Save the command, allowing it to be
  \begingroup
    \uncatcodespecials
    \catcode\idxargopen=1
    \catcode\idxargclose=2
    \catcode`\ =10   % Swallow multiple consequent spaces.
    \catcode`\^^I=10 % Just in case, to avoid dependence on
    \catcode`\^^M=5  % current meanings of tabs and newlines.
    \idx@fingetverbatimarg
}%
\def\idx@fingetverbatimarg#1{\endgroup\idx@getverbatimarg@cmd{#1}}%
\def\idx@getverboptarg#1{%
  \def\idx@optionaltemp{#1}% Save the command, allowing it to be more
  \let\idx@optionalnext = \relax
  \begingroup
    \if@idxsee \catcode\idxargopen=1 \fi
    \@futurenonspacelet\idx@optionalnext\idx@bracketcheck
}%
\def\idx@bracketcheck{%
  \ifx [\idx@optionalnext
    \endgroup % Cancel \catcode\idxargopen=1.
    \expandafter\idx@finbracketcheck
  \else
    \endgroup % Cancel \catcode\idxargopen=1.
    \let\@optionalarg = \empty
    \expandafter\idx@optionaltemp
  \fi
}%
\def\idx@finbracketcheck{%
  \begingroup
    \uncatcodespecials
    \catcode`\ =10   % Swallow multiple consequent spaces.
    \catcode`\^^I=10 % Just in case, to avoid dependence on
    \catcode`\^^M=5  % current meanings of tabs and newlines.
    \idx@@getoptionalarg
}%
\def\idx@@getoptionalarg[#1]{%
  \endgroup
  \def\@optionalarg{#1}%
  \idx@optionaltemp
}%
{\catcode`\&=0
\gdef\idx@scanterm#1{%
  \edef\idx@scanterm@nl@catcode{\the\catcode`\^^M\relax}%
  \catcode`\^^M=5
  \scantokens{#1&endinput}%
  \catcode`\^^M=\idx@scanterm@nl@catcode
}}%
\def\@idx{\idx@getverbatimarg\@finidx}%
\def\@finidx#1{%
  \idx@scanterm{#1}% Produce TERM as output.
  \@idxcollect{#1}{}%
}%
\def\@sidx{\idx@getverbatimarg\@finsidx}%
\def\@finsidx#1{\@idxmaintoks = {#1}\idx@getverboptarg\@finfinsidx}%
\def\@finfinsidx{%
  \@idxsubtoks = \expandafter{\@optionalarg}%
  \@idxtokscollect
}%
\def\idxsortkeysep{@}% This `@' is catcode 11, but it doesn't matter.
\def\@idxconstructmarked#1#2#3{%
  \toks@ = {#2}% The control sequence.
  \toks2 = {#3}% The term.
  \edef\temp{\the\toks2 \idxsortkeysep \the\toks@{\the\toks2}}%
  #1 = \expandafter{\temp}%
}%
\def\@idxmarked#1{\idx@getverbatimarg{\@finidxmarked{#1}}}%
\def\@finidxmarked#1#2{%
  \idx@scanterm{#1{#2}}% Produce \CS{TERM} as output.
  \@idxconstructmarked\@idxmaintoks{#1}{#2}%
  \@idxsubtoks = {}%
  \@idxtokscollect
}%
\def\@sidxmarked#1{\idx@getverbatimarg{\@finsidxmarked{#1}}}%
\def\@finsidxmarked#1#2{%
  \@idxconstructmarked\toks@{#1}{#2}%
  \edef\temp{{\the\toks@}}%
  \expandafter\@finsidx\temp
}%
\def\@idxsubmarked{%
  \let\sidxsubmarked@print\idxsubmarked@print
  \idx@getverbatimarg\@finsidxsubmarked
}%
\def\idxsubmarked@print{\expandafter\idx@scanterm\expandafter}%
\def\@sidxsubmarked{%
  \let\sidxsubmarked@print\gobble
  \idx@getverbatimarg\@finsidxsubmarked
}%
\def\@finsidxsubmarked#1{\@idxmaintoks = {#1}\@@finsidxsubmarked}% Reads TERM.
\def\@@finsidxsubmarked#1{\idx@getverbatimarg{\@@@finsidxsubmarked{#1}}}% Reads \CS.
\def\@@@finsidxsubmarked#1#2{% Reads {\CS}{SUBTERM}.
  \sidxsubmarked@print % Prints for \@idxsubmarked, gobbles for \@sidxsubmarked.
    {\the\@idxmaintoks\space #1{#2}}% Maybe produce `TERM \CS{SUBTERM}' as output.
  \@idxconstructmarked\@idxsubtoks{#1}{#2}%
  \@idxtokscollect
}%
\def\idxnameseparator{, }% as in `Tachikawa, Elizabeth'
\def\@idxcollectname#1#2{%
  \def\temp{#1}%
  \ifx\temp\empty
    \toks@ = {}%
  \else
    \toks@ = \expandafter{\idxnameseparator #1}%
  \fi
  \toks2 = {#2}%
  \edef\temp{\the\toks2 \the\toks@}%
}%
\def\@idxname{\idx@getverbatimarg\@finidxname}%
\def\@finidxname#1{\idx@getverbatimarg{\@finfinidxname{#1}}}%
\def\@finfinidxname#1#2{%
  \idx@scanterm{#1 #2}% Separate the names by a space in the output.
  \@idxcollectname{#1}{#2}%
  \expandafter\@idxcollect\expandafter{\temp}{}%
}%
\def\@sidxname{\idx@getverbatimarg\@finsidxname}%
\def\@finsidxname#1{\idx@getverbatimarg{\@finfinsidxname{#1}}}%
\def\@finfinsidxname#1#2{%
  \@idxcollectname{#1}{#2}%
  \expandafter\@finsidx\expandafter{\temp}%
}%
\let\indexfonts = \relax
\def\readindexfile#1{%
  \edef\@idxprefix{#1}%
  \testfileexistence[\indexfilebasename]{\@idxprefix nd}%
  \iffileexists \begingroup
    \ifx\begin\undefined
      \def\begin##1{\@beginindex}%
      \let\end = \@gobble
    \fi
    \input \indexfilebasename.\@idxprefix nd
    \singlecolumn
  \endgroup
  \else
    \message{No index file \indexfilebasename.\@idxprefix nd.}%
  \fi
}%
\def\@beginindex{%
  \let\item = \@indexitem
  \let\subitem = \@indexsubitem
  \let\subsubitem = \@indexsubsubitem
  \indexfonts
  \doublecolumns
  \parindent = 0pt
  \hookrun{beginindex}%
}%
\let\indexspace = \bigbreak
\let\afterindexterm = \quad
\newskip\aboveindexitemskipamount  \aboveindexitemskipamount = 0pt plus2pt
\def\aboveindexitemskip{\vskip\aboveindexitemskipamount}%
\def\@indexitem{\begingroup
  \@indexitemsetup
  \leftskip = 0pt
  \aboveindexitemskip
  \penalty-100 % Encourage page breaks before items.
  \def\par{\endgraf\endgroup\nobreak}%
}%
\def\@indexsubitem{%
  \@indexitemsetup
  \leftskip = 1em
}%
\def\@indexsubsubitem{%
  \@indexitemsetup
  \leftskip = 2em
}%
\def\@indexitemsetup{%
  \par
  \hangindent = 1em
  \raggedright
  \hyphenpenalty = 10000
  \hookrun{indexitem}%
}%
\def\seevariant{\it}%
\def\indexseeword{see}%
\def\indexsee{\idx@getverbatimarg\@finindexsee}%
\def\@finindexsee#1#2{{\seevariant \indexseeword\/ }\idx@scanterm{#1}}%
\def\indexseealsowords{see also}%
\def\indexseealso{\idx@getverbatimarg\@finindexseealso}%
\def\@finindexseealso#1#2{{\seevariant \indexseealsowords\/ }\idx@scanterm{#1}}%
\defineindex{i}%
\begingroup
  \catcode `\^^M = \active %
  \gdef\flushleft{%
    \def\@endjustifycmd{\@endflushleft}%
    \def\@eoljustifyaction{\null\hfil\break}%
    \let\@firstlinejustifyaction = \relax
    \@startjustify %
  }%
  \gdef\flushright{%
    \def\@endjustifycmd{\@endflushright}%
    \def\@eoljustifyaction{\break\null\hfil}%
    \def\@firstlinejustifyaction{\hfil\null}%
    \@startjustify %
  }%
  \gdef\center{%
    \def\@endjustifycmd{\@endcenter}%
    \def\@eoljustifyaction{\hfil\break\null\hfil}%
    \def\@firstlinejustifyaction{\hfil\null}%
    \@startjustify %
  }%
  \gdef\@startjustify{%
    \parskip = 0pt
    \catcode`\^^M = \active %
    \def^^M{\futurelet\next\@finjustifyreturn}%
    \def\@eateol##1^^M{%
      \def\temp{##1}%
      \@firstlinejustifyaction %
      \ifx\temp\empty\else \temp^^M\fi %
    }%
    \expandafter\aftergroup\@endjustifycmd %
    \checkenv \environmenttrue %
    \par\noindent %
    \@eateol %
  }%
  \gdef\@finjustifyreturn{%
    \@eoljustifyaction %
    \ifx\next^^M%
      \def\par{\endgraf\vskip\blanklineskipamount \global\let\par = \endgraf}%
      \@endjustifycmd %
      \noindent %
      \@firstlinejustifyaction %
    \fi %
  }%
\endgroup
\def\@endflushleft{\unpenalty{\parfillskip = 0pt plus1fil\par}\ignorespaces}%
\def\@endflushright{% Remove the \hfil\null\break we just put on.
   \unskip \setbox0=\lastbox \unpenalty
   {\parfillskip = 0pt \par}\ignorespaces
}%
\def\@endcenter{% Remove the \hfil\null\break we just put on.
   \unskip \setbox0=\lastbox \unpenalty
   {\parfillskip = 0pt plus1fil \par}\ignorespaces
}%
\newcount\abovecolumnspenalty   \abovecolumnspenalty = 10000
\newcount\@linestogo         % Lines remaining to process.
\newcount\@linestogoincolumn % Lines remaining in column.
\newcount\@columndepth       % Number of lines in a column.
\newdimen\@columnwidth       % Width of each column.
\newtoks\crtok  \crtok = {\cr}%
\newcount\currentcolumn
\def\makecolumns#1/#2: {\par \begingroup
   \@columndepth = #1
   \advance\@columndepth by -1
   \divide \@columndepth by #2
   \advance\@columndepth by 1
   \@linestogoincolumn = \@columndepth
   \@linestogo = #1
   \currentcolumn = 1
   \def\@endcolumnactions{%
      \ifnum \@linestogo<2 
         \the\crtok \egroup \endgroup \par % End \valign and \makecolumns.
      \else
         \global\advance\@linestogo by -1
         \ifnum\@linestogoincolumn<2
            \global\advance\currentcolumn by 1
            \global\@linestogoincolumn = \@columndepth
            \the\crtok
         \else
            &\global\advance\@linestogoincolumn by -1
         \fi
      \fi
   }%
   \makeactive\^^M
   \letreturn \@endcolumnactions
   \@columnwidth = \hsize
     \advance\@columnwidth by -\parindent
     \divide\@columnwidth by #2
   \penalty\abovecolumnspenalty
   \noindent % It's not a paragraph (usually).
   \valign\bgroup
     &\hbox to \@columnwidth{\strut \hsize = \@columnwidth ##\hfil}\cr
}%
\newcount\footnotenumber
\newcount\hlfootlabelnumber
\newdimen\footnotemarkseparation \footnotemarkseparation = .5em
\newskip\interfootnoteskip \interfootnoteskip = 0pt
\newtoks\everyfootnote
\newdimen\footnoterulewidth \footnoterulewidth = 2in
\newdimen\footnoteruleheight \footnoteruleheight = 0.4pt
\newdimen\belowfootnoterulespace \belowfootnoterulespace = 2.6pt
\let\@plainfootnote = \footnote
\let\@plainvfootnote = \vfootnote
\def\hlfootlabel{FOOT\the\hlfootlabelnumber}%
\def\hlfootbacklabel{FOOTB\the\hlfootlabelnumber}%
\def\@eplainfootnote#1{\let\@sf\empty % parameter #2 (the text) is read later
  \ifhmode\edef\@sf{\spacefactor\the\spacefactor}\/\fi
  \global\advance\hlfootlabelnumber by 1
  \hlstart@impl{foot}{\hlfootlabel}%
  \hldest@impl{footback}{\hlfootbacklabel}%
  #1%
  \hlend@impl{foot}%
  \@sf\vfootnote{#1}%
}%
\let\footnote\@eplainfootnote
\def\vfootnote#1{\insert\footins\bgroup
  \interlinepenalty\interfootnotelinepenalty
  \splittopskip\ht\strutbox % top baseline for broken footnotes
  \advance\splittopskip by \interfootnoteskip
  \splitmaxdepth\dp\strutbox
  \floatingpenalty\@MM
  \leftskip\z@skip \rightskip\z@skip \spaceskip\z@skip \xspaceskip\z@skip
  \everypar = {}%
  \parskip = 0pt % because of the vskip
  \ifnum\@numcolumns > 1 \hsize = \@normalhsize \fi
  \the\everyfootnote
  \vskip\interfootnoteskip
  \indent\llap{%
    \hlstart@impl{footback}{\hlfootbacklabel}%
    \hldest@impl{foot}{\hlfootlabel}%
    #1%
    \hlend@impl{footback}%
    \kern\footnotemarkseparation
  }%
  \footstrut\futurelet\next\fo@t
}%
\def\footnoterule{\dimen@ = \footnoteruleheight
  \advance\dimen@ by \belowfootnoterulespace
  \kern-\dimen@
  \hrule width\footnoterulewidth height\footnoteruleheight depth0pt
  \kern\belowfootnoterulespace
  \vskip-\interfootnoteskip
}%
\def\numberedfootnote{%
  \global\advance\footnotenumber by 1
  \@eplainfootnote{$^{\number\footnotenumber}$}%
}%
\newdimen\paperheight 
\ifnum\mag=1000
  \paperheight = 11in % No magnification (yet).
\else
  \paperheight = 11truein % We already have a magnification. 
\fi
\def\topmargin{\afterassignment\@finishtopmargin \dimen@}%
\def\@finishtopmargin{%
  \dimen2 = \voffset		% Remember the old \voffset.
  \voffset = \dimen@ \advance\voffset by -1truein
  \advance\dimen2 by -\voffset	% Compute the change in \voffset.
  \advance\vsize by \dimen2	% Change type area accordingly.
}%
\def\advancetopmargin{%
  \dimen@ = 0pt \afterassignment\@finishadvancetopmargin \advance\dimen@
}%
\def\@finishadvancetopmargin{%
  \advance\voffset by \dimen@
  \advance\vsize by -\dimen@
}%
\def\bottommargin{\afterassignment\@finishbottommargin \dimen@}%
\def\@finishbottommargin{%
  \@computebottommargin		% Result in \dimen2.
  \advance\dimen2 by -\dimen@	% Compute the change in the bottom margin.
  \advance\vsize by \dimen2	% Change the type area.
}%
\def\advancebottommargin{%
  \dimen@ = 0pt \afterassignment\@finishadvancebottommargin \advance\dimen@
}%
\def\@finishadvancebottommargin{%
  \advance\vsize by -\dimen@
}%
\def\@computebottommargin{%
  \dimen2 = \paperheight	% The total paper size.
  \advance\dimen2 by -\vsize	% Less the text size.
  \advance\dimen2 by -\voffset	% Less the offset at the top.
  \advance\dimen2 by -1truein	% Less the default offset.
}%
\newdimen\paperwidth
\ifnum\mag=1000
  \paperwidth = 8.5in % No magnification (yet).
\else
  \paperwidth = 8.5truein % We already have a magnification. 
\fi
\def\leftmargin{\afterassignment\@finishleftmargin \dimen@}%
\def\@finishleftmargin{%
  \dimen2 = \hoffset		% Remember the old \hoffset.
  \hoffset = \dimen@ \advance\hoffset by -1truein
  \advance\dimen2 by -\hoffset	% Compute the change in \hoffset.
  \advance\hsize by \dimen2	% Change type area accordingly.
}%
\def\advanceleftmargin{%
  \dimen@ = 0pt \afterassignment\@finishadvanceleftmargin \advance\dimen@
}%
\def\@finishadvanceleftmargin{%
  \advance\hoffset by \dimen@
  \advance\hsize by -\dimen@
}%
\def\rightmargin{\afterassignment\@finishrightmargin \dimen@}%
\def\@finishrightmargin{%
  \@computerightmargin		% Result in \dimen2.
  \advance\dimen2 by -\dimen@	% Compute the change in the right margin.
  \advance\hsize by \dimen2	% Change the type area.
}%
\def\advancerightmargin{%
  \dimen@ = 0pt \afterassignment\@finishadvancerightmargin \advance\dimen@
}%
\def\@finishadvancerightmargin{%
  \advance\hsize by -\dimen@
}%
\def\@computerightmargin{%
  \dimen2 = \paperwidth		% The total paper size.
  \advance\dimen2 by -\hsize	% Less the text size.
  \advance\dimen2 by -\hoffset	% Less the offset at the left.
  \advance\dimen2 by -1truein	% Less the default offset.
}%
\let\@plainm@g = \m@g
\def\m@g{\@plainm@g \paperwidth = 8.5 true in \paperheight = 11 true in}%
\newskip\abovecolumnskip \abovecolumnskip = \bigskipamount
\newskip\belowcolumnskip \belowcolumnskip = \bigskipamount
\newdimen\gutter \gutter = 2pc
\newbox\@partialpage
\newdimen\@normalhsize
\newdimen\@normalvsize  % The original (before multi-columns) \vsize.
\newtoks\previousoutput
\def\quadcolumns{\@columns4}%
\def\triplecolumns{\@columns3}%
\def\doublecolumns{\@columns2}%
\def\begincolumns#1{\ifcase#1\relax \or \singlecolumn \or \@columns2 \or
                            \@columns3 \or \@columns4 \else \relax \fi}%
\def\endcolumns{\singlecolumn}%
\let\@ndcolumns = \relax
\chardef\@numcolumns = 1
\mathchardef\@ejectpartialpenalty = 10141
\chardef\@col@minlines = 3
\chardef\@col@extralines = 3
\newdimen\@col@extraheight
\def\@columns#1{%
  \@ndcolumns
  \global\let\@ndcolumns = \@endcolumns
  \global\chardef\@numcolumns = #1
  \global\previousoutput = \expandafter{\the\output}%
  \global\output = {%
    \ifnum\outputpenalty = -\@ejectpartialpenalty
      \dimen@ = \vsize
      \advance\dimen@ by \@col@minlines\baselineskip
      \global\setbox\@partialpage =
        \vbox  \ifdim \pagetotal > \vsize  to \dimen@  \fi  {%
	  \unvbox255 \unskip
	}%
    \else
      \the\previousoutput
    \fi
  }%
  \vskip \abovecolumnskip
  \vskip \@col@minlines\baselineskip
  \penalty -\@ejectpartialpenalty
  \global\output = {\@columnoutput}%
  \global\@normalhsize = \hsize
  \global\@normalvsize = \vsize
  \count@ = \@numcolumns
  \advance\count@ by -1
  \global\advance\hsize by -\count@\gutter
  \global\divide\hsize by \@numcolumns
  \advance\vsize by -\ht\@partialpage
  \advance\vsize by -\ht\footins
  \ifvoid\footins\else \advance\vsize by -\skip\footins \fi
  \multiply\count\footins by \@numcolumns
  \advance\vsize by -\ht\topins
  \ifvoid\topins\else \advance\vsize by -\skip\topins \fi
  \multiply\count\topins by \@numcolumns
  \global\vsize = \@numcolumns\vsize
  \@col@extraheight=\@col@extralines\baselineskip
  \multiply\@col@extraheight by \@numcolumns
  \global\advance\vsize by \@col@extraheight
}%
\def\gutterbox{\vbox to \dimen0{\vfil\hbox{\hfil}\vfil}}%
\def\@columnsplit{%
  \splittopskip = \topskip
  \splitmaxdepth = \baselineskip
  \begingroup
    \vbadness = 10000
    \global\setbox1 = \vsplit255 to \dimen@  \global\wd1 = \hsize
    \global\setbox3 = \vsplit255 to \dimen@  \global\wd3 = \hsize
    \ifnum\@numcolumns > 2
      \global\setbox5 = \vsplit255 to \dimen@ \global\wd5 = \hsize
    \fi
    \ifnum\@numcolumns > 3
      \global\setbox7 = \vsplit255 to \dimen@ \global\wd7 = \hsize
    \fi
  \endgroup
  \setbox0 = \box255
  \global\setbox255 = \vbox{%
    \unvbox\@partialpage
    \ifcase\@numcolumns \relax\or\relax
      \or \hbox to \@normalhsize{\box1\hfil\gutterbox\hfil\box3}%
      \or \hbox to \@normalhsize{\box1\hfil\gutterbox\hfil\box3%
                                      \hfil\gutterbox\hfil\box5}%
      \or \hbox to \@normalhsize{\box1\hfil\gutterbox\hfil\box3%
                                      \hfil\gutterbox\hfil\box5%
                                      \hfil\gutterbox\hfil\box7}%
    \fi
  }%
  \setbox\@partialpage = \box0
}%
\def\@columnoutput{%
  \dimen@ = \ht255
    \advance\dimen@ by -\@col@extraheight
    \divide\dimen@ by \@numcolumns
  \@columnsplit
  \@recoverclubpenalty 
  \hsize = \@normalhsize % Local to \output's group.
  \vsize = \@normalvsize
  \the\previousoutput
  \unvbox\@partialpage
  \penalty\outputpenalty
  \global\vsize = \@numcolumns\@normalvsize
  \global\advance\vsize by \@col@extraheight
}%
\def\singlecolumn{%
  \@ndcolumns
  \chardef\@numcolumns = 1
  \vskip\belowcolumnskip
  \nointerlineskip
}%
\newbox\@singlecolumnbox 
\newdimen\column@pagegoal
\newdimen\column@vsize
\def\@endcolumns{%
  \global\let\@ndcolumns = \relax
  \par % Shouldn't start in horizontal mode.
  \column@pagegoal = \pagegoal
  \advance\column@pagegoal by-\@col@extraheight
  \ifdim \pagetotal > \column@pagegoal
    \column@vsize = \column@pagegoal
  \else
    \column@vsize = \pagetotal
  \fi
  \global\output = {\global\setbox1 = \box255}%
  \pagegoal = \pagetotal
  \break                     % Exercise the page builder, i.e., \output.
  \setbox2 = \box1           % Save material in box2 in case of overflow.
  \global\output = \expandafter{\the\previousoutput}%
  \setbox\@singlecolumnbox = \box\@partialpage
  \@balancecolumns
}%
\def\@balancecolumns{%
  \global\setbox255 = \copy2  % Retrieve what the fake \output set.
  \dimen@ = \column@vsize
    \divide\dimen@ by \@numcolumns
  \@columnsplit
  \ifvoid\@partialpage
    \global\vsize = \@normalvsize
    \global\hsize = \@normalhsize
    \dump@balanced@columns
    \let\next\relax
  \else
    \advance \column@vsize by \@numcolumns pt
    \test@spill@columns
    \ifspill@columns
      \begingroup
        \vsize = \@normalvsize
        \hsize = \@normalhsize
        \dump@balanced@columns
        \break
        \@recoverclubpenalty
      \endgroup
      \unvbox\@partialpage
      \let\next\@endcolumns
    \else
      \setbox0=\box\@partialpage % Merely to void \@partialpage.
      \let\next\@balancecolumns
    \fi
  \fi
  \next
}%
\def\dump@balanced@columns{%
  \ifvoid\topins\else\topinsert\unvbox\topins\endinsert\fi
  \unvbox\@singlecolumnbox
  \nointerlineskip
  \box255
}%
\newif\ifspill@columns
\def\test@spill@columns{%
  \spill@columnsfalse
  \ifdim \column@vsize > \column@pagegoal
    \ifvoid\footins
      \ifvoid\topins
        \spill@columnstrue
      \fi
    \fi
  \fi
}%
\def\@saveclubpenalty{% save the current value of \clubpenalty
  \edef\@recoverclubpenalty{%
     \global\clubpenalty=\the\clubpenalty\relax%
     \global\let\noexpand\@recoverclubpenalty\relax
  }% the \noexpand handles infinite self-reference
}%
\let\@recoverclubpenalty\relax
\newdimen\temp@dimen
\def\columnfill{%
  \par
  \dimen@ = \pagetotal  % The height of the text so far. 
  \temp@dimen = \vsize  % = \@numcolumns * columnheight
  \advance\temp@dimen by -\@col@extraheight
  \divide\temp@dimen by \@numcolumns  % Compute column height
  \loop
    \ifdim \dimen@ > \temp@dimen  % more material than a column?
      \advance \dimen@ by -\temp@dimen
      \advance \dimen@ by \topskip % fudge factor
  \repeat
  \advance \temp@dimen by -\dimen@
  \advance \temp@dimen by -\prevdepth
  \@saveclubpenalty 
  \clubpenalty=10000\relax
  \hrule height\temp@dimen width0pt depth0pt\relax  % fill space with rule
  \nointerlineskip
  \par
  \nointerlineskip
  \allowbreak \vfil % To allow a page break here.
  \relax
}%
\def\@hldest{%
  \def\hl@prefix{hldest}%
  \let\after@hl@getparam\hldest@aftergetparam
  \begingroup
    \hl@getparam
}%
\def\hldest@aftergetparam{%
  \ifvmode
    \hldest@driver
  \else
    \allowhyphens
    \smash{\ifx\hldest@opt@raise\empty \else \raise\hldest@opt@raise\fi
             \hbox{\hldest@driver}}%
    \allowhyphens
  \fi
  \endgroup
}%
\def\@hlstart{%
  \leavevmode
  \def\hl@prefix{hl}%
  \let\after@hl@getparam\hlstart@aftergetparam
  \begingroup
    \hl@getparam
}%
\def\hlstart@aftergetparam{%
  \ifx\color\undefined \else
    \ifx\hl@opt@color\empty \else
      \ifx\hl@opt@colormodel\empty
        \edef\temp{\noexpand\color{\hl@opt@color}}%
      \else
        \edef\temp{\noexpand\color[\hl@opt@colormodel]{\hl@opt@color}}%
      \fi
      \temp
    \fi
  \fi
  \hl@driver
}%
\def\hl@getparam#1#2{% We'll read #3 (LABEL) later.
  \edef\@hltype{#1}%
  \ifx\@hltype\empty
    \expandafter\let\expandafter\@hltype
      \csname \hl@prefix @type\endcsname
  \fi
  \expandafter\ifx\csname \hl@prefix @typeh@\@hltype\endcsname \relax
    \errmessage{Invalid hyperlink type `\@hltype'}%
  \fi
  \For\hl@arg:=#2\do{%
    \ifx\hl@arg\empty \else
      \expandafter\hl@set@opt\hl@arg=,%
    \fi
  }%
  \bgroup
    \uncatcodespecials
    \catcode`\{=1 \catcode`\}=2
    \@hl@getparam
}%
\def\@hl@getparam#1{%
  \egroup
  \edef\@hllabel{#1}%
  \after@hl@getparam
  \ignorespaces
}%
\def\hl@set@opt#1=#2,{%
  \expandafter\ifx\csname \hl@prefix @opt@#1\endcsname \relax
    \errmessage{Invalid hyperlink option `#1'}%
  \fi
  \if ,#2, % if #2 is empty, complain.
    \errmessage{Missing value for option `#1'}%
  \else
    \def\temp##1={##1}%
    \expandafter\edef\csname \hl@prefix @opt@#1\endcsname{\temp#2}%
  \fi
}%
\def\hldest@impl#1{%
  \expandafter\ifcase\csname hldest@on@#1\endcsname
    \relax\expandafter\gobble
  \else
    \toks@=\expandafter{\csname hldest@type@#1\endcsname}%
    \toks@ii=\expandafter{\csname hldest@opts@#1\endcsname}%
    \edef\temp{\noexpand\hldest{\the\toks@}{\the\toks@ii}}%
    \expandafter\temp
  \fi
}%
\def\hlstart@impl#1{%
  \expandafter\ifcase\csname hl@on@#1\endcsname
    \leavevmode\expandafter\gobble
  \else
    \toks@=\expandafter{\csname hl@type@#1\endcsname}%
    \toks@ii=\expandafter{\csname hl@opts@#1\endcsname}%
    \edef\temp{\noexpand\hlstart{\the\toks@}{\the\toks@ii}}%
    \expandafter\temp
  \fi
}%
\def\hlend@impl#1{%
  \expandafter\ifcase\csname hl@on@#1\endcsname
  \else
    \hlend
  \fi
}%
\def\hl@asterisk@word{*}%
\def\hl@opts@word{opts}%
\newif\if@params@override
\def\hldest@groups{definexref,xrdef,li,eq,bib,foot,footback,idx}%
\def\hl@groups{ref,xref,eq,cite,foot,footback,idx,url,hrefint,hrefext}%
\def\hldesttype{%
  \def\hl@prefix{hldest}%
  \def\hl@param{type}%
  \let\hl@all@groups\hldest@groups
  \futurelet\hl@excl\hl@param@read@excl
}%
\def\hldestopts{%
  \def\hl@prefix{hldest}%
  \def\hl@param{opts}%
  \let\hl@all@groups\hldest@groups
  \futurelet\hl@excl\hl@param@read@excl
}%
\def\hltype{%
  \def\hl@prefix{hl}%
  \def\hl@param{type}%
  \let\hl@all@groups\hl@groups
  \futurelet\hl@excl\hl@param@read@excl
}%
\def\hlopts{%
  \def\hl@prefix{hl}%
  \def\hl@param{opts}%
  \let\hl@all@groups\hl@groups
  \futurelet\hl@excl\hl@param@read@excl
}%
\def\hl@param@read@excl{%
  \ifx!\hl@excl
    \let\next\hl@param@read@opt@arg
    \@params@overridetrue
  \else
    \def\next{\hl@param@read@opt@arg{!}}%
    \@params@overridefalse
  \fi
  \next
}%
\def\hl@param@read@opt@arg#1{% #1 is the `!', ignore it.
  \@getoptionalarg\hl@setparam
}%
\def\@hl@setparam#1{%
  \ifx\@optionalarg\empty
    \hl@setparam@default{#1}% Set default.
  \else
    \let\hl@do@all@groups\gobble
    \For\hl@group:=\@optionalarg\do{%
      \ifx\hl@group\hl@asterisk@word
        \def\hl@do@all@groups{\let\@optionalarg\hl@all@groups \hl@setparam}%
      \else
        \hl@setparam@group{#1}%
      \fi
    }%
    \hl@do@all@groups{#1}%
  \fi
}%
\def\hl@setparam@group#1{%
  \ifx\hl@group\empty
    \hl@setparam@default{#1}%
  \else
    \expandafter\ifx\csname\hl@prefix @\hl@param @\hl@group\endcsname\relax
      \errmessage{Hyperlink group `\hl@prefix:\hl@param:\hl@group' is not defined}%
    \fi
    \ifx\hl@param\hl@opts@word
      \if@params@override
        \expandafter\let\csname\hl@prefix @\hl@param @\hl@group\endcsname\empty
      \fi
      \hl@update@opts@with@list{#1}% #1 will be used in the \for
    \else
      \ece\def{\hl@prefix @\hl@param @\hl@group}{#1}%
    \fi
  \fi
}%
\def\hl@setparam@default#1{%
  \ifx\hl@param\hl@opts@word
    \For\hl@opt:=#1\do{%
      \ifx\hl@opt\empty \else
        \expandafter\hl@set@opt\hl@opt=,%
      \fi
    }%
  \else
    \expandafter\ifx\csname\hl@prefix @\hl@param\endcsname\relax
      \message{Default hyperlink parameter `\hl@prefix:\hl@param' is not defined}%
    \fi
    \ece\def{\hl@prefix @\hl@param}{#1}%
  \fi
}%
\def\hl@update@opts@with@list#1{%
  \global\expandafter\let\expandafter\hl@update@new@list
    \csname \hl@prefix @opts@\hl@group\endcsname
  \begingroup
    \For\hl@opt:=#1\do{%
      \hl@update@opts@with@opt
    }%
  \endgroup
  \ece\let{\hl@prefix @opts@\hl@group}\hl@update@new@list
}%
\def\hl@update@opts@with@opt{%
  \global\let\hl@update@old@list\hl@update@new@list
  \global\let\hl@update@new@list\empty
  \global\let\hl@update@new@opt\hl@opt
  \expandafter\hl@parse@opt@key\hl@opt=,%
  \let\hl@update@new@key\hl@update@key
  \global\let\hl@update@comma\empty
  \begingroup
    \for\hl@opt:=\hl@update@old@list\do{%
      \ifx\hl@opt\empty \else % Skip empty `options'.
        \expandafter\hl@parse@opt@key\hl@opt=,%
        \toks@=\expandafter{\hl@update@new@list}%
        \ifx\hl@update@key\hl@update@new@key
          \ifx\hl@update@new@opt\empty \else % Skip multiple options.
            \toks@ii=\expandafter{\hl@update@new@opt}%
            \xdef\hl@update@new@list{\the\toks@\hl@update@comma\the\toks@ii}%
            \global\let\hl@update@new@opt\empty
            \global\def\hl@update@comma{,}%
          \fi
        \else
          \toks@ii=\expandafter{\hl@opt}%
          \xdef\hl@update@new@list{\the\toks@\hl@update@comma\the\toks@ii}%
          \global\def\hl@update@comma{,}%
        \fi
      \fi
    }%
  \endgroup
  \ifx\hl@update@new@opt\empty \else
    \toks@=\expandafter{\hl@update@new@list}%
    \toks@ii=\expandafter{\hl@update@new@opt}%
    \xdef\hl@update@new@list{\the\toks@\hl@update@comma\the\toks@ii}%
  \fi
}%
\def\hl@parse@opt@key#1=#2,{\def\hl@update@key{#1}}%
\def\hldest@opt@raise{\normalbaselineskip}%
\def\hl@opt@colormodel{cmyk}%
\def\hl@opt@color{0.28,1,1,0.35}%
\def\hldest@on@definexref{0}%
\def\hldest@on@xrdef{0}%
\def\hldest@on@li{0}%
\def\hldest@on@eq{0}% \eqdef and friends
\def\hldest@on@bib{0}% \biblabelprint (BibTeX)
\def\hldest@on@foot{0}% \footnote / \numberedfootnote
\def\hldest@on@footback{0}% back-ref for \footnote / \numberedfootnote
\def\hldest@on@idx{0}% both `page' dests and `exact' dests
\let\hldest@type@definexref\empty
\let\hldest@type@xrdef\empty
\let\hldest@type@li\empty
\let\hldest@type@eq\empty % \eqdef and friends
\let\hldest@type@bib\empty % \biblabelprint (BibTeX)
\let\hldest@type@foot\empty % \footnote / \numberedfootnote
\let\hldest@type@footback\empty % back-ref for \footnote / \numberedfootnote
\let\hldest@type@idx\empty % both `page' dests and `exact' dests
\let\hldest@opts@definexref\empty
\let\hldest@opts@xrdef\empty
\let\hldest@opts@li\empty
\def\hldest@opts@eq{raise=1.7\normalbaselineskip}% \eqdef and friends
\let\hldest@opts@bib\empty % \biblabelprint (BibTeX)
\let\hldest@opts@foot\empty % \footnote / \numberedfootnote
\let\hldest@opts@footback\empty % back-ref for \footnote / \numberedfootnote
\let\hldest@opts@idx\empty % both `page' dests and `exact' dests
\def\hl@on@ref{0}% \refn and \xrefn, \ref, \refs
\def\hl@on@xref{0}%
\def\hl@on@eq{0}% \eqref and \eqrefn
\def\hl@on@cite{0}% \cite (BibTeX)
\def\hl@on@foot{0}% \footnote / \numberedfootnote
\def\hl@on@footback{0}% back-reference for \footnote / \numberedfootnote
\def\hl@on@idx{0}%
\def\hl@on@url{0}% \url from url.sty
\def\hl@on@hrefint{0}% \href with internal #labels
\def\hl@on@hrefext{0}% \href with external labels (URLs)
\let\hl@type@ref\empty % \refn and \xrefn, \ref, \refs
\let\hl@type@xref\empty
\let\hl@type@eq\empty % \eqref and \eqrefn
\let\hl@type@cite\empty % \cite (BibTeX)
\let\hl@type@foot\empty % \footnote / \numberedfootnote
\let\hl@type@footback\empty % back-reference for \footnote / \numberedfootnote
\let\hl@type@idx\empty
\let\hl@type@url\empty % \url from url.sty (this will be set to `url' by
\let\hl@type@hrefint\empty % \href with internal #labels
\let\hl@type@hrefext\empty % \href with external labels (URLs) (this
\let\hl@opts@ref\empty % \refn and \xrefn, \ref, \refs
\let\hl@opts@xref\empty
\let\hl@opts@eq\empty % \eqref and \eqrefn
\let\hl@opts@cite\empty % \cite (BibTeX)
\let\hl@opts@foot\empty % \footnote / \numberedfootnote
\let\hl@opts@footback\empty % back-reference for \footnote / \numberedfootnote
\let\hl@opts@idx\empty
\let\hl@opts@url\empty % \url from url.sty
\let\hl@opts@hrefint\empty % \href with internal #labels
\let\hl@opts@hrefext\empty % \href with external labels (URLs)
\def\@hlon{\@hlonoff@value@stub{hl}\@@hlon1 }%
\def\@hloff{\@hlonoff@value@stub{hl}\@@hloff0 }%
\def\@hldeston{\@hlonoff@value@stub{hldest}\@@hldeston1 }%
\def\@hldestoff{\@hlonoff@value@stub{hldest}\@@hldestoff0 }%
\def\@hlonoff@value@stub#1#2#3{%
  \def\hl@prefix{#1}%
  \let\hl@on@empty#2%
  \def\hl@value{#3}%
  \expandafter\let\expandafter\hl@all@groups
    \csname \hl@prefix @groups\endcsname
  \@getoptionalarg\@finhlswitch
}%
\def\@finhlswitch{%
  \ifx\@optionalarg\empty
    \hl@on@empty
  \fi
  \let\hl@do@all@groups\relax
  \For\hl@group:=\@optionalarg\do{%
    \ifx\hl@group\hl@asterisk@word
      \let\@optionalarg\hl@all@groups
      \let\hl@do@all@groups\@finhlswitch
    \else
      \ifx\hl@group\empty
        \hl@on@empty
      \else
        \expandafter\ifx\csname\hl@prefix @on@\hl@group\endcsname \relax
          \errmessage{Hyperlink group `\hl@prefix:on:\hl@group'
                      is not defined}%
        \fi
        \ece\edef{\hl@prefix @on@\hl@group}{\hl@value}%
      \fi
    \fi
  }%
  \hl@do@all@groups
}%
\def\@@hlon{%
  \let\hlstart\@hlstart
  \let\hlend\@hlend
}%
\def\@@hloff{%
  \def\hlstart##1##2##3{\leavevmode\ignorespaces}%
  \let\hlend\relax
}%
\def\@@hldeston{%
  \let\hldest\@hldest
}%
\def\@@hldestoff{%
  \def\hldest##1##2##3{\ignorespaces}%
}%
\def\hl@idxexact@word{idxexact}%
\def\hl@idxpage@word{idxpage}%
\def\hl@idxnone@word{idxnone}%
\def\hl@raw@word{raw}%
\def\enablehyperlinks{\@getoptionalarg\@finenablehyperlinks}%
\def\@finenablehyperlinks{%
  \let\hl@selecteddriver\empty
  \def\hldest@place@idx{0}%
  \for\hl@arg:=\@optionalarg\do{%
    \ifx\hl@arg\hl@idxexact@word
      \def\hldest@place@idx{1}%
    \else
      \ifx\hl@arg\hl@idxnone@word
        \def\hldest@place@idx{-1}%
      \else
        \ifx\hl@arg\hl@idxpage@word
          \def\hldest@place@idx{0}%
        \else
          \let\hl@selecteddriver\hl@arg
        \fi
      \fi
    \fi
  }%
  \ifx\hl@selecteddriver\empty
    \ifpdf
      \def\hl@selecteddriver{pdftex}%
      \message{^^JEplain: using `pdftex' hyperlink driver.}%
    \else
      \def\hl@selecteddriver{hypertex}%
      \message{^^JEplain: using `hypertex' hyperlink driver.}%
    \fi
  \else
    \expandafter\ifx\csname hldriver@\hl@selecteddriver\endcsname \relax
      \errmessage{No hyperlink driver `\hl@selecteddriver' available}%
    \fi
  \fi
  \let\hl@setparam\@hl@setparam
  \csname hldriver@\hl@selecteddriver\endcsname
  \def\@finenablehyperlinks{\errmessage{Hyperlink driver `\hl@selecteddriver'
                                        already selected}}%
  \let\hldriver@nolinks\undefined
  \let\hldriver@hypertex\undefined
  \let\hldriver@pdftex \undefined
  \let\hldriver@dvipdfm\undefined
  \let\hloff\@hloff
  \let\hlon\@hlon
  \let\hldestoff\@hldestoff
  \let\hldeston\@hldeston
  \hlon[*,]\hloff[foot,footback]%
  \hldeston[*,]\hldestoff[foot,footback]%
}%
\def\hldriver@nolinks{%
  \def\@hldest##1##2##3{%
    \edef\temp{\write-1{hldest: ##3}}%
    \ifvmode
      \temp
    \else
      \allowhyphens
      \expandafter\smash\expandafter{\temp}%
      \allowhyphens
    \fi
    \ignorespaces
  }%
  \def\@hlstart##1##2##3{%
    \leavevmode
    \begingroup % Start the color group.
    \edef\temp{\write-1{hlstart: ##3}}%
    \temp
    \ignorespaces
  }%
  \def\@hlend{%
    \edef\temp{\write-1{hlend}}%
    \temp
    \endgroup % End the color group from \@hlstart.
  }%
  \let\hl@setparam\gobble
}%
{\catcode`\#=\other
\gdef\hlhash{#}}%
\def\hldriver@hypertex{%
  \def\hldest@type{xyz}%
  \let\hldest@opt@cmd \empty
  \def\hldest@driver{%
    \ifx\@hltype\hl@raw@word
      \csname \hldest@opt@cmd \endcsname
    \else
      \special{html:<a name="\@hllabel">}\special{html:</a>}%
    \fi
  }%
  \let\hldest@typeh@raw \empty
  \let\hldest@typeh@xyz \empty
  \def\hl@type{name}%
  \ifx\hl@type@url\empty
    \def\hl@type@url{url}%
  \fi
  \ifx\hl@type@hrefext\empty
    \def\hl@type@hrefext{url}%
  \fi
  \let\hl@opt@cmd  \empty
  \let\hl@opt@ext  \empty
  \let\hl@opt@file \empty
  \def\hl@driver{%
    \ifx\@hltype\hl@raw@word
      \csname \hl@opt@cmd \endcsname
    \else
      \def\hlstart@preamble{html:<a href="}%
      \csname hl@typeh@\@hltype\endcsname
    \fi
  }%
  \let\hl@typeh@raw \empty
  \def\hl@typeh@name{\special{\hlstart@preamble \hlhash\@hllabel">}}%
  \def\hl@typeh@filename{%
    \special{%
      \hlstart@preamble
        file:\hl@opt@file\hl@opt@ext
        \ifempty\@hllabel \else \hlhash\@hllabel\fi
      ">%
    }%
  }%
  \def\hl@typeh@url{%
    \special{%
      \hlstart@preamble
        \@hllabel
      ">%
    }%
  }%
  \def\@hlend{\special{html:</a>}\endgroup}% End the group from \@hlstart.
}%
\def\hldriver@pdftex{%
\ifpdf % PDF output is enabled.
  \def\hldest@type{xyz}%
  \let\hldest@opt@width  \empty
  \let\hldest@opt@height \empty
  \let\hldest@opt@depth  \empty
  \let\hldest@opt@zoom   \empty
  \let\hldest@opt@cmd    \empty
  \def\hldest@driver{%
    \ifx\@hltype\hl@raw@word
      \csname \hldest@opt@cmd \endcsname
    \else
      \pdfdest name{\@hllabel}\@hltype
        \csname hldest@typeh@\@hltype\endcsname
    \fi
  }%
  \let\hldest@typeh@raw   \empty
  \let\hldest@typeh@fit   \empty
  \let\hldest@typeh@fith  \empty
  \let\hldest@typeh@fitv  \empty
  \let\hldest@typeh@fitb  \empty
  \let\hldest@typeh@fitbh \empty
  \let\hldest@typeh@fitbv \empty
  \def\hldest@typeh@fitr{%
    \ifx\hldest@opt@width  \empty \else width  \hldest@opt@width  \fi
    \ifx\hldest@opt@height \empty \else height \hldest@opt@height \fi
    \ifx\hldest@opt@depth  \empty \else depth  \hldest@opt@depth  \fi
  }%
  \def\hldest@typeh@xyz{%
    \ifx\hldest@opt@zoom\empty \else zoom \hldest@opt@zoom \fi
  }%
  \def\hl@type{name}%
  \ifx\hl@type@url\empty
    \def\hl@type@url{url}%
  \fi
  \ifx\hl@type@hrefext\empty
    \def\hl@type@hrefext{url}%
  \fi
  \let\hl@opt@width   \empty
  \let\hl@opt@height  \empty
  \let\hl@opt@depth   \empty
  \def\hl@opt@bstyle  {S}%
  \def\hl@opt@bwidth  {1}%
  \let\hl@opt@bcolor  \empty
  \let\hl@opt@hlight  \empty
  \let\hl@opt@bdash   \empty
  \let\hl@opt@pagefit \empty
  \let\hl@opt@cmd     \empty
  \let\hl@opt@file    \empty
  \let\hl@opt@newwin  \empty
  \def\hl@driver{%
    \ifx\@hltype\hl@raw@word
      \csname \hl@opt@cmd \endcsname
    \else
      \let\hl@BSspec\relax % construct
      \ifx\hl@opt@bstyle \empty
        \ifx\hl@opt@bwidth \empty
          \ifx\hl@opt@bdash \empty
            \let\hl@BSspec\empty % don't construct
          \fi
        \fi
      \fi
      \def\hlstart@preamble{%
        \pdfstartlink
          \ifx\hl@opt@width  \empty \else width  \hl@opt@width  \fi
          \ifx\hl@opt@height \empty \else height \hl@opt@height \fi
          \ifx\hl@opt@depth  \empty \else depth  \hl@opt@depth \fi
          attr{%
            \ifx\hl@opt@bcolor\empty\else /C[\hl@opt@bcolor]\fi
            \ifx\hl@opt@hlight\empty\else /H/\hl@opt@hlight\fi
            \ifx\hl@BSspec\relax
              /BS<<%
                /Type/Border%
                \ifx\hl@opt@bstyle\empty\else /S/\hl@opt@bstyle\fi
                \ifx\hl@opt@bwidth\empty\else /W \hl@opt@bwidth\fi
                \ifx\hl@opt@bdash\empty \else /D[\hl@opt@bdash]\fi
              >>%
            \fi
          }%
      }%
      \csname hl@typeh@\@hltype\endcsname
    \fi
  }%
  \let\hl@typeh@raw\empty
  \def\hl@typeh@name{\hlstart@preamble goto name{\@hllabel}}%
  \def\hl@typeh@num{\hlstart@preamble  goto num \@hllabel}%
  \def\hl@typeh@page{%
    \count@=\@hllabel
    \advance\count@ by-1
    \hlstart@preamble
    user{%
      /Subtype/Link%
      /Dest%
        [\the\count@
          \ifx\hl@opt@pagefit\empty/Fit\else\hl@opt@pagefit\fi]%
    }%
  }%
  \def\hl@typeh@filename{\hl@file{(\@hllabel)}}%
  \def\hl@typeh@filepage{%
    \count@=\@hllabel
    \advance\count@ by-1
    \hl@file{%
      [\the\count@ \ifx\hl@opt@pagefit\empty/Fit\else\hl@opt@pagefit\fi]%
    }%
  }%
  \def\hl@file##1{%
    \hlstart@preamble
    user{%
      /Subtype/Link%
      /A<<%
        /Type/Action%
        /S/GoToR%
        /D##1%
        /F(\hl@opt@file)%
        \ifx\hl@opt@newwin\empty \else
          /NewWindow \ifcase\hl@opt@newwin false\else true\fi
        \fi
      >>%
    }%
  }%
  \def\hl@typeh@url{%
    \hlstart@preamble
    user{%
      /Subtype/Link%
      /A<<%
        /Type/Action%
        /S/URI%
        /URI(\@hllabel)%
      >>%
    }%
  }%
  \def\@hlend{\pdfendlink\endgroup}% End the group from the \@hlstart.
\else % PDF output is not enabled.
  \message{Eplain warning: `pdftex' hyperlink driver: PDF output is^^J
           \space not enabled, falling back on `nolinks' driver.}%
  \hldriver@nolinks
\fi
}%
\def\hldriver@dvipdfm{%
  \def\hldest@type{xyz}%
  \let\hldest@opt@left   \empty
  \let\hldest@opt@top    \empty
  \let\hldest@opt@right  \empty
  \let\hldest@opt@bottom \empty
  \let\hldest@opt@zoom   \empty
  \let\hldest@opt@cmd    \empty
  \def\hldest@driver{%
    \ifx\@hltype\hl@raw@word
      \csname \hldest@opt@cmd \endcsname
    \else
      \def\hldest@preamble{%
        pdf: dest (\@hllabel) [@thispage
      }%
      \csname hldest@typeh@\@hltype\endcsname
    \fi
  }%
  \let\hldest@typeh@raw\empty
  \def\hldest@typeh@fit{%
    \special{\hldest@preamble /Fit]}%
  }%
  \def\hldest@typeh@fith{%
    \special{\hldest@preamble /FitH
      \ifx\hldest@opt@top\empty @ypos \else \hldest@opt@top \fi]}%
  }%
  \def\hldest@typeh@fitv{%
    \special{\hldest@preamble /FitV
      \ifx\hldest@opt@left\empty @xpos \else \hldest@opt@left \fi]}%
  }%
  \def\hldest@typeh@fitb{%
    \special{\hldest@preamble /FitB]}%
  }%
  \def\hldest@typeh@fitbh{%
    \special{\hldest@preamble /FitBH
      \ifx\hldest@opt@top\empty @ypos \else \hldest@opt@top \fi]}%
  }%
  \def\hldest@typeh@fitbv{%
    \special{\hldest@preamble /FitBV
      \ifx\hldest@opt@left\empty @xpos \else \hldest@opt@left \fi]}%
  }%
  \def\hldest@typeh@fitr{%
    \special{\hldest@preamble /FitR
      \ifx\hldest@opt@left\empty @xpos\else\hldest@opt@left\fi\space
      \ifx\hldest@opt@bottom\empty @ypos\else\hldest@opt@bottom\fi\space
      \ifx\hldest@opt@right\empty @xpos\else\hldest@opt@right\fi\space
      \ifx\hldest@opt@top\empty @ypos\else\hldest@opt@top \fi]}%
  }%
  \def\hldest@typeh@xyz{%
    \begingroup
      \ifx\hldest@opt@zoom\empty
        \count1=\z@ \count2=\z@
      \else
        \count2=\hldest@opt@zoom
        \count1=\count2 \divide\count1 by 1000
        \count3=\count1 \multiply\count3 by 1000
        \advance\count2 by -\count3
      \fi
      \special{\hldest@preamble /XYZ
        \ifx\hldest@opt@left\empty @xpos\else\hldest@opt@left\fi\space
        \ifx\hldest@opt@top\empty @ypos\else\hldest@opt@top\fi\space
        \the\count1.\the\count2]}%
    \endgroup
  }%
  \def\hl@type{name}%
  \ifx\hl@type@url\empty
    \def\hl@type@url{url}%
  \fi
  \ifx\hl@type@hrefext\empty
    \def\hl@type@hrefext{url}%
  \fi
  \def\hl@opt@bstyle  {S}%
  \def\hl@opt@bwidth  {1}%
  \let\hl@opt@bcolor  \empty
  \let\hl@opt@hlight  \empty
  \let\hl@opt@bdash   \empty
  \let\hl@opt@pagefit \empty
  \let\hl@opt@cmd     \empty
  \let\hl@opt@file    \empty
  \let\hl@opt@newwin  \empty
  \def\hl@driver{%
    \ifx\@hltype\hl@raw@word
      \csname \hl@opt@cmd \endcsname
    \else
      \let\hl@BSspec\relax % construct
      \ifx\hl@opt@bstyle \empty
        \ifx\hl@opt@bwidth \empty
          \ifx\hl@opt@bdash \empty
            \let\hl@BSspec\empty % don't construct
          \fi
        \fi
      \fi
      \def\hlstart@preamble{%
        pdf: beginann
          <<%
            /Type/Annot%
            /Subtype/Link%
            \ifx\hl@opt@bcolor\empty\else /C[\hl@opt@bcolor]\fi
            \ifx\hl@opt@hlight\empty\else /H/\hl@opt@hlight\fi
            \ifx\hl@BSspec\relax
              /BS<<%
                /Type/Border%
                \ifx\hl@opt@bstyle\empty\else /S/\hl@opt@bstyle\fi
                \ifx\hl@opt@bwidth\empty\else /W \hl@opt@bwidth\fi
                \ifx\hl@opt@bdash\empty \else /D[\hl@opt@bdash]\fi
              >>%
            \fi
      }%
      \csname hl@typeh@\@hltype\endcsname
    \fi
  }%
  \let\hl@typeh@raw\empty
  \def\hl@typeh@name{\special{\hlstart@preamble /Dest(\@hllabel)>>}}%
  \def\hl@typeh@page{%
    \count@=\@hllabel
    \advance\count@ by-1
    \special{%
      \hlstart@preamble
      /Dest[\the\count@
            \ifx\hl@opt@pagefit\empty/Fit\else\hl@opt@pagefit\fi]%
     >>%
    }%
  }%
  \def\hl@typeh@filename{\hl@file{(\@hllabel)}}%
  \def\hl@typeh@filepage{%
    \count@=\@hllabel
    \advance\count@ by-1
    \hl@file{%
      [\the\count@ \ifx\hl@opt@pagefit\empty/Fit\else\hl@opt@pagefit\fi]%
    }%
  }%
  \def\hl@file##1{%
    \special{%
      \hlstart@preamble
      /A<<%
        /Type/Action%
        /S/GoToR%
        /D##1%
        /F(\hl@opt@file)%
        \ifx\hl@opt@newwin\empty \else
          /NewWindow \ifcase\hl@opt@newwin false\else true\fi
        \fi
      >>%
     >>%
    }%
  }%
  \def\hl@typeh@url{%
    \special{%
      \hlstart@preamble
      /A<<%
        /Type/Action%
        /S/URI%
        /URI(\@hllabel)%
      >>%
     >>%
    }%
  }%
  \def\@hlend{\special{pdf: endann}\endgroup}% End the group from \@hlstart.
}%
\def\href{%
  \bgroup
    \uncatcodespecials
    \catcode`\{=1 \catcode`\}=2
    \@href
}%
\def\@href#1{% We'll read #2 (TEXT) later.
  \egroup
  \edef\@hreftmp{\ifempty{#1}{}\fi}% Parameter stuffing for \@@href.
  \expandafter\@@href\@hreftmp#1\@@
}%
\def\href@end@int{\hlend@impl{hrefint}}%
\def\href@end@ext{\hlend@impl{hrefext}}%
\def\@@href#1#2\@@{%
  \def\@hreftmp{#1}%
  \ifx\@hreftmp\hlhash
    \let\href@end\href@end@int
    \hlstart@impl{hrefint}{#2}%
  \else
    \let\href@end\href@end@ext
    \hlstart@impl{hrefext}{#1#2}%
  \fi
  \@@@href
}%
\def\@@@href{%
  \futurelet\@hreftmp\href@
}%
\def\href@{%
  \ifcat\bgroup\noexpand\@hreftmp
    \let\@hreftmp\href@@
  \else
    \let\@hreftmp\href@@@
  \fi
  \@hreftmp
}%
\def\href@@{\bgroup\aftergroup\href@end \let\@hreftmp}%
\def\href@@@#1{#1\href@end}%
\def\hldeston{\errmessage{Please enable hyperlinks with
  \string\enablehyperlinks\space before using hyperlink commands
  (consider selecting the `nolinks' driver to ignore all hyperlink
  commands in your document)}}%
\let\hldestoff\hldeston \let\hlon\hldeston \let\hloff\hldeston
\let\hlstart\hldeston \let\hlend\hldeston \let\hldest\hldeston
\let\hl@setparam\hldeston
\@hloff[*]\@hldestoff[*]%
\newif\ifusepkg@miniltx@loaded
\newcount\usepkg@recursion@level
\def\usepkg@rcrs{\the\usepkg@recursion@level}%
\let\usepkg@at@begin@document\empty
\let\usepkg@at@end@of@package\empty
\def\usepkg@word@autopict{autopict}%
\def\usepkg@word@psfrag{psfrag}%
\long\def\beginpackages#1\endpackages{%
  \let\usepackage\real@usepackage
  \let\DoNotLoadEpstopdf=t
  \let\eplaininput=\input
  #1%
  \usepkg@at@begin@document
  \global\let\usepkg@at@begin@document\empty
  \global\let\usepackage\fake@usepackage
  \let\packageinput=\input
  \let\input=\eplaininput
  \ifx\resetatcatcode\@undefined \else\resetatcatcode \fi
}%
\def\fake@usepackage{\errmessage{You should not use \string\usepackage\space outside of^^J
  \@spaces\string\beginpackages...\string\endpackages\space environment}%
}%
\def\eplain@RequirePackage{%
  \global\ece\let{usepkg@save@pkg\usepkg@rcrs}\usepkg@pkg
  \global\ece\let{usepkg@save@options\usepkg@rcrs}\usepkg@options
  \global\ece\let{usepkg@save@date\usepkg@rcrs}\usepkg@date
  \global\ece\let{usepkg@at@end@of@package\usepkg@rcrs}\usepkg@at@end@of@package
  \global\advance\usepkg@recursion@level by\@ne
  \real@usepackage
}%
\let\usepackage\fake@usepackage
\def\real@usepackage{\@getoptionalarg\@finusepackage}%
\def\@finusepackage#1{%
  \let\usepkg@options\@optionalarg
  \ifempty{#1}%
    \errmessage{No packages specified}%
  \fi
  \ifusepkg@miniltx@loaded \else
    \testfileexistence[miniltx]{tex}%
    \if@fileexists
      \input miniltx.tex
      \global\usepkg@miniltx@loadedtrue
      \global\let\RequirePackage\eplain@RequirePackage
      \global\let\DeclareOption\eplain@DeclareOption
      \global\let\PassOptionsToPackage\eplain@PassOptionsToPackage
      \global\let\ExecuteOptions\eplain@ExecuteOptions
      \gdef\ProcessOptions{\@ifstar\eplain@ProcessOptions
                                   \eplain@ProcessOptions}%
      \global\let\AtBeginDocument\eplain@AtBeginDocument
      \global\let\AtEndOfPackage\eplain@AtEndOfPackage
      \global\let\ProvidesFile\eplain@ProvidesFile
      \global\let\ProvidesPackage\eplain@ProvidesPackage
    \else
      \errmessage{miniltx.tex not found, cannot load LaTeX packages}%
    \fi
  \fi
  \@ifnextchar[%]
    {\@finfinusepackage{#1}}%
    {\@finfinusepackage{#1}[]}%
}%
\def\@finfinusepackage#1[#2]{%
  \edef\usepkg@date{#2}%
  \let\usepkg@load@list\empty
  \for\usepkg@pkg:=#1\do{%
    \toks@=\expandafter{\usepkg@load@list}%
    \edef\usepkg@load@list{%
      \the\toks@
      \noexpand\usepkg@load@pkg{\usepkg@pkg}%
    }%
  }%
  \usepkg@load@list
  \ifnum\usepkg@recursion@level>0
    \global\advance\usepkg@recursion@level by\m@ne
    \expandafter\let\expandafter\usepkg@pkg\csname usepkg@save@pkg\usepkg@rcrs\endcsname
    \expandafter\let\expandafter\usepkg@options\csname usepkg@save@options\usepkg@rcrs\endcsname
    \expandafter\let\expandafter\usepkg@date\csname usepkg@save@date\usepkg@rcrs\endcsname
    \expandafter\let\expandafter\usepkg@at@end@of@package\csname usepkg@at@end@of@package\usepkg@rcrs\endcsname
    \global\ece\let{usepkg@save@pkg\usepkg@rcrs}\undefined
    \global\ece\let{usepkg@save@options\usepkg@rcrs}\undefined
    \global\ece\let{usepkg@save@date\usepkg@rcrs}\undefined
    \global\ece\let{usepkg@at@end@of@package\usepkg@rcrs}\undefined
  \fi
}%
\def\usepkg@load@pkg#1{%
  \def\usepkg@pkg{#1}%
  \ifx\usepkg@pkg\usepkg@word@autopict
    \testfileexistence[picture]{tex}%
    \if@fileexists \else
      \errmessage{Loader `picture.tex' for package `\usepkg@pkg' not found}%
    \fi
  \else
    \ifx\usepkg@pkg\usepkg@word@psfrag
      \testfileexistence[psfrag]{tex}%
      \if@fileexists \else
        \errmessage{Loader `psfrag.tex' for package `\usepkg@pkg' not found}%
      \fi
    \fi
  \fi
  \ifundefined{ver@\usepkg@pkg.sty}%
    \expandafter\@finusepkg@load@pkg
  \else
    \immediate\write-1{^^J\linenumber Eplain: package `\usepkg@pkg' already
             loaded, skipping reloading}%
  \fi
}%
\def\@finusepkg@load@pkg{%
  \testfileexistence[\usepkg@pkg]{sty}%
  \if@fileexists \else
    \errmessage{Package `\usepkg@pkg' not found}%
  \fi
  \expandafter\let\expandafter\temp\csname usepkg@options@\usepkg@pkg\endcsname
  \ifx\temp\relax
    \let\temp\empty
  \fi
  \ifx\temp\empty
    \let\temp\usepkg@options
  \else
    \ifx\usepkg@options\empty \else
      \edef\temp{\temp,\usepkg@options}%
    \fi
  \fi
  \global\ece\let{usepkg@options@\usepkg@pkg}\temp
  \let\usepackage\eplain@RequirePackage
  \global\let\usepkg@at@end@of@package\empty
  \ifx\usepkg@pkg\usepkg@word@autopict
    \input picture.tex
  \else
    \ifx\usepkg@pkg\usepkg@word@psfrag
      \input \usepkg@pkg.tex
    \else
      \input \usepkg@pkg.sty
    \fi
  \fi
  \usepkg@at@end@of@package
  \global\let\usepkg@at@end@of@package\empty
  \let\usepackage\real@usepackage
  \global\ece\let{usepkg@options@\usepkg@pkg}\undefined
  \def\Url@HyperHook##1{\hlstart@impl{url}{\Url@String}##1\hlend@impl{url}}%
}%
\def\eplain@DeclareOption#1#2{%
  \toks@{#2}%
  \expandafter\xdef\csname usepkg@option@\usepkg@pkg @#1\endcsname{\the\toks@}%
}%
\def\eplain@PassOptionsToPackage#1#2{%
  \ifempty{#1}\else
    \for\usepkg@temp:=#2\do{%
      \expandafter\let\expandafter\temp\csname usepkg@options@\usepkg@temp\endcsname
      \ifx\temp\relax
        \let\temp\empty
      \fi
      \ifx\temp\empty
        \edef\temp{#1}%
      \else
        \edef\temp{\temp,#1}%
      \fi
      \global\ece\let{usepkg@options@\usepkg@temp}\temp
    }%
  \fi
}%
\def\usepkg@exec@options#1{%
  \for\CurrentOption:=#1\do{%
    \expandafter\let\expandafter\usepkg@temp
      \csname usepkg@option@\usepkg@pkg @\CurrentOption\endcsname
    \ifx\usepkg@temp\relax
      \expandafter\let\expandafter\temp\csname usepkg@option@\usepkg@pkg @*\endcsname
      \ifx\temp\relax
        \errmessage{Unknown option `\CurrentOption' to package `\usepkg@pkg'}%
      \else
        \temp
      \fi
    \else
      \usepkg@temp
    \fi
  }%
}%
\let\eplain@ExecuteOptions\usepkg@exec@options
\def\eplain@ProcessOptions{%
  \expandafter\usepkg@exec@options\csname usepkg@options@\usepkg@pkg\endcsname
}%
\def\usepkg@accumulate@cmds#1#2{%
  \toks@=\expandafter{#1}%
  \toks@ii={#2}%
  \xdef#1{\the\toks@\the\toks@ii}%
}%
\def\eplain@AtBeginDocument{\usepkg@accumulate@cmds\usepkg@at@begin@document}%
\def\eplain@AtEndOfPackage{\usepkg@accumulate@cmds\usepkg@at@end@of@package}%
\def\eplain@ProvidesPackage#1{%
  \@ifnextchar[%]
    {\eplain@pr@videpackage{#1.sty}}{\eplain@pr@videpackage#1[]}%
}%
\def\eplain@pr@videpackage#1[#2]{%
  \wlog{#1: #2}%
  \expandafter\xdef\csname ver@#1\endcsname{#2}%
  \@ifl@t@r{#2}\usepkg@date{}%
    {\message{Warning: you have requested package `\usepkg@pkg', version \usepkg@date,^^J
       \@spaces but only version `\csname ver@#1\endcsname' is available.}}%
}%
\def\eplain@ProvidesFile#1{%
  \@ifnextchar[%]
    {\eplain@pr@videfile{#1}}{\eplain@pr@videfile#1[]}%
}%
\def\eplain@pr@videfile#1[#2]{%
  \wlog{#1: #2}%
  \expandafter\xdef\csname ver@#1\endcsname{#2}%
}%
\def\@ifl@ter#1#2{%
  \expandafter\@ifl@t@r
    \csname ver@#2.#1\endcsname
}%
\def\@ifl@t@r#1#2{%
  \ifnum\expandafter\@parse@version#1//00\@nil<%
        \expandafter\@parse@version#2//00\@nil
    \expandafter\@secondoftwo
  \else
    \expandafter\@firstoftwo
  \fi
}%
\def\@parse@version#1/#2/#3#4#5\@nil{#1#2#3#4 }%
\let\ttfamily\tt
\def\strip@prefix#1>{}%
\def\@ifpackageloaded#1{%
  \expandafter\ifx\csname ver@#1.sty\endcsname\relax
    \expandafter\@secondoftwo
  \else
    \expandafter\@firstoftwo
  \fi
}%
\long\def\g@addto@macro#1#2{%
  \begingroup
    \toks@\expandafter{#1#2}%
    \xdef#1{\the\toks@}%
  \endgroup
}%
\def\PackageWarning#1#2{%
  \begingroup
    \newlinechar=10 %
    \def\MessageBreak{%
      ^^J(#1)\@spaces\@spaces\@spaces\@spaces
    }%
    \immediate\write16{^^JPackage #1 Warning: #2\on@line.^^J}%
  \endgroup
}%
\def\PackageWarningNoLine#1#2{%
  \PackageWarning{#1}{#2\@gobble}%
}%
\def\on@line{ on input line \the\inputlineno}%
\def\@spaces{\space\space\space\space}%
\def\@inmatherr#1{%
   \relax
   \ifmmode
     \errmessage{The command is invalid in math mode}%
   \fi
}%
\let\protected@edef\edef
\let\wlog = \@plainwlog
\catcode`@ = \@eplainoldatcode
\def\fmtname{eplain}%
\def\eplain{t}%
{\edef\plainversion{\fmtversion}%
 \xdef\fmtversion{3.6: 30 September 2013 (and plain \plainversion)}%
}%
