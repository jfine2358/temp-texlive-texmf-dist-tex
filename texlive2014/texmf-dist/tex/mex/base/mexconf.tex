% This is MEXCONF.TEX - a part of the Polish plain TeX and LaTeX formats.
% Version 1.05,  18 December 1993
%
% Authors: Marek Ry\'cko & Bogus\l{}aw Jackowski
%
% This macro file belongs to the public domain
% under the conditions specified by the author of TeX:
%
%   ``Macro files like PLAIN.TEX should not be changed in any way,
%     except with respect to preloaded fonts,
%     unless the changes are authorized by the authors of the macros.''
%
%                                           Donald E. Knuth
%
% You are entitled to modify this file obeying the rules specified below.
%
% For details see MEXINFO.ENG or MEXINFO.POL.

\message{MeX configuration,}

% The following lines define which hyphenation patterns
% are to be included in the current configuration of the MeX format.
% We assume that fonts with different positions of Polish diacritical
% characters can be used within the same document and each font layout
% requires its own hyphenation patterns.
% The standard positions of Polish diacritical characters in TeX's fonts
% is the layout accepted at the 5th European TeX Conference in Cork (1990);
% this layout is called here `PL'.
%
% The standard configuration of the MeX format includes the hyphenation
% patterns for:
%   - the English language (OK for all fonts),
%   - the Polish language used with the fonts in the `PL' layout.
% In such configuration switching to another font layout disables
% hyphenation for the Polish language.
%
% However, the hyphenation patterns for other layouts of fonts including
% Polish diacritical letters are prepared and if memory space in the
% particular instalation of TeX permits the user is free to use INITEX
% to generate a new MeX or LaMeX format containing these patterns.
% Just `unpercent' some of the following lines.

\hyph@nate{english}
\hyph@nate{pl}
%\hyph@nate{pone}
%\hyph@nate{mazovia}
%\hyph@nate{latintwo}

% ``deafault'' means -- at the beginning
% (unpercent only one possibility)
\def\d@faultlayout
    {pl}
    %{mazovia}
    %{latintwo}
    %{pone}
\def\d@faultlanguage
    {\polish}
    %{\english}
\def\d@faultprefixing
    {\nonprefixing}
    %{\prefixing}
\def\d@faultspacing
    {\frenchspacing}
    %{\nonfrenchspacing}

% in PLAIN.TeX and LFONTS.TeX files the prefix:
\def\@ldpref % exactly two characters
    {cm}
% is changed to:
\def\n@wpref % an arbitrary number of characters
    {pl}

\def\pl@in % the name of the file containing the original plain TeX format
    % (usually: `plain.tex')
    {plain.tex}
\def\lpl@in % the name of the file containing the original lplain file
    % (usually: `lplain.tex')
    {lplain.tex}
\def\lf@nts % the name of the file containing the original lfonts file
    % (usually: `lfonts.tex')
    {lfonts.tex}
\def\l@tex % the name of the file containing the original latex file
    % (usually: `latex.tex')
    {latex.tex}
\def\P@lhyphen % name of the file containing Polish hyphenation patterns
    % the patterns must be in the prefix notation
%    {plhyph.tex}
%% the above assumption is not necessary, we start to use unicode patterns (StaW, 30.06.2008).
    {loadhyph-pl.tex}
\def\@nghyphen % name of the file containing the US-English
    % hyphenation patterns (usually `hyphen.tex')
    {hyphen.tex}
\endinput
