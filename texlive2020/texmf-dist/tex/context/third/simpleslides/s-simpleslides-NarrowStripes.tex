%D \module
%D   [      file=simpleslides-s-NarrowStripes,
%D        version=2009.03.30
%D          title=\CONTEXT\ Style File,
%D       subtitle=Presentation Module NarrowStripes,
%D         author=Aditya Mahajan and Thomas A. Schmitz,
%D           date=\currentdate,
%D      copyright={Aditya Mahajan and Thomas A. Schmitz}]
%C
%C Copyright 2007 Aditya Mahajan and Thomas A. Schmitz
%C This file may be distributed under the GNU General Public License v. 2.0.

%D This file provides the \quotation{NarrowStripes} style for the presentation
%D module. It is loaded at runtime. The theme for this style is inspired by the
%D \quotation{Berkeley} theme of the \LaTeX\ Beamer package.

\writestatus{simpleslides}{loading NarrowStripes style}

\startmodule[simpleslides-s-NarrowStripes]

\unprotect

%D First, we change the page layout.

\setuplayout [width=fit,
              leftmargin=1.5cm,
	      rightmargin=0cm,
              leftmargindistance=1.8cm,
              rightmargindistance=0pt,
              height=fit, 
              header=2.5cm, 
              footer=0cm, 
              topspace=.4cm, 
              backspace=3.2cm,
	      cutspace=3.7cm,
              bottomspace=0cm,
              bottom=0pt,
              location=singlesided]

%D We also specify the position of the slidetitle.

\setuplayer[simpleslides:layer:slidetitle]
    [width=\paperwidth,
    height=\paperheight,
    x=32mm]

%D These macros are used for placing figures/pictures:

\define\NormalHeight{\textheight}
\define\NormalWidth{.476\textwidth}
\define\PictureFrameHeight{\textheight}
\define\PictureFrameWidth{.476\textwidth}

%D This module has three color schemes, blue, green and red.

\startsetups simpleslides:setups:blue
\definecolor [simpleslides:backgroundcolor]             [s=.95]
\definecolor [simpleslides:altcontrastcolor]            [r=0,g=0,b=1]
\definecolor [simpleslides:variantcolor]                [r=.69,g=.69,b=.97]
\definecolor [simpleslides:contrastcolor]               [b=.8]
\definecolor [simpleslides:itemize:color]               [b=.8]
\stopsetups

\startsetups simpleslides:setups:red
\definecolor [simpleslides:itemize:color]               [r=.8]
\definecolor [simpleslides:backgroundcolor]             [s=.95]
\definecolor [simpleslides:altcontrastcolor]            [r=1]
\definecolor [simpleslides:variantcolor]                [b=.69,g=.69,r=.97]
\definecolor [simpleslides:contrastcolor]               [r=.8]
\stopsetups

\startsetups simpleslides:setups:green
\definecolor [simpleslides:itemize:color]               [g=.4]
\definecolor [simpleslides:backgroundcolor]             [s=.95]
\definecolor [simpleslides:altcontrastcolor]            [g=.4]
\definecolor [simpleslides:variantcolor]                [b=.68,r=.68,g=.79]
\definecolor [simpleslides:contrastcolor]               [g=.4]
\stopsetups

%D Now we choose the scheme that the user asked for

\doifsetupselse{simpleslides:setups:\moduleparameter{simpleslides}{color}}
    {\setups{simpleslides:setups:\moduleparameter{simpleslides}{color}}}
    {\setups{simpleslides:setups:blue}}

%D We let Metapost calculate the background:

\definetextext[simpleslides:sometxt:text] {\TaspresentSometxtText}

\unexpanded\def\TaspresentSometxtText#1%
  {\framed[\c!frame=\v!off, \c!width=2.25cm, \c!height=2.25cm]
   {\switchtobodyfont[12pt]\color[simpleslides:contrastcolor]{#1}}}

\startuseMPgraphic{simpleslides:MP:ornament}
StartPage ;

save a,b,c ; numeric a,b,c ;
a = 2.25cm ;
b = 0.4 cm ;
c = PaperHeight - a/2 ;

z1 = ulcorner Page shifted (0,-a) ;
z2 = ulcorner Page shifted (0,-a-b) ;
z3 = urcorner Page shifted (0,-a-b) ;
z4 = urcorner Page shifted (0,-a) ;
z5 = ulcorner Page shifted (a,0) ;
z6 = ulcorner Page shifted (a+b,0) ;
z7 = llcorner Page shifted (a+b,0) ;
z8 = llcorner Page shifted (a,0) ;
z9 = ulcorner Page shifted (a,-a) ;
z10 = ulcorner Page shifted (a+b,-a) ;
z11 = ulcorner Page shifted (a+b,-a-b) ;
z12 = ulcorner Page shifted (a,-a-b) ;

save p; path p[] ;
p[1] = z1 -- z2 -- z3 -- z4 -- cycle ;
p[2] = z5 -- z6 -- z7 -- z8 -- cycle ;
p[3] = z9 -- z10 --z11 -- z12 -- cycle ;

linear_shade(p[1],1,
             \MPcolor{simpleslides:backgroundcolor},
             \MPcolor{simpleslides:altcontrastcolor}) ;

linear_shade(p[2],2,
             \MPcolor{simpleslides:altcontrastcolor},
             \MPcolor{simpleslides:backgroundcolor}) ;

fill p[3] withcolor \MPcolor{simpleslides:variantcolor} ;

if PageNumber > 1:
	draw \sometxt[simpleslides:sometxt:text]{\folio} shifted (0,PaperHeight-a) ;
fi ;
StopPage ;
\stopuseMPgraphic

\defineoverlay
  [simpleslides:background:ornament]
  [\useMPgraphic{simpleslides:MP:ornament}]

\defineoverlay 
  [simpleslides:background:title] 
  [\useMPgraphic{simpleslides:MP:ornament}] 

%D this sets up the title page:

\setupTitle
  [\c!title\c!color={simpleslides:contrastcolor},
   \c!author\c!color={simpleslides:contrastcolor},
   \c!date\c!color={simpleslides:contrastcolor}]

%D The slide title is typeset in a layer

\setupSlideTitle
  [\c!color={simpleslides:contrastcolor},
   \c!alternative=layer,
   \c!align=\v!center,
   \c!width=\textwidth,
   \c!height=2.25cm,
   \c!after=]

%D The symbol for the first level of itemizations. 

\definesymbol[1][\useMPgraphic{simpleslides:itemize:square}]
\setupitemize[1][inmargin][color={simpleslides:itemize:color}]

\protect
\stopmodule

\endinput

