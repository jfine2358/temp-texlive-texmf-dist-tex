%%
%% This is file `smartdiagramlibrarycore.commands.code.tex',
%% generated with the docstrip utility.
%%
%% The original source files were:
%%
%% smartdiagram.dtx  (with options: `commands')
%% * * * * * * * * * * * * * * * * * * * * * * * * * * * * * * * * * * * * * * *
%% smartdiagram --- Automatic creation of smart diagrams from lists of items.
%% 
%% E-mail: claudio <dot> fiandrino <at> gmail <dot> com
%% 
%% Released under the LaTeX Project Public License v1.3c or later
%% See http://www.latex-project.org/lppl.txt
%% * * * * * * * * * * * * * * * * * * * * * * * * * * * * * * * * * * * * * * *
%% 

\NewDocumentCommand{\smartdiagram}{r[] m}{%
   \StrCut{#1}{:}\diagramtype\option
   \IfNoValueTF{#1}{% true-no value 1
      \PackageError{smartdiagram}%
      {Type of the diagram not inserted. Please insert it}%
      {Example: \protect\smartdiagram[flow diagram]}}
   {%false-no value 1
   \IfStrEq{\diagramtype}{}{%
      \PackageError{smartdiagram}{Type of the diagram not inserted. Please insert it}
      {Example: \protect\smartdiagram[flow diagram]}
   }{}
   \IfStrEq{\diagramtype}{circular diagram}{% true-circular diagram
   \begin{tikzpicture}[every node/.style={align=center,let hypenation}]

   \foreach \smitem [count=\xi] in {#2}  {\global\let\maxsmitem\xi}

   \foreach \smitem [count=\xi] in {#2}{%
   \IfStrEq{\option}{clockwise}{% true-clockwise-circular diagram
     \pgfmathtruncatemacro{\angle}{180+360/\maxsmitem*\xi}
   }{% false-clockwise-circular diagram
     \pgfmathtruncatemacro{\angle}{360/\maxsmitem*\xi}
   }
   \edef\col{\@nameuse{color@\xi}}
   \IfStrEq{\option}{clockwise}{% true-clockwise-circular diagram
     \node[module,drop shadow] (module\xi)
      at (-\angle:\sm@core@circulardistance) {\smitem };
   }{% false-clockwise-circular diagram
     \node[module,drop shadow] (module\xi)
      at (\angle:\sm@core@circulardistance) {\smitem };
   }
   }%
   \foreach \smitem [count=\xi] in {#2}{%
   \ifnum\xi=\maxsmitem
     \ifcircularfinalarrowdisabled
       \relax
     \else
       \pgfmathtruncatemacro{\xj}{mod(\xi, \maxsmitem) + 1)}
       \edef\col{\@nameuse{color@\xj}}
       \IfStrEq{\option}{clockwise}{% true-clockwise-circular diagram
         \draw[diagram arrow type,shorten <=0.3cm,shorten >=0.3cm]
          (module\xj) to[bend right] (module\xi);
         }{% false-clockwise-circular diagram
          \draw[diagram arrow type,shorten <=0.3cm,shorten >=0.3cm]
           (module\xj) to[bend left] (module\xi);
         }
     \fi
   \else
     \pgfmathtruncatemacro{\xj}{mod(\xi, \maxsmitem) + 1)}
     \edef\col{\@nameuse{color@\xj}}
     \IfStrEq{\option}{clockwise}{% true-clockwise-circular diagram
       \draw[diagram arrow type,shorten <=0.3cm,shorten >=0.3cm]
         (module\xj) to[bend right] (module\xi);
     }{% false-clockwise-circular diagram
        \draw[diagram arrow type,shorten <=0.3cm,shorten >=0.3cm]
         (module\xj) to[bend left] (module\xi);
       }

   \fi
   }%
   \end{tikzpicture}
   }{}% end-circular diagram
   \IfStrEq{\diagramtype}{flow diagram}{% true-flow diagram
   \begin{tikzpicture}[every node/.style={align=center,let hypenation}]

   \foreach \smitem [count=\xi] in {#2}  {\global\let\maxsmitem\xi}

   \foreach \smitem [count=\xi] in {#2}{%
   \edef\col{\@nameuse{color@\xi}}
   \IfStrEq{\option}{horizontal}{% true-horizontal-flow diagram
     \path let \n1 = {int(0-\xi)}, \n2={0+\xi*\sm@core@modulexsep} in
         node[module,drop shadow] (module\xi) at +(\n2,0) {\smitem};
   }{% false-horizontal-flow diagram
     \path let \n1 = {int(0-\xi)}, \n2={0-\xi*\sm@core@moduleysep} in
         node[module,drop shadow] (module\xi) at +(0,\n2) {\smitem};
   }
   }%

   \foreach \smitem [count=\xi] in {#2}{%
   \pgfmathtruncatemacro{\xj}{mod(\xi, \maxsmitem) + 1)}
   \edef\col{\@nameuse{color@\xj}}
   \ifnum\xi<\maxsmitem
   \begin{pgfonlayer}{smart diagram arrow back}
   \draw[diagram arrow type] (module\xj) -- (module\xi);
   \end{pgfonlayer}
   \fi
   % last arrow - not display it in background - check if disabled
   \ifbackarrowdisabled
     \relax
   \else
     \ifnum\xi=\maxsmitem
       \IfStrEq{\option}{horizontal}{% true-horizontal-flow diagram
         \tikzset{square arrow/.style={%
            to path={-- ++(0,\sm@core@backarrowdistance) -| (\tikztotarget)}
            }%
         }%
         \draw[diagram arrow type, square arrow]
          (module\xj.north) to (module\xi.north);
       }{% false-horizontal-flow diagram
         \tikzset{square arrow/.style={%
            to path={-- ++(\sm@core@backarrowdistance,0) |- (\tikztotarget)}
            }%
         }%
         \draw[diagram arrow type,square arrow]
           (module\xj.east) to (module\xi);
       }%
     \fi
   \fi
   }%
   \end{tikzpicture}
   }{}% end-flow diagram
   \IfStrEq{\diagramtype}{descriptive diagram}{% true-descr. diagram
   \begin{tikzpicture}[every node/.style={align=center,let hypenation}]
   \foreach \smitem [count=\xi] in {#2}{%
   \edef\col{\@nameuse{color@\xi}}
   \foreach \subitem [count=\xii] in \smitem{%
   \ifnumequal{\xii}{1}{% true
   \node[description title,drop shadow]
    (module-title\xi) at (0,0-\xi*\sm@core@descriptiveitemsysep) {\subitem};
   }{}
   \ifnumequal{\xii}{2}{% true
   \node[description,drop shadow](module\xi)
    at (0,0-\xi*\sm@core@descriptiveitemsysep) {\subitem};
   }{}
   }%
   }%
   \end{tikzpicture}
   }{}% end-descr. diagram
   \IfStrEq{\diagramtype}{bubble diagram}{% true-bubble diagram
   \begin{tikzpicture}[every node/.style={align=center,let hypenation}]
   \foreach \smitem [count=\xi] in {#2}{\global\let\maxsmitem\xi}
   \pgfmathtruncatemacro\actualnumitem{\maxsmitem-1}
   \foreach \smitem [count=\xi] in {#2}{%
   \ifnumequal{\xi}{1}{ %true
   \node[bubble center node](center bubble){\smitem};
   }{%false
   \pgfmathtruncatemacro{\xj}{\xi-1}
   \pgfmathtruncatemacro{\angle}{360/\actualnumitem*\xj}
   \edef\col{\@nameuse{color@\xj}}
   \node[bubble node] (module\xi)
       at (center bubble.\angle) {\smitem };
   }%
   }%
   \end{tikzpicture}
   }{}%end-bubble diagram
   \IfStrEq{\diagramtype}{constellation diagram}{% true-const diagram
   \begin{tikzpicture}[every node/.style={align=center,let hypenation}]
   \foreach \smitem [count=\xi] in {#2}{\global\let\maxsmitem\xi}
   \pgfmathtruncatemacro\actualnumitem{\maxsmitem-1}
   \foreach \smitem [count=\xi] in {#2}{%
   \ifnumequal{\xi}{1}{ %true
   \node[planet](planet){\smitem};
   }{%false
   \pgfmathtruncatemacro{\xj}{\xi-1}
   \pgfmathtruncatemacro{\angle}{360/\actualnumitem*\xj}
   \edef\col{\@nameuse{color@\xj}}
   \node[satellite] (satellite\xi)
       at (\angle:\sm@core@distanceplanetsatellite) {\smitem };
   \draw[connection planet satellite] (planet) -- (satellite\xi);
   }%
   }%
   \end{tikzpicture}
   }{}%end-const diagram
   \IfStrEq{\diagramtype}{connected constellation diagram}{% true-conn const diagram
   \begin{tikzpicture}[every node/.style={align=center,let hypenation}]
   \foreach \smitem [count=\xi] in {#2}{\global\let\maxsmitem\xi}
   \pgfmathtruncatemacro\actualnumitem{\maxsmitem-1}
   \foreach \smitem [count=\xi] in {#2}{%
   \ifnumequal{\xi}{1}{ %true
   \node[planet](planet){\smitem};
   }{%false
   \pgfmathtruncatemacro{\xj}{\xi-1}
   \pgfmathtruncatemacro{\angle}{360/\actualnumitem*\xj}
   \edef\col{\@nameuse{color@\xj}}
   \node[satellite] (satellite\xj)
        at (\angle:\sm@core@distanceplanetsatellite) {\smitem };
   }%
   }%
   \foreach \smitem [count=\xi] in {#2}{%
      \ifnumgreater{\xi}{1}{ %true
      \pgfmathtruncatemacro{\xj}{\xi-1}
      \edef\col{\@nameuse{color@\xj}}
      \pgfmathtruncatemacro{\xk}{mod(\xj,\actualnumitem) +1}
      \path[connection planet satellite,-]
         (satellite\xj) edge[bend right] (satellite\xk);
   }{}
   }%
   \end{tikzpicture}
   }{}%end-connected constellation diagram
   \IfStrEq{\diagramtype}{priority descriptive diagram}{% true-priority descriptive diagram
   \pgfmathparse{subtract(\sm@core@priorityarrowwidth,\sm@core@priorityarrowheadextend)}
   \pgfmathsetmacro\sm@core@priorityticksize{\pgfmathresult/2}
   \pgfmathsetmacro\arrowtickxshift{(\sm@core@priorityarrowwidth-\sm@core@priorityticksize)/2}
   \begin{tikzpicture}[every node/.style={align=center,let hypenation}]
   \foreach \smitem [count=\xi] in {#2}{\global\let\maxsmitem\xi}
   \foreach \smitem [count=\xi] in {#2}{%
   \edef\col{\@nameuse{color@\xi}}
    \node[description,drop shadow](module\xi)
    at (0,0+\xi*\sm@core@descriptiveitemsysep) {\smitem};
\draw[line width=\sm@core@prioritytick,\col]
      ([xshift=-\arrowtickxshift pt]module\xi.base west)--
 ($([xshift=-\arrowtickxshift pt]module\xi.base west)-(\sm@core@priorityticksize pt,0)$);
   }%
   \coordinate (A) at (module1);
   \coordinate (B) at (module\maxsmitem);
   \CalcHeight(A,B){heightmodules}
   \pgfmathadd{\heightmodules}{\sm@core@priorityarrowheightadvance}
   \pgfmathsetmacro{\distancemodules}{\pgfmathresult}
   \pgfmathsetmacro\arrowxshift{\sm@core@priorityarrowwidth/2}
   \begin{pgfonlayer}{background}
   \node[priority arrow] at ([xshift=-\arrowxshift pt]module1.south west){};
   \end{pgfonlayer}
   \end{tikzpicture}
   }{}% end-priority descriptive diagram
   \IfStrEq{\diagramtype}{sequence diagram}{% true-sequence diagram
   \begin{tikzpicture}[every node/.style={align=center,let hypenation}]
   \foreach \x[count=\xi, count=\prevx from 0] in {#2}{%
   \edef\col{\@nameuse{color@\xi}}
   \ifnum\xi=1
     \node[sequence item] (sequence-item\xi) {\x};
   \else
     \node[sequence item,anchor=west] (sequence-item\xi) at (sequence-item\prevx.east) {\x};
   \fi
   }
   \end{tikzpicture}
   }{}% end-sequence diagram
}% end-no value 1
}% end-command
\NewDocumentCommand{\smartdiagramanimated}{r[] m}{%
   \StrCut{#1}{:}\diagramtype\option
   \IfNoValueTF{#1}{% true-no value 1
      \PackageError{smartdiagram}{Type of the diagram not inserted. Please insert it}
      {Example: \protect\smartdiagram[flow diagram]}}
   {%false-no value 1
   \IfStrEq{\diagramtype}{}{%
      \PackageError{smartdiagram}{Type of the diagram not inserted. Please insert it}
      {Example: \protect\smartdiagram[flow diagram]}
   }{}
   \IfStrEq{\diagramtype}{circular diagram}{% true-circular diagram
   \begin{tikzpicture}[every node/.style={align=center,let hypenation}]
   \foreach \smitem [count=\xi] in {#2}  {\global\let\maxsmitem\xi}
   \foreach \smitem [count=\xi] in {#2}{%
   \IfStrEq{\option}{clockwise}{% true-clockwise-circular diagram
     \pgfmathtruncatemacro{\angle}{180+360/\maxsmitem*\xi}
   }{% false-clockwise-circular diagram
     \pgfmathtruncatemacro{\angle}{360/\maxsmitem*\xi}
   }
   \edef\col{\@nameuse{color@\xi}}
   \IfStrEq{\option}{clockwise}{% true-clockwise-circular diagram
      \node[module,
        drop shadow={smvisible on=<\xi->},
        smvisible on=<\xi->] (module\xi)
       at (-\angle:\sm@core@circulardistance) {\smitem};
   }{% false-clockwise-circular diagram
      \node[module,
        drop shadow={smvisible on=<\xi->},
        smvisible on=<\xi->] (module\xi)
       at (\angle:\sm@core@circulardistance) {\smitem};
   }
   }%
   \foreach \smitem [count=\xi] in {#2}{%
   \ifnum\xi=\maxsmitem
     \ifcircularfinalarrowdisabled
       \relax
     \else
       \pgfmathtruncatemacro{\xj}{mod(\xi, \maxsmitem) + 1)}
       \pgfmathtruncatemacro{\adv}{\xi + 1)}
       \edef\col{\@nameuse{color@\xj}}
       \IfStrEq{\option}{clockwise}{% true-clockwise-circular diagram
         \draw[diagram arrow type,shorten <=0.3cm,shorten >=0.3cm,
        smvisible on=<\adv->]
          (module\xj) to[bend right] (module\xi);
         }{% false-clockwise-circular diagram
          \draw[diagram arrow type,shorten <=0.3cm,shorten >=0.3cm,
        smvisible on=<\adv->]
           (module\xj) to[bend left] (module\xi);
         }
     \fi
   \else
     \pgfmathtruncatemacro{\xj}{mod(\xi, \maxsmitem) + 1)}
     \pgfmathtruncatemacro{\adv}{\xi + 1)}
     \edef\col{\@nameuse{color@\xj}}
     \IfStrEq{\option}{clockwise}{% true-clockwise-circular diagram
       \draw[diagram arrow type,shorten <=0.3cm,shorten >=0.3cm,
        smvisible on=<\adv->]
         (module\xj) to[bend right] (module\xi);
     }{% false-clockwise-circular diagram
        \draw[diagram arrow type,shorten <=0.3cm,shorten >=0.3cm,
          smvisible on=<\adv->]
         (module\xj) to[bend left] (module\xi);
       }

   \fi
   }%
   \end{tikzpicture}
   }{}% end-circular diagram
   \IfStrEq{\diagramtype}{flow diagram}{% true-flow diagram
   \begin{tikzpicture}[every node/.style={align=center,let hypenation}]

   \foreach \smitem [count=\xi] in {#2}  {\global\let\maxsmitem\xi}

   \foreach \smitem [count=\xi] in {#2}{%
   \edef\col{\@nameuse{color@\xi}}
   \IfStrEq{\option}{horizontal}{% true-horizontal-flow diagram
     \path let \n1 = {int(0-\xi)}, \n2={0+\xi*\sm@core@modulexsep}
        in node[module,drop shadow={smvisible on=<\xi->},
        smvisible on=<\xi->] (module\xi) at +(\n2,0) {\smitem};
   }{% false-horizontal-flow diagram
     \path let \n1 = {int(0-\xi)}, \n2={0-\xi*\sm@core@moduleysep}
        in node[module,drop shadow={smvisible on=<\xi->},
        smvisible on=<\xi->] (module\xi) at +(0,\n2) {\smitem};
   }
   }%

   \foreach \smitem [count=\xi] in {#2}{%
   \pgfmathtruncatemacro{\xj}{mod(\xi, \maxsmitem) + 1)}
   \edef\col{\@nameuse{color@\xj}}
   \ifnum\xi<\maxsmitem
   \begin{pgfonlayer}{smart diagram arrow back}
   \draw[diagram arrow type,smvisible on=<\xi->]
     (module\xj) -- (module\xi);
   \end{pgfonlayer}
   \fi
   % last arrow - not display it in background - check if disabled
   \ifbackarrowdisabled
     \relax
   \else
     \ifnum\xi=\maxsmitem
       \IfStrEq{\option}{horizontal}{% true-horizontal-flow diagram
         \tikzset{square arrow/.style={
           to path={-- ++(0,\sm@core@backarrowdistance) -| (\tikztotarget)}
           }
         }
         \draw[diagram arrow type, square arrow,smvisible on=<\xi->]
          (module\xj.north) to (module\xi.north);
       }{% false-horizontal-flow diagram
         \tikzset{square arrow/.style={
           to path={-- ++(\sm@core@backarrowdistance,0) |- (\tikztotarget)}
           }
         }
         \draw[diagram arrow type,square arrow,smvisible on=<\xi->]
           (module\xj.east) to (module\xi);
       }
     \fi
   \fi
   }%
   \end{tikzpicture}
   }{}% end-flow diagram
   \IfStrEq{\diagramtype}{descriptive diagram}{% true-descriptive diagram
   \begin{tikzpicture}[every node/.style={align=center,let hypenation}]
   \foreach \smitem [count=\xi] in {#2}{%
   \edef\col{\@nameuse{color@\xi}}

   \foreach \subitem [count=\xii] in \smitem{%
      \pgfmathtruncatemacro\subitemvisible{\xi}
   \ifnumequal{\xii}{1}{% true
   \node[description title,drop shadow, smvisible on=<\subitemvisible->]
   (module-title\xi) at (0,0-\xi*\sm@core@descriptiveitemsysep) {\subitem};\pause
   }{}
   \ifnumequal{\xii}{2}{% true
   \node[description,drop shadow,smvisible on=<\subitemvisible->]
   (module\xi)at (0,0-\xi*\sm@core@descriptiveitemsysep) {\subitem};\pause
   }{}
   }%
   }%
   \end{tikzpicture}
   }{}% end-descriptive diagram
   \IfStrEq{\diagramtype}{bubble diagram}{% true-bubble diagram
   \begin{tikzpicture}[every node/.style={align=center,let hypenation}]
   \foreach \smitem [count=\xi] in {#2}{\global\let\maxsmitem\xi}
   \pgfmathtruncatemacro\actualnumitem{\maxsmitem-1}
   \foreach \smitem [count=\xi] in {#2}{%
   \ifnumequal{\xi}{1}{ %true
   \node[bubble center node, smvisible on=<\xi->](center bubble){\smitem};
   }{%false
   \pgfmathtruncatemacro{\xj}{\xi-1}
   \pgfmathtruncatemacro{\angle}{360/\actualnumitem*\xj}
   \edef\col{\@nameuse{color@\xj}}
   \node[bubble node, smvisible on=<\xi->](module\xi)
        at (center bubble.\angle) {\smitem };
   }%
   }%
   \end{tikzpicture}
   }{}%end-bubble diagram
   \IfStrEq{\diagramtype}{constellation diagram}{% true-const diagram
   \begin{tikzpicture}[every node/.style={align=center,let hypenation}]
   \foreach \smitem [count=\xi] in {#2}{\global\let\maxsmitem\xi}
   \pgfmathtruncatemacro\actualnumitem{\maxsmitem-1}
   \foreach \smitem [count=\xi] in {#2}{%
   \ifnumequal{\xi}{1}{ %true
   \node[planet, smvisible on=<\xi->](planet){\smitem};
   }{%false
   \pgfmathtruncatemacro{\xj}{\xi-1}
   \pgfmathtruncatemacro{\angle}{360/\actualnumitem*\xj}
   \edef\col{\@nameuse{color@\xj}}
   \node[satellite, smvisible on=<\xi->] (satellite\xi)
    at (\angle:\sm@core@distanceplanetsatellite) {\smitem };
   \draw[connection planet satellite, smvisible on=<\xi->]
    (planet) -- (satellite\xi);
 }%
   }%
   \end{tikzpicture}
   }{}%end-constellation diagram
   \IfStrEq{\diagramtype}{connected constellation diagram}{% true-conn const diagram
   \begin{tikzpicture}[every node/.style={align=center,let hypenation}]
   \foreach \smitem [count=\xi] in {#2}{\global\let\maxsmitem\xi}
   \pgfmathtruncatemacro\actualnumitem{\maxsmitem-1}
   \foreach \smitem [count=\xi] in {#2}{%
   \ifnumequal{\xi}{1}{ %true
   \node[planet,smvisible on=<\xi->](planet){\smitem};
   }{%false
   \pgfmathtruncatemacro{\xj}{\xi-1}
   \pgfmathtruncatemacro{\angle}{360/\actualnumitem*\xj}
   \edef\col{\@nameuse{color@\xj}}
   \node[satellite,smvisible on=<\xi->] (satellite\xj)
    at (\angle:\sm@core@distanceplanetsatellite) {\smitem };
   }%
   }%
   \foreach \smitem [count=\xi] in {#2}{%
      \ifnumgreater{\xi}{1}{ %true
      \pgfmathtruncatemacro{\xj}{\xi-1}
      \edef\col{\@nameuse{color@\xj}}
      \pgfmathtruncatemacro{\xk}{mod(\xj,\actualnumitem) +1}
      \pgfmathtruncatemacro{\smvisible}{\xi+1}
      \path[connection planet satellite,-,smvisible on=<\smvisible->]
       (satellite\xj) edge[bend right] (satellite\xk);
   }{}
   }%
   \end{tikzpicture}
   }{}%end-connected constellation diagram
   \IfStrEq{\diagramtype}{priority descriptive diagram}{% true-priority descriptive diagram
   \pgfmathparse{subtract(\sm@core@priorityarrowwidth,\sm@core@priorityarrowheadextend)}
   \pgfmathsetmacro\sm@core@priorityticksize{\pgfmathresult/2}
   \pgfmathsetmacro\arrowtickxshift{(\sm@core@priorityarrowwidth-\sm@core@priorityticksize)/2}
   \begin{tikzpicture}[every node/.style={align=center,let hypenation}]
   \foreach \smitem [count=\xi] in {#2}{\global\let\maxsmitem\xi}
   \foreach \smitem [count=\xi] in {#2}{%
   \edef\col{\@nameuse{color@\xi}}
   \pgfmathtruncatemacro\smvisible{\xi+1}
   \node[description,drop shadow={smvisible on=<\smvisible->},smvisible on=<\smvisible->]
    (module\xi) at (0,0+\xi*\sm@core@descriptiveitemsysep) {\smitem};
\draw[line width=\sm@core@prioritytick,\col,smvisible on=<\smvisible->]
 ([xshift=-\arrowtickxshift pt]module\xi.base west)--
 ($([xshift=-\arrowtickxshift pt]module\xi.base west)-(\sm@core@priorityticksize pt,0)$);
   }%
   \coordinate (A) at (module1);
   \coordinate (B) at (module\maxsmitem);
   \CalcHeight(A,B){heightmodules}
   \pgfmathadd{\heightmodules}{\sm@core@priorityarrowheightadvance}
   \pgfmathsetmacro{\distancemodules}{\pgfmathresult}
   \pgfmathsetmacro\arrowxshift{\sm@core@priorityarrowwidth/2}
   \begin{pgfonlayer}{background}
   \node[priority arrow] at ([xshift=-\arrowxshift pt]module1.south west){};
   \end{pgfonlayer}
   \end{tikzpicture}
   }{}% end-priority descriptive diagram
   \IfStrEq{\diagramtype}{sequence diagram}{% true-sequence diagram
   \begin{tikzpicture}[every node/.style={align=center,let hypenation}]
   \foreach \x[count=\xi, count=\prevx from 0] in {#2}{%
   \edef\col{\@nameuse{color@\xi}}
   \ifnum\xi=1
     \node[sequence item,smvisible on=<\xi->] (sequence-item\xi) {\x};
   \else
     \node[sequence item,anchor=west,smvisible on=<\xi->]
      (sequence-item\xi) at (sequence-item\prevx.east) {\x};
   \fi
   }
   \end{tikzpicture}
   }{}% end-sequence diagram
   }% end-no value 1
}% end-command

%% 
%% Copyright (C) 2012-2013 by Claudio Fiandrino
%% E-mail: <claudio <dot> fiandrino <at> gmail <dot> com>
%% 
%% This work may be distributed and/or modified under the
%% conditions of the LaTeX Project Public License (LPPL), either
%% version 1.3c of this license or (at your option) any later
%% version.  The latest version of this license is in the file:
%% 
%% http://www.latex-project.org/lppl.txt
%% 
%% This work is "maintained" (as per LPPL maintenance status) by
%% Claudio Fiandrino.
%% 
%% This work consists of the file  smartdiagram.dtx
%% and the derived files           smartdiagram.ins,
%%                                 smartdiagram.pdf,
%%                                 smartdiagramlibrarycore.definitions.code.tex,
%%                                 smartdiagramlibrarycore.styles.code.tex,
%%                                 smartdiagramlibrarycore.commands.code.tex
%%                                 smartdiagramlibraryadditions.code.tex and
%%                                 smartdiagram.sty.
%% 
%%
%% End of file `smartdiagramlibrarycore.commands.code.tex'.
