%%
%% This is file `leipzig.tex',
%% generated with the docstrip utility.
%%
%% The original source files were:
%%
%% leipzig.dtx  (with options: `abbrvs')
%% ----------------------------------------------------------------
%% leipzig --- A package to typeset and index linguistic gloss abbreviations.
%% E-mail: natalie.a.weber@gmail.com
%% Released under the LaTeX Project Public License v1.3c or later
%% See http://www.latex-project.org/lppl.txt
%% ----------------------------------------------------------------
%% 
%% Copyright (C) 2019 by Natalie Weber <natalie.a.weber@gmail.com>
%% 
%% This work may be distributed and/or modified under the
%% conditions of the LaTeX Project Public License, either version 1.3
%% of this license or (at your option) any later version.
%% The latest version of this license is in
%% 
%%   http://www.latex-project.org/lppl.txt
%% 
%% and version 1.3 or later is part of all distributions of LaTeX
%% version 2005/12/01 or later.
%% 
%% This work has the LPPL maintenance status `maintained'.
%% 
%% The Current Maintainer of this work is Natalie Weber.
%% 
%% This work consists of the file leipzig.dtx,
%% and the derived files
%%                                 README.md,
%%                                 leipzig.ins,
%%                                 leipzig.tex,
%%                                 leipzig.sty, and
%%                                 leipzig.pdf
%% 





%%%%%%%%%%%%%%%%%%%%%%%%%%%%%%%%%%%%%
%%  This is a simple list of newcommands which create shortcuts for    %%
%%  standard linguistic glosses (see the Leipzig Glossing rules,       %%
%%  http://www.eva.mpg.de/lingua/resources/glossing-rules.php         %%
%%%%%%%%%%%%%%%%%%%%%%%%%%%%%%%%%%%%%


\newleipzig{aarg}{a}{agent} %agent-like argument of
\newleipzig{abl}{abl}{ab\-la\-tive} %ablative
\newleipzig{abs}{abs}{ab\-so\-lu\-tive} %absolutive
\newleipzig{acc}{acc}{ac\-cusa\-tive} %accusative
\newleipzig{adj}{adj}{ad\-jec\-tive} %adjective
\newleipzig{adv}{adv}{ad\-ver\-bial} %adverb(ial)
\newleipzig{agr}{agr}{agreement} %agreement
\newleipzig{all}{all}{al\-la\-tive} %allative
\newleipzig{antip}{antip}{anti\-pas\-sive} %antipassive
\newleipzig{appl}{appl}{ap\-plica\-tive} %applicative
\newleipzig{art}{art}{article} %article
\newleipzig{aux}{aux}{aux\-il\-iary} %auxiliary
\newleipzig{ben}{ben}{bene\-fac\-tive} %benefactive
\newleipzig{caus}{caus}{causative} %causative
\newleipzig{clf}{clf}{clas\-si\-fi\-er} %classifier
\newleipzig{com}{com}{comi\-ta\-tive} %comitative
\newleipzig{comp}{comp}{com\-ple\-men\-ti\-zer} %complementizer
\newleipzig{compl}{compl}{com\-ple\-tive} %completive
\newleipzig{cond}{cond}{con\-di\-tion\-al} %conditional
\newleipzig{cop}{cop}{cop\-u\-la} %copula
\newleipzig{cvb}{cvb}{con\-verb} %converb
\newleipzig{dat}{dat}{da\-tive} %dative
\newleipzig{decl}{decl}{declarative}     %declarative
\newleipzig{def}{def}{definite}          %definite
\newleipzig{dem}{dem}{demonstrative}     %demonstrative
\newleipzig{det}{det}{determiner}        %determiner
\newleipzig{dist}{dist}{dis\-tal}          %distal
\newleipzig{distr}{distr}{dis\-tri\-bu\-tive}  %distributive
\newleipzig{du}{du}{dual}                %dual
\newleipzig{dur}{dur}{dur\-ative}          %durative
\newleipzig{erg}{erg}{erg\-ative}          %ergative
\newleipzig{excl}{excl}{ex\-clu\-sive}       %exclusive
\newleipzig{f}{f}{feminine}              %feminine
\newleipzig{foc}{foc}{focus}             %focus
\newleipzig{fut}{fut}{future}            %future
\newleipzig{gen}{gen}{gen\-i\-tive}          %genitive
\newleipzig{imp}{imp}{imperative}        %imperative
\newleipzig{incl}{incl}{inclusive}       %inclusive
\newleipzig{ind}{ind}{indicative}        %indicative
\newleipzig{indf}{indf}{indefinite}      %indefinite
\newleipzig{inf}{inf}{in\-fini\-tive}        %infinitive
\newleipzig{ins}{ins}{instrumental}      %instrumental
\newleipzig{intr}{intr}{in\-tran\-si\-tive}    %intransitive
\newleipzig{ipfv}{ipfv}{im\-per\-fec\-tive}    %imperfective
\newleipzig{irr}{irr}{ir\-real\-is}          %irrealis
\newleipzig{loc}{loc}{loc\-ative}          %locative
\newleipzig{m}{m}{masculine}             %masculine
\newleipzig{n}{n}{neuter}                %neuter
\newleipzig{neg}{neg}{negative}          %negation, negative
\newleipzig{nmlz}{nmlz}{nom\-i\-nal\-iz\-er}    %nominalizer/nominalization
\newleipzig{nom}{nom}{nom\-in\-ative}        %nominative
\newleipzig{obj}{obj}{object}            %object
\newleipzig{obl}{obl}{ob\-lique}           %oblique
\newleipzig{parg}{p}{patient}            %patient
\newleipzig{pass}{pass}{passive}        %passive
\newleipzig{pfv}{pfv}{per\-fec\-tive}        %perfective
\newleipzig{pl}{pl}{plural}              %plural
\newleipzig{poss}{poss}{possessive}      %possessive
\newleipzig{pred}{pred}{pred\-i\-ca\-tive}     %predicative
\newleipzig{prf}{prf}{perfect}           %perfect
\newleipzig{prs}{prs}{present}           %present
\newleipzig{prog}{prog}{progressive}     %progressive
\newleipzig{proh}{proh}{prohibitive}     %prohibitive
\newleipzig{prox}{prox}{prox\-i\-mal}        %proximal/proximate
\newleipzig{pst}{pst}{past}              %past
\newleipzig{ptcp}{ptcp}{participle}      %participle
\newleipzig{purp}{purp}{pur\-po\-sive}       %purposive
\newleipzig{q}{q}{question particle}     %question particle/marker
\newleipzig{quot}{quot}{quot\-ative}       %quotative
\newleipzig{recp}{recp}{recip\-ro\-cal}      %reciprocal
\newleipzig{refl}{refl}{reflexive}       %reflexive
\newleipzig{rel}{rel}{relative}          %relative
\newleipzig{res}{res}{re\-sul\-ta\-tive}       %resultative
\newleipzig{sarg}{s}{argument of intransitive verb}
                                        %single argument of intransitive verb
\newleipzig{sbj}{sbj}{subject}           %subject
\newleipzig{sbjv}{sbjv}{sub\-junc\-tive}     %subjunctive
\newleipzig{sg}{sg}{singular}            %singular
\newleipzig{top}{top}{topic}             %topic
\newleipzig{tr}{tr}{tran\-si\-tive}          %transitive
\newleipzig{voc}{voc}{voc\-ative}          %vocative

%%  For backwards compatibility with older versions of the leipzig package,
%%  where `subjunctive' was incorrectly abbreviated to SUBJ.

\providecommand{\Subj}{}
\let\Subj\Sbjv

%%  Define short versions of person + number:
\newleipzig{first}{1}{first person}%
\newleipzig{second}{2}{second person}%
\newleipzig{third}{3}{third person}%

\newcommand{\Fsg}{{\First}{\Sg}}%
\newcommand{\Fdu}{{\First}{\Du}}%
\newcommand{\Fpl}{{\First}{\Pl}}%
\newcommand{\Ssg}{{\Second}{\Sg}}%
\newcommand{\Sdu}{{\Second}{\Du}}%
\newcommand{\Spl}{{\Second}{\Pl}}%
\newcommand{\Tsg}{{\Third}{\Sg}}%
\newcommand{\Tdu}{{\Third}{\Du}}%
\newcommand{\Tpl}{{\Third}{\Pl}}%

%% 
%% Copyright (C) 2019 by Natalie Weber <natalie.a.weber@gmail.com>
%% 
%% This work may be distributed and/or modified under the
%% conditions of the LaTeX Project Public License, either version 1.3
%% of this license or (at your option) any later version.
%% The latest version of this license is in
%% 
%%   http://www.latex-project.org/lppl.txt
%% 
%% and version 1.3 or later is part of all distributions of LaTeX
%% version 2005/12/01 or later.
%%
%% End of file `leipzig.tex'.
