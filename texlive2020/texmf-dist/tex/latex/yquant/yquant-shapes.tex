% To draw the wires, we need to determine border angles. However, pgf's approach for this (giving the border point that lies on a line from the center to a desired probe point) is not suitable for this. Instead, we need to give a point that is projected perpendicularly onto the shape.
\def\pgf@sh@anchorxproj#1{%
   \csgdef{pgf@anchor@\pgf@sm@shape@name @xproj}##1{\pgf@process{##1}#1}%
}%
\def\pgf@sh@inheritanchorxproj[from=#1]{%
   \global\csletcs{pgf@anchor@\pgf@sm@shape@name @xproj}{pgf@anchor@#1@xproj}%
}%
\def\pgf@sh@anchoryproj#1{%
   \csgdef{pgf@anchor@\pgf@sm@shape@name @yproj}##1{\pgf@process{##1}#1}%
}%
\def\pgf@sh@inheritanchoryproj[from=#1]{%
   \global\csletcs{pgf@anchor@\pgf@sm@shape@name @yproj}{pgf@anchor@#1@yproj}%
}%
\patchcmd%
   \pgfdeclareshape%
   {\let\anchorborder=\pgf@sh@anchorborder}%
   {\let\anchorborder=\pgf@sh@anchorborder%
    \let\anchorxproj=\pgf@sh@anchorxproj%
    \let\inheritanchorxproj=\pgf@sh@inheritanchorxproj
    \let\anchoryproj=\pgf@sh@anchoryproj%
    \let\inheritanchoryproj=\pgf@sh@inheritanchoryproj}%
   {}%
   {\PackageError{yquant.sty}%
                 {Failed to patch \string\pgfdeclareshape}%
                 {yquant could not provide a necessary extension to pgf.}}%
% This is mostly a copy of \pgfpointshapeborder, but we do not perform the center-anchor shift.
\def\pgfpointshapexproj#1#2{%
   % Ok, check whether #1 is known!
   \ifcsname pgf@sh@ns@#1\endcsname%
      \pgf@process{%
         \edef\pgfreferencednodename{#1}% for use inside of anchors.
         % MW install special macros
         \csname pgf@sh@ma@#1\endcsname% MW
         % install special coordinates
         \csname pgf@sh@np@#1\endcsname%
         \pgf@process{%
            \pgf@process{\pgfpointtransformed{#2}}%
            \pgfsettransform{\csname pgf@sh@nt@#1\endcsname}%
            \pgftransforminvert%
            \pgf@pos@transform@glob%
            \pgf@xa=\pgf@x%
            \pgf@ya=\pgf@y%
            \csname pgf@anchor@\csname pgf@sh@ns@#1\endcsname @xproj\endcsname{\pgfqpoint{\pgf@xa}{\pgf@ya}}%
            \pgfsettransform{\csname pgf@sh@nt@#1\endcsname}%
            \pgf@pos@transform@glob%
         }%
         % Add inter picture transformation
         \pgf@shape@interpictureshift{#1}%
         % Undo current transformation
         \pgftransforminvert%
         \pgf@pos@transform@glob%
      }%
   \else%
      \pgferror{No shape named #1 is known}%
      \pgfpointorigin%
   \fi%
}%
\def\ifpgfpointshapexproj#1{%
   \pgfutil@ifundefined{pgf@sh@ns@#1}%
      {\pgferror{No shape named #1 is known}}%
      {\ifcsname pgf@anchor@\csname pgf@sh@ns@#1\endcsname @xproj\endcsname}%
}
\let\pgfpointshapeyproj=\pgfpointshapexproj%
\patchcmd%
   \pgfpointshapeyproj%
   {xproj}{yproj}%
   {}{\PackageError{yquant.sty}{Failed to provide \string\pgfpointshapeyproj}{}}%
\def\ifpgfpointshapeyproj#1{%
   \pgfutil@ifundefined{pgf@sh@ns@#1}%
      {\pgferror{No shape named #1 is known}}%
      {\ifcsname pgf@anchor@\csname pgf@sh@ns@#1\endcsname @yproj\endcsname}%
}

% Calculate the intersection of an ellipse centered at the origin with radii #1 and #2 with a horizontal line at position #3. Result goes to \pgf@xa, and it is the right intersection point.
\protected\def\yquant@shape@ellipse@xfromy#1#2#3{%
   \ifdim#3>#2\relax%
      \global\pgf@xa=0pt %
   \else%
      \ifdim-\dimexpr#3\relax>#2\relax%
         \global\pgf@xa=0pt %
      \else%
         \begingroup%
            % Here, we essentially do #1*sqrt(1-(#3/#2)^2)
            \dimen2=#2\relax%
            \dimen3=#3\relax%
            % if we divide by a dimension, it is internally converted to sp, so we divide by its pt-value and again by 65536. Same for multiplication. etex fuses muldiv to 64bit, so we don't get overflows.
            % calculate the sqrt; but \pgfmathsqrt@ expects a real number without dimension suffix. It internally does \expandafter\pgfmath@x#1pt\relax, so just gobble the additional pt.
            % TODO: is there a better way, exploiting perhaps a representation in sp?
            \pgfmathsqrt@{\the\dimexpr1pt-\dimen3*\dimen3/\dimen2*65536/\dimen2\relax%
                          \@gobbletwo}%
            \global\pgf@xa=\pgfmathresult\dimexpr#1\relax%
         \endgroup%
      \fi%
   \fi%
}

\pgfdeclareshape{yquant-text}{%
   \inheritsavedanchors[from=rectangle]%
   \foreach \anc in {center, mid, base, north, south, west, mid west, base west, north west, south west, east, mid east, base east, north east, south east} {%
      \inheritanchor[from=rectangle]{\anc}%
   }%
   \global\cslet{pgf@anchor@yquant-text@circuit}\pgf@anchor@rectangle@center%
   \inheritanchorborder[from=rectangle]%
   \inheritbackgroundpath[from=rectangle]%
   \anchorxproj{%
      \pgf@xa=\pgf@x%
      \pgf@ya=\pgf@y%
      % The origin is at the left baseline of the text, i.e. to the left we have the inner + outer xsep, the the right there's the text + inner + outer xsep.
      \northeast%
      \pgf@xb=.5\pgf@x%
      \southwest%
      \advance \pgf@xb by .5\pgf@x%
      \ifdim\pgf@xa>\pgf@xb%
         % to the right
         \northeast%
      % else we already called \southwest
      \fi%
      \pgf@y=\pgf@ya%
   }%
   \anchoryproj{%
      \pgf@xa=\pgf@x%
      \pgf@ya=\pgf@y%
      % The origin is at the left baseline of the text, i.e. to the top we have the text height + inner + outer ysep, the the bottom there's the text depth + inner + outer ysep.
      \northeast%
      \pgf@yb=.5\pgf@y%
      \southwest%
      \advance \pgf@yb by .5\pgf@y%
      \ifdim\pgf@ya>\pgf@yb%
         % to the top
         \northeast%
      % else we already called \southwest
      \fi%
      \pgf@x=\pgf@xa%
   }
}

\pgfdeclareshape{yquant-rectangle}{%
   % BEGIN_FOLD Saved anchors
   \saveddimen\xradius{%
      \pgfmathsetlength\pgf@x{\pgfkeysvalueof{/tikz/x radius}}%
      \ifdim\wd\pgfnodeparttextbox=0pt %
         \pgf@xa=0pt %
      \else%
         \pgfmathsetlength\pgf@xa{\pgfkeysvalueof{/pgf/inner xsep}}%
      \fi%
      \ifdim\dimexpr.5\wd\pgfnodeparttextbox+\pgf@xa\relax>\pgf@x%
         \pgf@x=\dimexpr.5\wd\pgfnodeparttextbox+\pgf@xa\relax%
      \fi%
   }%
   \saveddimen\yradius{%
      \pgfmathsetlength\pgf@x{\pgfkeysvalueof{/tikz/y radius}}%
      \ifdim\dimexpr\ht\pgfnodeparttextbox+\dp\pgfnodeparttextbox\relax=0pt %
         \pgf@xa=0pt %
      \else%
         \pgfmathsetlength\pgf@xa{\pgfkeysvalueof{/pgf/inner ysep}}%
      \fi%
      \@tempdima=\dimexpr.5\ht\pgfnodeparttextbox+.5\dp\pgfnodeparttextbox+\pgf@xa\relax%
      \ifdim\@tempdima>\pgf@x%
         \pgf@x=\@tempdima%
      \fi%
   }%
   \savedanchor\stext{%
      \pgfqpoint%
         {-.5\wd\pgfnodeparttextbox}%
         {-\dimexpr.5\ht\pgfnodeparttextbox-.5\dp\pgfnodeparttextbox\relax}%
   }%
   % END_FOLD
   % BEGIN_FOLD Operator anchors
   \global\cslet{pgf@anchor@yquant-rectangle@center}\pgfpointorigin%
   \anchor{text}{\stext}%
   \anchor{north}%
          {\pgfqpoint{0pt}%
                     {\yradius}}%
   \anchor{north east}%
          {\pgfqpoint{\xradius}%
                     {\yradius}}%
   \anchor{east}%
          {\pgfqpoint{\xradius}%
                     {0pt}}%
   \anchor{south east}%
          {\pgfqpoint{\xradius}%
                     {-\yradius}}%
   \anchor{south}%
          {\pgfqpoint{0pt}%
                     {-\yradius}}%
   \anchor{south west}%
          {\pgfqpoint{-\xradius}%
                     {-\yradius}}%
   \anchor{west}%
          {\pgfqpoint{-\xradius}% \dimexpr not really necessary...
                     {0pt}}%
   \anchor{north west}%
          {\pgfqpoint{-\xradius}%
                     {\yradius}}%
   \anchorborder{%
      \@tempdima=\pgf@x%
      \@tempdimb=\pgf@y%
      \pgfpointborderrectangle{\pgfqpoint{\@tempdima}{\@tempdimb}}%
                              {\pgfqpoint{\xradius}{\yradius}}%
   }%
   \anchorxproj{%
      \ifdim\pgf@x>0pt %
         % to the right
         \pgf@x=\xradius\relax%
      \else%
         % to the left
         \pgf@x=-\xradius\relax%
      \fi%
   }%
   \anchoryproj{%
      \ifdim\pgf@y>0pt %
         % to the top
         \pgf@y=\yradius\relax%
      \else%
         % to the bottom
         \pgf@y=-\yradius\relax%
      \fi%
   }%
   % END_FOLD
   % BEGIN_FOLD Circuit-related anchors
   \global\csletcs{pgf@anchor@yquant-rectangle@circuit}%
                  {pgf@anchor@yquant-rectangle@center}%
   % END_FOLD
   % BEGIN_FOLD Path
   \backgroundpath{%
      \pgfpathrectanglecorners%
         {\pgfqpoint{-\xradius}{\yradius}}%
         {\pgfqpoint{\xradius}{-\yradius}}%
   }%
   % END_FOLD
}

\pgfdeclareshape{yquant-circle}{%
   \inheritsavedanchors[from=yquant-rectangle]%
   \foreach \anc in {center, north, east, south, west, circuit, text} {%
      \inheritanchor[from=yquant-rectangle]{\anc}%
   }%
   \anchor{north east}%
          {\pgfqpoint{.707107\dimexpr\xradius\relax}%
                     {.707107\dimexpr\yradius\relax}}%
   \anchor{south east}%
          {\pgfqpoint{.707107\dimexpr\xradius\relax}%
                     {-.707107\dimexpr\yradius\relax}}%
   \anchor{south west}%
          {\pgfqpoint{-.707107\dimexpr\xradius\relax}%
                     {-.707107\dimexpr\yradius\relax}}%
   \anchor{north west}%
          {\pgfqpoint{-.707107\dimexpr\xradius\relax}%
                     {.707107\dimexpr\yradius\relax}}%
   \anchorborder{%
      \@tempdima=\pgf@x%
      \@tempdimb=\pgf@y%
      \pgfpointborderellipse{\pgfqpoint{\@tempdima}{\@tempdimb}}%
                            {\pgfqpoint{.707107\dimexpr\xradius\relax}%
                                       {.707107\dimexpr\yradius\relax}}%
   }%
   \anchorxproj{%
      \yquant@shape@ellipse@xfromy\xradius\yradius\pgf@y%
      \ifdim\pgf@x>0pt %
         % to the right
         \pgf@x=\pgf@xa%
      \else%
         % to the left
         \pgf@x=-\pgf@xa%
      \fi%
   }%
   \anchoryproj{%
      \yquant@shape@ellipse@xfromy\yradius\xradius\pgf@x%
      \ifdim\pgf@y>0pt %
         % to the top
         \pgf@y=\pgf@xa%
      \else%
         % to the bottom
         \pgf@y=-\pgf@xa%
      \fi%
   }%
   \foregroundpath{%
      \pgfpathellipse{\pgfpointorigin}%
                     {\pgfqpoint{\xradius}{0pt}}%
                     {\pgfqpoint{0pt}{\yradius}}%
   }%
}

\pgfdeclareshape{yquant-zz}{%
   % Here, the radii have a special meaning, we don't put text into the shape
   \saveddimen\xradius{%
      \pgfmathsetlength\pgf@x{\pgfkeysvalueof{/tikz/x radius}}%
   }%
   \saveddimen\yradius{%
      \pgfmathsetlength\pgf@x{\pgfkeysvalueof{/tikz/y radius}}%
   }%
   \foreach \anc in {center, north, east, south, west, circuit} {%
      \inheritanchor[from=yquant-circle]{\anc}%
   }%
   \anchor{north east}%
          {\pgfqpoint{.707107\dimexpr\xradius\relax}%
                     {\dimexpr\yradius-.292893\dimexpr\xradius\relax\relax}}%
   \anchor{south east}%
          {\pgfqpoint{.707107\dimexpr\xradius\relax}%
                     {\dimexpr-\yradius+.292893\dimexpr\xradius\relax\relax}}%
   \anchor{south west}%
          {\pgfqpoint{-.707107\dimexpr\xradius\relax}%
                     {\dimexpr-\yradius+.292893\dimexpr\xradius\relax\relax}}%
   \anchor{north west}%
          {\pgfqpoint{-.707107\dimexpr\xradius\relax}%
                     {\dimexpr\yradius-.292893\dimexpr\xradius\relax\relax}}%
   % TODO: this is not really the correct border anchor
   \inheritanchorborder[from=yquant-rectangle]%
   \anchorxproj{%
      \ifdim\pgf@y<0pt %
         \pgf@ya=\dimexpr\pgf@y+\yradius-\xradius\relax%
      \else%
         \pgf@ya=\dimexpr\pgf@y-\yradius+\xradius\relax%
      \fi%
      \yquant@shape@ellipse@xfromy\xradius\xradius\pgf@ya%
      \ifdim\pgf@x>0pt %
         % to the right
         \pgf@x=\pgf@xa%
      \else%
         % to the left
         \pgf@x=-\pgf@xa%
      \fi%
   }%
   \anchoryproj{%
      \yquant@shape@ellipse@xfromy\xradius\xradius\pgf@xa%
      \ifdim\pgf@y>0pt %
         % to the top
         \pgf@y=\dimexpr\yradius-\xradius+\pgf@xa\relax%
      \else%
         % to the bottom
         \pgf@y=-\dimexpr\yradius-\xradius+\pgf@xa\relax%
      \fi%
   }%
   \backgroundpath{%
      \pgfpathmoveto{\pgfqpoint{0pt}{\dimexpr\yradius-2\dimexpr\xradius\relax\relax}}%
      \pgfpathlineto{\pgfqpoint{0pt}{-\dimexpr\yradius-2\dimexpr\xradius\relax\relax}}%
      \pgfpathcircle{\pgfqpoint{0pt}{\dimexpr\yradius-\xradius\relax}}%
                    {\xradius}%
      \pgfpathcircle{\pgfqpoint{0pt}{\dimexpr\xradius-\yradius\relax}}%
                    {\xradius}%
   }%
}

\pgfdeclareshape{yquant-xx}{%
   \inheritsavedanchors[from=yquant-zz]%
   \savedmacro\ifconnector{%
      \let\ifconnector=\ifyquant@config@multi@line%
   }%
   \foreach \anc in {center, north, north east, east, south east, south, south west, west, north west, circuit} {%
      \inheritanchor[from=yquant-rectangle]{\anc}%
   }%
   % TODO: this is not really the correct border anchor
   \inheritanchorborder[from=yquant-rectangle]%
   \inheritanchorxproj[from=yquant-rectangle]%
   \inheritanchoryproj[from=yquant-rectangle]%
   % Draw the operator itself
   \backgroundpath{%
      \ifconnector%
         \pgfpathmoveto{\pgfqpoint{0pt}{\dimexpr\yradius-2\dimexpr\xradius\relax\relax}}%
         \pgfpathlineto{\pgfqpoint{0pt}{-\dimexpr\yradius-2\dimexpr\xradius\relax\relax}}%
      \fi%
      \pgfpathrectanglecorners%
         {\pgfqpoint{-\xradius}{\dimexpr\yradius\relax}}%
         {\pgfqpoint{\xradius}{\dimexpr\yradius-2\dimexpr\xradius\relax\relax}}%
      \pgfpathrectanglecorners%
         {\pgfqpoint{-\xradius}{-\yradius}}%
         {\pgfqpoint{\xradius}{-\dimexpr\yradius-2\dimexpr\xradius\relax\relax}}%
   }%
}

\pgfdeclareshape{yquant-slash}{%
   \inheritsavedanchors[from=yquant-zz]%
   \foreach \anc in {center, north, north east, east, south east, south, south west, west, north west, circuit} {%
      \inheritanchor[from=yquant-rectangle]{\anc}%
   }%
   \inheritanchorborder[from=yquant-rectangle]%
   \backgroundpath{%
      \pgfpathmoveto{\pgfqpoint{\xradius}{\yradius}}%
      \pgfpathlineto{\pgfqpoint{-\xradius}{-\yradius}}%
   }%
}

\pgfdeclareshape{yquant-swap}{%
   \inheritsavedanchors[from=yquant-xx]%
   \foreach \anc in {center, north, north east, east, south east, south, south west, west, north west, circuit} {%
      \inheritanchor[from=yquant-rectangle]{\anc}%
   }%
   \inheritanchorborder[from=yquant-rectangle]%
   \inheritanchoryproj[from=yquant-rectangle]%
   \backgroundpath{%
      % Connector
      \ifconnector%
         \pgfpathmoveto{\pgfqpoint{0pt}{\dimexpr\yradius-\xradius\relax}}%
         \pgfpathlineto{\pgfqpoint{0pt}{-\dimexpr\yradius-\xradius\relax}}%
      \fi%
      % Upper cross
      \pgfpathmoveto{\pgfqpoint{-\xradius}{\yradius}}%
      \pgfpathlineto{\pgfqpoint{\xradius}%
                               {\dimexpr\yradius-2\dimexpr\xradius\relax\relax}}%
      \pgfpathmoveto{\pgfqpoint{\xradius}{\yradius}}%
      \pgfpathlineto{\pgfqpoint{-\xradius}
                               {\dimexpr\yradius-2\dimexpr\xradius\relax\relax}}%
      % Lower cross
      \pgfpathmoveto{\pgfqpoint{-\xradius}%
                               {-\dimexpr\yradius-2\dimexpr\xradius\relax\relax}}%
      \pgfpathlineto{\pgfqpoint{\xradius}{-\yradius}}%
      \pgfpathmoveto{\pgfqpoint{\xradius}%
                               {-\dimexpr\yradius-2\dimexpr\xradius\relax\relax}}%
      \pgfpathlineto{\pgfqpoint{-\xradius}{-\yradius}}%
   }%
}

\pgfdeclareshape{yquant-oplus}{%
   \inheritsavedanchors[from=yquant-zz]%
   \foreach \anc in {center, north, north east, east, south east, south, south west, west, north west, circuit} {%
      \inheritanchor[from=yquant-circle]{\anc}%
   }%
   \inheritanchorborder[from=yquant-circle]%
   \inheritanchorxproj[from=yquant-circle]%
   \inheritanchoryproj[from=yquant-circle]%
   \backgroundpath{%
      \pgfpathmoveto{\pgfqpoint{0pt}{\yradius}}%
      \pgfpathlineto{\pgfqpoint{0pt}{-\yradius}}%
      \pgfpathmoveto{\pgfqpoint{-\xradius}{0pt}}%
      \pgfpathlineto{\pgfqpoint{\xradius}{0pt}}%
      \pgfpathellipse{\pgfpointorigin}%
                     {\pgfqpoint{\xradius}{0pt}}%
                     {\pgfqpoint{0pt}{\yradius}}%
   }%
}

\pgfdeclareshape{yquant-measure}{%
   \saveddimen\xradius{%
      \pgfmathsetlength\pgf@x{\pgfkeysvalueof{/tikz/x radius}}%
      \ifdim.5\wd\pgfnodeparttextbox>\pgf@x%
         \pgf@x=.5\wd\pgfnodeparttextbox%
      \fi%
   }%
   \saveddimen\yradius{%
      \pgfmathsetlength\pgf@x{\pgfkeysvalueof{/tikz/y radius}}%
      \ifdim\dimexpr\ht\pgfnodeparttextbox+\dp\pgfnodeparttextbox\relax=0pt %
         \ifdim\pgf@x<1.25mm %
            \pgf@x=1.25mm %
         \fi%
      \else%
         % We need the 2.3mm for the meter sign, the height of the text plus a minimum separation of 2pt
         \pgf@y=.5\dimexpr\ht\pgfnodeparttextbox+\dp\pgfnodeparttextbox+2.3mm+4pt\relax%
         \ifdim\pgf@x<\pgf@y%
            \pgf@x=\pgf@y%
         \fi%
      \fi%
   }%
   \savedanchor\stext{%
      \pgfqpoint%
         {-.5\wd\pgfnodeparttextbox}%
         {\dp\pgfnodeparttextbox}%
   }%
   \saveddimen\textheight{%
      \pgf@x=\ht\pgfnodeparttextbox%
   }
   \foreach \anc in {center, north, north east, east, south east, south, south west, west, north west, circuit} {%
      \inheritanchor[from=yquant-rectangle]{\anc}%
   }%
   \anchor{text}{%
      \stext%
      \pgf@y=\dimexpr-\yradius+1pt+\pgf@y\relax%
   }%
   \inheritanchorborder[from=yquant-rectangle]%
   \inheritanchorxproj[from=yquant-rectangle]%
   \inheritanchoryproj[from=yquant-rectangle]%
   \backgroundpath{%
      \pgfpathrectanglecorners%
         {\pgfqpoint{-\xradius}{\yradius}}%
         {\pgfqpoint{\xradius}{-\yradius}}%
   }
   \beforebackgroundpath{%
      % Make sure the meter does not extend beyond the box (we are in a scope here)
      \path [clip]
         (-\xradius, \yradius) rectangle (\xradius, -\yradius);%
      % The position of the meter symbol depends on the presence of the text. If there is no text, we just vertically center. If there is some text, we shift the symbol upwards from the text until there is no overlap any more.
      \csname pgf@anchor@yquant-measure@text\endcsname%
      \advance\pgf@y by \textheight\relax%
      \ifdim\pgf@y<-1.15mm %
         \@tempdima=-1.15mm %
      \else%
         \@tempdima=\dimexpr\pgf@y+2pt\relax%
      \fi%
      \path [/yquant/operators/every measure meter]
         (-2.25mm, \@tempdima) arc[start angle=160, end angle=20,%
                                   x radius=2.25mm, y radius=1.4mm]
         (0, \@tempdima) -- ++(1.6mm, 2.3mm);
   }%
}

\pgfdeclareshape{yquant-dmeter}{%
   \saveddimen\xradius{%
      \pgfmathsetlength\pgf@x{\pgfkeysvalueof{/tikz/x radius}}%
      \ifdim\wd\pgfnodeparttextbox=0pt %
         \pgf@xa=0pt %
      \else%
         \pgfmathsetlength\pgf@xa{\pgfkeysvalueof{/pgf/inner xsep}+.5mm}%
      \fi%
      \ifdim\dimexpr.5\wd\pgfnodeparttextbox+\pgf@xa\relax>\pgf@x%
         \pgf@x=\dimexpr.5\wd\pgfnodeparttextbox+\pgf@xa\relax%
      \fi%
   }%
   \saveddimen\yradius{%
      \pgfmathsetlength\pgf@x{\pgfkeysvalueof{/tikz/y radius}}%
      \ifdim\dimexpr\ht\pgfnodeparttextbox+\dp\pgfnodeparttextbox\relax=0pt %
         \pgf@xa=0pt %
      \else%
         \pgfmathsetlength\pgf@xa{\pgfkeysvalueof{/pgf/inner ysep}}%
      \fi%
      \@tempdima=\dimexpr.5\ht\pgfnodeparttextbox+.5\dp\pgfnodeparttextbox+\pgf@xa\relax%
      \ifdim\@tempdima>\pgf@x%
         \pgf@x=\@tempdima%
      \fi%
   }%
   \savedanchor\stext{%
      \pgfqpoint%
         {-\dimexpr.5\wd\pgfnodeparttextbox+1mm\relax}%
         {-\dimexpr.5\ht\pgfnodeparttextbox-.5\dp\pgfnodeparttextbox\relax}%
   }%
   \foreach \anc in {center, north, south, south west, west, north west, circuit, text} {%
      \inheritanchor[from=yquant-rectangle]{\anc}%
   }%
   \foreach \anc in {north east, east, south east} {%
      \inheritanchor[from=yquant-circle]{\anc}%
   }%
%   \anchor{text}{%
%      \stext%
%      \pgf@x=-.5\dimexpr\xradius-\pgf@x\relax%
%   }%
   \anchorborder{%
      \@tempdima=\pgf@x%
      \@tempdimb=\pgf@y%
      \ifdim\pgf@x>\dimexpr\xradius-3mm\relax%
         \pgfpointborderellipse{\pgfqpoint{\dimexpr\@tempdima-\xradius+3mm\relax}%
                                          {\@tempdimb}}%
                               {\pgfqpoint{3mm}%
                                          {.707107\dimexpr\yradius\relax}}%
         \advance\pgf@x by \dimexpr\xradius-3mm\relax%
      \else%
         \pgfpointborderrectangle{\pgfqpoint{\@tempdima}{\@tempdimb}}%
                                 {\pgfqpoint{\xradius}{\yradius}}%
      \fi%
   }%
   \anchorxproj{%
      \ifdim\pgf@x>\dimexpr\xradius-3mm\relax%
         % to the right
         \advance\pgf@x by -\dimexpr\xradius-3mm\relax%
         \yquant@shape@ellipse@xfromy{3mm}\yradius\pgf@y%
         \pgf@x=\dimexpr\pgf@xa+\xradius-3mm\relax%
      \else%
         % to the left
         \pgf@x=-\xradius\relax%
      \fi%
   }%
   \anchoryproj{%
      \ifdim\pgf@x>\dimexpr\xradius-3mm\relax%
         % to the right
         \advance\pgf@x by -\dimexpr\xradius-3mm\relax%
         \yquant@shape@ellipse@xfromy\yradius{3mm}\pgf@x%
         \advance\pgf@x by \dimexpr\xradius-3mm\relax%
         \ifdim\pgf@y>0pt %
            % to the top
            \pgf@y=\pgf@xa%
         \else%
            % to the bottom
            \pgf@y=-\pgf@xa%
         \fi%
      \else%
         % to the left
         \ifdim\pgf@y>0pt %
            % to the top
            \pgf@y=\yradius\relax%
         \else%
            \pgf@y=-\yradius\relax%
         \fi%
      \fi%
   }%
   \backgroundpath{%
      \pgfpathmoveto{\pgfqpoint{-\xradius}{\yradius}}%
      \pgfpathlineto{\pgfqpoint{\dimexpr\xradius-3mm\relax}{\yradius}}%
      \pgfpatharc{90}{-90}{3mm and \yradius}%
      \pgfpathlineto{\pgfqpoint{-\xradius}{-\yradius}}%
      \pgfpathclose%
   }%
}

\pgfdeclareshape{yquant-barrier}{%
   % Here, the radii have a special meaning, we don't put text into the shape
   \saveddimen\xradius{%
      \pgf@x=.5\pgflinewidth%
   }%
   \savedanchor\shorten{%
      \pgfqpoint\pgf@shorten@end@additional\pgf@shorten@start@additional%
   }%
   \saveddimen\yradius{%
      \pgfmathsetlength\pgf@x{\pgfkeysvalueof{/tikz/y radius}+.5*\yquant@config@register@sep}%
   }%
   \foreach \anc in {center, north, north east, east, south east, south, south west, west, north west, circuit} {%
      \inheritanchor[from=yquant-rectangle]{\anc}%
   }%
   \inheritanchorborder[from=yquant-rectangle]%
   \backgroundpath{%
      \pgfsetlinewidth{\xradius}%
      \shorten%
      \pgf@xa=\dimexpr\yradius+\pgf@x\relax%
      \pgf@ya=\dimexpr\yradius+\pgf@y\relax%
      \pgfpathmoveto{\pgfqpoint{0pt}{\pgf@xa}}%
      \pgfpathlineto{\pgfqpoint{0pt}{-\pgf@ya}}%
   }%
}