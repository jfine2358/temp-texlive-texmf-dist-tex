%D \module
%D   [       file=luatex-plain,
%D        version=2009.12.01,
%D          title=\LUATEX\ Macros,
%D       subtitle=Plain Format,
%D         author=Hans Hagen,
%D           date=\currentdate,
%D      copyright={PRAGMA ADE \& \CONTEXT\ Development Team}]

\input plain

\directlua {tex.enableprimitives('', tex.extraprimitives())}

% We assume that pdf is used.

\ifdefined\pdfextension
    \input luatex-pdf \relax
\fi

\outputmode 1

% \outputmode 0 \magnification\magstep5

% We set the page dimensions because otherwise the backend does weird things
% when we have for instance this on a line of its own:
%
%   \hbox to 100cm {\hss wide indeed\hss}
%
% The page dimension calculation is a fuzzy one as there are some compensations
% for the \hoffset and \voffset and such. I remember long discussions and much
% trial and error in figuring this out during pdftex development times. Where
% a dvi driver will project on a papersize (and thereby clip) the pdf backend
% has to deal with the lack of a page concept on tex by some guessing. Normally
% a macro package will set the dimensions to something reasonable anyway.

\pagewidth   8.5truein
\pageheight 11.0truein

% We load some code at runtime:

\everyjob \expandafter {%
    \the\everyjob
    %D \module
%D   [       file=luatex-basics,
%D        version=2009.12.01,
%D          title=\LUATEX\ Support Macros,
%D       subtitle=Attribute Allocation,
%D         author=Hans Hagen,
%D           date=\currentdate,
%D      copyright={PRAGMA ADE \& \CONTEXT\ Development Team}]

%D As soon as we feel the need this file will file will contain an extension
%D to the standard plain register allocation. For the moment we stick to a
%D rather dumb attribute allocator. We start at 256 because we don't want
%D any interference with the attributes used in the font handler.

\ifx\newattribute\undefined \else \endinput \fi

\newcount \lastallocatedattribute \lastallocatedattribute=255

\def\newattribute#1%
  {\global\advance\lastallocatedattribute 1
   \attributedef#1\lastallocatedattribute}

% maybe we will have luatex-basics.lua some day for instance when more
% (pdf) primitives have moved to macros)

\directlua {

    gadgets = gadgets or { } % reserved namespace

    gadgets.functions = { }
    local registered  = {}

    function gadgets.functions.reverve()
        local numb = newtoken.scan_int()
        local name = newtoken.scan_string()
        local okay = string.gsub(name,"[\string\\ ]","")
        registered[okay] = numb
        texio.write_nl("reserving lua function '"..okay.."' with number "..numb)
    end

    function gadgets.functions.register(name,f)
        local okay = string.gsub(name,"[\string\\ ]","")
        local numb = registered[okay]
        if numb then
            texio.write_nl("registering lua function '"..okay.."' with number "..numb)
            lua.get_functions_table()[numb] = f
        else
            texio.write_nl("lua function '"..okay.."' is not reserved")
        end
    end

}

\newcount\lastallocatedluafunction

\def\newluafunction#1%
  {\ifdefined#1\else
     \global\advance\lastallocatedluafunction 1
     \global\chardef#1\lastallocatedluafunction
     \directlua{gadgets.functions.reserve()}#1{\detokenize{#1}}%
   \fi}

% an example of usage (if we ever support it it will go to the plain gadgets module):
%
% \directlua {
%
%     local cct = nil
%     local chr = nil
%
%     gadgets.functions.register("UcharcatLuaOne",function()
%         chr = newtoken.scan_int()
%         cct = tex.getcatcode(chr)
%         tex.setcatcode(chr,newtoken.scan_int())
%         tex.sprint(unicode.utf8.char(chr))
%     end)
%
%     gadgets.functions.register("UcharcatLuaTwo",function()
%         tex.setcatcode(chr,cct)
%     end)
%
% }
%
% \def\Ucharcat
%   {\expandafter\expandafter\expandafter\luafunction
%    \expandafter\expandafter\expandafter\UcharcatLuaTwo
%    \luafunction\UcharcatLuaOne}
%
% A:\the\catcode65:\Ucharcat 65 11:A:\the\catcode65\par
% A:\the\catcode65:\Ucharcat 65  5:A:\the\catcode65\par
% A:\the\catcode65:\Ucharcat 65 11:A:\the\catcode65\par


\endinput
%
    %D \module
%D   [       file=luatex-fonts,
%D        version=2009.12.01,
%D          title=\LUATEX\ Support Macros,
%D       subtitle=Generic \OPENTYPE\ Font Handler,
%D         author=Hans Hagen,
%D      copyright={PRAGMA ADE \& \CONTEXT\ Development Team}]

%D \subject{Welcome}
%D
%D This file is one of a set of basic functionality enhancements
%D for \LUATEX\ derived from the \CONTEXT\ \MKIV\ code base. Please
%D don't polute the \type {luatex-*} namespace with code not coming
%D from the \CONTEXT\ development team as we may add more files.
%D
%D As this is an experimental setup, it might not always work out as
%D expected. Around \LUATEX\ version 0.50 we expect the code to be
%D more or less okay.
%D
%D This file implements a basic font system for a bare \LUATEX\
%D system. By default \LUATEX\ only knows about the classic \TFM\
%D fonts but it can read other font formats and pass them to \LUA.
%D With some glue code one can then construct a suitable \TFM\
%D representation that \LUATEX\ can work with. For more advanced font
%D support a bit more code is needed that needs to be hooked
%D into the callback mechanism.
%D
%D This file is currently rather simple: it just loads the \LUA\ file
%D with the same name. An example of a \type {luatex.tex} file that is
%D just plain \TEX:
%D
%D \starttyping
%D \catcode`\{=1 % left brace is begin-group character
%D \catcode`\}=2 % right brace is end-group character
%D
%D \input plain
%D
%D \everyjob\expandafter{\the\everyjob\input luatex-fonts\relax}
%D
%D \dump
%D \stoptyping
%D
%D We could load the \LUA\ file in \type {\everyjob} but maybe some
%D day we need more here.
%D
%D When defining a font you can use two prefixes. A \type {file:}
%D prefix forced a file search, while a \type {name:} prefix will
%D result in consulting the names database. Such a database can be
%D generated with:
%D
%D \starttyping
%D mtxrun --usekpse --script fonts --names
%D \stoptyping
%D
%D This will generate a file \type {luatex-fonts-names.lua} that has
%D to be placed in a location where it can be found by \KPSE. Beware:
%D the \type {--kpseonly} flag is only used outside \CONTEXT\ and
%D provides very limited functionality, just enough for this task.
%D
%D The code loaded here does not come out of thin air, but is mostly
%D shared with \CONTEXT, however, in that macropackage we go beyond
%D what is provided here. When you use the code packaged here you
%D need to keep a few things in mind:
%D
%D \startitemize
%D
%D \item This subsystem will be extended, improved etc. in about the
%D same pace as \CONTEXT\ \MKIV. However, because \CONTEXT\ provides a
%D rather high level of integration not all features will be supported
%D in the same quality. Use \CONTEXT\ if you want more goodies.
%D
%D \item There is no official \API\ yet, which means that using
%D functions implemented here is at your own risk, in the sense that
%D names and namespaces might change. There will be a minimal \API\
%D defined once \LUATEX\ version 1.0 is out. Instead of patching the
%D files it's better to overload functions if needed.
%D
%D \item The modules are not stripped too much, which makes it
%D possible to benefit from improvements in the code that take place
%D in the perspective of \CONTEXT\ development. They might be split a
%D bit more in due time so the baseline might become smaller.
%D
%D \item The code is maintained and tested by the \CONTEXT\
%D development team. As such it might be better suited for this macro
%D package and integration in other systems might demand some
%D additional wrapping. Problems can be reported to the team but as we
%D use \CONTEXT\ \MKIV\ as baseline, you'd better check if the problem
%D is a general \CONTEXT\ problem too.
%D
%D \item The more high level support for features that is provided in
%D \CONTEXT\ is not part of the code loaded here as it makes no sense
%D elsewhere. Some experimental features are not part of this code
%D either but some might show up later.
%D
%D \item Math font support will be added but only in its basic form
%D once that the Latin Modern and \TEX\ Gyre math fonts are
%D available.
%D
%D \item At this moment the more nifty speed-ups are not enabled
%D because they work in tandem with the alternative file handling
%D that \CONTEXT\ uses. Maybe around \LUATEX\ 1.0 we will bring some
%D speedup into this code too (if it pays off at all).
%D
%D \item The code defines a few global tables. If this code is used
%D in a larger perspective then you can best make sure that no
%D conflicts occur. The \CONTEXT\ package expects users to work in
%D their own namespace (\type {userdata}, \type {thirddata}, \type
%D {moduledata} or \type {document}. The team takes all freedom to
%D use any table at the global level but will not use tables that are
%D named after macro packages. Later the \CONTEXT\ might operate in
%D a more controlled namespace but it has a low priority.
%D
%D \item There is some tracing code present but this is not enabled
%D and not supported outside \CONTEXT\ either as it integrates quite
%D tightly into \CONTEXT. In case of problems you can use \CONTEXT\
%D for tracking down problems.
%D
%D \item Patching the code in distributions is dangerous as it might
%D fix your problem but introduce new ones for \CONTEXT. So, best keep
%D the original code as it is.
%D
%D \item Attributes are (automatically) taken from the range 127-255 so
%D you'd best not use these yourself.
%D
%D \stopitemize
%D
%D If this all sounds a bit tricky, keep in mind that it makes no sense
%D for us to maintain multiple code bases and we happen to use \CONTEXT.
%D
%D For more details about how the font subsystem works we refer to
%D publications in \TEX\ related journals, the \CONTEXT\ documentation,
%D and the \CONTEXT\ wiki.

\directlua {
    if not fonts then
        dofile(kpse.find_file("luatex-fonts.lua","tex"))
    end
}

\endinput
%
    %D \module
%D   [       file=luatex-math,
%D        version=2013.04.29,
%D          title=\LUATEX\ Support Macros,
%D       subtitle=An exmaple of math,
%D         author=Hans Hagen,
%D      copyright={PRAGMA ADE \& \CONTEXT\ Development Team}]

%D This module is in no way a complete plain math implementation. I made this file
%D because I needed it for a tutorial for (mostly) plain \TEX\ users. There are
%D several ways to support math in \LUATEX, and this is just one of them. It was the
%D quickest hack I could come up with and it stays somewhat close to the traditional
%D approach (and thereby far from the \CONTEXT\ way). This file is mainly meant for
%D Boguslaw Jackowski.

%D In the perspective of the TUG Lucida Opentype project Bruno Voisin checked the code
%D and definitions below and suggested some improvements.

% we provide a remap feature

\ifdefined\directlua
    \directlua{dofile(kpse.find_file('luatex-math.lua'))}
\fi

% a bunch of fonts:

\let \teni    = \relax
\let \seveni  = \relax
\let \fivei   = \relax
\let \tensy   = \relax
\let \sevensy = \relax
\let \fivesy  = \relax
\let \tenex   = \relax
\let \sevenbf = \relax
\let \fivebf  = \relax

\def\latinmodern
  {\font\tenrm   = file:lmroman10-regular.otf:+liga;+kern;+tlig;+trep at 10pt
   \font\sevenrm = file:lmroman7-regular.otf:+liga;+kern;+tlig;+trep  at  7pt
   \font\fiverm  = file:lmroman5-regular.otf:+liga;+kern;+tlig;+trep  at  5pt
   %
   \font\tentt   = file:lmmono10-regular.otf                               at 10pt
   \font\tensl   = file:lmromanslant10-regular.otf:+liga;+kern;+tlig;+trep at 10pt
   \font\tenit   = file:lmroman10-italic.otf:+liga;+kern;+tlig;+trep       at 10pt
   \font\tenbf   = file:lmroman10-bold.otf:+liga;+kern;+tlig;+trep         at 10pt
   \font\tenbi   = file:lmroman10-bolditalic.otf:+liga;+kern;+tlig;+trep   at 10pt
   %
   \font\tenos   = file:lmroman10-regular.otf:+onum;+liga;+kern;+tlig;+trep at 10pt
   \font\sevenos = file:lmroman7-regular.otf:+onum;+liga;+kern;+tlig;+trep  at  7pt
   \font\fiveos  = file:lmroman5-regular.otf:+onum;+liga;+kern;+tlig;+trep  at  5pt
   %
   \font\mathfonttextupright         = file:latinmodern-math.otf:script=math;ssty=0;mathsize=yes at 10pt
   \font\mathfontscriptupright       = file:latinmodern-math.otf:script=math;ssty=1;mathsize=yes at  7pt
   \font\mathfontscriptscriptupright = file:latinmodern-math.otf:script=math;ssty=2;mathsize=yes at  5pt
   %
   \textfont         0 = \mathfonttextupright
   \scriptfont       0 = \mathfontscriptupright
   \scriptscriptfont 0 = \mathfontscriptscriptupright
   %
   \textfont         1 = \tenos
   \scriptfont       1 = \sevenos
   \scriptscriptfont 1 = \fiveos
   %
   \tenrm}

\def\lucidabright
  {\font\tenrm   = file:lucidabrightot.otf:+liga;+kern;+tlig;+trep at 10pt
   \font\sevenrm = file:lucidabrightot.otf:+liga;+kern;+tlig;+trep at  7pt
   \font\fiverm  = file:lucidabrightot.otf:+liga;+kern;+tlig;+trep at  5pt
   %
   \font\tentt   = file:lucidasanstypewriterot.otf                             at 10pt
   \font\tensl   = file:lucidabrightot.otf:slant=0.167;+liga;+kern;+tlig;+trep at 10pt
   \font\tenit   = file:lucidabrightot-italic.otf:+liga;+kern;+tlig;+trep      at 10pt
   \font\tenbf   = file:lucidabrightot-demi.otf:+liga;+kern;+tlig;+trep        at 10pt
   \font\tenbi   = file:lucidabrightot-demiitalic.otf:+liga;+kern;+tlig;+trep  at 10pt
   %
   \font\tenos   = file:lucidabrightot.otf:+onum;+liga;+kern;+tlig;+trep at 10pt
   \font\sevenos = file:lucidabrightot.otf:+onum;+liga;+kern;+tlig;+trep at  7pt
   \font\fiveos  = file:lucidabrightot.otf:+onum;+liga;+kern;+tlig;+trep at  5pt
   %
   \font\mathfonttextupright         = file:lucidabrightmathot.otf:script=math;ssty=0;mathsize=yes at 10pt
   \font\mathfontscriptupright       = file:lucidabrightmathot.otf:script=math;ssty=1;mathsize=yes at  7pt
   \font\mathfontscriptscriptupright = file:lucidabrightmathot.otf:script=math;ssty=2;mathsize=yes at  5pt
   %
   \textfont         0 = \mathfonttextupright
   \scriptfont       0 = \mathfontscriptupright
   \scriptscriptfont 0 = \mathfontscriptscriptupright
   %
   \textfont         1 = \tenos
   \scriptfont       1 = \sevenos
   \scriptscriptfont 1 = \fiveos
   %
   \tenrm}

\ifdefined\directlua
    \directlua {
        if arguments["mtx:lucidabright"] then
            tex.print("\string\\lucidabright")
        else
            tex.print("\string\\latinmodern")
        end
    }
\fi

\newtoks\everymathrm
\newtoks\everymathcal
\newtoks\everymathit
\newtoks\everymathsl
\newtoks\everymathbf
\newtoks\everymathbi
\newtoks\everymathtt

% the following commands switch text as well as math

\def\rm{\fam0\relax\the\everymathrm\relax\tenrm\relax}
\def\it{\fam0\relax\the\everymathit\relax\tenit\relax}
\def\sl{\fam0\relax\the\everymathsl\relax\tensl\relax}
\def\bf{\fam0\relax\the\everymathbf\relax\tenbf\relax}
\def\bi{\fam0\relax\the\everymathbi\relax\tenbi\relax}
\def\tt{\fam0\relax\the\everymathtt\relax\tentt\relax}

% tex is fast enough for this kind of assignments:

\everymathrm {%
    % codes
    \Umathcode"0041="0"0"0041%
    \Umathcode"0042="0"0"0042%
    \Umathcode"0043="0"0"0043%
    \Umathcode"0044="0"0"0044%
    \Umathcode"0045="0"0"0045%
    \Umathcode"0046="0"0"0046%
    \Umathcode"0047="0"0"0047%
    \Umathcode"0048="0"0"0048%
    \Umathcode"0049="0"0"0049%
    \Umathcode"004A="0"0"004A%
    \Umathcode"004B="0"0"004B%
    \Umathcode"004C="0"0"004C%
    \Umathcode"004D="0"0"004D%
    \Umathcode"004E="0"0"004E%
    \Umathcode"004F="0"0"004F%
    \Umathcode"0050="0"0"0050%
    \Umathcode"0051="0"0"0051%
    \Umathcode"0052="0"0"0052%
    \Umathcode"0053="0"0"0053%
    \Umathcode"0054="0"0"0054%
    \Umathcode"0055="0"0"0055%
    \Umathcode"0056="0"0"0056%
    \Umathcode"0057="0"0"0057%
    \Umathcode"0058="0"0"0058%
    \Umathcode"0059="0"0"0059%
    \Umathcode"005A="0"0"005A%
    \Umathcode"0061="0"0"0061%
    \Umathcode"0062="0"0"0062%
    \Umathcode"0063="0"0"0063%
    \Umathcode"0064="0"0"0064%
    \Umathcode"0065="0"0"0065%
    \Umathcode"0066="0"0"0066%
    \Umathcode"0067="0"0"0067%
    \Umathcode"0068="0"0"0068%
    \Umathcode"0069="0"0"0069%
    \Umathcode"006A="0"0"006A%
    \Umathcode"006B="0"0"006B%
    \Umathcode"006C="0"0"006C%
    \Umathcode"006D="0"0"006D%
    \Umathcode"006E="0"0"006E%
    \Umathcode"006F="0"0"006F%
    \Umathcode"0070="0"0"0070%
    \Umathcode"0071="0"0"0071%
    \Umathcode"0072="0"0"0072%
    \Umathcode"0073="0"0"0073%
    \Umathcode"0074="0"0"0074%
    \Umathcode"0075="0"0"0075%
    \Umathcode"0076="0"0"0076%
    \Umathcode"0077="0"0"0077%
    \Umathcode"0078="0"0"0078%
    \Umathcode"0079="0"0"0079%
    \Umathcode"007A="0"0"007A%
    \Umathcode"0391="0"0"0391%
    \Umathcode"0392="0"0"0392%
    \Umathcode"0393="0"0"0393%
    \Umathcode"0394="0"0"0394%
    \Umathcode"0395="0"0"0395%
    \Umathcode"0396="0"0"0396%
    \Umathcode"0397="0"0"0397%
    \Umathcode"0398="0"0"0398%
    \Umathcode"0399="0"0"0399%
    \Umathcode"039A="0"0"039A%
    \Umathcode"039B="0"0"039B%
    \Umathcode"039C="0"0"039C%
    \Umathcode"039D="0"0"039D%
    \Umathcode"039E="0"0"039E%
    \Umathcode"039F="0"0"039F%
    \Umathcode"03A0="0"0"03A0%
    \Umathcode"03A1="0"0"03A1%
    \Umathcode"03A3="0"0"03A3%
    \Umathcode"03A4="0"0"03A4%
    \Umathcode"03A5="0"0"03A5%
    \Umathcode"03A6="0"0"03A6%
    \Umathcode"03A7="0"0"03A7%
    \Umathcode"03A8="0"0"03A8%
    \Umathcode"03A9="0"0"03A9%
    \Umathcode"03B1="0"0"03B1%
    \Umathcode"03B2="0"0"03B2%
    \Umathcode"03B3="0"0"03B3%
    \Umathcode"03B4="0"0"03B4%
    \Umathcode"03B5="0"0"03B5%
    \Umathcode"03B6="0"0"03B6%
    \Umathcode"03B7="0"0"03B7%
    \Umathcode"03B8="0"0"03B8%
    \Umathcode"03B9="0"0"03B9%
    \Umathcode"03BA="0"0"03BA%
    \Umathcode"03BB="0"0"03BB%
    \Umathcode"03BC="0"0"03BC%
    \Umathcode"03BD="0"0"03BD%
    \Umathcode"03BE="0"0"03BE%
    \Umathcode"03BF="0"0"03BF%
    \Umathcode"03C0="0"0"03C0%
    \Umathcode"03C1="0"0"03C1%
    \Umathcode"03C2="0"0"03C2%
    \Umathcode"03C3="0"0"03C3%
    \Umathcode"03C4="0"0"03C4%
    \Umathcode"03C5="0"0"03C5%
    \Umathcode"03C6="0"0"03C6%
    \Umathcode"03C7="0"0"03C7%
    \Umathcode"03C8="0"0"03C8%
    \Umathcode"03C9="0"0"03C9%
    \Umathcode"03D1="0"0"03D1%
    \Umathcode"03D5="0"0"03D5%
    \Umathcode"03D6="0"0"03D6%
    \Umathcode"03F0="0"0"03F0%
    \Umathcode"03F1="0"0"03F1%
    \Umathcode"03F4="0"0"03F4%
    \Umathcode"03F5="0"0"03F5%
    \Umathcode"2202="0"0"2202%
    \Umathcode"2207="0"0"2207%
    % commands
    \Umathchardef\Alpha     "0"0"000391%
    \Umathchardef\Beta      "0"0"000392%
    \Umathchardef\Gamma     "0"0"000393%
    \Umathchardef\Delta     "0"0"000394%
    \Umathchardef\Epsilon   "0"0"000395%
    \Umathchardef\Zeta      "0"0"000396%
    \Umathchardef\Eta       "0"0"000397%
    \Umathchardef\Theta     "0"0"000398%
    \Umathchardef\Iota      "0"0"000399%
    \Umathchardef\Kappa     "0"0"00039A%
    \Umathchardef\Lambda    "0"0"00039B%
    \Umathchardef\Mu        "0"0"00039C%
    \Umathchardef\Nu        "0"0"00039D%
    \Umathchardef\Xi        "0"0"00039E%
    \Umathchardef\Omicron   "0"0"00039F%
    \Umathchardef\Pi        "0"0"0003A0%
    \Umathchardef\Rho       "0"0"0003A1%
    \Umathchardef\Sigma     "0"0"0003A3%
    \Umathchardef\Tau       "0"0"0003A4%
    \Umathchardef\Upsilon   "0"0"0003A5%
    \Umathchardef\Phi       "0"0"0003A6%
    \Umathchardef\Chi       "0"0"0003A7%
    \Umathchardef\Psi       "0"0"0003A8%
    \Umathchardef\Omega     "0"0"0003A9%
    \Umathchardef\alpha     "0"0"0003B1%
    \Umathchardef\beta      "0"0"0003B2%
    \Umathchardef\gamma     "0"0"0003B3%
    \Umathchardef\delta     "0"0"0003B4%
    \Umathchardef\varepsilon"0"0"0003B5%
    \Umathchardef\zeta      "0"0"0003B6%
    \Umathchardef\eta       "0"0"0003B7%
    \Umathchardef\theta     "0"0"0003B8%
    \Umathchardef\iota      "0"0"0003B9%
    \Umathchardef\kappa     "0"0"0003BA%
    \Umathchardef\lambda    "0"0"0003BB%
    \Umathchardef\mu        "0"0"0003BC%
    \Umathchardef\nu        "0"0"0003BD%
    \Umathchardef\xi        "0"0"0003BE%
    \Umathchardef\omicron   "0"0"0003BF%
    \Umathchardef\pi        "0"0"0003C0%
    \Umathchardef\rho       "0"0"0003C1%
    \Umathchardef\varsigma  "0"0"0003C2%
    \Umathchardef\sigma     "0"0"0003C3%
    \Umathchardef\tau       "0"0"0003C4%
    \Umathchardef\upsilon   "0"0"0003C5%
    \Umathchardef\varphi    "0"0"0003C6%
    \Umathchardef\chi       "0"0"0003C7%
    \Umathchardef\psi       "0"0"0003C8%
    \Umathchardef\omega     "0"0"0003C9%
    \Umathchardef\vartheta  "0"0"0003D1%
    \Umathchardef\phi       "0"0"0003D5%
    \Umathchardef\varpi     "0"0"0003D6%
    \Umathchardef\varkappa  "0"0"0003F0%
    \Umathchardef\varrho    "0"0"0003F1%
    \Umathchardef\epsilon   "0"0"0003F5%
    \Umathchardef\varTheta  "0"0"0003F4%
    \Umathchardef\digamma   "0"0"0003DC%
    \relax
}

\everymathit {%
    % codes
    \Umathcode"0041="0"0"1D434%
    \Umathcode"0042="0"0"1D435%
    \Umathcode"0043="0"0"1D436%
    \Umathcode"0044="0"0"1D437%
    \Umathcode"0045="0"0"1D438%
    \Umathcode"0046="0"0"1D439%
    \Umathcode"0047="0"0"1D43A%
    \Umathcode"0048="0"0"1D43B%
    \Umathcode"0049="0"0"1D43C%
    \Umathcode"004A="0"0"1D43D%
    \Umathcode"004B="0"0"1D43E%
    \Umathcode"004C="0"0"1D43F%
    \Umathcode"004D="0"0"1D440%
    \Umathcode"004E="0"0"1D441%
    \Umathcode"004F="0"0"1D442%
    \Umathcode"0050="0"0"1D443%
    \Umathcode"0051="0"0"1D444%
    \Umathcode"0052="0"0"1D445%
    \Umathcode"0053="0"0"1D446%
    \Umathcode"0054="0"0"1D447%
    \Umathcode"0055="0"0"1D448%
    \Umathcode"0056="0"0"1D449%
    \Umathcode"0057="0"0"1D44A%
    \Umathcode"0058="0"0"1D44B%
    \Umathcode"0059="0"0"1D44C%
    \Umathcode"005A="0"0"1D44D%
    \Umathcode"0061="0"0"1D44E%
    \Umathcode"0062="0"0"1D44F%
    \Umathcode"0063="0"0"1D450%
    \Umathcode"0064="0"0"1D451%
    \Umathcode"0065="0"0"1D452%
    \Umathcode"0066="0"0"1D453%
    \Umathcode"0067="0"0"1D454%
    \Umathcode"0068="0"0"0210E%
    \Umathcode"0069="0"0"1D456%
    \Umathcode"006A="0"0"1D457%
    \Umathcode"006B="0"0"1D458%
    \Umathcode"006C="0"0"1D459%
    \Umathcode"006D="0"0"1D45A%
    \Umathcode"006E="0"0"1D45B%
    \Umathcode"006F="0"0"1D45C%
    \Umathcode"0070="0"0"1D45D%
    \Umathcode"0071="0"0"1D45E%
    \Umathcode"0072="0"0"1D45F%
    \Umathcode"0073="0"0"1D460%
    \Umathcode"0074="0"0"1D461%
    \Umathcode"0075="0"0"1D462%
    \Umathcode"0076="0"0"1D463%
    \Umathcode"0077="0"0"1D464%
    \Umathcode"0078="0"0"1D465%
    \Umathcode"0079="0"0"1D466%
    \Umathcode"007A="0"0"1D467%
    \Umathcode"0391="0"0"1D6E2%
    \Umathcode"0392="0"0"1D6E3%
    \Umathcode"0393="0"0"1D6E4%
    \Umathcode"0394="0"0"1D6E5%
    \Umathcode"0395="0"0"1D6E6%
    \Umathcode"0396="0"0"1D6E7%
    \Umathcode"0397="0"0"1D6E8%
    \Umathcode"0398="0"0"1D6E9%
    \Umathcode"0399="0"0"1D6EA%
    \Umathcode"039A="0"0"1D6EB%
    \Umathcode"039B="0"0"1D6EC%
    \Umathcode"039C="0"0"1D6ED%
    \Umathcode"039D="0"0"1D6EE%
    \Umathcode"039E="0"0"1D6EF%
    \Umathcode"039F="0"0"1D6F0%
    \Umathcode"03A0="0"0"1D6F1%
    \Umathcode"03A1="0"0"1D6F2%
    \Umathcode"03A3="0"0"1D6F4%
    \Umathcode"03A4="0"0"1D6F5%
    \Umathcode"03A5="0"0"1D6F6%
    \Umathcode"03A6="0"0"1D6F7%
    \Umathcode"03A7="0"0"1D6F8%
    \Umathcode"03A8="0"0"1D6F9%
    \Umathcode"03A9="0"0"1D6FA%
    \Umathcode"03B1="0"0"1D6FC%
    \Umathcode"03B2="0"0"1D6FD%
    \Umathcode"03B3="0"0"1D6FE%
    \Umathcode"03B4="0"0"1D6FF%
    \Umathcode"03B5="0"0"1D700%
    \Umathcode"03B6="0"0"1D701%
    \Umathcode"03B7="0"0"1D702%
    \Umathcode"03B8="0"0"1D703%
    \Umathcode"03B9="0"0"1D704%
    \Umathcode"03BA="0"0"1D705%
    \Umathcode"03BB="0"0"1D706%
    \Umathcode"03BC="0"0"1D707%
    \Umathcode"03BD="0"0"1D708%
    \Umathcode"03BE="0"0"1D709%
    \Umathcode"03BF="0"0"1D70A%
    \Umathcode"03C0="0"0"1D70B%
    \Umathcode"03C1="0"0"1D70C%
    \Umathcode"03C2="0"0"1D70D%
    \Umathcode"03C3="0"0"1D70E%
    \Umathcode"03C4="0"0"1D70F%
    \Umathcode"03C5="0"0"1D710%
    \Umathcode"03C6="0"0"1D711%
    \Umathcode"03C7="0"0"1D712%
    \Umathcode"03C8="0"0"1D713%
    \Umathcode"03C9="0"0"1D714%
    \Umathcode"03D1="0"0"1D717%
    \Umathcode"03D5="0"0"1D719%
    \Umathcode"03D6="0"0"1D71B%
    \Umathcode"03F0="0"0"1D718%
    \Umathcode"03F1="0"0"1D71A%
    \Umathcode"03F4="0"0"1D6F3%
    \Umathcode"03F5="0"0"1D716%
    \Umathcode"2202="0"0"1D715%
    \Umathcode"2207="0"0"1D6FB%
    % commands
    \Umathchardef\Alpha     "0"0"01D6E2%
    \Umathchardef\Beta      "0"0"01D6E3%
    \Umathchardef\Gamma     "0"0"01D6E4%
    \Umathchardef\Delta     "0"0"01D6E5%
    \Umathchardef\Epsilon   "0"0"01D6E6%
    \Umathchardef\Zeta      "0"0"01D6E7%
    \Umathchardef\Eta       "0"0"01D6E8%
    \Umathchardef\Theta     "0"0"01D6E9%
    \Umathchardef\Iota      "0"0"01D6EA%
    \Umathchardef\Kappa     "0"0"01D6EB%
    \Umathchardef\Lambda    "0"0"01D6EC%
    \Umathchardef\Mu        "0"0"01D6ED%
    \Umathchardef\Nu        "0"0"01D6EE%
    \Umathchardef\Xi        "0"0"01D6EF%
    \Umathchardef\Omicron   "0"0"01D6F0%
    \Umathchardef\Pi        "0"0"01D6F1%
    \Umathchardef\Rho       "0"0"01D6F2%
    \Umathchardef\Sigma     "0"0"01D6F4%
    \Umathchardef\Tau       "0"0"01D6F5%
    \Umathchardef\Upsilon   "0"0"01D6F6%
    \Umathchardef\Phi       "0"0"01D6F7%
    \Umathchardef\Chi       "0"0"01D6F8%
    \Umathchardef\Psi       "0"0"01D6F9%
    \Umathchardef\Omega     "0"0"01D6FA%
    \Umathchardef\alpha     "0"0"01D6FC%
    \Umathchardef\beta      "0"0"01D6FD%
    \Umathchardef\gamma     "0"0"01D6FE%
    \Umathchardef\delta     "0"0"01D6FF%
    \Umathchardef\varepsilon"0"0"01D700%
    \Umathchardef\zeta      "0"0"01D701%
    \Umathchardef\eta       "0"0"01D702%
    \Umathchardef\theta     "0"0"01D703%
    \Umathchardef\iota      "0"0"01D704%
    \Umathchardef\kappa     "0"0"01D705%
    \Umathchardef\lambda    "0"0"01D706%
    \Umathchardef\mu        "0"0"01D707%
    \Umathchardef\nu        "0"0"01D708%
    \Umathchardef\xi        "0"0"01D709%
    \Umathchardef\omicron   "0"0"01D70A%
    \Umathchardef\pi        "0"0"01D70B%
    \Umathchardef\rho       "0"0"01D70C%
    \Umathchardef\varsigma  "0"0"01D70D%
    \Umathchardef\sigma     "0"0"01D70E%
    \Umathchardef\tau       "0"0"01D70F%
    \Umathchardef\upsilon   "0"0"01D710%
    \Umathchardef\varphi    "0"0"01D711%
    \Umathchardef\chi       "0"0"01D712%
    \Umathchardef\psi       "0"0"01D713%
    \Umathchardef\omega     "0"0"01D714%
    \Umathchardef\epsilon   "0"0"01D716%
    \Umathchardef\vartheta  "0"0"01D717%
    \Umathchardef\varkappa  "0"0"01D718%
    \Umathchardef\phi       "0"0"01D719%
    \Umathchardef\varrho    "0"0"01D71A%
    \Umathchardef\varpi     "0"0"01D71B%
    \Umathchardef\varTheta  "0"0"01D6F3%
    \Umathchardef\digamma   "0"0"0003DC%
    \relax
}

\everymathsl {%
    \the\everymathit
}

\everymathbf {%
    % codes
    \Umathcode"0030="0"0"1D7CE%
    \Umathcode"0031="0"0"1D7CF%
    \Umathcode"0032="0"0"1D7D0%
    \Umathcode"0033="0"0"1D7D1%
    \Umathcode"0034="0"0"1D7D2%
    \Umathcode"0035="0"0"1D7D3%
    \Umathcode"0036="0"0"1D7D4%
    \Umathcode"0037="0"0"1D7D5%
    \Umathcode"0038="0"0"1D7D6%
    \Umathcode"0039="0"0"1D7D7%
    \Umathcode"0041="0"0"1D400%
    \Umathcode"0042="0"0"1D401%
    \Umathcode"0043="0"0"1D402%
    \Umathcode"0044="0"0"1D403%
    \Umathcode"0045="0"0"1D404%
    \Umathcode"0046="0"0"1D405%
    \Umathcode"0047="0"0"1D406%
    \Umathcode"0048="0"0"1D407%
    \Umathcode"0049="0"0"1D408%
    \Umathcode"004A="0"0"1D409%
    \Umathcode"004B="0"0"1D40A%
    \Umathcode"004C="0"0"1D40B%
    \Umathcode"004D="0"0"1D40C%
    \Umathcode"004E="0"0"1D40D%
    \Umathcode"004F="0"0"1D40E%
    \Umathcode"0050="0"0"1D40F%
    \Umathcode"0051="0"0"1D410%
    \Umathcode"0052="0"0"1D411%
    \Umathcode"0053="0"0"1D412%
    \Umathcode"0054="0"0"1D413%
    \Umathcode"0055="0"0"1D414%
    \Umathcode"0056="0"0"1D415%
    \Umathcode"0057="0"0"1D416%
    \Umathcode"0058="0"0"1D417%
    \Umathcode"0059="0"0"1D418%
    \Umathcode"005A="0"0"1D419%
    \Umathcode"0061="0"0"1D41A%
    \Umathcode"0062="0"0"1D41B%
    \Umathcode"0063="0"0"1D41C%
    \Umathcode"0064="0"0"1D41D%
    \Umathcode"0065="0"0"1D41E%
    \Umathcode"0066="0"0"1D41F%
    \Umathcode"0067="0"0"1D420%
    \Umathcode"0068="0"0"1D421%
    \Umathcode"0069="0"0"1D422%
    \Umathcode"006A="0"0"1D423%
    \Umathcode"006B="0"0"1D424%
    \Umathcode"006C="0"0"1D425%
    \Umathcode"006D="0"0"1D426%
    \Umathcode"006E="0"0"1D427%
    \Umathcode"006F="0"0"1D428%
    \Umathcode"0070="0"0"1D429%
    \Umathcode"0071="0"0"1D42A%
    \Umathcode"0072="0"0"1D42B%
    \Umathcode"0073="0"0"1D42C%
    \Umathcode"0074="0"0"1D42D%
    \Umathcode"0075="0"0"1D42E%
    \Umathcode"0076="0"0"1D42F%
    \Umathcode"0077="0"0"1D430%
    \Umathcode"0078="0"0"1D431%
    \Umathcode"0079="0"0"1D432%
    \Umathcode"007A="0"0"1D433%
    \Umathcode"0391="0"0"1D6A8%
    \Umathcode"0392="0"0"1D6A9%
    \Umathcode"0393="0"0"1D6AA%
    \Umathcode"0394="0"0"1D6AB%
    \Umathcode"0395="0"0"1D6AC%
    \Umathcode"0396="0"0"1D6AD%
    \Umathcode"0397="0"0"1D6AE%
    \Umathcode"0398="0"0"1D6AF%
    \Umathcode"0399="0"0"1D6B0%
    \Umathcode"039A="0"0"1D6B1%
    \Umathcode"039B="0"0"1D6B2%
    \Umathcode"039C="0"0"1D6B3%
    \Umathcode"039D="0"0"1D6B4%
    \Umathcode"039E="0"0"1D6B5%
    \Umathcode"039F="0"0"1D6B6%
    \Umathcode"03A0="0"0"1D6B7%
    \Umathcode"03A1="0"0"1D6B8%
    \Umathcode"03A3="0"0"1D6BA%
    \Umathcode"03A4="0"0"1D6BB%
    \Umathcode"03A5="0"0"1D6BC%
    \Umathcode"03A6="0"0"1D6BD%
    \Umathcode"03A7="0"0"1D6BE%
    \Umathcode"03A8="0"0"1D6BF%
    \Umathcode"03A9="0"0"1D6C0%
    \Umathcode"03B1="0"0"1D6C2%
    \Umathcode"03B2="0"0"1D6C3%
    \Umathcode"03B3="0"0"1D6C4%
    \Umathcode"03B4="0"0"1D6C5%
    \Umathcode"03B5="0"0"1D6C6%
    \Umathcode"03B6="0"0"1D6C7%
    \Umathcode"03B7="0"0"1D6C8%
    \Umathcode"03B8="0"0"1D6C9%
    \Umathcode"03B9="0"0"1D6CA%
    \Umathcode"03BA="0"0"1D6CB%
    \Umathcode"03BB="0"0"1D6CC%
    \Umathcode"03BC="0"0"1D6CD%
    \Umathcode"03BD="0"0"1D6CE%
    \Umathcode"03BE="0"0"1D6CF%
    \Umathcode"03BF="0"0"1D6D0%
    \Umathcode"03C0="0"0"1D6D1%
    \Umathcode"03C1="0"0"1D6D2%
    \Umathcode"03C2="0"0"1D6D3%
    \Umathcode"03C3="0"0"1D6D4%
    \Umathcode"03C4="0"0"1D6D5%
    \Umathcode"03C5="0"0"1D6D6%
    \Umathcode"03C6="0"0"1D6D7%
    \Umathcode"03C7="0"0"1D6D8%
    \Umathcode"03C8="0"0"1D6D9%
    \Umathcode"03C9="0"0"1D6DA%
    \Umathcode"03D1="0"0"1D6DD%
    \Umathcode"03D5="0"0"1D6DF%
    \Umathcode"03D6="0"0"1D6E1%
    \Umathcode"03F0="0"0"1D6DE%
    \Umathcode"03F1="0"0"1D6E0%
    \Umathcode"03F4="0"0"1D6B9%
    \Umathcode"03F5="0"0"1D6DC%
    \Umathcode"2202="0"0"1D6DB%
    \Umathcode"2207="0"0"1D6C1%
    % commands
    \Umathchardef\Alpha     "0"0"01D6A8%
    \Umathchardef\Beta      "0"0"01D6A9%
    \Umathchardef\Gamma     "0"0"01D6AA%
    \Umathchardef\Delta     "0"0"01D6AB%
    \Umathchardef\Epsilon   "0"0"01D6AC%
    \Umathchardef\Zeta      "0"0"01D6AD%
    \Umathchardef\Eta       "0"0"01D6AE%
    \Umathchardef\Theta     "0"0"01D6AF%
    \Umathchardef\Iota      "0"0"01D6B0%
    \Umathchardef\Kappa     "0"0"01D6B1%
    \Umathchardef\Lambda    "0"0"01D6B2%
    \Umathchardef\Mu        "0"0"01D6B3%
    \Umathchardef\Nu        "0"0"01D6B4%
    \Umathchardef\Xi        "0"0"01D6B5%
    \Umathchardef\Omicron   "0"0"01D6B6%
    \Umathchardef\Pi        "0"0"01D6B7%
    \Umathchardef\Rho       "0"0"01D6B8%
    \Umathchardef\Sigma     "0"0"01D6BA%
    \Umathchardef\Tau       "0"0"01D6BB%
    \Umathchardef\Upsilon   "0"0"01D6BC%
    \Umathchardef\Phi       "0"0"01D6BD%
    \Umathchardef\Chi       "0"0"01D6BE%
    \Umathchardef\Psi       "0"0"01D6BF%
    \Umathchardef\Omega     "0"0"01D6C0%
    \Umathchardef\alpha     "0"0"01D6C2%
    \Umathchardef\beta      "0"0"01D6C3%
    \Umathchardef\gamma     "0"0"01D6C4%
    \Umathchardef\delta     "0"0"01D6C5%
    \Umathchardef\varepsilon"0"0"01D6C6%
    \Umathchardef\zeta      "0"0"01D6C7%
    \Umathchardef\eta       "0"0"01D6C8%
    \Umathchardef\theta     "0"0"01D6C9%
    \Umathchardef\iota      "0"0"01D6CA%
    \Umathchardef\kappa     "0"0"01D6CB%
    \Umathchardef\lambda    "0"0"01D6CC%
    \Umathchardef\mu        "0"0"01D6CD%
    \Umathchardef\nu        "0"0"01D6CE%
    \Umathchardef\xi        "0"0"01D6CF%
    \Umathchardef\omicron   "0"0"01D6D0%
    \Umathchardef\pi        "0"0"01D6D1%
    \Umathchardef\rho       "0"0"01D6D2%
    \Umathchardef\varsigma  "0"0"01D6D3%
    \Umathchardef\sigma     "0"0"01D6D4%
    \Umathchardef\tau       "0"0"01D6D5%
    \Umathchardef\upsilon   "0"0"01D6D6%
    \Umathchardef\varphi    "0"0"01D6D7%
    \Umathchardef\chi       "0"0"01D6D8%
    \Umathchardef\psi       "0"0"01D6D9%
    \Umathchardef\omega     "0"0"01D6DA%
    \Umathchardef\epsilon   "0"0"01D6DC%
    \Umathchardef\vartheta  "0"0"01D6DD%
    \Umathchardef\varkappa  "0"0"01D6DE%
    \Umathchardef\phi       "0"0"01D6DF%
    \Umathchardef\varrho    "0"0"01D6E0%
    \Umathchardef\varpi     "0"0"01D6E1%
    \Umathchardef\varTheta  "0"0"01D6B9%
    \Umathchardef\digamma   "0"0"01D7CA%
    \relax
}

\everymathbi {%
    % codes
    \Umathcode"0030="0"0"1D7CE%
    \Umathcode"0031="0"0"1D7CF%
    \Umathcode"0032="0"0"1D7D0%
    \Umathcode"0033="0"0"1D7D1%
    \Umathcode"0034="0"0"1D7D2%
    \Umathcode"0035="0"0"1D7D3%
    \Umathcode"0036="0"0"1D7D4%
    \Umathcode"0037="0"0"1D7D5%
    \Umathcode"0038="0"0"1D7D6%
    \Umathcode"0039="0"0"1D7D7%
    \Umathcode"0041="0"0"1D468%
    \Umathcode"0042="0"0"1D469%
    \Umathcode"0043="0"0"1D46A%
    \Umathcode"0044="0"0"1D46B%
    \Umathcode"0045="0"0"1D46C%
    \Umathcode"0046="0"0"1D46D%
    \Umathcode"0047="0"0"1D46E%
    \Umathcode"0048="0"0"1D46F%
    \Umathcode"0049="0"0"1D470%
    \Umathcode"004A="0"0"1D471%
    \Umathcode"004B="0"0"1D472%
    \Umathcode"004C="0"0"1D473%
    \Umathcode"004D="0"0"1D474%
    \Umathcode"004E="0"0"1D475%
    \Umathcode"004F="0"0"1D476%
    \Umathcode"0050="0"0"1D477%
    \Umathcode"0051="0"0"1D478%
    \Umathcode"0052="0"0"1D479%
    \Umathcode"0053="0"0"1D47A%
    \Umathcode"0054="0"0"1D47B%
    \Umathcode"0055="0"0"1D47C%
    \Umathcode"0056="0"0"1D47D%
    \Umathcode"0057="0"0"1D47E%
    \Umathcode"0058="0"0"1D47F%
    \Umathcode"0059="0"0"1D480%
    \Umathcode"005A="0"0"1D481%
    \Umathcode"0061="0"0"1D482%
    \Umathcode"0062="0"0"1D483%
    \Umathcode"0063="0"0"1D484%
    \Umathcode"0064="0"0"1D485%
    \Umathcode"0065="0"0"1D486%
    \Umathcode"0066="0"0"1D487%
    \Umathcode"0067="0"0"1D488%
    \Umathcode"0068="0"0"1D489%
    \Umathcode"0069="0"0"1D48A%
    \Umathcode"006A="0"0"1D48B%
    \Umathcode"006B="0"0"1D48C%
    \Umathcode"006C="0"0"1D48D%
    \Umathcode"006D="0"0"1D48E%
    \Umathcode"006E="0"0"1D48F%
    \Umathcode"006F="0"0"1D490%
    \Umathcode"0070="0"0"1D491%
    \Umathcode"0071="0"0"1D492%
    \Umathcode"0072="0"0"1D493%
    \Umathcode"0073="0"0"1D494%
    \Umathcode"0074="0"0"1D495%
    \Umathcode"0075="0"0"1D496%
    \Umathcode"0076="0"0"1D497%
    \Umathcode"0077="0"0"1D498%
    \Umathcode"0078="0"0"1D499%
    \Umathcode"0079="0"0"1D49A%
    \Umathcode"007A="0"0"1D49B%
    \Umathcode"0391="0"0"1D71C%
    \Umathcode"0392="0"0"1D71D%
    \Umathcode"0393="0"0"1D71E%
    \Umathcode"0394="0"0"1D71F%
    \Umathcode"0395="0"0"1D720%
    \Umathcode"0396="0"0"1D721%
    \Umathcode"0397="0"0"1D722%
    \Umathcode"0398="0"0"1D723%
    \Umathcode"0399="0"0"1D724%
    \Umathcode"039A="0"0"1D725%
    \Umathcode"039B="0"0"1D726%
    \Umathcode"039C="0"0"1D727%
    \Umathcode"039D="0"0"1D728%
    \Umathcode"039E="0"0"1D729%
    \Umathcode"039F="0"0"1D72A%
    \Umathcode"03A0="0"0"1D72B%
    \Umathcode"03A1="0"0"1D72C%
    \Umathcode"03A3="0"0"1D72E%
    \Umathcode"03A4="0"0"1D72F%
    \Umathcode"03A5="0"0"1D730%
    \Umathcode"03A6="0"0"1D731%
    \Umathcode"03A7="0"0"1D732%
    \Umathcode"03A8="0"0"1D733%
    \Umathcode"03A9="0"0"1D734%
    \Umathcode"03B1="0"0"1D736%
    \Umathcode"03B2="0"0"1D737%
    \Umathcode"03B3="0"0"1D738%
    \Umathcode"03B4="0"0"1D739%
    \Umathcode"03B5="0"0"1D73A%
    \Umathcode"03B6="0"0"1D73B%
    \Umathcode"03B7="0"0"1D73C%
    \Umathcode"03B8="0"0"1D73D%
    \Umathcode"03B9="0"0"1D73E%
    \Umathcode"03BA="0"0"1D73F%
    \Umathcode"03BB="0"0"1D740%
    \Umathcode"03BC="0"0"1D741%
    \Umathcode"03BD="0"0"1D742%
    \Umathcode"03BE="0"0"1D743%
    \Umathcode"03BF="0"0"1D744%
    \Umathcode"03C0="0"0"1D745%
    \Umathcode"03C1="0"0"1D746%
    \Umathcode"03C2="0"0"1D747%
    \Umathcode"03C3="0"0"1D748%
    \Umathcode"03C4="0"0"1D749%
    \Umathcode"03C5="0"0"1D74A%
    \Umathcode"03C6="0"0"1D74B%
    \Umathcode"03C7="0"0"1D74C%
    \Umathcode"03C8="0"0"1D74D%
    \Umathcode"03C9="0"0"1D74E%
    \Umathcode"03D1="0"0"1D751%
    \Umathcode"03D5="0"0"1D753%
    \Umathcode"03D6="0"0"1D755%
    \Umathcode"03F0="0"0"1D752%
    \Umathcode"03F1="0"0"1D754%
    \Umathcode"03F4="0"0"1D72D%
    \Umathcode"03F5="0"0"1D750%
    \Umathcode"2202="0"0"1D74F%
    \Umathcode"2207="0"0"1D735%
    % commands
    \Umathchardef\Alpha     "0"0"01D71C%
    \Umathchardef\Beta      "0"0"01D71D%
    \Umathchardef\Gamma     "0"0"01D71E%
    \Umathchardef\Delta     "0"0"01D71F%
    \Umathchardef\Epsilon   "0"0"01D720%
    \Umathchardef\Zeta      "0"0"01D721%
    \Umathchardef\Eta       "0"0"01D722%
    \Umathchardef\Theta     "0"0"01D723%
    \Umathchardef\Iota      "0"0"01D724%
    \Umathchardef\Kappa     "0"0"01D725%
    \Umathchardef\Lambda    "0"0"01D726%
    \Umathchardef\Mu        "0"0"01D727%
    \Umathchardef\Nu        "0"0"01D728%
    \Umathchardef\Xi        "0"0"01D729%
    \Umathchardef\Omicron   "0"0"01D72A%
    \Umathchardef\Pi        "0"0"01D72B%
    \Umathchardef\Rho       "0"0"01D72C%
    \Umathchardef\Sigma     "0"0"01D72E%
    \Umathchardef\Tau       "0"0"01D72F%
    \Umathchardef\Upsilon   "0"0"01D730%
    \Umathchardef\Phi       "0"0"01D731%
    \Umathchardef\Chi       "0"0"01D732%
    \Umathchardef\Psi       "0"0"01D733%
    \Umathchardef\Omega     "0"0"01D734%
    \Umathchardef\alpha     "0"0"01D736%
    \Umathchardef\beta      "0"0"01D737%
    \Umathchardef\gamma     "0"0"01D738%
    \Umathchardef\delta     "0"0"01D739%
    \Umathchardef\varepsilon"0"0"01D73A%
    \Umathchardef\zeta      "0"0"01D73B%
    \Umathchardef\eta       "0"0"01D73C%
    \Umathchardef\theta     "0"0"01D73D%
    \Umathchardef\iota      "0"0"01D73E%
    \Umathchardef\kappa     "0"0"01D73F%
    \Umathchardef\lambda    "0"0"01D740%
    \Umathchardef\mu        "0"0"01D741%
    \Umathchardef\nu        "0"0"01D742%
    \Umathchardef\xi        "0"0"01D743%
    \Umathchardef\omicron   "0"0"01D744%
    \Umathchardef\pi        "0"0"01D745%
    \Umathchardef\rho       "0"0"01D746%
    \Umathchardef\varsigma  "0"0"01D747%
    \Umathchardef\sigma     "0"0"01D748%
    \Umathchardef\tau       "0"0"01D749%
    \Umathchardef\upsilon   "0"0"01D74A%
    \Umathchardef\varphi    "0"0"01D74B%
    \Umathchardef\chi       "0"0"01D74C%
    \Umathchardef\psi       "0"0"01D74D%
    \Umathchardef\omega     "0"0"01D74E%
    \Umathchardef\epsilon   "0"0"01D750%
    \Umathchardef\vartheta  "0"0"01D751%
    \Umathchardef\varkappa  "0"0"01D752%
    \Umathchardef\phi       "0"0"01D753%
    \Umathchardef\varrho    "0"0"01D754%
    \Umathchardef\varpi     "0"0"01D755%
    \Umathchardef\varTheta  "0"0"01D72D%
    \Umathchardef\digamma   "0"0"01D7CA%
    \relax
}

\everymathtt {%
    % not done
}

% Nothing special here:

\let\mit\it

% We use a special family for this, not that oldstyle in math makes
% much sense, it's more that in good old tex oldstyle was taken from
% math fonts. So, just something compatible:

\def\oldstyle{\fam1\relax\tenos\relax}

% Again a text and math one and it had better be used grouped.

\def\cal{\fam0\relax\the\everymathcal\relax\tenit\relax}

\everymathcal {%
    \Umathcode"0041="0"0"1D49C% A
    \Umathcode"0042="0"0"0212C% B
    \Umathcode"0043="0"0"1D49E% C
    \Umathcode"0044="0"0"1D49F% D
    \Umathcode"0045="0"0"02130% E
    \Umathcode"0046="0"0"02131% F
    \Umathcode"0047="0"0"1D4A2% G
    \Umathcode"0048="0"0"0210B% H
    \Umathcode"0049="0"0"02110% I
    \Umathcode"004A="0"0"1D4A5% J
    \Umathcode"004B="0"0"1D4A6% K
    \Umathcode"004C="0"0"02112% L
    \Umathcode"004D="0"0"02133% M
    \Umathcode"004E="0"0"1D4A9% N
    \Umathcode"004F="0"0"1D4AA% O
    \Umathcode"0050="0"0"1D4AB% P
    \Umathcode"0051="0"0"1D4AC% Q
    \Umathcode"0052="0"0"0211B% R
    \Umathcode"0053="0"0"1D4AE% S
    \Umathcode"0054="0"0"1D4AF% T
    \Umathcode"0055="0"0"1D4B0% U
    \Umathcode"0056="0"0"1D4B1% V
    \Umathcode"0057="0"0"1D4B2% W
    \Umathcode"0058="0"0"1D4B3% X
    \Umathcode"0059="0"0"1D4B4% Y
    \Umathcode"005A="0"0"1D4B5% Z
    \Umathcode"0061="0"0"1D4B6% a
    \Umathcode"0062="0"0"1D4B7% b
    \Umathcode"0063="0"0"1D4B8% c
    \Umathcode"0064="0"0"1D4B9% d
    \Umathcode"0065="0"0"0212F% e
    \Umathcode"0066="0"0"1D4BB% f
    \Umathcode"0067="0"0"0210A% g
    \Umathcode"0068="0"0"1D4BD% h
    \Umathcode"0069="0"0"1D4BE% i
    \Umathcode"006A="0"0"1D4BF% j
    \Umathcode"006B="0"0"1D4C0% k
    \Umathcode"006C="0"0"1D4C1% l
    \Umathcode"006D="0"0"1D4C2% m
    \Umathcode"006E="0"0"1D4C3% n
    \Umathcode"006F="0"0"02134% o
    \Umathcode"0070="0"0"1D4C5% p
    \Umathcode"0071="0"0"1D4C6% q
    \Umathcode"0072="0"0"1D4C7% r
    \Umathcode"0073="0"0"1D4C8% s
    \Umathcode"0074="0"0"1D4C9% t
    \Umathcode"0075="0"0"1D4CA% u
    \Umathcode"0076="0"0"1D4CB% v
    \Umathcode"0077="0"0"1D4CC% w
    \Umathcode"0078="0"0"1D4CD% x
    \Umathcode"0079="0"0"1D4CE% y
    \Umathcode"007A="0"0"1D4CF% z
}

\Udelcode  "00021 = "0 "00021
\Udelcode  "00028 = "0 "00028
\Udelcode  "00028 = "0 "00028
\Udelcode  "00029 = "0 "00029
\Udelcode  "00029 = "0 "00029
\Udelcode  "0002F = "0 "0002F
\Udelcode  "0002F = "0 "0002F
\Udelcode  "0002F = "0 "02044
\Udelcode  "0003F = "0 "0003F
\Udelcode  "0005B = "0 "0005B
\Udelcode  "0005B = "0 "0005B
\Udelcode  "0005D = "0 "0005D
\Udelcode  "0005D = "0 "0005D
\Udelcode  "0007B = "0 "0007B
\Udelcode  "0007B = "0 "0007B
\Udelcode  "0007C = "0 "0007C
\Udelcode  "0007C = "0 "0007C
\Udelcode  "0007C = "0 "0007C
\Udelcode  "0007C = "0 "0007C
\Udelcode  "0007C = "0 "0007C
\Udelcode  "0007D = "0 "0007D
\Udelcode  "0007D = "0 "0007D
\Udelcode  "02016 = "0 "02016
\Udelcode  "02016 = "0 "02016
\Udelcode  "02016 = "0 "02016
\Udelcode  "02016 = "0 "02016
\Udelcode  "02016 = "0 "02016
\Udelcode  "02044 = "0 "02044
\Udelcode  "02044 = "0 "02044
\Udelcode  "02308 = "0 "02308
\Udelcode  "02308 = "0 "02308
\Udelcode  "02308 = "0 "02308
\Udelcode  "02308 = "0 "02308
\Udelcode  "02308 = "0 "02308
\Udelcode  "02309 = "0 "02309
\Udelcode  "02309 = "0 "02309
\Udelcode  "02309 = "0 "02309
\Udelcode  "02309 = "0 "02309
\Udelcode  "02309 = "0 "02309
\Udelcode  "0230A = "0 "0230A
\Udelcode  "0230A = "0 "0230A
\Udelcode  "0230B = "0 "0230B
\Udelcode  "0230B = "0 "0230B
\Udelcode  "0231C = "0 "0231C
\Udelcode  "0231C = "0 "0231C
\Udelcode  "0231D = "0 "0231D
\Udelcode  "0231D = "0 "0231D
\Udelcode  "0231E = "0 "0231E
\Udelcode  "0231E = "0 "0231E
\Udelcode  "0231F = "0 "0231F
\Udelcode  "0231F = "0 "0231F
\Udelcode  "023B0 = "0 "023B0
\Udelcode  "023B0 = "0 "023B0
\Udelcode  "023B1 = "0 "023B1
\Udelcode  "023B1 = "0 "023B1
\Udelcode  "027E6 = "0 "027E6
\Udelcode  "027E6 = "0 "027E6
\Udelcode  "027E7 = "0 "027E7
\Udelcode  "027E7 = "0 "027E7
\Udelcode  "027E8 = "0 "027E8
\Udelcode  "027E8 = "0 "027E8
\Udelcode  "027E9 = "0 "027E9
\Udelcode  "027E9 = "0 "027E9
\Udelcode  "027EA = "0 "027EA
\Udelcode  "027EA = "0 "027EA
\Udelcode  "027EB = "0 "027EB
\Udelcode  "027EB = "0 "027EB
\Udelcode  "027EE = "0 "027EE
\Udelcode  "027EE = "0 "027EE
\Udelcode  "027EF = "0 "027EF
\Udelcode  "027EF = "0 "027EF

\Umathcode "00021 = "5 "0 "00021
\Umathcode "00022 = "0 "0 "00022
\Umathcode "00027 = "0 "0 "00027
\Umathcode "00028 = "4 "0 "00028
\Umathcode "00029 = "5 "0 "00029
\Umathcode "0002A = "2 "0 "02217
\Umathcode "0002B = "2 "0 "0002B
\Umathcode "0002C = "6 "0 "0002C
\Umathcode "0002D = "2 "0 "02212
\Umathcode "0002E = "6 "0 "0002E
\Umathcode "0002F = "4 "0 "02044
\Umathcode "0003A = "3 "0 "0003A
\Umathcode "0003B = "6 "0 "0003B
\Umathcode "0003C = "3 "0 "0003C
\Umathcode "0003D = "3 "0 "0003D
\Umathcode "0003E = "3 "0 "0003E
\Umathcode "0003F = "5 "0 "0003F
\Umathcode "0005B = "4 "0 "0005B
\Umathcode "0005C = "0 "0 "0005C
\Umathcode "0005D = "5 "0 "0005D
\Umathcode "0007B = "4 "0 "0007B
\Umathcode "0007C = "0 "0 "0007C
\Umathcode "0007D = "5 "0 "0007D
\Umathcode "000A5 = "0 "0 "000A5
\Umathcode "000A7 = "0 "0 "000A7
\Umathcode "000AC = "0 "0 "000AC
\Umathcode "000B1 = "2 "0 "000B1
\Umathcode "000B6 = "0 "0 "000B6
\Umathcode "000B7 = "2 "0 "000B7
\Umathcode "000D7 = "2 "0 "000D7
\Umathcode "000F0 = "0 "0 "000F0
\Umathcode "000F7 = "2 "0 "000F7
\Umathcode "00338 = "3 "0 "00338
\Umathcode "003F0 = "0 "0 "003F0
\Umathcode "02016 = "0 "0 "02016
\Umathcode "02020 = "2 "0 "02020
\Umathcode "02021 = "2 "0 "02021
\Umathcode "02022 = "2 "0 "02022
\Umathcode "02026 = "0 "0 "02026
\Umathcode "02032 = "0 "0 "02032
\Umathcode "02033 = "0 "0 "02033
\Umathcode "02034 = "0 "0 "02034
\Umathcode "02044 = "0 "0 "02044
\Umathcode "0207A = "2 "0 "0207A
\Umathcode "0207B = "2 "0 "0207B
\Umathcode "020DD = "0 "0 "020DD
\Umathcode "020DE = "0 "0 "020DE
\Umathcode "020DF = "0 "0 "020DF
\Umathcode "02111 = "0 "0 "02111
\Umathcode "02113 = "0 "0 "02113
\Umathcode "02118 = "0 "0 "02118
\Umathcode "0211C = "0 "0 "0211C
\Umathcode "02132 = "0 "0 "02132
\Umathcode "02135 = "0 "0 "02135
\Umathcode "02136 = "0 "0 "02136
\Umathcode "02137 = "0 "0 "02137
\Umathcode "02138 = "0 "0 "02138
\Umathcode "02141 = "0 "0 "02141
\Umathcode "02142 = "0 "0 "02142
\Umathcode "02143 = "0 "0 "02143
\Umathcode "02144 = "0 "0 "02144
\Umathcode "02145 = "0 "0 "02145
\Umathcode "02146 = "0 "0 "02146
\Umathcode "02147 = "0 "0 "02147
\Umathcode "02148 = "0 "0 "02148
\Umathcode "02149 = "0 "0 "02149
\Umathcode "0214A = "0 "0 "0214A
\Umathcode "0214B = "2 "0 "0214B
\Umathcode "02190 = "3 "0 "02190
\Umathcode "02191 = "3 "0 "02191
\Umathcode "02192 = "3 "0 "02192
\Umathcode "02193 = "3 "0 "02193
\Umathcode "02194 = "3 "0 "02194
\Umathcode "02195 = "3 "0 "02195
\Umathcode "02196 = "3 "0 "02196
\Umathcode "02197 = "3 "0 "02197
\Umathcode "02198 = "3 "0 "02198
\Umathcode "02199 = "3 "0 "02199
\Umathcode "0219A = "3 "0 "0219A
\Umathcode "0219B = "3 "0 "0219B
\Umathcode "0219C = "3 "0 "0219C
\Umathcode "0219D = "3 "0 "0219D
\Umathcode "0219E = "3 "0 "0219E
\Umathcode "0219F = "3 "0 "0219F
\Umathcode "021A0 = "3 "0 "021A0
\Umathcode "021A1 = "3 "0 "021A1
\Umathcode "021A2 = "3 "0 "021A2
\Umathcode "021A3 = "3 "0 "021A3
\Umathcode "021A4 = "3 "0 "021A4
\Umathcode "021A5 = "3 "0 "021A5
\Umathcode "021A6 = "3 "0 "021A6
\Umathcode "021A7 = "3 "0 "021A7
\Umathcode "021A8 = "0 "0 "021A8
\Umathcode "021A9 = "3 "0 "021A9
\Umathcode "021AA = "3 "0 "021AA
\Umathcode "021AB = "3 "0 "021AB
\Umathcode "021AC = "3 "0 "021AC
\Umathcode "021AD = "3 "0 "021AD
\Umathcode "021AE = "3 "0 "021AE
\Umathcode "021AF = "3 "0 "021AF
\Umathcode "021B0 = "3 "0 "021B0
\Umathcode "021B1 = "3 "0 "021B1
\Umathcode "021B2 = "3 "0 "021B2
\Umathcode "021B3 = "3 "0 "021B3
\Umathcode "021B4 = "0 "0 "021B4
\Umathcode "021B5 = "0 "0 "021B5
\Umathcode "021B6 = "3 "0 "021B6
\Umathcode "021B7 = "3 "0 "021B7
\Umathcode "021B8 = "3 "0 "021B8
\Umathcode "021B9 = "3 "0 "021B9
\Umathcode "021BA = "3 "0 "021BA
\Umathcode "021BB = "3 "0 "021BB
\Umathcode "021BC = "3 "0 "021BC
\Umathcode "021BD = "3 "0 "021BD
\Umathcode "021BE = "3 "0 "021BE
\Umathcode "021BF = "3 "0 "021BF
\Umathcode "021C0 = "3 "0 "021C0
\Umathcode "021C1 = "3 "0 "021C1
\Umathcode "021C2 = "3 "0 "021C2
\Umathcode "021C3 = "3 "0 "021C3
\Umathcode "021C4 = "3 "0 "021C4
\Umathcode "021C5 = "3 "0 "021C5
\Umathcode "021C6 = "3 "0 "021C6
\Umathcode "021C7 = "3 "0 "021C7
\Umathcode "021C8 = "3 "0 "021C8
\Umathcode "021C9 = "3 "0 "021C9
\Umathcode "021CA = "3 "0 "021CA
\Umathcode "021CB = "3 "0 "021CB
\Umathcode "021CC = "3 "0 "021CC
\Umathcode "021CD = "3 "0 "021CD
\Umathcode "021CE = "3 "0 "021CE
\Umathcode "021CF = "3 "0 "021CF
\Umathcode "021D0 = "3 "0 "021D0
\Umathcode "021D1 = "3 "0 "021D1
\Umathcode "021D2 = "3 "0 "021D2
\Umathcode "021D3 = "3 "0 "021D3
\Umathcode "021D4 = "3 "0 "021D4
\Umathcode "021D5 = "3 "0 "021D5
\Umathcode "021D6 = "3 "0 "021D6
\Umathcode "021D7 = "3 "0 "021D7
\Umathcode "021D8 = "3 "0 "021D8
\Umathcode "021D9 = "3 "0 "021D9
\Umathcode "021DA = "3 "0 "021DA
\Umathcode "021DB = "3 "0 "021DB
\Umathcode "021DC = "3 "0 "021DC
\Umathcode "021DD = "3 "0 "021DD
\Umathcode "021DE = "3 "0 "021DE
\Umathcode "021DF = "3 "0 "021DF
\Umathcode "021E0 = "3 "0 "021E0
\Umathcode "021E1 = "3 "0 "021E1
\Umathcode "021E2 = "3 "0 "021E2
\Umathcode "021E3 = "3 "0 "021E3
\Umathcode "021E4 = "3 "0 "021E4
\Umathcode "021E5 = "3 "0 "021E5
\Umathcode "021E6 = "0 "0 "021E6
\Umathcode "021E7 = "0 "0 "021E7
\Umathcode "021E8 = "0 "0 "021E8
\Umathcode "021E9 = "0 "0 "021E9
\Umathcode "021EB = "0 "0 "021EB
\Umathcode "021F4 = "3 "0 "021F4
\Umathcode "021F5 = "3 "0 "021F5
\Umathcode "021F6 = "3 "0 "021F6
\Umathcode "021F7 = "3 "0 "021F7
\Umathcode "021F8 = "3 "0 "021F8
\Umathcode "021F9 = "3 "0 "021F9
\Umathcode "021FA = "3 "0 "021FA
\Umathcode "021FB = "3 "0 "021FB
\Umathcode "021FC = "3 "0 "021FC
\Umathcode "021FD = "3 "0 "021FD
\Umathcode "021FE = "3 "0 "021FE
\Umathcode "021FF = "3 "0 "021FF
\Umathcode "02200 = "0 "0 "02200
\Umathcode "02201 = "0 "0 "02201
\Umathcode "02202 = "0 "0 "02202
\Umathcode "02203 = "0 "0 "02203
\Umathcode "02204 = "0 "0 "02204
\Umathcode "02205 = "0 "0 "02205
\Umathcode "02208 = "3 "0 "02208
\Umathcode "02209 = "3 "0 "02209
\Umathcode "0220B = "3 "0 "0220B
\Umathcode "0220C = "3 "0 "0220C
\Umathcode "0220F = "1 "0 "0220F
\Umathcode "02210 = "1 "0 "02210
\Umathcode "02211 = "1 "0 "02211
\Umathcode "02212 = "2 "0 "02212
\Umathcode "02213 = "2 "0 "02213
\Umathcode "02214 = "2 "0 "02214
\Umathcode "02216 = "2 "0 "02216
\Umathcode "02217 = "2 "0 "02217
\Umathcode "02218 = "2 "0 "02218
\Umathcode "02219 = "2 "0 "02219
\Umathcode "0221D = "3 "0 "0221D
\Umathcode "0221E = "0 "0 "0221E
\Umathcode "0221F = "0 "0 "0221F
\Umathcode "02220 = "0 "0 "02220
\Umathcode "02221 = "0 "0 "02221
\Umathcode "02222 = "0 "0 "02222
\Umathcode "02223 = "2 "0 "02223
\Umathcode "02224 = "2 "0 "02224
\Umathcode "02225 = "3 "0 "02225
\Umathcode "02226 = "3 "0 "02226
\Umathcode "02227 = "2 "0 "02227
\Umathcode "02228 = "2 "0 "02228
\Umathcode "02229 = "2 "0 "02229
\Umathcode "0222A = "2 "0 "0222A
\Umathcode "0222B = "1 "0 "0222B
\Umathcode "0222C = "1 "0 "0222C
\Umathcode "0222D = "1 "0 "0222D
\Umathcode "0222E = "1 "0 "0222E
\Umathcode "0222F = "1 "0 "0222F
\Umathcode "02230 = "1 "0 "02230
\Umathcode "02231 = "1 "0 "02231
\Umathcode "02232 = "1 "0 "02232
\Umathcode "02233 = "1 "0 "02233
\Umathcode "02234 = "3 "0 "02234
\Umathcode "02235 = "3 "0 "02235
\Umathcode "02236 = "6 "0 "02236
\Umathcode "02237 = "3 "0 "02237
\Umathcode "02238 = "2 "0 "02238
\Umathcode "02239 = "3 "0 "02239
\Umathcode "0223C = "3 "0 "0223C
\Umathcode "0223D = "3 "0 "0223D
\Umathcode "02240 = "2 "0 "02240
\Umathcode "02241 = "3 "0 "02241
\Umathcode "02242 = "3 "0 "02242
\Umathcode "02243 = "3 "0 "02243
\Umathcode "02244 = "3 "0 "02244
\Umathcode "02245 = "3 "0 "02245
\Umathcode "02246 = "3 "0 "02246
\Umathcode "02247 = "3 "0 "02247
\Umathcode "02248 = "3 "0 "02248
\Umathcode "02249 = "3 "0 "02249
\Umathcode "0224A = "3 "0 "0224A
\Umathcode "0224C = "3 "0 "0224C
\Umathcode "0224D = "3 "0 "0224D
\Umathcode "0224E = "3 "0 "0224E
\Umathcode "02250 = "3 "0 "02250
\Umathcode "02251 = "3 "0 "02251
\Umathcode "02252 = "3 "0 "02252
\Umathcode "02253 = "3 "0 "02253
\Umathcode "02254 = "3 "0 "02254
\Umathcode "02255 = "3 "0 "02255
\Umathcode "02256 = "3 "0 "02256
\Umathcode "02257 = "3 "0 "02257
\Umathcode "02259 = "3 "0 "02259
\Umathcode "0225A = "3 "0 "0225A
\Umathcode "0225B = "3 "0 "0225B
\Umathcode "0225C = "3 "0 "0225C
\Umathcode "0225D = "3 "0 "0225D
\Umathcode "0225E = "3 "0 "0225E
\Umathcode "0225F = "3 "0 "0225F
\Umathcode "02260 = "3 "0 "02260
\Umathcode "02261 = "3 "0 "02261
\Umathcode "02262 = "3 "0 "02262
\Umathcode "02263 = "3 "0 "02263
\Umathcode "02264 = "3 "0 "02264
\Umathcode "02265 = "3 "0 "02265
\Umathcode "02266 = "3 "0 "02266
\Umathcode "02267 = "3 "0 "02267
\Umathcode "02268 = "3 "0 "02268
\Umathcode "02269 = "3 "0 "02269
\Umathcode "0226A = "3 "0 "0226A
\Umathcode "0226B = "3 "0 "0226B
\Umathcode "0226C = "3 "0 "0226C
\Umathcode "0226D = "3 "0 "0226D
\Umathcode "0226E = "3 "0 "0226E
\Umathcode "0226F = "3 "0 "0226F
\Umathcode "02270 = "3 "0 "02270
\Umathcode "02271 = "3 "0 "02271
\Umathcode "02272 = "3 "0 "02272
\Umathcode "02273 = "3 "0 "02273
\Umathcode "02274 = "3 "0 "02274
\Umathcode "02275 = "3 "0 "02275
\Umathcode "02276 = "3 "0 "02276
\Umathcode "02277 = "3 "0 "02277
\Umathcode "02278 = "3 "0 "02278
\Umathcode "02279 = "3 "0 "02279
\Umathcode "0227A = "3 "0 "0227A
\Umathcode "0227B = "3 "0 "0227B
\Umathcode "0227C = "3 "0 "0227C
\Umathcode "0227D = "3 "0 "0227D
\Umathcode "0227E = "3 "0 "0227E
\Umathcode "0227F = "3 "0 "0227F
\Umathcode "02280 = "3 "0 "02280
\Umathcode "02281 = "3 "0 "02281
\Umathcode "02282 = "3 "0 "02282
\Umathcode "02283 = "3 "0 "02283
\Umathcode "02284 = "3 "0 "02284
\Umathcode "02285 = "3 "0 "02285
\Umathcode "02286 = "3 "0 "02286
\Umathcode "02287 = "3 "0 "02287
\Umathcode "02288 = "3 "0 "02288
\Umathcode "02289 = "3 "0 "02289
\Umathcode "0228A = "3 "0 "0228A
\Umathcode "0228B = "3 "0 "0228B
\Umathcode "0228E = "2 "0 "0228E
\Umathcode "0228F = "3 "0 "0228F
\Umathcode "02290 = "3 "0 "02290
\Umathcode "02291 = "2 "0 "02291
\Umathcode "02292 = "2 "0 "02292
\Umathcode "02293 = "2 "0 "02293
\Umathcode "02294 = "2 "0 "02294
\Umathcode "02295 = "2 "0 "02295
\Umathcode "02296 = "2 "0 "02296
\Umathcode "02297 = "2 "0 "02297
\Umathcode "02298 = "2 "0 "02298
\Umathcode "02299 = "2 "0 "02299
\Umathcode "0229A = "2 "0 "0229A
\Umathcode "0229B = "2 "0 "0229B
\Umathcode "0229C = "2 "0 "0229C
\Umathcode "0229D = "2 "0 "0229D
\Umathcode "0229E = "2 "0 "0229E
\Umathcode "0229F = "2 "0 "0229F
\Umathcode "022A0 = "2 "0 "022A0
\Umathcode "022A1 = "2 "0 "022A1
\Umathcode "022A2 = "3 "0 "022A2
\Umathcode "022A3 = "3 "0 "022A3
\Umathcode "022A4 = "0 "0 "022A4
\Umathcode "022A5 = "0 "0 "022A5
\Umathcode "022A7 = "3 "0 "022A7
\Umathcode "022A8 = "3 "0 "022A8
\Umathcode "022A9 = "3 "0 "022A9
\Umathcode "022AA = "3 "0 "022AA
\Umathcode "022AB = "3 "0 "022AB
\Umathcode "022AC = "3 "0 "022AC
\Umathcode "022AD = "3 "0 "022AD
\Umathcode "022AE = "3 "0 "022AE
\Umathcode "022AF = "3 "0 "022AF
\Umathcode "022B2 = "2 "0 "022B2
\Umathcode "022B3 = "2 "0 "022B3
\Umathcode "022B8 = "3 "0 "022B8
\Umathcode "022BA = "2 "0 "022BA
\Umathcode "022BB = "2 "0 "022BB
\Umathcode "022BC = "2 "0 "022BC
\Umathcode "022C0 = "1 "0 "022C0
\Umathcode "022C1 = "1 "0 "022C1
\Umathcode "022C2 = "1 "0 "022C2
\Umathcode "022C3 = "1 "0 "022C3
\Umathcode "022C4 = "2 "0 "022C4
\Umathcode "022C5 = "2 "0 "022C5
\Umathcode "022C6 = "2 "0 "022C6
\Umathcode "022C7 = "2 "0 "022C7
\Umathcode "022C8 = "3 "0 "022C8
\Umathcode "022C9 = "2 "0 "022C9
\Umathcode "022CA = "2 "0 "022CA
\Umathcode "022CB = "2 "0 "022CB
\Umathcode "022CC = "2 "0 "022CC
\Umathcode "022CE = "2 "0 "022CE
\Umathcode "022CF = "2 "0 "022CF
\Umathcode "022D0 = "3 "0 "022D0
\Umathcode "022D1 = "3 "0 "022D1
\Umathcode "022D2 = "2 "0 "022D2
\Umathcode "022D3 = "2 "0 "022D3
\Umathcode "022D4 = "3 "0 "022D4
\Umathcode "022D6 = "2 "0 "022D6
\Umathcode "022D7 = "2 "0 "022D7
\Umathcode "022D8 = "3 "0 "022D8
\Umathcode "022D9 = "3 "0 "022D9
\Umathcode "022DA = "3 "0 "022DA
\Umathcode "022DB = "3 "0 "022DB
\Umathcode "022DC = "3 "0 "022DC
\Umathcode "022DD = "3 "0 "022DD
\Umathcode "022DE = "3 "0 "022DE
\Umathcode "022DF = "3 "0 "022DF
\Umathcode "022E0 = "3 "0 "022E0
\Umathcode "022E1 = "3 "0 "022E1
\Umathcode "022E2 = "3 "0 "022E2
\Umathcode "022E3 = "3 "0 "022E3
\Umathcode "022E4 = "3 "0 "022E4
\Umathcode "022E5 = "3 "0 "022E5
\Umathcode "022E6 = "3 "0 "022E6
\Umathcode "022E7 = "3 "0 "022E7
\Umathcode "022E8 = "3 "0 "022E8
\Umathcode "022E9 = "3 "0 "022E9
\Umathcode "022EA = "3 "0 "022EA
\Umathcode "022EB = "3 "0 "022EB
\Umathcode "022EC = "3 "0 "022EC
\Umathcode "022ED = "3 "0 "022ED
\Umathcode "022EE = "0 "0 "022EE
\Umathcode "022EF = "0 "0 "022EF
\Umathcode "022F0 = "0 "0 "022F0
\Umathcode "022F1 = "0 "0 "022F1
\Umathcode "02300 = "0 "0 "02300
\Umathcode "02308 = "4 "0 "02308
\Umathcode "02309 = "5 "0 "02309
\Umathcode "0230A = "4 "0 "0230A
\Umathcode "0230B = "5 "0 "0230B
\Umathcode "0231C = "4 "0 "0231C
\Umathcode "0231D = "5 "0 "0231D
\Umathcode "0231E = "4 "0 "0231E
\Umathcode "0231F = "5 "0 "0231F
\Umathcode "02322 = "3 "0 "02322
\Umathcode "02323 = "3 "0 "02323
\Umathcode "023B0 = "4 "0 "023B0
\Umathcode "023B1 = "5 "0 "023B1
\Umathcode "024C7 = "0 "0 "024C7
\Umathcode "024C8 = "0 "0 "024C8
\Umathcode "025A0 = "0 "0 "025A0
\Umathcode "025A1 = "0 "0 "025A1
\Umathcode "025A2 = "0 "0 "025A2
\Umathcode "025B2 = "2 "0 "025B2
\Umathcode "025B3 = "0 "0 "025B3
\Umathcode "025B6 = "2 "0 "025B6
\Umathcode "025B7 = "2 "0 "025B7
\Umathcode "025BC = "2 "0 "025BC
\Umathcode "025BD = "2 "0 "025BD
\Umathcode "025C0 = "2 "0 "025C0
\Umathcode "025C1 = "2 "0 "025C1
\Umathcode "025CA = "0 "0 "025CA
\Umathcode "025EF = "2 "0 "025EF
\Umathcode "02605 = "0 "0 "02605
\Umathcode "02660 = "0 "0 "02660
\Umathcode "02661 = "0 "0 "02661
\Umathcode "02662 = "0 "0 "02662
\Umathcode "02663 = "0 "0 "02663
\Umathcode "02666 = "0 "0 "02666
\Umathcode "0266D = "0 "0 "0266D
\Umathcode "0266E = "0 "0 "0266E
\Umathcode "0266F = "0 "0 "0266F
\Umathcode "02713 = "0 "0 "02713
\Umathcode "02720 = "0 "0 "02720
\Umathcode "027E6 = "4 "0 "027E6
\Umathcode "027E7 = "5 "0 "027E7
\Umathcode "027E8 = "4 "0 "027E8
\Umathcode "027E9 = "5 "0 "027E9
\Umathcode "027EA = "4 "0 "027EA
\Umathcode "027EB = "5 "0 "027EB
\Umathcode "027EE = "4 "0 "027EE
\Umathcode "027EF = "5 "0 "027EF
\Umathcode "027F5 = "3 "0 "027F5
\Umathcode "027F6 = "3 "0 "027F6
\Umathcode "027F7 = "3 "0 "027F7
\Umathcode "027F8 = "3 "0 "027F8
\Umathcode "027F9 = "3 "0 "027F9
\Umathcode "027FA = "3 "0 "027FA
\Umathcode "027FB = "3 "0 "027FB
\Umathcode "027FC = "3 "0 "027FC
\Umathcode "027FD = "3 "0 "027FD
\Umathcode "027FE = "3 "0 "027FE
\Umathcode "027FF = "3 "0 "027FF
\Umathcode "02906 = "3 "0 "02906
\Umathcode "02907 = "3 "0 "02907
\Umathcode "0290A = "3 "0 "0290A
\Umathcode "0290B = "3 "0 "0290B
\Umathcode "0290C = "3 "0 "0290C
\Umathcode "0290D = "3 "0 "0290D
\Umathcode "02911 = "3 "0 "02911
\Umathcode "02916 = "3 "0 "02916
\Umathcode "02917 = "3 "0 "02917
\Umathcode "02921 = "3 "0 "02921
\Umathcode "02922 = "3 "0 "02922
\Umathcode "02923 = "3 "0 "02923
\Umathcode "02924 = "3 "0 "02924
\Umathcode "02925 = "3 "0 "02925
\Umathcode "02926 = "3 "0 "02926
\Umathcode "02A00 = "1 "0 "02A00
\Umathcode "02A01 = "1 "0 "02A01
\Umathcode "02A02 = "1 "0 "02A02
\Umathcode "02A03 = "1 "0 "02A03
\Umathcode "02A04 = "1 "0 "02A04
\Umathcode "02A05 = "1 "0 "02A05
\Umathcode "02A06 = "1 "0 "02A06
\Umathcode "02A09 = "1 "0 "02A09
\Umathcode "02A3F = "2 "0 "02A3F
\Umathcode "02A7D = "3 "0 "02A7D
\Umathcode "02A7E = "3 "0 "02A7E
\Umathcode "02A85 = "3 "0 "02A85
\Umathcode "02A86 = "3 "0 "02A86
\Umathcode "02A87 = "3 "0 "02A87
\Umathcode "02A88 = "3 "0 "02A88
\Umathcode "02A89 = "3 "0 "02A89
\Umathcode "02A8A = "3 "0 "02A8A
\Umathcode "02A8B = "3 "0 "02A8B
\Umathcode "02A8C = "3 "0 "02A8C
\Umathcode "02A95 = "3 "0 "02A95
\Umathcode "02A96 = "3 "0 "02A96
\Umathcode "02AAF = "3 "0 "02AAF
\Umathcode "02AB0 = "3 "0 "02AB0
\Umathcode "02AB1 = "3 "0 "02AB1
\Umathcode "02AB2 = "3 "0 "02AB2
\Umathcode "02AB3 = "3 "0 "02AB3
\Umathcode "02AB4 = "3 "0 "02AB4
\Umathcode "02AB5 = "3 "0 "02AB5
\Umathcode "02AB6 = "3 "0 "02AB6
\Umathcode "02AB7 = "3 "0 "02AB7
\Umathcode "02AB8 = "3 "0 "02AB8
\Umathcode "02AB9 = "3 "0 "02AB9
\Umathcode "02ABA = "3 "0 "02ABA
\Umathcode "02AC5 = "3 "0 "02AC5
\Umathcode "02AC6 = "3 "0 "02AC6
\Umathcode "02ACB = "3 "0 "02ACB
\Umathcode "02ACC = "3 "0 "02ACC
\Umathcode "12035 = "0 "0 "12035
\Umathcode "1D6A4 = "0 "0 "1D6A4
\Umathcode "1D6A5 = "0 "0 "1D6A5
\Umathcode "1D6FB = "0 "0 "1D6FB
\Umathcode "1D717 = "0 "0 "1D717
\Umathcode "1D718 = "0 "0 "1D718

% gaps .. done in lua (as example)

% \Umathcode "1D455 = "0 "0 "0210E
% \Umathcode "1D49D = "0 "0 "0212C
% \Umathcode "1D4A0 = "0 "0 "02130
% \Umathcode "1D4A1 = "0 "0 "02131
% \Umathcode "1D4A3 = "0 "0 "0210B
% \Umathcode "1D4A4 = "0 "0 "02110
% \Umathcode "1D4A7 = "0 "0 "02112
% \Umathcode "1D4A8 = "0 "0 "02133
% \Umathcode "1D4AD = "0 "0 "0211B
% \Umathcode "1D4BA = "0 "0 "0212F
% \Umathcode "1D4BC = "0 "0 "0210A
% \Umathcode "1D4C4 = "0 "0 "02134
% \Umathcode "1D506 = "0 "0 "0212D
% \Umathcode "1D50B = "0 "0 "0210C
% \Umathcode "1D50C = "0 "0 "02111
% \Umathcode "1D515 = "0 "0 "0211C
% \Umathcode "1D51D = "0 "0 "02128
% \Umathcode "1D53A = "0 "0 "02102
% \Umathcode "1D53F = "0 "0 "0210D
% \Umathcode "1D545 = "0 "0 "02115
% \Umathcode "1D547 = "0 "0 "02119
% \Umathcode "1D548 = "0 "0 "0211A
% \Umathcode "1D549 = "0 "0 "0211D
% \Umathcode "1D551 = "0 "0 "02124

% initialization

\the\everymathit

% a couple of definitions (we could also use \mathchardef):

\def\acute                     {\Umathaccent"0"0"0000B4 }
\def\acwopencirclearrow        {\Umathchar  "3"0"0021BA }
\def\aleph                     {\Umathchar  "0"0"002135 }
\def\Alpha                     {\Umathchar  "0"0"000391 }
\def\alpha                     {\Umathchar  "0"0"0003B1 }
\def\amalg                     {\Umathchar  "2"0"002A3F }
\def\angle                     {\Umathchar  "0"0"002220 }
\def\Angstrom                  {\Umathchar  "0"0"00212B }
\def\approx                    {\Umathchar  "3"0"002248 }
\def\approxEq                  {\Umathchar  "3"0"002245 }
\def\approxeq                  {\Umathchar  "3"0"00224A }
\def\approxnEq                 {\Umathchar  "3"0"002247 }
\def\arrowvert                 {\Umathchar  "0"0"00007C }
\def\Arrowvert                 {\Umathchar  "0"0"002016 }
\def\ast                       {\Umathchar  "2"0"002217 }
\def\ast                       {\Umathchar  "2"0"002217 }
\def\asymp                     {\Umathchar  "3"0"00224D }
\def\backepsilon               {\Umathchar  "0"0"0003F6 }
\def\backprime                 {\Umathchar  "0"0"012035 }
\def\backsim                   {\Umathchar  "3"0"00223D }
\def\backslash                 {\Umathchar  "0"0"00005C }
\def\bar                       {\Umathaccent"0"0"0000AF }
\def\barleftarrow              {\Umathchar  "3"0"0021E4 }
\def\barleftarrowrightarrowbar {\Umathchar  "3"0"0021B9 }
\def\barovernorthwestarrow     {\Umathchar  "3"0"0021B8 }
\def\barwedge                  {\Umathchar  "2"0"0022BC }
\def\because                   {\Umathchar  "3"0"002235 }
\def\Beta                      {\Umathchar  "0"0"000392 }
\def\beta                      {\Umathchar  "0"0"0003B2 }
\def\beth                      {\Umathchar  "0"0"002136 }
\def\between                   {\Umathchar  "3"0"00226C }
\def\bigcap                    {\Umathchar  "1"0"0022C2 }
\def\bigcirc                   {\Umathchar  "2"0"0025EF }
\def\bigcircle                 {\Umathchar  "2"0"0020DD }
\def\bigcircle                 {\Umathchar  "2"0"0020DD }
\def\bigcup                    {\Umathchar  "1"0"0022C3 }
\def\bigdiamond                {\Umathchar  "0"0"0020DF }
\def\bigodot                   {\Umathchar  "1"0"002A00 }
\def\bigoplus                  {\Umathchar  "1"0"002A01 }
\def\bigotimes                 {\Umathchar  "1"0"002A02 }
\def\bigsqcap                  {\Umathchar  "1"0"002A05 }
\def\bigsqcup                  {\Umathchar  "1"0"002A06 }
\def\bigsquare                 {\Umathchar  "0"0"0020DE }
\def\bigstar                   {\Umathchar  "0"0"002605 }
\def\bigtimes                  {\Umathchar  "1"0"002A09 }
\def\bigtriangledown           {\Umathchar  "2"0"0025BD }
\def\bigtriangleup             {\Umathchar  "2"0"0025B3 }
\def\bigudot                   {\Umathchar  "1"0"002A03 }
\def\biguplus                  {\Umathchar  "1"0"002A04 }
\def\bigvee                    {\Umathchar  "1"0"0022C1 }
\def\bigwedge                  {\Umathchar  "1"0"0022C0 }
\def\blacklozenge              {\Umathchar  "0"0"002666 }
\def\blacksquare               {\Umathchar  "0"0"0025A0 }
\def\blacktriangle             {\Umathchar  "2"0"0025B2 }
\def\blacktriangledown         {\Umathchar  "2"0"0025BC }
\def\blacktriangleleft         {\Umathchar  "2"0"0025C0 }
\def\blacktriangleright        {\Umathchar  "2"0"0025B6 }
\def\bot                       {\Umathchar  "0"0"0022A5 }
\def\bowtie                    {\Umathchar  "3"0"0022C8 }
\def\Box                       {\Umathchar  "0"0"0025A1 }
\def\boxdot                    {\Umathchar  "2"0"0022A1 }
\def\boxminus                  {\Umathchar  "2"0"00229F }
\def\boxplus                   {\Umathchar  "2"0"00229E }
\def\boxtimes                  {\Umathchar  "2"0"0022A0 }
%def\braceld                   {\Umathchar  "0"0"000000 }
%def\bracerd                   {\Umathchar  "0"0"000000 }
%def\bracelu                   {\Umathchar  "0"0"000000 }
%def\braceru                   {\Umathchar  "0"0"000000 }
\def\breve                     {\Umathaccent"0"0"0002D8 }
\def\bullet                    {\Umathchar  "2"0"002022 }
\def\bullet                    {\Umathchar  "2"0"002022 }
\def\Bumpeq                    {\Umathchar  "3"0"00224E }
\def\cap                       {\Umathchar  "2"0"002229 }
\def\Cap                       {\Umathchar  "2"0"0022D2 }
\def\carriagereturn            {\Umathchar  "0"0"0021B5 }
\def\cdot                      {\Umathchar  "2"0"0022C5 }
\def\cdotp                     {\Umathchar  "6"0"0022C5 }
\def\cdots                     {\Umathchar  "0"0"0022EF }
\def\centerdot                 {\Umathchar  "2"0"0000B7 }
\def\check                     {\Umathaccent"0"0"0002C7 }
\def\checkmark                 {\Umathchar  "0"0"002713 }
\def\Chi                       {\Umathchar  "0"0"0003A7 }
\def\chi                       {\Umathchar  "0"0"0003C7 }
\def\circ                      {\Umathchar  "2"0"002218 }
\def\circeq                    {\Umathchar  "3"0"002257 }
\def\circlearrowleft           {\Umathchar  "3"0"0021BB }
\def\circlearrowright          {\Umathchar  "3"0"0021BA }
\def\circledast                {\Umathchar  "2"0"00229B }
\def\circledcirc               {\Umathchar  "2"0"00229A }
\def\circleddash               {\Umathchar  "2"0"00229D }
\def\circledequals             {\Umathchar  "2"0"00229C }
\def\circledR                  {\Umathchar  "0"0"0024C7 }
\def\circledS                  {\Umathchar  "0"0"0024C8 }
\def\circleonrightarrow        {\Umathchar  "3"0"0021F4 }
\def\clubsuit                  {\Umathchar  "0"0"002663 }
\def\colon                     {\Umathchar  "6"0"002236 }
\def\colonequals               {\Umathchar  "3"0"002254 }
\def\complement                {\Umathchar  "0"0"002201 }
\def\complexes                 {\Umathchar  "0"0"002102 }
\def\cong                      {\Umathchar  "3"0"002245 }
\def\coprod                    {\Umathchar  "1"0"002210 }
\def\cup                       {\Umathchar  "2"0"00222A }
\def\Cup                       {\Umathchar  "2"0"0022D3 }
\def\curlyeqprec               {\Umathchar  "3"0"0022DE }
\def\curlyeqsucc               {\Umathchar  "3"0"0022DF }
\def\curlyvee                  {\Umathchar  "2"0"0022CE }
\def\curlywedge                {\Umathchar  "2"0"0022CF }
\def\curvearrowleft            {\Umathchar  "3"0"0021B6 }
\def\curvearrowright           {\Umathchar  "3"0"0021B7 }
\def\cwopencirclearrow         {\Umathchar  "3"0"0021BB }
\def\dag                       {\Umathchar  "0"0"002020 }
\def\dagger                    {\Umathchar  "2"0"002020 }
\def\daleth                    {\Umathchar  "0"0"002138 }
\def\dasharrow                 {\Umathchar  "3"0"0021E2 }
\def\dashedleftarrow           {\Umathchar  "3"0"00290C }
\def\dashedrightarrow          {\Umathchar  "3"0"00290D }
\def\dashv                     {\Umathchar  "3"0"0022A3 }
\def\ddag                      {\Umathchar  "0"0"002021 }
\def\ddagger                   {\Umathchar  "2"0"002021 }
\def\dddot                     {\Umathaccent"0"0"0020DB }
\def\ddot                      {\Umathaccent"0"0"0000A8 }
\def\ddots                     {\Umathchar  "0"0"0022F1 }
\def\Ddownarrow                {\Umathchar  "3"0"00290B }
\def\definedeq                 {\Umathchar  "3"0"00225D }
\def\Delta                     {\Umathchar  "0"0"000394 }
\def\delta                     {\Umathchar  "0"0"0003B4 }
\def\diamond                   {\Umathchar  "2"0"0022C4 }
\def\diamondsuit               {\Umathchar  "0"0"002662 }
\def\differentialD             {\Umathchar  "0"0"002145 }
\def\differentiald             {\Umathchar  "0"0"002146 }
\def\digamma                   {\Umathchar  "0"0"0003DC }
\def\div                       {\Umathchar  "2"0"0000F7 }
\def\divideontimes             {\Umathchar  "2"0"0022C7 }
\def\divides                   {\Umathchar  "2"0"002223 }
\def\dot                       {\Umathaccent"0"0"0002D9 }
\def\doteq                     {\Umathchar  "3"0"002250 }
\def\Doteq                     {\Umathchar  "3"0"002251 }
\def\doteqdot                  {\Umathchar  "3"0"002251 }
\def\dotminus                  {\Umathchar  "2"0"002238 }
\def\dotplus                   {\Umathchar  "2"0"002214 }
\def\dots                      {\Umathchar  "0"0"002026 }
\def\dottedrightarrow          {\Umathchar  "3"0"002911 }
\def\doublecap                 {\Umathchar  "2"0"0022D2 }
\def\doublecup                 {\Umathchar  "2"0"0022D3 }
\def\doubleprime               {\Umathchar  "0"0"002033 }
\def\downarrow                 {\Umathchar  "3"0"002193 }
\def\Downarrow                 {\Umathchar  "3"0"0021D3 }
\def\downdasharrow             {\Umathchar  "3"0"0021E3 }
\def\downdownarrows            {\Umathchar  "3"0"0021CA }
\def\downharpoonleft           {\Umathchar  "3"0"0021C3 }
\def\downharpoonright          {\Umathchar  "3"0"0021C2 }
\def\downuparrows              {\Umathchar  "3"0"0021F5 }
\def\downwhitearrow            {\Umathchar  "0"0"0021E9 }
\def\downzigzagarrow           {\Umathchar  "3"0"0021AF }
\def\ell                       {\Umathchar  "0"0"002113 }
\def\emptyset                  {\Umathchar  "0"0"002205 }
\def\Epsilon                   {\Umathchar  "0"0"000395 }
\def\epsilon                   {\Umathchar  "0"0"0003F5 }
\def\eq                        {\Umathchar  "3"0"00003D }
\def\eqcirc                    {\Umathchar  "3"0"002256 }
\def\eqgtr                     {\Umathchar  "3"0"0022DD }
\def\eqless                    {\Umathchar  "3"0"0022DC }
\def\eqsim                     {\Umathchar  "3"0"002242 }
\def\eqslantgtr                {\Umathchar  "3"0"002A96 }
\def\eqslantless               {\Umathchar  "3"0"002A95 }
\def\equalscolon               {\Umathchar  "3"0"002255 }
\def\equiv                     {\Umathchar  "3"0"002261 }
\def\Eta                       {\Umathchar  "0"0"000397 }
\def\eta                       {\Umathchar  "0"0"0003B7 }
\def\eth                       {\Umathchar  "0"0"0000F0 }
\def\Eulerconst                {\Umathchar  "0"0"002107 }
\def\exists                    {\Umathchar  "0"0"002203 }
\def\exponentiale              {\Umathchar  "0"0"002147 }
\def\fallingdotseq             {\Umathchar  "3"0"002252 }
\def\Finv                      {\Umathchar  "0"0"002132 }
\def\flat                      {\Umathchar  "0"0"00266D }
\def\forall                    {\Umathchar  "0"0"002200 }
\def\frown                     {\Umathchar  "3"0"002322 }
\def\Game                      {\Umathchar  "0"0"002141 }
\def\Gamma                     {\Umathchar  "0"0"000393 }
\def\gamma                     {\Umathchar  "0"0"0003B3 }
\def\ge                        {\Umathchar  "3"0"002265 }
\def\geq                       {\Umathchar  "3"0"002265 }
\def\geqq                      {\Umathchar  "3"0"002267 }
\def\geqslant                  {\Umathchar  "3"0"002A7E }
\def\gets                      {\Umathchar  "3"0"002190 }
\def\gg                        {\Umathchar  "3"0"00226B }
\def\ggg                       {\Umathchar  "3"0"0022D9 }
\def\gggtr                     {\Umathchar  "3"0"0022D9 }
\def\gimel                     {\Umathchar  "0"0"002137 }
\def\gnapprox                  {\Umathchar  "3"0"002A8A }
\def\gneqq                     {\Umathchar  "3"0"002269 }
\def\gnsim                     {\Umathchar  "3"0"0022E7 }
\def\grave                     {\Umathaccent"0"0"000060 }
\def\gt                        {\Umathchar  "3"0"00003E }
\def\gtrapprox                 {\Umathchar  "3"0"002A86 }
\def\gtrdot                    {\Umathchar  "2"0"0022D7 }
\def\gtreqless                 {\Umathchar  "3"0"0022DB }
\def\gtreqqless                {\Umathchar  "3"0"002A8C }
\def\gtrless                   {\Umathchar  "3"0"002277 }
\def\gtrsim                    {\Umathchar  "3"0"002273 }
\def\hat                       {\Umathaccent"0"0"0002C6 }
\def\hbar                      {\Umathchar  "0"0"00210F }
\def\heartsuit                 {\Umathchar  "0"0"002661 }
\def\hookleftarrow             {\Umathchar  "3"0"0021A9 }
\def\hookrightarrow            {\Umathchar  "3"0"0021AA }
\def\hslash                    {\Umathchar  "0"0"00210F }
\def\iiint                     {\Umathchar  "1"0"00222D }
\def\iiintop                   {\Umathchar  "0"0"00222D }
\def\iint                      {\Umathchar  "1"0"00222C }
\def\iintop                    {\Umathchar  "0"0"00222C }
\def\Im                        {\Umathchar  "0"0"002111 }
\def\imaginaryi                {\Umathchar  "0"0"002148 }
\def\imaginaryj                {\Umathchar  "0"0"002149 }
\def\imath                     {\Umathchar  "0"0"01D6A4 }
\def\imply                     {\Umathchar  "3"0"0021D2 }
\def\in                        {\Umathchar  "0"0"002208 }
\def\infty                     {\Umathchar  "0"0"00221E }
\def\int                       {\Umathchar  "1"0"00222B }
\def\intclockwise              {\Umathchar  "1"0"002231 }
\def\integers                  {\Umathchar  "0"0"002124 }
\def\intercal                  {\Umathchar  "2"0"0022BA }
\def\intop                     {\Umathchar  "0"0"00222B }
\def\Iota                      {\Umathchar  "0"0"000399 }
\def\iota                      {\Umathchar  "0"0"0003B9 }
\def\jmath                     {\Umathchar  "0"0"01D6A5 }
\def\Join                      {\Umathchar  "3"0"0022C8 }
\def\Kappa                     {\Umathchar  "0"0"00039A }
\def\kappa                     {\Umathchar  "0"0"0003BA }
\def\Lambda                    {\Umathchar  "0"0"00039B }
\def\lambda                    {\Umathchar  "0"0"0003BB }
\def\land                      {\Umathchar  "2"0"002227 }
\def\langle                    {\Udelimiter "4"0"0027E8 }
\def\lbrace                    {\Udelimiter "4"0"00007B }
\def\lbrack                    {\Udelimiter "4"0"00005B }
\def\lceil                     {\Udelimiter "4"0"002308 }
\def\lceiling                  {\Udelimiter "4"0"002308 }
\def\ldotp                     {\Umathchar  "6"0"00002E }
\def\ldots                     {\Umathchar  "0"0"002026 }
\def\Ldsh                      {\Umathchar  "3"0"0021B2 }
\def\le                        {\Umathchar  "3"0"002264 }
\def\leadsto                   {\Umathchar  "3"0"0021DD }
\def\leftarrow                 {\Umathchar  "3"0"002190 }
\def\Leftarrow                 {\Umathchar  "3"0"0021D0 }
\def\leftarrowtail             {\Umathchar  "3"0"0021A2 }
\def\leftarrowtriangle         {\Umathchar  "3"0"0021FD }
\def\leftdasharrow             {\Umathchar  "3"0"0021E0 }
\def\leftharpoondown           {\Umathchar  "3"0"0021BD }
\def\leftharpoonup             {\Umathchar  "3"0"0021BC }
\def\leftleftarrows            {\Umathchar  "3"0"0021C7 }
\def\leftrightarrow            {\Umathchar  "3"0"002194 }
\def\Leftrightarrow            {\Umathchar  "3"0"0021D4 }
\def\leftrightarrows           {\Umathchar  "3"0"0021C6 }
\def\leftrightarrowtriangle    {\Umathchar  "3"0"0021FF }
\def\leftrightharpoons         {\Umathchar  "3"0"0021CB }
\def\leftrightsquigarrow       {\Umathchar  "3"0"0021AD }
\def\leftsquigarrow            {\Umathchar  "3"0"0021DC }
\def\leftthreetimes            {\Umathchar  "2"0"0022CB }
\def\leftwavearrow             {\Umathchar  "3"0"00219C }
\def\leftwhitearrow            {\Umathchar  "0"0"0021E6 }
\def\leq                       {\Umathchar  "3"0"002264 }
\def\leqq                      {\Umathchar  "3"0"002266 }
\def\leqslant                  {\Umathchar  "3"0"002A7D }
\def\lessapprox                {\Umathchar  "3"0"002A85 }
\def\lessdot                   {\Umathchar  "2"0"0022D6 }
\def\lesseqgtr                 {\Umathchar  "3"0"0022DA }
\def\lesseqqgtr                {\Umathchar  "3"0"002A8B }
\def\lessgtr                   {\Umathchar  "3"0"002276 }
\def\lesssim                   {\Umathchar  "3"0"002272 }
\def\lfloor                    {\Udelimiter "4"0"00230A }
\def\lgroup                    {\Udelimiter "4"0"0027EE }
\def\lhook                     {\Umathchar  "3"0"0FE322 }
\def\lhooknwarrow              {\Umathchar  "3"0"002923 }
\def\lhooksearrow              {\Umathchar  "3"0"002925 }
\def\linefeed                  {\Umathchar  "0"0"0021B4 }
\def\ll                        {\Umathchar  "3"0"00226A }
\def\llangle                   {\Udelimiter "4"0"0027EA }
\def\llbracket                 {\Udelimiter "4"0"0027E6 }
\def\llcorner                  {\Udelimiter "4"0"00231E }
\def\Lleftarrow                {\Umathchar  "3"0"0021DA }
\def\lll                       {\Umathchar  "3"0"0022D8 }
\def\llless                    {\Umathchar  "3"0"0022D8 }
\def\lmoustache                {\Udelimiter "4"0"0023B0 }
\def\lnapprox                  {\Umathchar  "3"0"002A89 }
\def\lneq                      {\Umathchar  "3"0"002A87 }
\def\lneqq                     {\Umathchar  "3"0"002268 }
\def\lnot                      {\Umathchar  "0"0"0000AC }
\def\lnsim                     {\Umathchar  "3"0"0022E6 }
\def\longleftarrow             {\Umathchar  "3"0"0027F5 }
\def\Longleftarrow             {\Umathchar  "3"0"0027F8 }
\def\longleftrightarrow        {\Umathchar  "3"0"0027F7 }
\def\Longleftrightarrow        {\Umathchar  "3"0"0027FA }
\def\longmapsfrom              {\Umathchar  "3"0"0027FB }
\def\Longmapsfrom              {\Umathchar  "3"0"0027FD }
\def\longmapsto                {\Umathchar  "3"0"0027FC }
\def\Longmapsto                {\Umathchar  "3"0"0027FE }
\def\longrightarrow            {\Umathchar  "3"0"0027F6 }
\def\Longrightarrow            {\Umathchar  "3"0"0027F9 }
\def\longrightsquigarrow       {\Umathchar  "3"0"0027FF }
\def\looparrowleft             {\Umathchar  "3"0"0021AB }
\def\looparrowright            {\Umathchar  "3"0"0021AC }
\def\lor                       {\Umathchar  "2"0"002228 }
\def\lozenge                   {\Umathchar  "0"0"0025CA }
\def\lparent                   {\Udelimiter "4"0"000028 }
\def\lrcorner                  {\Udelimiter "5"0"00231F }
\def\Lsh                       {\Umathchar  "3"0"0021B0 }
\def\lt                        {\Umathchar  "3"0"00003C }
\def\ltimes                    {\Umathchar  "2"0"0022C9 }
\def\lvert                     {\Udelimiter "4"0"00007C }
\def\lVert                     {\Udelimiter "4"0"002016 }
\def\maltese                   {\Umathchar  "0"0"002720 }
\def\mapsdown                  {\Umathchar  "3"0"0021A7 }
\def\mapsfrom                  {\Umathchar  "3"0"0021A4 }
\def\Mapsfrom                  {\Umathchar  "3"0"002906 }
\def\mapsfromchar              {\Umathchar  "3"0"0FE324 }
\def\mapsto                    {\Umathchar  "3"0"0021A6 }
\def\Mapsto                    {\Umathchar  "3"0"002907 }
\def\mapstochar                {\Umathchar  "3"0"0FE321 }
\def\mapsup                    {\Umathchar  "3"0"0021A5 }
\def\mathring                  {\Umathaccent"0"0"0002DA }
\def\measuredangle             {\Umathchar  "0"0"002221 }
\def\measuredeq                {\Umathchar  "3"0"00225E }
\def\mho                       {\Umathchar  "0"0"002127 }
\def\mid                       {\Umathchar  "3"0"00007C }
\def\minus                     {\Umathchar  "2"0"002212 }
\def\minuscolon                {\Umathchar  "2"0"002239 }
\def\models                    {\Umathchar  "3"0"0022A7 }
\def\mp                        {\Umathchar  "2"0"002213 }
\def\Mu                        {\Umathchar  "0"0"00039C }
\def\mu                        {\Umathchar  "0"0"0003BC }
\def\multimap                  {\Umathchar  "3"0"0022B8 }
\def\napprox                   {\Umathchar  "3"0"002249 }
\def\napproxEq                 {\Umathchar  "3"0"002246 }
\def\nasymp                    {\Umathchar  "3"0"00226D }
\def\natural                   {\Umathchar  "0"0"00266E }
\def\naturalnumbers            {\Umathchar  "0"0"002115 }
\def\ncong                     {\Umathchar  "3"0"002246 }
\def\ndivides                  {\Umathchar  "2"0"002224 }
\def\ne                        {\Umathchar  "3"0"002260 }
\def\nearrow                   {\Umathchar  "3"0"002197 }
\def\Nearrow                   {\Umathchar  "3"0"0021D7 }
\def\neg                       {\Umathchar  "0"0"0000AC }
\def\negativesign              {\Umathchar  "2"0"00207B }
\def\neq                       {\Umathchar  "3"0"002260 }
\def\nequiv                    {\Umathchar  "3"0"002262 }
\def\neswarrow                 {\Umathchar  "3"0"002922 }
\def\nexists                   {\Umathchar  "0"0"002204 }
\def\ngeq                      {\Umathchar  "3"0"002271 }
\def\ngtr                      {\Umathchar  "3"0"00226F }
\def\ngtrless                  {\Umathchar  "3"0"002279 }
\def\ngtrsim                   {\Umathchar  "3"0"002275 }
\def\nHdownarrow               {\Umathchar  "3"0"0021DF }
\def\nHuparrow                 {\Umathchar  "3"0"0021DE }
\def\ni                        {\Umathchar  "3"0"00220B }
\def\nin                       {\Umathchar  "3"0"002209 }
\def\nleftarrow                {\Umathchar  "3"0"00219A }
\def\nLeftarrow                {\Umathchar  "3"0"0021CD }
\def\nleftrightarrow           {\Umathchar  "3"0"0021AE }
\def\nLeftrightarrow           {\Umathchar  "3"0"0021CE }
\def\nleq                      {\Umathchar  "3"0"002270 }
\def\nless                     {\Umathchar  "3"0"00226E }
\def\nlessgtr                  {\Umathchar  "3"0"002278 }
\def\nlesssim                  {\Umathchar  "3"0"002274 }
\def\nmid                      {\Umathchar  "3"0"002224 }
\def\nni                       {\Umathchar  "3"0"00220C }
\def\not                       {\Umathchar  "3"0"000338 }
\def\notin                     {\Umathchar  "3"0"002209 }
\def\nowns                     {\Umathchar  "3"0"00220C }
\def\nparallel                 {\Umathchar  "3"0"002226 }
\def\nprec                     {\Umathchar  "3"0"002280 }
\def\npreccurlyeq              {\Umathchar  "3"0"0022E0 }
\def\nrightarrow               {\Umathchar  "3"0"00219B }
\def\nRightarrow               {\Umathchar  "3"0"0021CF }
\def\nsim                      {\Umathchar  "3"0"002241 }
\def\nsimeq                    {\Umathchar  "3"0"002244 }
\def\nsqsubseteq               {\Umathchar  "3"0"0022E2 }
\def\nsqsupseteq               {\Umathchar  "3"0"0022E3 }
\def\nsubset                   {\Umathchar  "3"0"002284 }
\def\nsubseteq                 {\Umathchar  "3"0"002288 }
\def\nsucc                     {\Umathchar  "3"0"002281 }
\def\nsucccurlyeq              {\Umathchar  "3"0"0022E1 }
\def\nsupset                   {\Umathchar  "3"0"002285 }
\def\nsupseteq                 {\Umathchar  "3"0"002289 }
\def\ntriangleleft             {\Umathchar  "3"0"0022EB }
\def\ntrianglelefteq           {\Umathchar  "3"0"0022EC }
\def\ntriangleright            {\Umathchar  "3"0"0022EA }
\def\ntrianglerighteq          {\Umathchar  "3"0"0022ED }
\def\Nu                        {\Umathchar  "0"0"00039D }
\def\nu                        {\Umathchar  "0"0"0003BD }
\def\nvdash                    {\Umathchar  "3"0"0022AC }
\def\nvDash                    {\Umathchar  "3"0"0022AD }
\def\nVdash                    {\Umathchar  "3"0"0022AE }
\def\nVDash                    {\Umathchar  "3"0"0022AF }
\def\nvleftarrow               {\Umathchar  "3"0"0021F7 }
\def\nVleftarrow               {\Umathchar  "3"0"0021FA }
\def\nvleftrightarrow          {\Umathchar  "3"0"0021F9 }
\def\nVleftrightarrow          {\Umathchar  "3"0"0021FC }
\def\nvrightarrow              {\Umathchar  "3"0"0021F8 }
\def\nVrightarrow              {\Umathchar  "3"0"0021FB }
\def\nwarrow                   {\Umathchar  "3"0"002196 }
\def\Nwarrow                   {\Umathchar  "3"0"0021D6 }
\def\nwsearrow                 {\Umathchar  "3"0"002921 }
\def\odot                      {\Umathchar  "2"0"002299 }
\def\ohm                       {\Umathchar  "0"0"002126 }
\def\oiiint                    {\Umathchar  "1"0"002230 }
\def\oiint                     {\Umathchar  "1"0"00222F }
\def\oint                      {\Umathchar  "1"0"00222E }
\def\ointclockwise             {\Umathchar  "1"0"002232 }
\def\ointctrclockwise          {\Umathchar  "1"0"002233 }
\def\Omega                     {\Umathchar  "0"0"0003A9 }
\def\omega                     {\Umathchar  "0"0"0003C9 }
\def\Omicron                   {\Umathchar  "0"0"00039F }
\def\omicron                   {\Umathchar  "0"0"0003BF }
\def\ominus                    {\Umathchar  "2"0"002296 }
\def\oplus                     {\Umathchar  "2"0"002295 }
\def\oslash                    {\Umathchar  "2"0"002298 }
\def\otimes                    {\Umathchar  "2"0"002297 }
\def\overbar                   {\Umathaccent"0"0"00203E }
\def\overbrace                 {\Umathaccent"0"0"0023DE }
\def\overbracket               {\Umathaccent"0"0"0023B4 }
\def\overparent                {\Umathaccent"0"0"0023DC }
\def\owns                      {\Umathchar  "3"0"00220B }
\def\P                         {\Umathchar  "0"0"0000B6 }
\def\parallel                  {\Umathchar  "3"0"002225 }
\def\partial                   {\Umathchar  "0"0"002202 }
\def\perp                      {\Umathchar  "3"0"0022A5 }
\def\Phi                       {\Umathchar  "0"0"0003A6 }
\def\phi                       {\Umathchar  "0"0"0003D5 }
\def\Pi                        {\Umathchar  "0"0"0003A0 }
\def\pi                        {\Umathchar  "0"0"0003C0 }
\def\pitchfork                 {\Umathchar  "3"0"0022D4 }
\def\Plankconst                {\Umathchar  "0"0"00210E }
\def\pm                        {\Umathchar  "2"0"0000B1 }
\def\positivesign              {\Umathchar  "2"0"00207A }
\def\prec                      {\Umathchar  "3"0"00227A }
\def\precapprox                {\Umathchar  "3"0"002AB7 }
\def\preccurlyeq               {\Umathchar  "3"0"00227C }
\def\preceq                    {\Umathchar  "3"0"002AAF }
\def\preceqq                   {\Umathchar  "3"0"002AB3 }
\def\precnapprox               {\Umathchar  "3"0"002AB9 }
\def\precneq                   {\Umathchar  "3"0"002AB1 }
\def\precneqq                  {\Umathchar  "3"0"002AB5 }
\def\precnsim                  {\Umathchar  "3"0"0022E8 }
\def\precsim                   {\Umathchar  "3"0"00227E }
\def\prime                     {\Umathchar  "0"0"002032 }
\def\primes                    {\Umathchar  "0"0"002119 }
\def\prod                      {\Umathchar  "1"0"00220F }
\def\PropertyLine              {\Umathchar  "0"0"00214A }
\def\propto                    {\Umathchar  "3"0"00221D }
\def\Psi                       {\Umathchar  "0"0"0003A8 }
\def\psi                       {\Umathchar  "0"0"0003C8 }
\def\questionedeq              {\Umathchar  "3"0"00225F }
\def\rangle                    {\Udelimiter "5"0"0027E9 }
\def\rationals                 {\Umathchar  "0"0"00211A }
\def\rbrace                    {\Udelimiter "5"0"00007D }
\def\rbrack                    {\Udelimiter "5"0"00005D }
\def\rceil                     {\Udelimiter "5"0"002309 }
\def\rceiling                  {\Udelimiter "5"0"002309 }
\def\Rdsh                      {\Umathchar  "3"0"0021B3 }
\def\Re                        {\Umathchar  "0"0"00211C }
\def\reals                     {\Umathchar  "0"0"00211D }
\def\Relbar                    {\Umathchar  "3"0"00003D }
\def\relbar                    {\Umathchar  "3"0"002212 }
\def\restriction               {\Umathchar  "3"0"0021BE }
\def\rfloor                    {\Udelimiter "5"0"00230B }
\def\rgroup                    {\Udelimiter "5"0"0027EF }
\def\Rho                       {\Umathchar  "0"0"0003A1 }
\def\rho                       {\Umathchar  "0"0"0003C1 }
\def\rhook                     {\Umathchar  "3"0"0FE323 }
\def\rhooknearrow              {\Umathchar  "3"0"002924 }
\def\rhookswarrow              {\Umathchar  "3"0"002926 }
\def\rightangle                {\Umathchar  "0"0"00221F }
\def\rightarrow                {\Umathchar  "3"0"002192 }
\def\Rightarrow                {\Umathchar  "3"0"0021D2 }
\def\rightarrowbar             {\Umathchar  "3"0"0021E5 }
\def\rightarrowtail            {\Umathchar  "3"0"0021A3 }
\def\rightarrowtriangle        {\Umathchar  "3"0"0021FE }
\def\rightdasharrow            {\Umathchar  "3"0"0021E2 }
\def\rightharpoondown          {\Umathchar  "3"0"0021C1 }
\def\rightharpoonup            {\Umathchar  "3"0"0021C0 }
\def\rightleftarrows           {\Umathchar  "3"0"0021C4 }
\def\rightleftharpoons         {\Umathchar  "3"0"0021CC }
\def\rightrightarrows          {\Umathchar  "3"0"0021C9 }
\def\rightsquigarrow           {\Umathchar  "3"0"0021DD }
\def\rightthreearrows          {\Umathchar  "3"0"0021F6 }
\def\rightthreetimes           {\Umathchar  "2"0"0022CC }
\def\rightwavearrow            {\Umathchar  "3"0"00219D }
\def\rightwhitearrow           {\Umathchar  "0"0"0021E8 }
\def\risingdotseq              {\Umathchar  "3"0"002253 }
\def\rmoustache                {\Udelimiter "5"0"0023B1 }
\def\rneq                      {\Umathchar  "3"0"002A88 }
\def\rparent                   {\Udelimiter "5"0"000029 }
\def\rrangle                   {\Udelimiter "5"0"0027EB }
\def\rrbracket                 {\Udelimiter "5"0"0027E7 }
\def\Rrightarrow               {\Umathchar  "3"0"0021DB }
\def\Rsh                       {\Umathchar  "3"0"0021B1 }
\def\rtimes                    {\Umathchar  "2"0"0022CA }
\def\rvert                     {\Udelimiter "5"0"00007C }
\def\rVert                     {\Udelimiter "5"0"002016 }
\def\S                         {\Umathchar  "0"0"0000A7 }
\def\searrow                   {\Umathchar  "3"0"002198 }
\def\Searrow                   {\Umathchar  "3"0"0021D8 }
\def\setminus                  {\Umathchar  "2"0"002216 }
\def\sharp                     {\Umathchar  "0"0"00266F }
\def\Sigma                     {\Umathchar  "0"0"0003A3 }
\def\sigma                     {\Umathchar  "0"0"0003C3 }
\def\sim                       {\Umathchar  "3"0"00223C }
\def\simeq                     {\Umathchar  "3"0"002243 }
\def\slash                     {\Umathchar  "0"0"002044 }
\def\smile                     {\Umathchar  "3"0"002323 }
\def\solidus                   {\Udelimiter "5"0"002044 }
\def\spadesuit                 {\Umathchar  "0"0"002660 }
\def\sphericalangle            {\Umathchar  "0"0"002222 }
\def\sqcap                     {\Umathchar  "2"0"002293 }
\def\sqcup                     {\Umathchar  "2"0"002294 }
\def\sqsubset                  {\Umathchar  "3"0"00228F }
\def\sqsubseteq                {\Umathchar  "2"0"002291 }
\def\sqsubsetneq               {\Umathchar  "3"0"0022E4 }
\def\sqsupset                  {\Umathchar  "3"0"002290 }
\def\sqsupseteq                {\Umathchar  "2"0"002292 }
\def\sqsupsetneq               {\Umathchar  "3"0"0022E5 }
\def\square                    {\Umathchar  "0"0"0025A1 }
\def\squaredots                {\Umathchar  "3"0"002237 }
\def\star                      {\Umathchar  "2"0"0022C6 }
\def\stareq                    {\Umathchar  "3"0"00225B }
\def\subset                    {\Umathchar  "3"0"002282 }
\def\Subset                    {\Umathchar  "3"0"0022D0 }
\def\subseteq                  {\Umathchar  "3"0"002286 }
\def\subseteqq                 {\Umathchar  "3"0"002AC5 }
\def\subsetneq                 {\Umathchar  "3"0"00228A }
\def\subsetneqq                {\Umathchar  "3"0"002ACB }
\def\succ                      {\Umathchar  "3"0"00227B }
\def\succapprox                {\Umathchar  "3"0"002AB8 }
\def\succcurlyeq               {\Umathchar  "3"0"00227D }
\def\succeq                    {\Umathchar  "3"0"002AB0 }
\def\succeqq                   {\Umathchar  "3"0"002AB4 }
\def\succnapprox               {\Umathchar  "3"0"002ABA }
\def\succneq                   {\Umathchar  "3"0"002AB2 }
\def\succneqq                  {\Umathchar  "3"0"002AB6 }
\def\succnsim                  {\Umathchar  "3"0"0022E9 }
\def\succsim                   {\Umathchar  "3"0"00227F }
\def\sum                       {\Umathchar  "1"0"002211 }
\def\supset                    {\Umathchar  "3"0"002283 }
\def\Supset                    {\Umathchar  "3"0"0022D1 }
\def\supseteq                  {\Umathchar  "3"0"002287 }
\def\supseteqq                 {\Umathchar  "3"0"002AC6 }
\def\supsetneq                 {\Umathchar  "3"0"00228B }
\def\supsetneqq                {\Umathchar  "3"0"002ACC }
\def\surd                      {\Umathchar  "2"0"00221A }
\def\swarrow                   {\Umathchar  "3"0"002199 }
\def\Swarrow                   {\Umathchar  "3"0"0021D9 }
\def\Tau                       {\Umathchar  "0"0"0003A4 }
\def\tau                       {\Umathchar  "0"0"0003C4 }
\def\therefore                 {\Umathchar  "3"0"002234 }
\def\Theta                     {\Umathchar  "0"0"000398 }
\def\theta                     {\Umathchar  "0"0"0003B8 }
\def\tilde                     {\Umathaccent"0"0"0002DC }
\def\times                     {\Umathchar  "2"0"0000D7 }
\def\to                        {\Umathchar  "3"0"002192 }
\def\top                       {\Umathchar  "0"0"0022A4 }
\def\triangle                  {\Umathchar  "0"0"0025B3 }
\def\triangledown              {\Umathchar  "2"0"0025BD }
\def\triangleleft              {\Umathchar  "2"0"0025C1 }
\def\triangleq                 {\Umathchar  "3"0"00225C }
\def\triangleright             {\Umathchar  "2"0"0025B7 }
\def\tripleprime               {\Umathchar  "0"0"002034 }
\def\turnediota                {\Umathchar  "0"0"002129 }
\def\twoheaddownarrow          {\Umathchar  "3"0"0021A1 }
\def\twoheadleftarrow          {\Umathchar  "3"0"00219E }
\def\twoheadrightarrow         {\Umathchar  "3"0"0021A0 }
\def\twoheadrightarrowtail     {\Umathchar  "3"0"002916 }
\def\twoheaduparrow            {\Umathchar  "3"0"00219F }
\def\udots                     {\Umathchar  "0"0"0022F0 }
\def\ulcorner                  {\Udelimiter "4"0"00231C }
\def\underbar          {\Umathaccent bottom "0"0"00203E }
\def\underbrace        {\Umathaccent bottom "0"0"0023DF }
\def\underbracket      {\Umathaccent bottom "0"0"0023B5 }
\def\underparent       {\Umathaccent bottom "0"0"0023DD }
\def\upand                     {\Umathchar  "2"0"00214B }
\def\uparrow                   {\Umathchar  "3"0"002191 }
\def\Uparrow                   {\Umathchar  "3"0"0021D1 }
\def\updasharrow               {\Umathchar  "3"0"0021E1 }
\def\updownarrow               {\Umathchar  "3"0"002195 }
\def\Updownarrow               {\Umathchar  "3"0"0021D5 }
\def\updownarrowbar            {\Umathchar  "0"0"0021A8 }
\def\updownarrows              {\Umathchar  "3"0"0021C5 }
\def\upharpoonleft             {\Umathchar  "3"0"0021BF }
\def\upharpoonright            {\Umathchar  "3"0"0021BE }
\def\uplus                     {\Umathchar  "2"0"00228E }
\def\Upsilon                   {\Umathchar  "0"0"0003A5 }
\def\upsilon                   {\Umathchar  "0"0"0003C5 }
\def\upuparrows                {\Umathchar  "3"0"0021C8 }
\def\upwhitearrow              {\Umathchar  "0"0"0021E7 }
\def\urcorner                  {\Udelimiter "5"0"00231D }
\def\Uuparrow                  {\Umathchar  "3"0"00290A }
\def\varepsilon                {\Umathchar  "0"0"0003B5 }
\def\varkappa                  {\Umathchar  "0"0"0003F0 }
\def\varkappa                  {\Umathchar  "0"0"0003F0 }
\def\varnothing                {\Umathchar  "0"0"002300 }
\def\varphi                    {\Umathchar  "0"0"0003C6 }
\def\varpi                     {\Umathchar  "0"0"0003D6 }
\def\varrho                    {\Umathchar  "0"0"01D71A }
\def\varsigma                  {\Umathchar  "0"0"0003C2 }
\def\vartheta                  {\Umathchar  "0"0"01D717 }
\def\varTheta                  {\Umathchar  "0"0"0003D1 }
\def\vdash                     {\Umathchar  "3"0"0022A2 }
\def\vDash                     {\Umathchar  "3"0"0022A8 }
\def\Vdash                     {\Umathchar  "3"0"0022A9 }
\def\VDash                     {\Umathchar  "3"0"0022AB }
\def\vdots                     {\Umathchar  "0"0"0022EE }
\def\vec                       {\Umathaccent"0"0"0020D7 }
\def\vee                       {\Umathchar  "2"0"002228 }
\def\veebar                    {\Umathchar  "2"0"0022BB }
\def\veeeq                     {\Umathchar  "3"0"00225A }
\def\vert                      {\Udelimiter "0"0"00007C }
\def\Vert                      {\Udelimiter "0"0"002016 }
\def\Vvdash                    {\Umathchar  "3"0"0022AA }
\def\wedge                     {\Umathchar  "2"0"002227 }
\def\wedgeeq                   {\Umathchar  "3"0"002259 }
\def\whitearrowupfrombar       {\Umathchar  "0"0"0021EB }
\def\widehat                   {\Umathaccent"0"0"000302 }
\def\widetilde                 {\Umathaccent"0"0"000303 }
\def\wp                        {\Umathchar  "0"0"002118 }
\def\wr                        {\Umathchar  "2"0"002240 }
\def\Xi                        {\Umathchar  "0"0"00039E }
\def\xi                        {\Umathchar  "0"0"0003BE }
\def\yen                       {\Umathchar  "0"0"0000A5 }
\def\Zeta                      {\Umathchar  "0"0"000396 }
\def\zeta                      {\Umathchar  "0"0"0003B6 }

%D The following are suggested by Bruno. As I don't use plain and as the above are
%D taken from text unicode greek I suppose his list is better:

\def\alpha                     {\Umathchar  "0"0"01D6FC }
\def\beta                      {\Umathchar  "0"0"01D6FD }
\def\chi                       {\Umathchar  "0"0"01D712 }
\def\delta                     {\Umathchar  "0"0"01D6FF }
\def\digamma                   {\Umathchar  "0"0"0003DC }
\def\epsilon                   {\Umathchar  "0"0"01D716 }
\def\eta                       {\Umathchar  "0"0"01D702 }
\def\gamma                     {\Umathchar  "0"0"01D6FE }
\def\iota                      {\Umathchar  "0"0"01D704 }
\def\kappa                     {\Umathchar  "0"0"01D705 }
\def\lambda                    {\Umathchar  "0"0"01D706 }
\def\mu                        {\Umathchar  "0"0"01D707 }
\def\nu                        {\Umathchar  "0"0"01D708 }
\def\omega                     {\Umathchar  "0"0"01D714 }
\def\omicron                   {\Umathchar  "0"0"01D70A }
\def\phi                       {\Umathchar  "0"0"01D719 }
\def\pi                        {\Umathchar  "0"0"01D70B }
\def\psi                       {\Umathchar  "0"0"01D713 }
\def\rho                       {\Umathchar  "0"0"01D70C }
\def\sigma                     {\Umathchar  "0"0"01D70E }
\def\tau                       {\Umathchar  "0"0"01D70F }
\def\theta                     {\Umathchar  "0"0"01D703 }
\def\upsilon                   {\Umathchar  "0"0"01D710 }
\def\varepsilon                {\Umathchar  "0"0"01D700 }
\def\varkappa                  {\Umathchar  "0"0"01D718 }
\def\varphi                    {\Umathchar  "0"0"01D711 }
\def\varpi                     {\Umathchar  "0"0"01D71B }
\def\varrho                    {\Umathchar  "0"0"01D71A }
\def\varsigma                  {\Umathchar  "0"0"01D70D }
\def\vartheta                  {\Umathchar  "0"0"01D717 }
\def\xi                        {\Umathchar  "0"0"01D709 }
\def\zeta                      {\Umathchar  "0"0"01D701 }

\def\varTheta                  {\Umathchar  "0"0"0003F4 }

% a few definitions:

\def\sqrt     {\Uroot "0 "221A{}}
\def\root#1\of{\Uroot "0 "221A{#1}}

% \skewchar\teni='177 \skewchar\seveni='177 \skewchar\fivei='177
% \skewchar\tensy='60 \skewchar\sevensy='60 \skewchar\fivesy='60

\chardef\% = "25
\chardef\& = "26
\chardef\# = "23
\chardef\$ = "24
\chardef\_ = "5F

\let\ss        ß
\let\ae        æ
\let\oe        œ
\let\o         ø
\let\AE        Æ
\let\OE        Œ
\let\O         Ø
\let\i         ı
\let\j         ȷ
\let\aa        å
\let\l         ł
\let\L         Ł
\let\AA        Å
\let\copyright ©
\let\S         §
\let\P         ¶
\let\dag       †
\let\ddag      ‡
\let\pounds    £

% just use utf

\def\`#1{#1^^^^0300}
\def\'#1{#1^^^^0301}
\def\^#1{#1^^^^0302}
\def\~#1{#1^^^^0303}
\def\=#1{#1^^^^0304}
\def\u#1{#1^^^^0306}
\def\.#1{#1^^^^0307}
\def\"#1{#1^^^^0308}
\def\r#1{#1^^^^030a} % not in plain
\def\H#1{#1^^^^030b}
\def\v#1{#1^^^^030c}
\def\d#1{#1^^^^0323}
\def\c#1{#1^^^^0327}
\def\k#1{#1^^^^0328} % not in plain
\def\b#1{#1^^^^0331}

% for Bruno, when he tests this file with xetex:

\ifdefined\directlua \else

    \catcode`@=11

    \def\sqrt{\Uradical "0 "221A }

    \def\root#1\of
      {\setbox\rootbox\hbox\bgroup
          $\m@th\scriptscriptstyle{#1}$%
       \egroup
       \mathpalette\r@@t}

    \catcode`@=12

\fi

\endinput
%
    %D \module
%D   [       file=luatex-fonts,
%D        version=2009.12.01,
%D          title=\LUATEX\ Support Macros,
%D       subtitle=Generic \OPENTYPE\ Font Handler,
%D         author=Hans Hagen,
%D      copyright={PRAGMA ADE \& \CONTEXT\ Development Team}]

%D Cf. discussion on \CONTEXT\ list:

% \savinghyphcodes1

\directlua {
    dofile(kpse.find_file("luatex-languages.lua","tex"))
}

\def\loadpatterns#1{\directlua{tex.language = languages.loadpatterns("#1")}}

\endinput
%
    %D \module
%D   [       file=luatex-mplib,
%D        version=2009.12.01,
%D          title=\LUATEX\ Support Macros,
%D       subtitle=\METAPOST\ to \PDF\ conversion,
%D         author=Taco Hoekwater \& Hans Hagen,
%D      copyright=see context related readme files]

%D This is the companion to the \LUA\ module \type {supp-mpl.lua}. Further
%D embedding is up to others. A simple example of usage in plain \TEX\ is:
%D
%D \starttyping
%D \pdfoutput=1
%D
%D \input luatex-mplib.tex
%D
%D \setmplibformat{plain}
%D
%D \mplibcode
%D   beginfig(1);
%D     draw fullcircle
%D       scaled 10cm
%D       withcolor red
%D       withpen pencircle xscaled 4mm yscaled 2mm rotated 30 ;
%D   endfig;
%D \endmplibcode
%D
%D \end
%D \stoptyping

\def\setmplibformat#1{\def\mplibformat{#1}}

\def\setupmplibcatcodes
  {\catcode`\{=12 \catcode`\}=12 \catcode`\#=12 \catcode`\^=12 \catcode`\~=12
   \catcode`\_=12 \catcode`\%=12 \catcode`\&=12 \catcode`\$=12 }

\def\mplibcode
  {\bgroup
   \setupmplibcatcodes
   \domplibcode}

\long\def\domplibcode#1\endmplibcode
  {\egroup
   \directlua{metapost.process('\mplibformat',[[#1]])}}

%D We default to \type {plain} \METAPOST:

\def\mplibformat{plain}

%D We use a dedicated scratchbox:

\ifx\mplibscratchbox\undefined \newbox\mplibscratchbox \fi

%D Now load the needed \LUA\ code.

\directlua{dofile(kpse.find_file('luatex-mplib.lua'))}

%D The following code takes care of encapsulating the literals:

\def\startMPLIBtoPDF#1#2#3#4%
  {\hbox\bgroup
   \xdef\MPllx{#1}\xdef\MPlly{#2}%
   \xdef\MPurx{#3}\xdef\MPury{#4}%
   \xdef\MPwidth{\the\dimexpr#3bp-#1bp\relax}%
   \xdef\MPheight{\the\dimexpr#4bp-#2bp\relax}%
   \parskip0pt%
   \leftskip0pt%
   \parindent0pt%
   \everypar{}%
   \setbox\mplibscratchbox\vbox\bgroup
   \noindent}

\def\stopMPLIBtoPDF
  {\egroup
   \setbox\mplibscratchbox\hbox
     {\hskip-\MPllx bp%
      \raise-\MPlly bp%
      \box\mplibscratchbox}%
   \setbox\mplibscratchbox\vbox to \MPheight
     {\vfill
      \hsize\MPwidth
      \wd\mplibscratchbox0pt%
      \ht\mplibscratchbox0pt%
      \dp\mplibscratchbox0pt%
      \box\mplibscratchbox}%
   \wd\mplibscratchbox\MPwidth
   \ht\mplibscratchbox\MPheight
   \box\mplibscratchbox
   \egroup}

%D The body of picture, except for text items, is taken care of by:

\ifnum\pdfoutput>0
    \let\MPLIBtoPDF\pdfliteral
\else
    \def\MPLIBtoPDF#1{\special{pdf:literal direct #1}} % not ok yet
\fi

%D Text items have a special handler:

\def\MPLIBtextext#1#2#3#4#5%
  {\begingroup
   \setbox\mplibscratchbox\hbox
     {\font\temp=#1 at #2bp%
      \temp
      #3}%
   \setbox\mplibscratchbox\hbox
     {\hskip#4 bp%
      \raise#5 bp%
      \box\mplibscratchbox}%
   \wd\mplibscratchbox0pt%
   \ht\mplibscratchbox0pt%
   \dp\mplibscratchbox0pt%
   \box\mplibscratchbox
   \endgroup}

\endinput
%
    %D \module
%D   [       file=luatex-gadgets,
%D        version=2015.05.12,
%D          title=\LUATEX\ Support Macros,
%D       subtitle=Useful stuff from articles,
%D         author=Hans Hagen,
%D           date=\currentdate,
%D      copyright={PRAGMA ADE \& \CONTEXT\ Development Team}]

\directlua{dofile(resolvers.findfile('luatex-gadgets.lua'))}

% optional removal of marked content
%
% before\marksomething{gone}{\em HERE}\unsomething{gone}after
% before\marksomething{kept}{\em HERE}\unsomething{gone}after
% \marksomething{gone}{\em HERE}\unsomething{gone}last
% \marksomething{kept}{\em HERE}\unsomething{gone}last

\def\setmarksignal  #1{\directlua{gadgets.marking.setsignal(\number#1)}}
\def\marksomething#1#2{{\directlua{gadgets.marking.mark("#1")}{#2}}}
\def\unsomething    #1{\directlua{gadgets.marking.remove("#1")}}

\newattribute\gadgetmarkattribute \setmarksignal\gadgetmarkattribute

\endinput
%
}

% We also patch the version number:

\edef\fmtversion{\fmtversion+luatex}

\automatichyphenmode=1

\dump
