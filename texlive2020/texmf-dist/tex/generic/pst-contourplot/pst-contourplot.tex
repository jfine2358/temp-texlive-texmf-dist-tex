%% Package `pst-contourplot.tex'
%%
%% This program can be redistributed and/or modified under
%% the terms of the LaTeX Project Public License Distributed
%% from CTAN archives in directory macros/latex/base/lppl.txt.
%%
%% DESCRIPTION:
%% `pst-contourplot'  Extension of PSTricks
%% for drawing implicit functions f(x,y)=0
%% This package uses the "marching squares" algorithm.
%%
%% Author  :   <manuel.luque27@gmail.com>
%%
\csname PSTCONTOURPLOTLoaded\endcsname
\let \PSTCONTOURPLOTLoaded\endinput
\ifx\PSTricksLoaded\endinput\else\input pstricks.tex\fi
\ifx\PSTXKeyLoaded\endinput \else\input pst-xkey.tex\fi
\def\fileversion{0.6}
\def\filedate{2018/07/14}
\message{`PST' v\fileversion, \filedate}
%
\edef\PstAtCode{\the\catcode`\@} \catcode`\@=11\relax
\pst@addfams{pst-contourplot}
\define@key[psset]{pst-contourplot}{a}{\edef\psk@contourplot@a{#1 }} %  dimension d'une cellule
%
\define@boolkey[psset]{pst-contourplot}[pst@]{grid}[true]{}
\define@boolkey[psset]{pst-contourplot}[Pst@]{Fill}[true]{}
\define@boolkey[psset]{pst-contourplot}[Pst@]{ReverseColors}[true]{}
\psset[pst-contourplot]{a=0.025,grid=false,Fill=false,ReverseColors=false}
% 04 juillet 2018
\define@boolkey[psset]{pst-contourplot}[Pst@]{Draw}[true]{}
\psset[pst-contourplot]{Draw=true}
% 6 juillet
\define@key[psset]{pst-contourplot}{function}{\edef\psk@contourplot@function{#1 }} %  la fonction implicite
\define@key[psset]{pst-contourplot}{ChoicePoints}{\edef\psk@contourplot@ChoicePoints{#1 }} %  choix des points pour les fl�ches
\psset[pst-contourplot]{function=,ChoicePoints=}
% 14 juillet
\define@boolkey[psset]{pst-contourplot}[Pst@]{WriteData}[true]{}
\psset[pst-contourplot]{WriteData=false}
\define@key[psset]{pst-contourplot}{FileName}{\edef\psk@contourplot@FileName{#1}} %  nom du fichier
\psset[pst-contourplot]{FileName=PointsCurve}
%
\def\psContourPlot{\def\pst@par{}\pst@object{psContourPlot}}
\def\psContourPlot@i(#1,#2)(#3,#4){% les limites
\begin@SpecialObj
\addto@pscode{%
100 dict begin
/cm {\pst@number\psunit mul } bind def % mise � l'�chelle
/Div { dup 0 eq { pop } { div } ifelse } def  	% control the Division
% cote du carre-cellule
/a \psk@contourplot@a def % en pts
 \ifPst@algebraic
      /fonction (\psk@contourplot@function) tx@AlgToPs begin AlgToPs end cvx def
    \else
      /fonction  { \psk@contourplot@function } def
    \fi
    /function {2 dict begin
      /y exch def
      /x exch def
      fonction end
    } def
\ifPst@ReverseColors /flag 0 def \else /flag 1 def \fi
/tabCels [
#1 a #3 {/x exch def
#2 a #4 {/y exch def
[ x y
  x a add y
  x a add y a add
  x y a add]
} for
} for
] def
 % indexer les sommets avec leur poids
 % 0 si f(x,y)<=0
 % 1 si f(x,y)>0
/calcValeur {
  function
  0 le {(0)}{(1)} ifelse
 } def
% https://en.wikibooks.org/wiki/PostScript_FAQ
/concatstringarray  %  [(a) (b) ... (z)] --> (ab...z)
    { 0 1 index { length add } forall string
      0 3 2 roll
        { 3 copy putinterval
          length add
        }
      forall pop
    } bind def
/interpolate { % x2 y2 x1 y1 interpolation lineaire => x y
7 dict begin
    /y1 exch def /x1 exch def
    /y2 exch def /x2 exch def
    /v1 x1 y1 function def
    /v2 x2 y2 function def
    /delta v1 v2 v1 sub Div def
    x1 delta x2 x1 sub mul sub % x
    y1 delta y2 y1 sub mul sub % y
  end
} def
/tabIndex [
#1 a #3 {/x exch def
#2 a #4 {/y exch def
[
x y                 calcValeur
x a add y 			calcValeur
x a add y a add     calcValeur
x y a add			calcValeur
]
concatstringarray
} for
} for
] def
%%%%%%%%%%%%%%%%%%%%%%%%%%%%%%%%%%%%%%%%%%%%
\ifPst@Fill
gsave
1 setlinejoin
\pst@usecolor\psfillcolor
0 1 tabCels length 1 sub {/i exch def
/valuesVertice tabIndex i get def
/cellule tabCels i get def
% les sommets
/S0 {cellule 0 2 getinterval aload pop } def
/S1 {cellule 2 2 getinterval aload pop } def
/S2 {cellule 4 2 getinterval aload pop } def
/S3 {cellule 6 2 getinterval aload pop } def
flag 0 eq {valuesVertice (1111) eq {
S0 /y exch def /x exch def
/part {
newpath
x cm y cm moveto
a cm 0 rlineto
0 a cm rlineto
a neg cm 0 rlineto
closepath
} def
part fill part stroke
 } if
 }{
valuesVertice (0000) eq  {
S0 /y exch def /x exch def
/part {
newpath
x cm y cm moveto
a cm 0 rlineto
0 a cm rlineto
a neg cm 0 rlineto
closepath
} def
part fill part stroke
} if
} ifelse
%%%%%%%%%%%%%%%%%%%%%%%%%%%%%%%%%%%%%%%%%%%%
valuesVertice (1000) eq {
   S0 S1 interpolate /yA exch def /xA exch def
   S0 S3 interpolate /yB exch def /xB exch def
   newpath
flag 1 eq {
   xA cm yA cm moveto
   S1 cm exch cm exch lineto
   S2 cm exch cm exch lineto
   S3 cm exch cm exch lineto
   xB cm yB cm lineto
   }{
   S0 cm exch cm exch moveto
   xA cm yA cm lineto
   xB cm yB cm lineto
   } ifelse
   closepath
   fill
   } if
valuesVertice (0100) eq {
   S1 S0 interpolate /yA exch def /xA exch def
   S1 S2 interpolate /yB exch def /xB exch def
   newpath
flag 1 eq {
   xA cm yA cm moveto
   xB cm yB cm lineto
   S2 cm exch cm exch lineto
   S3 cm exch cm exch lineto
   S0 cm exch cm exch lineto
    }{
   xA cm yA cm moveto
   xB cm yB cm lineto
   S1 cm exch cm exch lineto
    }ifelse
   closepath
   fill
   } if
valuesVertice (1100) eq {
   S1 S2 interpolate /yA exch def /xA exch def
   S0 S3 interpolate /yB exch def /xB exch def
   newpath
flag 1 eq {
   xA cm yA cm moveto
   S2 cm exch cm exch lineto
   S3 cm exch cm exch lineto
   xB cm yB cm lineto
   }{
   S0 cm exch cm exch moveto
   S1 cm exch cm exch lineto
   xA cm yA cm lineto
   xB cm yB cm lineto
   }ifelse
   closepath
   fill
   } if
valuesVertice (0001) eq {
   S3 S0 interpolate /yA exch def /xA exch def
   S3 S2 interpolate /yB exch def /xB exch def
   newpath
flag 1 eq {
   xA cm yA cm moveto
   S0 cm exch cm exch lineto
   S1 cm exch cm exch lineto
   S2 cm exch cm exch lineto
   xB cm yB cm lineto
   }{
   xA cm yA cm moveto
   xB cm yB cm lineto
   S3 cm exch cm exch lineto
   } ifelse
   closepath
   fill
   } if
valuesVertice (1001) eq {
   S0 S1 interpolate /yA exch def /xA exch def
   S3 S2 interpolate /yB exch def /xB exch def
   newpath
flag 1 eq {
   xA cm yA cm moveto
   S1 cm exch cm exch lineto
   S2 cm exch cm exch lineto
   xB cm yB cm lineto
   }{
   xA cm yA cm moveto
   xB cm yB cm lineto
   S3 cm exch cm exch lineto
   S0 cm exch cm exch lineto
   } ifelse
   closepath
   fill
   } if
valuesVertice (0101) eq { % ambig�it�
% lever l'ambig�it�
% calculer les cccordonn�es du centre de la cellule
S0 /y0 exch def /x0 exch def
S2 /y2 exch def /x2 exch def
/xI x0 x2 add 2 div def
/yI y0 y2 add 2 div def
   S0 S1 interpolate /yC exch def /xC exch def
   S0 S3 interpolate /yD exch def /xD exch def
   S2 S1 interpolate /yB exch def /xB exch def
   S2 S3 interpolate /yA exch def /xA exch def
xI yI function 0 flag 0 eq {ge}{le} ifelse {
  newpath
  xA cm yA cm moveto
  xB cm yB cm lineto
  S2 cm exch cm exch lineto
   closepath
   fill
  newpath
  xD cm yD cm moveto
  S0 cm exch cm exch lineto
  xC cm yC cm lineto
   closepath
   fill
  }{
  newpath
  xA cm yA cm moveto
  xB cm yB cm lineto
  S1 cm exch cm exch lineto
  xC cm yC cm lineto
  S0 cm exch cm exch lineto
  xD cm yD cm lineto
   closepath
   fill
   } ifelse
   } if
valuesVertice (1101) eq {
   S2 S1 interpolate /yA exch def /xA exch def
   S2 S3 interpolate /yB exch def /xB exch def
   newpath
flag 1 eq {
   xA cm yA cm moveto
   S2 cm exch cm exch lineto
   xB cm yB cm lineto
   }{
   xA cm yA cm moveto
   xB cm yB cm lineto
   S3 cm exch cm exch lineto
   S0 cm exch cm exch lineto
   S1 cm exch cm exch lineto
   }ifelse
   closepath
   fill
   } if
valuesVertice (0010) eq {
   S2 S1 interpolate /yA exch def /xA exch def
   S2 S3 interpolate /yB exch def /xB exch def
   newpath
flag 1 eq {
   S0 cm exch cm exch moveto
   S1 cm exch cm exch lineto
   xA cm yA cm lineto
   xB cm yB cm lineto
   S3 cm exch cm exch lineto
   }{
   xA cm yA cm moveto
   S2 cm exch cm exch lineto
   xB cm yB cm lineto
   } ifelse
   closepath
   fill
   } if
valuesVertice (1010) eq {  % ambig�it�
% lever l'ambig�it�
% calculer les cccordonn�es du centre de la cellule
S0 /y0 exch def /x0 exch def
S2 /y2 exch def /x2 exch def
/xI x0 x2 add 2 div def
/yI y0 y2 add 2 div def
   S0 S1 interpolate /yC exch def /xC exch def
   S0 S3 interpolate /yD exch def /xD exch def
   S2 S1 interpolate /yB exch def /xB exch def
   S2 S3 interpolate /yA exch def /xA exch def
xI yI function 0 flag 0 eq {ge}{le} ifelse {
  newpath
  xA cm yA cm moveto
  xB cm yB cm lineto
  S2 cm exch cm exch lineto
   closepath
%   fill
  newpath
  xD cm yD cm moveto
  S0 cm exch cm exch lineto
  xC cm yC cm lineto
   closepath
   fill
 }{
  newpath
  xA cm yA cm moveto
  xB cm yB cm lineto
  S1 cm exch cm exch lineto
  xC cm yC cm lineto
  S0 cm exch cm exch lineto
  xD cm yD cm lineto
   closepath
   fill
   } ifelse
   } if
valuesVertice (0110) eq {
   S0 S1 interpolate /yA exch def /xA exch def
   S3 S2 interpolate /yB exch def /xB exch def
   newpath
flag 1 eq {
   S0 cm exch cm exch moveto
   xA cm yA cm lineto
   xB cm yB cm lineto
   S3 cm exch cm exch lineto
   }{
   xA cm yA cm moveto
   S1 cm exch cm exch lineto
   S2 cm exch cm exch lineto
   xB cm yB cm lineto
   } ifelse
   closepath
   fill
   } if
valuesVertice (1110) eq { % OK
   S3 S0 interpolate /yA exch def /xA exch def
   S3 S2 interpolate /yB exch def /xB exch def
   newpath
flag 1 eq {
   xA cm yA cm  moveto
   xB cm yB cm lineto
   S3 cm exch cm exch lineto
   }{
   xA cm yA cm  moveto
   xB cm yB cm lineto
   S2 cm exch cm exch lineto
   S1 cm exch cm exch lineto
   S0 cm exch cm exch lineto
   } ifelse
   closepath
   fill
   } if
valuesVertice (0011) eq {
   S0 S3 interpolate /yA exch def /xA exch def
   S1 S2 interpolate /yB exch def /xB exch def
   newpath
flag 1 eq {
   S0 cm exch cm exch moveto
   S1 cm exch cm exch lineto
   xB cm yB cm lineto
   xA cm yA cm lineto
   }{
   xA cm yA cm moveto
   xB cm yB cm lineto
   S2 cm exch cm exch lineto
   S3 cm exch cm exch lineto
   } ifelse
   closepath
   fill
   } if
valuesVertice (1011) eq {
   S1 S0 interpolate /yA exch def /xA exch def
   S1 S2 interpolate /yB exch def /xB exch def
   newpath
flag 1 eq {
   xA cm yA cm  moveto
   S1 cm exch cm exch lineto
   xB cm yB cm lineto
   }{
   xA cm yA cm  moveto
   xB cm yB cm lineto
   S2 cm exch cm exch lineto
   S3 cm exch cm exch lineto
   S0 cm exch cm exch lineto
   } ifelse
   closepath
   fill
   } if
valuesVertice (0111) eq {
   S0 S1 interpolate /yA exch def /xA exch def
   S0 S3 interpolate /yB exch def /xB exch def
   newpath
flag 1 eq {
   S0 cm exch cm exch moveto
   xA cm yA cm lineto
   xB cm yB cm lineto
   }{
   xA cm yA cm moveto
   S1 cm exch cm exch lineto
   S2 cm exch cm exch lineto
   S3 cm exch cm exch lineto
   xB cm yB cm lineto
   } ifelse
   closepath
   fill
   } if
 }for
grestore
\fi
% la grille
\ifpst@grid
gsave
0.25 setlinewidth
0.5 setgray
#1 a #3 {/x exch def
 x cm #2 cm moveto
 x cm #4 cm lineto
 stroke
 } for
#2 a #4 {/y exch def
 #1 cm y cm moveto
 #3 cm y cm lineto
 stroke
 } for
newpath
#1 cm #2 cm moveto
#3 cm #2 cm lineto
#3 cm #4 cm lineto
#1 cm #4 cm lineto
closepath
stroke
grestore
\fi
% collecter un tableau de valeurs
/LesPoints [
0 1 tabCels length 1 sub {/i exch def
/valuesVertice tabIndex i get def
/cellule tabCels i get def
% les sommets
/S0 {cellule 0 2 getinterval aload pop } def
/S1 {cellule 2 2 getinterval aload pop } def
/S2 {cellule 4 2 getinterval aload pop } def
/S3 {cellule 6 2 getinterval aload pop } def
valuesVertice (0000) eq valuesVertice (1111) eq or {} if
valuesVertice (1000) eq {
   [S0 S1 interpolate cm exch cm exch]
   [S0 S3 interpolate cm exch cm exch]
   } if
valuesVertice (0100) eq {
   [S1 S0 interpolate cm exch cm exch]
   [S1 S2 interpolate cm exch cm exch]
   } if
valuesVertice (1100) eq { % OK
   [S1 S2 interpolate cm exch cm exch]
   [S0 S3 interpolate cm exch cm exch]
   } if
valuesVertice (0001) eq {
   [S3 S0 interpolate cm exch cm exch]
   [S3 S2 interpolate cm exch cm exch]
   } if
valuesVertice (1001) eq {
   [S0 S1 interpolate cm exch cm exch]
   [S3 S2 interpolate cm exch cm exch]
   } if
valuesVertice (0101) eq { % ambig�it�
% lever l'ambig�it�
% calculer les coordonn�es du centre de la cellule
S0 /y0 exch def /x0 exch def
S2 /y2 exch def /x2 exch def
/xI x0 x2 add 2 div def
/yI y0 y2 add 2 div def
xI yI function 0 ge {
   [S0 S1 interpolate cm exch cm exch]
   [S0 S3 interpolate cm exch cm exch]
   [S2 S1 interpolate cm exch cm exch]
   [S2 S3 interpolate cm exch cm exch]
  }{
   [S1 S0 interpolate cm exch cm exch]
   [S1 S2 interpolate cm exch cm exch]
   [S3 S2 interpolate cm exch cm exch]
   [S3 S0 interpolate cm exch cm exch]
   } ifelse
   } if
valuesVertice (1101) eq { % OK
   [S2 S1 interpolate cm exch cm exch]
   [S2 S3 interpolate cm exch cm exch]
   } if
valuesVertice (0010) eq { %OK
   [S2 S1 interpolate cm exch cm exch]
   [S2 S3 interpolate cm exch cm exch]
   } if
valuesVertice (1010) eq {  % ambig�it�
% lever l'ambig�it�
% calculer les coordonn�es du centre de la cellule
S0 /y0 exch def /x0 exch def
S2 /y2 exch def /x2 exch def
/xI x0 x2 add 2 div def
/yI y0 y2 add 2 div def
xI yI function 0 ge {
   [S1 S0 interpolate cm exch cm exch]
   [S1 S2 interpolate cm exch cm exch]
   [S3 S2 interpolate cm exch cm exch]
   [S3 S0 interpolate cm exch cm exch]
   }{
% or
   [S0 S1 interpolate cm exch cm exch]
   [S0 S3 interpolate cm exch cm exch]
   [S2 S1 interpolate cm exch cm exch]
   [S2 S3 interpolate cm exch cm exch]
   } ifelse
   } if
valuesVertice (0110) eq {
   [S0 S1 interpolate cm exch cm exch]
   [S3 S2 interpolate cm exch cm exch]
   } if
valuesVertice (1110) eq { % OK
   [S3 S0 interpolate cm exch cm exch]
   [S3 S2 interpolate cm exch cm exch]
   } if
valuesVertice (0011) eq {
   [S0 S3 interpolate cm exch cm exch]
   [S1 S2 interpolate cm exch cm exch]
   } if
valuesVertice (1011) eq {
   [S1 S0 interpolate cm exch cm exch]
   [S1 S2 interpolate cm exch cm exch]
   } if
valuesVertice (0111) eq {
   [S0 S1 interpolate cm exch cm exch]
   [S0 S3 interpolate cm exch cm exch]
   } if
} for
 ] def
\ifPst@WriteData
(\psk@contourplot@FileName.dat) (w) file /filePoints exch def
0 1 LesPoints length 1 sub {/i exch def
  LesPoints i get aload pop /y exch def /x exch def
    filePoints 91 write %% [
  filePoints  x 15 string cvs writestring
  filePoints  32 write %% espace
  filePoints  y 15 string cvs writestring
    filePoints 93 write %% ]
  filePoints  10 write %% CR
  } for
 filePoints closefile
\fi
% le dessin
\ifshowpoints
0 1 LesPoints length 1 sub {/i exch def
  LesPoints i get aload pop \psk@@dotsize \psk@@@dotsize CLW mul add 2 div 0 360 arc fill
  } for
\fi
\ifPst@Draw
0 2 LesPoints length 2 sub {/i exch def
  LesPoints i get aload pop moveto LesPoints i 1 add get aload pop lineto
  stroke
  } for
\fi
% fl�cher
% dessin d'une fl�che
/CLW /currentlinewidth load def
/fleche {
 4 dict begin
 /y1 exch def /x1 exch def
 /y2 exch def /x2 exch def
  gsave
  x1 y1 moveto
  y2 y1 sub x2 x1 sub Atan rotate
   1 CLW mul  2 CLW mul rlineto
  -5 CLW mul -2 CLW mul rlineto
   5 CLW mul -2 CLW mul rlineto
  closepath
  fill
  grestore
 end
} def
/ChoicePoints [\psk@contourplot@ChoicePoints] def
ChoicePoints length 0 ne {
0 1 ChoicePoints length 1 sub {/i exch def
/UnPoint ChoicePoints i get def
 UnPoint abs 2 mod 1 eq {/UnPoint UnPoint 1 sub def} if
UnPoint LesPoints length ge {/UnPoint LesPoints length 5 sub def} if
UnPoint LesPoints length neg le {/UnPoint LesPoints length 4 sub neg def} if
UnPoint 0 ge {
 LesPoints UnPoint get aload pop
 LesPoints UnPoint 1 add get aload pop
 fleche
} {
 LesPoints UnPoint abs 1 add get aload pop
 LesPoints UnPoint abs get aload pop
 fleche
} ifelse
} for
} if
end
 }% fin du code ps
 \end@SpecialObj
 }% % fin de la commande PSTricks
%
% read data
%
\def\psReadData{\def\pst@par{}\pst@object{psReadData}}
\def\psReadData@i{%
\begin@SpecialObj
\addto@pscode{%
 /LesPoints [(\psk@contourplot@FileName.dat) run ] def
% le dessin
\ifshowpoints
0 1 LesPoints length 1 sub {/i exch def
  LesPoints i get aload pop \psk@@dotsize \psk@@@dotsize CLW mul add 2 div 0 360 arc fill
  } for
\fi
\ifPst@Draw
0 2 LesPoints length 2 sub {/i exch def
  LesPoints i get aload pop moveto LesPoints i 1 add get aload pop lineto
  stroke
  } for
\fi
% fl�cher
% dessin d'une fl�che
/CLW /currentlinewidth load def
/fleche {
 4 dict begin
 /y1 exch def /x1 exch def
 /y2 exch def /x2 exch def
  gsave
  x1 y1 moveto
  y2 y1 sub x2 x1 sub Atan rotate
   1 CLW mul  2 CLW mul rlineto
  -5 CLW mul -2 CLW mul rlineto
   5 CLW mul -2 CLW mul rlineto
  closepath
  fill
  grestore
 end
} def
/ChoicePoints [\psk@contourplot@ChoicePoints] def
ChoicePoints length 0 ne {
0 1 ChoicePoints length 1 sub {/i exch def
/UnPoint ChoicePoints i get def
 UnPoint abs 2 mod 1 eq {/UnPoint UnPoint 1 sub def} if
 UnPoint LesPoints length ge {/UnPoint LesPoints length 4 sub def} if
 UnPoint LesPoints length neg le {/UnPoint LesPoints length 4 sub neg def} if
UnPoint 0 ge {
 LesPoints UnPoint get aload pop
 LesPoints UnPoint 1 add get aload pop
 fleche
} {
 LesPoints UnPoint abs 1 add get aload pop
 LesPoints UnPoint abs get aload pop
 fleche
} ifelse
} for
} if
 }% fin du code ps
 \end@SpecialObj
 }% % fin de la commande PSTricks
\catcode`\@=\PstAtCode\relax
\endinput 