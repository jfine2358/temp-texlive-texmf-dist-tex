%% macros for TeX Gyre Pagella and math fonts from pxfonts package
%% Jacek Mierczy\'nski, Staszek Wawrykiewicz, ver. 0.95 (03.02.2007) 
%% Public domain.
%% -----
% basic unit, can be defined before \input qtxmath.tex and changed to dd ;-)
\ifx\PT\undefined\def\PT{pt}\fi
\ifx\altg\undefined\def\altg{}\fi
\ifx\fontenc\undefined\def\fontenc{qx}\fi
% -- fonts %%% to remember: (qx-qplr) and sc !!!
\font\tenrm=\fontenc-qplr at10\PT \font\sevenrm=\fontenc-qplr at7\PT 
\font\fiverm=\fontenc-qplr at5\PT \font\eightrm=\fontenc-qplr at8\PT 
\font\ninerm=\fontenc-qplr at9\PT \font\sixrm=\fontenc-qplr at6\PT
%
\font\teni=qpxmi\altg\space at10\PT 
\font\seveni=qpxmi\altg\space at7\PT 
\font\fivei=qpxmi\altg\space at5\PT
\font\sixi=qpxmi\altg\space  at6\PT  
\font\eighti=qpxmi\altg\space  at8\PT 
\font\ninei=qpxmi\altg\space  at9\PT
%
\font\tenbf=\fontenc-qplb at10\PT \font\sevenbf=\fontenc-qplb at7\PT 
\font\fivebf=\fontenc-qplb at5\PT \font\sixbf=\fontenc-qplb at6\PT  
\font\eightbf=\fontenc-qplb at8\PT \font\ninebf=\fontenc-qplb at9\PT
%
\font\tenit=\fontenc-qplri at10\PT \font\nineit=\fontenc-qplri at9\PT 
\font\eightit=\fontenc-qplri at8\PT
%
\font\tensy=pxsy at10\PT \font\sevensy=pxsy at7\PT \font\fivesy=pxsy at5\PT
\font\eightsy=pxsy at8\PT \font\ninesy=pxsy at9\PT
\font\sixsy=pxsy at6\PT
\font\tenex=pxex at10\PT \font\eightex=pxex at9\PT \font\nineex=pxex at9\PT
%
\font\tencsc=\fontenc-qplr-sc at10\PT \font\eightcsc=\fontenc-qplr-sc at8\PT
\font\ninecsc=\fontenc-qplr-sc at9\PT
%
\skewchar\teni='177 \skewchar\ninei='177 \skewchar\eighti='177
\skewchar\seveni='177 \skewchar\fivei='177
\skewchar\tensy='60 \skewchar\ninesy='60 \skewchar\eightsy='60
\skewchar\sevensy='60 \skewchar\fivesy='60
%
% --- 8point ---
\def\eightpoint{\def\rm{\fam0\eightrm}%
\textfont0\eightrm \scriptfont0=\sixrm \scriptscriptfont0=\fiverm
\textfont1\eighti  \scriptfont1=\sixi  \scriptscriptfont1=\fivei
\textfont2\eightsy \scriptfont2=\sixsy   \scriptscriptfont2=\fivesy
\textfont3\eightex \scriptfont3=\eightex \scriptscriptfont3=\eightex
\def\mit{\fam1}\def\oldstyle{\fam1\eighti}\def\cal{\fam2}%
\textfont\itfam=\eightit \def\it{\fam\itfam\eightit}%
\textfont\bffam=\eightbf \scriptfont\bffam=\sixbf\scriptscriptfont\bffam=\fivebf
\def\bf{\fam\bffam\eightbf}%
%\def\tt{\fam\ttfam\eighttt}%
%\textfont\ttfam=\eighttt
%%\tt \ttglue=.5em plus.25em minus.15em
\let\sc=\eightcsc
\normalbaselineskip=10\PT
\setbox\strutbox=\hbox{\vrule height7.6\PT depth 2.4\PT width0pt}%
\normalbaselines\rm}
%% --- 9point ---
\def\ninepoint{\def\rm{\fam0\ninerm}%
\textfont0=\ninerm \scriptfont0=\sixrm \scriptscriptfont0=\fiverm
\textfont1=\ninei  \scriptfont1=\sixi  \scriptscriptfont1=\fivei
\textfont2=\ninesy \scriptfont2=\sixsy  \scriptscriptfont2=\fivesy
\textfont3=\nineex \scriptfont3=\nineex \scriptscriptfont3=\nineex
\def\mit{\fam1}\def\oldstyle{\fam1\ninei}\def\cal{\fam2}%
\textfont\itfam=\nineit \def\it{\fam\itfam\nineit}%
\textfont\bffam=\ninebf \scriptfont\bffam=\sixbf \scriptscriptfont\bffam=\fivebf
\def\bf{\fam\bffam\ninebf}%
\let\sc=\ninecsc
%\def\tt{\fam\ttfam\ninett}%
%\textfont\ttfam=\ninett
%%\tt \ttglue=.5em plus.25em minus.15em
\normalbaselineskip=11\PT
\setbox\strutbox=\hbox{\vrule height8\PT depth3\PT width0pt}%
\normalbaselines\rm}
% --- 10point ---
\def\tenpoint{\def\rm{\fam0\tenrm}%
\textfont0=\tenrm \scriptfont0=\sevenrm \scriptscriptfont0=\fiverm
\textfont1=\teni  \scriptfont1=\seveni  \scriptscriptfont1=\fivei
\textfont2=\tensy \scriptfont2=\sevensy \scriptscriptfont2=\fivesy
\textfont3=\tenex \scriptfont3=\tenex \scriptscriptfont3=\tenex
\def\mit{\fam1 }\def\oldstyle{\fam1\teni}\def\cal{\fam2}%
\def\it{\fam\itfam\tenit}%
\textfont\itfam=\tenit
%\def\sl{\fam\slfam\tensl} % \sl is family 5
%\textfont\slfam=\tensl
\def\bf{\fam\bffam\tenbf}%
\textfont\bffam=\tenbf \scriptfont\bffam=\sevenbf
\scriptscriptfont\bffam=\fivebf
\let\sc=\tencsc
%\textfont\ttfam=\tentt
%\def\tt{\fam\ttfam\tentt}
%%\tt \ttglue=.5em plus.25em minus.15em
\normalbaselineskip=12\PT plus.1\PT minus.1\PT
\setbox\strutbox=\hbox{\vrule height8.8\PT depth3.6\PT width0pt}%
\normalbaselines\rm}
\tenpoint %% default
%% ------
\mathcode`\+="218B
\mathcode`\=="318C
\mathchardef\Gamma="0180
\mathchardef\Delta="0181
\mathchardef\Theta="0182
\mathchardef\Lambda="0183
\mathchardef\Xi="0184
\mathchardef\Pi="0185
\mathchardef\Sigma="0186
\mathchardef\Upsilon="0187
\mathchardef\Phi="0188
\mathchardef\Psi="0189
\mathchardef\Omega="018A
\mathchardef\xleq="318E
\mathchardef\xgeq="318F
%% dodatkowe symbole (additional symbols)
\mathchardef\varGamma="0100
\mathchardef\varDelta="0101
\mathchardef\varTheta="0102
\mathchardef\varLambda="0103
\mathchardef\varXi="0104
\mathchardef\varPi="0105
\mathchardef\varSigma="0106
\mathchardef\varUpsilon="0107
\mathchardef\varPhi="0108
\mathchardef\varPsi="0109
\mathchardef\varOmega="010A
\mathchardef\varg="0190
%%
\def\ecmathaccdef{%
\def\acute{\mathaccent"7001 }
\def\grave{\mathaccent"7000 }
\def\ddot{\mathaccent"7004 }
\def\tilde{\mathaccent"7003 }
\def\bar{\mathaccent"7009 }
\def\breve{\mathaccent"7008 }
\def\check{\mathaccent"7007 }
\def\hat{\mathaccent"7002 }
\def\dot{\mathaccent"700A }
}
%
\def\qxfontenc{qx}\def\rmfontenc{rm}\def\csfontenc{cs}
%
\ifx\fontenc\qxfontenc\else
   \ifx\fontenc\rmfontenc\else
      \ifx\fontenc\csfontenc\else
      \ecmathaccdef
      \fi
   \fi
\fi

\endinput
