%D \module
%D   [      file=simpleslides-s-ThickStripes,
%D        version=2009.03.30
%D          title=\CONTEXT\ Style File,
%D       subtitle=Presentation Module --- Thick Stripes style,
%D         author=Aditya Mahajan and Thomas A. Schmitz,
%D           date=\currentdate,
%D      copyright={Aditya Mahajan and Thomas A. Schmitz}]
%C
%C Copyright 2007 Aditya Mahajan and Thomas A. Schmitz
%C This file may be distributed under the GNU General Public License v. 2.0.

%D This file provides the \quotation{ThickStripes} style for the presentation
%D module. It is loaded at runtime. The theme for this style is inspired by the
%D \quotation{Berkeley} theme of the \LaTeX\ Beamer package.

\writestatus{simpleslides}{loading Thick Stripes style}

\startmodule[simpleslides-s-ThickStripes]

\unprotect

%D First, we change the page layout to have more space on the top and the
%D left.

\setuplayout [width=fit,
              leftmargin=1.5cm,
	      rightmargin=0cm,
              leftmargindistance=.9cm,
              rightmargindistance=0pt,
              height=fit, 
              header=2.5cm, 
              footer=0cm, 
              topspace=.4cm, 
              backspace=2.9cm,
	      cutspace=2.8cm,
              bottomspace=0cm,
              bottom=0pt,
              location=singlesided]

%D These macros are used for placing figures/pictures:

\define\NormalHeight        {\textheight}
\define\NormalWidth         {.476\textwidth}
\define\PictureFrameHeight  {\textheight}
\define\PictureFrameWidth   {.476\textwidth}

\setuplayer
  [simpleslides:layer:slidetitle]
  [x=29mm]

%D We define our color scheme

\definecolor [simpleslides:backgroundcolor]   [s=.9]
\definecolor [simpleslides:variantcolor]      [r=.15,g=.15,b=.525]
\definecolor [simpleslides:contrastcolor]     [r=.2,g=.2,b=.7]
\definecolor [simpleslides:altcontrastcolor] [s=.4]
\definecolor [simpleslides:itemize:color]     [simpleslides:backgroundcolor]


%D We use \METAPOST to draw the background. The background consists of two
%D stripes and a "clock" at the bottom.

\startuseMPgraphic{simpleslides:MP:title}
StartPage ;
fill Page withcolor \MPcolor{simpleslides:backgroundcolor} ;
StopPage ;
\stopuseMPgraphic

\startuseMPgraphic{simpleslides:MP:ornament}
StartPage ;
save a, b,  theta;
numeric a, b,  theta ;

save t ;
pair  t[] ;

a = 2.5cm ;
b = 1.5cm ;

z[1] = ulcorner Page shifted (a,0) ;
z[2] = llcorner Page shifted (a,0) ;
z[3] = ulcorner Page shifted (0,-a) ;
z[4] = urcorner Page shifted (0,-a) ;
z[5] = ulcorner Page shifted (a,-a) ;

save p ;
path p[] ;
p[1] = ulcorner Page -- z[1] -- z[2] -- llcorner Page -- cycle ;
p[2] = ulcorner Page -- urcorner Page -- z[4] -- z[3] -- cycle ;
p[3] = ulcorner Page -- z[1] -- z[5] -- z[3] -- cycle ;

fill Page withcolor \MPcolor{simpleslides:backgroundcolor} ;
fill p[1] withcolor \MPcolor{simpleslides:contrastcolor} ;
fill p[2] withcolor \MPcolor{simpleslides:contrastcolor} ;
fill p[3] withcolor \MPcolor{simpleslides:variantcolor} ;

pickup pencircle scaled 5pt ;

if NOfPages > 1:
	theta = (PageNumber - 1)/(NOfPages - 1) ;
	p[4] = unitcircle scaled b rotated 90 shifted (b + ((a-b)/2),(a-b)/2) ;
	fill p[4] withcolor \MPcolor{simpleslides:backgroundcolor} ;
	t[0] = center p[4] ;
	t[1] = point 1 along p[4] ;
	t[2] = point -theta along p[4] ;
	t[3] = point -theta/2 along p[4] ;
	p[5] = t[0] -- t[1] .. t[3] .. t[2] -- cycle ;
	fill p[5] withcolor \MPcolor{simpleslides:altcontrastcolor} ;
fi ;
StopPage ;
\stopuseMPgraphic

%D We define these backgrounds as overlays:

\defineoverlay
  [simpleslides:background:ornament]
  [\useMPgraphic{simpleslides:MP:ornament}]

\defineoverlay 
  [simpleslides:background:title] 
  [\useMPgraphic{simpleslides:MP:ornament}] 

%D We want the title to placed in color. 

\setupTitle[\c!headcolor={simpleslides:contrastcolor}]

%D We want the slide title on the top

\setupSlideTitle
   [\c!after=,
    \c!alternative=layer,
    \c!width=\textwidth,
    \c!height=2.5cm,
    \c!color=simpleslides:backgroundcolor]

%D The symbol for the first level of itemizations. 

\definesymbol[1][\useMPgraphic{simpleslides:itemize:triangle}]
\setupitemize[1][inmargin][color=simpleslides:backgroundcolor]

\protect
\stopmodule

\endinput

