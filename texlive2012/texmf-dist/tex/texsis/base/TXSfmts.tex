%% file: TXSfmts.tex - Special Document Formats - TeXsis version 2.18
%% @(#) $Id: TXSfmts.tex,v 18.0 1999/07/09 17:24:29 myers Exp $
%======================================================================*
%
%       Several standard document formats are defined in TeXsis. All of
% these set up the appropriate dimensions, title page macros, etc., and
% call \texsis:
%
%       \preprint               Preprint format. Makes a standard title
%                               page from \title, \author, etc. and puts the
%                               manuscript on the following pages.
%
%       \Manuscript             Like \preprint, but printed with 
%                               \TrueDoubleSpacing for submission to 
%                               journals like Physical Review
%
%       \paper                  Like \preprint but puts \title, \author,
%                               etc. material at top of first page
%                               instead of on a separate title page.
%
%       \book                   For books, with chapters and sections,
%                               and odd/even page numbering
%
%       \slides                 Slides/transparencies format, in 24 pt type,
%                               special formatting.
%
%       \letter                 Standard business letter; see TXSletr.tex
%
%       \memo                   Standard memo; see TXSmemo.tex.
%
%       \referee                Referee reports; similar to \memo
%
%       \twinprint              Preprint in twin page format, with
%                               title page in full size landscape.
%
%       \twinformat             Twin page formatting for other styles.
%
%       More specialized document layouts are available in ``style'' 
%  files, which are only loaded when they are called for.  This is 
%  accomplished with the \autoload macro described below.  It's possible 
%  (and easy) to add new style files for any desired document layout.
%
%       The specialized document layouts which are currently loaded 
%  from style files [named in brackets] are:
%
%       \nuclproc               The ``NUCLPROC'' style for Nuclear
%                               Physics Proceedings. [nuclproc.txs]
%
%       \NorthHolland           North Holland proceedings format,
%                               single column, left justified titles.
%                               [Elsevier.txs]
%
%       \NorthHollandTwo        North Holland double column proceedings
%                               format, left justified titles. [Elsevier.txs]
%
%       \WorldScientific        World Scientific proceedings format.
%                               [WorldSci.txs]
%
%       \IEEEproceedings        Conference paper in double column IEEE
%                               format. [IEEE.txs]
%
%       \IEEEreduced            Same as \IEEEproceedings but reduced
%                               by 75% to fit on 8.5 x 11 paper. [IEEE.txs]
%
%       \AIPproceedings         Conference paper in the American Institute
%                               of Physics (AIP) format. [AIP.txs]
%
%======================================================================*
\message{Layout macros.}
\catcode`@=11
     
%--------------------------------------------------*
% Paper Layout: This prints the \titlepage material at the top of
% the first page, not on a separate page.
     
\def\paper{%                            % Paper:
   \auxswitchtrue                       % save tags and labels in .aux file
   \refswitchtrue                       % save references in .ref file
   \texsis                              % Initialize
   \def\titlepage{%                     % \paper title page
      \bgroup                           % so changes don't leak out
      \let\title=\Title                 % \title inside \titlepage
      \let\endmode=\relax               % \endmode will end a field
      \pageno=1}%                       % title page is page 1 by default
   \def\endtitlepage{%                  % \paper end title page
      \endmode                          % end any open field
      \goodbreak\bigskip                % no new page, but maybe break here
      \egroup}%                         % everything back the way it was
   \autoparens                          % auto-sizing of parens
   \quoteon                             % automatic begin/end quotes
   }
     
\def\Tbf{\fourteenpoint\bf}%         % 14pt bold title
\def\tbf{\twelvepoint\bf}%           % 12pt bold section head

%--------------------------------------------------*
% Preprint Layout: This prints the \titlepage material on a separate
% page with a \banner at the top.
     
\def\preprint{% Document layout for Preprints
   \auxswitchtrue                       % save tags and labels in .aux file
   \refswitchtrue                       % save references in .ref file
   \texsis                              % Initialize TeXsis
   \def\titlepage{%                     % \preprint title page
      \bgroup                           % so changes don't leak out
      \pageno=1                         % title page is page 1 by default
      \let\title=\Title                 % \title inside \titlepage
      \let\endmode=\relax               % \endmode will end a field
      \banner}%                         % prints the banner at top of page 1
   \def\endtitlepage{%                  % \preprint end title page
      \endmode                          % end any open field
      \vfil\eject                       % new page
      \egroup}%                         % everything back the way it was
   \autoparens                          % auto-sizing of parens
   \quoteon                             % automatic begin/end quotes
   }
     
%--------------------------------------------------*
% \Manuscript is a layout similar to \preprint, but printed
% in \TrueDoubleSpacing and with other things set the way Physical
% Review wants them for submitted manuscripts.  For an even better
% version use \PhysRevManuscript or \PhysRev\Manuscript.

\def\Manuscript{% layout for manuscripts to be submitted to journals
   \preprint                    % start with preprint form
   \showsectIDfalse             % no section number in equation numbers
   \showchaptIDfalse            % no chapter number in equation numbers
   \def\SectionStyle##1{\uppercase       % section numbers in upper case
         \expandafter{\romannumeral ##1}}%     roman numerals
   \RomanTablestrue             % roman numerals for table numbers
   \TablesLast                  % tables at the end
   \FiguresLast                 % figures at the end
   \TrueDoubleSpacing           % wider interline spacing for manu
   \def\everyabstract{\TrueDoubleSpacing}
   \def\Tbf{\fourteenpoint\bf\TrueDoubleSpacing}%       % 14pt bold title
   \def\refFormat{\TrueDoubleSpacing}%
   }

\autoload\PhysRevManuscript{PhysRev.txs}% More better version in style file
        
%--------------------------------------------------*
% \book format is similar to \thesis but without the special 
% formatting.  Use for books, long review articles, etc.
     
\def\book{% document layout for a book or similar document
   \ContentsSwitchtrue                  % contents page ON
   \refswitchtrue                       % save references
   \auxswitchtrue                       % .AUX file for forward references
   \texsis                              % initialize TeXsis
   \RunningHeadstrue                    % running headlines
   \bookpagenumbers                     % page numbers for book binding
   \def\titlepage{%                     % \book title page
      \bgroup                           % any changes local
      \pageno=-1                        % start on page i
      \let\title=\Title                 % \title inside \titlepage
      \let\endmode=\relax               % \endmode will end a field
      \def\FootText{\relax}}%           %  no number on title page
   \def\endtitlepage{%                  % end title page for \book
      \endmode                          % end any open field
      \vfil\eject                       % end title page
      \egroup%                          % close group from \titlepage
      \pageno=1}%                       % 
   \def\abstract{%                      % abstract for \book
      \endmode                          % end previous field
      \bigskip\bigskip\medskip          % skip down some
      \bgroup\singlespaced              % abstract has same spacing
         \let\endmode=\endabstract      % to end \abstract
         \narrower\narrower             %
         \everyabstract}%               %
   \def\endabstract{%                   % end abstract for \book
      \medskip\egroup\bigskip}%         %
   \def\FootText{--\ \tenrm\folio\ --}% % for 1st page of chapter
   \def\Tbf{\sixteenpoint\bf}%          % Chapter Titles in 16pt bold
   \def\tbf{\fourteenpoint\bf}%         % Section titles in 12pt bold
   \twelvepoint                         % 12pt type for most of doc
   \doublespaced                        % double spacing
   \autoparens                          % auto-sizing of parens
   \quoteon                             % auto quote matching
   }                                    %
     
%--------------------------------------------------*
%  \thesis is loaded from a style file now because it will be
%  different at different installations.

\autoload\thesis{thesis.txs}
\autoload\UTthesis{thesis.txs}
\autoload\YaleThesis{thesis.txs}

%--------------------------------------------------*
% \letter is a letter format for typing letters, following the syntax
% in Appendix F of the TeXbook.  See TXSletr.tex for details.
     
\def\Letter{%   document format for letters
   \ContentsSwitchfalse                         % no table of contents
   \refswitchfalse                              % no reference list
   \auxswitchfalse                              % no forward references
   \texsis                                      % initialize
   \singlespaced                                % single space default
   \LetterFormat}                               % Setup using \LetterFormat

\def\letter{\Letter}                            % synonym
     
%--------------------------------------------------*
% \memo Format for memoranda, as defined in TXSmemo.tex

\def\Memo{%  document format for memos
   \ContentsSwitchfalse                         % no table of contents
   \refswitchfalse                              % no reference list
   \auxswitchfalse                              % no forward references
   \texsis                                      % initialize
   \singlespaced                                % single space default
   \MemoFormat}                                 % Setup using \MemoFormat

\def\memo{\Memo}                                % synonym

%--------------------------------------------------*
% \Referee format for Referee Reports, as defined in TXSmemo.tex

\def\Referee{% document format for referee reports
   \ContentsSwitchfalse                         % no table of contents
   \auxswitchfalse                              % no need for an .aux file
   \refswitchfalse                              % no reference list
   \texsis                                      % initialize
   \RefReptFormat}                              % Setup using \MemoFormat

\def\referee{\Referee}                          % synonym
     
%--------------------------------------------------*
%       Landscape Layout: for printing ``sideways'' on the laser printer
%
% \Landscape sets up the page size for sideways (``landscape mode'')
% printing, and calls \LandscapeSpecial to signal this to the printer.
% What signal is given to the printer driver (with \special) depends
% on which filter or driver you are using.  You may have to change
% the definition of \LandscapeSpecial to suit your installation, but
% please do this in TXSsite.tex, not this file.

\def\Landscape{% set page size for printing ``sideways''
   \texsis              % 12pt, double spaced, etc
   \hsize=9in           % wide in the side
   \vsize=6.5in         % short and stocky
   \voffset=.5in        % and a little space from the ``margin''
   \nopagenumbers       % default is no page numbering
   \LandscapeSpecial    % actually turns on landscape mode   
}

\def\landscape{\Landscape}                      % synonym
     
% \LandscapeSpecial turns on landscape mode on the laser printer.
% How this is done is very site dependent, but the default is for
% Rokicki's dvips.  Change this _in_TXSsite.tex_ if you need to.

\def\LandscapeSpecial{\special{papersize=11in,8.5in}}


%--------------------------------------------------*
%   \slides makes slides or transparencies in default 24pt type with
% sensible defaults. It also defines \bl for a blank line and
% \np for new page.
%
\def\slides{%                                   % standard stuff except 12pt:
   \quoteon                                     % automatic quote balancing on
   \autoparens                                  % automatic paren balancing on
   \ATlock                                      % now @ is no longer a letter
   \pageno=1                                    % start on page one
   \twentyfourpoint                             % 24 point type
   \doublespaced                                % double spaced
   \raggedright\tolerance=2000                  % very ragged right 
   \hyphenpenalty=500                           % hyphens are bad
   \raggedbottom                                % ragged bottom
   \nopagenumbers                               % no numbers
   \hoffset=-.25in \hsize=7.0in                 % to fit 8.5in x 10.5in
   \voffset=-.25in \vsize=9.0in                 % to fit 8.5in x 10.5in
   \parindent=30pt                              % for big type
   \def\bl{\vskip\normalbaselineskip}%          % blank line
   \def\np{\vfill\eject}%                       % new page
   \def\nospace{\nulldelimiterspace=0pt%        % autoscale...
      \mathsurround=0pt}%                       %
   \def\big##1{{\hbox{$\left##1%                % ...\big
      \vbox to2ex{}\right.\nospace$}}}%         %
   \def\Big##1{{\hbox{$\left##1%                % ...\Big
      \vbox to2.5ex{}\right.\nospace$}}}%       %
   \def\bigg##1{{\hbox{$\left##1%               % ...\bigg
       \vbox to3ex{}\right.\nospace$}}}%        %
   \def\Bigg##1{{\hbox{$\left##1%               % ...\Bigg
      \vbox to4ex{}\right.\nospace$}}}%
  }

%-----------------------------------------------------------------------
% TWIN PAGE OUTPUT - using macros in twin.txs

\autoload\twinout{twin.txs}             % two pages on one piece of paper


%      \twinprint sets up a twin column preprint, with the first page
% full with in \twelvepoint and the rest in \tenpoint \twinout format.

\def\twinprint{% preprint with two pages on one
   \preprint                                    % start with \prepriwt
   \let\t@tl@=\title                           % based on \title
   \def\title{\vskip-1.5in\t@tl@}%             % reduce space
   \let\endt@tlep@ge=\endtitlepage             % save it
   \def\endtitlepage{\endt@tlep@ge             % usual \endtitlepage
       \twinformat}%                           % twin column default
}


%      \twinformat sets up the standard defaults for \twinout, with
% 10pt fonts and no headline offset.

\def\twinformat{%                              % 2 pages/page format
   \tenpoint\doublespaced                      % ten point size
   \def\Tbf{\twelvebf}\def\tbf{\tenbf}%        % title fonts
   \headlineoffset=0pt                         % no extra offset
   \twinout}                                   % \twinout format

% >>> EOF TXSfmts.tex <<<
