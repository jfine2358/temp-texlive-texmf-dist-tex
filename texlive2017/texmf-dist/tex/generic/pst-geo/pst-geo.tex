%%
%% This is file `pst-geo.tex',
%%
%% IMPORTANT NOTICE:
%%
%% Package `pst-geo'
%%
%% Manuel Luque <Mluque5130@aol.com> (France)
%% Giuseppe Matarazzo <joematara@hotmail.com> (Italy)
%% Herbert Voss <hvoss@tug.org> (Germany)
%%
%% December 07, 2016
%%
%% This program can be redistributed and/or modified under the terms
%% of the LaTeX Project Public License Distributed from CTAN archives
%% in directory macros/latex/base/lppl.txt.
%%
%% DESCRIPTION:
%%   `pst-geo' is a PSTricks package to create geographical objects
%%
%%%%%%%%%%%%%%%%%%%%%%%%%%%%%%%%%%%%%%%%%%%%%%%%%%%%%%%%%%%%%%%%%%%%%%%%%%%%%%%%%%%%%%%%

\csname PSTGeo\endcsname
\let\PSTGeoLoaded\endinput

% Require PSTricks
\ifx\PSTricksLoaded\endinput\else\input pstricks.tex\fi
\ifx\PSTnodesLoaded\endinput\else\input pst-node.tex\fi
\ifx\PSTXKeyLoaded\endinput\else\input pst-xkey\fi
%
\edef\PstAtCode{\the\catcode`\@} \catcode`\@=11\relax


\pst@addfams{pst-geo}
%
\def\fileversion{0.06}
\def\filedate{2016/12/08}
\message{`PST-geo' v\fileversion, \filedate\space (ML/HV)}
%
\pstheader{pst-geo.pro}

\define@key[psset]{pst-geo}{path}[data]{\def\psk@geo@Path{#1}} % chemin des donn�es
\define@key[psset]{pst-geo}{level}[1]{\def\psk@geo@level{#1}}
\define@key[psset]{pst-geo}{type}[1]{\def\psk@geo@type{#1}}
% type=1 ==> Mercator
% type=2 ==> Lambert
% type=3 ==> simple
% type=4 ==> Sanson-Flamsteed
% type=5 ==> cylindrical
% type=6 ==> Babinet
% type=7 ==> Collignon
% type=8 ==> Bonne
\psset{path=data,level=1,type=1}
\define@key[psset]{pst-geo}{n}[n]{\def\psk@geo@n{#1}}
\psset{n=1.77245} % sqrt(pi) % pour Collignon
\define@key[psset]{pst-geo}{limiteL}[180]{\def\psk@geo@limiteL{#1}}
\psset{limiteL=180} % longitude limite
\define@key[psset]{pst-geo}{latitude0}[45]{\def\psk@geo@latitudeO{#1}}
\psset{latitude0=45}  % pour Bonne latitude 0
\define@key[psset]{pst-geo}{longitude0}[0]{\def\psk@geo@longitudeO{#1}}
\psset{longitude0=0}  % pour Bonne longitude 0
\define@key[psset]{pst-geo}{increment}[10]{\def\psk@geo@increment{#1}}
\psset{increment=10}
%
\define@key[psset]{pst-geo}{MapFillColor}%[{[rgb]{0.99, 0.95, 0.7}}]
{\pst@getcolor{#1}\psk@geo@Fillcolor} % couleur des continents
\psset{MapFillColor={[rgb]{0.99, 0.95, 0.7}}}
\define@key[psset]{pst-geo}{SeaFillColor}%[{[rgb]{0.5, 0.8, 0.8}}]
{\pst@getcolor{#1}\psk@geo@Seacolor} % sea color
\psset{SeaFillColor={[rgb]{0.5, 0.8, 0.8}}}
\define@key[psset]{pst-geo}{LakeFillColor}%[{[rgb]{0.2, 0.2, 0.9}}]
{\pst@getcolor{#1}\psk@geo@Lakecolor} % lake color
\psset{LakeFillColor={[rgb]{0.2, 0.2, 0.9}}}
\define@key[psset]{pst-geo}{RiverFillColor}%[{[rgb]{0, 0, 1}}]
{\pst@getcolor{#1}\psk@geo@Rivercolor} % river color
\psset{RiverFillColor={[rgb]{0, 0, 1}}}
%
%\define@boolkey[psset]{pst-3dplot}[Pst@]{Debug}[true]{}% 
%
\define@boolkey[psset]{pst-geo}[Pst@geo@]{capitals}[true]{}
\define@boolkey[psset]{pst-geo}[Pst@geo@]{cities}[true]{}
\define@boolkey[psset]{pst-geo}[Pst@geo@]{rivers}[true]{}
\define@boolkey[psset]{pst-geo}[Pst@geo@]{borders}[true]{}
\define@boolkey[psset]{pst-geo}[Pst@geo@]{maillage}[true]{}
\psset{cities=false,capitals=false,rivers=false,borders,maillage}
\define@boolkey[psset]{pst-geo}[Pst@geo@]{Fill}[true]{}
\psset{Fill}
\define@boolkey[psset]{pst-geo}[Pst@geo@]{USA}[true]{}
\define@boolkey[psset]{pst-geo}[Pst@geo@]{Australia}[true]{}
\define@boolkey[psset]{pst-geo}[Pst@geo@]{Mexico}[true]{}
\psset{USA=false,Australia=false,Mexico=false}
%
\define@key[psset]{pst-geo}{increment}[10]{\def\psk@geo@increment{#1}}
\define@boolkey[psset]{pst-geo}[Pst@geo@]{Europe}[true]{}
\define@boolkey[psset]{pst-geo}[Pst@geo@]{Asia}[true]{}
\define@boolkey[psset]{pst-geo}[Pst@geo@]{Africa}[true]{}
\define@boolkey[psset]{pst-geo}[Pst@geo@]{Northamerica}[true]{}
\define@boolkey[psset]{pst-geo}[Pst@geo@]{Southamerica}[true]{}
\psset{increment=10,Europe,Asia=false,Africa=false,Northamerica=false,Southamerica=false}
\define@key[psset]{pst-geo}{all}[true]{\psset{increment=10,Europe,Asia,Africa,Northamerica,Southamerica}}

\def\psNodeLabelStyle{}
\newpsstyle{psNodeMapStyle}{fillstyle=solid,fillcolor=yellow!80,framesep=1pt,linestyle=none,opacity=0.4}
\define@key[psset]{pst-geo}{mapCountry}[all]{\def\psk@geo@mapCountry{#1}}
\define@key[psset]{pst-geo}{nodeWidth}[1mm]{\def\psk@geo@nodeWidth{#1}}
\psset{mapCountry=all,nodeWidth=1mm}

\definecolor{orange}{cmyk}{0,0.61,0.87,0}
\definecolor{darkgreen}{cmyk}{1,0,1,0.5}


\def\geo@variablesMap#1{%
% 0 2d
% 1 2dII
% 2 3d
% 3 3dII
  \ifcase #1 % 2d
    /Coeff_mercatorY {2 div 45 add dup sin exch cos div ln 57.2958 mul} def % Mercator
    /Coeff_simpleY {1 mul} def % simple
    /Coeff_lambertY {sin 57.2958 mul} def % Lambert
    /Coeff_sanson_flamsteedY {1 mul} def % Sanson_Flamstead
    /Coeff_cylindricalY { dup sin exch cos div 57.2958 mul} def % Cylindrical
    /Coeff_babinetY { sin 2 sqrt mul 57.2958 mul} def % Babinet
    /n \psk@geo@n\space def
    /Coeff_collignonY { 0 ge { 1 2 sqrt 90 Y sub 2 div sin mul sub 57.2958 mul n mul}
       { 1 2 sqrt 90 Y add 2 div sin mul sub 57.2958 mul n mul neg} ifelse }def % Collignon
    /Coeff_BonneY {pop /RHO l0 dup cos exch sin div Y neg l0 add 57.2958 div add def
      /Theta Y cos X L0 sub mul RHO div def
      Theta cos RHO mul neg 57.2958 mul }def % Bonne
%
    /Coeff_mercatorX {1 mul} def % Mercator
    /Coeff_simpleX {1 mul} def % simple
    /Coeff_lambertX {1 mul} def % Lambert
    /Coeff_sanson_flamsteedX {Y cos mul} def % Sanson_Flamstead
    /Coeff_cylindricalX { 1 mul } def % Cylindrical
    /Coeff_babinetX { 2 mul 2 sqrt mul 3.1415926 div Y cos mul} def % Babinet
    /Coeff_collignonX { 2 mul 2 sqrt mul 90 Y abs sub 2 div sin mul n div} def % Collignon
    /l0 \psk@geo@latitudeO\space def
    /L0 \psk@geo@longitudeO\space def
    /Coeff_BonneX { pop /RHO l0 dup cos exch sin div Y neg l0 add 57.2958 div add def
      /Theta Y cos X L0 sub mul RHO div def
        Theta sin RHO mul 57.2958 mul } def % Bonne
    \ifcase\psk@geo@type\or
      /CoeffY {Coeff_mercatorY} def
      /CoeffX {Coeff_mercatorX} def
      /StartY -85 def
    \or
      /CoeffY {Coeff_simpleY} def
      /CoeffX {Coeff_simpleX} def
      /StartY -90 def
    \or
      /CoeffY {Coeff_lambertY} def
      /CoeffX {Coeff_lambertX} def
      /StartY -90 def
    \or
      /CoeffY {Coeff_sanson_flamsteedY} def
      /CoeffX {Coeff_sanson_flamsteedX} def
      /StartY -90 def
    \or
      /CoeffY {Coeff_cylindricalY} def
      /CoeffX {Coeff_cylindricalX} def
      /StartY -85 def
    \or
      /CoeffY {Coeff_babinetY} def
      /CoeffX {Coeff_babinetX} def
      /StartY -90 def
    \or
      /CoeffY {Coeff_collignonY} def
      /CoeffX {Coeff_collignonX} def
      /StartY -90 def
    \or
      /CoeffY {Coeff_BonneY} def
      /CoeffX {Coeff_BonneX} def
      /StartY -90 def
    \fi
  \or % 2dII
    /Coeff_mercatorY { 2 div 45 add dup sin exch cos div ln 57.2958 mul } def % Mercator
    /Coeff_simpleY {1 mul } def % simple
    /Coeff_lambertY {sin 57.2958 mul } def % Lambert
    /Coeff_sanson_flamsteedY {1 mul } def % Sanson_Flamstead
    /Coeff_cylindricalY { dup sin exch cos div 57.2958 mul } def % Cylindrical
    /Coeff_babinetY { sin 2 sqrt mul 57.2958 mul } def % Babinet
    /n \psk@geo@n\space def
    /Coeff_collignonY { 0 ge { 1 2 sqrt 90 Y sub 2 div sin mul sub 57.2958 mul n mul }
                      { 1 2 sqrt 90 Y add 2 div sin mul sub 57.2958 mul n mul neg} ifelse } def % Collignon
    /Coeff_BonneY { pop /RHO l0 dup cos exch sin div Y neg l0 add 57.2958 div add def
                    /Theta Y cos X L0 sub mul RHO div def
                    Theta cos RHO mul neg 57.2958 mul } def % Bonne
    /Coeff_mercatorX {1 mul} def % Mercator
    /Coeff_simpleX {1 mul} def % simple
    /Coeff_lambertX {1 mul} def % Lambert
    /Coeff_sanson_flamsteedX {Y cos mul} def % Sanson_Flamstead
    /Coeff_cylindricalX { 1 mul } def % Cylindrical
    /Coeff_babinetX { 2 mul 2 sqrt mul 3.1415926 div Y cos mul} def % Babinet
    /Coeff_collignonX { 2 mul 2 sqrt mul 90 Y abs sub 2 div sin mul n div } def % Collignon
    /l0 \psk@geo@latitudeO\space def
    /L0 \psk@geo@longitudeO\space def
    /Coeff_BonneX {pop /RHO l0 dup cos exch sin div Y neg l0 add 57.2958 div add def
               /Theta Y cos X L0 sub mul RHO div def
               Theta sin RHO mul 57.2958 mul } def % Bonne
    \ifcase\psk@geo@type\or % type 1  % hv 2004-05-04
      /CoeffY {Coeff_mercatorY} def
      /CoeffX {Coeff_mercatorX} def
      /StartY -85 def
    \or % type 2  % hv 2004-05-04
      /CoeffY {Coeff_simpleY} def
      /CoeffX {Coeff_simpleX} def
      /StartY -90 def
    \or % type 3  % hv 2004-05-04
      /CoeffY {Coeff_lambertY} def
      /CoeffX {Coeff_lambertX} def
      /StartY -90 def
    \or % type 4  % hv 2004-05-04
      /CoeffY {Coeff_sanson_flamsteedY} def
      /CoeffX {Coeff_sanson_flamsteedX} def
      /StartY -90 def
    \or % type 5  % hv 2004-05-04
      /CoeffY {Coeff_cylindricalY} def
      /CoeffX {Coeff_cylindricalX} def
      /StartY -85 def
    \or % type 6  % hv 2004-05-04
      /CoeffY {Coeff_babinetY} def
      /CoeffX {Coeff_babinetX} def
      /StartY -90 def
    \or % type 7  % hv 2004-05-04
      /CoeffY {Coeff_collignonY} def
      /CoeffX {Coeff_collignonX} def
      /StartY -90 def
    \else % every other is type 8  % hv 2004-05-04
      /CoeffY {Coeff_BonneY} def
      /CoeffX {Coeff_BonneX} def
      /StartY -90 def
    \fi
    /step \psk@geo@level\space def % hv 2004-05-04
    /increment \psk@geo@increment\space def
  \or % 3d
    /RotX \psk@geo@RotX\space def
    /RotY \psk@geo@RotY\space def
    /RotZ \psk@geo@RotZ\space def
    /THETA \psk@geo@Theta\space def
    /PHI \psk@geo@Phi\space def
    /Rsphere \psk@geo@Radius\space def
    /Dobs \psk@geo@Dobs\space def
    /DScreen \psk@geo@Ecran\space def
    /xunit {\pst@number\psxunit\space mul} def
    /yunit {\pst@number\psyunit\space mul} def
    /runit {\pst@number\psrunit\space mul} def
  \or
    /RotX \psk@geo@RotX\space def
    /RotY \psk@geo@RotY\space def
    /RotZ \psk@geo@RotZ\space def
    /THETA \psk@geo@Theta\space def
    /PHI \psk@geo@Phi\space def
    /Rsphere \psk@geo@Radius\space def
    /Dobs \psk@geo@Dobs\space def
    /DScreen \psk@geo@Ecran\space def
  \fi % \ifcase
  /level \psk@geo@level\space def % hv 2004-05-04
  /step \psk@geo@level\space def % hv 2004-05-04
}
\newcount\TypeProjection
\def\WorldMap{\pst@object{WorldMap}}
\def\WorldMap@i{%
%  \psset{#1}%
  \begin@ClosedObj
  \addto@pscode{
    \geo@variablesMap{0}
    (\psk@geo@Path/river.dat) run
    (\psk@geo@Path/pcoast.dat) run
    (\psk@geo@Path/pisland.dat) run
    (\psk@geo@Path/pborder.dat) run
    (\psk@geo@Path/plake.dat) run
    (\psk@geo@Path/usa.dat) run
    (\psk@geo@Path/aus.dat) run
    (\psk@geo@Path/mexico.dat) run
    1 setlinejoin 0 0 translate
    /xunit {\pst@number\psxunit\space mul 20 div } def
    /yunit {\pst@number\psyunit\space mul 20 div } def
    /increment \psk@geo@increment\space def
    /limiteLongitude \psk@geo@limiteL\space def
%
    \ifPst@geo@Fill
      gsave
      newpath
      /X limiteLongitude neg def
      /Y StartY def
      X CoeffX xunit Y CoeffY yunit moveto
      StartY 2 StartY neg { /Y exch def X CoeffX xunit Y CoeffY yunit lineto } for
      /X limiteLongitude def
      StartY neg -2 StartY { /Y exch def X CoeffX xunit Y CoeffY yunit lineto } for
      closepath
%     /Color {\psk@geo@Seacolor} def
%     Color setrgbcolor
      \pst@usecolor\psk@geo@Seacolor fill
      grestore
    \fi
    /CalculsPoints {
      /region exch def
      newpath
      /nbr region length def % nombre de r�gions
      region 0 get aload pop
      /Y exch def /X exch def
      X CoeffX xunit Y CoeffY yunit moveto
      /ncount 0 def % hv 2004-05-04
      0 1 nbr 1 sub {
        /ncount ncount 1 add def % hv 2004-05-04
        region exch get aload pop
        ncount step mod 0 eq nbr ncount sub 0 lt or { % hv 2004-05-04
          /Y exch def /X exch def
          X CoeffX xunit Y CoeffY yunit lineto }{ pop pop } ifelse % hv 2004-05-04
      } for
    } def
%
    /Draw { {
        CalculsPoints
        \ifPst@geo@Fill closepath gsave Color fill grestore \fi
        stroke
      } forall
    } def
    \ifPst@geo@borders border \else coasts \fi
    /Color { \pst@usecolor\psk@geo@Fillcolor } def
    Draw islands Draw
    \ifPst@geo@Mexico mexico Draw \fi
    \ifPst@geo@USA usa Draw \fi
    \ifPst@geo@Australia australia Draw \fi
    /Color { \pst@usecolor\psk@geo@Lakecolor } def
    lake Draw
    \ifPst@geo@rivers
      /DrawRivers { { CalculsPoints \pst@usecolor\psk@geo@Rivercolor stroke } forall } def
      river DrawRivers
    \fi
%
    \ifPst@geo@maillage
    % meridiens
      limiteLongitude neg increment limiteLongitude {% for X
        /X exch def
        newpath
        /Y StartY def
        X CoeffX xunit StartY CoeffY yunit moveto
        StartY increment 10 div StartY neg { % for Y
          /Y exch def
          X CoeffX xunit Y CoeffY yunit lineto
        } for
        gsave
        X 0 eq {1 0 0 setrgbcolor}{0.2 setgray} ifelse
        0.25 setlinewidth
        stroke
        grestore
      } for
      % parall�les
      StartY increment StartY neg {% for Y
        /Y exch def
        newpath
        /X limiteLongitude neg def
        X CoeffX xunit Y CoeffY yunit moveto
        limiteLongitude neg increment 10 div limiteLongitude { % for X
          /X exch def
          X CoeffX xunit Y CoeffY yunit lineto
        } for
        % 11 janvier 2004
        Y 0 eq {1 0 0 setrgbcolor}{0.2 setgray} ifelse
        0.25 setlinewidth
        stroke
      } for
    \fi
    /DrawCities {
      /CITY exch def
      /Rayon exch def
      /nbr CITY length def % nombre de villes
      0 1 nbr 1 sub {
        /compteur exch def
        CITY compteur get aload pop
        /X exch def /Y exch def
        gsave
        1 0 0 setrgbcolor newpath
        X CoeffX xunit Y CoeffY yunit
        Rayon 0 360 arc closepath fill
        grestore
      } for
    } def
    \ifPst@geo@cities (\psk@geo@Path/c-cap.dat) run 1.5 % rayon du point
      citycapitals DrawCities (\psk@geo@Path/c-sub.dat) run 0.5 % rayon du point
      CitySub DrawCities
    \fi
    \ifPst@geo@capitals (\psk@geo@Path/c-cap.dat) run 1.5 citycapitals DrawCities \fi
  }% end pscode
  \end@ClosedObj
}
%
\def\pnodeMap(#1,#2)#3{%   hv 2004-05-09
  \pst@killglue
  \pnode(!
      \geo@variablesMap{0}
      /Y #2 def
      /X #1 def
      X CoeffX 20 div Y CoeffY 20 div){#3}%
}
%
%------------------- hv 2004-05-08 ----------------------
%
%\mapput[90](2.316667,48.85000){Paris}
%
\def\mapput{\@ifnextchar[{\mapput@i}{\mapput@i[90]}}
\def\mapput@i[#1](#2,#3){\@ifnextchar[{\mapput@iii[#1](#2,#3)}{\mapput@ii[#1](#2,#3)}}
\def\mapput@ii[#1](#2,#3)#4{\mapput@iii[#1](#2,#3)[#4]{#4}}
\def\mapput@iii[#1](#2,#3)[#4]#5{%
  \@ifnextchar[{\mapput@iv[#1](#2,#3)[#4]#5}{\mapput@iv[#1](#2,#3)[#4]#5[\@empty]}}
\def\mapput@iv[#1](#2,#3)[#4]#5[#6]{{%
% [angle](longitude,lattitude)[node name]{node label}[country]
  \def\pst@tempb{all}
  \ifx\pst@tempb\psk@geo@mapCountry
    \mapput@v[#1](#2,#3)[#4]{#5}
  \else
    \def\pst@tempb{#6}
%    \typeout{Compare \pst@tempb{} with \psk@geo@mapCountry}
    \ifx\pst@tempb\psk@geo@mapCountry
      \mapput@v[#1](#2,#3)[#4]{#5}
    \fi%
  \fi%
}}
\def\mapput@v[#1](#2,#3)[#4]#5{{%
  \pnodeMap(#2,#3){#4}
%  \typeout{Node: #4}
  \pst@dima=\psk@geo@nodeWidth
  \pscircle[style=psNodeMapStyle](#4){\pst@dima}
  \pscircle*(#4){0.5\pst@dima}
  \uput{1.2\pst@dima}[#1](#4){\psframebox[style=psNodeMapStyle]{\psNodeLabelStyle #5}}%
}}

%%%%%%%%%%%%%%%%%%%%%%%%%%%%%%%%%%%%%%%%%%%%% 2DII %%%%%%%%%%%%%%%%%%%%%%%%%%%%%%%%%%%%%%%%%%%%%%%%%
 %
\newcount\TypeProjection
\def\WorldMapII{\pst@object{WorldMapII}}
\def\WorldMapII@i{{%
  \use@par
  \begin@ClosedObj
  \addto@pscode{
    \geo@variablesMap{1}
    1 setlinejoin
    0 0 translate
    /xunit {\pst@number\psxunit\space mul 20 div } def
    /yunit {\pst@number\psyunit\space mul 20 div } def
    /limiteLongitude \psk@geo@limiteL\space def
%
    /CalculsPoints {
      /region exch def
       gsave
      newpath
      /nbr region length def % nombre de points
      region nbr 2 sub get
      /Y exch def
      region nbr 1 sub get
      /X exch def
      X CoeffX xunit Y CoeffY yunit moveto
      /ncount 0 def % hv 2004-05-04
      nbr 2 sub -2 2  {
        /Counter exch def % hv 2004-05-04
        /ncount ncount 1 add def % hv 2004-05-04
        ncount step ge { % hv 2004-05-04
          region Counter get
          /Y exch def
          region Counter 1 add get
          /X exch def
          X CoeffX xunit Y CoeffY yunit lineto
          /ncount 0 def % hv 2004-05-04
        }{ /ncount ncount 1 add def } ifelse % hv 2004-05-04
     } for
    } def
%
    /CalculsPointsAsia {
      /region exch def
      gsave
      newpath
      /nbr region length def % nombre de points
      region nbr 2 sub get
      /Y exch def
      region nbr 1 sub get
      /X exch def
      X 0 le {/X X 360 add def} if
      X CoeffX xunit Y CoeffY yunit moveto
      /ncount 0 def % hv 2004-05-04
      nbr 2 sub -2 2  {
        /Counter exch def % hv 2004-05-04
        /ncount ncount 1 add def % hv 2004-05-04
        ncount step ge { % hv 2004-05-04
          region Counter get
          /Y exch def
          region Counter 1 add get
          /X exch def
          X 0 le {/X X 360 add def} if
          X CoeffX xunit Y CoeffY yunit lineto
          /ncount 0 def % hv 2004-05-04
        }{ /ncount ncount 1 add def } ifelse % hv 2004-05-04
      } for
    } def
%
    /Draw {
      { CalculsPoints
        stroke
      } forall
    } def
%
    /DrawAsia {
      { CalculsPointsAsia
        stroke
      } forall
    } def
    /DrawRivers {
      { CalculsPoints
        gsave
        0 0 1 setrgbcolor
        stroke
        grestore
      } forall
    } def
%
    /DrawRiversAsia {
      { CalculsPointsAsia
        gsave
        0 0 1 setrgbcolor
        stroke
        grestore
      } forall
    } def
    /DrawBdy {
      { CalculsPoints
        gsave
        1 0 0 setrgbcolor
        stroke
        grestore
      } forall
    } def
    \ifPst@geo@Europe
      (\psk@geo@Path/europe-cil_II.dat) run
      europe-cil
      Draw
      \ifPst@geo@rivers
        (\psk@geo@Path/europe-riv_II.dat) run
        europe-riv
        DrawRivers
      \fi
      \ifPst@geo@borders
        (\psk@geo@Path/europe-bdy_II.dat) run
        europe-bdy
        DrawBdy
      \fi
    \fi
%
    \ifPst@geo@Africa
      (\psk@geo@Path/africa-cil_II.dat) run
      africa-cil
      Draw
      \ifPst@geo@rivers
        (\psk@geo@Path/africa-riv_II.dat) run
        africa-riv
        DrawRivers
      \fi
      \ifPst@geo@borders
        (\psk@geo@Path/africa-bdy_II.dat) run
        africa-bdy
        DrawBdy
      \fi
    \fi
%
    \ifPst@geo@Asia
      (\psk@geo@Path/asia-cil_II.dat) run
      asia-cil
      DrawAsia
      (\psk@geo@Path/asia-isl_II.dat) run
      asia-isl
      Draw
      \ifPst@geo@rivers
        (\psk@geo@Path/asia-riv_II.dat) run
        asia-riv
        DrawRiversAsia
      \fi
      \ifPst@geo@borders
        (\psk@geo@Path/asia-bdy_II.dat) run
        asia-bdy
        DrawBdy
      \fi
    \fi
    \ifPst@geo@Northamerica
      (\psk@geo@Path/northamerica-cil_II.dat) run northamerica-cil Draw
    \ifPst@geo@rivers
      (\psk@geo@Path/northamerica-riv_II.dat) run northamerica-riv DrawRivers
      \fi
      \ifPst@geo@borders
        (\psk@geo@Path/northamerica-bdy_II.dat) run
        (\psk@geo@Path/northamerica-pby_II.dat) run
        northamerica-bdy
        DrawBdy
        northamerica-pby
        DrawBdy
      \fi
    \fi
    \ifPst@geo@Southamerica
      (\psk@geo@Path/southamerica-cil_II.dat) run
      southamerica-cil
      Draw
      (\psk@geo@Path/southamerica-arc_II.dat) run
      southamerica-arc
      DrawAsia
      \ifPst@geo@rivers
        (\psk@geo@Path/southamerica-riv_II.dat) run
        southamerica-riv
        DrawRivers
      \fi
      \ifPst@geo@borders
        (\psk@geo@Path/southamerica-bdy_II.dat) run
        southamerica-bdy
        DrawBdy
      \fi
    \fi
%
    \ifPst@geo@maillage
% meridiens
      limiteLongitude neg increment limiteLongitude {% for X
        /X exch def
        newpath
        /Y StartY def
        X CoeffX xunit StartY CoeffY yunit moveto
        StartY increment 10 div StartY neg { % for Y
          /Y exch def
          X CoeffX xunit Y CoeffY yunit lineto
        } for
        gsave
% 11 janvier 2004
        X 0 eq { 1 0 0 setrgbcolor }{ 0.2 setgray } ifelse
%
        0.25 setlinewidth
        stroke
        grestore
      } for
% parall�les
      StartY increment StartY neg {% for Y
        /Y exch def
        newpath
        /X limiteLongitude neg def
        X CoeffX xunit Y CoeffY yunit moveto
        limiteLongitude neg increment 10 div limiteLongitude { % for X
          /X exch def
          X CoeffX xunit Y CoeffY yunit lineto
        } for
% 11 janvier 2004
        Y 0 eq {1 0 0 setrgbcolor}{0.2 setgray} ifelse
%
        0.25 setlinewidth
        stroke
      } for
    \fi
    /DrawCities {
      /CITY exch def
      /Rayon exch def
      /nbr CITY length def % nombre de villes
      0 1 nbr 1 sub {
        /compteur exch def
        CITY compteur get aload pop
        /X exch def /Y exch def
        gsave
        1 0 0 setrgbcolor newpath
        X CoeffX xunit Y CoeffY yunit
        Rayon 0 360 arc closepath fill
        grestore
      } for
    } def
    \ifPst@geo@cities
% les capitales
      (\psk@geo@Path/c-cap.dat) run
      1.5 % rayon du point
      citycapitals
      DrawCities
% les autres villes
      (\psk@geo@Path/c-sub.dat) run
      0.5 % rayon du point
      CitySub
      DrawCities
    \fi
    \ifPst@geo@capitals
% les capitales
      (\psk@geo@Path/c-cap.dat) run
      1.5
      citycapitals
      DrawCities
    \fi
  }%
  \end@ClosedObj
  }\ignorespaces}
%
%%============================================ 3d =======================================================
%
\define@key[psset]{pst-geo}{RotX}{\def\psk@geo@RotX{#1}} % rotation autour de Ox en degr�s
\define@key[psset]{pst-geo}{RotY}{\def\psk@geo@RotY{#1}} % rotation autour de Oy en degr�s
\define@key[psset]{pst-geo}{RotZ}{\def\psk@geo@RotZ{#1}} % rotation autour de OZ en degr�s
\define@key[psset]{pst-geo}{Radius}{\def\psk@geo@Radius{#1}} % rayon de la sph�re
%
\define@key[psset]{pst-geo}{THETA}{\def\psk@geo@Theta{#1}}
\define@key[psset]{pst-geo}{PHI}{\def\psk@geo@Phi{#1}}
\define@key[psset]{pst-geo}{Dobs}{\def\psk@geo@Dobs{#1}}
\define@key[psset]{pst-geo}{Decran}{\def\psk@geo@Ecran{#1}}
\psset[pst-geo]{THETA=0,PHI=45,Dobs=20,Decran=25,RotX=0,RotY=0,RotZ=0,Radius=5}
%
\define@key[psset]{pst-geo}{Day}{\def\psk@geo@J{#1}}
\define@key[psset]{pst-geo}{Month}{\def\psk@geo@M{#1}}
\define@key[psset]{pst-geo}{Year}{\def\psk@geo@A{#1}}
\define@key[psset]{pst-geo}{hour}{\def\psk@geo@hour{#1}}
\psset[pst-geo]{Day=\number\day,Month=\number\month,Year=\number\year,hour=12}
%
\define@key[psset]{pst-geo}{gridmapdiv}{\def\psk@geo@mapgriddiv{#1}}
\define@key[psset]{pst-geo}{longitudeMeridien}{\def\psk@geo@longitudeMeridien{#1}}
\newdimen\psmeridienwidth
\define@key[psset]{pst-geo}{meridienwidth}[1pt]{\pssetlength\psmeridienwidth{#1}}
\define@key[psset]{pst-geo}{meridiencolor}[red]{\pst@getcolor{#1}\psmeridiencolor}
\psset[pst-geo]{meridiencolor=red,meridienwidth=1pt,longitudeMeridien=0,gridmapdiv=10}
%
\define@key[psset]{pst-geo}{latitudeParallel}[0]{\def\psk@geo@latitudeParallel{#1}}
\newdimen\psparallelwidth
\define@key[psset]{pst-geo}{parallelwidth}[1pt]{\pssetlength\psparallelwidth{#1}}
\define@key[psset]{pst-geo}{parallelcolor}{\pst@getcolor{#1}\psparallelcolor}
\psset[pst-geo]{parallelcolor=red,latitudeParallel=0,parallelwidth=1pt}
%
\definecolor{terre}{rgb}{0.99,0.95,0.7}
\definecolor{mer}{rgb}{0.5,0.8,0.8}
\definecolor{darkblue}{cmyk}{1,1,0,0.7}

\define@key[psset]{pst-geo}{mapcolor}{\pst@getcolor{#1}\psmapcolor}
\define@key[psset]{pst-geo}{bordercolor}[black]{\pst@getcolor{#1}\psbordercolor}
\define@key[psset]{pst-geo}{islandcolor}[black]{\pst@getcolor{#1}\psislandcolor}
\define@key[psset]{pst-geo}{coastcolor}[black]{\pst@getcolor{#1}\pscoastcolor}
\psset[pst-geo]{mapcolor=terre,bordercolor=black,islandcolor=black,coastcolor=black}
\define@key[psset]{pst-geo}{oceancolor}{\pst@getcolor{#1}\psoceancolor}
\psset[pst-geo]{oceancolor=mer}
\define@key[psset]{pst-geo}{rivercolor}{\pst@getcolor{#1}\psrivercolor}
\psset[pst-geo]{rivercolor=blue}
\define@key[psset]{pst-geo}{wfraczoncolor}{\pst@getcolor{#1}\pswfraczoncolor}
\psset[pst-geo]{wfraczoncolor=red}
\define@key[psset]{pst-geo}{wmaglincolor}{\pst@getcolor{#1}\pswmaglincolor}
\psset[pst-geo]{wmaglincolor=darkblue}
\define@key[psset]{pst-geo}{ridgecolor}{\pst@getcolor{#1}\psridgecolor}
\psset[pst-geo]{ridgecolor=red}
\define@key[psset]{pst-geo}{transfrmcolor}{\pst@getcolor{#1}\pstransfrmcolor}
\psset[pst-geo]{transfrmcolor=orange}
\define@key[psset]{pst-geo}{trenchcolor}{\pst@getcolor{#1}\pstrenchcolor}
\psset[pst-geo]{trenchcolor=darkgreen}
\define@key[psset]{pst-geo}{gridmapcolor}{\pst@getcolor{#1}\psgridmapcolor}
\psset[pst-geo]{gridmapcolor=black}
\define@key[psset]{pst-geo}{circlecolor}{\pst@getcolor{#1}\pscirclecolor}
\psset[pst-geo]{circlecolor=blue}
\define@key[psset]{pst-geo}{circlesep}{\def\psk@geo@circlesep{#1}}
\psset[pst-geo]{circlesep=2} % en degres
\newdimen\pscirclewidth
\define@key[psset]{pst-geo}{circlewidth}{\pssetlength\pscirclewidth{#1}}
\psset[pst-geo]{circlewidth=.5pt}
\newdimen\psgridmapwidth
\define@key[psset]{pst-geo}{gridmapwidth}[0.8pt]{\pssetlength\psgridmapwidth{#1}}
\psset[pst-geo]{gridmapwidth=.8pt}
\newdimen\psborderwidth
\define@key[psset]{pst-geo}{borderwidth}[0.8pt]{\pssetlength\psborderwidth{#1}}
\psset[pst-geo]{borderwidth=.8pt}
\newdimen\pscoastwidth
\define@key[psset]{pst-geo}{coastwidth}[0.8pt]{\pssetlength\pscoastwidth{#1}}
\psset[pst-geo]{coastwidth=.8pt}
\newdimen\pswfraczonwidth
\define@key[psset]{pst-geo}{wfraczonwidth}[0.8pt]{\pssetlength\pswfraczonwidth{#1}}
\psset[pst-geo]{wfraczonwidth=.8pt}
\newdimen\pswmaglinwidth
\define@key[psset]{pst-geo}{wmaglinwidth}[0.8pt]{\pssetlength\pswmaglinwidth{#1}}
\psset[pst-geo]{wmaglinwidth=.8pt}
\newdimen\psridgewidth
\define@key[psset]{pst-geo}{ridgewidth}[2pt]{\pssetlength\psridgewidth{#1}}
\psset[pst-geo]{ridgewidth=2pt}
%
\define@boolkey[psset]{pst-geo}[Pst@geo@]{islands}[true]{}
\define@boolkey[psset]{pst-geo}[Pst@geo@]{France}[true]{}
\define@boolkey[psset]{pst-geo}[Pst@geo@]{Germany}[true]{}
\define@boolkey[psset]{pst-geo}[Pst@geo@]{Canada}[true]{}
\define@boolkey[psset]{pst-geo}[Pst@geo@]{lakes}[true]{}
\psset{islands,France=false,Canada=false,lakes=false,Germany=false}
%
\define@boolkey[psset]{pst-geo}[Pst@geo@]{gridmap}[true]{}
\define@boolkey[psset]{pst-geo}[Pst@geo@]{coasts}[true]{}
\define@boolkey[psset]{pst-geo}[Pst@geo@]{wfraczon}[true]{}
\define@boolkey[psset]{pst-geo}[Pst@geo@]{ridge}[true]{}
\define@boolkey[psset]{pst-geo}[Pst@geo@]{wmaglin}[true]{}
\psset{gridmap,coasts=false,wfraczon=false,ridge=false,wmaglin=false}
%
\define@boolkey[psset]{pst-geo}[Pst@geo@]{circles}[true]{}
\define@boolkey[psset]{pst-geo}[Pst@geo@]{visibility}[true]{}
\define@boolkey[psset]{pst-geo}[Pst@geo@]{blueEarth}[true]{}
\define@boolkey[psset]{pst-geo}[Pst@geo@]{daynight}[true]{}
\psset{circles,visibility,blueEarth,daynight=false}
%	
\def\WorldMapThreeD{\pst@object{WorldMapThreeD}}
\def\WorldMapThreeD@i{%
\begin@ClosedObj
\addto@pscode{
  /increment \psk@geo@mapgriddiv\space def
  \geo@variablesMap{2}
  /mapcolor  {\pst@usecolor\psmapcolor currentrgbcolor} def
  /bordercolor  {\pst@usecolor\psbordercolor currentrgbcolor} def
  /coastcolor  {\pst@usecolor\pscoastcolor currentrgbcolor} def
  /islandcolor  {\pst@usecolor\psislandcolor currentrgbcolor} def
  /maillagecolor  {\pst@usecolor\psgridmapcolor currentrgbcolor} def
  /oceancolor  {\pst@usecolor\psoceancolor currentrgbcolor} def
  /rivercolor  {\pst@usecolor\psrivercolor currentrgbcolor} def
  /wfraczoncolor {\pst@usecolor\pswfraczoncolor currentrgbcolor} def
  /wmaglincolor {\pst@usecolor\pswmaglincolor currentrgbcolor} def
  /ridgecolor {\pst@usecolor\psridgecolor currentrgbcolor} def
  /transfrmcolor {\pst@usecolor\pstransfrmcolor currentrgbcolor} def
  /trenchcolor {\pst@usecolor\pstrenchcolor currentrgbcolor} def
  /circlecolor  {\pst@usecolor\pscirclecolor currentrgbcolor} def
  /circlewidth {\pst@number\pscirclewidth SLW} def
  /circlesep \psk@geo@circlesep\space def
  /maillagewidth {\pst@number\psgridmapwidth SLW} def
  /borderwidth {\pst@number\psborderwidth SLW} def
  /coastwidth {\pst@number\pscoastwidth SLW} def
  /wfraczonwidth {\pst@number\pswfraczonwidth SLW} def
  /wmaglinwidth {\pst@number\pswmaglinwidth SLW} def
  /ridgewidth {\pst@number\psridgewidth SLW} def
  /longitude_meridien \psk@geo@longitudeMeridien\space def
  /meridienwidth {\pst@number\psmeridienwidth SLW} def
  /meridiencolor  {\pst@usecolor\psmeridiencolor currentrgbcolor} def
  /latitude_parallel \psk@geo@latitudeParallel\space def
  /parallelwidth {\pst@number\psparallelwidth SLW} def
  /parallelcolor  {\pst@usecolor\psparallelcolor currentrgbcolor} def
  /year \psk@geo@A\space def
  /month \psk@geo@M\space def
  /day \psk@geo@J\space def
   month 3 lt {/month month 12 add def /year year 1 sub def} if
  /AA year 100 div truncate def
  /BB 2 AA sub AA 4 div truncate add def
  /CC 365.25 year mul truncate def
  /DD 30.6001 month 1 add mul truncate def
  /julien BB CC add DD add day add 1720994.5 add def
  /hour \psk@geo@hour\space def
  /jours julien 2454367 sub def
  /AngleRayonsSoleil latitude_parallel  23.5 sin 360 jours mul 365.25 div sin mul arccos sub def
   /u1 0 def /u2 AngleRayonsSoleil cos def /u3 AngleRayonsSoleil sin def
%  /Julien jd hour 24 div add def
  %
  1 setlinejoin
  0 0 translate
  /pst@fill { \psk@opacityalpha .setopacityalpha fill } def
   tx@geoDict begin
   MatriceTransformationZXY
%
  /RsphereScreen Rsphere DScreen mul Dobs div
    1 Rsphere dup mul Dobs dup mul div sub sqrt div runit def
%
/GrandCercle {
  % cercle vu par l'observateur
  0 0 RsphereScreen 0 360 arc
  closepath
} def
%
\ifPst@geo@visibility
  /condition { 0 ge } def
\else
 /condition { 0 le } def
\fi
%
\ifPst@geo@daynight
    the_night
\fi
%
\ifPst@geo@blueEarth
  gsave
  newpath
  oceancolor
  GrandCercle
  pst@fill  % fill
  grestore
\fi
%
\ifPst@geo@circles
gsave
 oceans_seas_hatched
grestore
\fi
%
\ifPst@geo@borders
  (\psk@geo@Path/pborder.dat) run
gsave
GrandCercle
clip
border dup /REGION exch def
CompteurRegions
{CalculsPointsRegion
  closepath
   gsave
  mapcolor
  pst@fill  % fill
  grestore
  borderwidth
  bordercolor
  stroke
  }forall
grestore
\fi
%
\ifPst@geo@islands
  (\psk@geo@Path/pisland.dat) run
gsave
GrandCercle
clip
islands dup /REGION exch def
CompteurRegions
{CalculsPointsRegion
  closepath
   gsave
  mapcolor
  pst@fill  % fill
  grestore
  borderwidth
  islandcolor
   stroke
  }forall
grestore
\fi
%
\ifPst@geo@France
  (\psk@geo@Path/france.dat) run
gsave
GrandCercle
clip
france dup /REGION exch def
CompteurRegions
{CalculsPointsRegion
  closepath
   gsave
 mapcolor
  pst@fill  % fill
  grestore
  borderwidth
  bordercolor
  stroke
  }forall
grestore
\fi
%
\ifPst@geo@Germany
  (\psk@geo@Path/germany.dat) run
  gsave
  GrandCercle
  clip
  germany dup /REGION exch def
  CompteurRegions
  {CalculsPointsRegion
    closepath
    gsave
    mapcolor
    pst@fill  % fill
    grestore
    borderwidth
    bordercolor
    stroke
  } forall
grestore
\fi
%
\ifPst@geo@lakes
 (\psk@geo@Path/plake.dat) run
gsave
GrandCercle
clip
lake dup /REGION exch def
CompteurRegions
{CalculsPointsRegion
  closepath
   gsave
  oceancolor
  pst@fill  % fill
  grestore
%  borderwidth
%  bordercolor
0 0 1 setrgbcolor
  stroke
  }forall
grestore
\fi
%
\ifPst@geo@coasts
  (\psk@geo@Path/pcoast.dat) run
gsave
GrandCercle
clip
coasts dup /REGION exch def
CompteurRegions
{CalculsPointsRegion
   coastwidth
   coastcolor
   stroke
  }forall
grestore
\fi
%
\ifPst@geo@USA
  (\psk@geo@Path/usa.dat) run
gsave
GrandCercle
clip
usa dup /REGION exch def
CompteurRegions
{CalculsPointsRegion
   closepath
   gsave
   mapcolor
   pst@fill  % fill
   grestore
  borderwidth
  bordercolor
   stroke
  }forall
grestore
\fi
%
\ifPst@geo@Mexico
  (\psk@geo@Path/mex.dat) run
gsave
GrandCercle
clip
mexico dup /REGION exch def
CompteurRegions
{CalculsPointsRegion
   closepath
   gsave
   mapcolor
   pst@fill  % fill
   grestore
  borderwidth
  bordercolor
   stroke
  }forall
grestore
\fi
%
\ifPst@geo@Australia
  (\psk@geo@Path/aus.dat) run
gsave
GrandCercle
clip
australia dup /REGION exch def
CompteurRegions
{CalculsPointsRegion
   closepath
   gsave
   mapcolor
   pst@fill  % fill
   grestore
  borderwidth
  bordercolor
   stroke
  }forall
grestore
\fi
%
\ifPst@geo@Canada
  (\psk@geo@Path/canada.dat) run
gsave
GrandCercle
clip
canada dup /REGION exch def
CompteurRegions
{CalculsPointsRegion
   closepath
   gsave
   mapcolor
   pst@fill  % fill
   grestore
  borderwidth
  bordercolor
   stroke
  }forall
grestore
\fi
%
\ifPst@geo@wfraczon
  (\psk@geo@Path/wfraczon.dat) run
GrandCercle
clip
wfraczon dup /REGION exch def
CompteurRegions
{CalculsPointsRegion
  wfraczoncolor
  wfraczonwidth
   stroke}forall
\fi
%
\ifPst@geo@wmaglin
  (\psk@geo@Path/wmaglin.dat) run
GrandCercle
clip
wmaglin dup /REGION exch def
CompteurRegions
{CalculsPointsRegion
  wmaglincolor
  wmaglinwidth
   stroke}forall
\fi
%
\ifPst@geo@ridge
  (\psk@geo@Path/ridge.dat) run
gsave
GrandCercle
clip
ridge dup /REGION exch def
CompteurRegions
{CalculsPointsRegion
  ridgecolor
  ridgewidth
   stroke}forall
grestore
  (\psk@geo@Path/transfrm.dat) run
gsave
GrandCercle
clip
transfrm dup /REGION exch def
CompteurRegions
{CalculsPointsRegion
  transfrmcolor
  ridgewidth
   stroke}forall
grestore
  (\psk@geo@Path/trench.dat) run
gsave
GrandCercle
clip
trench dup /REGION exch def
CompteurRegions
{CalculsPointsRegion
  trenchcolor
  ridgewidth
   stroke}forall
grestore
\fi
%
\ifPst@geo@rivers
  (\psk@geo@Path/river.dat) run
GrandCercle
clip
river dup /REGION exch def
CompteurRegions
{CalculsPointsRegion
   rivercolor
   stroke}forall
\fi
%
\ifPst@geo@cities
% les capitales
  (\psk@geo@Path/c-cap.dat) run
  1.5
  citycapitals
  DrawCities
% les autres villes
  (\psk@geo@Path/c-sub.dat) run
  0.5
  CitySub
DrawCities
\fi
\ifPst@geo@capitals
% les capitales
  (\psk@geo@Path/c-cap.dat) run
  2
  citycapitals
  DrawCities
\fi
%
\ifPst@geo@gridmap
    MaillageSphere
\fi
end
}%
\end@ClosedObj}
% placer une ville en particulier
% par sa longitude et sa latitude en degr�s
% si la ville n'est pas visible
% elle est rejet�e hors de l'�cran
\def\pnodeMapIIID{\pst@object{pnodeMapIIID}}
\def\pnodeMapIIID@i(#1,#2)#3{%
   \begin@SpecialObj
   \pnode(!
   \geo@variablesMap{2}
   tx@geoDict begin
   MatriceTransformationZXY
        /Xpoint {#2 cos #1 cos mul Rsphere mul} def
        /Ypoint {#2 cos #1 sin mul Rsphere mul} def
        /Zpoint {#2 sin Rsphere mul } def
   CalculsPointsAfterTransformations
%  CalcCoordinates
%  /YcoorCity exch def /XcoorCity exch def
%   Test
%   PS 0 ge %
%    {XcoorCity YcoorCity }
%    {/XcoorCity 1000 def /YcoorCity 1000 def}
%    ifelse
formulesTroisD
     Xi Yi end){#3}%
   \end@SpecialObj%
   \ignorespaces}%
%
%
\def\mapputIIID{\@ifnextchar[{\mapputIIID@i}{\mapputIIID@i[90]}}
\def\mapputIIID@i[#1](#2,#3){\@ifnextchar[{\mapputIIID@iii[#1](#2,#3)}{\mapputIIID@ii[#1](#2,#3)}}
\def\mapputIIID@ii[#1](#2,#3)#4{\mapputIIID@iii[#1](#2,#3)[#4]{#4}}
\def\mapputIIID@iii[#1](#2,#3)[#4]#5{%
  \@ifnextchar[{\mapputIIID@iv[#1](#2,#3)[#4]#5}{\mapputIIID@iv[#1](#2,#3)[#4]#5[\@empty]}}
\def\mapputIIID@iv[#1](#2,#3)[#4]#5[#6]{{%
% [angle](longitude,lattitude)[node name]{node label}[country]
  \def\pst@tempb{all}
  \ifx\pst@tempb\psk@geo@mapCountry
    \mapputIIID@v[#1](#2,#3)[#4]{#5}%
  \else
    \def\pst@tempb{#6}%
%    \typeout{Compare \pst@tempb{} with \psk@geo@mapCountry}
    \ifx\pst@tempb\psk@geo@mapCountry
      \mapputIIID@v[#1](#2,#3)[#4]{#5}
    \fi%
  \fi%
}}
\def\mapputIIID@v[#1](#2,#3)[#4]#5{%
  \pnodeMapIIID(#2,#3){#4}
  \pst@dima=\psk@geo@nodeWidth
  \pscircle[style=psNodeMapStyle](#4){\pst@dima}
  \pscircle*(#4){0.5\pst@dima}
  \ifx\relax#5\relax\else
    \uput{1.2\pst@dima}[#1](#4){\psframebox[style=psNodeMapStyle]{#5}}%
  \fi}
%
% dessiner un m�ridien de longitude donn�e
% longitude en degr�s
\def\psmeridien{\pst@object{psmeridien}}
\def\psmeridien@i#1{%
   \pst@killglue
   \begingroup
   \begin@SpecialObj%
   \addto@pscode{%
   \geo@variablesMap{2}
   /longitude_meridien #1 def
   /meridienwidth {\pst@number\psmeridienwidth SLW} def
   /meridiencolor  {\pst@usecolor\psmeridiencolor currentrgbcolor} def
   tx@geoDict begin
   MatriceTransformationZXY
  \ifPst@geo@visibility
    /condition { 0 ge } def
  \else
   /condition { 0 le } def
  \fi
   meridien
   end}%
   \end@SpecialObj%
   \endgroup%
   \ignorespaces%
}
%
% dessiner un cercle parall�le � l'�quateur de latitude donn�e
% latitude en degr�s
\def\psparallel{\pst@object{psparallel}}
\def\psparallel@i#1{%
   \pst@killglue
   \begingroup
   \begin@SpecialObj%
   \addto@pscode{%
   \geo@variablesMap{2}
   /latitude_parallel #1 def
   /parallelwidth {\pst@number\psparallelwidth SLW} def
   /parallelcolor  {\pst@usecolor\psparallelcolor currentrgbcolor} def
   tx@geoDict begin
   MatriceTransformationZXY
     \ifPst@geo@visibility
    /condition { 0 ge } def
    \else
    /condition { 0 le } def
    \fi
   Parallel
   parallelwidth
   parallelcolor
   stroke
   end}%
   \end@SpecialObj%
   \endgroup%
   \ignorespaces%
}
%%%%%%%%%%%%%%%%%%%%%%%%%%%%
% le dessin du globe tellure
%%%%%%%%%%%%%%%%%%%%%%%%%%%%


\newpsstyle{Globe}{circles=true}
\newpsstyle{night}{fillstyle=solid,fillcolor=black,opacity=0.8}
\define@key[psset]{pst-geo}{styleGlobe}{\def\psk@styleGlobe{#1}}
\define@key[psset]{pst-geo}{styleNight}{\def\psk@styleNight{#1}}
\psset[pst-geo]{styleGlobe=Globe,styleNight=night}

\def\psGlobeTellure{\pst@object{psGlobeTellure}}
\def\psGlobeTellure@i(#1,#2)#3{% longitude, latitude et nom de la ville
   \begin@SpecialObj%
    \pnode(!/ux 0 def /uy 90 #2 sub sin def /uz 90 #2 sub cos def 0 0){ttt}%
    \psset{THETA=0,PHI=0}%
    {\psset{RotX=#2 90 sub,RotZ=90 #1 add}
    \WorldMapThreeD[style=\psk@styleGlobe,daynight=false]%
    \mapputIIID(#1,#2){#3}}%
   \WorldMapThreeD[style=\psk@styleNight,daynight=true,circles=false,blueEarth=false,
                         borders=false,islands=false,
                         rivers=false,lakes=false,
                         gridmap=false,latitudeParallel=#2]%
   \psframe*(! -3 \psk@geo@Radius\space neg -1.5 add)(!3 \psk@geo@Radius\space neg -0.5 add)
   \rput(! 0 \psk@geo@Radius\space neg -1 add){\white\psk@geo@hour h le \psk@geo@J/\psk@geo@M/\psk@geo@A}
   \end@SpecialObj%
   \ignorespaces}%

%%%%%%%%%%%%%%%%%%%%%%%%%

% �picentre d'un tremblement de terre
% Earthquake Epicenters
%
\define@key[psset]{pst-geo}{waves}{\edef\psk@geo@waves{#1}}
\psset{waves=3} % nombre de cercles concentriques
\define@key[psset]{pst-geo}{Rmax}{\edef\psk@geo@Rmax{#1}}
\psset{Rmax=1500} % rayon maximum du s�isme en km

\def\psepicenter{\pst@object{psepicenter}}
\def\psepicenter@i(#1,#2)#3{%
  \begin@ClosedObj%
  \addto@pscode{
  \geo@variablesMap{2}
    /longitude #1 def
    /latitude #2 def
    /nbre \psk@geo@waves\space def
    /dphi \psk@geo@Rmax\space 0.008983 mul def
    /circlecolor  {\pst@usecolor\pscirclecolor currentrgbcolor} def
    /circlewidth {\pst@number\pscirclewidth SLW} def
    tx@geoDict begin
    circlecolor
    circlewidth
    nbre dphi longitude latitude
    ondes
    end }%
  \ifx\relax#3\relax\else\mapputIIID(#1,#2){#3}\fi%
  \end@ClosedObj%
  \ignorespaces}
%
%%%%%%%%%%%%%%%%%%%%%%%%%%%%%%%%%%%%%%% 3dII %%%%%%%%%%%%%%%%%%%%%%%%%%%%%%%%%
%
\def\WorldMapThreeDII{\pst@object{WorldMapThreeDII}}
\def\WorldMapThreeDII@i{%
  \addbefore@par{fillstyle=solid}
  \begin@ClosedObj
  \addto@pscode{
    \geo@variablesMap{3}
    1 setlinejoin
    0 0 translate
    /increment \psk@geo@increment\space def
    /xunit {\pst@number\psxunit\space mul}def
    /yunit {\pst@number\psyunit\space mul}def
    /runit {\pst@number\psrunit\space mul}def
    /pst@fill { \psk@opacityalpha .setopacityalpha fill } def
    gsave
    newpath
    0.7 0.9 0.9 setrgbcolor
% cercle vu par l'observateur
    /RsphereScreen Rsphere DScreen mul Dobs div
      1 Rsphere dup mul Dobs dup mul div sub sqrt div runit def
    0 0 RsphereScreen 0 360 arc
    closepath
    \ifx\psk@fillstyle\relax\else fill \fi
    grestore
    /condition {PSfacette 0 le} def
    /departPhi -90 def
    tx@geoIIDict begin
      MatriceTransformation
      /Draw {
        { CalculsPoints
          stroke
        } forall
      }def
%
      /DrawRivers {
        { CalculsPoints
          gsave
          0 0 1 setrgbcolor
          CLW 0.25 mul
          stroke
          grestore
        } forall
      } def
%
      /DrawBdy {
        { CalculsPoints
          gsave
          1 0 0 setrgbcolor
          stroke
          grestore
        } forall
      } def
\ifPst@geo@maillage MaillageSphereII \fi
\ifPst@geo@Europe
    (\psk@geo@Path/europe-cil_II.dat) run
 europe-cil
  Draw
 \ifPst@geo@rivers
(\psk@geo@Path/europe-riv_II.dat) run
  europe-riv
  DrawRivers
  \fi
 \ifPst@geo@borders
(\psk@geo@Path/europe-bdy_II.dat) run
 europe-bdy
 DrawBdy
 \fi
\fi
%
\ifPst@geo@Africa
  (\psk@geo@Path/africa-cil_II.dat) run
  africa-cil
  Draw
  \ifPst@geo@rivers
    (\psk@geo@Path/africa-riv_II.dat) run
    africa-riv
    DrawRivers
  \fi
  \ifPst@geo@borders
    (\psk@geo@Path/africa-bdy_II.dat) run
    africa-bdy
    DrawBdy
  \fi
\fi
%
\ifPst@geo@Asia
  (\psk@geo@Path/asia-cil_II.dat) run
  asia-cil
  Draw
  \ifPst@geo@rivers
    (\psk@geo@Path/asia-riv_II.dat) run
    asia-riv
    DrawRivers
  \fi
  \ifPst@geo@borders
    (\psk@geo@Path/asia-bdy_II.dat) run
    asia-bdy
    DrawBdy
  \fi
\fi
%
\ifPst@geo@Northamerica
(\psk@geo@Path/northamerica-cil_II.dat) run
 northamerica-cil
  Draw
\ifPst@geo@rivers
(\psk@geo@Path/northamerica-riv_II.dat) run
  northamerica-riv
  DrawRivers
  \fi
\ifPst@geo@borders
(\psk@geo@Path/northamerica-bdy_II.dat) run
(\psk@geo@Path/northamerica-pby_II.dat) run
  northamerica-bdy DrawBdy northamerica-pby DrawBdy
\fi
\fi
%
\ifPst@geo@Southamerica
(\psk@geo@Path/southamerica-cil_II.dat) run
 southamerica-cil
  Draw
\ifPst@geo@rivers
(\psk@geo@Path/southamerica-riv_II.dat) run
  southamerica-riv
  DrawRivers
 \fi
\ifPst@geo@borders
(\psk@geo@Path/southamerica-bdy_II.dat) run
 southamerica-bdy
  DrawBdy
\fi
\fi
%
\ifPst@geo@cities
% les capitales
  (\psk@geo@Path/c-cap.dat) run
  1.5 % rayon du point
  citycapitals
  DrawCities
% les autres villes
  (\psk@geo@Path/c-sub.dat) run
  0.5 % rayon du point
  CitySub
DrawCities
\fi
\ifPst@geo@capitals
% les capitales
  (\psk@geo@Path/citycapitals.dat) run
  1.5
  citycapitals
  DrawCities
\fi
end
}%
\end@ClosedObj
\ignorespaces
}
%

%
\catcode`\@=\PstAtCode\relax
\endinput
