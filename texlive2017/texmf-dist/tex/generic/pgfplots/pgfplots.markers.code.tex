%--------------------------------------------
%
% Package pgfplots
%
% Provides a user-friendly interface to create function plots (normal
% plots, semi-logplots and double-logplots).
% 
% It is based on Till Tantau's PGF package.
%
% Copyright 2007-2013 by Christian Feuersänger.
%
% This program is free software: you can redistribute it and/or modify
% it under the terms of the GNU General Public License as published by
% the Free Software Foundation, either version 3 of the License, or
% (at your option) any later version.
% 
% This program is distributed in the hope that it will be useful,
% but WITHOUT ANY WARRANTY; without even the implied warranty of
% MERCHANTABILITY or FITNESS FOR A PARTICULAR PURPOSE.  See the
% GNU General Public License for more details.
% 
% You should have received a copy of the GNU General Public License
% along with this program.  If not, see <http://www.gnu.org/licenses/>.
%
%--------------------------------------------
	
\def\pgfplots@markers@visphase@name@none{none}%
\def\pgfplots@markers@visphase@name@afterpath@clipped{afterpath@clipped}%
\def\pgfplots@markers@visphase@name@afterpath@unclipped{afterpath@unclipped}%
\def\pgfplots@markers@visphase@name@endaxis@unclipped{endaxis@unclipped}%

% defines \pgfplotsretval to contain the visualization phase used by
% the current plot.
\def\pgfplots@markers@survey@set@visphasename{%
	\begingroup
	%
	% this validates+normalizes \tikz@plot@mark:
	\pgfplots@markers@check@for@default@plot@mark
	\ifx\tikz@plot@mark\pgfutil@empty
		\ifpgfplots@scatterplotenabled
			\def\b@pgfplots@has@plot@mark{1}%
		\else
			\def\b@pgfplots@has@plot@mark{0}%
		\fi
	\else
		\def\b@pgfplots@has@plot@mark{1}%
	\fi
	%
	% this sets \ifpgfplots@clip@marker@paths:
	\pgfplots@markers@prepare@layered@graphics@marks
	%
	\if0\b@pgfplots@has@plot@mark
		% no markers. Ok, we do not need a visualization phase for
		% markers.
		\let\pgfplotsretval=\pgfplots@markers@visphase@name@none
	\else
		\ifpgfplots@clip@marker@paths
			\let\pgfplotsretval=\pgfplots@markers@visphase@name@afterpath@clipped
		\else
			\if2\pgfplots@clipmode
				% clip mode=individual:
				\let\pgfplotsretval=\pgfplots@markers@visphase@name@afterpath@unclipped
			\else
				% clip mode=global
				\let\pgfplotsretval=\pgfplots@markers@visphase@name@endaxis@unclipped
			\fi
		\fi
	\fi
	%
	\pgfmath@smuggleone\pgfplotsretval
	\endgroup
}%

\def\pgfplots@markers@visphase@after@axis@prepare{%
	\global\pgfplotslistnewempty\pgfplots@stored@markerlist
}

% Draw any markers for which drawing has been postponed.
%
% The motivation here is that markers should be drawn on top of
% their plots, and AFTER the clip path.
%
\def\pgfplots@markers@visphase@after@axis{%
	\begingroup
	\pgfplotslistforeach\pgfplots@stored@plotlist\as\pgfplots@loc@TMPa{%
		\expandafter\pgfplots@stored@plotlist@EXTRACTENTRY\pgfplots@loc@TMPa
%\message{^^JProcessing stored plot with precommand '\meaning\pgfplots@stored@current@precmd';^^J  pgfplots@plotcmd '\meaning\pgfplots@stored@current@cmd'^^J postcommand '\meaning\pgfplots@stored@current@postcmd'^^J}%
%\message{DATA: '\meaning\pgfplots@stored@current@data';^^J}%
		% The precmd sets all required variables needed to
		% finalize a plot.
		% @see the stream preparation routines.
		\pgfplots@stored@current@precmd
		\ifx\pgfplots@markers@visphase@name\pgfplots@markers@visphase@name@endaxis@unclipped
			\ifx\pgfplots@stored@current@cmd\pgfutil@empty
				% Ah: some \pgfplotsextra command like \draw,\path,
				% \node or even \scope or \endscope.
				% Has been done earlier.
			\else
				\begingroup
				% reconfigure...
				\pgfkeyssetvalue{/pgfplots/on layer}{\pgfkeysvalueof{/pgfplots/mark layer}}%
				%
				% choose the one that is necessary for us here:
				\let\pgfplots@markers@visphase@name=\pgfplots@markers@visphase@name@afterpath@unclipped
				%
				% provide an empty list of phase names here in order
				% to turn off standard visualization phases: they
				% have been applied already.
				\pgfplots@stored@plotlist@visualize@selected@phases{}%
				\endgroup
			\fi
			\pgfplots@stored@current@postcmd
		\fi
	}%
	\endgroup
}

\def\pgfplots@execute@at@begin@plot@visualization@internal{%
	\global\let\pgfplots@stored@markerlist@last=\relax
}%

\def\pgfplots@markers@install@plotmark@handler{%
	% note: I can't check on tikz@transform because it can be '\relax'.
	\pgfplots@gettikzinternal@keyval{mark indices}{tikz@mark@list}{}%
	\pgfplots@gettikzinternal@keyval{mark}{tikz@plot@mark}{}%
	%
	%
	% do not reset \tikz@options: draw color may be acquired
	% from 'current plot style'
	%\let\tikz@options=\pgfutil@empty%
	\let\tikz@transform=\pgfutil@empty%
	\tikzset{every plot/.try,every mark}%
	%
	% This sets colors:
	\tikz@options
	%
	% This sets the \iftikz@mode@draw etc:
	%\tikz@mode
	% FIXME: using 'color=blue' will NOT activate filltrue!
	% So: if 'tikz@mode' *contains* 'fillfalse', I know what to do... 
	% but all other cases are not clear
	%--------------------------------------------------
	% \iftikz@mode@draw
	% \else
	% 	% Override the marker codes: force 'draw=none'
	% 	% even if the markers likes to be stroked:
	% 	\let\pgfusepathqfillstroke=\pgfusepathqfill
	% \fi
	% \iftikz@mode@fill
	% \else
	% 	% Override the marker codes: force 'fill=none'
	% 	% even if the markers likes to be filled:
	% 	\let\pgfusepathqfillstroke=\pgfusepathqstroke
	% \fi
	%-------------------------------------------------- 
	%
	% this here is the MAIN marker code.
	% It may be modified if scatter plot is enabled, see below.
	\def\pgfplots@loc@TMPa{%
		% the transformation has been reset before processing plot
		% marks.
		% Consequently, it applies only transformation options in "every mark".
		\tikz@transform
		\pgfuseplotmark{\tikz@plot@mark}%
	}%
	\ifpgfplots@scatterplotenabled
		% Scatter plots work like this:
		%
		% <compute per-point meta info>
		% /pgfplots/scatter/@pre marker code
		% <marker code, lowlevel>
		% /pgfplots/scatter/@post marker code
		%
		% -> that's all. The Rest is configurable with style which
		%  (re)define '@pre marker code' and '@post marker code' (see
		%  the docs for details).
		%
		% Prepare arguments for '@pre/@post' macros:
		\t@pgfplots@toka={%
			\begingroup
			% prepare programmatic access to '/data point/x' etc:
   			\pgfplotspointgetcoordinates
			\pgfplotsaxisvisphasetransformpointmeta
			\pgfkeysvalueof{/pgfplots/scatter/@pre marker code/.@cmd}\pgfeov
		}%
		\t@pgfplots@tokb=\expandafter{\pgfplots@loc@TMPa}%
		\t@pgfplots@tokc={%
			\pgfkeysvalueof{/pgfplots/scatter/@post marker code/.@cmd}\pgfeov
			\endgroup
		}%
		\edef\pgfplots@loc@TMPa{%
			\the\t@pgfplots@toka
			\the\t@pgfplots@tokb
			\the\t@pgfplots@tokc
		}%
		\if r\pgfplots@scatter@position
			% scatter/position=relative
			%
			% tikz transforms the coordinate system such that "(0,0)"
			% is the point where the mark should be placed.
		\else
			% scatter/position=absolute
			%
			% ah - we do not want tikz's transformation.
			%
			% Disable it by means of \aftergroup hackery:
			\t@pgfplots@tokc=\expandafter{\pgfplots@loc@TMPa}%
			\xdef\pgfplots@scatter@plot@mark{%
				\noexpand\begingroup
				\the\t@pgfplots@tokc
				\noexpand\endgroup
			}%
			\def\pgfplots@loc@TMPa{\aftergroup\pgfplots@scatter@plot@mark}%
		\fi
	\fi
	\def\pgfplots@loc@TMPb##1{%
		\ifx\tikz@mark@list\pgfutil@empty%
			\pgfplothandlermark{##1}%
		\else
			\pgfplothandlermarklisted{##1}{\tikz@mark@list}%
		\fi
	}%
	\t@pgfplots@tokc=\expandafter\expandafter\expandafter{\expandafter\pgfplots@loc@TMPb\expandafter{\pgfplots@loc@TMPa}}%
	\edef\pgfplots@basiclevel@plothandler{\the\t@pgfplots@tokc}%
	\expandafter\def\expandafter\pgfplots@basiclevel@plothandler\expandafter{%
		\pgfplots@basiclevel@plothandler
		\global\let\pgfplots@markers@plotstreamstart=\pgf@plotstreamstart
		\def\pgf@plotstreamstart{%
			\pgfplots@markers@plotstreamstart
			\global\let\pgfplots@markers@mark@handler@@=\pgf@plotstreampoint
			\global\let\pgf@plotstreampoint=\pgfplots@markers@mark@handler
		}%
	}%
	\let\tikz@plot@handler=\pgfplots@basiclevel@plothandler
	%
	\pgfplots@basiclevel@plothandler
	%
}%

\def\pgfplots@markers@mark@handler#1{%
	\ifpgfplots@clip
		% check for clip marker paths:
		\pgfplotsaxisifcontainspoint{\pgfplots@markers@mark@handler@@{#1}}{}%
	\else
		\pgfplots@markers@mark@handler@@{#1}
	\fi
}%

\def\pgfplots@if@has@plot@marks#1#2{%
	% make sure there is a mark set!
	\pgfplots@gettikzinternal@keyval{mark}{tikz@plot@mark}{}%
	\def\pgfplots@loc@TMPa{none}%
	\ifx\tikz@plot@mark\pgfplots@loc@TMPa
		% this here happens only in older versions of pgf.
		\let\tikz@plot@mark\pgfutil@empty
	\fi
	\ifx\tikz@plot@mark\pgfutil@empty
		#2\relax
	\else
		#1\relax
	\fi
}%

\def\pgfplots@markers@check@for@default@plot@mark{%
	\pgfplots@if@has@plot@marks{}{%
		\pgfplots@getcurrent@plothandler\pgfplots@basiclevel@plothandler
		\ifx\pgfplots@basiclevel@plothandler\pgfplothandlerdiscard
			% oh, the "only marks" plot handler-- and no plot mark!?
			% what's that!?
			\ifpgfplots@scatterplotenabled
				% scatter handles this in /pgfplots/scatter/true
				% automatically.
			\else
				% that appears to be nonsense...
				\def\tikz@plot@mark{*}%
			\fi
		\fi
	}%
}%


% Used inside of /pgfplots/scatter/classes :
\def\pgfplots@scatter@classes@pre@marker@code{%
	\pgfutil@ifundefined{pgfp@scatter@class@\pgfplotspointmeta}{%
		\let\pgfplots@loc@TMPa=\pgfplotspointmeta
		%
		% ups - no styles available? Maybe something went
		% wrong with the 'scatter src' key. Check whether it
		% was accidentally a numerical style:
		\if1\csname pgfpmeta@\pgfplotspointmetainputhandler @issymbolic\endcsname
		\else
			% ok, be fault tolerant and round to an integer:
			\pgfplotscoordmath{meta}{tofixed}{\pgfplotspointmeta}%
			\begingroup
			\pgfkeys{/pgf/number format/precision=0}%
			\expandafter\pgfmathroundto\expandafter{\pgfmathresult}%
			\pgfmath@smuggleone\pgfmathresult
			\endgroup
			\let\pgfplotspointmeta=\pgfmathresult
		\fi
		% now, check again:
		\pgfutil@ifundefined{pgfp@scatter@class@\pgfplotspointmeta}{%
			% still not possible? Then, try truncating the
			% number to an integer.
			\expandafter\pgfutil@in@\expandafter.\expandafter{\pgfplotspointmeta}%
			\ifpgfutil@in@
				\def\pgfplots@loc@TMPb##1.##2\relax{\def\pgfplotspointmeta{##1}}%
				\expandafter\pgfplots@loc@TMPb\pgfplots@loc@TMPa\relax
			\fi
			% now, check again:
			\pgfutil@ifundefined{pgfp@scatter@class@\pgfplotspointmeta}{%
				\pgfutil@ifundefined{pgfp@scatter@WARNING@\pgfplotspointmeta}{%
					\pgfplotswarning{scatter classes no such class}\pgfplotspointmeta\pgfeov%
					\expandafter\gdef\csname pgfp@scatter@WARNING@\pgfplotspointmeta\endcsname{ALREADY CHECKED}%
				}{}%
				\def\pgfplots@loc@TMPa{}%
			}{%
				\expandafter\let\expandafter\pgfplots@loc@TMPa\csname pgfp@scatter@class@\pgfplotspointmeta\endcsname
			}%
		}{%
			\expandafter\let\expandafter\pgfplots@loc@TMPa\csname pgfp@scatter@class@\pgfplotspointmeta\endcsname
		}%
	}{%
		\expandafter\let\expandafter\pgfplots@loc@TMPa\csname pgfp@scatter@class@\pgfplotspointmeta\endcsname
	}%
	\expandafter\scope\expandafter[\pgfplots@loc@TMPa]%
}%

% validates 'mark layer' such that it can be queried afterwards.
\def\pgfplots@markers@prepare@layered@graphics@marks{%
	\ifpgfplots@layered@graphics
		\pgfkeysgetvalue{/pgfplots/mark layer}\pgfplots@loc@TMPa
		\edef\pgfplots@loc@TMPa{\pgfplots@loc@TMPa}%
		\ifx\pgfplots@loc@TMPa\pgfutil@empty
			\def\pgfplots@loc@TMPa{auto}%
		\fi
		\def\pgfplots@loc@TMPb{auto}%
		\ifx\pgfplots@loc@TMPa\pgfplots@loc@TMPb
			% COMBINATION: 'activate layers,mark layer=auto' --> keep
			% backwards compatibility with 'clip marker paths'
			% -> nothing to do.
			\pgfkeyslet{/pgfplots/mark layer}\pgfutil@empty%
		\else
			% mark layer=like plot
			\def\pgfplots@loc@TMPb{like plot}%
			\ifx\pgfplots@loc@TMPa\pgfplots@loc@TMPb
				% draw markers together with their plot (no matter
				% on which layer):
				\pgfplots@clip@marker@pathstrue
				\pgfkeyslet{/pgfplots/mark layer}\pgfutil@empty%
			\else
				\edef\pgfplots@loc@TMPb{\pgfkeysvalueof{/pgfplots/on layer}}%
				\ifx\pgfplots@loc@TMPa\pgfplots@loc@TMPb
					% nothing to do; they are on the same layer
					% anyway: (this is the same as 'like plot'
					\pgfplots@clip@marker@pathstrue
					\pgfkeyslet{/pgfplots/mark layer}\pgfutil@empty%
				\else
					% OK, marks will be plotted on their own layer.
				\fi
			\fi
		\fi
	\else
		% reset its value:
		\pgfkeyslet{/pgfplots/mark layer}\pgfutil@empty%
	\fi
}%

