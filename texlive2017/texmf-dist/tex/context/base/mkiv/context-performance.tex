% language=uk

\usemodule[art-01,abr-01]

\starttext

\subject {About performance}

If you use \PDFTEX, \XETEX\ and \LUATEX\ you will notice a difference in speed.
Don't draw conclusions too easily from simple tests like:

\starttyping
\dorecurse{1000}{test\page}
\stoptyping

or (also tests the file system):

\starttyping
\dorecurse{1000}{\input tufte\blank}
\stoptyping

The wide engines \XETEX\ and \LUATEX\ have more work to do than \PDFTEX, because
the input is \UTF\ and \UNICODE\ fonts are used. Of course running \PDFTEX\ on
extensive \UTF\ input will compensate it a bit.

Comparing \XETEX\ and \PDFTEX\ (using \MKII) and \LUATEX\ (using \MKIV) is kind
of useless anyway because the \LUATEX\ with \MKIV\ combination is not only doing
more advanced things, which costs time, but at the other hand has more efficient
alternatives, like for instance using \MPLIB, which gains a lot.

So, it's best to compare speeds with a mixed content document: multiple fonts,
text and math, images, \METAPOST\ graphics, structural components, tables, etc.

On the average \PDFTEX\ is the fastest, but offering less functionality, while
\LUATEX\ with \MKIV\ is often faster than \XETEX\ with \MKII. On complex products
like the \METAFUN\ manual or when processing complex \XML\ files a \LUATEX\ is
much faster than a \PDFTEX\ run.

There is some startup time involved which is normally not that much, and initial
font loading is also not really a burden, but of course for a few page document
it brings down the number of pages processed per second. Normalizing the input
takes a bit but applying \OPENTYPE\ font features takes much more. If you find
unacceptable bottlenecks just let me know (but better first check performance in
the other engines). Of course inefficient coding of styles (massive font switches
where a simple one could do) are no reason for a complaint. Keep in mind:
flexibility comes at a price.

A lot of time went into making sure that \CONTEXT\ runs efficiently on \LUATEX\
and we keep improving the performance. This is not so much an engine issue but
more one of the macro package. Of course what is true for \CONTEXT\ \MKIV\ can be
different for other macro packages but comparing with them makes no sense because
the differences in functionality.

There are many features in \CONTEXT\ \MKIV\ that demand analysis of the text
stream and when a spot is found where that feature has to be applied some actions
takes place. Most such features don't have an impact on overall performance (due
to the analysis) unless they are used. Of course font processing has the largest
impact, but they are also the most flexible and extensible feature, especially
when you use dynamic features. And font processing is always enabled. Speed has
been improved over time and we're currently at a point that further speedup makes
not much sense and would only make the code more complex. This is not to say that
we cannot improve performance in the future.

Hans Hagen \crlf Hasselt NL

\stoptext
