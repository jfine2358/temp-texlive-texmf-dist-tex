We humans, more than any other species, edit our ncRNA molecules to a remarkable
degree. Not even other primates carry out this reaction as well as we do. We also
edit particularly extensively in the brain. This makes editing of ncRNA an
attractive candidate process to explain why we are mentally so much more
sophisticated than our primate relatives, even though we share so much of our DNA
template in common.

In some ways, this is the beauty of ncRNAs. They create a relatively safe method
for organisms to use to alter various aspects if cellular regulation. Evolution
has probably favoured this mechanism because it is simply too risky to try to
improve function by changing proteins. Proteins, you see, are the Mary Poppins of
the cell. They are \quote {practically perfect in every way}.
