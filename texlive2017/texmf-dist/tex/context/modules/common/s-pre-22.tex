%D \module
%D   [      file=s-pre-22,
%D        version=2000.08.07,
%D          title=\CONTEXT\ Style File,
%D       subtitle=Presentation Environment 22,
%D         author=Hans Hagen,
%D           date=\currentdate,
%D      copyright={PRAGMA ADE \& \CONTEXT\ Development Team}]
%C
%C This module is part of the \CONTEXT\ macro||package and is
%C therefore copyrighted by \PRAGMA. See mreadme.pdf for
%C details.

%D This style was made on behalf of the \PDFTEX\ presentation
%D at \TUG\ 2000. It cycled a summary of each talk, with name
%D and title. When documenting this style, I changed
%D reprocessing into pushing on layers.
%D
%D A \quote {problem} like this can be solved in several ways:
%D
%D \startitemize
%D \item writing a lot of semi||complex \TEX\ code as shown
%D \item keeping track of positions and draw everything on the
%D      page layer
%D \item defining an overlay for each summary and changing the
%D      order when flushing
%D \item maintaining a so called field stack
%D \stopitemize
%D
%D We go for the first method. We assume that summaries are
%D simple text snippets.

\startmode[asintended] \setupbodyfont[lbr] \stopmode

\setupbodyfont[14.4pt]

%D We use the whole page area.

\setuppapersize
  [S6][S6]

\setuplayout
  [topspace=0cm,
   backspace=0cm,
   header=0pt,
   footer=0pt,
   width=middle,
   height=middle]

%D We define a couple of matching colors and gray scales.
%D Watch out, some are really meant to look dim.

\setupcolors
  [state=start]

\definecolor[PageColor] [s=.50]
\definecolor[TextColor] [s=.80]
\definecolor[DoneColor] [s=.65]

\definecolor[TopColor] [r=.5,g=.6,b=.7]
\definecolor[BotColor] [r=.6,g=.7,b=.5]
\definecolor[DotColor] [r=.7,g=.5,b=.6]

%D We will use foreground colors. Because these can interfere
%D with the colors they overload, we can bets make sure that
%D we don't have local colors.

\setupinteraction
  [state=start,
   color=,
   contrastcolor=]

%D The presentation is supposed to cycle automatically.

\setupinteractionscreen
  [option=max,
   delay=5]

\setuppagetransitions

%D We will use random positioning of objects.

\setupsystem
  [random=medium]

%D We have two kind of graphics: the page background and
%D the shape around the textual elements.

\defineoverlay [shape] [\uniqueMPgraphic{shape}]
\defineoverlay [page]  [\reuseMPgraphic{page}]

\startreusableMPgraphic{page}
  StartPage ;
    filldraw Page withcolor \MPcolor{PageColor} ;
    pickup pencircle scaled .375cm ;
    for i=1 upto 200 :
      drawdot center Page randomized (PaperWidth,PaperHeight)
        withcolor \MPcolor {DotColor} ;
    endfor ;
  StopPage ;
\stopreusableMPgraphic

\startuniqueMPgraphic{shape}
  path p ;
  p := unitsquare xyscaled(OverlayWidth,OverlayHeight) superellipsed .90 ;
  draw p withpen pencircle scaled .50cm withcolor \MPcolor{PageColor} ;
  fill p                                withcolor OverlayColor ;
  draw p withpen pencircle scaled .25cm withcolor OverlayLineColor ;
  currentpicture := currentpicture xysized(OverlayWidth,OverlayHeight) ;
\stopuniqueMPgraphic

%D The resizing at the end is needed to get a nice inverted
%D hyperlink when we click on it in a browser.

%D Behind the page we put a forward button:

\defineoverlay [forward] [\overlaybutton{forward}]

%D The content will be managed by means of two layers.

\definelayer [main] \defineoverlay [main] [\composedlayer{main}]
\definelayer [temp] \defineoverlay [temp] [\composedlayer{temp}]

%D The first layer will hold everything to be shown, while
%D the second one gets the data we currently focus on.
%D Therefore the first layer will not be flushed each page.

\setuplayer
  [main]
  [state=repeat]

%D All the overlays go onto the page area.

\setupbackgrounds
  [page]
  [background={page,forward,main,temp}]

%D We have to collect all data before we typeset it. Each
%D element will be typeset dim and bright. The dim
%D alternatives will be collected on the main layer, but each
%D bring one goes onto a box stack.

\initializeboxstack{Summary}
\initializeboxstack{Subtext}

%D The macros that take care of all this manipulations look
%D more complicated than they actually are. We use a
%D scratchbox to collect and inspect data. Also, because we
%D typeset each element twice, we need to make sure that we use
%D the same random seed for both.

\doglobal\newcounter\CurrentSummary

\def\StartSummary% bottom bot-title top-title
  {\dodoublegroupempty\doStartSummary}

\def\doStartSummary#1#2%
  {\doglobal\increment\CurrentSummary
   \setbox\scratchbox=\hbox{\strut#1}
   \getrandomseed\RandomSeed
   \setlayer[main]
     {\RandomSubtextBox{DoneColor}{BotColor}{BotColor}}
   \setrandomseed\RandomSeed
   \savebox{Subtext}{\CurrentSummary}
     {\RandomSubtextBox{TextColor}{BotColor}{black}}
   \setbox\scratchbox=\hbox \bgroup
     \setbox\scratchbox=\hbox{\bfb\setstrut\strut\quad#2\quad}%
     \SetAcceptableWidth
     \framed [offset=0pt,width=fit,frame=off,align=middle,strut=no]
       \bgroup \setupwhitespace[big]
         \doifsomething{#2}{\noindent\box\scratchbox\blank}}

\def\StopSummary
  {\egroup \egroup
   \getrandomseed\RandomSeed
   \setlayer[main]
     {\RandomSummaryBox{DoneColor}{TopColor}{TopColor}}
   \setrandomseed\RandomSeed
   \savebox{Summary}{\CurrentSummary}
     {\RandomSummaryBox{TextColor}{TopColor}{black}}}

%D A \type {\doStartSummary#1#2#3\StopSummary} could have been
%D used too but this one is less sensitive for catcode changes
%D (not that we expect problems like this in this kind of
%D application).

%D The width is either derived from the width ot the title or
%D at random. The final width of the box is detemined by the
%D content.

\def\SetAcceptableWidth
  {\scratchdimen=.5\makeupwidth
   \ifdim\wd\scratchbox>.5\makeupwidth
     \getrandomdimen\hsize{\wd\scratchbox}{.8\makeupwidth}%
   \else
     \getrandomdimen\hsize{.5\makeupwidth}{.7\makeupwidth}%
   \fi}

%D The subtext box goes at the bottom, somewhere in the right
%D corner.

\def\RandomSubtextBox#1#2#3%
  {\vbox to \makeupheight
     {\vfill
      \hbox to \makeupwidth
        {\hfill
         \button
           [offset=2ex,frame=off,background=shape,strut=no,
            backgroundcolor=#1,framecolor=#2,foregroundcolor=#3]
           {\copy\scratchbox}%
           [previouspage]%
         \getrandomdimen\scratchdimen{.5cm}{2.5cm}%
         \hskip\scratchdimen}
      \getrandomdimen\scratchdimen{.5cm}{1.5cm}
      \vskip \scratchdimen}}

%D The main text goes in the top half of the page, not to
%D far from the center. The last \type {\vskip} makes sure
%D that we don't clash with the subtexts.

\definereference[thispage][page(\CurrentSummary)]

\def\RandomSummaryBox#1#2#3%
  {\vbox to \makeupheight
     {\getrandomdimen\scratchdimen{.5cm}\makeupheight
      \vskip 0pt plus \scratchdimen
      \hbox to \makeupwidth
        {\getrandomdimen\scratchdimen{.5cm}\makeupwidth
         \hskip 0pt plus \scratchdimen
         \button
           [offset=3ex,frame=off,background=shape,strut=no,
            backgroundcolor=#1,framecolor=#2,foregroundcolor=#3]
           {\copy\scratchbox}%
           [thispage]%
         \getrandomdimen\scratchdimen{.5cm}\makeupwidth
         \hskip 0pt plus \scratchdimen}
      \getrandomdimen\scratchdimen{.5cm}\makeupheight
      \vskip 0pt plus \scratchdimen
      \vskip.2\makeupheight}}

%D Because we conly collect data, we hav eto make sure that at
%D some moment it is processed and flushed. The following loop
%D does this.

\def\BuildPage
  {\dorecurse{\CurrentSummary}
     {\startstandardmakeup
        \setlayer[temp]{\foundbox{Summary}\recurselevel}
        \setlayer[temp]{\foundbox{Subtext}\recurselevel}
      \stopstandardmakeup}}

%D We hook this macro into the \type {\stoptext} macro.

\appendtoks \BuildPage \to \everystoptext

%D We still need a title page.

\def\TitlePage%
  {\dodoublegroupempty\doTitlePage}

\long\def\doTitlePage#1#2%
  {\ifsecondargument
     \MakeTitlePage{#1}{#2}
   \else\iffirstargument
     \MakeTitlePage{\currentdate}{#1}
   \else
     \MakeTitlePage{\currentdate}{Welcome}
   \fi\fi}

\def\MakeTitlePage#1#2%
  {\StartSummary{#1}{#2}\StopSummary}

%D For old times sake:

\long\def\StartTopic#1\StopTopic{\StartSummary#1\StopSummary}

\doifnotmode{demo}{\endinput}

%D The demo text.

\starttext

\TitlePage{Indeed}{The Title Page}

\StartSummary{Alpha}{Title}
  A simple and not too long text just to show the topic.
  A simple and not too long text just to show the topic.
  A simple and not too long text just to show the topic.
\StopSummary

\StartSummary{Beta and Gamma}{Another Title}
  A simple and not too long text just to show the topic.
  A simple and not too long text just to show the topic.
\StopSummary

\StartSummary{Delta}{Some Title}
  A simple and not too long text just to show the topic.
\StopSummary

\StartSummary{Epsilon}{What A Title}
  A simple and not too long text just to show the topic.
  A simple and not too long text just to show the topic.
  A simple and not too long text just to show the topic.
\StopSummary

\StartSummary{Zeta, Eta and Theta}{Eh, A Title}
  A simple and not too long text just to show the topic.
  A simple and not too long text just to show the topic.
  A simple and not too long text just to show the topic.
\StopSummary

\StartSummary{Omega}
  A simple and not too long text just to show the topic.
  A simple and not too long text just to show the topic.
  A simple and not too long text just to show the topic.
  A simple and not too long text just to show the topic.
\StopSummary

\stoptext
