%  Format of examples: %
%
%  All leipzig gloss macros are of the form {\Abbrv}
%
%  where Abbrv starts with a capital letter, and is equivalent to
%  the gloss abbreviations suggested in the Leipzig glossing rules.
%
% That is, \Acc{} will print \textsc{acc}, etc.
%

\newleipzig{abl}{abl}{ablative}         %ablative
\newleipzig{abs}{abs}{absolutive}       %absolutive
\newleipzig{acc}{acc}{accusative}       %accusative
\newleipzig{adj}{adj}{adjective}        %adjective
\newleipzig{adv}{adv}{adverbial}        %adverb(ial)
\newleipzig{aarg}{a}{agent}             %agent-like argument of canonical transitive verb
\newleipzig{agr}{agr}{agreement}        %agreement
\newleipzig{all}{all}{allative}         %allative
\newleipzig{antip}{antip}{antipassive}  %antipassive
\newleipzig{appl}{appl}{applicative}    %applicative
\newleipzig{art}{art}{article}          %article
\newleipzig{aux}{aux}{auxiliary}        %auxiliary
\newleipzig{ben}{ben}{benefactive}      %benefactive
\newleipzig{caus}{caus}{causative}      %causative
\newleipzig{clf}{clf}{classifier}       %classifier
\newleipzig{com}{com}{comitative}       %comitative
\newleipzig{comp}{comp}{complementizer} %complementizer
\newleipzig{compl}{compl}{completive}   %completive
\newleipzig{cond}{cond}{conditional}    %conditional
\newleipzig{cop}{cop}{copula}           %copula
\newleipzig{cvb}{cvb}{converb}          %converb
\newleipzig{dat}{dat}{dative}           %dative
\newleipzig{decl}{decl}{declarative}    %declarative
\newleipzig{def}{def}{definite}         %definite
\newleipzig{dem}{dem}{demonstrative}    %demonstrative
\newleipzig{det}{det}{determiner}       %determiner
\newleipzig{dist}{dist}{distal}         %distal
\newleipzig{distr}{distr}{distributive} %distributive
\newleipzig{du}{du}{dual}               %dual
\newleipzig{dur}{dur}{durative}         %durative
\newleipzig{erg}{erg}{ergative}         %ergative
\newleipzig{excl}{excl}{exclusive}      %exclusive
\newleipzig{f}{f}{feminine}             %feminine
\newleipzig{foc}{foc}{focus}            %focus
\newleipzig{fut}{fut}{future}           %future
\newleipzig{gen}{gen}{genitive}         %genitive
\newleipzig{imp}{imp}{imperative}       %imperative
\newleipzig{incl}{incl}{inclusive}      %inclusive
\newleipzig{ind}{ind}{indicative}       %indicative
\newleipzig{indf}{indf}{indefinite}     %indefinite
\newleipzig{inf}{inf}{infinitive}       %infinitive
\newleipzig{ins}{ins}{instrumental}     %instrumental
\newleipzig{intr}{intr}{intransitive}   %intransitive
\newleipzig{impf}{impf}{imperfective}   %imperfective
\newleipzig{irr}{irr}{irrealis}         %irrealis
\newleipzig{loc}{loc}{locative}         %locative
\newleipzig{m}{m}{masculine}            %masculine
\newleipzig{n}{n}{neuter}               %neuter
\newleipzig{neg}{neg}{negative}         %negation, negative
\newleipzig{nmlz}{nmlz}{nominalizer}    %nominalizer/nominalization
\newleipzig{nom}{nom}{nominative}       %nominative
\newleipzig{obj}{obj}{object}           %object
\newleipzig{obl}{obl}{oblique}          %oblique
\newleipzig{pass}{pass}{passive}        %passive
\newleipzig{parg}{p}{patient}           %patient
\newleipzig{pfv}{pfv}{perfective}       %perfective
\newleipzig{pl}{pl}{plural}             %plural
\newleipzig{poss}{poss}{possessive}     %possessive
\newleipzig{pred}{pred}{predicative}    %predicative
\newleipzig{prf}{prf}{perfect}          %perfect
\newleipzig{prs}{prs}{present}          %present
\newleipzig{prog}{prog}{progressive}    %progressive
\newleipzig{proh}{proh}{prohibitive}    %prohibitive
\newleipzig{prox}{prox}{proximal}       %proximal/proximate
\newleipzig{pst}{pst}{past}             %past
\newleipzig{ptcp}{ptcp}{participle}     %participle
\newleipzig{purp}{purp}{purposive}      %purposive
\newleipzig{q}{q}{question particle}    %question particle/marker
\newleipzig{quot}{quot}{quotative}      %quotative
\newleipzig{recp}{recp}{reciprocal}     %reciprocal
\newleipzig{refl}{refl}{reflexive}      %reflexive
\newleipzig{rel}{rel}{relative}         %relative
\newleipzig{res}{res}{resultative}      %resultative
\newleipzig{sbj}{sbj}{subject}          %subject
\newleipzig{subj}{subj}{subjunctive}    %subjunctive
\newleipzig{sg}{sg}{singular}           %singular
\newleipzig{sarg}{s}{argument of intransitive verb}  
                                        %single argument of intransitive verb
\newleipzig{top}{top}{topic}            %topic
\newleipzig{tr}{tr}{transitive}         %transitive
\newleipzig{voc}{voc}{vocative}         %vocative

%%  Define short versions of person + number:
\newleipzig{first}{1}{first person}%
\newleipzig{second}{2}{second person}%
\newleipzig{third}{3}{third person}%

\newcommand{\Fsg}{{\First}{\Sg}}%
\newcommand{\Fdu}{{\First}{\Du}}%
\newcommand{\Fpl}{{\First}{\Pl}}%
\newcommand{\Ssg}{{\Second}{\Sg}}%
\newcommand{\Sdu}{{\Second}{\Du}}%
\newcommand{\Spl}{{\Second}{\Pl}}%
\newcommand{\Tsg}{{\Third}{\Sg}}%
\newcommand{\Tdu}{{\Third}{\Du}}%
\newcommand{\Tpl}{{\Third}{\Pl}}%