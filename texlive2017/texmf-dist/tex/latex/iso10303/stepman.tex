% stepman.tex   Description of option style files for STEP
\documentclass[wd,copyright,letterpaper]{isov2}
\usepackage{stepv13}
\usepackage{irv12}
\usepackage{apv12}
\usepackage{aicv1}
\usepackage{atsv11}
%%\usepackage{isomods}  % must come after the step packages
\usepackage{hyphenat}
\usepackage{comment}

\ifpdf
  \pdfoutput=1
  \usepackage[plainpages=false,
              pdfpagelabels,
              bookmarksnumbered,
              hyperindex=true
             ]{hyperref}
\fi

% general required preamble commands
\standard{ISO/WD 10303-3456}
\yearofedition{2002}
\languageofedition{(E)}
\renewcommand{\extrahead}{N 47b}  % add doc N number to headers
\partno{3456}
\series{documentation methods}
\doctitle{LaTeX package files for ISO 10303: User manual}
\ballotcycle{2}
% required preamble commands for an AP
\aptitle{implicit drawing}
\aicinaptrue % only if the AP uses AICs
\mapspectrue  % only if AP uses mapping specification
% required preamble commands for an ATS
\APtitle{abstract painting}
\APnumber{299}

\changemarkstrue

\makeindex

\setcounter{tocdepth}{3} % add more levels to table of contents
%
% Rest of preamble is some special macro definitions for this document only
%
\makeatletter
%
%   the \file{} command
%
\newcommand{\file}[1]{\textsf{#1}}
%
%   the \meta{} command
%
\begingroup
\obeyspaces%
\catcode`\^^M\active%
\gdef\meta{\begingroup\obeyspaces\catcode`\^^M\active%
\let^^M\do@space\let \do@space%
\def\-{\egroup\discretionary{-}{}{}\hbox\bgroup\it}%
\m@ta}%
\endgroup
\def\m@ta#1{\leavevmode\hbox\bgroup$<$\it#1\/$>$\egroup
    \endgroup}
\def\do@space{\egroup\space
    \hbox\bgroup\it\futurelet\next\sp@ce}
\def\sp@ce{\ifx\next\do@space\expandafter\sp@@ce\fi}
\def\sp@@ce#1{\futurelet\next\sp@ce}
%
% the \setlabel{id}{num} command
% this is based on the kernel \refstepcounter macro (ltxref.dtx)
%
%%\newcounter{lbl}
\ifpdf
  \newcommand{\setlabel}[2]{%
    \protected@write\@auxout{}{%
      \string\newlabel{#1}{{#2}{\thepage}{setlabel\relax}{label.#2}{}}}%
  }
\else
  \newcommand{\setlabel}[2]{%
    \protected@write\@auxout{}{%
      \string\newlabel{#1}{{#2}{\thepage}}}%
  }
\fi
%
% index a command
\newcommand{\bs}{\symbol{'134}}
\newcommand{\ixcom}[1]{\index{#1/ @{\tt \protect\bs #1}}}
% index an environment
\newcommand{\ixenv}[1]{\index{#1 @{\tt #1} (environment)}}
% index an option
\newcommand{\ixopt}[1]{\index{#1 @{\tt #1} (option)}}
% index a package
\newcommand{\ixpack}[1]{\index{#1 @\file{#1} (package)}}
% index a class
\newcommand{\ixclass}[1]{\index{#1 @\file{#1} (class)}}
% index in typewriter font
\newcommand{\ixtt}[1]{\index{#1@{\tt #1}}}
% index LaTeX
\newcommand{\ixltx}{\index{latex@\LaTeX}}
% index LaTeX 2e
\newcommand{\ixltxe}{\index{latex2e@\LaTeX 2e}}
% index LaTeX v2.09
\newcommand{\ixltxv}{\index{latex209@\LaTeX{} v2.09}}
% index a file
\newcommand{\ixfile}[1]{\index{#1@\file{#1}}}
\makeatother
%
%
%   set some labels
% step
\setlabel{;ssne}{A}
%%%\setlabel{;sior}{B}
%%%\setlabel{;scil}{C}
\setlabel{;seg}{D}
% aic
\setlabel{;sesl}{4}
% ap
\setlabel{;sireq}{4}
\setlabel{;suof}{4.1}
\setlabel{;sao}{4.2}
\setlabel{;saa}{4.3}
\setlabel{;saim}{5}
\setlabel{;smap}{5.1}
\setlabel{;saesl}{5.2}
\setlabel{;scr}{6}
\setlabel{;saeel}{A}
\setlabel{;sasn}{B}
\setlabel{;simreq}{C}
\setlabel{;spics}{D}
\setlabel{;saam}{F}
\setlabel{;sarm}{G}
\setlabel{;saeg}{H}
\setlabel{;scil}{J}
\setlabel{tabB1}{B.1}
\setlabel{;uof1}{5.1.2}
\setlabel{;uoflast}{5.1.4}
%
% define a new length
\newlength{\prwlen}
%
% new (La)TeX macros
\newcommand{\latex}{LaTeX}
\newcommand{\tex}{TeX}
%
%%%%%%%%%% END SPECIAL MACROS
%
%   end of preamble
%
\begin{document}


\STEPcover{
%\scivnumber{987}
\wg{EC}
\docnumber{47b}
\oldwg{EC}
\olddocnumber{47a}
\docdate{2002/09/04}
%\partnumber{3456}
%\doctitle{LaTeX package files for ISO 10303: User manual}
%\status{Working draft}
%\primcont
\abstract{This document describes and illustrates the \latex{} macros
for typesetting ISO~10303. The International Organisation for
Standardisation (ISO) has specified editorial directives for all 
international standards published by them. The \latex{} macros
described here were developed to meet additional editorial directives 
for ISO~10303. } % end abstract
\keywords{\latex, document preparation, typesetting ISO standards}
%\dateprojo{May 1996}
\owner{Peter R. Wilson}
\address{Boeing Commercial Airplane\newline
            PO Box 3701 \newline
            MS 2R-97 \newline
            Seattle, WA 98124-2207 \newline
            USA}
\telephone{+1 (206) 544-0589}
\fax{+1 (206) 544-5889}
\email{\url{peter.r.wilson@boeing.com}}
\altowner{Peter R. Wilson}
\altaddress{Boeing Commercial Airplane \newline
            PO Box 3701 \newline
            MS 2R-97 \newline
            Seattle, WA 98124-2207 \newline
            USA}
\alttelephone{+1 (206) 544-0589}
\altfax{+1 (206) 544-5889}
\altemail{\url{peter.r.wilson@boeing.com}}
\comread{\draftctr This document serves two purposes. Firstly, it provides a description
         of the current \latex{} macros for ISO 10303. Secondly, the source
         can be used as an example of using the \latex{} commands.
         Although the document is written as though it were a
         standard, it is not, and is not intended to become, 
         a standard.} %end comread
} % end of STEPcover

\Foreword

\fwdshortlist
\endForeword%
{Annexes A, B and C are}  % normative annexes
{Annexes D, E and F are} % informative annexes

\begin{Introduction}%%%%%%%%%%{documentation methods}

    This part of ISO 10303 specifies the \latex{} facilities specifically 
designed for use in preparing the various parts of this standard.

\begin{majorsublist}
\item the \file{step} package facility;
\item the \file{ir} package facility;
\item the \file{ap} package facility;
%%%\item the \file{am} package facility;
\item the \file{aic} package facility;
\item the \file{ats} package facility.
\end{majorsublist}

    This part of ISO~10303 is intended to be used in conjunction with
\textit{\latex{} for ISO standards: User manual}
which is based in part upon material in the ISO/IEC Directives,
Part 2 (\textit{Rules for the structure and drafting of International 
Standards, Fourth edition}).
The \latex{} facilities described here are based as well
upon the specifications given in ISO TC184/SC4 N1217n 
(\textit{SC4 Supplementary directives --- Rules for the structure
and drafting of SC4 standards for industrial data}).


\sclause*{Overview}


    This document describes a set of \latex{} macro files for use within
ISO~10303, commonly called STEP (STandard for the Exchange of Product
model data). The electronic source of this  document 
also provides an example of the use of these files.

    The current set of macro files have been developed by 
Peter Wilson (\url{peter.r.wilson@boeing.com}) from a macro file developed
by Kent Reed (NIST) for \latex{} v2.09. In turn, this was a revision of
files originally created by Phil Spiby (CADDETC), based on earlier work 
by Phil Kennicott (GE).\footnote{In mid 1994 \latex{} was upgraded 
from version 2.09 to what is called \latex 2e. The files described in 
this document are only applicable to \latex 2e (support for \latex{} v2.09
was dropped in September 1997).}


\begin{anote} 
It is important to remember that these macro files are only compatible with 
\latex 2e.
\end{anote} % end anote

    Documents produced with the \latex{} files have been twice reviewed 
by the ISO Editorial Board in Geneva for conformance to their 
typographical requirements. The first review was of a set of Draft 
International Standard documents. This review resulted in some changes 
to the style files. The second review was of a set of twelve 
International Standard documents (ISO 10303:1994). Likewise, this
review led to changes in the style files to bring the documents into 
conformance.

    With the issuance of the first STEP release, the opportunity was 
taken to provide a new baseline release of the package files. 
In particular, one STEP specific package file is available for all 
STEP parts, while others contain only commands relevant to the 
documentation of particular series of parts. The range of package 
files may be extended in the future to cater for 
documentation specific to all STEP parts.

   The 1997 baseline release was also designed to cater for the 
fact that a major update of \latex{} to \latex 2e took place during 1994.
\latex 2e is the only officially supported version of \latex.

    Because ISO standard documents have a very structured layout, the 
\file{isov2} class and the package files described here have been 
designed to reflect the logical document structure to a much greater 
extent than the `standard' \latex{} files. 

    With ISO's move toward accepting documents in PDF and HTML, 
the advent of second
editions of some of the STEP parts, and a new edition of the STEP
Supplementary Directives, a 2002
baseline release has been developed and is documented here. 



\end{Introduction}

\stepparttitle{Documentation methods: LaTeX package files for ISO 10303:
User manual}


\scopeclause

This part of ISO~10303 describes a set of \ixltx\latex{} facilities for typesetting
documents according to the ISO/IEC Directives Part 2, together with the 
Supplementary Directives for drafting and presentation of ISO~10303.

\begin{inscope}{part of ISO~10303}
\item use of \latex{} for preparing ISO~10303 documents.
\end{inscope}

\begin{outofscope}{part of ISO~10303}
\item use of \latex{} for preparing ISO standard documents in general;
\item use of \latex{} in general;
\item use of other document preparation systems.
\end{outofscope}

\textbf{IMPORTANT:} The preparation of this document has been partly
funded by the US Government and is not subject to copyright.
Any copyright notices within the document are for illustrative purposes only.

\normrefsclause \label{sec:nrefs}

\normrefbp{part of ISO~10303}
\begin{nreferences}

\isref{ISO/IEC Directives, Part 2}{Rules for the structure and drafting 
           of International Standards, Fourth edition.}

\isref{ISO TC 184/SC4 N1217:2001(E)}{SC4 Supplementary directives --- 
       Rules for the structure and drafting of SC4 standards for 
       industrial data.}

%\isref{ISO 10303-1:1994}{Industrial automation systems and integration ---
%        Product data representation and exchange --- 
%        Part 1: Overview and fundamental principles.}
\nrefparti

%\isref{ISO 10303-11:1994}{Industrial automation systems and integration --- 
%        Product data representation and exchange --- 
%        Part 11: Description methods:
%        The EXPRESS language reference manual.}
\nrefpartxi

%\disref{ISO/TR 10303-12:---}{Industrial automation systems and integration ---
%        Product data representation and exchange ---
%        Part 12: Description methods:
%        The EXPRESS-I language reference manual.}
\nrefpartxii

%\disref{ISO/IEC 8824-1:---}{Information technology ---
%       Open systems interconnection ---
%       Abstract syntax notation one (ASN.1) ---
%       Part 1: Specification of basic notation.}
\nrefasni

\disref{P. R. WILSON:---}{LaTeX for ISO standards: User manual.}

\end{nreferences}

\defabbclause
%\clause{Terms, definitions, and abbreviations}

\partidefhead
%\sclause{Terms defined in ISO 10303-1}

    This part of ISO~10303 makes use of the following terms defined in 
ISO~10303-1:

\begin{olddefinitions}
\olddefinition{application protocol (AP)} \index{Application Protocol}
                                            \index{AP}
\olddefinition{integrated resource} \index{Integrated Resource}
\end{olddefinitions}


\otherdefhead
%\sclause{Other definitions}

    For the purposes of this part of ISO~10303, the following definitions
apply.

\begin{definitions}
\definition{boilerplate}{Text whose wording is fixed and which has been
agreed to be present in a specific type of document.} \index{boilerplate}
\definition{style file}{A set of \latex{} macros assembled into a 
single file with an extension \file{.sty}.}
            \index{style file}
\definition{package file}{A style file for use with \latex 2e.}
            \index{package file}
\definition{facility}{A generic term for a set of \latex{} macros
          assembled for a common purpose. The macros may be defined in
          either a style file or a package file.}\index{facility}

\end{definitions}

\abbsubclause
%\sclause{Abbreviations}

    For the purposes of this part of ISO 10303, the following abbreviations
 apply.

\begin{symbols}
\symboldef{AIC}{Application Interpreted Construct} \index{AIC}
\symboldef{AM}{Application Module} \index{AM}
\symboldef{AP}{Application Protocol}  \index{AP}
\symboldef{DIS}{Draft International Standard} \index{DIS}
\symboldef{IS}{International Standard}         \index{IS}
\symboldef{ISOD}{ISO/IEC Directives, Part 2} \index{ISOD} \index{ISO/IEC Directives}
\symboldef{SD}{Supplementary Directives --- 
  \textit{SC4 Supplementary directives --- Rules for the structure and
   drafting of SC4 standards for industrial data}}\index{SD}\index{Supplementary Directives}
\symboldef{IS-REVIEW}{The ISO Editorial Board review (September 1994) of 
            twelve IS documents
            for conformance to ISO typographical and 
            layout requirements.} \index{IS-REVIEW}
\end{symbols}



\clause{Conformance requirements}  \label{sec:iconform}

    The facility files shall not be modified in any manner.

    If there is a need to modify any of the macro definitions then this
shall be done using the \latex{} 
\verb|\renewcommand|\ixcom{renewcommand} and/or the
\verb|\renewenvironment|\ixcom{renewenvironment}
commands. These shall be placed in a new \file{.sty} file (or files) 
which shall be called in within the preamble\index{preamble} of the 
document being typeset.

    There shall be no author specified \verb|\label{...}| commands where
the first two characters of the label are \verb|;s| (semicolon and `s');
the creation of labels starting with these characters is reserved to the 
maintainer of the facility files.

\begin{anote} For conformance to the \file{isov2} class, author specified
labels starting with the characters \verb|;i| (semicolon and `i') are
prohibited.
\end{anote}


\fcandaclause
%\clause{Fundamental concepts and assumptions}

    It is assumed that the reader of this document is familiar with the
\ixltx\latex{} document preparation system and in particular
with the \file{isov2}\ixclass{isov2} class and associated facilities 
described in 
\textit{LaTeX for ISO standards: User manual}.

\begin{note}Reference~\bref{lamport} describes the
      \latex{} system.
\end{note} % end note

    The reader is also assumed to be familiar with the ISO/IEC Directives 
Part~2 (ISOD)\index{ISOD} and
the SC4 Supplementary directives for the structure and drafting of 
SC4 standards (SD).\index{ISOD}\index{SD}

    If there are any discrepancies between the layout and wording of this 
document and the requirements of the ISOD or the SD,
then the requirements in those documents shall be
followed for ISO~10303 standard documents.

    The packages described herein have been designed to be used with
the \file{isov2}\ixclass{isov2} document class. It is highly unlikely that the
packages will perform at all with any other \latex{} document class.

    Because of many revisions over the years to the packages described
herein, a naming convention has been adopted for the package files.
The naming convention is that the
primary name of the file is suffixed by \file{v\#}, where
\file{\#} is the primary version number of the file in question.
All file primary names have been limited to a maximum of eight characters.

\begin{note}Table~\ref{tab:curfiles} shows the versions of the files
that were current at the time of publication.
\ixpack{step}\ixfile{stepv13.sty}
\ixpack{ir}\ixfile{irv12.sty}
\ixpack{ap}\ixfile{apv12.sty}
\ixpack{aic}\ixfile{aicv1.sty}
\ixpack{ats}\ixfile{atsv11.sty}
%%%\ixpack{am}\ixfile{amv1.sty}
\end{note} % end note

\begin{table}
\centering
\caption{File versions current at publication time} \label{tab:curfiles}
\begin{tabular}{|l|l|l|} \hline
\textbf{Facility} & \textbf{File}   & \textbf{Version} \\ \hline\hline
\file{step}    & \file{stepv13.sty} & v1.3.2 \\
\file{ir}      & \file{irv12.sty}   & v1.2   \\
\file{ap}      & \file{apv12.sty}   & v1.2   \\
%%%\file{am}      & \file{amv1.sty}    & v1.0   \\
\file{aic}     & \file{aicv1.sty}   & v1.0   \\
\file{ats}     & \file{atsv11.sty}  & v1.1   \\ 
\hline
\end{tabular}
\end{table}


\begin{note}
This document is not, and is never intended to become,
 a standard, although it has been laid out in a 
similar, but not necessarily identical, manner.
\end{note} % end note


\clearpage
\clause{The \file{step} package facility}

    The \file{step}\ixpack{step} package facility provides commands 
and environments 
applicable to all the ISO~10303 series of documents.

\sclause{Preamble commands}

    Certain commands shall be put in the preamble\index{preamble}
of any document.

    The command 
\verb|\partno{|\meta{number}\verb|}|\ixcom{partno}
is used to specify the Part number of the ISO~10303 standard
(e.g., \verb|\partno{3456}|).

    The command
\verb|\series{|\meta{series title}\verb|}|\ixcom{series}
is used to specify the name of the ISO~10303 series of which the Part 
is a member (e.g., \verb|\series{application modules}|).

    The command
\verb|\doctitle{|\meta{informal title}\verb|}|\ixcom{doctitle}
is used to specify the title to be used on the cover sheet.
For example: \\
\verb|\doctitle{LaTeX package files for ISO 10303: User manual}|

    The command
\verb|\ballotcycle{|\meta{number}\verb|}|\ixcom{ballotcycle}
is used to specify the ballot cycle number for the document
(e.g., \verb|\ballotcycle{2}|).

    The command\ixcom{ifhaspatents}
\verb|\haspatentstrue|\ixcom{haspatentstrue} shall be put in the
preamble when the document includes identified patented material;
otherwise the command \verb|\haspatentsfalse|\ixcom{haspatentsfalse}
may, but need not, be used instead.

    The \verb|\extrahead|\ixcom{extrahead} macro, from the \file{isov2}
class, shall be defined to be the document
number (e.g., \verb|\renewcommand{\extrahead}{47a}|).



\begin{anote}
The commands \verb|\standard|\ixcom{standard}, 
\verb|\yearofedition|\ixcom{yearofedition} and 
\verb|\languageofedition|\ixcom{languageofedition} from the \file{isov2}
class must also be put in the preamble.
\end{anote}


\sclause{Cover page}

    The command \verb+\STEPcover{+\meta{commands}\verb+}+\ixcom{STEPcover}
produces a cover page for a STEP document. 
The complete list of commands is shown below.

\begin{itemize}
\item \verb+\wg{+\meta{working group}\verb+}+\ixcom{wg}
      the working 
      group or other committee producing the document e.g., WG 5
\item \verb+\docnumber{+\meta{number}\verb+}+\ixcom{docnumber}
       the number
       of the document e.g., 156
\item \verb+\docdate{+\meta{date}\verb+}+\ixcom{docdate}
       date of 
       publication e.g., 1993/07/03
\item \verb+\oldwg{+\meta{working group}\verb+}+\ixcom{oldwg}
       superseded 
       working group e.g., WG 1
\item \verb+\olddocnumber{+\meta{number}\verb+}+\ixcom{olddocnumber}
        number of previous document e.g., 107
\item \verb+\abstract{+\meta{text}\verb+}+\ixcom{abstract}
        an abstract 
        of the document
\item \verb+\keywords{+\meta{text}\verb+}+\ixcom{keywords}
        for listing 
        relevant keywords
\item \verb+\owner{+\meta{text}\verb+}+\ixcom{owner}
         name of the project leader
\item \verb+\address{+\meta{text}\verb+}+\ixcom{address}
        address of the project leader
\item \verb+\telephone{+\meta{number}\verb+}+\ixcom{telephone}
         the project leader's telephone number
\item \verb+\fax{+\meta{number}\verb+}+\ixcom{fax}
         the project leader's fax number
\item \verb+\email{+\meta{text}\verb+}+\ixcom{email}
        Email address of the project leader
\item \verb+\altowner{+\meta{text}\verb+}+\ixcom{altowner}
         name of the editor of the document
\item \verb+\altaddress{+\meta{text}\verb+}+\ixcom{altaddress}
         the editor's address 
\item \verb+\alttelephone{+\meta{number}\verb+}+\ixcom{alttelephone}
         the editor's telephone number
\item \verb+\altfax{+\meta{number}\verb+}+\ixcom{altfax}
         the editor's fax number
\item \verb+\altemail{+\meta{text}\verb+}+\ixcom{altemail}
           the editor's Email address 
\item \verb+\comread{+\meta{text}\verb+}+\ixcom{comread}
           comments to 
           the reader
\end{itemize}

    Use only those commands within \verb|\STEPcover| that are relevant 
to the purposes at hand. The order of the commands within 
\verb|\STEPcover| is immaterial.

\begin{example}
The commands used to produce the cover sheet for one version of this 
document were:
\begin{verbatim}
\STEPcover{
\wg{EC}
\docnumber{41}
\oldwg{EC}
\olddocnumber{35}
\docdate{1994/08/19}
\abstract{This document describes the \latex{} style files created for ISO~10303.
          It also describes the program GenIndex which provides some 
          capabilities to assist in the creation of indexes for \latex{}
          documents in general.}
\keywords{\latex, Style file, GenIndex, Index}
\owner{Peter R Wilson}
\address{NIST\newline
         Bldg. 220, Room A127 \newline
         Gaithersburg, MD 20899 \newline
         USA }
\telephone{+1 (301) 975-2976}
\email{\texttt{pwilson@cme.nist.gov}}
\altowner{Tony Day}
\altaddress{Sikorsky Aircraft}
\comread{This document serves two purposes. Firstly, it provides a description
         of the current \latex{} style file for ISO 10303. Secondly, the source
         can be used as an example of using the \latex{} commands.} % end comread
} % end of STEPcover
\end{verbatim}
Note the use of the \verb|\newline| command instead of \verb|\\| in 
the argument of the \verb|\address| command to indicate a new line. The
\verb|\newline| is needed to ensure satisfactory conversion to HTML.
\end{example} % end example

    The macro \verb|\draftctr|\ixcom{draftctr} generates boilerplate that
may be used in the `Comments to Reader' section of a cover page.
\begin{example}
The \latex{} source \verb|\draftctr This document \ldots| prints:

\draftctr This document \ldots
\end{example}

\sclause{Heading commands}

    The commands described in this subclause specify various `standard'
clause headings.

\ssclause{The Foreword commands}

    The \verb+\Foreword+\ixcom{Foreword} command specifies that a 
table of contents, list of figures and a list of tables be produced. 
Page numbering is roman style and the table of contents starts on page iii.
A new unnumbered clause entitled Foreword is started containing both 
ISO required boilerplate and boilerplate\index{boilerplate}
text specific to ISO 10303.


    Any text may be written after the \verb|\Foreword| command. The
Foreword clause is ended by the 
\verb+\endForeword{+\meta{norm annexes}\verb+}{+\meta{inf annexes}\verb+}+ 
command.\ixcom{endForeword} This command takes two parameters.
\begin{enumerate}
\item \meta{norm annexes} A phrase that starts the sentence 
      `\meta{norm annexes} a normative part of this part \ldots'.
     If there are no normative annexes, then use an empty
     argument (i.e., \verb|{}| with no spaces between the braces).
\item \meta{inf annexes} A phrase that starts the sentence 
     `\meta{inf annexes} for information only.'.
     If there are no informative annexes, then use an
     empty argument.
\end{enumerate}

    The \verb|\endForeword| command produces some additional 
boilerplate\index{boilerplate} text specifically for ISO 10303. 

\begin{example}
The \latex{} source for the Foreword for this document is:
\begin{verbatim}
\Foreword
\fwdshortlist
\endForeword
{Annexes A, B and C are}  % normative annexes
{Annexes D, E and F are} % informative annexes
\end{verbatim}
\end{example} % end example


    The \verb|\fwdshortlist|\ixcom{fwdshortlist} command 
produces boilerplate text for inclusion in the foreword referencing
the STEP parts and series. 
\begin{example}
In this document, the command \verb|\fwdshortlist| prints:

\fwdshortlist
\end{example}

    The \verb|\steptrid|\ixcom{steptrid} command
produces boilerplate text for inclusion in the foreword describing 
the creators of a STEP Technical Report.

\begin{example}
The \latex{} command \verb|\steptrid| in this document prints:
  
\steptrid
\end{example}


\ssclause{The Introduction environment}

    The 
\verb+\begin{Introduction}+\ixenv{Introduction}
environment starts a new unnumbered clause 
entitled Introduction and adds some boilerplate\index{boilerplate}
text specifically for ISO~10303.

\begin{example}
    The following \latex{} source was used to specify the Introduction 
to this document. \label{ex:intro}
\begin{verbatim}
\begin{Introduction}

    This part of ISO 10303 specifies the \latex{} facilities 
specifically designed for use in preparing the various parts of 
this standard.

\begin{majorsublist}
\item the \file{step} package facility;
\item the \file{ir} package facility;
\item the \file{ap} package facility;
\item the \file{aic} package facility;
\item the \file{atc} package facility.
\end{majorsublist}

    This part of ISO 10303 is intended to be used ...

\sclause*{Overview}

    This document describes a set of \latex{} files for use
within ISO~10303 ...

\end{Introduction}
\end{verbatim}
\end{example} % end example


\ssclause{The stepparttitle command}

   The \verb+\stepparttitle{+\meta{part title}\verb+}+\ixcom{stepparttitle}
command produces the title for
an ISO~10303 part, where \meta{part title} is the title of the part.

\begin{anexample}The title for this document was produced using:
\begin{verbatim}
\stepparttitle{Documentation methods:
               LaTeX package files for ISO 10303: User manual}
\end{verbatim}
\end{anexample} % end example


\ssclause{Other headings}

    Most of these commands take no parameters. They start document clauses
with particular titles. The commands that take no parameters are listed
in \tref{tab:noparamhead}. Some of these headings commands have predefined
labels, which are also listed in the table.
\ixcom{partidefhead}
\ixcom{otherdefhead}
\ixcom{introsubhead}
\ixcom{fcandasubhead}
\ixcom{shortnamehead}
\ixcom{picshead}
\ixcom{objreghead}
\ixcom{docidhead}
\ixcom{schemaidhead}
\ixcom{expresshead}
\ixcom{listingshead}
\ixcom{expressghead}
%%\ixcom{modelscopehead}
\ixcom{techdischead}
\ixcom{exampleshead}

\begin{anote}
 In the tables, C = clause, SC = subclause, SSC = subsubclause,
NA = normative annex, IA = informative annex.
\end{anote} % end note

\settowidth{\prwlen}{\quad Protocol Implementation Conformance Statement}
\begin{table}
\centering
\caption{STEP package parameterless heading commands}
\label{tab:noparamhead}
\begin{tabular}{|l|c|p{\prwlen}|l|} \hline
\textbf{Command} & \textbf{Clause} & \textbf{Default text} & \textbf{Label} \\ \hline
\verb|\partidefhead| & SC & Terms defined in ISO 10303-1 &  \\
\verb|\otherdefhead| & SC & Other definitions & \\
\verb|\introsubhead| & SC & Introduction &  \\
\verb|\fcandasubhead| & SC & Fundamental concepts and assumptions & \\
\verb|\shortnamehead| & NA & Short names of entities & \verb|;ssne| \\
\verb|\picshead| & NA & Protocol Implementation Conformance Statement (PICS) proforma & \verb|;spics| \\
\verb|\objreghead| & NA & Information object registration  & \verb|;sior| \\
\verb|\docidhead| & SC & Document identification & \\
\verb|\schemaidhead| & SC & Schema identification &  \\
\verb|\expresshead| & IA & \Express{} listing &  \\
\verb|\listingshead| & IA & Computer interpretable listings & \verb|;scil| \\
\verb|\expressghead| & IA & \ExpressG\ diagrams & \verb|;seg| \\
%%%%\verb|\modelscopehead| & IA & Model scope & \verb|;sms| \\
\verb|\techdischead| & IA & Technical discussions & \verb|;std| \\ 
\verb|\exampleshead| & IA & Examples & \verb|;sex| \\
\hline
\end{tabular}
\end{table}

    The commands listed in \tref{tab:paramhead} are equivalent to the
general sectioning commands, but are intended to indicate the start
of a particular documentation element. These commands take either one
or two parameters. The parameters are denoted in the column headed
`Parameterized title'.
\ixcom{refdefhead}
\ixcom{schemahead}
\ixcom{typehead}
\ixcom{entityhead}
\ixcom{rulehead}
\ixcom{functionhead}
\ixcom{atypehead}
\ixcom{anentityhead}
\ixcom{arulehead}
\ixcom{afunctionhead}
\ixcom{aschemaidhead}
\ixcom{singletypehead}
\ixcom{singleentityhead}
\ixcom{singlerulehead}
\ixcom{singlefunctionhead}

\begin{table}
\centering
\caption{STEP package parameterized heading commands}
\label{tab:paramhead}
\begin{tabular}{|l|c|l|} \hline
\textbf{Command} & \textbf{Clause} & \textbf{Parameterized title} \\ \hline
\verb|\refdefhead| & SC & Terms defined in \meta{ISO ref} \\
\verb|\schemahead| & C & \meta{schema name} \\
\verb|\singletypehead| & SC & \meta{schema name} type definition:
\meta{type name} \\
\verb|\typehead| & SC & \meta{schema name} type definitions \\
\verb|\atypehead| & SSC & \meta{type name} \\
\verb|\singleentityhead| & SC & \meta{schema name} entity definition:
\meta{entity name} \\
\verb|\entityhead| & SC & \meta{schema name} entity definitions \meta{group} \\
\verb|\anentityhead| & SSC & \meta{entity name} \\
\verb|\singlerulehead| & SC & \meta{schema name} rule definition:
\meta{rule name} \\
\verb|\rulehead| & SC & \meta{schema name} rule definitions \\
\verb|\arulehead| & SSC & \meta{rule name} \\
\verb|\singlefunctionhead| & SC & \meta{schema name} function definition:
\meta{function name} \\
\verb|\functionhead| & SC & \meta{schema name} function definitions \\
\verb|\afunctionhead| & SSC & \meta{function name} \\
\verb|\aschemaidhead| & SSC & \meta{schema name} identification \\ \hline
\end{tabular}
\end{table}

\sclause{Miscellaneous commands}

    The following commands provide some printing options for commonly 
occurring situations. The \verb|\nexp{}|\ixcom{nexp} command is intended 
to be used for printing \Express{} \index{express@{\Express}} entity names etc.
\begin{itemize}
\item The command \verb|\B{abc}|\ixcom{B} prints \B{abc}
\item The command \verb|\E{abc}|\ixcom{E} prints \E{abc}
\item The command \verb|\Express|\ixcom{Express} prints \Express{}
\item The command \verb|\ExpressG|\ixcom{ExpressG} prints \ExpressG{}
\item The command \verb|\ExpressI|\ixcom{ExpressI} prints \ExpressI{}
\item The command \verb|\ExpressX|\ixcom{ExpressX} prints \ExpressX{}
\item The command \verb|\BG{|\meta{mathsymbol}\verb|}|\ixcom{BG} prints 
      \meta{mathsymbol} in bold font.
\item The command \verb|\HASH|\ixcom{HASH} prints \HASH{}
\item The command \verb|\LT|\ixcom{LT} prints \LT{}
\item The command \verb|\LE|\ixcom{LE} prints \LE{}
\item The command \verb|\NE|\ixcom{NE} prints \NE{}
\item The command \verb|\INE|\ixcom{INE} prints \INE{}
\item The command \verb|\GE|\ixcom{GE} prints \GE{}
\item The command \verb|\GT|\ixcom{GT} prints \GT{}
\item The command \verb|\CAT|\ixcom{CAT} prints \CAT{}
%\item The command \verb|\HAT|\ixcom{HAT} prints \HAT{}
\item The command \verb|\QUES|\ixcom{QUES} prints \QUES{}
%\item The command \verb|\BS|\ixcom{BS} prints \BS{}
\item The command \verb|\IEQ|\ixcom{IEQ} prints \IEQ{}
\item The command \verb|\INEQ|\ixcom{INEQ} prints \INEQ{}
\item The command \verb|\nexp{an\_entity}|\ixcom{nexp} prints \nexp{an\_entity}
\item The command \verb|\xword{ExpResS\_KeyworD}|\ixcom{xword}
      prints \xword{ExpResS\_KeyworD}
\end{itemize}

The command \verb|\ix{|\meta{word or phrase}\verb|}|\ixcom{ix} both prints 
its parameter and also makes an index entry out of it.

The command \verb|\mnote{|\meta{Marginal note text}\verb|}|\ixcom{mnote}
prints its parameter as a 
marginal note. \mnote{Quite a lot of marginal note text.}
Remember, though, that marginal notes are only printed when the 
\file{isov2}\ixclass{isov2} class \file{draft}\ixopt{draft} option
is used. Marginal notes are not allowed by ISO.

\ssclause{Standard reference commands}

    Many parts of STEP use the same normative or informative references.
The most common of these are provided via commands. The currently available 
commands are listed in \tref{tab:nrefc}.
\ixcom{nrefasni}
\ixcom{nrefparti}
\ixcom{nrefpartxi}
\ixcom{nrefpartxii}
\ixcom{nrefpartxxi}
\ixcom{nrefpartxxii}
\ixcom{nrefpartxxxi}
\ixcom{nrefpartxxxii}
\ixcom{nrefpartxli}
\ixcom{nrefpartxlii}
\ixcom{nrefpartxliii}

    The naming convention used for references to parts of ISO~10303 is to
end the command name with the number of the part expressed in lower case 
Roman numerals. Should further references to parts of ISO~10303 be added later, 
the same naming convention will be used.

\begin{table}
\centering
\caption{Commands for common references to standards} \label{tab:nrefc}
\begin{tabular}{|l|l|} \hline
\textbf{Standard} & \textbf{Command} \\ \hline
ISO/IEC 8824-1 & \verb|\nrefasni| \\
ISO 10303-1    & \verb|\nrefparti|  \\
ISO 10303-11   & \verb|\nrefpartxi|  \\
ISO 10303-12   & \verb|\nrefpartxii|  \\
ISO 10303-21   & \verb|\nrefpartxxi|  \\
ISO 10303-22   & \verb|\nrefpartxxii|  \\
ISO 10303-31   & \verb|\nrefpartxxxi|  \\
ISO 10303-32   & \verb|\nrefpartxxxii|  \\
ISO 10303-41   & \verb|\nrefpartxli|  \\
ISO 10303-42   & \verb|\nrefpartxlii|  \\
ISO 10303-43   & \verb|\nrefpartxliii|  \\ \hline
\end{tabular}
\end{table}


\begin{example} The normative references in this document were input as:
\begin{verbatim}
\begin{nreferences}
\isref{ISO/IEC Directives, Part 2}{Rules for the structure and drafting 
       International Standards, Fourth edition.}
\isref{...}
\nrefparti
\nrefpartxi
\nrefpartxii
\nrefasni
\disref{P. R. WILSON:---}{LaTeX for ISO standards: User manual.}
\end{nreferences}
\end{verbatim}
\end{example}

\begin{anote}
For the commands providing references to STEP parts, the part number 
is denoted by lowercase Roman numerals. Should further reference
commands be provided for other STEP parts, then the same naming scheme
will be used.
\end{anote}

    Some informative bibliographic reference commands are also provided.

The command \verb|\bibidefo|\ixcom{bibidefo} produces the reference
entry to the IDEF0 document and \verb|\brefidefo|\ixcom{brefidefo}
can be used for citing the reference in the body of the document.

The commands \verb|\bibidefix|\ixcom{bibidefix} and
\verb|\bibieeedefix|\ixcom{bibieeedefix} produce the reference entry
to the original FIPS version of IDEF1X and the IEEE version of IDEF1X
respectively. 
The command \verb|\brefidefix|\ixcom{brefidefix} can be used for
citing an IDEF1X reference in the body of the document.

   IDEF0 and IDEF1X are references \brefidefo{} and \brefidefix{}
in the bibliography.


\begin{example} Part of the bibliography for this document looks like:
\begin{verbatim}
\begin{references}
...
\reference{BRYAN, M.,}{SGML --- An Author's Guide to the Standard Generalized
           Markup Language,}{Addison-Wesley Publishing Co., 1988. }\label{bryan}
\bibidefo
\bibieeeidefix
\reference{RESSLER, S.,}{The National PDES Testbed Mail Server User's Guide,}
           {NSTIR 4508, National Institute of Standards and Technology,
           Gaithersburg, MD 20899. January 1991.} \label{ressler}
...
\end{references}
\end{verbatim}
\end{example}

\begin{example}The source for one of the sentences above was:
\begin{verbatim}
IDEF0 and IDEF1X are references \brefidefo{} and \brefidefix{} in the bibliography.
\end{verbatim}
\end{example}

    
\sclause{Commands for documenting EXPRESS code} \index{express@\Express\}


    The Supplementary Directives\index{SD} specify the layout of the 
documentation of \Express{} code. The following commands are intended 
to serve two purposes:
\begin{enumerate}
\item To provide environments for the documentation of entity 
      attributes, etc.;
\item To provide begin and end tags around all the \Express{} code 
      documentation.
\end{enumerate}

    This latter purpose is to provide an enabling capability for the 
automatic extraction of portions of the documentation of an 
\Express{} model so that they could be placed into another document. 
For example, tools could be developed that would automatically extract 
pieces of resource model documentation and place them into an AP document.

\begin{anote}
This document uses the \file{hyphenat}\ixpack{hyphenat} 
package which enables automatic hyphenation of `words'
containing the underscore character command 
%(\verb|\_|\index{_/@\verb|\_|}). 
(\verb|\_|\index{_/@\texttt{\bs\_}}). 
Such words would normally have to
be coded as \verb|long\_\-word| to ensure potential hyphenation 
at the position of the underscore. When using the \file{hyphenat} package
it is an error to put the \verb|\-|\ixcom{-} discretionary
hyphen command after the underscore command as this then stops further
hyphenation.
\end{anote}


\ssclause{Environments ecode, eicode and excode}

    The \verb|ecode|\ixenv{ecode} environment is for 
tagging \Express{} code. It prints the appropriate title
and sets up the relevant fonts.

\begin{anexample} The following \latex{} source code:
\begin{verbatim}
\begin{ecode}\ixent{an\_entity}
\begin{verbatm}  % read verbatm as verbatim
*)
ENTITY an_entity;
  attr : REAL;
END_ENTITY;
(*
\end{verbatm}   % read verbatm as verbatim
\end{ecode}
\end{verbatim}

produces:

\begin{ecode}\ixent{an\_entity}
\begin{verbatim}
*)
ENTITY an_entity;
  attr : REAL;
END_ENTITY;
(*
\end{verbatim}
\end{ecode}
\end{anexample} % end example

    Similarly, the \verb|eicode|\ixenv{eicode} and
\verb|excode|\ixenv{excode} environments are for tagging \ExpressI{} 
and \ExpressX{} code and setting up the relevant titles and fonts.


\ssclause{Environment attrlist}

    The \verb|attrlist|\ixenv{attrlist} environment produces 
the heading for attribute definitions and sets up 
a \verb|description|\ixenv{description} list.

\begin{anexample}The following \latex{} source code:
\begin{verbatim}
\begin{attrlist}
\item[attr\_1] The \ldots
\item[attr\_2] This \ldots
\end{attrlist}
\end{verbatim}

produces:

\begin{attrlist}
\item[attr\_1] The \ldots
\item[attr\_2] This \ldots
\end{attrlist}
\end{anexample} % end example

\ssclause{Environment fproplist}

    The \verb|fproplist|\ixenv{fproplist} environment is similar to 
\verb|attrlist|\ixenv{attrlist} except that it is for
formal propositions.

\begin{anexample}The following \latex{} source code:
\begin{verbatim}
\begin{fproplist}
\item[un\_1] The value of \ldots\ shall be unique.
\item[gt\_0] The value of \ldots\ shall be greater than zero.
\end{fproplist}
\end{verbatim}

produces:

\begin{fproplist}
\item[un\_1] The value of \ldots\ shall be unique.
\item[gt\_0] The value of \ldots\ shall be greater than zero.
\end{fproplist}
\end{anexample} % end example

\ssclause{Other listing environments}

    The environments \verb|iproplist|\ixenv{iproplist}, 
\verb|enumlist|\ixenv{enumlist}, and \verb|arglist|\ixenv{arglist} are
similar to \verb|attrlist|\ixenv{attrlist}.
 Respectively they are environments for
informal propositions, enumerated items, and argument definitions.

\ssclause{Indexing}

    The command \verb|\ixent{|\meta{entity}\verb|}|\ixcom{ixent} 
generates an index
entry for the entity \meta{entity}.

    There are similar macros, each of which takes the name of the 
declaration as its argument, for indexing the other \Express{} declarations:
\verb|\ixenum|\ixcom{ixenum} for enumeration,
\verb|\ixfun|\ixcom{ixfun} for function,
\verb|\ixproc|\ixcom{ixproc} for procedure,
\verb|\ixrule|\ixcom{ixrule} for rule,
\verb|\ixsc|\ixcom{ixsc} for subtype\_constraint,
\verb|\ixschema|\ixcom{ixschema} for schema,
\verb|\ixselect|\ixcom{ixselect} for select, and
\verb|\ixtype|\ixcom{ixtype} for type.

\ssclause{Documentation tagging}

    Several environments are defined to tag the general documentation 
of \Express{} code. \index{express@\Express\}

    The environment \verb+\begin{espec}{+\meta{name}\verb+}+\ixenv{espec}
may be used to enclose, and give a name to, a complete specification 
block for an \Express{} entity. There are analogous environments --- 
\verb+fspec+\ixenv{fspec}, 
\verb+rspec+\ixenv{rspec}, 
\verb+sspec+\ixenv{sspec}, and
\verb+tspec+\ixenv{tspec} --- 
for functions, rules, schemas and types respectively.

    The \verb|dtext|\ixenv{dtext} environment may be used to anonymously 
enclose descriptive text.

\begin{example}\label{ex:code} Here is the suggested tagged documentation 
style for part of an \Express{} model.
\begin{verbatim}
%\ssclause{committee\_def}
\begin{espec}{committee_def}
\begin{dtext}
    A committee is composed of an odd number of people. 
Each committee also has a name.
    The ideal size of a committee is less than three.

\begin{anote} Figures and tables may also be placed here. \end{anote} % end note
\end{dtext}
\begin{ecode}\ixent{committee\_def}
\begin{verbatm} % read verbatm as verbatim
*)
ENTITY committee_def;
  title   : name;
  members : SET [1:?] OF person;
DERIVE
  ideal : BOOLEAN := SIZEOF(members) = 1;
UNIQUE
  un1 : title;
WHERE
  odd_members : ODD(SIZEOF(members));
END_ENTITY;
(*
\end{verbatm}   % read verbatm as verbatim
\end{ecode}
\begin{attrlist}
\item[title] The name of the committee.
\item[members] The people who form the committee.
\item[ideal] TRUE if there is only one person 
             on the committee.
             That is, if the committee is the ideal size.
\end{attrlist}
\begin{fproplist}
\item[un1] The \nexp{title} of the committee shall be unique.
\item[odd\_members] There shall be an odd number of people 
                    on the committee.
\end{fproplist}
\begin{iproplist}
\item[chair] The members of a committee shall appoint one of 
             their number as
             chair of the committee.
\end{iproplist}
\end{espec}
\end{verbatim}
\end{example} % end example

\begin{example}
The code in \eref{ex:code} produces the following result:

\begin{espec}{committee_def}
\begin{dtext}
    A committee is composed of an odd number of people. 
Each committee also has a name.
The ideal size of a committee is less than three.

\begin{anote} Figures and tables may also be placed here. \end{anote} % end note
\end{dtext}
\begin{ecode}\ixent{committee\_def}
\begin{verbatim}
*)
ENTITY committee_def;
  title   : name;
  members : SET [1:?] OF person;
DERIVE
  ideal : BOOLEAN := SIZEOF(members) = 1;
UNIQUE
  un1 : title;
WHERE
  odd_members : ODD(SIZEOF(members));
END_ENTITY;
(*
\end{verbatim}   % read verbatm as verbatim
\end{ecode}
\begin{attrlist}
\item[title] The name of the committee.
\item[members] The people who form the committee.
\item[ideal] TRUE if there is only one person on the committee. That is, if
             the committee is the ideal size.
\end{attrlist}
\begin{fproplist}
\item[un1] The \nexp{title} of the committee shall be unique.
\item[odd\_members] There shall be an odd number of people on the committee.
\end{fproplist}
\begin{iproplist}
\item[chair] The members of a committee shall appoint one of their number as
             chair of the committee.
\end{iproplist}
\end{espec}

\end{example} % end example


\sclause{Commands producing boilerplate text} \index{boilerplate}

    The following commands produce boilerplate text as specified by the 
Supplementary Directives\index{SD}.

\begin{anote}
 In the examples, 
the parameters of those commands that
take them have been specified in 
\textit{this font style} so their effects can
be seen in the resulting printed text.
\end{anote}

\ssclause{Definition of \ExpressG}

    The \verb|\expressgdef|\ixcom{expressgdef} prints the boilerplate
for where the definition of \ExpressG{} can be found.

\begin{anexample}
The command \verb|\expressgdef| prints: 

\expressgdef
\end{anexample} 

\ssclause{Major subdivision listing}

    The \verb|majorsublist|\ixenv{majorsublist}
environment prints the boilerplate for the heading of a listing of
major subdivisions of the standard and starts an itemized list.
An illustration of its use is given in \eref{ex:intro} 
on page~\pageref{ex:intro}.

The heading text is produced by the 
\verb|\majorsubname|\ixcom{majorsubname} command.

\begin{anexample} The command \verb|\majorsubname| command prints:

\majorsubname

\end{anexample}

\ssclause{Schema introduction}

    The command \verb|\schemahead{|\meta{schema name}\verb|}|\ixcom{schemahead} prints the heading for a schema clause.

    The command \verb+\schemaintro{+\meta{schema name}\verb+}+\ixcom{schemaintro} 
produces the boilerplate for the introduction to an \Express{} schema
clause.

\begin{anexample}The command \verb|\schemaintro{\nexp{this\_schema}}| prints:

\schemaintro{\nexp{this\_schema}}
\end{anexample}
  


\ssclause{Short names of entities}

    The command \verb|\shortnamehead|\ixcom{shortnamehead} prints the
heading for the short names annex.

    The command \verb|\shortnames|\ixcom{shortnames} 
produces the boilerplate for the
introduction to the annex listing short names.

\begin{anexample}The command \verb|\shortnames| prints:

\shortnames 
\end{anexample} %end example

\ssclause{Registration commands}

    The command \verb|\objreghead|\ixcom{objreghead} prints the heading
for the information object registration annex.

    The command \verb|\docidhead|\ixcom{docidhead} prints the heading
for the document identification subclause.


    The command 
\verb+\docreg{+\meta{version no}\verb+}+\ixcom{docreg}
produces the boilerplate for document registration. The command takes
one parameter:
\meta{version no} which is the version number.\footnote{The
SD say that the version number should be 1 for a first edition IS.
The version number is incremented by one for each corrigenda,
amendment or new edition.}

\begin{example}The command \verb|\docreg{1}|
         prints:

\docreg{\textit{1}} 
\end{example} % end example

    The command \verb|\schemaidhead|\ixcom{schemaidhead} prints the heading
for the schema identification subclause. 
The command 
\verb|\aschemaidhead{|\meta{schema name}\verb|}|\ixcom{aschemaidhead} 
prints the heading for a particular schema identification subsubclause.

    The command
\verb+\schemareg{+\meta{version no}\verb+}{+\meta{p2}\verb+}{+\meta{p3}\verb+}{+\meta{p4}\verb+}{+\meta{p5}\verb+}{+\meta{p6}\verb+}+\ixcom{schemareg} produces the boilerplate concerning
schema registration. The command takes six parameters.
\begin{enumerate}
\item \meta{version no} The version number;
\item \meta{p2} The name of an \Express{} schema (with underscores);
\item \meta{p3} The number of the schema object (typically 1);
\item \meta{p4} The name of the schema, with hyphens replacing any
                  underscores in the name;
\item \meta{p5} The number identifying the schema;
\item \meta{p6} The clause or annex in which the schema is defined.
\end{enumerate}

\begin{example}The command \\
 \verb|\schemareg{1}{a\_schema}{3}{a-schema}{5}{clause 6}|
prints:

\schemareg{\textit{1}}{a\_schema}{\textit{3}}{\textit{a-schema}}{\textit{5}}{\textit{clause 6}}
\end{example} % end example


\ssclause{Computer interpretable listings} 

    The command \verb|\listingshead|\ixcom{listingshead} prints the
heading for the computer interpretable listings annex.

    The command 
\verb|\expurls{|\meta{short}\verb|}{|\meta{express}\verb|}|\ixcom{expurls}
produces the boilerplate for the introduction to the annex 
listing short names and \Express, where \meta{short} is the URL for the short
names and \meta{express} is the URL for the \Express.

\begin{anexample} The command 
  \verb|\expurls{http:/www.short/}{http://www.express/}| prints:

\expurls{http://www.short/}{http://www.express/}

\end{anexample}

\clearpage
\clause{The \file{ir} package facility} 

    The \file{ir}\ixpack{ir} package provides commands and environments
specifically for the ISO~10303 Integrated Resources series of documents.

    Use of this package requires the use of the \file{step}\ixpack{step} 
package.

\sclause{Boilerplate commands}

    The \file{ir} package modifies the \verb|\fwdshortlist|\ixcom{fwdshortlist}
command to produce extra IR-specific boilerplate.

    The following commands produce boilerplate text as specified by the 
SD\index{SD}.


\ssclause{Integrated resource EXPRESS-G} 

    The command \verb|\expressghead|\ixcom{expressghead} prints the
heading for the \ExpressG{} diagrams annex.

    The command \verb+\irexpressg+\ixcom{irexpressg} 
produces the boilerplate for the introduction to the integrated 
resource \ExpressG{} annex.
\index{expressg@\ExpressG\}

\begin{anexample}The command \verb|\irexpressg| prints:

\irexpressg

 \end{anexample} % end example

%%%%%%%%%%%%%%%%%%%%%%%%%%%%%%
%%%%\end{document}
%%%%%%%%%%%%%%%%%%%%%%%%%%%%%%



\clearpage
\clause{The \file{ap} package facility}

    The \file{ap}\ixpack{ap} package provides commands and environments
specifically for the ISO~10303 Application Protocol series of documents.

    Use of this package requires the use of the \file{step}\ixpack{step} 
package.

\sclause{Preamble commands}

    Certain commands shall be put in the preamble of an AP document.

    The command 
\verb+\aptitle{+\meta{title of AP}\verb+}+\ixcom{aptitle}
shall be put into the preamble. \index{preamble} The parameter shall be of 
such a form that
it will read naturally in a sentence of the form: 
`\ldots for the \meta{title of AP} application protocol.'.

\begin{anexample}
  For the purposes of later examples, the command
\verb|\aptitle{|\texttt{\theap}\verb|}| has been put in the preamble
of this document.
\end{anexample} % end example

    If the AP makes use of one or more
AICs\index{AIC}, then the command \verb|\aicinaptrue|\ixcom{aicinaptrue} 
shall be put in the document preamble.

   If a mapping specification is used instead of a mapping table,
the command \verb|\mapspectrue|\ixcom{mapspectrue} shall be put
in the preamble. If mapping templates are used then 
\verb|\maptemplatetrue|\ixcom{maptemplatetrue} shall also be put in the
preamble.

    If IDEF1X is used instead of \ExpressG{} as the graphical form for the
ARM, then \verb|\idefixtrue|\ixcom{idefixtrue} shall be put in the preamble.

  

\sclause{Heading commands}

    These commands start document clauses with particular titles. The
commands that take no parameters are listed in \tref{tab:apnpheads}.
Some of these commands have predefined labels, which are also listed in 
the table.
\ixcom{inforeqhead}
\ixcom{uofhead}
\ixcom{applobjhead}
\ixcom{applasserthead}
\ixcom{aimhead}
\ixcom{maptablehead}
\ixcom{templateshead}
\ixcom{aimshortexphead}
\ixcom{confreqhead}
\ixcom{aimlongexphead}
\ixcom{aimshortnameshead}
\ixcom{impreqhead}
\ixcom{aamhead}
\ixcom{aamdefhead}
\ixcom{aamfighead}
\ixcom{armhead}
\ixcom{aimexpressghead}
\ixcom{aimexpresshead}
\ixcom{apusagehead}

\settowidth{\prwlen}{\quad Application activity model definitions}
\begin{table}
\centering
\caption{AP package parameterless heading commands}
\label{tab:apnpheads}
\begin{tabular}{|l|c|p{\prwlen}|l|} \hline
\textbf{Command} & \textbf{Clause} & \textbf{Default text} & \textbf{Label} \\ \hline
\verb|\inforeqhead| & C & Information requirements & \verb|;sireq| \\
\verb|\uofhead| & SC & Units of functionality  & \verb|;suof| \\
\verb|\applobjhead| & SC & Application objects  & \verb|;sao| \\
\verb|\applasserthead| & SC & Application assertions  & \verb|;saa| \\
\verb|\aimhead| & C & Application interpreted model & \verb|;saim| \\
\verb|\mappinghead| & SC & Mapping table, or & \verb|;smap| \\
                    &    & Mapping specification  & \verb|;smap| \\
\verb|\templateshead| & SSC & Mapping templates &    \\
\verb|\aimshortexphead| & SC & AIM \Express{} short listing & \verb|;saesl| \\
\verb|\confreqheadhead| & C & Conformance requirements & \verb|;scr| \\
\verb|\aimlongexphead| & NA & AIM \Express{} expanded listing  & \verb|;saeel| \\
\verb|\aimshortnameshead| & NA & AIM short names  & \verb|;sasn| \\
\verb|\impreqhead| & NA & Implementation method specific requirements & \verb|;simreq| \\
\verb|\aamhead| & IA & Application activity model & \verb|;saam| \\
\verb|\aamdefhead| & SC & Application activity model definitions and abbreviations & \verb|| \\
\verb|\aamfighead| & SC & Application activity model diagrams & \verb|| \\
\verb|\armhead| & IA & Application reference model & \verb|;sarm| \\
\verb|\aimexpressghead| & IA & AIM \ExpressG{} & \verb|;saeg| \\
\verb|\aimexpresshead| & IA & AIM \Express{} listing & \verb|| \\
\verb|\apusagehead| & IA & Application protocol usage guide & \verb|;sapug| \\
 \hline
\end{tabular}
\end{table}

    The commands listed in \tref{tab:appheads} take parameters.
\ixcom{auofhead}
\ixcom{mapuofhead}
\ixcom{mapobjecthead}
\ixcom{mapattributehead}

\begin{table}
\centering
\caption{AP package parameterized heading commands}
\label{tab:appheads}
\begin{tabular}{|l|c|l|} \hline
\textbf{Command} & \textbf{Clause} & \textbf{Parameterized title} \\ \hline
\verb|\auofhead| & SSC & \meta{UoF} \\ 
\verb|\mapuofhead| & SSC & \meta{UoF} \\
\verb|\mapobjecthead| & SSSC & \meta{application object} \\
\verb|\mapattribhead| & SSSSC & \meta{attribute} \\
\hline
\end{tabular}
\end{table}

\sclause{Boilerplate commands}

    The following commands produce boilerplate text as specified by the 
SD\index{SD}.

\begin{anote}
 In the examples, the parameters of those commands that
take them have been specified in 
\textit{this font style} so their effects can
be seen in the resulting printed text.
\end{anote}

\ssclause{AP introduction}

    The command \verb|\apextraintro|\ixcom{apextraintro} produces extra
boilerplate for the Introduction to an AP.

\begin{anexample}The command \verb|\apextraintro| prints:

\apextraintro
\end{anexample} %end example

\ssclause{AP scope}

    The command \verb+\apscope{+\meta{application purpose and context}\verb+}+\ixcom{apscope} 
produces the boilerplate for the start of an AP scope\index{scope} clause.

\begin{anexample}The command \verb|\apscope{application purpose and context.}|
         prints:

\apscope{\textit{application purpose and context.}} 
\end{anexample} 

\ssclause{AP information requirements}

  The command \verb|\inforeqhead|\ixcom{inforeqhead} prints the
heading for the information requirements clause.

  The command \verb+\apinforeq{+\meta{AP purpose}\verb+}+\ixcom{apinforeq} 
produces the boilerplate for the clause.

\begin{anexample}The command \verb|\apinforeq{AP purpose.}| prints: 

\apinforeq{\textit{AP purpose.}} 
\end{anexample} % end example

\ssclause{AP UoF}

    The command \verb|\uofhead|\ixcom{uofhead} prints the heading
for the UoF subclause.

    The environment 
\verb+\begin{apuof}+\meta{item list}\verb+\end{apuof}+\ixenv{apuof} 
produces the boilerplate for the introduction to the clause.

\begin{anexample} Remembering that \verb|\aptitle|\ixcom{aptitle}
                  was set to \texttt{\theap} in the preamble,
                  the commands
\begin{verbatim}
\begin{apuof}
\item Name of UoF1;
\item Name of UoF2;
\item Name of UoFn.
\end{apuof}
\end{verbatim}
prints:

\begin{apuof}
\item Name of UoF1;
\item Name of UoF2;
\item Name of UoFn.
\end{apuof}

\end{anexample}

\ssclause{AP application objects}

    The command \verb|\applobjhead|\ixcom{applobjhead} prints the
heading for the application objects subclause.

    The command \verb|\apapplobj|\ixcom{apapplobj} produces the 
boilerplate for the introduction to the clause.

\begin{anexample} Remembering that \verb|\aptitle|\ixcom{aptitle}
                  was set to \texttt{\theap} in the preamble,
                  the command \verb|\apapplobj| prints:

\apapplobj

\end{anexample}

\ssclause{AP assertions}

    The command \verb|\applasserthead|\ixcom{applasserthead} prints the
heading for the application assertions subclause.

    The command \verb|\apassert|\ixcom{apassert}
produces the boilerplate for the clause.

\begin{anexample} Remembering that \verb|\aptitle|\ixcom{aptitle}
                  was set to \texttt{\theap} in the preamble,
                  the command \verb|\apassert| prints:

\apassert

\end{anexample}


\ssclause{AP mapping table/specification}

    The command \verb|\mappinghead|\ixcom{mappinghead} prints
the heading for the mapping table or mapping specification subclause.
The heading text depends on whether or not 
\verb|\mapspectrue|\ixcom{mapspectrue} was put in the preamble.

    The command \verb|\apmapping|\ixcom{apmapping} 
produces the boilerplate for the introduction to the AP mapping table
or specification clause.

\begin{anote}AICs are included in the boilerplate only if the command
\verb|\aicinaptrue|\ixcom{aicinaptrue} is included
in the preamble.
\end{anote}

\begin{example}By default, or when \verb|\mapspecfalse| is in
the preamble, the command \verb|\apmapping|
         prints: \mapspecfalse

\apmapping
\end{example} % end example

\begin{example}When \verb|\mapspectrue| is in the preamble, the command \verb|\apmapping|
         prints: \mapspectrue

\apmapping
\end{example} % end example

\sssclause{AP mapping templates}

    The command \verb|\aptemplatehead|\ixcom{aptemplatehead} prints
the heading for the mapping template subclause (if any).

    The command \verb|\apmaptemplate|\ixcom{apmaptemplate} prints
the boilerplate for the introduction to the clause. This refers to the
UoFs in the AP. The first of the UoFs shall be labelled as 
\verb|\label{;uof1}| and the last of the UoFs shall be
labelled as \verb|\label{;uoflast}|.

\begin{example} If there are three UoFs, then there should be headings
of the form:
\begin{verbatim}
\mapuofhead{First UoF}\label{;uof1}
...
\mapuofhead{Second UoF}...
...
\mapuofhead{Third UoF}\label{;uoflast}
...
\end{verbatim}
\end{example}

\begin{example} Assuming that there are three UoFs as in the previous example, 
the command \verb|\apmaptemplate| prints:

\apmaptemplate
\end{example} % end example

    The command \verb|\sstemplates|\ixcom{sstemplates} prints
the two subclauses for the \xword{subtype} and \xword{SuPeRtype} templates.

\begin{example} \label{ex:sstemplates} In this document, 
and noting that the clause
numbering is not the same as in a real AP document, 
the command \verb|\sstemplates|
         prints:

\sstemplates

\end{example} % end example


\sssclause{Template headings}

    There are three headings used within a mapping template.

    The command \verb|\signature|\ixcom{signature} prints the underlined
Mapping signature header.

    The command \verb|\parameters|\ixcom{parameters} prints the underlined
Parameter definition header.

    The command \verb|\body|\ixcom{body} prints the underlined
Template body header.

\begin{anexample} The results of using the \verb|\signature|\ixcom{signature}
and \verb|\parameters|\ixcom{parameters} commands were illustrated
in \eref{ex:sstemplates} on \pref{ex:sstemplates}.
\end{anexample}



\ssclause{AIM short EXPRESS listing}

    The command \verb|\aimshortexphead|\ixcom{aimshortexphead} prints
the heading for the AIM EXPRESS short listing subclause.

    The command \verb|\apshortexpress|\ixcom{apshortexpress} produces 
the boilerplate for the
first paragraph of the clause.

\begin{anote}AICs are included in the boilerplate only if the command
\verb|\aicinaptrue|\ixcom{aicinaptrue} is included in the preamble.
\end{anote}

\begin{example}
The command \verb|\apshortexpress| without \verb|\aicinaptrue|
in the preamble produces:

\aicinapfalse
\apshortexpress

\end{example} % end example

\begin{example}
With \verb|\aicinaptrue| set in the preamble the command
\verb|\apshortexpress| produces the following:

\aicinaptrue
\apshortexpress
\end{example} % end example


\ssclause{AP conformance}

    The command \verb|\confreqhead|\ixcom{confreqhead} prints the
heading for the conformance requirements clause.

    The command 
\verb+\apconformance{+\meta{implementation methods}\verb+}+\ixcom{apconformance} 
produces the boilerplate for the introduction to the clause.

    The environment 
\verb+\begin{apconformclasses}+\meta{item list}\verb+\end{apconformclasses}+\ixenv{apconformclasses} 
provides some additional boilerplate.

\begin{example}The command \verb|\apconformance{ISO 10303-21, ISO 10303-22}|
         prints:

\apconformance{\textit{ISO 10303-21, ISO 10303-22}} 
\end{example} % end example

\begin{example}The commands
  \begin{verbatim}
\begin{apconformclasses}
\item first class;
\item second class;
\item last class.
\end{apconformclasses}
\end{verbatim}
         print:

\begin{apconformclasses}
\item first class;
\item second class;
\item last class.
\end{apconformclasses}
\end{example}


\ssclause{EXPRESS expanded listing}

    The command \verb|\aimlongexphead|\ixcom{aimlongexphead} prints
the heading for the AIM expanded listing clause.

    The command \verb|\aimlongexp|\ixcom{aimlongexp} 
produces the boilerplate for the introduction to the clause.

\begin{anexample}The command \verb|\aimlongexp|
         prints:

\aimlongexp 
\end{anexample} % end example

\ssclause{AIM short names}

    The command \verb|\aimshortnamehead|\ixcom{aimshortnamehead} prints
the heading for the AIM short names annex.

    The command \verb|\apshortnames|\ixcom{apshortnames} 
produces the boilerplate for the introduction to the AP short name annex.

\begin{anexample}The command \verb|\apshortnames|
         prints:

\apshortnames 
\end{anexample} % end example

\ssclause{Implementation requirements}

    the command \verb|\impreqhead|\ixcom{impreqhead} prints the heading
for implementation method-specific reguirements.

    The command \verb+\apimpreq{+\meta{schema name}\verb+}+\ixcom{apimpreq}
produces the boilerplate for the requirements on exchange structure.

\begin{anexample}The command \verb|\apimpreq{schema\_name}|
         prints:

\apimpreq{\textit{schema\_name}} 
\end{anexample} % end example


\ssclause{AP PICS}

    The command \verb|\picshead|\ixcom{picshead}, 
from the \file{step}\ixpack{step} package,
prints the heading for the PICS annex.

    The command \verb|\picsannex|\ixcom{picsannex}
produces the boilerplate for the start of the AP PICS annex.

\begin{anexample}The command \verb|\picsannex|
         prints:

\picsannex 
\end{anexample} % end example

\ssclause{AAM annex}

    The command \verb|\aamhead|\ixcom{aamhead} prints the heading for
the AAM annex.


    The command \verb|\apaamintro|\ixcom{apaamintro} 
 produces the introductory boilerplate for the introduction of
the AP annex on application activity models.

\begin{anexample}
  The command \verb|\apaamintro| prints:

\apaamintro

\end{anexample} % end example

\ssclause{AP AAM definitions}

    The command \verb|\aamdefhead|\ixcom{aamdefhead} prints the heading
for the AAM definitions subclause.

    The command \verb|\apaamdefs|\ixcom{apaamdefs} produces 
the boilerplate at the start of
the AP subclause on AAM definitions and abbreviations.

\begin{anexample}
  The command \verb|\apaamdefs| prints:

\apaamdefs
\end{anexample} % end example

\ssclause{AAM diagrams annex}

    The command \verb|\aamfighead|\ixcom{aamfighead} prints the heading
for the AAM diagrams subclause.

    The command 
\verb|\aamfigrange{|\meta{figure range}\verb|}|\ixcom{aamfigrange} 
is used to store the activity model diagram figure range for later use.

\begin{example}
    For the purposes of this document we set
\begin{verbatim}
\aamfigrange{figures F.1 through F.n}
\end{verbatim}

\aamfigrange{\textit{figures F.1 through F.n}}

\end{example}

    The command \verb+\aamfigures+\ixcom{aamfigures}
produces the boilerplate for the introduction to an APs AAM figure
subclause.

\begin{example} Noting that we have set 
\verb|\aamfigrange{figures F.1 through F.n}|\ixcom{aamfigrange}, 
the command \verb|\aamfigures| prints:

\aamfigures

\end{example}

\ssclause{ARM annex}

    The command \verb|\armhead|\ixcom{armhead} prints the heading for the
ARM annex.

    The command 
\verb+\armintro+\ixcom{armintro}
produces the boilerplate for the introduction to the ARM figures.

\begin{anexample}The command 
          \verb|\armintro| 
         prints:

\armintro 
\end{anexample} % end example

\ssclause{AIM EXPRESS-G annex}

    The command \verb|\aimexpressghead|\ixcom{aimexpressghead} 
prints the heading for the AIM \ExpressG{} annex.
 

    The command 
\verb+\aimexpressg+\ixcom{aimexpressg}
produces the boilerplate for the introduction to an AP's AIM \ExpressG{}
model. 

\begin{anexample}The command \verb|\aimexpressg|
         prints:

\aimexpressg
\end{anexample} % end example

\ssclause{AIM EXPRESS listing}

    The command \verb|\aimexpresshead|\ixcom{aimexpresshead} prints
the heading for the AIM listing annex.

%    The command \verb|\aimexplisting|\ixcom{aimexplisting}
%produces the boilerplate for the introduction to an AIMs short name and
%\Express{} listing.
%
%
%\begin{example}The command \verb|\aimexplisting|
%         prints:
%
%\aimexplisting 
%\end{example}

    The command 
\verb|\apexpurls{|\meta{short}\verb|}{|\meta{express}\verb|]|\ixcom{apexpurls}
produces the boilerplate for the introduction to the AP annex
listing short names and \Express, where \meta{short} is the URL for the short
names and \meta{express} is the URL for the \Express.

\begin{anexample} The command \verb|\apexpurls{http:/www.short/}{http://www.express/}|
prints:

\apexpurls{http://www.short/}{http://www.express/}

\end{anexample}

%%%%%%%%%%%%%%%%%%%%%%%%%%%%%%
%%%%%%\end{document}
%%%%%%%%%%%%%%%%%%%%%%%%%%%%%%



\clearpage
\clause{The \file{aic} package facility}

    The \file{aic}\ixpack{aic} package
provides commands and environments specifically
for the ISO~10303 Application Interpreted Construct series of
documents.

    The use of this package requires the use of the 
\file{step}\ixpack{step} package.

\sclause{Heading commands}

    The commands described in this subclause start document clauses with
particular titles.

    The commands that take no parameters are listed in \tref{tab:aicnpheads}.
\ixcom{aicshortexphead}

\begin{table}[btp]
\centering
\caption{AIC package parameterless heading commands}
\label{tab:aicnpheads}
\begin{tabular}{|l|c|l|l|} \hline
\textbf{Command} & \textbf{Clause} & \textbf{Default text} & \textbf{Label} \\ \hline
\verb|\aicshortexphead| & C & \Express{} short listing & \verb|;sesl| \\
\hline
\end{tabular}
\end{table}

\sclause{Boilerplate commands}

    The following commands produce boilerplate text as specified by the
Supplementary Directives. 


\ssclause{Introduction text}

    The command \verb|\aicextraintro|\ixcom{aicextraintro}
prints additional boilerplate for the Introduction to an AIC.

\begin{anexample}The command \verb|\aicextraintro|
         prints:

\aicextraintro
\end{anexample}

\ssclause{Definition of AIC}

    The command \verb|\aicdef|\ixcom{aicdef}
prints the definition of `AIC'. It shall only be used within the
\verb|definitions|\ixenv{definitions} environment.

\begin{anexample}The commands:
         \begin{verbatim}
         \begin{definitions}
         \aicdef
         \end{definitions}
         \end{verbatim}
         produce:

\begin{definitions}
\aicdef
\end{definitions}
\end{anexample} % end example

\ssclause{Short EXPRESS listing}

    The command \verb|\aicshortexphead|\ixcom{aicshortexphead} prints
the heading for the AIC short \Express{} annex.

    The command \verb|\aicshortexpintro|\ixcom{aicshortexpintro}
prints boilerplate for the introduction to the short \Express{} listing.

\begin{anexample}The command \verb|\aicshortexpintro|
         prints:

\aicshortexpintro  
\end{anexample} % end example

\ssclause{EXPRESS-G figures}

    The command \verb|\expressghead|\ixcom{expressghead}, 
from the \file{step} package, prints the heading for the \ExpressG{} diagrams
annex.

    The command 
\verb+\aicexpressg+\ixcom{aicexpressg} 
prints boilerplate for the introduction to the \ExpressG\ figures.

\begin{anexample}The command \verb|\aicexpressg|
         prints:

\aicexpressg
\end{anexample}

%%%%%%%%%%%%%%%%%%%%%%%%%
%%%\end{document}
%%%%%%%%%%%%%%%%%%%%%%%%%

\clearpage
\clause{The \file{ats} package facility}

    The \file{ats}\ixpack{ats} package
provides commands and environments specifically
for the ISO~10303 Abstract Test Suite series of
documents.

    The use of this package requires the use of the 
\file{step}\ixpack{step} package.

\sclause{Preamble commands}

    Certain commands shall be put in the preamble\index{preamble} 
of an ATS document.

    The command 
\verb+\APnumber{+\meta{number}\verb+}+\ixcom{APnumber} shall be put 
in the preamble,
where \meta{number} is the ISO 10303 part number of the corresponding AP.

\begin{example}
For the purposes of later examples, the command
\verb+\APnumber{+\texttt{\theAPpartno}\verb+}+ has been put in the preamble.
of this document.
\end{example}

    The command 
\verb+\APtitle{+\meta{title of AP}\verb+}+\ixcom{APtitle} shall be put 
in the preamble,
where \meta{title of AP} is the ISO 10303 part title of the
corresponding AP. This must be given in such a manner that it reads
sensibly in a sentence of the form `\ldots for ISO 10303-299,
application protocol \meta{title of AP}.'

\begin{example}
For the purposes of later examples, the command
\verb+\APtitle{+\texttt{\theAPtitle}\verb+}+ 
has been put in the preamble of this document.
\end{example}

    The command
\verb+\mapspectrue+\ixcom{mapspectrue}
shall be put in the preamble if the AP uses a mapping specification rather
than a mapping table.

\sclause{Heading commands}

    These commands start document clauses with particular titles.
The commands that take no parameters are listed in \tref{tab:atshead}.
\ixcom{purposeshead}
\ixcom{domainpurposehead}
\ixcom{aepurposehead}
\ixcom{apobjhead}
\ixcom{apasserthead}
\ixcom{aimpurposehead}
%%%\ixcom{extrefpurposehead}
\ixcom{implementpurposehead}
%%%\ixcom{rulepurposehead}
\ixcom{otherpurposehead}
\ixcom{gtpvchead}
\ixcom{generalpurposehead}
\ixcom{gvcatchead}
\ixcom{gvcprehead}
\ixcom{gvcposthead}
\ixcom{atchead}
\ixcom{prehead}
\ixcom{posthead}
\ixcom{confclassannexhead}
\ixcom{postipfilehead}
%%%\ixcom{excludepurposehead}
\ixcom{atsusagehead}

\settowidth{\prwlen}{\quad General verdict criteria for all abstract}
\begin{table}
\centering
\caption{ATS package parameterless heading commands} \label{tab:atshead}
\begin{tabular}{|l|c|p{\prwlen}|} \hline
\textbf{Command}             & \textbf{Clause} & \textbf{Default text} \\ \hline
\verb|\purposeshead|         & C   & Test purposes  \\
\verb|\aepurposehead|        & SC  & Application element test purposes \\
\verb|\aimpurposehead|       & SC  & AIM test purposes \\
\verb|\implementpurposehead| & SC  & Implementation method test purposes \\
\verb|\domainpurposehead|    & SC  & Domain test purposes \\
\verb|\otherpurposehead|     & SC  & Other test purposes \\

\verb|\gtpvchead|            & C   & General test purposes and verdict criteria \\
\verb|\generalpurposehead|   & SC  & General test purposes \\
\verb|\gvcatchead|           & SC  & General verdict criteria for all abstract test cases \\
\verb|\gvcprehead|           & SC  & General verdict criteria for preprocessor abstract test cases \\
\verb|\gvcposthead|          & SC  & General verdict criteria for postprocessor abstract test cases \\

\verb|\atchead|              & C   & Abstract test cases \\
\verb|\prehead|              & SSC & Preprocessor \\
\verb|\precoveredhead|       & SSSC & Test purposes covered \\
\verb|\preinputhead|         & SSSC & Input specification \\
\verb|\precriteriahead|      & SSSC & Verdict criteria \\
\verb|\preconstraintshead|   & SSSC & Constraints on values \\
\verb|\preexechead|          & SSSC & Execution sequence \\
\verb|\preextrahead|         & SSSC & Extra details \\


\verb|\posthead|             & SSC & Postprocessor \\
\verb|\postcoveredhead|       & SSSC & Test purposes coverage \\
\verb|\postinputhead|         & SSSC & Input specification \\
\verb|\postcriteriahead|      & SSSC & Verdict criteria \\
\verb|\postexechead|          & SSSC & Execution sequence \\
\verb|\postextrahead|         & SSSC & Extra details \\

\verb|\confclassannexhead|   & NA  & Conformance classes \\
\verb|\postipfilehead|       & NA  & Postprocessor input specification file names \\

\verb|\atsusagehead|         & IA  & Usage scenarios \\

\verb|\apasserthead|         & SSC & Application assertions \\
%%%\verb|\extrefpurposehead|    & SC  & External reference test purposes \\
%%%%\verb|\rulepurposehead|      & SC  & \rulepurposename\  \\
%%%\verb|\excludepurposehead|   & NA  & Excluded test purposes \\ 
\hline
\end{tabular}
\end{table}


    The commands that take a parameter are listed in \tref{tab:atsphead}.
\ixcom{apobjhead}
\ixcom{aimenthead}
\ixcom{atctitlehead}
\ixcom{confclasshead}

\begin{table}
\centering
\caption{ATS package parameterized heading commands} \label{tab:atsphead}
\begin{tabular}{|l|c|l|} \hline
Command               & Clause & Parameterized title \\ \hline
\verb|\apobjhead|     & SSC & \meta{Application object n}  \\
\verb|\aimenthead|    & SSC & \meta{Entity name} \\
\verb|\atctitlehead|  & SC  & \meta{Title} \\
\verb|\confclasshead| & SC  & Conformance class \meta{number}  \\ \hline
\end{tabular}
\end{table}


\sclause{Keyword commands}

    Several keyword (headings) are used in documenting a test case.
\latex{} commands for these keywords are given in \tref{tab:atskey}.
\ixcom{atssummary}
\ixcom{atscovered}
\ixcom{atsinput}
\ixcom{atsconstraints}
\ixcom{atsverdict}
\ixcom{atsexecution}
\ixcom{atsextra}

\begin{table}
\centering
\caption{ATS package keyword commands} \label{tab:atskey}
\begin{tabular}{|l|l|} \hline
Command                & Effect \\ \hline
\verb|\atssummary|     & \atssummary{} \\
\verb|\atscovered|     & \atscovered{} \\
\verb|\atsinput|       & \atsinput{} \\
\verb|\atsconstraints| & \atsconstraints{} \\
\verb|\atsverdict|     & \atsverdict{} \\
\verb|\atsexecution|   & \atsexecution{} \\
\verb|\atsextra|       & \atsextra{} \\ \hline
\end{tabular}
\end{table}

\sclause{Boilerplate commands}

    The following commands produce boilerplate text.

\begin{anote}
 In the examples, the
parameters of those commands that take them have been specified in
\textit{this font style} so that their
effects can be seen in the printed text.
\end{anote}

\ssclause{ATS introduction}

    The command 
\verb|\atsintroendbp|\ixcom{atsintroendbp}
 produces the boilerplate
for the end of the Introduction to an ATS.

\begin{anexample} Remembering that in the preamble 
        \verb|\APnumber|\ixcom{APnumber} was set to \texttt{\theAPpartno} 
        and \verb|\APtitle|\ixcom{APtitle} was set to \texttt{\theAPtitle},
the command \verb|\atsintroendbp| prints:

\atsintroendbp

\end{anexample}


\ssclause{ATS scope}

    The command \verb|\scopeclause|\ixcom{scopeclause}, from the \file{isov2}
class, prints the heading for the Scope clause.

    The command 
\verb|\atsscopebp|\ixcom{atsscopebp}
produces boilerplate for an ATS \textit{Scope}
clause.

\begin{anexample}  Remembering that in the preamble 
        \verb|\APnumber|\ixcom{APnumber} was set to \texttt{\theAPpartno}, 
the command \verb|\atsscopebp| prints:

\atsscopebp

\end{anexample}

\ssclause{Test purpose}

    The command \verb|\purposehead|\ixcom{purposehead} prints the heading
for the test purposes clause.

    The command \verb|\atspurposebp|\ixcom{atspurposebp} 
prints boilerplate for the introduction to the clause.

\begin{anexample}  Remembering that in the preamble 
        \verb|\APnumber|\ixcom{APnumber} was set to \texttt{\theAPpartno}, 
the command \verb|\atspurposebp| prints:

\atspurposebp

\end{anexample}

\ssclause{Application element test purposes}

    The command \verb|\aepurposehead|\ixcom{aepurposehead} prints the
heading for the application element test purposes subclause.

    The command 
\verb|\aetpbp|\ixcom{aetpbp}
prints boilerplate for the clause.

\begin{anexample} Remembering that in the preamble 
        \verb|\APnumber|\ixcom{APnumber} was set to \texttt{\theAPpartno}, 
the command \verb|\aetpbp| prints:

\aetpbp

\end{anexample}

\ssclause{AIM test purposes}

    The command \verb|\aimpurposehead|\ixcom{aimpurposehead} prints the
heading for the AIM test purposes subclause.

    The command 
\verb|\aimtpbp|\ixcom{aimtpbp}
prints boilerplate for the clause.

\begin{anexample} Remembering that in the preamble 
        \verb|\APnumber|\ixcom{APnumber} was set to \texttt{\theAPpartno}, 
the command \verb|\aimtpbp| prints:

\aimtpbp

\end{anexample}

\ssclause{Implementation method test purposes}

    The command \verb|\implementpurposehead|\ixcom{implementpurposehead} prints the
heading for the implementation method test purposes subclause.

    The command 
\verb|\atsimtpbp|\ixcom{atsimtpbp}
prints boilerplate for the clause.

\begin{anexample} Remembering that in the preamble 
        \verb|\APnumber|\ixcom{APnumber} was set to \texttt{\theAPpartno}, 
the command \verb|\atsimtpbp| prints:

\atsimtpbp

\end{anexample}



\ssclause{General test purposes and verdict criteria}

    The command \verb|\gtpvchead|\ixcom{gtpvchead} prints the heading
for the general test purposes and verdict criteria clause.

    The command 
\verb|\atsgtpvcbp|\ixcom{atsgtpvcbp} 
prints boilerplate for the clause


\begin{anexample} The command \verb|\atsgtpvcbp| prints:

\atsgtpvcbp
\end{anexample}

\ssclause{General test purposes}

    The command \verb|\generalpurposehead|\ixcom{generalpurposehead} prints
the heading for the general test purposes subclause.

    The command 
\verb|\gtpbp|\ixcom{gtpbp}
prints boilerplate for the suclause.

\begin{anexample} Remembering that in the preamble 
        \verb|\APnumber|\ixcom{APnumber} was set to \texttt{\theAPpartno}, 
the command \verb|\gtpbp| prints:

\gtpbp

\end{anexample}

\ssclause{General verdict criteria}

    The command \verb|\gvcatchead|\ixcom{gvcatchead} prints the
heading for the general verdict criteria for all cases subclause.

    The command 
\verb|\gvatcbp|\ixcom{gvatcbp}
prints boilerplate for the subclause.

\begin{anexample} Remembering that in the preamble 
        \verb|\APnumber|\ixcom{APnumber} was set to \texttt{\theAPpartno}, 
the command \verb|\gvatcbp| prints:

\gvatcbp

\end{anexample}

\ssclause{General verdict criteria for preprocessor}

    The command \verb|\gvcprehead|\ixcom{gvcprehead} prints the
heading for the general verdict criteria for preprocessor cases subclause.

    The command 
\verb|\gvcprebp|\ixcom{gvcprebp} 
prints boilerplate for the subclause.

\begin{anexample} Remembering that in the preamble 
        \verb|\APnumber|\ixcom{APnumber} was set to \texttt{\theAPpartno}, 
the command \verb|\gvcprebp| prints:

\gvcprebp

\end{anexample}

\ssclause{General verdict criteria for postprocessor}


    The command \verb|\gvcposthead|\ixcom{gvcposthead} prints the
heading for the general verdict criteria for postprocessor cases subclause.

    The command 
\verb|\gvcpostbp|\ixcom{gvcpostbp} 
prints boilerplate for the subclause.

\begin{anexample} Remembering that in the preamble 
        \verb|\APnumber|\ixcom{APnumber} was set to \texttt{\theAPpartno}, 
the command \verb|\gvcpostbp| prints:

\gvcpostbp

\end{anexample}

\ssclause{Abstract test cases}

    The command \verb|\atchead|\ixcom{atchead} prints the heading
for the abstract test cases clause.

    The command 
\verb|\atcbp|\ixcom{atcbp} 
prints the first paragraph of the boilerplate for the clause.

\begin{example} The command \verb|\atcbp| prints:

\atcbp
\end{example}

    The command
\verb|\atcbpii|\ixcom{atcbpii}
prints paragraphs~3 and onwards of the boilerplate.

\begin{example} The command \verb|\atcbpii|
prints:

\atcbpii

\end{example}

\ssclause{Preprocessor}

    The command \verb|\prehead|\ixcom{prehead} prints the title
for the preprocessor subsubclause.

    The command
\verb|\atcpretpc|\ixcom{atcpretpc}
prints boilerplate for the subclause.

\begin{anexample} The command \verb|\atcpretpc| prints:

\atcpretpc
\end{anexample}

\ssclause{Postprocessor}

    The command \verb|\posthead|\ixcom{posthead} prints the title
for the postrocessor subsubclause.

    The command
\verb|\atcposttpc|\ixcom{atcposttpc}
prints boilerplate for the subclause.

\begin{anexample} The command \verb|\atcposttpc| prints:

\atcposttpc
\end{anexample}



\ssclause{Conformance class}

    The command \verb|\confclassannexhead|\ixcom{confclassannexhead}
prints the heading for the conformance classes annex heading.

    The command 
\verb|\atsnoclassesbp|\ixcom{atsnoclassesbp} 
prints the entire boilerplate for the
\textit{Conformance class} annex when the AP has no conformance classes.

\begin{example} Remembering that in the preamble 
        \verb|\APnumber|\ixcom{APnumber} was set to \texttt{\theAPpartno}, 
the command \verb|\atsnoclassesbp| prints:

\atsnoclassesbp

\end{example}

    The command \verb|\confclasshead{|\meta{number}\verb|}|\ixcom{confclasshead}
prints the heading for a conformance class \meta{number} subclause.

    The command 
\verb|\confclassbp{|\meta{number}\verb|}|\ixcom{confclassbp}
prints the
boilerplate for the introduction to a conformance class subclause, where
\meta{number} is the number of the conformance class.

\begin{example} Remembering that in the preamble 
        \verb|\APnumber|\ixcom{APnumber} was set to \texttt{\theAPpartno}, 
the command \verb|\confclassbp{27}| prints:

\confclassbp{\textit{27}}

\end{example}

\ssclause{Postprocessor input specification file names}

    The command \verb|\postipfilehead|\ixcom{postipfilehead} prints
the heading for the postprocessor input file names annex.

    The command 
\verb|\pisfbp{|\meta{12 or 21}\verb|}{|\meta{url}\verb|}{|\meta{ref}\verb|}|\ixcom{pisfbp} 
prints the boilerplate for the annex.

\begin{anexample} The command 
\verb|\pisfbp{12}{http://www.mel.nist.gov/step/parts/parts3456/wd}{\ref{TabB1}}| prints:

\pisfbp{12}{http://www.mel.nist.gov/step/parts/part3456/wd}{\ref{tabB1}}

\end{anexample}

%%%%%%%%%%%%%%%%%%%%%%%%%%%
%%%\end{document}
%%%%%%%%%%%%%%%%%%%%%%%%%%%

\normannex{Additional commands} \label{anx:extraiso}

\sclause{Internal commands}

    The code implementing the various facilities includes many commands
not described in the body of this document. Any command that includes
the commercial at sign (\verb|@|) in its name shall not be used by any author;
the implementer of the package code reserves the right to modify or delete
these at any time without giving any notice.

   Internal commands that have names consisting only of letters may be
used in a document at the author's own risk. These may be changed, but 
if so notification will be given.

\sclause{Boilerplate}

    Much of the boilerplate text is maintained in separate \file{.tex}
files and many of the commands that generate boilerplate merely 
\verb|\input|
the appropriate file.




%%%%%%%%%%%%%%%%%%%%%%%%%%
%%%\end{document}
%%%%%%%%%%%%%%%%%%%%%%%%%%

\normannex{Ordering of LaTeX commands} \label{anx:lord}

    The \latex{} commands to produce an ISO~10303 document are:
\begin{verbatim}
\documentclass[<options>]{isov2}
\usepackage{stepv13}                                % required package
\usepackage{irv12}                                  % for an IR document
\usepackage{apv12}                                  % for an AP document
\usepackage{aicv1}                                  % for an AIC document
\usepackage{atsv11}                                 % for an ATS document
\usepackage[<options>]{<name>}                      % additional packages
\standard{<standard identifier>}
\yearofedition{<year>}
\languageofedition{<parenthesized code letter>}
\partno{<part number>}
\series{<series title>}
\doctitle{<title on cover page>}
\ballotcycle{<number>}
\aptitle{<title of AP>}  % if doc is an AP
\aicinaptrue             % if doc is an AP that uses AICs
\mapspectrue             % if doc is an AP that uses mapping spec.
\APnumber{<number>}      % if doc is an ATS
\APtitle{<title>}        % if doc is an ATS
\mapspectrue             % if doc is an ATS and AP uses mapping spec.
  % other preamble commands
\begin{document}
\STEPcover{< title commands >}
\Foreword                            % start Foreword & ISO boilerplate
  \fwdshortlist                      % STEP boilerplate
\endForeword{<param1>}{<param2>}     % end Foreword & boilerplate
\begin{Introduction}                 % start Introduction & boilerplate
  \aicextraintro             % extra boilerplate for an AIC
  \apextraintro                % extra boilerplate for an AP
  % your text 
\end{Introduction}
\stepparttitle{<Part title>}
\scopeclause                         % Clause 1: Scope clause
  \apscope{<AP purpose>}             % boilerplate if an AP
   % text of scope
\normrefsclause                      % Clause 2: Normative references
  \normrefbp{<document type>}        % boilerplate
  \begin{nreferences}
    % \isref{}{} and/or \disref{}{} list of normative references
  \end{nreferences}
\defclause                           % definitions clause
  \partidefhead                      % defs from Part1 subclause
    % olddefinition list
  \refdefhead{<ISO 10303-NN>}        % defs from Part NN subclause
    % olddefinition list
  \otherdefhead                      % defs in this part
    % definition list
\symabbclause                        % Symbols & abbreviations clause
  % symbol lists
% THE BODY OF THE DOCUMENT
\bibannex                            % optional; the final Bibliography 
  % bibliography listing
% the index
\end{document}
\end{verbatim}


\sclause{Body of a resource document} \index{integrated resource}

    The body of a resource document has the following structure:

\begin{verbatim}
\schemahead{<Schema name>}         % repeat for each schema
  \introsubhead                    % intro subclause
     % text
  \fcandasubhead                   % concepts subclause
     % text
  \typehead{<Schema>}              % if type defs
     \atypehead{<type>}            % type heading     
  \entityhead{<Schema>}{<group>}   % if entity defs
     \anentityhead{<entity>}       % entity heading
  \rulehead{<Schema>}              % if rule defs
     \arulehead{<rule>}            % rule heading
  \functionhead{<Schema>}          % if function defs
     \afunctionhead{<function>}    % function heading
% repeat above for each schema
\shortnamehead                     % Annex A: Short names of entities
  \irshortnames                    % boilerplate
  % list of short names
\objreghead                        % Annex B: Information object registration
  \docidhead                       % Document identification subclause
    \docreg{<param1>}                                   % boilerplate
  \schemaidhead                    % Schema identification subclause
% Either (for single schema)
     \schemareg{<6 parameters>}    % boilerplate
% Or (for multiple schemas) repeat:
     \aschemaidhead{<schema name>} % Schema id subsubclause
       \schemareg{<6 parameters>}
\listingshead                       % Annex C: Computer interpretable listings
  \expurls{<short>}{<express>}      % boilerplate
\expressghead                       % Annex D: EXPRESS-G figures
  \irexpressg                       % boilerplate
  %  EXPRESS-G diagrams
\techdischead                       % optional Technical discussions
  % text
\exampleshead                       % optional Examples
  % text
\end{verbatim}


\sclause{Body of an application protocol} \index{AP}

    The body of an AP document has the following structure:

\begin{verbatim}
\inforeqhead                   % Clause 4: Information requirements
  \apinforeq{<param1>}         % boilerplate
  \uofhead                     % Clause 4.1: Units of functionality
    \begin{apuof}              % boilerplate
      % \item list of UoFs
    \end{apuof}
    \auofhead{<UoF1>}          % repeat for each UoF
      % text
    \applobjhead               % Clause 4.2: Application objects
      \apapplobj               % boilerplate
        % text
    \applasserthead            % Clause 4.3: Application assertions
      \apassert                % boilerplate
        % text
\aimhead                       % Clause 5: Application interpreted model
  \maptablehead                % Clause 5.1: Mapping table/specification
    \apmapping                 % boilerplate
    \maptemplatehead           % if mapping templates used
      \apmaptemplate           % template boilerplate
      \sstemplates             % sup/sub templates
      \templatehead
        % text
     \mapuofhead{<Uof>}        % mapping for <UoF>
       \mapobjecthead{<object>}
         % mapping for <object>
         \mapattributehead{<attr>}
           % mapping for <attr>
  \aimshortexphead             % Clause 5.2: AIM EXPRESS short listing
    \apshortexpress            % boilerplate
      % text
\confreqhead                   % Clause 6: Conformance requirements
  \apconformance{<param1>}     % boilerplate
  \begin{apconformclasses}     % optional boilerplate
    % \item list
  \end{apconformclasses}
     % text
\aimlongexphead                % Annex A: AIM EXPRESS expanded listing
  \aimlongexp                  % boilerplate
     % text
\aimshortnameshead             % Annex B: AIM short names
  \apshortnames                % boilerplate
     % text
\impreqhead                    % Annex C: Impl. specific reqs
  \apimpreq{<schema name>}     % boilerplate
\picshead                      % Annex D: PICS
  \picsannex                   % boilerplate
     % text
\objreghead                    % Annex E: Information object registration
  \docidhead                   % Annex E.1: Document identification
    \docreg{<param1>}          % boilerplate
  \schemaidhead                % Annex E.2: Schema identification
    \apschemareg{<6 params>}   % boilerplate
\aamhead                       % Annex F: Application activity model
  \aamfigrange{<figure range>} % Figure range for AAM diagrams
  \apaamintro                  % boilerplate
      % text
  \aamdefhead                  % Annex F.1: AAM defs and abbreviations
    \apaamdefs                 % boilerplate
       % text
  \aamfighead                  % Annex F.2: AAM diagrams
    \aamfigures                % boilerplate
       % IDEF0 diagrams
\armhead                       % Annex G: Application reference model
   \armintro                   % boilerplate
     % ARM figures
\aimexpressghead               % Annex H: AIM EXPRESS-G
  \aimexpressg                 % boilerplate
     % AIM figures
\listingshead                  % Annex J: Computer interpretable listings
  \apexpurls{<short>}{<express>}   % boilerplate
\apusagehead                   % optional Annex: AP usage
   % text
\techdischead                  % optional Annex: Technical discussions
   % text
\end{verbatim}

\sclause{Body of an AIC} \index{AIC}

    The body of an AIC document has the following structure:

\begin{verbatim}
\aicshortexphead               % Clause 4: EXPRESS short listing
  \aicshortexpintro            % boilerplate
  \fcandasubhead               % Clause 4.1 fundamental concepts
    % text
  \typehead{<Schema>}          % if type definitions
     \atypehead{<type>}        % repeat for each type
  \entityhead{<Schema>}{}      % if entity defs
     \anentityhead{<entity>}   % repeat for each entity
  \functionhead{<Schema>}      % if function defs
     \afunctionhead{<function>} % repeat for each function
\shortnamehead                 % Annex A: Short names of entities
  \shortnames                  % boilerplate
\objreghead                    % Annex B: Information object registration
  \docidhead                   % Annex B.1: Document identification
    \docreg{<version no>}      % boilerplate
  \schemaidhead                % Annex B.2: Schema identification
    \schemareg{<6 parameters>} % boilerplate
\expressghead               % Annex C: EXPRESS-G diagrams
  \aicexpressg               % boilerplate
\listingshead                  % Annex D: Computer interpretable listings
  \expurls                     % boilerplate
\techdischead                  % optional Annex: Technical discussions
\end{verbatim}

\sclause{Body of an ATS document}\index{ATS}

    The body of an Abstract Test Suite
document has the following structure:

\begin{verbatim}
\purposeshead                  % Clause 4: Test purposes
  \atspurposebp    % boilerplate
  \aepurposehead               % 4.1 Application element test purposes
    \aetpbp                    % boilerplate
    \apobjhead{<object>}       % 4.1.n
    ...
  \aimpurposehead              % 4.2 AIM test purposes
    \aimtpbp                   % boilerplate
    \aimenthead{<entity>}      % 4.2.n
    ...
  \implementpurposehead        % (optional) 4.3 Implementation t.p
    \atsimtpbp                 % boilerplate
    % text
  \domainpurposehead           % (optional) 4.2+ Domain test purposes
    % text
  \otherpurposehead            % (optional) 4.2+ Other test purposes
    % text
\gtpvchead                     % Clause 5: General t.p and verdict criteria
  \atsgtpvcbp                  % boilerplate
  \generalpurposehead          % 5.1 General test purposes
    \gtpbp                     % boilerplate
    ...
  \gvcatchead                  % 5.2 General verdict criteria for all ATC
    \gvatcbp                   % boilerplate
    ...
  \gvcprehead                  % 5.3 General verdict criteria for preprocessor
    \gvcprebp                  % boilerplate
     ...
  \gvcposthead                 % 5.4 General verdict criteria for postprocessor
    \gvcpostbp                 % boilerplate
    ...
\atchead                       % Clause 6: Abstract test cases
  \atcbp                       % boilerplate (para 1)
  % your para 2
  \atcbpii                     % boilerplate (paras 3+)
  \atctitlehead{<title>}       % 6.n an abstract test case
    \prehead                   % 6.n.1 Preprocessor
      \precoveredhead..        % Test purposes covered
        \atcpretpc             % boilerplate
      \preinputhead            % Input specification
        % text
      \precriteriahead         % Verdict criteria
        % text
      \preconstrainthead       % Constraints on values
        % text
      \preexechead             % (optional) Execution sequence
        % text
      \preextrahead            % (optional) Extra details
        % text
    \posthead                  % 6.n.2 Postprocessor
      \postcoveredhead         % Test purposes covered
        % text
        \atcposttpc            % boilerplate
      \postinputhead           % Input specification
        % text
      \postcriteriahead        % Verdict criteria
        % text
      \postexechead            % (optional) Execution sequence
        % text
      \postextra               % (optional) Extra details
        % text
\confclassannexhead            % Annex A: Conformance classes
  \atsnoclassesbp              % boilerplate if no conformance classes, else
  \confclasshead{<number>}     % A.n Conformance class <number>
    \confclassbp{<number>}     % boilerplate
    % text
  ...
\postipfilehead                % Annex B: Postprocessor input file names
  \pisfbp{..}{..}{..}                      % boilerplate
   ...
\objreghead                    % Annex C: Information object registration
  \docreg{<partno>}            % registration boilerplate
\atsusagehead                  % Annex D: Usage scenarios
  % text
\end{verbatim}



%%%%%%%%%%%%%%%%%%%%%%%%%%%%%%%%%%%%%%%
% object registration annex
\objreghead

\docreg{-1}

%%%%%%%%%%%%%%%%%%%%%%%%%%%%%%%%%%%%%%

%%%%%%%%%%%%%%%%%%%%%%%%%
%%%\end{document}
%%%%%%%%%%%%%%%%%%%%%%%%%

\infannex{Deprecated, deleted, new and modified commands}

    This release has involved many internal changes to the \latex{}
\file{.sty} files. In particular boilerplate text is, as far as possible,
maintained in external \file{.tex} files in order to save memory
space within the \latex{} processor. 


%%%%%%%%%%%%%%%%%%%%%%%%%
%%%\end{document}
%%%%%%%%%%%%%%%%%%%%%%%%%

\sclause{New commands}

    The commands that are new in this release are:

\begin{itemize}
%%%%%%%%%%%%%%%%%%%%%%%%%% STEP %%%%%%%%%%%%%%%%%%%%%%%%%%%
\item \verb|\bibieeeidefo|\ixcom{bibieeeidefo} STEP: reference to IDEF0 document;
\item \verb|\exampleshead|\ixcom{exampleshead} STEP: clause heading;
\item \verb|\expressgdef|\ixcom{expressgdef} STEP: location of \ExpressG{} definition;

\item \verb|\Theseries|\ixcom{Theseries} STEP: print \verb|\series| argument;
\item \verb|\theseries|\ixcom{theseries} STEP: print \verb|\series| argument in lowercase;
\item \verb|\ifanir|,\ixcom{ifanir} 
      \verb|\anirtrue|,\ixcom{anirtrue}
      \verb|\anirfalse|\ixcom{anirfalse} STEP: flag for an IR document;
\item \verb|\ifhaspatents|,\ixcom{ifhaspatents} 
      \verb|\haspatentstrue|,\ixcom{haspatentstrue} 
      \verb|\haspatentsfalse|\ixcom{haspatentsfalse} STEP: flag for known patents;
\item \verb|\ifmapspec|,\ixcom{ifmapspec} 
      \verb|\mapspectrue|,\ixcom{mapspectrue} 
      \verb|\mapspecfalse|\ixcom{mapspecfalse} STEP: flag for mapping specification;

\item \verb|\ixent|\ixcom{ixent} STEP: index an \Express{} \xword{entity};
\item \verb|\ixenum|\ixcom{ixenum} STEP: index an \Express{}  \xword{enumeration};
\item \verb|\ixfun|\ixcom{ixfun} STEP: index an \Express{}  \xword{function};
\item \verb|\ixproc|\ixcom{ixproc} STEP: index an \Express{}  \xword{procedure};
\item \verb|\ixrule|\ixcom{ixrule} STEP: index an \Express{}  \xword{rule};
\item \verb|\ixsc|\ixcom{ixsc} STEP: index an \Express{}  \xword{subtype\_constraint};
\item \verb|\ixschema|\ixcom{ixschema} STEP: index an \Express{}  \xword{schema};
\item \verb|\ixselect|\ixcom{ixselect} STEP: index an \Express{}  \xword{select};
\item \verb|\ixtype|\ixcom{ixtype} STEP: index an \Express{}  \xword{type};

\item \verb|\maptableorspec|\ixcom{maptableorspec} STEP: prints `table' or `specification';

\item \verb|\xword|\ixcom{xword} STEP: prints an \Express{} keyword;

%%%%%%%%%%%%%%%%%%%%%%%%%%%%%%%% AP %%%%%%%%%%%%%%%%%%%%%%%%%%%%%%%


\item \verb|\apmaptemplate|\ixcom{apmaptemplate} AP: boilerplate;
\item \verb|\apusagehead|\ixcom{apusagehead} AP: clause heading;
\item \verb|\ifidefix|,\ixcom{ifidefix} 
      \verb|\idefixtrue|,\ixcom{idefixtrue} 
      \verb|\idefixfalse|\ixcom{idefixfalse} AP: flag for an IDEF1X ARM;
\item \verb|\ifmaptemplate|,\ixcom{ifmaptemplate} 
      \verb|\maptemplatetrue|,\ixcom{maptemplatetrue} 
      \verb|\maptemplatefalse|\ixcom{maptemplatefalse} AP: flag for 
            using mapping templates;
\item \verb|\mapattributehead|\ixcom{mapattributehead} AP: clause heading;
\item \verb|\mapobjecthead|\ixcom{mapobjecthead} AP: clause heading;
\item \verb|\mapuofhead|\ixcom{mapuofhead} AP: clause heading;
\item \verb|\sstemplates|\ixcom{sstemplates} AP: boilerplate;
\item \verb|\templateshead|\ixcom{templateshead} AP: clause heading;
%%% \item \verb|\apmappingspec|\ixcom{} internal (not used?) 

%%%%%%%%%%%%%%%%%%%%%%%%%%%%%%%% ATS %%%%%%%%%%%%%%%%%%%%%%%%%%%%%

\item \verb|\atcposttpc|\ixcom{atcposttpc} ATS: boilerplate;
\item \verb|\atcpretpc|\ixcom{atcpretpc} ATS: boilerplate;
\item \verb|\atsimtpbp|\ixcom{atsimtpbp} ATS: boilerplate;
\item \verb|\atsusagehead|\ixcom{atsusagehead} ATS: clause heading.


\end{itemize}





\sclause{Modified commands}

    The commands that have been modified in this release are:

\begin{itemize}

%%%%%%%%%%%%%%%%%%%%%%%%%%%% STEP %%%%%%%%%%%%%%%%%%%%%%%%%%%%%

\item STEP: The \verb|\Introduction|\ixcom{Introduction} command is
      now the \verb|Introduction|\ixenv{Introduction} environment,
      with no argument;

%%%%%%%%%%%%%%%%%%%%%%%%%%%%%% IR %%%%%%%%%%%%%%%%%%%%%%%%%%%%%%%

\item \verb|\irexpressg|\ixcom{irexpressg} IR: takes no argument;

%%%%%%%%%%%%%%%%%%%%%%%%%%%%%% AP %%%%%%%%%%%%%%%%%%%%%%%%%%%%%%%

\item \verb|\aimexpressg|\ixcom{aimexpressg} AP: takes no argument;

%%%%%%%%%%%%%%%%%%%%%%%%%%%%%% AIC %%%%%%%%%%%%%%%%%%%%%%%%%%%%%%%

\item \verb|\aicexpressg|\ixcom{aicexpressg} AIC: takes no argument;


%%%%%%%%%%%%%%%%%%%%%%%%%%%%%% ATS %%%%%%%%%%%%%%%%%%%%%%%%%%%%%%%

\item \verb|\atcbpii|\ixcom{atcbpii} ATS:  takes no argument;
\item \verb|\atspurposebp|\ixcom{atspurposebp} ATS: takes no argument;
\item \verb|\pisfbp|\ixcom{pisfbp} ATS: takes 3 arguments.


\end{itemize}





\sclause{Deleted commands}

    The commands that have been deleted in this release are:

\begin{itemize}

%%%%%%%%%%%%%%%%%%%%%%%%%%%%%% STEP %%%%%%%%%%%%%%%%%%%%%%%%%%%%%

\item \verb|\fwddivlist|\ixcom{fwddivlist} STEP: used in Foreword;
\item \verb|\fwdpartslist|\ixcom{fwdpartslist} STEP: used in Foreword;

\item \verb|\introend|\ixcom{introend} STEP: was for use at the end of the
      Introduction;

%%%%%%%%%%%%%%%%%%%%%%%%%%%%%% IR %%%%%%%%%%%%%%%%%%%%%%%%%%%%%

\item \verb|\irschemaintro|\ixcom{irschemaintro} IR:
      has been replaced by
      \verb|\schemaintro|\ixcom{schemaintro};

%%%%%%%%%%%%%%%%%%%%%%%%%%%%%% AP %%%%%%%%%%%%%%%%%%%%%%%%%%%%%

\item \verb|\apintroend|\ixcom{apintroend} AP:
      has been replaced by
      \verb|\apextraintro|\ixcom{apextraintro};

\item \verb|\apschemareg|\ixcom{apschemareg} AP: use 
           \verb|\schemareg|\ixcom{schemareg} instead;

\item \verb|\apmappingtable|\ixcom{apmappingtable} AP:
      has been replaced by
      \verb|\apmapping|\ixcom{apmapping};

\item \verb|\armfigures|\ixcom{armfigures} AP:
      has been replaced by
      \verb|\armintro|\ixcom{armintro};

\item \verb|\maptablehead|\ixcom{maptablehead} AP:
      has been replaced by
      \verb|\mappinghead|\ixcom{mappinghead};

\item \verb|\modelscopehead|\ixcom{modelscopehead} AP: was the heading
      for a `Model scope' annex;

%%%%%%%%%%%%%%%%%%%%%%%%%%%%%% AIC %%%%%%%%%%%%%%%%%%%%%%%%%%%%%

\item \verb|\aicexpressghead|\ixcom{aicexpressghead} AIC: 
      use \verb|\expressghead|\ixcom{expressghead} instead;
\item \verb|\aicshortnames|\ixcom{aicshortnames} AIC: use
      \verb|\expurls|\ixcom{expurls} instead;
\item \verb|\aicshortnameshead|\ixcom{aicshortnameshead} AIC: 
      use \verb|\shortnamehead|\ixcom{shortnamehead} instead;

%%%%%%%%%%%%%%%%%%%%%%%%%%%%%% ATS %%%%%%%%%%%%%%%%%%%%%%%%%%%%%


\item \verb|\excludepurposehead|\ixcom{excludepurposehead} ATS: 
      was the heading for an `Exclude purposes' clause.

\end{itemize}


%%%%%%%%%%%%%%%%%%%%%%%
%%%\end{document}
%%%%%%%%%%%%%%%%%%%%%%%


% sgmlannx.tex    latex and SGML

\infannex{LaTeX, the Web, and *ML} \label{anx:sgml} \index{SGML}

    ISO are becoming more interested in electronic sources for their
standards as well as the traditional camera-ready copy. Acronyms like
PDF, HTML, SGML and XML have been bandied about. Fortunately documents
written using \latex{} are well placed to be provided in a variety of 
electronic formats. A comprehensive treatment of \latex{} with respect
to this topic is provided by Goossens and Rahtz~\bref{lwebcom}.

    SGML (Standard Generalized Markup Language) is a document tagging 
language that is described in ISO~8879~\bref{sgml} and whose usage is described 
in~\bref{bryan}, among others. The principal
mover behind SGML is Charles Goldfarb from IBM, who has authored a detailed 
handbook~\bref{goldfarb} on the SGML standard.

    The concepts lying behind both \latex{} and SGML are similar, but on the face
of it they are distinctly different in both syntax and capabilities. ISO is
migrating towards electronic versions of its standard documents and, naturally, 
would prefer these to be SGML tagged. 
     Like \latex, SGML has a
concept of style files, which are termed DTDs, and both systems support
powerful macro-like capabilities. SGML provides for logical document
markup and not typesetting --- commercial SGML systems often use
\tex{} or \latex{} as their printing engine, as does the NIST SGML
environment for ISO~10303~\bref{pandl}.



NIST have SGML tagged some STEP documents 
using manual methods, which are time consuming and expensive. 
In about 1997 there was a NIST 
effort underway to develop an auto-tagger that would (semi-) automatically 
convert
a \latex{} tagged document to one with SGML tags. This tool assumed a
fixed set of \latex{} macros and a fixed DTD.
 The design of an auto-tagger
essentially boils down to being able to convert from a source document tagged
according to a \latex{} style file to one which is tagged according to an
SGML DTD.
    Fully automatic conversion is really only possible if the authors'
of the documents to be translated avoid using any `non-standard' macros within
their documents. There is a program called \file{ltx2x}\index{ltx2x} available
from SOLIS, which replaces \latex{} commands within a document with
user-defined text strings~\bref{ltx2x}. This can be used as a basis for
a \latex{} to whatever auto-tagger, provided the \latex{} commands are not
too exotic.

    HTML is a simple markup language, based on SGML, and is used for the
publication of many documents on the Web. XML is a subset of SGML and appears
to being taken up by every man and his dog as \emph{the} document markup
language. HTML is being recast in terms of XML instead of SGML. PDF is a page
description language that is a popular format for display of documents 
on the Web.

    \latex{} documents can be output in PDF by using pdfLaTeX. Instead
of a \file{.dvi} file being produced a \file{.pdf} file is output directly.
The best 
results are obtained when PostScript fonts rather than Knuth's cm fonts 
are used. Noting that the \file{isov2} class provides an \verb|\ifpdf| command,
a general form for documents to be processed by either \latex{} or pdfLaTeX
is
\begin{verbatim}
\documentclass{isov2}
\usepackage{times}     % PostScript fonts Times, Courier, Helvetica
\ifpdf
  \pdfoutput=1         % request PDF output
  \usepackage[pdftex]{graphicx}
\else
  \usepackage{graphicx}
\fi
...
\end{verbatim}

    There are several converters available to transform a \latex{} document 
into an HTML document, but like \file{ltx2x} they generally do their own
parsing of the source file, and unlike \file{ltx2x} are typically limited
to only generating HTML. Eitan Gurari's \file{TeX4ht}\index{TeX4ht} 
suite is a notable
exception (see Chapter~4 and Appendix~B of~\bref{lwebcom}). It uses the 
\file{.dvi} file as input, so that all the parsing is done by \tex, and can be
configured to generate a wide variety of output formats.
A set of \file{TeX4ht} configuration files are available for converting
STEP \latex{} documents into HTML\footnote{Later, configuration files for XML
output will be developed.}.

    It is highly recommended that for the purposes of ISO~10303, document editors
refrain from defining their own \latex{} macros. If new generally applicable
\latex{} commands are found to be necessary, these should be sent to the
editor of this document for incorporation into
the \file{isov2}\ixclass{isov2} class, the \file{step}\ixpack{step}
package and/or appropriate other packages.

    Some other points to watch when writing \latex{} documents that will assist
in translations into *ML are given below. Typically, attention to these points
will make it easier to parse the \latex{} source.

\begin{itemize}
\item Avoid using the \verb|\label|\ixcom{label} command within
      clause headings or captions. It can just as easily be placed immediately
      after these constructs.
\item Avoid using the \verb|\index|\ixcom{index} command within
      clause headings or captions. It can just as easily be placed immediately
      after these constructs.
\item Use all the specified tagging constructs when defining an \Express{} 
      model --- this will also assist any program that attempts to extract
      \Express{} source code and descriptive text from a document.
\end{itemize}



\infannex{Obtaining LaTeX and friends} \label{anx:getstuff}

    \latex{} is a freely available document typesetting system. There are many
public domain additions to the basic system, for example the \file{iso.cls}
and \file{step.sty} styles. The information below gives pointers to where
you can obtain \latex{} etc., from the\index{Internet} Internet. 


    \latex{} runs on a wide variety of hardware, from PCs to Crays. Source to build
a \latex{} system is freely available via anonymous ftp\index{ftp} 
from what is called CTAN\index{CTAN}
(Comprehensive \tex\ Archive Network). 
There are three sites; pick the one nearest to you.
\begin{itemize}
\item \url{ftp.dante.de} CTAN in Germany;
\item \url{ftp.tex.ac.uk} CTAN in the UK;
\item \url{ctan.tug.org} CTAN in the USA;
\end{itemize}
The top level CTAN directory 
for \latex{} and friends is \url{/tex-archive}. CTAN contains a wide variety
of (La)TeX sources, style files, and software tools and scripts to assist in
document processing.

\begin{anote}
CTAN is maintained by the \tex{} Users Group (TUG). Their homepage
\isourl{http://www.tug.org} should be consulted for the current list of CTAN sites and mirrors.
\end{anote}

\begin{comment}

\sclause{SOLIS} \index{SOLIS}

    SOLIS is the \textit{SC4 On Line Information Service}. It contains many electronic
sources of STEP related documents. The relevant top level directory is
\url{pub/subject/sc4}.
 In particular, SOLIS contains the source for this document
and the \file{.sty} files, as well as other \latex{} related files. 
The \latex{} root directory is \url{sc4/editing/latex}.
The latest versions of the \latex{}
related files are kept in the sub-directory \url{latex/current}.
Some \latex{} related programs are also available in the 
\url{latex/programs} sub-directory.

    There are several ways of accessing SOLIS; instructions
are detailed by Ressler~\bref{ressler} and Rinaudot~\bref{rinaudot}. 
Copies of these reports may be obtained by telephoning the
IPO Office at \verb|+1 (301) 975-3983|, although they are probably somewhat
dated by now.
The simplest method is to point your browser at the following URL: \\
\isourl{http://www.nist.gov/sc4}

\end{comment}

\bibannex
\label{biblio}

\begin{references}
\reference{LAMPORT, L.,}{LaTeX --- A Document Preparation System,}
            {Addison-Wesley Publishing Co., 2nd edition, 1994} \label{lamport}
\reference{WILSON, P. R.,}{LaTeX for standards: The LaTeX package files
            user manual,}%
           {NISTIR, 
                   National Institute of Standards and Technology,
           Gaithersburg, MD 20899. June 1996.} \label{doc:isorot}
\reference{GOOSENS, M., MITTELBACH, F. and SAMARIN, A.,}{%
           The LaTeX Companion,}
           {Addison-Wesley Publishing Co., 1994} \label{goosens}
\reference{GOOSENS, M. and RAHTZ, S.,}{%
           The LaTeX Web Companion --- Integrating TeX, HTML, and XML,}
           {Addison-Wesley Publishing Co., 1999} \label{lwebcom}
\reference{CHEN, P. and HARRISON, M.A.,}{Index preparation and
           processing,}{Software--Practice and Experience, 19(9):897--915,
           September 1988.} \label{chen}
%\reference{KOPKA, H. and DALY, P.W.,}{A Guide to LaTeX,}
%           {Addison-Wesley Publishing Co., 1993.} \label{kopka}
\reference{ISO 8879:1986,}{Information processing --- 
                                Text and office systems ---
           Standard Generalized Markup Language (SGML)}{} \label{sgml}
\reference{GOLDFARB, C.F.,}{The SGML Handbook,}
           {Oxford University Press, 1990.} \label{goldfarb}
\reference{BRYAN, M.,}{SGML --- An Author's Guide to the Standard Generalized
           Markup Language,}{Addison-Wesley Publishing Co., 1988. }\label{bryan}
\reference{PHILLIPS, L., and LUBELL, J.,}{An SGML Environment for STEP,}%
           {NISTIR 5515, 
                   National Institute of Standards and Technology,
           Gaithersburg, MD 20899. November 1994.} \label{pandl}
\reference{WILSON, P. R.,}{LTX2X: A LaTeX to X Auto-tagger,}%
           {NISTIR, 
                   National Institute of Standards and Technology,
           Gaithersburg, MD 20899. June 1996.} \label{ltx2x}
\bibidefo
\bibieeeidefix
\begin{comment}
\reference{RESSLER, S.,}{The National PDES Testbed Mail Server User's Guide,}
           {NSTIR 4508, National Institute of Standards and Technology,
           Gaithersburg, MD 20899. January 1991.} \label{ressler}
\reference{RINAUDOT, G. R.,}{STEP On Line Information Service (SOLIS),}
          {NISTIR 5511, National Institute of Standards and Technology,
          Gaithersburg, MD 20899. October 1994. } \label{rinaudot}
\end{comment}
\end{references}

    
%  the INDEX
% stepman.tex   Description of option style files for STEP
\documentclass[wd,copyright,letterpaper]{isov2}
\usepackage{stepv13}
\usepackage{irv12}
\usepackage{apv12}
\usepackage{aicv1}
\usepackage{atsv11}
%%\usepackage{isomods}  % must come after the step packages
\usepackage{hyphenat}
\usepackage{comment}

\ifpdf
  \pdfoutput=1
  \usepackage[plainpages=false,
              pdfpagelabels,
              bookmarksnumbered,
              hyperindex=true
             ]{hyperref}
\fi

% general required preamble commands
\standard{ISO/WD 10303-3456}
\yearofedition{2002}
\languageofedition{(E)}
\renewcommand{\extrahead}{N 47b}  % add doc N number to headers
\partno{3456}
\series{documentation methods}
\doctitle{LaTeX package files for ISO 10303: User manual}
\ballotcycle{2}
% required preamble commands for an AP
\aptitle{implicit drawing}
\aicinaptrue % only if the AP uses AICs
\mapspectrue  % only if AP uses mapping specification
% required preamble commands for an ATS
\APtitle{abstract painting}
\APnumber{299}

\changemarkstrue

\makeindex

\setcounter{tocdepth}{3} % add more levels to table of contents
%
% Rest of preamble is some special macro definitions for this document only
%
\makeatletter
%
%   the \file{} command
%
\newcommand{\file}[1]{\textsf{#1}}
%
%   the \meta{} command
%
\begingroup
\obeyspaces%
\catcode`\^^M\active%
\gdef\meta{\begingroup\obeyspaces\catcode`\^^M\active%
\let^^M\do@space\let \do@space%
\def\-{\egroup\discretionary{-}{}{}\hbox\bgroup\it}%
\m@ta}%
\endgroup
\def\m@ta#1{\leavevmode\hbox\bgroup$<$\it#1\/$>$\egroup
    \endgroup}
\def\do@space{\egroup\space
    \hbox\bgroup\it\futurelet\next\sp@ce}
\def\sp@ce{\ifx\next\do@space\expandafter\sp@@ce\fi}
\def\sp@@ce#1{\futurelet\next\sp@ce}
%
% the \setlabel{id}{num} command
% this is based on the kernel \refstepcounter macro (ltxref.dtx)
%
%%\newcounter{lbl}
\ifpdf
  \newcommand{\setlabel}[2]{%
    \protected@write\@auxout{}{%
      \string\newlabel{#1}{{#2}{\thepage}{setlabel\relax}{label.#2}{}}}%
  }
\else
  \newcommand{\setlabel}[2]{%
    \protected@write\@auxout{}{%
      \string\newlabel{#1}{{#2}{\thepage}}}%
  }
\fi
%
% index a command
\newcommand{\bs}{\symbol{'134}}
\newcommand{\ixcom}[1]{\index{#1/ @{\tt \protect\bs #1}}}
% index an environment
\newcommand{\ixenv}[1]{\index{#1 @{\tt #1} (environment)}}
% index an option
\newcommand{\ixopt}[1]{\index{#1 @{\tt #1} (option)}}
% index a package
\newcommand{\ixpack}[1]{\index{#1 @\file{#1} (package)}}
% index a class
\newcommand{\ixclass}[1]{\index{#1 @\file{#1} (class)}}
% index in typewriter font
\newcommand{\ixtt}[1]{\index{#1@{\tt #1}}}
% index LaTeX
\newcommand{\ixltx}{\index{latex@\LaTeX}}
% index LaTeX 2e
\newcommand{\ixltxe}{\index{latex2e@\LaTeX 2e}}
% index LaTeX v2.09
\newcommand{\ixltxv}{\index{latex209@\LaTeX{} v2.09}}
% index a file
\newcommand{\ixfile}[1]{\index{#1@\file{#1}}}
\makeatother
%
%
%   set some labels
% step
\setlabel{;ssne}{A}
%%%\setlabel{;sior}{B}
%%%\setlabel{;scil}{C}
\setlabel{;seg}{D}
% aic
\setlabel{;sesl}{4}
% ap
\setlabel{;sireq}{4}
\setlabel{;suof}{4.1}
\setlabel{;sao}{4.2}
\setlabel{;saa}{4.3}
\setlabel{;saim}{5}
\setlabel{;smap}{5.1}
\setlabel{;saesl}{5.2}
\setlabel{;scr}{6}
\setlabel{;saeel}{A}
\setlabel{;sasn}{B}
\setlabel{;simreq}{C}
\setlabel{;spics}{D}
\setlabel{;saam}{F}
\setlabel{;sarm}{G}
\setlabel{;saeg}{H}
\setlabel{;scil}{J}
\setlabel{tabB1}{B.1}
\setlabel{;uof1}{5.1.2}
\setlabel{;uoflast}{5.1.4}
%
% define a new length
\newlength{\prwlen}
%
% new (La)TeX macros
\newcommand{\latex}{LaTeX}
\newcommand{\tex}{TeX}
%
%%%%%%%%%% END SPECIAL MACROS
%
%   end of preamble
%
\begin{document}


\STEPcover{
%\scivnumber{987}
\wg{EC}
\docnumber{47b}
\oldwg{EC}
\olddocnumber{47a}
\docdate{2002/09/04}
%\partnumber{3456}
%\doctitle{LaTeX package files for ISO 10303: User manual}
%\status{Working draft}
%\primcont
\abstract{This document describes and illustrates the \latex{} macros
for typesetting ISO~10303. The International Organisation for
Standardisation (ISO) has specified editorial directives for all 
international standards published by them. The \latex{} macros
described here were developed to meet additional editorial directives 
for ISO~10303. } % end abstract
\keywords{\latex, document preparation, typesetting ISO standards}
%\dateprojo{May 1996}
\owner{Peter R. Wilson}
\address{Boeing Commercial Airplane\newline
            PO Box 3701 \newline
            MS 2R-97 \newline
            Seattle, WA 98124-2207 \newline
            USA}
\telephone{+1 (206) 544-0589}
\fax{+1 (206) 544-5889}
\email{\url{peter.r.wilson@boeing.com}}
\altowner{Peter R. Wilson}
\altaddress{Boeing Commercial Airplane \newline
            PO Box 3701 \newline
            MS 2R-97 \newline
            Seattle, WA 98124-2207 \newline
            USA}
\alttelephone{+1 (206) 544-0589}
\altfax{+1 (206) 544-5889}
\altemail{\url{peter.r.wilson@boeing.com}}
\comread{\draftctr This document serves two purposes. Firstly, it provides a description
         of the current \latex{} macros for ISO 10303. Secondly, the source
         can be used as an example of using the \latex{} commands.
         Although the document is written as though it were a
         standard, it is not, and is not intended to become, 
         a standard.} %end comread
} % end of STEPcover

\Foreword

\fwdshortlist
\endForeword%
{Annexes A, B and C are}  % normative annexes
{Annexes D, E and F are} % informative annexes

\begin{Introduction}%%%%%%%%%%{documentation methods}

    This part of ISO 10303 specifies the \latex{} facilities specifically 
designed for use in preparing the various parts of this standard.

\begin{majorsublist}
\item the \file{step} package facility;
\item the \file{ir} package facility;
\item the \file{ap} package facility;
%%%\item the \file{am} package facility;
\item the \file{aic} package facility;
\item the \file{ats} package facility.
\end{majorsublist}

    This part of ISO~10303 is intended to be used in conjunction with
\textit{\latex{} for ISO standards: User manual}
which is based in part upon material in the ISO/IEC Directives,
Part 2 (\textit{Rules for the structure and drafting of International 
Standards, Fourth edition}).
The \latex{} facilities described here are based as well
upon the specifications given in ISO TC184/SC4 N1217n 
(\textit{SC4 Supplementary directives --- Rules for the structure
and drafting of SC4 standards for industrial data}).


\sclause*{Overview}


    This document describes a set of \latex{} macro files for use within
ISO~10303, commonly called STEP (STandard for the Exchange of Product
model data). The electronic source of this  document 
also provides an example of the use of these files.

    The current set of macro files have been developed by 
Peter Wilson (\url{peter.r.wilson@boeing.com}) from a macro file developed
by Kent Reed (NIST) for \latex{} v2.09. In turn, this was a revision of
files originally created by Phil Spiby (CADDETC), based on earlier work 
by Phil Kennicott (GE).\footnote{In mid 1994 \latex{} was upgraded 
from version 2.09 to what is called \latex 2e. The files described in 
this document are only applicable to \latex 2e (support for \latex{} v2.09
was dropped in September 1997).}


\begin{anote} 
It is important to remember that these macro files are only compatible with 
\latex 2e.
\end{anote} % end anote

    Documents produced with the \latex{} files have been twice reviewed 
by the ISO Editorial Board in Geneva for conformance to their 
typographical requirements. The first review was of a set of Draft 
International Standard documents. This review resulted in some changes 
to the style files. The second review was of a set of twelve 
International Standard documents (ISO 10303:1994). Likewise, this
review led to changes in the style files to bring the documents into 
conformance.

    With the issuance of the first STEP release, the opportunity was 
taken to provide a new baseline release of the package files. 
In particular, one STEP specific package file is available for all 
STEP parts, while others contain only commands relevant to the 
documentation of particular series of parts. The range of package 
files may be extended in the future to cater for 
documentation specific to all STEP parts.

   The 1997 baseline release was also designed to cater for the 
fact that a major update of \latex{} to \latex 2e took place during 1994.
\latex 2e is the only officially supported version of \latex.

    Because ISO standard documents have a very structured layout, the 
\file{isov2} class and the package files described here have been 
designed to reflect the logical document structure to a much greater 
extent than the `standard' \latex{} files. 

    With ISO's move toward accepting documents in PDF and HTML, 
the advent of second
editions of some of the STEP parts, and a new edition of the STEP
Supplementary Directives, a 2002
baseline release has been developed and is documented here. 



\end{Introduction}

\stepparttitle{Documentation methods: LaTeX package files for ISO 10303:
User manual}


\scopeclause

This part of ISO~10303 describes a set of \ixltx\latex{} facilities for typesetting
documents according to the ISO/IEC Directives Part 2, together with the 
Supplementary Directives for drafting and presentation of ISO~10303.

\begin{inscope}{part of ISO~10303}
\item use of \latex{} for preparing ISO~10303 documents.
\end{inscope}

\begin{outofscope}{part of ISO~10303}
\item use of \latex{} for preparing ISO standard documents in general;
\item use of \latex{} in general;
\item use of other document preparation systems.
\end{outofscope}

\textbf{IMPORTANT:} The preparation of this document has been partly
funded by the US Government and is not subject to copyright.
Any copyright notices within the document are for illustrative purposes only.

\normrefsclause \label{sec:nrefs}

\normrefbp{part of ISO~10303}
\begin{nreferences}

\isref{ISO/IEC Directives, Part 2}{Rules for the structure and drafting 
           of International Standards, Fourth edition.}

\isref{ISO TC 184/SC4 N1217:2001(E)}{SC4 Supplementary directives --- 
       Rules for the structure and drafting of SC4 standards for 
       industrial data.}

%\isref{ISO 10303-1:1994}{Industrial automation systems and integration ---
%        Product data representation and exchange --- 
%        Part 1: Overview and fundamental principles.}
\nrefparti

%\isref{ISO 10303-11:1994}{Industrial automation systems and integration --- 
%        Product data representation and exchange --- 
%        Part 11: Description methods:
%        The EXPRESS language reference manual.}
\nrefpartxi

%\disref{ISO/TR 10303-12:---}{Industrial automation systems and integration ---
%        Product data representation and exchange ---
%        Part 12: Description methods:
%        The EXPRESS-I language reference manual.}
\nrefpartxii

%\disref{ISO/IEC 8824-1:---}{Information technology ---
%       Open systems interconnection ---
%       Abstract syntax notation one (ASN.1) ---
%       Part 1: Specification of basic notation.}
\nrefasni

\disref{P. R. WILSON:---}{LaTeX for ISO standards: User manual.}

\end{nreferences}

\defabbclause
%\clause{Terms, definitions, and abbreviations}

\partidefhead
%\sclause{Terms defined in ISO 10303-1}

    This part of ISO~10303 makes use of the following terms defined in 
ISO~10303-1:

\begin{olddefinitions}
\olddefinition{application protocol (AP)} \index{Application Protocol}
                                            \index{AP}
\olddefinition{integrated resource} \index{Integrated Resource}
\end{olddefinitions}


\otherdefhead
%\sclause{Other definitions}

    For the purposes of this part of ISO~10303, the following definitions
apply.

\begin{definitions}
\definition{boilerplate}{Text whose wording is fixed and which has been
agreed to be present in a specific type of document.} \index{boilerplate}
\definition{style file}{A set of \latex{} macros assembled into a 
single file with an extension \file{.sty}.}
            \index{style file}
\definition{package file}{A style file for use with \latex 2e.}
            \index{package file}
\definition{facility}{A generic term for a set of \latex{} macros
          assembled for a common purpose. The macros may be defined in
          either a style file or a package file.}\index{facility}

\end{definitions}

\abbsubclause
%\sclause{Abbreviations}

    For the purposes of this part of ISO 10303, the following abbreviations
 apply.

\begin{symbols}
\symboldef{AIC}{Application Interpreted Construct} \index{AIC}
\symboldef{AM}{Application Module} \index{AM}
\symboldef{AP}{Application Protocol}  \index{AP}
\symboldef{DIS}{Draft International Standard} \index{DIS}
\symboldef{IS}{International Standard}         \index{IS}
\symboldef{ISOD}{ISO/IEC Directives, Part 2} \index{ISOD} \index{ISO/IEC Directives}
\symboldef{SD}{Supplementary Directives --- 
  \textit{SC4 Supplementary directives --- Rules for the structure and
   drafting of SC4 standards for industrial data}}\index{SD}\index{Supplementary Directives}
\symboldef{IS-REVIEW}{The ISO Editorial Board review (September 1994) of 
            twelve IS documents
            for conformance to ISO typographical and 
            layout requirements.} \index{IS-REVIEW}
\end{symbols}



\clause{Conformance requirements}  \label{sec:iconform}

    The facility files shall not be modified in any manner.

    If there is a need to modify any of the macro definitions then this
shall be done using the \latex{} 
\verb|\renewcommand|\ixcom{renewcommand} and/or the
\verb|\renewenvironment|\ixcom{renewenvironment}
commands. These shall be placed in a new \file{.sty} file (or files) 
which shall be called in within the preamble\index{preamble} of the 
document being typeset.

    There shall be no author specified \verb|\label{...}| commands where
the first two characters of the label are \verb|;s| (semicolon and `s');
the creation of labels starting with these characters is reserved to the 
maintainer of the facility files.

\begin{anote} For conformance to the \file{isov2} class, author specified
labels starting with the characters \verb|;i| (semicolon and `i') are
prohibited.
\end{anote}


\fcandaclause
%\clause{Fundamental concepts and assumptions}

    It is assumed that the reader of this document is familiar with the
\ixltx\latex{} document preparation system and in particular
with the \file{isov2}\ixclass{isov2} class and associated facilities 
described in 
\textit{LaTeX for ISO standards: User manual}.

\begin{note}Reference~\bref{lamport} describes the
      \latex{} system.
\end{note} % end note

    The reader is also assumed to be familiar with the ISO/IEC Directives 
Part~2 (ISOD)\index{ISOD} and
the SC4 Supplementary directives for the structure and drafting of 
SC4 standards (SD).\index{ISOD}\index{SD}

    If there are any discrepancies between the layout and wording of this 
document and the requirements of the ISOD or the SD,
then the requirements in those documents shall be
followed for ISO~10303 standard documents.

    The packages described herein have been designed to be used with
the \file{isov2}\ixclass{isov2} document class. It is highly unlikely that the
packages will perform at all with any other \latex{} document class.

    Because of many revisions over the years to the packages described
herein, a naming convention has been adopted for the package files.
The naming convention is that the
primary name of the file is suffixed by \file{v\#}, where
\file{\#} is the primary version number of the file in question.
All file primary names have been limited to a maximum of eight characters.

\begin{note}Table~\ref{tab:curfiles} shows the versions of the files
that were current at the time of publication.
\ixpack{step}\ixfile{stepv13.sty}
\ixpack{ir}\ixfile{irv12.sty}
\ixpack{ap}\ixfile{apv12.sty}
\ixpack{aic}\ixfile{aicv1.sty}
\ixpack{ats}\ixfile{atsv11.sty}
%%%\ixpack{am}\ixfile{amv1.sty}
\end{note} % end note

\begin{table}
\centering
\caption{File versions current at publication time} \label{tab:curfiles}
\begin{tabular}{|l|l|l|} \hline
\textbf{Facility} & \textbf{File}   & \textbf{Version} \\ \hline\hline
\file{step}    & \file{stepv13.sty} & v1.3.2 \\
\file{ir}      & \file{irv12.sty}   & v1.2   \\
\file{ap}      & \file{apv12.sty}   & v1.2   \\
%%%\file{am}      & \file{amv1.sty}    & v1.0   \\
\file{aic}     & \file{aicv1.sty}   & v1.0   \\
\file{ats}     & \file{atsv11.sty}  & v1.1   \\ 
\hline
\end{tabular}
\end{table}


\begin{note}
This document is not, and is never intended to become,
 a standard, although it has been laid out in a 
similar, but not necessarily identical, manner.
\end{note} % end note


\clearpage
\clause{The \file{step} package facility}

    The \file{step}\ixpack{step} package facility provides commands 
and environments 
applicable to all the ISO~10303 series of documents.

\sclause{Preamble commands}

    Certain commands shall be put in the preamble\index{preamble}
of any document.

    The command 
\verb|\partno{|\meta{number}\verb|}|\ixcom{partno}
is used to specify the Part number of the ISO~10303 standard
(e.g., \verb|\partno{3456}|).

    The command
\verb|\series{|\meta{series title}\verb|}|\ixcom{series}
is used to specify the name of the ISO~10303 series of which the Part 
is a member (e.g., \verb|\series{application modules}|).

    The command
\verb|\doctitle{|\meta{informal title}\verb|}|\ixcom{doctitle}
is used to specify the title to be used on the cover sheet.
For example: \\
\verb|\doctitle{LaTeX package files for ISO 10303: User manual}|

    The command
\verb|\ballotcycle{|\meta{number}\verb|}|\ixcom{ballotcycle}
is used to specify the ballot cycle number for the document
(e.g., \verb|\ballotcycle{2}|).

    The command\ixcom{ifhaspatents}
\verb|\haspatentstrue|\ixcom{haspatentstrue} shall be put in the
preamble when the document includes identified patented material;
otherwise the command \verb|\haspatentsfalse|\ixcom{haspatentsfalse}
may, but need not, be used instead.

    The \verb|\extrahead|\ixcom{extrahead} macro, from the \file{isov2}
class, shall be defined to be the document
number (e.g., \verb|\renewcommand{\extrahead}{47a}|).



\begin{anote}
The commands \verb|\standard|\ixcom{standard}, 
\verb|\yearofedition|\ixcom{yearofedition} and 
\verb|\languageofedition|\ixcom{languageofedition} from the \file{isov2}
class must also be put in the preamble.
\end{anote}


\sclause{Cover page}

    The command \verb+\STEPcover{+\meta{commands}\verb+}+\ixcom{STEPcover}
produces a cover page for a STEP document. 
The complete list of commands is shown below.

\begin{itemize}
\item \verb+\wg{+\meta{working group}\verb+}+\ixcom{wg}
      the working 
      group or other committee producing the document e.g., WG 5
\item \verb+\docnumber{+\meta{number}\verb+}+\ixcom{docnumber}
       the number
       of the document e.g., 156
\item \verb+\docdate{+\meta{date}\verb+}+\ixcom{docdate}
       date of 
       publication e.g., 1993/07/03
\item \verb+\oldwg{+\meta{working group}\verb+}+\ixcom{oldwg}
       superseded 
       working group e.g., WG 1
\item \verb+\olddocnumber{+\meta{number}\verb+}+\ixcom{olddocnumber}
        number of previous document e.g., 107
\item \verb+\abstract{+\meta{text}\verb+}+\ixcom{abstract}
        an abstract 
        of the document
\item \verb+\keywords{+\meta{text}\verb+}+\ixcom{keywords}
        for listing 
        relevant keywords
\item \verb+\owner{+\meta{text}\verb+}+\ixcom{owner}
         name of the project leader
\item \verb+\address{+\meta{text}\verb+}+\ixcom{address}
        address of the project leader
\item \verb+\telephone{+\meta{number}\verb+}+\ixcom{telephone}
         the project leader's telephone number
\item \verb+\fax{+\meta{number}\verb+}+\ixcom{fax}
         the project leader's fax number
\item \verb+\email{+\meta{text}\verb+}+\ixcom{email}
        Email address of the project leader
\item \verb+\altowner{+\meta{text}\verb+}+\ixcom{altowner}
         name of the editor of the document
\item \verb+\altaddress{+\meta{text}\verb+}+\ixcom{altaddress}
         the editor's address 
\item \verb+\alttelephone{+\meta{number}\verb+}+\ixcom{alttelephone}
         the editor's telephone number
\item \verb+\altfax{+\meta{number}\verb+}+\ixcom{altfax}
         the editor's fax number
\item \verb+\altemail{+\meta{text}\verb+}+\ixcom{altemail}
           the editor's Email address 
\item \verb+\comread{+\meta{text}\verb+}+\ixcom{comread}
           comments to 
           the reader
\end{itemize}

    Use only those commands within \verb|\STEPcover| that are relevant 
to the purposes at hand. The order of the commands within 
\verb|\STEPcover| is immaterial.

\begin{example}
The commands used to produce the cover sheet for one version of this 
document were:
\begin{verbatim}
\STEPcover{
\wg{EC}
\docnumber{41}
\oldwg{EC}
\olddocnumber{35}
\docdate{1994/08/19}
\abstract{This document describes the \latex{} style files created for ISO~10303.
          It also describes the program GenIndex which provides some 
          capabilities to assist in the creation of indexes for \latex{}
          documents in general.}
\keywords{\latex, Style file, GenIndex, Index}
\owner{Peter R Wilson}
\address{NIST\newline
         Bldg. 220, Room A127 \newline
         Gaithersburg, MD 20899 \newline
         USA }
\telephone{+1 (301) 975-2976}
\email{\texttt{pwilson@cme.nist.gov}}
\altowner{Tony Day}
\altaddress{Sikorsky Aircraft}
\comread{This document serves two purposes. Firstly, it provides a description
         of the current \latex{} style file for ISO 10303. Secondly, the source
         can be used as an example of using the \latex{} commands.} % end comread
} % end of STEPcover
\end{verbatim}
Note the use of the \verb|\newline| command instead of \verb|\\| in 
the argument of the \verb|\address| command to indicate a new line. The
\verb|\newline| is needed to ensure satisfactory conversion to HTML.
\end{example} % end example

    The macro \verb|\draftctr|\ixcom{draftctr} generates boilerplate that
may be used in the `Comments to Reader' section of a cover page.
\begin{example}
The \latex{} source \verb|\draftctr This document \ldots| prints:

\draftctr This document \ldots
\end{example}

\sclause{Heading commands}

    The commands described in this subclause specify various `standard'
clause headings.

\ssclause{The Foreword commands}

    The \verb+\Foreword+\ixcom{Foreword} command specifies that a 
table of contents, list of figures and a list of tables be produced. 
Page numbering is roman style and the table of contents starts on page iii.
A new unnumbered clause entitled Foreword is started containing both 
ISO required boilerplate and boilerplate\index{boilerplate}
text specific to ISO 10303.


    Any text may be written after the \verb|\Foreword| command. The
Foreword clause is ended by the 
\verb+\endForeword{+\meta{norm annexes}\verb+}{+\meta{inf annexes}\verb+}+ 
command.\ixcom{endForeword} This command takes two parameters.
\begin{enumerate}
\item \meta{norm annexes} A phrase that starts the sentence 
      `\meta{norm annexes} a normative part of this part \ldots'.
     If there are no normative annexes, then use an empty
     argument (i.e., \verb|{}| with no spaces between the braces).
\item \meta{inf annexes} A phrase that starts the sentence 
     `\meta{inf annexes} for information only.'.
     If there are no informative annexes, then use an
     empty argument.
\end{enumerate}

    The \verb|\endForeword| command produces some additional 
boilerplate\index{boilerplate} text specifically for ISO 10303. 

\begin{example}
The \latex{} source for the Foreword for this document is:
\begin{verbatim}
\Foreword
\fwdshortlist
\endForeword
{Annexes A, B and C are}  % normative annexes
{Annexes D, E and F are} % informative annexes
\end{verbatim}
\end{example} % end example


    The \verb|\fwdshortlist|\ixcom{fwdshortlist} command 
produces boilerplate text for inclusion in the foreword referencing
the STEP parts and series. 
\begin{example}
In this document, the command \verb|\fwdshortlist| prints:

\fwdshortlist
\end{example}

    The \verb|\steptrid|\ixcom{steptrid} command
produces boilerplate text for inclusion in the foreword describing 
the creators of a STEP Technical Report.

\begin{example}
The \latex{} command \verb|\steptrid| in this document prints:
  
\steptrid
\end{example}


\ssclause{The Introduction environment}

    The 
\verb+\begin{Introduction}+\ixenv{Introduction}
environment starts a new unnumbered clause 
entitled Introduction and adds some boilerplate\index{boilerplate}
text specifically for ISO~10303.

\begin{example}
    The following \latex{} source was used to specify the Introduction 
to this document. \label{ex:intro}
\begin{verbatim}
\begin{Introduction}

    This part of ISO 10303 specifies the \latex{} facilities 
specifically designed for use in preparing the various parts of 
this standard.

\begin{majorsublist}
\item the \file{step} package facility;
\item the \file{ir} package facility;
\item the \file{ap} package facility;
\item the \file{aic} package facility;
\item the \file{atc} package facility.
\end{majorsublist}

    This part of ISO 10303 is intended to be used ...

\sclause*{Overview}

    This document describes a set of \latex{} files for use
within ISO~10303 ...

\end{Introduction}
\end{verbatim}
\end{example} % end example


\ssclause{The stepparttitle command}

   The \verb+\stepparttitle{+\meta{part title}\verb+}+\ixcom{stepparttitle}
command produces the title for
an ISO~10303 part, where \meta{part title} is the title of the part.

\begin{anexample}The title for this document was produced using:
\begin{verbatim}
\stepparttitle{Documentation methods:
               LaTeX package files for ISO 10303: User manual}
\end{verbatim}
\end{anexample} % end example


\ssclause{Other headings}

    Most of these commands take no parameters. They start document clauses
with particular titles. The commands that take no parameters are listed
in \tref{tab:noparamhead}. Some of these headings commands have predefined
labels, which are also listed in the table.
\ixcom{partidefhead}
\ixcom{otherdefhead}
\ixcom{introsubhead}
\ixcom{fcandasubhead}
\ixcom{shortnamehead}
\ixcom{picshead}
\ixcom{objreghead}
\ixcom{docidhead}
\ixcom{schemaidhead}
\ixcom{expresshead}
\ixcom{listingshead}
\ixcom{expressghead}
%%\ixcom{modelscopehead}
\ixcom{techdischead}
\ixcom{exampleshead}

\begin{anote}
 In the tables, C = clause, SC = subclause, SSC = subsubclause,
NA = normative annex, IA = informative annex.
\end{anote} % end note

\settowidth{\prwlen}{\quad Protocol Implementation Conformance Statement}
\begin{table}
\centering
\caption{STEP package parameterless heading commands}
\label{tab:noparamhead}
\begin{tabular}{|l|c|p{\prwlen}|l|} \hline
\textbf{Command} & \textbf{Clause} & \textbf{Default text} & \textbf{Label} \\ \hline
\verb|\partidefhead| & SC & Terms defined in ISO 10303-1 &  \\
\verb|\otherdefhead| & SC & Other definitions & \\
\verb|\introsubhead| & SC & Introduction &  \\
\verb|\fcandasubhead| & SC & Fundamental concepts and assumptions & \\
\verb|\shortnamehead| & NA & Short names of entities & \verb|;ssne| \\
\verb|\picshead| & NA & Protocol Implementation Conformance Statement (PICS) proforma & \verb|;spics| \\
\verb|\objreghead| & NA & Information object registration  & \verb|;sior| \\
\verb|\docidhead| & SC & Document identification & \\
\verb|\schemaidhead| & SC & Schema identification &  \\
\verb|\expresshead| & IA & \Express{} listing &  \\
\verb|\listingshead| & IA & Computer interpretable listings & \verb|;scil| \\
\verb|\expressghead| & IA & \ExpressG\ diagrams & \verb|;seg| \\
%%%%\verb|\modelscopehead| & IA & Model scope & \verb|;sms| \\
\verb|\techdischead| & IA & Technical discussions & \verb|;std| \\ 
\verb|\exampleshead| & IA & Examples & \verb|;sex| \\
\hline
\end{tabular}
\end{table}

    The commands listed in \tref{tab:paramhead} are equivalent to the
general sectioning commands, but are intended to indicate the start
of a particular documentation element. These commands take either one
or two parameters. The parameters are denoted in the column headed
`Parameterized title'.
\ixcom{refdefhead}
\ixcom{schemahead}
\ixcom{typehead}
\ixcom{entityhead}
\ixcom{rulehead}
\ixcom{functionhead}
\ixcom{atypehead}
\ixcom{anentityhead}
\ixcom{arulehead}
\ixcom{afunctionhead}
\ixcom{aschemaidhead}
\ixcom{singletypehead}
\ixcom{singleentityhead}
\ixcom{singlerulehead}
\ixcom{singlefunctionhead}

\begin{table}
\centering
\caption{STEP package parameterized heading commands}
\label{tab:paramhead}
\begin{tabular}{|l|c|l|} \hline
\textbf{Command} & \textbf{Clause} & \textbf{Parameterized title} \\ \hline
\verb|\refdefhead| & SC & Terms defined in \meta{ISO ref} \\
\verb|\schemahead| & C & \meta{schema name} \\
\verb|\singletypehead| & SC & \meta{schema name} type definition:
\meta{type name} \\
\verb|\typehead| & SC & \meta{schema name} type definitions \\
\verb|\atypehead| & SSC & \meta{type name} \\
\verb|\singleentityhead| & SC & \meta{schema name} entity definition:
\meta{entity name} \\
\verb|\entityhead| & SC & \meta{schema name} entity definitions \meta{group} \\
\verb|\anentityhead| & SSC & \meta{entity name} \\
\verb|\singlerulehead| & SC & \meta{schema name} rule definition:
\meta{rule name} \\
\verb|\rulehead| & SC & \meta{schema name} rule definitions \\
\verb|\arulehead| & SSC & \meta{rule name} \\
\verb|\singlefunctionhead| & SC & \meta{schema name} function definition:
\meta{function name} \\
\verb|\functionhead| & SC & \meta{schema name} function definitions \\
\verb|\afunctionhead| & SSC & \meta{function name} \\
\verb|\aschemaidhead| & SSC & \meta{schema name} identification \\ \hline
\end{tabular}
\end{table}

\sclause{Miscellaneous commands}

    The following commands provide some printing options for commonly 
occurring situations. The \verb|\nexp{}|\ixcom{nexp} command is intended 
to be used for printing \Express{} \index{express@{\Express}} entity names etc.
\begin{itemize}
\item The command \verb|\B{abc}|\ixcom{B} prints \B{abc}
\item The command \verb|\E{abc}|\ixcom{E} prints \E{abc}
\item The command \verb|\Express|\ixcom{Express} prints \Express{}
\item The command \verb|\ExpressG|\ixcom{ExpressG} prints \ExpressG{}
\item The command \verb|\ExpressI|\ixcom{ExpressI} prints \ExpressI{}
\item The command \verb|\ExpressX|\ixcom{ExpressX} prints \ExpressX{}
\item The command \verb|\BG{|\meta{mathsymbol}\verb|}|\ixcom{BG} prints 
      \meta{mathsymbol} in bold font.
\item The command \verb|\HASH|\ixcom{HASH} prints \HASH{}
\item The command \verb|\LT|\ixcom{LT} prints \LT{}
\item The command \verb|\LE|\ixcom{LE} prints \LE{}
\item The command \verb|\NE|\ixcom{NE} prints \NE{}
\item The command \verb|\INE|\ixcom{INE} prints \INE{}
\item The command \verb|\GE|\ixcom{GE} prints \GE{}
\item The command \verb|\GT|\ixcom{GT} prints \GT{}
\item The command \verb|\CAT|\ixcom{CAT} prints \CAT{}
%\item The command \verb|\HAT|\ixcom{HAT} prints \HAT{}
\item The command \verb|\QUES|\ixcom{QUES} prints \QUES{}
%\item The command \verb|\BS|\ixcom{BS} prints \BS{}
\item The command \verb|\IEQ|\ixcom{IEQ} prints \IEQ{}
\item The command \verb|\INEQ|\ixcom{INEQ} prints \INEQ{}
\item The command \verb|\nexp{an\_entity}|\ixcom{nexp} prints \nexp{an\_entity}
\item The command \verb|\xword{ExpResS\_KeyworD}|\ixcom{xword}
      prints \xword{ExpResS\_KeyworD}
\end{itemize}

The command \verb|\ix{|\meta{word or phrase}\verb|}|\ixcom{ix} both prints 
its parameter and also makes an index entry out of it.

The command \verb|\mnote{|\meta{Marginal note text}\verb|}|\ixcom{mnote}
prints its parameter as a 
marginal note. \mnote{Quite a lot of marginal note text.}
Remember, though, that marginal notes are only printed when the 
\file{isov2}\ixclass{isov2} class \file{draft}\ixopt{draft} option
is used. Marginal notes are not allowed by ISO.

\ssclause{Standard reference commands}

    Many parts of STEP use the same normative or informative references.
The most common of these are provided via commands. The currently available 
commands are listed in \tref{tab:nrefc}.
\ixcom{nrefasni}
\ixcom{nrefparti}
\ixcom{nrefpartxi}
\ixcom{nrefpartxii}
\ixcom{nrefpartxxi}
\ixcom{nrefpartxxii}
\ixcom{nrefpartxxxi}
\ixcom{nrefpartxxxii}
\ixcom{nrefpartxli}
\ixcom{nrefpartxlii}
\ixcom{nrefpartxliii}

    The naming convention used for references to parts of ISO~10303 is to
end the command name with the number of the part expressed in lower case 
Roman numerals. Should further references to parts of ISO~10303 be added later, 
the same naming convention will be used.

\begin{table}
\centering
\caption{Commands for common references to standards} \label{tab:nrefc}
\begin{tabular}{|l|l|} \hline
\textbf{Standard} & \textbf{Command} \\ \hline
ISO/IEC 8824-1 & \verb|\nrefasni| \\
ISO 10303-1    & \verb|\nrefparti|  \\
ISO 10303-11   & \verb|\nrefpartxi|  \\
ISO 10303-12   & \verb|\nrefpartxii|  \\
ISO 10303-21   & \verb|\nrefpartxxi|  \\
ISO 10303-22   & \verb|\nrefpartxxii|  \\
ISO 10303-31   & \verb|\nrefpartxxxi|  \\
ISO 10303-32   & \verb|\nrefpartxxxii|  \\
ISO 10303-41   & \verb|\nrefpartxli|  \\
ISO 10303-42   & \verb|\nrefpartxlii|  \\
ISO 10303-43   & \verb|\nrefpartxliii|  \\ \hline
\end{tabular}
\end{table}


\begin{example} The normative references in this document were input as:
\begin{verbatim}
\begin{nreferences}
\isref{ISO/IEC Directives, Part 2}{Rules for the structure and drafting 
       International Standards, Fourth edition.}
\isref{...}
\nrefparti
\nrefpartxi
\nrefpartxii
\nrefasni
\disref{P. R. WILSON:---}{LaTeX for ISO standards: User manual.}
\end{nreferences}
\end{verbatim}
\end{example}

\begin{anote}
For the commands providing references to STEP parts, the part number 
is denoted by lowercase Roman numerals. Should further reference
commands be provided for other STEP parts, then the same naming scheme
will be used.
\end{anote}

    Some informative bibliographic reference commands are also provided.

The command \verb|\bibidefo|\ixcom{bibidefo} produces the reference
entry to the IDEF0 document and \verb|\brefidefo|\ixcom{brefidefo}
can be used for citing the reference in the body of the document.

The commands \verb|\bibidefix|\ixcom{bibidefix} and
\verb|\bibieeedefix|\ixcom{bibieeedefix} produce the reference entry
to the original FIPS version of IDEF1X and the IEEE version of IDEF1X
respectively. 
The command \verb|\brefidefix|\ixcom{brefidefix} can be used for
citing an IDEF1X reference in the body of the document.

   IDEF0 and IDEF1X are references \brefidefo{} and \brefidefix{}
in the bibliography.


\begin{example} Part of the bibliography for this document looks like:
\begin{verbatim}
\begin{references}
...
\reference{BRYAN, M.,}{SGML --- An Author's Guide to the Standard Generalized
           Markup Language,}{Addison-Wesley Publishing Co., 1988. }\label{bryan}
\bibidefo
\bibieeeidefix
\reference{RESSLER, S.,}{The National PDES Testbed Mail Server User's Guide,}
           {NSTIR 4508, National Institute of Standards and Technology,
           Gaithersburg, MD 20899. January 1991.} \label{ressler}
...
\end{references}
\end{verbatim}
\end{example}

\begin{example}The source for one of the sentences above was:
\begin{verbatim}
IDEF0 and IDEF1X are references \brefidefo{} and \brefidefix{} in the bibliography.
\end{verbatim}
\end{example}

    
\sclause{Commands for documenting EXPRESS code} \index{express@\Express\}


    The Supplementary Directives\index{SD} specify the layout of the 
documentation of \Express{} code. The following commands are intended 
to serve two purposes:
\begin{enumerate}
\item To provide environments for the documentation of entity 
      attributes, etc.;
\item To provide begin and end tags around all the \Express{} code 
      documentation.
\end{enumerate}

    This latter purpose is to provide an enabling capability for the 
automatic extraction of portions of the documentation of an 
\Express{} model so that they could be placed into another document. 
For example, tools could be developed that would automatically extract 
pieces of resource model documentation and place them into an AP document.

\begin{anote}
This document uses the \file{hyphenat}\ixpack{hyphenat} 
package which enables automatic hyphenation of `words'
containing the underscore character command 
%(\verb|\_|\index{_/@\verb|\_|}). 
(\verb|\_|\index{_/@\texttt{\bs\_}}). 
Such words would normally have to
be coded as \verb|long\_\-word| to ensure potential hyphenation 
at the position of the underscore. When using the \file{hyphenat} package
it is an error to put the \verb|\-|\ixcom{-} discretionary
hyphen command after the underscore command as this then stops further
hyphenation.
\end{anote}


\ssclause{Environments ecode, eicode and excode}

    The \verb|ecode|\ixenv{ecode} environment is for 
tagging \Express{} code. It prints the appropriate title
and sets up the relevant fonts.

\begin{anexample} The following \latex{} source code:
\begin{verbatim}
\begin{ecode}\ixent{an\_entity}
\begin{verbatm}  % read verbatm as verbatim
*)
ENTITY an_entity;
  attr : REAL;
END_ENTITY;
(*
\end{verbatm}   % read verbatm as verbatim
\end{ecode}
\end{verbatim}

produces:

\begin{ecode}\ixent{an\_entity}
\begin{verbatim}
*)
ENTITY an_entity;
  attr : REAL;
END_ENTITY;
(*
\end{verbatim}
\end{ecode}
\end{anexample} % end example

    Similarly, the \verb|eicode|\ixenv{eicode} and
\verb|excode|\ixenv{excode} environments are for tagging \ExpressI{} 
and \ExpressX{} code and setting up the relevant titles and fonts.


\ssclause{Environment attrlist}

    The \verb|attrlist|\ixenv{attrlist} environment produces 
the heading for attribute definitions and sets up 
a \verb|description|\ixenv{description} list.

\begin{anexample}The following \latex{} source code:
\begin{verbatim}
\begin{attrlist}
\item[attr\_1] The \ldots
\item[attr\_2] This \ldots
\end{attrlist}
\end{verbatim}

produces:

\begin{attrlist}
\item[attr\_1] The \ldots
\item[attr\_2] This \ldots
\end{attrlist}
\end{anexample} % end example

\ssclause{Environment fproplist}

    The \verb|fproplist|\ixenv{fproplist} environment is similar to 
\verb|attrlist|\ixenv{attrlist} except that it is for
formal propositions.

\begin{anexample}The following \latex{} source code:
\begin{verbatim}
\begin{fproplist}
\item[un\_1] The value of \ldots\ shall be unique.
\item[gt\_0] The value of \ldots\ shall be greater than zero.
\end{fproplist}
\end{verbatim}

produces:

\begin{fproplist}
\item[un\_1] The value of \ldots\ shall be unique.
\item[gt\_0] The value of \ldots\ shall be greater than zero.
\end{fproplist}
\end{anexample} % end example

\ssclause{Other listing environments}

    The environments \verb|iproplist|\ixenv{iproplist}, 
\verb|enumlist|\ixenv{enumlist}, and \verb|arglist|\ixenv{arglist} are
similar to \verb|attrlist|\ixenv{attrlist}.
 Respectively they are environments for
informal propositions, enumerated items, and argument definitions.

\ssclause{Indexing}

    The command \verb|\ixent{|\meta{entity}\verb|}|\ixcom{ixent} 
generates an index
entry for the entity \meta{entity}.

    There are similar macros, each of which takes the name of the 
declaration as its argument, for indexing the other \Express{} declarations:
\verb|\ixenum|\ixcom{ixenum} for enumeration,
\verb|\ixfun|\ixcom{ixfun} for function,
\verb|\ixproc|\ixcom{ixproc} for procedure,
\verb|\ixrule|\ixcom{ixrule} for rule,
\verb|\ixsc|\ixcom{ixsc} for subtype\_constraint,
\verb|\ixschema|\ixcom{ixschema} for schema,
\verb|\ixselect|\ixcom{ixselect} for select, and
\verb|\ixtype|\ixcom{ixtype} for type.

\ssclause{Documentation tagging}

    Several environments are defined to tag the general documentation 
of \Express{} code. \index{express@\Express\}

    The environment \verb+\begin{espec}{+\meta{name}\verb+}+\ixenv{espec}
may be used to enclose, and give a name to, a complete specification 
block for an \Express{} entity. There are analogous environments --- 
\verb+fspec+\ixenv{fspec}, 
\verb+rspec+\ixenv{rspec}, 
\verb+sspec+\ixenv{sspec}, and
\verb+tspec+\ixenv{tspec} --- 
for functions, rules, schemas and types respectively.

    The \verb|dtext|\ixenv{dtext} environment may be used to anonymously 
enclose descriptive text.

\begin{example}\label{ex:code} Here is the suggested tagged documentation 
style for part of an \Express{} model.
\begin{verbatim}
%\ssclause{committee\_def}
\begin{espec}{committee_def}
\begin{dtext}
    A committee is composed of an odd number of people. 
Each committee also has a name.
    The ideal size of a committee is less than three.

\begin{anote} Figures and tables may also be placed here. \end{anote} % end note
\end{dtext}
\begin{ecode}\ixent{committee\_def}
\begin{verbatm} % read verbatm as verbatim
*)
ENTITY committee_def;
  title   : name;
  members : SET [1:?] OF person;
DERIVE
  ideal : BOOLEAN := SIZEOF(members) = 1;
UNIQUE
  un1 : title;
WHERE
  odd_members : ODD(SIZEOF(members));
END_ENTITY;
(*
\end{verbatm}   % read verbatm as verbatim
\end{ecode}
\begin{attrlist}
\item[title] The name of the committee.
\item[members] The people who form the committee.
\item[ideal] TRUE if there is only one person 
             on the committee.
             That is, if the committee is the ideal size.
\end{attrlist}
\begin{fproplist}
\item[un1] The \nexp{title} of the committee shall be unique.
\item[odd\_members] There shall be an odd number of people 
                    on the committee.
\end{fproplist}
\begin{iproplist}
\item[chair] The members of a committee shall appoint one of 
             their number as
             chair of the committee.
\end{iproplist}
\end{espec}
\end{verbatim}
\end{example} % end example

\begin{example}
The code in \eref{ex:code} produces the following result:

\begin{espec}{committee_def}
\begin{dtext}
    A committee is composed of an odd number of people. 
Each committee also has a name.
The ideal size of a committee is less than three.

\begin{anote} Figures and tables may also be placed here. \end{anote} % end note
\end{dtext}
\begin{ecode}\ixent{committee\_def}
\begin{verbatim}
*)
ENTITY committee_def;
  title   : name;
  members : SET [1:?] OF person;
DERIVE
  ideal : BOOLEAN := SIZEOF(members) = 1;
UNIQUE
  un1 : title;
WHERE
  odd_members : ODD(SIZEOF(members));
END_ENTITY;
(*
\end{verbatim}   % read verbatm as verbatim
\end{ecode}
\begin{attrlist}
\item[title] The name of the committee.
\item[members] The people who form the committee.
\item[ideal] TRUE if there is only one person on the committee. That is, if
             the committee is the ideal size.
\end{attrlist}
\begin{fproplist}
\item[un1] The \nexp{title} of the committee shall be unique.
\item[odd\_members] There shall be an odd number of people on the committee.
\end{fproplist}
\begin{iproplist}
\item[chair] The members of a committee shall appoint one of their number as
             chair of the committee.
\end{iproplist}
\end{espec}

\end{example} % end example


\sclause{Commands producing boilerplate text} \index{boilerplate}

    The following commands produce boilerplate text as specified by the 
Supplementary Directives\index{SD}.

\begin{anote}
 In the examples, 
the parameters of those commands that
take them have been specified in 
\textit{this font style} so their effects can
be seen in the resulting printed text.
\end{anote}

\ssclause{Definition of \ExpressG}

    The \verb|\expressgdef|\ixcom{expressgdef} prints the boilerplate
for where the definition of \ExpressG{} can be found.

\begin{anexample}
The command \verb|\expressgdef| prints: 

\expressgdef
\end{anexample} 

\ssclause{Major subdivision listing}

    The \verb|majorsublist|\ixenv{majorsublist}
environment prints the boilerplate for the heading of a listing of
major subdivisions of the standard and starts an itemized list.
An illustration of its use is given in \eref{ex:intro} 
on page~\pageref{ex:intro}.

The heading text is produced by the 
\verb|\majorsubname|\ixcom{majorsubname} command.

\begin{anexample} The command \verb|\majorsubname| command prints:

\majorsubname

\end{anexample}

\ssclause{Schema introduction}

    The command \verb|\schemahead{|\meta{schema name}\verb|}|\ixcom{schemahead} prints the heading for a schema clause.

    The command \verb+\schemaintro{+\meta{schema name}\verb+}+\ixcom{schemaintro} 
produces the boilerplate for the introduction to an \Express{} schema
clause.

\begin{anexample}The command \verb|\schemaintro{\nexp{this\_schema}}| prints:

\schemaintro{\nexp{this\_schema}}
\end{anexample}
  


\ssclause{Short names of entities}

    The command \verb|\shortnamehead|\ixcom{shortnamehead} prints the
heading for the short names annex.

    The command \verb|\shortnames|\ixcom{shortnames} 
produces the boilerplate for the
introduction to the annex listing short names.

\begin{anexample}The command \verb|\shortnames| prints:

\shortnames 
\end{anexample} %end example

\ssclause{Registration commands}

    The command \verb|\objreghead|\ixcom{objreghead} prints the heading
for the information object registration annex.

    The command \verb|\docidhead|\ixcom{docidhead} prints the heading
for the document identification subclause.


    The command 
\verb+\docreg{+\meta{version no}\verb+}+\ixcom{docreg}
produces the boilerplate for document registration. The command takes
one parameter:
\meta{version no} which is the version number.\footnote{The
SD say that the version number should be 1 for a first edition IS.
The version number is incremented by one for each corrigenda,
amendment or new edition.}

\begin{example}The command \verb|\docreg{1}|
         prints:

\docreg{\textit{1}} 
\end{example} % end example

    The command \verb|\schemaidhead|\ixcom{schemaidhead} prints the heading
for the schema identification subclause. 
The command 
\verb|\aschemaidhead{|\meta{schema name}\verb|}|\ixcom{aschemaidhead} 
prints the heading for a particular schema identification subsubclause.

    The command
\verb+\schemareg{+\meta{version no}\verb+}{+\meta{p2}\verb+}{+\meta{p3}\verb+}{+\meta{p4}\verb+}{+\meta{p5}\verb+}{+\meta{p6}\verb+}+\ixcom{schemareg} produces the boilerplate concerning
schema registration. The command takes six parameters.
\begin{enumerate}
\item \meta{version no} The version number;
\item \meta{p2} The name of an \Express{} schema (with underscores);
\item \meta{p3} The number of the schema object (typically 1);
\item \meta{p4} The name of the schema, with hyphens replacing any
                  underscores in the name;
\item \meta{p5} The number identifying the schema;
\item \meta{p6} The clause or annex in which the schema is defined.
\end{enumerate}

\begin{example}The command \\
 \verb|\schemareg{1}{a\_schema}{3}{a-schema}{5}{clause 6}|
prints:

\schemareg{\textit{1}}{a\_schema}{\textit{3}}{\textit{a-schema}}{\textit{5}}{\textit{clause 6}}
\end{example} % end example


\ssclause{Computer interpretable listings} 

    The command \verb|\listingshead|\ixcom{listingshead} prints the
heading for the computer interpretable listings annex.

    The command 
\verb|\expurls{|\meta{short}\verb|}{|\meta{express}\verb|}|\ixcom{expurls}
produces the boilerplate for the introduction to the annex 
listing short names and \Express, where \meta{short} is the URL for the short
names and \meta{express} is the URL for the \Express.

\begin{anexample} The command 
  \verb|\expurls{http:/www.short/}{http://www.express/}| prints:

\expurls{http://www.short/}{http://www.express/}

\end{anexample}

\clearpage
\clause{The \file{ir} package facility} 

    The \file{ir}\ixpack{ir} package provides commands and environments
specifically for the ISO~10303 Integrated Resources series of documents.

    Use of this package requires the use of the \file{step}\ixpack{step} 
package.

\sclause{Boilerplate commands}

    The \file{ir} package modifies the \verb|\fwdshortlist|\ixcom{fwdshortlist}
command to produce extra IR-specific boilerplate.

    The following commands produce boilerplate text as specified by the 
SD\index{SD}.


\ssclause{Integrated resource EXPRESS-G} 

    The command \verb|\expressghead|\ixcom{expressghead} prints the
heading for the \ExpressG{} diagrams annex.

    The command \verb+\irexpressg+\ixcom{irexpressg} 
produces the boilerplate for the introduction to the integrated 
resource \ExpressG{} annex.
\index{expressg@\ExpressG\}

\begin{anexample}The command \verb|\irexpressg| prints:

\irexpressg

 \end{anexample} % end example

%%%%%%%%%%%%%%%%%%%%%%%%%%%%%%
%%%%\end{document}
%%%%%%%%%%%%%%%%%%%%%%%%%%%%%%



\clearpage
\clause{The \file{ap} package facility}

    The \file{ap}\ixpack{ap} package provides commands and environments
specifically for the ISO~10303 Application Protocol series of documents.

    Use of this package requires the use of the \file{step}\ixpack{step} 
package.

\sclause{Preamble commands}

    Certain commands shall be put in the preamble of an AP document.

    The command 
\verb+\aptitle{+\meta{title of AP}\verb+}+\ixcom{aptitle}
shall be put into the preamble. \index{preamble} The parameter shall be of 
such a form that
it will read naturally in a sentence of the form: 
`\ldots for the \meta{title of AP} application protocol.'.

\begin{anexample}
  For the purposes of later examples, the command
\verb|\aptitle{|\texttt{\theap}\verb|}| has been put in the preamble
of this document.
\end{anexample} % end example

    If the AP makes use of one or more
AICs\index{AIC}, then the command \verb|\aicinaptrue|\ixcom{aicinaptrue} 
shall be put in the document preamble.

   If a mapping specification is used instead of a mapping table,
the command \verb|\mapspectrue|\ixcom{mapspectrue} shall be put
in the preamble. If mapping templates are used then 
\verb|\maptemplatetrue|\ixcom{maptemplatetrue} shall also be put in the
preamble.

    If IDEF1X is used instead of \ExpressG{} as the graphical form for the
ARM, then \verb|\idefixtrue|\ixcom{idefixtrue} shall be put in the preamble.

  

\sclause{Heading commands}

    These commands start document clauses with particular titles. The
commands that take no parameters are listed in \tref{tab:apnpheads}.
Some of these commands have predefined labels, which are also listed in 
the table.
\ixcom{inforeqhead}
\ixcom{uofhead}
\ixcom{applobjhead}
\ixcom{applasserthead}
\ixcom{aimhead}
\ixcom{maptablehead}
\ixcom{templateshead}
\ixcom{aimshortexphead}
\ixcom{confreqhead}
\ixcom{aimlongexphead}
\ixcom{aimshortnameshead}
\ixcom{impreqhead}
\ixcom{aamhead}
\ixcom{aamdefhead}
\ixcom{aamfighead}
\ixcom{armhead}
\ixcom{aimexpressghead}
\ixcom{aimexpresshead}
\ixcom{apusagehead}

\settowidth{\prwlen}{\quad Application activity model definitions}
\begin{table}
\centering
\caption{AP package parameterless heading commands}
\label{tab:apnpheads}
\begin{tabular}{|l|c|p{\prwlen}|l|} \hline
\textbf{Command} & \textbf{Clause} & \textbf{Default text} & \textbf{Label} \\ \hline
\verb|\inforeqhead| & C & Information requirements & \verb|;sireq| \\
\verb|\uofhead| & SC & Units of functionality  & \verb|;suof| \\
\verb|\applobjhead| & SC & Application objects  & \verb|;sao| \\
\verb|\applasserthead| & SC & Application assertions  & \verb|;saa| \\
\verb|\aimhead| & C & Application interpreted model & \verb|;saim| \\
\verb|\mappinghead| & SC & Mapping table, or & \verb|;smap| \\
                    &    & Mapping specification  & \verb|;smap| \\
\verb|\templateshead| & SSC & Mapping templates &    \\
\verb|\aimshortexphead| & SC & AIM \Express{} short listing & \verb|;saesl| \\
\verb|\confreqheadhead| & C & Conformance requirements & \verb|;scr| \\
\verb|\aimlongexphead| & NA & AIM \Express{} expanded listing  & \verb|;saeel| \\
\verb|\aimshortnameshead| & NA & AIM short names  & \verb|;sasn| \\
\verb|\impreqhead| & NA & Implementation method specific requirements & \verb|;simreq| \\
\verb|\aamhead| & IA & Application activity model & \verb|;saam| \\
\verb|\aamdefhead| & SC & Application activity model definitions and abbreviations & \verb|| \\
\verb|\aamfighead| & SC & Application activity model diagrams & \verb|| \\
\verb|\armhead| & IA & Application reference model & \verb|;sarm| \\
\verb|\aimexpressghead| & IA & AIM \ExpressG{} & \verb|;saeg| \\
\verb|\aimexpresshead| & IA & AIM \Express{} listing & \verb|| \\
\verb|\apusagehead| & IA & Application protocol usage guide & \verb|;sapug| \\
 \hline
\end{tabular}
\end{table}

    The commands listed in \tref{tab:appheads} take parameters.
\ixcom{auofhead}
\ixcom{mapuofhead}
\ixcom{mapobjecthead}
\ixcom{mapattributehead}

\begin{table}
\centering
\caption{AP package parameterized heading commands}
\label{tab:appheads}
\begin{tabular}{|l|c|l|} \hline
\textbf{Command} & \textbf{Clause} & \textbf{Parameterized title} \\ \hline
\verb|\auofhead| & SSC & \meta{UoF} \\ 
\verb|\mapuofhead| & SSC & \meta{UoF} \\
\verb|\mapobjecthead| & SSSC & \meta{application object} \\
\verb|\mapattribhead| & SSSSC & \meta{attribute} \\
\hline
\end{tabular}
\end{table}

\sclause{Boilerplate commands}

    The following commands produce boilerplate text as specified by the 
SD\index{SD}.

\begin{anote}
 In the examples, the parameters of those commands that
take them have been specified in 
\textit{this font style} so their effects can
be seen in the resulting printed text.
\end{anote}

\ssclause{AP introduction}

    The command \verb|\apextraintro|\ixcom{apextraintro} produces extra
boilerplate for the Introduction to an AP.

\begin{anexample}The command \verb|\apextraintro| prints:

\apextraintro
\end{anexample} %end example

\ssclause{AP scope}

    The command \verb+\apscope{+\meta{application purpose and context}\verb+}+\ixcom{apscope} 
produces the boilerplate for the start of an AP scope\index{scope} clause.

\begin{anexample}The command \verb|\apscope{application purpose and context.}|
         prints:

\apscope{\textit{application purpose and context.}} 
\end{anexample} 

\ssclause{AP information requirements}

  The command \verb|\inforeqhead|\ixcom{inforeqhead} prints the
heading for the information requirements clause.

  The command \verb+\apinforeq{+\meta{AP purpose}\verb+}+\ixcom{apinforeq} 
produces the boilerplate for the clause.

\begin{anexample}The command \verb|\apinforeq{AP purpose.}| prints: 

\apinforeq{\textit{AP purpose.}} 
\end{anexample} % end example

\ssclause{AP UoF}

    The command \verb|\uofhead|\ixcom{uofhead} prints the heading
for the UoF subclause.

    The environment 
\verb+\begin{apuof}+\meta{item list}\verb+\end{apuof}+\ixenv{apuof} 
produces the boilerplate for the introduction to the clause.

\begin{anexample} Remembering that \verb|\aptitle|\ixcom{aptitle}
                  was set to \texttt{\theap} in the preamble,
                  the commands
\begin{verbatim}
\begin{apuof}
\item Name of UoF1;
\item Name of UoF2;
\item Name of UoFn.
\end{apuof}
\end{verbatim}
prints:

\begin{apuof}
\item Name of UoF1;
\item Name of UoF2;
\item Name of UoFn.
\end{apuof}

\end{anexample}

\ssclause{AP application objects}

    The command \verb|\applobjhead|\ixcom{applobjhead} prints the
heading for the application objects subclause.

    The command \verb|\apapplobj|\ixcom{apapplobj} produces the 
boilerplate for the introduction to the clause.

\begin{anexample} Remembering that \verb|\aptitle|\ixcom{aptitle}
                  was set to \texttt{\theap} in the preamble,
                  the command \verb|\apapplobj| prints:

\apapplobj

\end{anexample}

\ssclause{AP assertions}

    The command \verb|\applasserthead|\ixcom{applasserthead} prints the
heading for the application assertions subclause.

    The command \verb|\apassert|\ixcom{apassert}
produces the boilerplate for the clause.

\begin{anexample} Remembering that \verb|\aptitle|\ixcom{aptitle}
                  was set to \texttt{\theap} in the preamble,
                  the command \verb|\apassert| prints:

\apassert

\end{anexample}


\ssclause{AP mapping table/specification}

    The command \verb|\mappinghead|\ixcom{mappinghead} prints
the heading for the mapping table or mapping specification subclause.
The heading text depends on whether or not 
\verb|\mapspectrue|\ixcom{mapspectrue} was put in the preamble.

    The command \verb|\apmapping|\ixcom{apmapping} 
produces the boilerplate for the introduction to the AP mapping table
or specification clause.

\begin{anote}AICs are included in the boilerplate only if the command
\verb|\aicinaptrue|\ixcom{aicinaptrue} is included
in the preamble.
\end{anote}

\begin{example}By default, or when \verb|\mapspecfalse| is in
the preamble, the command \verb|\apmapping|
         prints: \mapspecfalse

\apmapping
\end{example} % end example

\begin{example}When \verb|\mapspectrue| is in the preamble, the command \verb|\apmapping|
         prints: \mapspectrue

\apmapping
\end{example} % end example

\sssclause{AP mapping templates}

    The command \verb|\aptemplatehead|\ixcom{aptemplatehead} prints
the heading for the mapping template subclause (if any).

    The command \verb|\apmaptemplate|\ixcom{apmaptemplate} prints
the boilerplate for the introduction to the clause. This refers to the
UoFs in the AP. The first of the UoFs shall be labelled as 
\verb|\label{;uof1}| and the last of the UoFs shall be
labelled as \verb|\label{;uoflast}|.

\begin{example} If there are three UoFs, then there should be headings
of the form:
\begin{verbatim}
\mapuofhead{First UoF}\label{;uof1}
...
\mapuofhead{Second UoF}...
...
\mapuofhead{Third UoF}\label{;uoflast}
...
\end{verbatim}
\end{example}

\begin{example} Assuming that there are three UoFs as in the previous example, 
the command \verb|\apmaptemplate| prints:

\apmaptemplate
\end{example} % end example

    The command \verb|\sstemplates|\ixcom{sstemplates} prints
the two subclauses for the \xword{subtype} and \xword{SuPeRtype} templates.

\begin{example} \label{ex:sstemplates} In this document, 
and noting that the clause
numbering is not the same as in a real AP document, 
the command \verb|\sstemplates|
         prints:

\sstemplates

\end{example} % end example


\sssclause{Template headings}

    There are three headings used within a mapping template.

    The command \verb|\signature|\ixcom{signature} prints the underlined
Mapping signature header.

    The command \verb|\parameters|\ixcom{parameters} prints the underlined
Parameter definition header.

    The command \verb|\body|\ixcom{body} prints the underlined
Template body header.

\begin{anexample} The results of using the \verb|\signature|\ixcom{signature}
and \verb|\parameters|\ixcom{parameters} commands were illustrated
in \eref{ex:sstemplates} on \pref{ex:sstemplates}.
\end{anexample}



\ssclause{AIM short EXPRESS listing}

    The command \verb|\aimshortexphead|\ixcom{aimshortexphead} prints
the heading for the AIM EXPRESS short listing subclause.

    The command \verb|\apshortexpress|\ixcom{apshortexpress} produces 
the boilerplate for the
first paragraph of the clause.

\begin{anote}AICs are included in the boilerplate only if the command
\verb|\aicinaptrue|\ixcom{aicinaptrue} is included in the preamble.
\end{anote}

\begin{example}
The command \verb|\apshortexpress| without \verb|\aicinaptrue|
in the preamble produces:

\aicinapfalse
\apshortexpress

\end{example} % end example

\begin{example}
With \verb|\aicinaptrue| set in the preamble the command
\verb|\apshortexpress| produces the following:

\aicinaptrue
\apshortexpress
\end{example} % end example


\ssclause{AP conformance}

    The command \verb|\confreqhead|\ixcom{confreqhead} prints the
heading for the conformance requirements clause.

    The command 
\verb+\apconformance{+\meta{implementation methods}\verb+}+\ixcom{apconformance} 
produces the boilerplate for the introduction to the clause.

    The environment 
\verb+\begin{apconformclasses}+\meta{item list}\verb+\end{apconformclasses}+\ixenv{apconformclasses} 
provides some additional boilerplate.

\begin{example}The command \verb|\apconformance{ISO 10303-21, ISO 10303-22}|
         prints:

\apconformance{\textit{ISO 10303-21, ISO 10303-22}} 
\end{example} % end example

\begin{example}The commands
  \begin{verbatim}
\begin{apconformclasses}
\item first class;
\item second class;
\item last class.
\end{apconformclasses}
\end{verbatim}
         print:

\begin{apconformclasses}
\item first class;
\item second class;
\item last class.
\end{apconformclasses}
\end{example}


\ssclause{EXPRESS expanded listing}

    The command \verb|\aimlongexphead|\ixcom{aimlongexphead} prints
the heading for the AIM expanded listing clause.

    The command \verb|\aimlongexp|\ixcom{aimlongexp} 
produces the boilerplate for the introduction to the clause.

\begin{anexample}The command \verb|\aimlongexp|
         prints:

\aimlongexp 
\end{anexample} % end example

\ssclause{AIM short names}

    The command \verb|\aimshortnamehead|\ixcom{aimshortnamehead} prints
the heading for the AIM short names annex.

    The command \verb|\apshortnames|\ixcom{apshortnames} 
produces the boilerplate for the introduction to the AP short name annex.

\begin{anexample}The command \verb|\apshortnames|
         prints:

\apshortnames 
\end{anexample} % end example

\ssclause{Implementation requirements}

    the command \verb|\impreqhead|\ixcom{impreqhead} prints the heading
for implementation method-specific reguirements.

    The command \verb+\apimpreq{+\meta{schema name}\verb+}+\ixcom{apimpreq}
produces the boilerplate for the requirements on exchange structure.

\begin{anexample}The command \verb|\apimpreq{schema\_name}|
         prints:

\apimpreq{\textit{schema\_name}} 
\end{anexample} % end example


\ssclause{AP PICS}

    The command \verb|\picshead|\ixcom{picshead}, 
from the \file{step}\ixpack{step} package,
prints the heading for the PICS annex.

    The command \verb|\picsannex|\ixcom{picsannex}
produces the boilerplate for the start of the AP PICS annex.

\begin{anexample}The command \verb|\picsannex|
         prints:

\picsannex 
\end{anexample} % end example

\ssclause{AAM annex}

    The command \verb|\aamhead|\ixcom{aamhead} prints the heading for
the AAM annex.


    The command \verb|\apaamintro|\ixcom{apaamintro} 
 produces the introductory boilerplate for the introduction of
the AP annex on application activity models.

\begin{anexample}
  The command \verb|\apaamintro| prints:

\apaamintro

\end{anexample} % end example

\ssclause{AP AAM definitions}

    The command \verb|\aamdefhead|\ixcom{aamdefhead} prints the heading
for the AAM definitions subclause.

    The command \verb|\apaamdefs|\ixcom{apaamdefs} produces 
the boilerplate at the start of
the AP subclause on AAM definitions and abbreviations.

\begin{anexample}
  The command \verb|\apaamdefs| prints:

\apaamdefs
\end{anexample} % end example

\ssclause{AAM diagrams annex}

    The command \verb|\aamfighead|\ixcom{aamfighead} prints the heading
for the AAM diagrams subclause.

    The command 
\verb|\aamfigrange{|\meta{figure range}\verb|}|\ixcom{aamfigrange} 
is used to store the activity model diagram figure range for later use.

\begin{example}
    For the purposes of this document we set
\begin{verbatim}
\aamfigrange{figures F.1 through F.n}
\end{verbatim}

\aamfigrange{\textit{figures F.1 through F.n}}

\end{example}

    The command \verb+\aamfigures+\ixcom{aamfigures}
produces the boilerplate for the introduction to an APs AAM figure
subclause.

\begin{example} Noting that we have set 
\verb|\aamfigrange{figures F.1 through F.n}|\ixcom{aamfigrange}, 
the command \verb|\aamfigures| prints:

\aamfigures

\end{example}

\ssclause{ARM annex}

    The command \verb|\armhead|\ixcom{armhead} prints the heading for the
ARM annex.

    The command 
\verb+\armintro+\ixcom{armintro}
produces the boilerplate for the introduction to the ARM figures.

\begin{anexample}The command 
          \verb|\armintro| 
         prints:

\armintro 
\end{anexample} % end example

\ssclause{AIM EXPRESS-G annex}

    The command \verb|\aimexpressghead|\ixcom{aimexpressghead} 
prints the heading for the AIM \ExpressG{} annex.
 

    The command 
\verb+\aimexpressg+\ixcom{aimexpressg}
produces the boilerplate for the introduction to an AP's AIM \ExpressG{}
model. 

\begin{anexample}The command \verb|\aimexpressg|
         prints:

\aimexpressg
\end{anexample} % end example

\ssclause{AIM EXPRESS listing}

    The command \verb|\aimexpresshead|\ixcom{aimexpresshead} prints
the heading for the AIM listing annex.

%    The command \verb|\aimexplisting|\ixcom{aimexplisting}
%produces the boilerplate for the introduction to an AIMs short name and
%\Express{} listing.
%
%
%\begin{example}The command \verb|\aimexplisting|
%         prints:
%
%\aimexplisting 
%\end{example}

    The command 
\verb|\apexpurls{|\meta{short}\verb|}{|\meta{express}\verb|]|\ixcom{apexpurls}
produces the boilerplate for the introduction to the AP annex
listing short names and \Express, where \meta{short} is the URL for the short
names and \meta{express} is the URL for the \Express.

\begin{anexample} The command \verb|\apexpurls{http:/www.short/}{http://www.express/}|
prints:

\apexpurls{http://www.short/}{http://www.express/}

\end{anexample}

%%%%%%%%%%%%%%%%%%%%%%%%%%%%%%
%%%%%%\end{document}
%%%%%%%%%%%%%%%%%%%%%%%%%%%%%%



\clearpage
\clause{The \file{aic} package facility}

    The \file{aic}\ixpack{aic} package
provides commands and environments specifically
for the ISO~10303 Application Interpreted Construct series of
documents.

    The use of this package requires the use of the 
\file{step}\ixpack{step} package.

\sclause{Heading commands}

    The commands described in this subclause start document clauses with
particular titles.

    The commands that take no parameters are listed in \tref{tab:aicnpheads}.
\ixcom{aicshortexphead}

\begin{table}[btp]
\centering
\caption{AIC package parameterless heading commands}
\label{tab:aicnpheads}
\begin{tabular}{|l|c|l|l|} \hline
\textbf{Command} & \textbf{Clause} & \textbf{Default text} & \textbf{Label} \\ \hline
\verb|\aicshortexphead| & C & \Express{} short listing & \verb|;sesl| \\
\hline
\end{tabular}
\end{table}

\sclause{Boilerplate commands}

    The following commands produce boilerplate text as specified by the
Supplementary Directives. 


\ssclause{Introduction text}

    The command \verb|\aicextraintro|\ixcom{aicextraintro}
prints additional boilerplate for the Introduction to an AIC.

\begin{anexample}The command \verb|\aicextraintro|
         prints:

\aicextraintro
\end{anexample}

\ssclause{Definition of AIC}

    The command \verb|\aicdef|\ixcom{aicdef}
prints the definition of `AIC'. It shall only be used within the
\verb|definitions|\ixenv{definitions} environment.

\begin{anexample}The commands:
         \begin{verbatim}
         \begin{definitions}
         \aicdef
         \end{definitions}
         \end{verbatim}
         produce:

\begin{definitions}
\aicdef
\end{definitions}
\end{anexample} % end example

\ssclause{Short EXPRESS listing}

    The command \verb|\aicshortexphead|\ixcom{aicshortexphead} prints
the heading for the AIC short \Express{} annex.

    The command \verb|\aicshortexpintro|\ixcom{aicshortexpintro}
prints boilerplate for the introduction to the short \Express{} listing.

\begin{anexample}The command \verb|\aicshortexpintro|
         prints:

\aicshortexpintro  
\end{anexample} % end example

\ssclause{EXPRESS-G figures}

    The command \verb|\expressghead|\ixcom{expressghead}, 
from the \file{step} package, prints the heading for the \ExpressG{} diagrams
annex.

    The command 
\verb+\aicexpressg+\ixcom{aicexpressg} 
prints boilerplate for the introduction to the \ExpressG\ figures.

\begin{anexample}The command \verb|\aicexpressg|
         prints:

\aicexpressg
\end{anexample}

%%%%%%%%%%%%%%%%%%%%%%%%%
%%%\end{document}
%%%%%%%%%%%%%%%%%%%%%%%%%

\clearpage
\clause{The \file{ats} package facility}

    The \file{ats}\ixpack{ats} package
provides commands and environments specifically
for the ISO~10303 Abstract Test Suite series of
documents.

    The use of this package requires the use of the 
\file{step}\ixpack{step} package.

\sclause{Preamble commands}

    Certain commands shall be put in the preamble\index{preamble} 
of an ATS document.

    The command 
\verb+\APnumber{+\meta{number}\verb+}+\ixcom{APnumber} shall be put 
in the preamble,
where \meta{number} is the ISO 10303 part number of the corresponding AP.

\begin{example}
For the purposes of later examples, the command
\verb+\APnumber{+\texttt{\theAPpartno}\verb+}+ has been put in the preamble.
of this document.
\end{example}

    The command 
\verb+\APtitle{+\meta{title of AP}\verb+}+\ixcom{APtitle} shall be put 
in the preamble,
where \meta{title of AP} is the ISO 10303 part title of the
corresponding AP. This must be given in such a manner that it reads
sensibly in a sentence of the form `\ldots for ISO 10303-299,
application protocol \meta{title of AP}.'

\begin{example}
For the purposes of later examples, the command
\verb+\APtitle{+\texttt{\theAPtitle}\verb+}+ 
has been put in the preamble of this document.
\end{example}

    The command
\verb+\mapspectrue+\ixcom{mapspectrue}
shall be put in the preamble if the AP uses a mapping specification rather
than a mapping table.

\sclause{Heading commands}

    These commands start document clauses with particular titles.
The commands that take no parameters are listed in \tref{tab:atshead}.
\ixcom{purposeshead}
\ixcom{domainpurposehead}
\ixcom{aepurposehead}
\ixcom{apobjhead}
\ixcom{apasserthead}
\ixcom{aimpurposehead}
%%%\ixcom{extrefpurposehead}
\ixcom{implementpurposehead}
%%%\ixcom{rulepurposehead}
\ixcom{otherpurposehead}
\ixcom{gtpvchead}
\ixcom{generalpurposehead}
\ixcom{gvcatchead}
\ixcom{gvcprehead}
\ixcom{gvcposthead}
\ixcom{atchead}
\ixcom{prehead}
\ixcom{posthead}
\ixcom{confclassannexhead}
\ixcom{postipfilehead}
%%%\ixcom{excludepurposehead}
\ixcom{atsusagehead}

\settowidth{\prwlen}{\quad General verdict criteria for all abstract}
\begin{table}
\centering
\caption{ATS package parameterless heading commands} \label{tab:atshead}
\begin{tabular}{|l|c|p{\prwlen}|} \hline
\textbf{Command}             & \textbf{Clause} & \textbf{Default text} \\ \hline
\verb|\purposeshead|         & C   & Test purposes  \\
\verb|\aepurposehead|        & SC  & Application element test purposes \\
\verb|\aimpurposehead|       & SC  & AIM test purposes \\
\verb|\implementpurposehead| & SC  & Implementation method test purposes \\
\verb|\domainpurposehead|    & SC  & Domain test purposes \\
\verb|\otherpurposehead|     & SC  & Other test purposes \\

\verb|\gtpvchead|            & C   & General test purposes and verdict criteria \\
\verb|\generalpurposehead|   & SC  & General test purposes \\
\verb|\gvcatchead|           & SC  & General verdict criteria for all abstract test cases \\
\verb|\gvcprehead|           & SC  & General verdict criteria for preprocessor abstract test cases \\
\verb|\gvcposthead|          & SC  & General verdict criteria for postprocessor abstract test cases \\

\verb|\atchead|              & C   & Abstract test cases \\
\verb|\prehead|              & SSC & Preprocessor \\
\verb|\precoveredhead|       & SSSC & Test purposes covered \\
\verb|\preinputhead|         & SSSC & Input specification \\
\verb|\precriteriahead|      & SSSC & Verdict criteria \\
\verb|\preconstraintshead|   & SSSC & Constraints on values \\
\verb|\preexechead|          & SSSC & Execution sequence \\
\verb|\preextrahead|         & SSSC & Extra details \\


\verb|\posthead|             & SSC & Postprocessor \\
\verb|\postcoveredhead|       & SSSC & Test purposes coverage \\
\verb|\postinputhead|         & SSSC & Input specification \\
\verb|\postcriteriahead|      & SSSC & Verdict criteria \\
\verb|\postexechead|          & SSSC & Execution sequence \\
\verb|\postextrahead|         & SSSC & Extra details \\

\verb|\confclassannexhead|   & NA  & Conformance classes \\
\verb|\postipfilehead|       & NA  & Postprocessor input specification file names \\

\verb|\atsusagehead|         & IA  & Usage scenarios \\

\verb|\apasserthead|         & SSC & Application assertions \\
%%%\verb|\extrefpurposehead|    & SC  & External reference test purposes \\
%%%%\verb|\rulepurposehead|      & SC  & \rulepurposename\  \\
%%%\verb|\excludepurposehead|   & NA  & Excluded test purposes \\ 
\hline
\end{tabular}
\end{table}


    The commands that take a parameter are listed in \tref{tab:atsphead}.
\ixcom{apobjhead}
\ixcom{aimenthead}
\ixcom{atctitlehead}
\ixcom{confclasshead}

\begin{table}
\centering
\caption{ATS package parameterized heading commands} \label{tab:atsphead}
\begin{tabular}{|l|c|l|} \hline
Command               & Clause & Parameterized title \\ \hline
\verb|\apobjhead|     & SSC & \meta{Application object n}  \\
\verb|\aimenthead|    & SSC & \meta{Entity name} \\
\verb|\atctitlehead|  & SC  & \meta{Title} \\
\verb|\confclasshead| & SC  & Conformance class \meta{number}  \\ \hline
\end{tabular}
\end{table}


\sclause{Keyword commands}

    Several keyword (headings) are used in documenting a test case.
\latex{} commands for these keywords are given in \tref{tab:atskey}.
\ixcom{atssummary}
\ixcom{atscovered}
\ixcom{atsinput}
\ixcom{atsconstraints}
\ixcom{atsverdict}
\ixcom{atsexecution}
\ixcom{atsextra}

\begin{table}
\centering
\caption{ATS package keyword commands} \label{tab:atskey}
\begin{tabular}{|l|l|} \hline
Command                & Effect \\ \hline
\verb|\atssummary|     & \atssummary{} \\
\verb|\atscovered|     & \atscovered{} \\
\verb|\atsinput|       & \atsinput{} \\
\verb|\atsconstraints| & \atsconstraints{} \\
\verb|\atsverdict|     & \atsverdict{} \\
\verb|\atsexecution|   & \atsexecution{} \\
\verb|\atsextra|       & \atsextra{} \\ \hline
\end{tabular}
\end{table}

\sclause{Boilerplate commands}

    The following commands produce boilerplate text.

\begin{anote}
 In the examples, the
parameters of those commands that take them have been specified in
\textit{this font style} so that their
effects can be seen in the printed text.
\end{anote}

\ssclause{ATS introduction}

    The command 
\verb|\atsintroendbp|\ixcom{atsintroendbp}
 produces the boilerplate
for the end of the Introduction to an ATS.

\begin{anexample} Remembering that in the preamble 
        \verb|\APnumber|\ixcom{APnumber} was set to \texttt{\theAPpartno} 
        and \verb|\APtitle|\ixcom{APtitle} was set to \texttt{\theAPtitle},
the command \verb|\atsintroendbp| prints:

\atsintroendbp

\end{anexample}


\ssclause{ATS scope}

    The command \verb|\scopeclause|\ixcom{scopeclause}, from the \file{isov2}
class, prints the heading for the Scope clause.

    The command 
\verb|\atsscopebp|\ixcom{atsscopebp}
produces boilerplate for an ATS \textit{Scope}
clause.

\begin{anexample}  Remembering that in the preamble 
        \verb|\APnumber|\ixcom{APnumber} was set to \texttt{\theAPpartno}, 
the command \verb|\atsscopebp| prints:

\atsscopebp

\end{anexample}

\ssclause{Test purpose}

    The command \verb|\purposehead|\ixcom{purposehead} prints the heading
for the test purposes clause.

    The command \verb|\atspurposebp|\ixcom{atspurposebp} 
prints boilerplate for the introduction to the clause.

\begin{anexample}  Remembering that in the preamble 
        \verb|\APnumber|\ixcom{APnumber} was set to \texttt{\theAPpartno}, 
the command \verb|\atspurposebp| prints:

\atspurposebp

\end{anexample}

\ssclause{Application element test purposes}

    The command \verb|\aepurposehead|\ixcom{aepurposehead} prints the
heading for the application element test purposes subclause.

    The command 
\verb|\aetpbp|\ixcom{aetpbp}
prints boilerplate for the clause.

\begin{anexample} Remembering that in the preamble 
        \verb|\APnumber|\ixcom{APnumber} was set to \texttt{\theAPpartno}, 
the command \verb|\aetpbp| prints:

\aetpbp

\end{anexample}

\ssclause{AIM test purposes}

    The command \verb|\aimpurposehead|\ixcom{aimpurposehead} prints the
heading for the AIM test purposes subclause.

    The command 
\verb|\aimtpbp|\ixcom{aimtpbp}
prints boilerplate for the clause.

\begin{anexample} Remembering that in the preamble 
        \verb|\APnumber|\ixcom{APnumber} was set to \texttt{\theAPpartno}, 
the command \verb|\aimtpbp| prints:

\aimtpbp

\end{anexample}

\ssclause{Implementation method test purposes}

    The command \verb|\implementpurposehead|\ixcom{implementpurposehead} prints the
heading for the implementation method test purposes subclause.

    The command 
\verb|\atsimtpbp|\ixcom{atsimtpbp}
prints boilerplate for the clause.

\begin{anexample} Remembering that in the preamble 
        \verb|\APnumber|\ixcom{APnumber} was set to \texttt{\theAPpartno}, 
the command \verb|\atsimtpbp| prints:

\atsimtpbp

\end{anexample}



\ssclause{General test purposes and verdict criteria}

    The command \verb|\gtpvchead|\ixcom{gtpvchead} prints the heading
for the general test purposes and verdict criteria clause.

    The command 
\verb|\atsgtpvcbp|\ixcom{atsgtpvcbp} 
prints boilerplate for the clause


\begin{anexample} The command \verb|\atsgtpvcbp| prints:

\atsgtpvcbp
\end{anexample}

\ssclause{General test purposes}

    The command \verb|\generalpurposehead|\ixcom{generalpurposehead} prints
the heading for the general test purposes subclause.

    The command 
\verb|\gtpbp|\ixcom{gtpbp}
prints boilerplate for the suclause.

\begin{anexample} Remembering that in the preamble 
        \verb|\APnumber|\ixcom{APnumber} was set to \texttt{\theAPpartno}, 
the command \verb|\gtpbp| prints:

\gtpbp

\end{anexample}

\ssclause{General verdict criteria}

    The command \verb|\gvcatchead|\ixcom{gvcatchead} prints the
heading for the general verdict criteria for all cases subclause.

    The command 
\verb|\gvatcbp|\ixcom{gvatcbp}
prints boilerplate for the subclause.

\begin{anexample} Remembering that in the preamble 
        \verb|\APnumber|\ixcom{APnumber} was set to \texttt{\theAPpartno}, 
the command \verb|\gvatcbp| prints:

\gvatcbp

\end{anexample}

\ssclause{General verdict criteria for preprocessor}

    The command \verb|\gvcprehead|\ixcom{gvcprehead} prints the
heading for the general verdict criteria for preprocessor cases subclause.

    The command 
\verb|\gvcprebp|\ixcom{gvcprebp} 
prints boilerplate for the subclause.

\begin{anexample} Remembering that in the preamble 
        \verb|\APnumber|\ixcom{APnumber} was set to \texttt{\theAPpartno}, 
the command \verb|\gvcprebp| prints:

\gvcprebp

\end{anexample}

\ssclause{General verdict criteria for postprocessor}


    The command \verb|\gvcposthead|\ixcom{gvcposthead} prints the
heading for the general verdict criteria for postprocessor cases subclause.

    The command 
\verb|\gvcpostbp|\ixcom{gvcpostbp} 
prints boilerplate for the subclause.

\begin{anexample} Remembering that in the preamble 
        \verb|\APnumber|\ixcom{APnumber} was set to \texttt{\theAPpartno}, 
the command \verb|\gvcpostbp| prints:

\gvcpostbp

\end{anexample}

\ssclause{Abstract test cases}

    The command \verb|\atchead|\ixcom{atchead} prints the heading
for the abstract test cases clause.

    The command 
\verb|\atcbp|\ixcom{atcbp} 
prints the first paragraph of the boilerplate for the clause.

\begin{example} The command \verb|\atcbp| prints:

\atcbp
\end{example}

    The command
\verb|\atcbpii|\ixcom{atcbpii}
prints paragraphs~3 and onwards of the boilerplate.

\begin{example} The command \verb|\atcbpii|
prints:

\atcbpii

\end{example}

\ssclause{Preprocessor}

    The command \verb|\prehead|\ixcom{prehead} prints the title
for the preprocessor subsubclause.

    The command
\verb|\atcpretpc|\ixcom{atcpretpc}
prints boilerplate for the subclause.

\begin{anexample} The command \verb|\atcpretpc| prints:

\atcpretpc
\end{anexample}

\ssclause{Postprocessor}

    The command \verb|\posthead|\ixcom{posthead} prints the title
for the postrocessor subsubclause.

    The command
\verb|\atcposttpc|\ixcom{atcposttpc}
prints boilerplate for the subclause.

\begin{anexample} The command \verb|\atcposttpc| prints:

\atcposttpc
\end{anexample}



\ssclause{Conformance class}

    The command \verb|\confclassannexhead|\ixcom{confclassannexhead}
prints the heading for the conformance classes annex heading.

    The command 
\verb|\atsnoclassesbp|\ixcom{atsnoclassesbp} 
prints the entire boilerplate for the
\textit{Conformance class} annex when the AP has no conformance classes.

\begin{example} Remembering that in the preamble 
        \verb|\APnumber|\ixcom{APnumber} was set to \texttt{\theAPpartno}, 
the command \verb|\atsnoclassesbp| prints:

\atsnoclassesbp

\end{example}

    The command \verb|\confclasshead{|\meta{number}\verb|}|\ixcom{confclasshead}
prints the heading for a conformance class \meta{number} subclause.

    The command 
\verb|\confclassbp{|\meta{number}\verb|}|\ixcom{confclassbp}
prints the
boilerplate for the introduction to a conformance class subclause, where
\meta{number} is the number of the conformance class.

\begin{example} Remembering that in the preamble 
        \verb|\APnumber|\ixcom{APnumber} was set to \texttt{\theAPpartno}, 
the command \verb|\confclassbp{27}| prints:

\confclassbp{\textit{27}}

\end{example}

\ssclause{Postprocessor input specification file names}

    The command \verb|\postipfilehead|\ixcom{postipfilehead} prints
the heading for the postprocessor input file names annex.

    The command 
\verb|\pisfbp{|\meta{12 or 21}\verb|}{|\meta{url}\verb|}{|\meta{ref}\verb|}|\ixcom{pisfbp} 
prints the boilerplate for the annex.

\begin{anexample} The command 
\verb|\pisfbp{12}{http://www.mel.nist.gov/step/parts/parts3456/wd}{\ref{TabB1}}| prints:

\pisfbp{12}{http://www.mel.nist.gov/step/parts/part3456/wd}{\ref{tabB1}}

\end{anexample}

%%%%%%%%%%%%%%%%%%%%%%%%%%%
%%%\end{document}
%%%%%%%%%%%%%%%%%%%%%%%%%%%

\normannex{Additional commands} \label{anx:extraiso}

\sclause{Internal commands}

    The code implementing the various facilities includes many commands
not described in the body of this document. Any command that includes
the commercial at sign (\verb|@|) in its name shall not be used by any author;
the implementer of the package code reserves the right to modify or delete
these at any time without giving any notice.

   Internal commands that have names consisting only of letters may be
used in a document at the author's own risk. These may be changed, but 
if so notification will be given.

\sclause{Boilerplate}

    Much of the boilerplate text is maintained in separate \file{.tex}
files and many of the commands that generate boilerplate merely 
\verb|\input|
the appropriate file.




%%%%%%%%%%%%%%%%%%%%%%%%%%
%%%\end{document}
%%%%%%%%%%%%%%%%%%%%%%%%%%

\normannex{Ordering of LaTeX commands} \label{anx:lord}

    The \latex{} commands to produce an ISO~10303 document are:
\begin{verbatim}
\documentclass[<options>]{isov2}
\usepackage{stepv13}                                % required package
\usepackage{irv12}                                  % for an IR document
\usepackage{apv12}                                  % for an AP document
\usepackage{aicv1}                                  % for an AIC document
\usepackage{atsv11}                                 % for an ATS document
\usepackage[<options>]{<name>}                      % additional packages
\standard{<standard identifier>}
\yearofedition{<year>}
\languageofedition{<parenthesized code letter>}
\partno{<part number>}
\series{<series title>}
\doctitle{<title on cover page>}
\ballotcycle{<number>}
\aptitle{<title of AP>}  % if doc is an AP
\aicinaptrue             % if doc is an AP that uses AICs
\mapspectrue             % if doc is an AP that uses mapping spec.
\APnumber{<number>}      % if doc is an ATS
\APtitle{<title>}        % if doc is an ATS
\mapspectrue             % if doc is an ATS and AP uses mapping spec.
  % other preamble commands
\begin{document}
\STEPcover{< title commands >}
\Foreword                            % start Foreword & ISO boilerplate
  \fwdshortlist                      % STEP boilerplate
\endForeword{<param1>}{<param2>}     % end Foreword & boilerplate
\begin{Introduction}                 % start Introduction & boilerplate
  \aicextraintro             % extra boilerplate for an AIC
  \apextraintro                % extra boilerplate for an AP
  % your text 
\end{Introduction}
\stepparttitle{<Part title>}
\scopeclause                         % Clause 1: Scope clause
  \apscope{<AP purpose>}             % boilerplate if an AP
   % text of scope
\normrefsclause                      % Clause 2: Normative references
  \normrefbp{<document type>}        % boilerplate
  \begin{nreferences}
    % \isref{}{} and/or \disref{}{} list of normative references
  \end{nreferences}
\defclause                           % definitions clause
  \partidefhead                      % defs from Part1 subclause
    % olddefinition list
  \refdefhead{<ISO 10303-NN>}        % defs from Part NN subclause
    % olddefinition list
  \otherdefhead                      % defs in this part
    % definition list
\symabbclause                        % Symbols & abbreviations clause
  % symbol lists
% THE BODY OF THE DOCUMENT
\bibannex                            % optional; the final Bibliography 
  % bibliography listing
% the index
\end{document}
\end{verbatim}


\sclause{Body of a resource document} \index{integrated resource}

    The body of a resource document has the following structure:

\begin{verbatim}
\schemahead{<Schema name>}         % repeat for each schema
  \introsubhead                    % intro subclause
     % text
  \fcandasubhead                   % concepts subclause
     % text
  \typehead{<Schema>}              % if type defs
     \atypehead{<type>}            % type heading     
  \entityhead{<Schema>}{<group>}   % if entity defs
     \anentityhead{<entity>}       % entity heading
  \rulehead{<Schema>}              % if rule defs
     \arulehead{<rule>}            % rule heading
  \functionhead{<Schema>}          % if function defs
     \afunctionhead{<function>}    % function heading
% repeat above for each schema
\shortnamehead                     % Annex A: Short names of entities
  \irshortnames                    % boilerplate
  % list of short names
\objreghead                        % Annex B: Information object registration
  \docidhead                       % Document identification subclause
    \docreg{<param1>}                                   % boilerplate
  \schemaidhead                    % Schema identification subclause
% Either (for single schema)
     \schemareg{<6 parameters>}    % boilerplate
% Or (for multiple schemas) repeat:
     \aschemaidhead{<schema name>} % Schema id subsubclause
       \schemareg{<6 parameters>}
\listingshead                       % Annex C: Computer interpretable listings
  \expurls{<short>}{<express>}      % boilerplate
\expressghead                       % Annex D: EXPRESS-G figures
  \irexpressg                       % boilerplate
  %  EXPRESS-G diagrams
\techdischead                       % optional Technical discussions
  % text
\exampleshead                       % optional Examples
  % text
\end{verbatim}


\sclause{Body of an application protocol} \index{AP}

    The body of an AP document has the following structure:

\begin{verbatim}
\inforeqhead                   % Clause 4: Information requirements
  \apinforeq{<param1>}         % boilerplate
  \uofhead                     % Clause 4.1: Units of functionality
    \begin{apuof}              % boilerplate
      % \item list of UoFs
    \end{apuof}
    \auofhead{<UoF1>}          % repeat for each UoF
      % text
    \applobjhead               % Clause 4.2: Application objects
      \apapplobj               % boilerplate
        % text
    \applasserthead            % Clause 4.3: Application assertions
      \apassert                % boilerplate
        % text
\aimhead                       % Clause 5: Application interpreted model
  \maptablehead                % Clause 5.1: Mapping table/specification
    \apmapping                 % boilerplate
    \maptemplatehead           % if mapping templates used
      \apmaptemplate           % template boilerplate
      \sstemplates             % sup/sub templates
      \templatehead
        % text
     \mapuofhead{<Uof>}        % mapping for <UoF>
       \mapobjecthead{<object>}
         % mapping for <object>
         \mapattributehead{<attr>}
           % mapping for <attr>
  \aimshortexphead             % Clause 5.2: AIM EXPRESS short listing
    \apshortexpress            % boilerplate
      % text
\confreqhead                   % Clause 6: Conformance requirements
  \apconformance{<param1>}     % boilerplate
  \begin{apconformclasses}     % optional boilerplate
    % \item list
  \end{apconformclasses}
     % text
\aimlongexphead                % Annex A: AIM EXPRESS expanded listing
  \aimlongexp                  % boilerplate
     % text
\aimshortnameshead             % Annex B: AIM short names
  \apshortnames                % boilerplate
     % text
\impreqhead                    % Annex C: Impl. specific reqs
  \apimpreq{<schema name>}     % boilerplate
\picshead                      % Annex D: PICS
  \picsannex                   % boilerplate
     % text
\objreghead                    % Annex E: Information object registration
  \docidhead                   % Annex E.1: Document identification
    \docreg{<param1>}          % boilerplate
  \schemaidhead                % Annex E.2: Schema identification
    \apschemareg{<6 params>}   % boilerplate
\aamhead                       % Annex F: Application activity model
  \aamfigrange{<figure range>} % Figure range for AAM diagrams
  \apaamintro                  % boilerplate
      % text
  \aamdefhead                  % Annex F.1: AAM defs and abbreviations
    \apaamdefs                 % boilerplate
       % text
  \aamfighead                  % Annex F.2: AAM diagrams
    \aamfigures                % boilerplate
       % IDEF0 diagrams
\armhead                       % Annex G: Application reference model
   \armintro                   % boilerplate
     % ARM figures
\aimexpressghead               % Annex H: AIM EXPRESS-G
  \aimexpressg                 % boilerplate
     % AIM figures
\listingshead                  % Annex J: Computer interpretable listings
  \apexpurls{<short>}{<express>}   % boilerplate
\apusagehead                   % optional Annex: AP usage
   % text
\techdischead                  % optional Annex: Technical discussions
   % text
\end{verbatim}

\sclause{Body of an AIC} \index{AIC}

    The body of an AIC document has the following structure:

\begin{verbatim}
\aicshortexphead               % Clause 4: EXPRESS short listing
  \aicshortexpintro            % boilerplate
  \fcandasubhead               % Clause 4.1 fundamental concepts
    % text
  \typehead{<Schema>}          % if type definitions
     \atypehead{<type>}        % repeat for each type
  \entityhead{<Schema>}{}      % if entity defs
     \anentityhead{<entity>}   % repeat for each entity
  \functionhead{<Schema>}      % if function defs
     \afunctionhead{<function>} % repeat for each function
\shortnamehead                 % Annex A: Short names of entities
  \shortnames                  % boilerplate
\objreghead                    % Annex B: Information object registration
  \docidhead                   % Annex B.1: Document identification
    \docreg{<version no>}      % boilerplate
  \schemaidhead                % Annex B.2: Schema identification
    \schemareg{<6 parameters>} % boilerplate
\expressghead               % Annex C: EXPRESS-G diagrams
  \aicexpressg               % boilerplate
\listingshead                  % Annex D: Computer interpretable listings
  \expurls                     % boilerplate
\techdischead                  % optional Annex: Technical discussions
\end{verbatim}

\sclause{Body of an ATS document}\index{ATS}

    The body of an Abstract Test Suite
document has the following structure:

\begin{verbatim}
\purposeshead                  % Clause 4: Test purposes
  \atspurposebp    % boilerplate
  \aepurposehead               % 4.1 Application element test purposes
    \aetpbp                    % boilerplate
    \apobjhead{<object>}       % 4.1.n
    ...
  \aimpurposehead              % 4.2 AIM test purposes
    \aimtpbp                   % boilerplate
    \aimenthead{<entity>}      % 4.2.n
    ...
  \implementpurposehead        % (optional) 4.3 Implementation t.p
    \atsimtpbp                 % boilerplate
    % text
  \domainpurposehead           % (optional) 4.2+ Domain test purposes
    % text
  \otherpurposehead            % (optional) 4.2+ Other test purposes
    % text
\gtpvchead                     % Clause 5: General t.p and verdict criteria
  \atsgtpvcbp                  % boilerplate
  \generalpurposehead          % 5.1 General test purposes
    \gtpbp                     % boilerplate
    ...
  \gvcatchead                  % 5.2 General verdict criteria for all ATC
    \gvatcbp                   % boilerplate
    ...
  \gvcprehead                  % 5.3 General verdict criteria for preprocessor
    \gvcprebp                  % boilerplate
     ...
  \gvcposthead                 % 5.4 General verdict criteria for postprocessor
    \gvcpostbp                 % boilerplate
    ...
\atchead                       % Clause 6: Abstract test cases
  \atcbp                       % boilerplate (para 1)
  % your para 2
  \atcbpii                     % boilerplate (paras 3+)
  \atctitlehead{<title>}       % 6.n an abstract test case
    \prehead                   % 6.n.1 Preprocessor
      \precoveredhead..        % Test purposes covered
        \atcpretpc             % boilerplate
      \preinputhead            % Input specification
        % text
      \precriteriahead         % Verdict criteria
        % text
      \preconstrainthead       % Constraints on values
        % text
      \preexechead             % (optional) Execution sequence
        % text
      \preextrahead            % (optional) Extra details
        % text
    \posthead                  % 6.n.2 Postprocessor
      \postcoveredhead         % Test purposes covered
        % text
        \atcposttpc            % boilerplate
      \postinputhead           % Input specification
        % text
      \postcriteriahead        % Verdict criteria
        % text
      \postexechead            % (optional) Execution sequence
        % text
      \postextra               % (optional) Extra details
        % text
\confclassannexhead            % Annex A: Conformance classes
  \atsnoclassesbp              % boilerplate if no conformance classes, else
  \confclasshead{<number>}     % A.n Conformance class <number>
    \confclassbp{<number>}     % boilerplate
    % text
  ...
\postipfilehead                % Annex B: Postprocessor input file names
  \pisfbp{..}{..}{..}                      % boilerplate
   ...
\objreghead                    % Annex C: Information object registration
  \docreg{<partno>}            % registration boilerplate
\atsusagehead                  % Annex D: Usage scenarios
  % text
\end{verbatim}



%%%%%%%%%%%%%%%%%%%%%%%%%%%%%%%%%%%%%%%
% object registration annex
\objreghead

\docreg{-1}

%%%%%%%%%%%%%%%%%%%%%%%%%%%%%%%%%%%%%%

%%%%%%%%%%%%%%%%%%%%%%%%%
%%%\end{document}
%%%%%%%%%%%%%%%%%%%%%%%%%

\infannex{Deprecated, deleted, new and modified commands}

    This release has involved many internal changes to the \latex{}
\file{.sty} files. In particular boilerplate text is, as far as possible,
maintained in external \file{.tex} files in order to save memory
space within the \latex{} processor. 


%%%%%%%%%%%%%%%%%%%%%%%%%
%%%\end{document}
%%%%%%%%%%%%%%%%%%%%%%%%%

\sclause{New commands}

    The commands that are new in this release are:

\begin{itemize}
%%%%%%%%%%%%%%%%%%%%%%%%%% STEP %%%%%%%%%%%%%%%%%%%%%%%%%%%
\item \verb|\bibieeeidefo|\ixcom{bibieeeidefo} STEP: reference to IDEF0 document;
\item \verb|\exampleshead|\ixcom{exampleshead} STEP: clause heading;
\item \verb|\expressgdef|\ixcom{expressgdef} STEP: location of \ExpressG{} definition;

\item \verb|\Theseries|\ixcom{Theseries} STEP: print \verb|\series| argument;
\item \verb|\theseries|\ixcom{theseries} STEP: print \verb|\series| argument in lowercase;
\item \verb|\ifanir|,\ixcom{ifanir} 
      \verb|\anirtrue|,\ixcom{anirtrue}
      \verb|\anirfalse|\ixcom{anirfalse} STEP: flag for an IR document;
\item \verb|\ifhaspatents|,\ixcom{ifhaspatents} 
      \verb|\haspatentstrue|,\ixcom{haspatentstrue} 
      \verb|\haspatentsfalse|\ixcom{haspatentsfalse} STEP: flag for known patents;
\item \verb|\ifmapspec|,\ixcom{ifmapspec} 
      \verb|\mapspectrue|,\ixcom{mapspectrue} 
      \verb|\mapspecfalse|\ixcom{mapspecfalse} STEP: flag for mapping specification;

\item \verb|\ixent|\ixcom{ixent} STEP: index an \Express{} \xword{entity};
\item \verb|\ixenum|\ixcom{ixenum} STEP: index an \Express{}  \xword{enumeration};
\item \verb|\ixfun|\ixcom{ixfun} STEP: index an \Express{}  \xword{function};
\item \verb|\ixproc|\ixcom{ixproc} STEP: index an \Express{}  \xword{procedure};
\item \verb|\ixrule|\ixcom{ixrule} STEP: index an \Express{}  \xword{rule};
\item \verb|\ixsc|\ixcom{ixsc} STEP: index an \Express{}  \xword{subtype\_constraint};
\item \verb|\ixschema|\ixcom{ixschema} STEP: index an \Express{}  \xword{schema};
\item \verb|\ixselect|\ixcom{ixselect} STEP: index an \Express{}  \xword{select};
\item \verb|\ixtype|\ixcom{ixtype} STEP: index an \Express{}  \xword{type};

\item \verb|\maptableorspec|\ixcom{maptableorspec} STEP: prints `table' or `specification';

\item \verb|\xword|\ixcom{xword} STEP: prints an \Express{} keyword;

%%%%%%%%%%%%%%%%%%%%%%%%%%%%%%%% AP %%%%%%%%%%%%%%%%%%%%%%%%%%%%%%%


\item \verb|\apmaptemplate|\ixcom{apmaptemplate} AP: boilerplate;
\item \verb|\apusagehead|\ixcom{apusagehead} AP: clause heading;
\item \verb|\ifidefix|,\ixcom{ifidefix} 
      \verb|\idefixtrue|,\ixcom{idefixtrue} 
      \verb|\idefixfalse|\ixcom{idefixfalse} AP: flag for an IDEF1X ARM;
\item \verb|\ifmaptemplate|,\ixcom{ifmaptemplate} 
      \verb|\maptemplatetrue|,\ixcom{maptemplatetrue} 
      \verb|\maptemplatefalse|\ixcom{maptemplatefalse} AP: flag for 
            using mapping templates;
\item \verb|\mapattributehead|\ixcom{mapattributehead} AP: clause heading;
\item \verb|\mapobjecthead|\ixcom{mapobjecthead} AP: clause heading;
\item \verb|\mapuofhead|\ixcom{mapuofhead} AP: clause heading;
\item \verb|\sstemplates|\ixcom{sstemplates} AP: boilerplate;
\item \verb|\templateshead|\ixcom{templateshead} AP: clause heading;
%%% \item \verb|\apmappingspec|\ixcom{} internal (not used?) 

%%%%%%%%%%%%%%%%%%%%%%%%%%%%%%%% ATS %%%%%%%%%%%%%%%%%%%%%%%%%%%%%

\item \verb|\atcposttpc|\ixcom{atcposttpc} ATS: boilerplate;
\item \verb|\atcpretpc|\ixcom{atcpretpc} ATS: boilerplate;
\item \verb|\atsimtpbp|\ixcom{atsimtpbp} ATS: boilerplate;
\item \verb|\atsusagehead|\ixcom{atsusagehead} ATS: clause heading.


\end{itemize}





\sclause{Modified commands}

    The commands that have been modified in this release are:

\begin{itemize}

%%%%%%%%%%%%%%%%%%%%%%%%%%%% STEP %%%%%%%%%%%%%%%%%%%%%%%%%%%%%

\item STEP: The \verb|\Introduction|\ixcom{Introduction} command is
      now the \verb|Introduction|\ixenv{Introduction} environment,
      with no argument;

%%%%%%%%%%%%%%%%%%%%%%%%%%%%%% IR %%%%%%%%%%%%%%%%%%%%%%%%%%%%%%%

\item \verb|\irexpressg|\ixcom{irexpressg} IR: takes no argument;

%%%%%%%%%%%%%%%%%%%%%%%%%%%%%% AP %%%%%%%%%%%%%%%%%%%%%%%%%%%%%%%

\item \verb|\aimexpressg|\ixcom{aimexpressg} AP: takes no argument;

%%%%%%%%%%%%%%%%%%%%%%%%%%%%%% AIC %%%%%%%%%%%%%%%%%%%%%%%%%%%%%%%

\item \verb|\aicexpressg|\ixcom{aicexpressg} AIC: takes no argument;


%%%%%%%%%%%%%%%%%%%%%%%%%%%%%% ATS %%%%%%%%%%%%%%%%%%%%%%%%%%%%%%%

\item \verb|\atcbpii|\ixcom{atcbpii} ATS:  takes no argument;
\item \verb|\atspurposebp|\ixcom{atspurposebp} ATS: takes no argument;
\item \verb|\pisfbp|\ixcom{pisfbp} ATS: takes 3 arguments.


\end{itemize}





\sclause{Deleted commands}

    The commands that have been deleted in this release are:

\begin{itemize}

%%%%%%%%%%%%%%%%%%%%%%%%%%%%%% STEP %%%%%%%%%%%%%%%%%%%%%%%%%%%%%

\item \verb|\fwddivlist|\ixcom{fwddivlist} STEP: used in Foreword;
\item \verb|\fwdpartslist|\ixcom{fwdpartslist} STEP: used in Foreword;

\item \verb|\introend|\ixcom{introend} STEP: was for use at the end of the
      Introduction;

%%%%%%%%%%%%%%%%%%%%%%%%%%%%%% IR %%%%%%%%%%%%%%%%%%%%%%%%%%%%%

\item \verb|\irschemaintro|\ixcom{irschemaintro} IR:
      has been replaced by
      \verb|\schemaintro|\ixcom{schemaintro};

%%%%%%%%%%%%%%%%%%%%%%%%%%%%%% AP %%%%%%%%%%%%%%%%%%%%%%%%%%%%%

\item \verb|\apintroend|\ixcom{apintroend} AP:
      has been replaced by
      \verb|\apextraintro|\ixcom{apextraintro};

\item \verb|\apschemareg|\ixcom{apschemareg} AP: use 
           \verb|\schemareg|\ixcom{schemareg} instead;

\item \verb|\apmappingtable|\ixcom{apmappingtable} AP:
      has been replaced by
      \verb|\apmapping|\ixcom{apmapping};

\item \verb|\armfigures|\ixcom{armfigures} AP:
      has been replaced by
      \verb|\armintro|\ixcom{armintro};

\item \verb|\maptablehead|\ixcom{maptablehead} AP:
      has been replaced by
      \verb|\mappinghead|\ixcom{mappinghead};

\item \verb|\modelscopehead|\ixcom{modelscopehead} AP: was the heading
      for a `Model scope' annex;

%%%%%%%%%%%%%%%%%%%%%%%%%%%%%% AIC %%%%%%%%%%%%%%%%%%%%%%%%%%%%%

\item \verb|\aicexpressghead|\ixcom{aicexpressghead} AIC: 
      use \verb|\expressghead|\ixcom{expressghead} instead;
\item \verb|\aicshortnames|\ixcom{aicshortnames} AIC: use
      \verb|\expurls|\ixcom{expurls} instead;
\item \verb|\aicshortnameshead|\ixcom{aicshortnameshead} AIC: 
      use \verb|\shortnamehead|\ixcom{shortnamehead} instead;

%%%%%%%%%%%%%%%%%%%%%%%%%%%%%% ATS %%%%%%%%%%%%%%%%%%%%%%%%%%%%%


\item \verb|\excludepurposehead|\ixcom{excludepurposehead} ATS: 
      was the heading for an `Exclude purposes' clause.

\end{itemize}


%%%%%%%%%%%%%%%%%%%%%%%
%%%\end{document}
%%%%%%%%%%%%%%%%%%%%%%%


% sgmlannx.tex    latex and SGML

\infannex{LaTeX, the Web, and *ML} \label{anx:sgml} \index{SGML}

    ISO are becoming more interested in electronic sources for their
standards as well as the traditional camera-ready copy. Acronyms like
PDF, HTML, SGML and XML have been bandied about. Fortunately documents
written using \latex{} are well placed to be provided in a variety of 
electronic formats. A comprehensive treatment of \latex{} with respect
to this topic is provided by Goossens and Rahtz~\bref{lwebcom}.

    SGML (Standard Generalized Markup Language) is a document tagging 
language that is described in ISO~8879~\bref{sgml} and whose usage is described 
in~\bref{bryan}, among others. The principal
mover behind SGML is Charles Goldfarb from IBM, who has authored a detailed 
handbook~\bref{goldfarb} on the SGML standard.

    The concepts lying behind both \latex{} and SGML are similar, but on the face
of it they are distinctly different in both syntax and capabilities. ISO is
migrating towards electronic versions of its standard documents and, naturally, 
would prefer these to be SGML tagged. 
     Like \latex, SGML has a
concept of style files, which are termed DTDs, and both systems support
powerful macro-like capabilities. SGML provides for logical document
markup and not typesetting --- commercial SGML systems often use
\tex{} or \latex{} as their printing engine, as does the NIST SGML
environment for ISO~10303~\bref{pandl}.



NIST have SGML tagged some STEP documents 
using manual methods, which are time consuming and expensive. 
In about 1997 there was a NIST 
effort underway to develop an auto-tagger that would (semi-) automatically 
convert
a \latex{} tagged document to one with SGML tags. This tool assumed a
fixed set of \latex{} macros and a fixed DTD.
 The design of an auto-tagger
essentially boils down to being able to convert from a source document tagged
according to a \latex{} style file to one which is tagged according to an
SGML DTD.
    Fully automatic conversion is really only possible if the authors'
of the documents to be translated avoid using any `non-standard' macros within
their documents. There is a program called \file{ltx2x}\index{ltx2x} available
from SOLIS, which replaces \latex{} commands within a document with
user-defined text strings~\bref{ltx2x}. This can be used as a basis for
a \latex{} to whatever auto-tagger, provided the \latex{} commands are not
too exotic.

    HTML is a simple markup language, based on SGML, and is used for the
publication of many documents on the Web. XML is a subset of SGML and appears
to being taken up by every man and his dog as \emph{the} document markup
language. HTML is being recast in terms of XML instead of SGML. PDF is a page
description language that is a popular format for display of documents 
on the Web.

    \latex{} documents can be output in PDF by using pdfLaTeX. Instead
of a \file{.dvi} file being produced a \file{.pdf} file is output directly.
The best 
results are obtained when PostScript fonts rather than Knuth's cm fonts 
are used. Noting that the \file{isov2} class provides an \verb|\ifpdf| command,
a general form for documents to be processed by either \latex{} or pdfLaTeX
is
\begin{verbatim}
\documentclass{isov2}
\usepackage{times}     % PostScript fonts Times, Courier, Helvetica
\ifpdf
  \pdfoutput=1         % request PDF output
  \usepackage[pdftex]{graphicx}
\else
  \usepackage{graphicx}
\fi
...
\end{verbatim}

    There are several converters available to transform a \latex{} document 
into an HTML document, but like \file{ltx2x} they generally do their own
parsing of the source file, and unlike \file{ltx2x} are typically limited
to only generating HTML. Eitan Gurari's \file{TeX4ht}\index{TeX4ht} 
suite is a notable
exception (see Chapter~4 and Appendix~B of~\bref{lwebcom}). It uses the 
\file{.dvi} file as input, so that all the parsing is done by \tex, and can be
configured to generate a wide variety of output formats.
A set of \file{TeX4ht} configuration files are available for converting
STEP \latex{} documents into HTML\footnote{Later, configuration files for XML
output will be developed.}.

    It is highly recommended that for the purposes of ISO~10303, document editors
refrain from defining their own \latex{} macros. If new generally applicable
\latex{} commands are found to be necessary, these should be sent to the
editor of this document for incorporation into
the \file{isov2}\ixclass{isov2} class, the \file{step}\ixpack{step}
package and/or appropriate other packages.

    Some other points to watch when writing \latex{} documents that will assist
in translations into *ML are given below. Typically, attention to these points
will make it easier to parse the \latex{} source.

\begin{itemize}
\item Avoid using the \verb|\label|\ixcom{label} command within
      clause headings or captions. It can just as easily be placed immediately
      after these constructs.
\item Avoid using the \verb|\index|\ixcom{index} command within
      clause headings or captions. It can just as easily be placed immediately
      after these constructs.
\item Use all the specified tagging constructs when defining an \Express{} 
      model --- this will also assist any program that attempts to extract
      \Express{} source code and descriptive text from a document.
\end{itemize}



\infannex{Obtaining LaTeX and friends} \label{anx:getstuff}

    \latex{} is a freely available document typesetting system. There are many
public domain additions to the basic system, for example the \file{iso.cls}
and \file{step.sty} styles. The information below gives pointers to where
you can obtain \latex{} etc., from the\index{Internet} Internet. 


    \latex{} runs on a wide variety of hardware, from PCs to Crays. Source to build
a \latex{} system is freely available via anonymous ftp\index{ftp} 
from what is called CTAN\index{CTAN}
(Comprehensive \tex\ Archive Network). 
There are three sites; pick the one nearest to you.
\begin{itemize}
\item \url{ftp.dante.de} CTAN in Germany;
\item \url{ftp.tex.ac.uk} CTAN in the UK;
\item \url{ctan.tug.org} CTAN in the USA;
\end{itemize}
The top level CTAN directory 
for \latex{} and friends is \url{/tex-archive}. CTAN contains a wide variety
of (La)TeX sources, style files, and software tools and scripts to assist in
document processing.

\begin{anote}
CTAN is maintained by the \tex{} Users Group (TUG). Their homepage
\isourl{http://www.tug.org} should be consulted for the current list of CTAN sites and mirrors.
\end{anote}

\begin{comment}

\sclause{SOLIS} \index{SOLIS}

    SOLIS is the \textit{SC4 On Line Information Service}. It contains many electronic
sources of STEP related documents. The relevant top level directory is
\url{pub/subject/sc4}.
 In particular, SOLIS contains the source for this document
and the \file{.sty} files, as well as other \latex{} related files. 
The \latex{} root directory is \url{sc4/editing/latex}.
The latest versions of the \latex{}
related files are kept in the sub-directory \url{latex/current}.
Some \latex{} related programs are also available in the 
\url{latex/programs} sub-directory.

    There are several ways of accessing SOLIS; instructions
are detailed by Ressler~\bref{ressler} and Rinaudot~\bref{rinaudot}. 
Copies of these reports may be obtained by telephoning the
IPO Office at \verb|+1 (301) 975-3983|, although they are probably somewhat
dated by now.
The simplest method is to point your browser at the following URL: \\
\isourl{http://www.nist.gov/sc4}

\end{comment}

\bibannex
\label{biblio}

\begin{references}
\reference{LAMPORT, L.,}{LaTeX --- A Document Preparation System,}
            {Addison-Wesley Publishing Co., 2nd edition, 1994} \label{lamport}
\reference{WILSON, P. R.,}{LaTeX for standards: The LaTeX package files
            user manual,}%
           {NISTIR, 
                   National Institute of Standards and Technology,
           Gaithersburg, MD 20899. June 1996.} \label{doc:isorot}
\reference{GOOSENS, M., MITTELBACH, F. and SAMARIN, A.,}{%
           The LaTeX Companion,}
           {Addison-Wesley Publishing Co., 1994} \label{goosens}
\reference{GOOSENS, M. and RAHTZ, S.,}{%
           The LaTeX Web Companion --- Integrating TeX, HTML, and XML,}
           {Addison-Wesley Publishing Co., 1999} \label{lwebcom}
\reference{CHEN, P. and HARRISON, M.A.,}{Index preparation and
           processing,}{Software--Practice and Experience, 19(9):897--915,
           September 1988.} \label{chen}
%\reference{KOPKA, H. and DALY, P.W.,}{A Guide to LaTeX,}
%           {Addison-Wesley Publishing Co., 1993.} \label{kopka}
\reference{ISO 8879:1986,}{Information processing --- 
                                Text and office systems ---
           Standard Generalized Markup Language (SGML)}{} \label{sgml}
\reference{GOLDFARB, C.F.,}{The SGML Handbook,}
           {Oxford University Press, 1990.} \label{goldfarb}
\reference{BRYAN, M.,}{SGML --- An Author's Guide to the Standard Generalized
           Markup Language,}{Addison-Wesley Publishing Co., 1988. }\label{bryan}
\reference{PHILLIPS, L., and LUBELL, J.,}{An SGML Environment for STEP,}%
           {NISTIR 5515, 
                   National Institute of Standards and Technology,
           Gaithersburg, MD 20899. November 1994.} \label{pandl}
\reference{WILSON, P. R.,}{LTX2X: A LaTeX to X Auto-tagger,}%
           {NISTIR, 
                   National Institute of Standards and Technology,
           Gaithersburg, MD 20899. June 1996.} \label{ltx2x}
\bibidefo
\bibieeeidefix
\begin{comment}
\reference{RESSLER, S.,}{The National PDES Testbed Mail Server User's Guide,}
           {NSTIR 4508, National Institute of Standards and Technology,
           Gaithersburg, MD 20899. January 1991.} \label{ressler}
\reference{RINAUDOT, G. R.,}{STEP On Line Information Service (SOLIS),}
          {NISTIR 5511, National Institute of Standards and Technology,
          Gaithersburg, MD 20899. October 1994. } \label{rinaudot}
\end{comment}
\end{references}

    
%  the INDEX
% stepman.tex   Description of option style files for STEP
\documentclass[wd,copyright,letterpaper]{isov2}
\usepackage{stepv13}
\usepackage{irv12}
\usepackage{apv12}
\usepackage{aicv1}
\usepackage{atsv11}
%%\usepackage{isomods}  % must come after the step packages
\usepackage{hyphenat}
\usepackage{comment}

\ifpdf
  \pdfoutput=1
  \usepackage[plainpages=false,
              pdfpagelabels,
              bookmarksnumbered,
              hyperindex=true
             ]{hyperref}
\fi

% general required preamble commands
\standard{ISO/WD 10303-3456}
\yearofedition{2002}
\languageofedition{(E)}
\renewcommand{\extrahead}{N 47b}  % add doc N number to headers
\partno{3456}
\series{documentation methods}
\doctitle{LaTeX package files for ISO 10303: User manual}
\ballotcycle{2}
% required preamble commands for an AP
\aptitle{implicit drawing}
\aicinaptrue % only if the AP uses AICs
\mapspectrue  % only if AP uses mapping specification
% required preamble commands for an ATS
\APtitle{abstract painting}
\APnumber{299}

\changemarkstrue

\makeindex

\setcounter{tocdepth}{3} % add more levels to table of contents
%
% Rest of preamble is some special macro definitions for this document only
%
\makeatletter
%
%   the \file{} command
%
\newcommand{\file}[1]{\textsf{#1}}
%
%   the \meta{} command
%
\begingroup
\obeyspaces%
\catcode`\^^M\active%
\gdef\meta{\begingroup\obeyspaces\catcode`\^^M\active%
\let^^M\do@space\let \do@space%
\def\-{\egroup\discretionary{-}{}{}\hbox\bgroup\it}%
\m@ta}%
\endgroup
\def\m@ta#1{\leavevmode\hbox\bgroup$<$\it#1\/$>$\egroup
    \endgroup}
\def\do@space{\egroup\space
    \hbox\bgroup\it\futurelet\next\sp@ce}
\def\sp@ce{\ifx\next\do@space\expandafter\sp@@ce\fi}
\def\sp@@ce#1{\futurelet\next\sp@ce}
%
% the \setlabel{id}{num} command
% this is based on the kernel \refstepcounter macro (ltxref.dtx)
%
%%\newcounter{lbl}
\ifpdf
  \newcommand{\setlabel}[2]{%
    \protected@write\@auxout{}{%
      \string\newlabel{#1}{{#2}{\thepage}{setlabel\relax}{label.#2}{}}}%
  }
\else
  \newcommand{\setlabel}[2]{%
    \protected@write\@auxout{}{%
      \string\newlabel{#1}{{#2}{\thepage}}}%
  }
\fi
%
% index a command
\newcommand{\bs}{\symbol{'134}}
\newcommand{\ixcom}[1]{\index{#1/ @{\tt \protect\bs #1}}}
% index an environment
\newcommand{\ixenv}[1]{\index{#1 @{\tt #1} (environment)}}
% index an option
\newcommand{\ixopt}[1]{\index{#1 @{\tt #1} (option)}}
% index a package
\newcommand{\ixpack}[1]{\index{#1 @\file{#1} (package)}}
% index a class
\newcommand{\ixclass}[1]{\index{#1 @\file{#1} (class)}}
% index in typewriter font
\newcommand{\ixtt}[1]{\index{#1@{\tt #1}}}
% index LaTeX
\newcommand{\ixltx}{\index{latex@\LaTeX}}
% index LaTeX 2e
\newcommand{\ixltxe}{\index{latex2e@\LaTeX 2e}}
% index LaTeX v2.09
\newcommand{\ixltxv}{\index{latex209@\LaTeX{} v2.09}}
% index a file
\newcommand{\ixfile}[1]{\index{#1@\file{#1}}}
\makeatother
%
%
%   set some labels
% step
\setlabel{;ssne}{A}
%%%\setlabel{;sior}{B}
%%%\setlabel{;scil}{C}
\setlabel{;seg}{D}
% aic
\setlabel{;sesl}{4}
% ap
\setlabel{;sireq}{4}
\setlabel{;suof}{4.1}
\setlabel{;sao}{4.2}
\setlabel{;saa}{4.3}
\setlabel{;saim}{5}
\setlabel{;smap}{5.1}
\setlabel{;saesl}{5.2}
\setlabel{;scr}{6}
\setlabel{;saeel}{A}
\setlabel{;sasn}{B}
\setlabel{;simreq}{C}
\setlabel{;spics}{D}
\setlabel{;saam}{F}
\setlabel{;sarm}{G}
\setlabel{;saeg}{H}
\setlabel{;scil}{J}
\setlabel{tabB1}{B.1}
\setlabel{;uof1}{5.1.2}
\setlabel{;uoflast}{5.1.4}
%
% define a new length
\newlength{\prwlen}
%
% new (La)TeX macros
\newcommand{\latex}{LaTeX}
\newcommand{\tex}{TeX}
%
%%%%%%%%%% END SPECIAL MACROS
%
%   end of preamble
%
\begin{document}


\STEPcover{
%\scivnumber{987}
\wg{EC}
\docnumber{47b}
\oldwg{EC}
\olddocnumber{47a}
\docdate{2002/09/04}
%\partnumber{3456}
%\doctitle{LaTeX package files for ISO 10303: User manual}
%\status{Working draft}
%\primcont
\abstract{This document describes and illustrates the \latex{} macros
for typesetting ISO~10303. The International Organisation for
Standardisation (ISO) has specified editorial directives for all 
international standards published by them. The \latex{} macros
described here were developed to meet additional editorial directives 
for ISO~10303. } % end abstract
\keywords{\latex, document preparation, typesetting ISO standards}
%\dateprojo{May 1996}
\owner{Peter R. Wilson}
\address{Boeing Commercial Airplane\newline
            PO Box 3701 \newline
            MS 2R-97 \newline
            Seattle, WA 98124-2207 \newline
            USA}
\telephone{+1 (206) 544-0589}
\fax{+1 (206) 544-5889}
\email{\url{peter.r.wilson@boeing.com}}
\altowner{Peter R. Wilson}
\altaddress{Boeing Commercial Airplane \newline
            PO Box 3701 \newline
            MS 2R-97 \newline
            Seattle, WA 98124-2207 \newline
            USA}
\alttelephone{+1 (206) 544-0589}
\altfax{+1 (206) 544-5889}
\altemail{\url{peter.r.wilson@boeing.com}}
\comread{\draftctr This document serves two purposes. Firstly, it provides a description
         of the current \latex{} macros for ISO 10303. Secondly, the source
         can be used as an example of using the \latex{} commands.
         Although the document is written as though it were a
         standard, it is not, and is not intended to become, 
         a standard.} %end comread
} % end of STEPcover

\Foreword

\fwdshortlist
\endForeword%
{Annexes A, B and C are}  % normative annexes
{Annexes D, E and F are} % informative annexes

\begin{Introduction}%%%%%%%%%%{documentation methods}

    This part of ISO 10303 specifies the \latex{} facilities specifically 
designed for use in preparing the various parts of this standard.

\begin{majorsublist}
\item the \file{step} package facility;
\item the \file{ir} package facility;
\item the \file{ap} package facility;
%%%\item the \file{am} package facility;
\item the \file{aic} package facility;
\item the \file{ats} package facility.
\end{majorsublist}

    This part of ISO~10303 is intended to be used in conjunction with
\textit{\latex{} for ISO standards: User manual}
which is based in part upon material in the ISO/IEC Directives,
Part 2 (\textit{Rules for the structure and drafting of International 
Standards, Fourth edition}).
The \latex{} facilities described here are based as well
upon the specifications given in ISO TC184/SC4 N1217n 
(\textit{SC4 Supplementary directives --- Rules for the structure
and drafting of SC4 standards for industrial data}).


\sclause*{Overview}


    This document describes a set of \latex{} macro files for use within
ISO~10303, commonly called STEP (STandard for the Exchange of Product
model data). The electronic source of this  document 
also provides an example of the use of these files.

    The current set of macro files have been developed by 
Peter Wilson (\url{peter.r.wilson@boeing.com}) from a macro file developed
by Kent Reed (NIST) for \latex{} v2.09. In turn, this was a revision of
files originally created by Phil Spiby (CADDETC), based on earlier work 
by Phil Kennicott (GE).\footnote{In mid 1994 \latex{} was upgraded 
from version 2.09 to what is called \latex 2e. The files described in 
this document are only applicable to \latex 2e (support for \latex{} v2.09
was dropped in September 1997).}


\begin{anote} 
It is important to remember that these macro files are only compatible with 
\latex 2e.
\end{anote} % end anote

    Documents produced with the \latex{} files have been twice reviewed 
by the ISO Editorial Board in Geneva for conformance to their 
typographical requirements. The first review was of a set of Draft 
International Standard documents. This review resulted in some changes 
to the style files. The second review was of a set of twelve 
International Standard documents (ISO 10303:1994). Likewise, this
review led to changes in the style files to bring the documents into 
conformance.

    With the issuance of the first STEP release, the opportunity was 
taken to provide a new baseline release of the package files. 
In particular, one STEP specific package file is available for all 
STEP parts, while others contain only commands relevant to the 
documentation of particular series of parts. The range of package 
files may be extended in the future to cater for 
documentation specific to all STEP parts.

   The 1997 baseline release was also designed to cater for the 
fact that a major update of \latex{} to \latex 2e took place during 1994.
\latex 2e is the only officially supported version of \latex.

    Because ISO standard documents have a very structured layout, the 
\file{isov2} class and the package files described here have been 
designed to reflect the logical document structure to a much greater 
extent than the `standard' \latex{} files. 

    With ISO's move toward accepting documents in PDF and HTML, 
the advent of second
editions of some of the STEP parts, and a new edition of the STEP
Supplementary Directives, a 2002
baseline release has been developed and is documented here. 



\end{Introduction}

\stepparttitle{Documentation methods: LaTeX package files for ISO 10303:
User manual}


\scopeclause

This part of ISO~10303 describes a set of \ixltx\latex{} facilities for typesetting
documents according to the ISO/IEC Directives Part 2, together with the 
Supplementary Directives for drafting and presentation of ISO~10303.

\begin{inscope}{part of ISO~10303}
\item use of \latex{} for preparing ISO~10303 documents.
\end{inscope}

\begin{outofscope}{part of ISO~10303}
\item use of \latex{} for preparing ISO standard documents in general;
\item use of \latex{} in general;
\item use of other document preparation systems.
\end{outofscope}

\textbf{IMPORTANT:} The preparation of this document has been partly
funded by the US Government and is not subject to copyright.
Any copyright notices within the document are for illustrative purposes only.

\normrefsclause \label{sec:nrefs}

\normrefbp{part of ISO~10303}
\begin{nreferences}

\isref{ISO/IEC Directives, Part 2}{Rules for the structure and drafting 
           of International Standards, Fourth edition.}

\isref{ISO TC 184/SC4 N1217:2001(E)}{SC4 Supplementary directives --- 
       Rules for the structure and drafting of SC4 standards for 
       industrial data.}

%\isref{ISO 10303-1:1994}{Industrial automation systems and integration ---
%        Product data representation and exchange --- 
%        Part 1: Overview and fundamental principles.}
\nrefparti

%\isref{ISO 10303-11:1994}{Industrial automation systems and integration --- 
%        Product data representation and exchange --- 
%        Part 11: Description methods:
%        The EXPRESS language reference manual.}
\nrefpartxi

%\disref{ISO/TR 10303-12:---}{Industrial automation systems and integration ---
%        Product data representation and exchange ---
%        Part 12: Description methods:
%        The EXPRESS-I language reference manual.}
\nrefpartxii

%\disref{ISO/IEC 8824-1:---}{Information technology ---
%       Open systems interconnection ---
%       Abstract syntax notation one (ASN.1) ---
%       Part 1: Specification of basic notation.}
\nrefasni

\disref{P. R. WILSON:---}{LaTeX for ISO standards: User manual.}

\end{nreferences}

\defabbclause
%\clause{Terms, definitions, and abbreviations}

\partidefhead
%\sclause{Terms defined in ISO 10303-1}

    This part of ISO~10303 makes use of the following terms defined in 
ISO~10303-1:

\begin{olddefinitions}
\olddefinition{application protocol (AP)} \index{Application Protocol}
                                            \index{AP}
\olddefinition{integrated resource} \index{Integrated Resource}
\end{olddefinitions}


\otherdefhead
%\sclause{Other definitions}

    For the purposes of this part of ISO~10303, the following definitions
apply.

\begin{definitions}
\definition{boilerplate}{Text whose wording is fixed and which has been
agreed to be present in a specific type of document.} \index{boilerplate}
\definition{style file}{A set of \latex{} macros assembled into a 
single file with an extension \file{.sty}.}
            \index{style file}
\definition{package file}{A style file for use with \latex 2e.}
            \index{package file}
\definition{facility}{A generic term for a set of \latex{} macros
          assembled for a common purpose. The macros may be defined in
          either a style file or a package file.}\index{facility}

\end{definitions}

\abbsubclause
%\sclause{Abbreviations}

    For the purposes of this part of ISO 10303, the following abbreviations
 apply.

\begin{symbols}
\symboldef{AIC}{Application Interpreted Construct} \index{AIC}
\symboldef{AM}{Application Module} \index{AM}
\symboldef{AP}{Application Protocol}  \index{AP}
\symboldef{DIS}{Draft International Standard} \index{DIS}
\symboldef{IS}{International Standard}         \index{IS}
\symboldef{ISOD}{ISO/IEC Directives, Part 2} \index{ISOD} \index{ISO/IEC Directives}
\symboldef{SD}{Supplementary Directives --- 
  \textit{SC4 Supplementary directives --- Rules for the structure and
   drafting of SC4 standards for industrial data}}\index{SD}\index{Supplementary Directives}
\symboldef{IS-REVIEW}{The ISO Editorial Board review (September 1994) of 
            twelve IS documents
            for conformance to ISO typographical and 
            layout requirements.} \index{IS-REVIEW}
\end{symbols}



\clause{Conformance requirements}  \label{sec:iconform}

    The facility files shall not be modified in any manner.

    If there is a need to modify any of the macro definitions then this
shall be done using the \latex{} 
\verb|\renewcommand|\ixcom{renewcommand} and/or the
\verb|\renewenvironment|\ixcom{renewenvironment}
commands. These shall be placed in a new \file{.sty} file (or files) 
which shall be called in within the preamble\index{preamble} of the 
document being typeset.

    There shall be no author specified \verb|\label{...}| commands where
the first two characters of the label are \verb|;s| (semicolon and `s');
the creation of labels starting with these characters is reserved to the 
maintainer of the facility files.

\begin{anote} For conformance to the \file{isov2} class, author specified
labels starting with the characters \verb|;i| (semicolon and `i') are
prohibited.
\end{anote}


\fcandaclause
%\clause{Fundamental concepts and assumptions}

    It is assumed that the reader of this document is familiar with the
\ixltx\latex{} document preparation system and in particular
with the \file{isov2}\ixclass{isov2} class and associated facilities 
described in 
\textit{LaTeX for ISO standards: User manual}.

\begin{note}Reference~\bref{lamport} describes the
      \latex{} system.
\end{note} % end note

    The reader is also assumed to be familiar with the ISO/IEC Directives 
Part~2 (ISOD)\index{ISOD} and
the SC4 Supplementary directives for the structure and drafting of 
SC4 standards (SD).\index{ISOD}\index{SD}

    If there are any discrepancies between the layout and wording of this 
document and the requirements of the ISOD or the SD,
then the requirements in those documents shall be
followed for ISO~10303 standard documents.

    The packages described herein have been designed to be used with
the \file{isov2}\ixclass{isov2} document class. It is highly unlikely that the
packages will perform at all with any other \latex{} document class.

    Because of many revisions over the years to the packages described
herein, a naming convention has been adopted for the package files.
The naming convention is that the
primary name of the file is suffixed by \file{v\#}, where
\file{\#} is the primary version number of the file in question.
All file primary names have been limited to a maximum of eight characters.

\begin{note}Table~\ref{tab:curfiles} shows the versions of the files
that were current at the time of publication.
\ixpack{step}\ixfile{stepv13.sty}
\ixpack{ir}\ixfile{irv12.sty}
\ixpack{ap}\ixfile{apv12.sty}
\ixpack{aic}\ixfile{aicv1.sty}
\ixpack{ats}\ixfile{atsv11.sty}
%%%\ixpack{am}\ixfile{amv1.sty}
\end{note} % end note

\begin{table}
\centering
\caption{File versions current at publication time} \label{tab:curfiles}
\begin{tabular}{|l|l|l|} \hline
\textbf{Facility} & \textbf{File}   & \textbf{Version} \\ \hline\hline
\file{step}    & \file{stepv13.sty} & v1.3.2 \\
\file{ir}      & \file{irv12.sty}   & v1.2   \\
\file{ap}      & \file{apv12.sty}   & v1.2   \\
%%%\file{am}      & \file{amv1.sty}    & v1.0   \\
\file{aic}     & \file{aicv1.sty}   & v1.0   \\
\file{ats}     & \file{atsv11.sty}  & v1.1   \\ 
\hline
\end{tabular}
\end{table}


\begin{note}
This document is not, and is never intended to become,
 a standard, although it has been laid out in a 
similar, but not necessarily identical, manner.
\end{note} % end note


\clearpage
\clause{The \file{step} package facility}

    The \file{step}\ixpack{step} package facility provides commands 
and environments 
applicable to all the ISO~10303 series of documents.

\sclause{Preamble commands}

    Certain commands shall be put in the preamble\index{preamble}
of any document.

    The command 
\verb|\partno{|\meta{number}\verb|}|\ixcom{partno}
is used to specify the Part number of the ISO~10303 standard
(e.g., \verb|\partno{3456}|).

    The command
\verb|\series{|\meta{series title}\verb|}|\ixcom{series}
is used to specify the name of the ISO~10303 series of which the Part 
is a member (e.g., \verb|\series{application modules}|).

    The command
\verb|\doctitle{|\meta{informal title}\verb|}|\ixcom{doctitle}
is used to specify the title to be used on the cover sheet.
For example: \\
\verb|\doctitle{LaTeX package files for ISO 10303: User manual}|

    The command
\verb|\ballotcycle{|\meta{number}\verb|}|\ixcom{ballotcycle}
is used to specify the ballot cycle number for the document
(e.g., \verb|\ballotcycle{2}|).

    The command\ixcom{ifhaspatents}
\verb|\haspatentstrue|\ixcom{haspatentstrue} shall be put in the
preamble when the document includes identified patented material;
otherwise the command \verb|\haspatentsfalse|\ixcom{haspatentsfalse}
may, but need not, be used instead.

    The \verb|\extrahead|\ixcom{extrahead} macro, from the \file{isov2}
class, shall be defined to be the document
number (e.g., \verb|\renewcommand{\extrahead}{47a}|).



\begin{anote}
The commands \verb|\standard|\ixcom{standard}, 
\verb|\yearofedition|\ixcom{yearofedition} and 
\verb|\languageofedition|\ixcom{languageofedition} from the \file{isov2}
class must also be put in the preamble.
\end{anote}


\sclause{Cover page}

    The command \verb+\STEPcover{+\meta{commands}\verb+}+\ixcom{STEPcover}
produces a cover page for a STEP document. 
The complete list of commands is shown below.

\begin{itemize}
\item \verb+\wg{+\meta{working group}\verb+}+\ixcom{wg}
      the working 
      group or other committee producing the document e.g., WG 5
\item \verb+\docnumber{+\meta{number}\verb+}+\ixcom{docnumber}
       the number
       of the document e.g., 156
\item \verb+\docdate{+\meta{date}\verb+}+\ixcom{docdate}
       date of 
       publication e.g., 1993/07/03
\item \verb+\oldwg{+\meta{working group}\verb+}+\ixcom{oldwg}
       superseded 
       working group e.g., WG 1
\item \verb+\olddocnumber{+\meta{number}\verb+}+\ixcom{olddocnumber}
        number of previous document e.g., 107
\item \verb+\abstract{+\meta{text}\verb+}+\ixcom{abstract}
        an abstract 
        of the document
\item \verb+\keywords{+\meta{text}\verb+}+\ixcom{keywords}
        for listing 
        relevant keywords
\item \verb+\owner{+\meta{text}\verb+}+\ixcom{owner}
         name of the project leader
\item \verb+\address{+\meta{text}\verb+}+\ixcom{address}
        address of the project leader
\item \verb+\telephone{+\meta{number}\verb+}+\ixcom{telephone}
         the project leader's telephone number
\item \verb+\fax{+\meta{number}\verb+}+\ixcom{fax}
         the project leader's fax number
\item \verb+\email{+\meta{text}\verb+}+\ixcom{email}
        Email address of the project leader
\item \verb+\altowner{+\meta{text}\verb+}+\ixcom{altowner}
         name of the editor of the document
\item \verb+\altaddress{+\meta{text}\verb+}+\ixcom{altaddress}
         the editor's address 
\item \verb+\alttelephone{+\meta{number}\verb+}+\ixcom{alttelephone}
         the editor's telephone number
\item \verb+\altfax{+\meta{number}\verb+}+\ixcom{altfax}
         the editor's fax number
\item \verb+\altemail{+\meta{text}\verb+}+\ixcom{altemail}
           the editor's Email address 
\item \verb+\comread{+\meta{text}\verb+}+\ixcom{comread}
           comments to 
           the reader
\end{itemize}

    Use only those commands within \verb|\STEPcover| that are relevant 
to the purposes at hand. The order of the commands within 
\verb|\STEPcover| is immaterial.

\begin{example}
The commands used to produce the cover sheet for one version of this 
document were:
\begin{verbatim}
\STEPcover{
\wg{EC}
\docnumber{41}
\oldwg{EC}
\olddocnumber{35}
\docdate{1994/08/19}
\abstract{This document describes the \latex{} style files created for ISO~10303.
          It also describes the program GenIndex which provides some 
          capabilities to assist in the creation of indexes for \latex{}
          documents in general.}
\keywords{\latex, Style file, GenIndex, Index}
\owner{Peter R Wilson}
\address{NIST\newline
         Bldg. 220, Room A127 \newline
         Gaithersburg, MD 20899 \newline
         USA }
\telephone{+1 (301) 975-2976}
\email{\texttt{pwilson@cme.nist.gov}}
\altowner{Tony Day}
\altaddress{Sikorsky Aircraft}
\comread{This document serves two purposes. Firstly, it provides a description
         of the current \latex{} style file for ISO 10303. Secondly, the source
         can be used as an example of using the \latex{} commands.} % end comread
} % end of STEPcover
\end{verbatim}
Note the use of the \verb|\newline| command instead of \verb|\\| in 
the argument of the \verb|\address| command to indicate a new line. The
\verb|\newline| is needed to ensure satisfactory conversion to HTML.
\end{example} % end example

    The macro \verb|\draftctr|\ixcom{draftctr} generates boilerplate that
may be used in the `Comments to Reader' section of a cover page.
\begin{example}
The \latex{} source \verb|\draftctr This document \ldots| prints:

\draftctr This document \ldots
\end{example}

\sclause{Heading commands}

    The commands described in this subclause specify various `standard'
clause headings.

\ssclause{The Foreword commands}

    The \verb+\Foreword+\ixcom{Foreword} command specifies that a 
table of contents, list of figures and a list of tables be produced. 
Page numbering is roman style and the table of contents starts on page iii.
A new unnumbered clause entitled Foreword is started containing both 
ISO required boilerplate and boilerplate\index{boilerplate}
text specific to ISO 10303.


    Any text may be written after the \verb|\Foreword| command. The
Foreword clause is ended by the 
\verb+\endForeword{+\meta{norm annexes}\verb+}{+\meta{inf annexes}\verb+}+ 
command.\ixcom{endForeword} This command takes two parameters.
\begin{enumerate}
\item \meta{norm annexes} A phrase that starts the sentence 
      `\meta{norm annexes} a normative part of this part \ldots'.
     If there are no normative annexes, then use an empty
     argument (i.e., \verb|{}| with no spaces between the braces).
\item \meta{inf annexes} A phrase that starts the sentence 
     `\meta{inf annexes} for information only.'.
     If there are no informative annexes, then use an
     empty argument.
\end{enumerate}

    The \verb|\endForeword| command produces some additional 
boilerplate\index{boilerplate} text specifically for ISO 10303. 

\begin{example}
The \latex{} source for the Foreword for this document is:
\begin{verbatim}
\Foreword
\fwdshortlist
\endForeword
{Annexes A, B and C are}  % normative annexes
{Annexes D, E and F are} % informative annexes
\end{verbatim}
\end{example} % end example


    The \verb|\fwdshortlist|\ixcom{fwdshortlist} command 
produces boilerplate text for inclusion in the foreword referencing
the STEP parts and series. 
\begin{example}
In this document, the command \verb|\fwdshortlist| prints:

\fwdshortlist
\end{example}

    The \verb|\steptrid|\ixcom{steptrid} command
produces boilerplate text for inclusion in the foreword describing 
the creators of a STEP Technical Report.

\begin{example}
The \latex{} command \verb|\steptrid| in this document prints:
  
\steptrid
\end{example}


\ssclause{The Introduction environment}

    The 
\verb+\begin{Introduction}+\ixenv{Introduction}
environment starts a new unnumbered clause 
entitled Introduction and adds some boilerplate\index{boilerplate}
text specifically for ISO~10303.

\begin{example}
    The following \latex{} source was used to specify the Introduction 
to this document. \label{ex:intro}
\begin{verbatim}
\begin{Introduction}

    This part of ISO 10303 specifies the \latex{} facilities 
specifically designed for use in preparing the various parts of 
this standard.

\begin{majorsublist}
\item the \file{step} package facility;
\item the \file{ir} package facility;
\item the \file{ap} package facility;
\item the \file{aic} package facility;
\item the \file{atc} package facility.
\end{majorsublist}

    This part of ISO 10303 is intended to be used ...

\sclause*{Overview}

    This document describes a set of \latex{} files for use
within ISO~10303 ...

\end{Introduction}
\end{verbatim}
\end{example} % end example


\ssclause{The stepparttitle command}

   The \verb+\stepparttitle{+\meta{part title}\verb+}+\ixcom{stepparttitle}
command produces the title for
an ISO~10303 part, where \meta{part title} is the title of the part.

\begin{anexample}The title for this document was produced using:
\begin{verbatim}
\stepparttitle{Documentation methods:
               LaTeX package files for ISO 10303: User manual}
\end{verbatim}
\end{anexample} % end example


\ssclause{Other headings}

    Most of these commands take no parameters. They start document clauses
with particular titles. The commands that take no parameters are listed
in \tref{tab:noparamhead}. Some of these headings commands have predefined
labels, which are also listed in the table.
\ixcom{partidefhead}
\ixcom{otherdefhead}
\ixcom{introsubhead}
\ixcom{fcandasubhead}
\ixcom{shortnamehead}
\ixcom{picshead}
\ixcom{objreghead}
\ixcom{docidhead}
\ixcom{schemaidhead}
\ixcom{expresshead}
\ixcom{listingshead}
\ixcom{expressghead}
%%\ixcom{modelscopehead}
\ixcom{techdischead}
\ixcom{exampleshead}

\begin{anote}
 In the tables, C = clause, SC = subclause, SSC = subsubclause,
NA = normative annex, IA = informative annex.
\end{anote} % end note

\settowidth{\prwlen}{\quad Protocol Implementation Conformance Statement}
\begin{table}
\centering
\caption{STEP package parameterless heading commands}
\label{tab:noparamhead}
\begin{tabular}{|l|c|p{\prwlen}|l|} \hline
\textbf{Command} & \textbf{Clause} & \textbf{Default text} & \textbf{Label} \\ \hline
\verb|\partidefhead| & SC & Terms defined in ISO 10303-1 &  \\
\verb|\otherdefhead| & SC & Other definitions & \\
\verb|\introsubhead| & SC & Introduction &  \\
\verb|\fcandasubhead| & SC & Fundamental concepts and assumptions & \\
\verb|\shortnamehead| & NA & Short names of entities & \verb|;ssne| \\
\verb|\picshead| & NA & Protocol Implementation Conformance Statement (PICS) proforma & \verb|;spics| \\
\verb|\objreghead| & NA & Information object registration  & \verb|;sior| \\
\verb|\docidhead| & SC & Document identification & \\
\verb|\schemaidhead| & SC & Schema identification &  \\
\verb|\expresshead| & IA & \Express{} listing &  \\
\verb|\listingshead| & IA & Computer interpretable listings & \verb|;scil| \\
\verb|\expressghead| & IA & \ExpressG\ diagrams & \verb|;seg| \\
%%%%\verb|\modelscopehead| & IA & Model scope & \verb|;sms| \\
\verb|\techdischead| & IA & Technical discussions & \verb|;std| \\ 
\verb|\exampleshead| & IA & Examples & \verb|;sex| \\
\hline
\end{tabular}
\end{table}

    The commands listed in \tref{tab:paramhead} are equivalent to the
general sectioning commands, but are intended to indicate the start
of a particular documentation element. These commands take either one
or two parameters. The parameters are denoted in the column headed
`Parameterized title'.
\ixcom{refdefhead}
\ixcom{schemahead}
\ixcom{typehead}
\ixcom{entityhead}
\ixcom{rulehead}
\ixcom{functionhead}
\ixcom{atypehead}
\ixcom{anentityhead}
\ixcom{arulehead}
\ixcom{afunctionhead}
\ixcom{aschemaidhead}
\ixcom{singletypehead}
\ixcom{singleentityhead}
\ixcom{singlerulehead}
\ixcom{singlefunctionhead}

\begin{table}
\centering
\caption{STEP package parameterized heading commands}
\label{tab:paramhead}
\begin{tabular}{|l|c|l|} \hline
\textbf{Command} & \textbf{Clause} & \textbf{Parameterized title} \\ \hline
\verb|\refdefhead| & SC & Terms defined in \meta{ISO ref} \\
\verb|\schemahead| & C & \meta{schema name} \\
\verb|\singletypehead| & SC & \meta{schema name} type definition:
\meta{type name} \\
\verb|\typehead| & SC & \meta{schema name} type definitions \\
\verb|\atypehead| & SSC & \meta{type name} \\
\verb|\singleentityhead| & SC & \meta{schema name} entity definition:
\meta{entity name} \\
\verb|\entityhead| & SC & \meta{schema name} entity definitions \meta{group} \\
\verb|\anentityhead| & SSC & \meta{entity name} \\
\verb|\singlerulehead| & SC & \meta{schema name} rule definition:
\meta{rule name} \\
\verb|\rulehead| & SC & \meta{schema name} rule definitions \\
\verb|\arulehead| & SSC & \meta{rule name} \\
\verb|\singlefunctionhead| & SC & \meta{schema name} function definition:
\meta{function name} \\
\verb|\functionhead| & SC & \meta{schema name} function definitions \\
\verb|\afunctionhead| & SSC & \meta{function name} \\
\verb|\aschemaidhead| & SSC & \meta{schema name} identification \\ \hline
\end{tabular}
\end{table}

\sclause{Miscellaneous commands}

    The following commands provide some printing options for commonly 
occurring situations. The \verb|\nexp{}|\ixcom{nexp} command is intended 
to be used for printing \Express{} \index{express@{\Express}} entity names etc.
\begin{itemize}
\item The command \verb|\B{abc}|\ixcom{B} prints \B{abc}
\item The command \verb|\E{abc}|\ixcom{E} prints \E{abc}
\item The command \verb|\Express|\ixcom{Express} prints \Express{}
\item The command \verb|\ExpressG|\ixcom{ExpressG} prints \ExpressG{}
\item The command \verb|\ExpressI|\ixcom{ExpressI} prints \ExpressI{}
\item The command \verb|\ExpressX|\ixcom{ExpressX} prints \ExpressX{}
\item The command \verb|\BG{|\meta{mathsymbol}\verb|}|\ixcom{BG} prints 
      \meta{mathsymbol} in bold font.
\item The command \verb|\HASH|\ixcom{HASH} prints \HASH{}
\item The command \verb|\LT|\ixcom{LT} prints \LT{}
\item The command \verb|\LE|\ixcom{LE} prints \LE{}
\item The command \verb|\NE|\ixcom{NE} prints \NE{}
\item The command \verb|\INE|\ixcom{INE} prints \INE{}
\item The command \verb|\GE|\ixcom{GE} prints \GE{}
\item The command \verb|\GT|\ixcom{GT} prints \GT{}
\item The command \verb|\CAT|\ixcom{CAT} prints \CAT{}
%\item The command \verb|\HAT|\ixcom{HAT} prints \HAT{}
\item The command \verb|\QUES|\ixcom{QUES} prints \QUES{}
%\item The command \verb|\BS|\ixcom{BS} prints \BS{}
\item The command \verb|\IEQ|\ixcom{IEQ} prints \IEQ{}
\item The command \verb|\INEQ|\ixcom{INEQ} prints \INEQ{}
\item The command \verb|\nexp{an\_entity}|\ixcom{nexp} prints \nexp{an\_entity}
\item The command \verb|\xword{ExpResS\_KeyworD}|\ixcom{xword}
      prints \xword{ExpResS\_KeyworD}
\end{itemize}

The command \verb|\ix{|\meta{word or phrase}\verb|}|\ixcom{ix} both prints 
its parameter and also makes an index entry out of it.

The command \verb|\mnote{|\meta{Marginal note text}\verb|}|\ixcom{mnote}
prints its parameter as a 
marginal note. \mnote{Quite a lot of marginal note text.}
Remember, though, that marginal notes are only printed when the 
\file{isov2}\ixclass{isov2} class \file{draft}\ixopt{draft} option
is used. Marginal notes are not allowed by ISO.

\ssclause{Standard reference commands}

    Many parts of STEP use the same normative or informative references.
The most common of these are provided via commands. The currently available 
commands are listed in \tref{tab:nrefc}.
\ixcom{nrefasni}
\ixcom{nrefparti}
\ixcom{nrefpartxi}
\ixcom{nrefpartxii}
\ixcom{nrefpartxxi}
\ixcom{nrefpartxxii}
\ixcom{nrefpartxxxi}
\ixcom{nrefpartxxxii}
\ixcom{nrefpartxli}
\ixcom{nrefpartxlii}
\ixcom{nrefpartxliii}

    The naming convention used for references to parts of ISO~10303 is to
end the command name with the number of the part expressed in lower case 
Roman numerals. Should further references to parts of ISO~10303 be added later, 
the same naming convention will be used.

\begin{table}
\centering
\caption{Commands for common references to standards} \label{tab:nrefc}
\begin{tabular}{|l|l|} \hline
\textbf{Standard} & \textbf{Command} \\ \hline
ISO/IEC 8824-1 & \verb|\nrefasni| \\
ISO 10303-1    & \verb|\nrefparti|  \\
ISO 10303-11   & \verb|\nrefpartxi|  \\
ISO 10303-12   & \verb|\nrefpartxii|  \\
ISO 10303-21   & \verb|\nrefpartxxi|  \\
ISO 10303-22   & \verb|\nrefpartxxii|  \\
ISO 10303-31   & \verb|\nrefpartxxxi|  \\
ISO 10303-32   & \verb|\nrefpartxxxii|  \\
ISO 10303-41   & \verb|\nrefpartxli|  \\
ISO 10303-42   & \verb|\nrefpartxlii|  \\
ISO 10303-43   & \verb|\nrefpartxliii|  \\ \hline
\end{tabular}
\end{table}


\begin{example} The normative references in this document were input as:
\begin{verbatim}
\begin{nreferences}
\isref{ISO/IEC Directives, Part 2}{Rules for the structure and drafting 
       International Standards, Fourth edition.}
\isref{...}
\nrefparti
\nrefpartxi
\nrefpartxii
\nrefasni
\disref{P. R. WILSON:---}{LaTeX for ISO standards: User manual.}
\end{nreferences}
\end{verbatim}
\end{example}

\begin{anote}
For the commands providing references to STEP parts, the part number 
is denoted by lowercase Roman numerals. Should further reference
commands be provided for other STEP parts, then the same naming scheme
will be used.
\end{anote}

    Some informative bibliographic reference commands are also provided.

The command \verb|\bibidefo|\ixcom{bibidefo} produces the reference
entry to the IDEF0 document and \verb|\brefidefo|\ixcom{brefidefo}
can be used for citing the reference in the body of the document.

The commands \verb|\bibidefix|\ixcom{bibidefix} and
\verb|\bibieeedefix|\ixcom{bibieeedefix} produce the reference entry
to the original FIPS version of IDEF1X and the IEEE version of IDEF1X
respectively. 
The command \verb|\brefidefix|\ixcom{brefidefix} can be used for
citing an IDEF1X reference in the body of the document.

   IDEF0 and IDEF1X are references \brefidefo{} and \brefidefix{}
in the bibliography.


\begin{example} Part of the bibliography for this document looks like:
\begin{verbatim}
\begin{references}
...
\reference{BRYAN, M.,}{SGML --- An Author's Guide to the Standard Generalized
           Markup Language,}{Addison-Wesley Publishing Co., 1988. }\label{bryan}
\bibidefo
\bibieeeidefix
\reference{RESSLER, S.,}{The National PDES Testbed Mail Server User's Guide,}
           {NSTIR 4508, National Institute of Standards and Technology,
           Gaithersburg, MD 20899. January 1991.} \label{ressler}
...
\end{references}
\end{verbatim}
\end{example}

\begin{example}The source for one of the sentences above was:
\begin{verbatim}
IDEF0 and IDEF1X are references \brefidefo{} and \brefidefix{} in the bibliography.
\end{verbatim}
\end{example}

    
\sclause{Commands for documenting EXPRESS code} \index{express@\Express\}


    The Supplementary Directives\index{SD} specify the layout of the 
documentation of \Express{} code. The following commands are intended 
to serve two purposes:
\begin{enumerate}
\item To provide environments for the documentation of entity 
      attributes, etc.;
\item To provide begin and end tags around all the \Express{} code 
      documentation.
\end{enumerate}

    This latter purpose is to provide an enabling capability for the 
automatic extraction of portions of the documentation of an 
\Express{} model so that they could be placed into another document. 
For example, tools could be developed that would automatically extract 
pieces of resource model documentation and place them into an AP document.

\begin{anote}
This document uses the \file{hyphenat}\ixpack{hyphenat} 
package which enables automatic hyphenation of `words'
containing the underscore character command 
%(\verb|\_|\index{_/@\verb|\_|}). 
(\verb|\_|\index{_/@\texttt{\bs\_}}). 
Such words would normally have to
be coded as \verb|long\_\-word| to ensure potential hyphenation 
at the position of the underscore. When using the \file{hyphenat} package
it is an error to put the \verb|\-|\ixcom{-} discretionary
hyphen command after the underscore command as this then stops further
hyphenation.
\end{anote}


\ssclause{Environments ecode, eicode and excode}

    The \verb|ecode|\ixenv{ecode} environment is for 
tagging \Express{} code. It prints the appropriate title
and sets up the relevant fonts.

\begin{anexample} The following \latex{} source code:
\begin{verbatim}
\begin{ecode}\ixent{an\_entity}
\begin{verbatm}  % read verbatm as verbatim
*)
ENTITY an_entity;
  attr : REAL;
END_ENTITY;
(*
\end{verbatm}   % read verbatm as verbatim
\end{ecode}
\end{verbatim}

produces:

\begin{ecode}\ixent{an\_entity}
\begin{verbatim}
*)
ENTITY an_entity;
  attr : REAL;
END_ENTITY;
(*
\end{verbatim}
\end{ecode}
\end{anexample} % end example

    Similarly, the \verb|eicode|\ixenv{eicode} and
\verb|excode|\ixenv{excode} environments are for tagging \ExpressI{} 
and \ExpressX{} code and setting up the relevant titles and fonts.


\ssclause{Environment attrlist}

    The \verb|attrlist|\ixenv{attrlist} environment produces 
the heading for attribute definitions and sets up 
a \verb|description|\ixenv{description} list.

\begin{anexample}The following \latex{} source code:
\begin{verbatim}
\begin{attrlist}
\item[attr\_1] The \ldots
\item[attr\_2] This \ldots
\end{attrlist}
\end{verbatim}

produces:

\begin{attrlist}
\item[attr\_1] The \ldots
\item[attr\_2] This \ldots
\end{attrlist}
\end{anexample} % end example

\ssclause{Environment fproplist}

    The \verb|fproplist|\ixenv{fproplist} environment is similar to 
\verb|attrlist|\ixenv{attrlist} except that it is for
formal propositions.

\begin{anexample}The following \latex{} source code:
\begin{verbatim}
\begin{fproplist}
\item[un\_1] The value of \ldots\ shall be unique.
\item[gt\_0] The value of \ldots\ shall be greater than zero.
\end{fproplist}
\end{verbatim}

produces:

\begin{fproplist}
\item[un\_1] The value of \ldots\ shall be unique.
\item[gt\_0] The value of \ldots\ shall be greater than zero.
\end{fproplist}
\end{anexample} % end example

\ssclause{Other listing environments}

    The environments \verb|iproplist|\ixenv{iproplist}, 
\verb|enumlist|\ixenv{enumlist}, and \verb|arglist|\ixenv{arglist} are
similar to \verb|attrlist|\ixenv{attrlist}.
 Respectively they are environments for
informal propositions, enumerated items, and argument definitions.

\ssclause{Indexing}

    The command \verb|\ixent{|\meta{entity}\verb|}|\ixcom{ixent} 
generates an index
entry for the entity \meta{entity}.

    There are similar macros, each of which takes the name of the 
declaration as its argument, for indexing the other \Express{} declarations:
\verb|\ixenum|\ixcom{ixenum} for enumeration,
\verb|\ixfun|\ixcom{ixfun} for function,
\verb|\ixproc|\ixcom{ixproc} for procedure,
\verb|\ixrule|\ixcom{ixrule} for rule,
\verb|\ixsc|\ixcom{ixsc} for subtype\_constraint,
\verb|\ixschema|\ixcom{ixschema} for schema,
\verb|\ixselect|\ixcom{ixselect} for select, and
\verb|\ixtype|\ixcom{ixtype} for type.

\ssclause{Documentation tagging}

    Several environments are defined to tag the general documentation 
of \Express{} code. \index{express@\Express\}

    The environment \verb+\begin{espec}{+\meta{name}\verb+}+\ixenv{espec}
may be used to enclose, and give a name to, a complete specification 
block for an \Express{} entity. There are analogous environments --- 
\verb+fspec+\ixenv{fspec}, 
\verb+rspec+\ixenv{rspec}, 
\verb+sspec+\ixenv{sspec}, and
\verb+tspec+\ixenv{tspec} --- 
for functions, rules, schemas and types respectively.

    The \verb|dtext|\ixenv{dtext} environment may be used to anonymously 
enclose descriptive text.

\begin{example}\label{ex:code} Here is the suggested tagged documentation 
style for part of an \Express{} model.
\begin{verbatim}
%\ssclause{committee\_def}
\begin{espec}{committee_def}
\begin{dtext}
    A committee is composed of an odd number of people. 
Each committee also has a name.
    The ideal size of a committee is less than three.

\begin{anote} Figures and tables may also be placed here. \end{anote} % end note
\end{dtext}
\begin{ecode}\ixent{committee\_def}
\begin{verbatm} % read verbatm as verbatim
*)
ENTITY committee_def;
  title   : name;
  members : SET [1:?] OF person;
DERIVE
  ideal : BOOLEAN := SIZEOF(members) = 1;
UNIQUE
  un1 : title;
WHERE
  odd_members : ODD(SIZEOF(members));
END_ENTITY;
(*
\end{verbatm}   % read verbatm as verbatim
\end{ecode}
\begin{attrlist}
\item[title] The name of the committee.
\item[members] The people who form the committee.
\item[ideal] TRUE if there is only one person 
             on the committee.
             That is, if the committee is the ideal size.
\end{attrlist}
\begin{fproplist}
\item[un1] The \nexp{title} of the committee shall be unique.
\item[odd\_members] There shall be an odd number of people 
                    on the committee.
\end{fproplist}
\begin{iproplist}
\item[chair] The members of a committee shall appoint one of 
             their number as
             chair of the committee.
\end{iproplist}
\end{espec}
\end{verbatim}
\end{example} % end example

\begin{example}
The code in \eref{ex:code} produces the following result:

\begin{espec}{committee_def}
\begin{dtext}
    A committee is composed of an odd number of people. 
Each committee also has a name.
The ideal size of a committee is less than three.

\begin{anote} Figures and tables may also be placed here. \end{anote} % end note
\end{dtext}
\begin{ecode}\ixent{committee\_def}
\begin{verbatim}
*)
ENTITY committee_def;
  title   : name;
  members : SET [1:?] OF person;
DERIVE
  ideal : BOOLEAN := SIZEOF(members) = 1;
UNIQUE
  un1 : title;
WHERE
  odd_members : ODD(SIZEOF(members));
END_ENTITY;
(*
\end{verbatim}   % read verbatm as verbatim
\end{ecode}
\begin{attrlist}
\item[title] The name of the committee.
\item[members] The people who form the committee.
\item[ideal] TRUE if there is only one person on the committee. That is, if
             the committee is the ideal size.
\end{attrlist}
\begin{fproplist}
\item[un1] The \nexp{title} of the committee shall be unique.
\item[odd\_members] There shall be an odd number of people on the committee.
\end{fproplist}
\begin{iproplist}
\item[chair] The members of a committee shall appoint one of their number as
             chair of the committee.
\end{iproplist}
\end{espec}

\end{example} % end example


\sclause{Commands producing boilerplate text} \index{boilerplate}

    The following commands produce boilerplate text as specified by the 
Supplementary Directives\index{SD}.

\begin{anote}
 In the examples, 
the parameters of those commands that
take them have been specified in 
\textit{this font style} so their effects can
be seen in the resulting printed text.
\end{anote}

\ssclause{Definition of \ExpressG}

    The \verb|\expressgdef|\ixcom{expressgdef} prints the boilerplate
for where the definition of \ExpressG{} can be found.

\begin{anexample}
The command \verb|\expressgdef| prints: 

\expressgdef
\end{anexample} 

\ssclause{Major subdivision listing}

    The \verb|majorsublist|\ixenv{majorsublist}
environment prints the boilerplate for the heading of a listing of
major subdivisions of the standard and starts an itemized list.
An illustration of its use is given in \eref{ex:intro} 
on page~\pageref{ex:intro}.

The heading text is produced by the 
\verb|\majorsubname|\ixcom{majorsubname} command.

\begin{anexample} The command \verb|\majorsubname| command prints:

\majorsubname

\end{anexample}

\ssclause{Schema introduction}

    The command \verb|\schemahead{|\meta{schema name}\verb|}|\ixcom{schemahead} prints the heading for a schema clause.

    The command \verb+\schemaintro{+\meta{schema name}\verb+}+\ixcom{schemaintro} 
produces the boilerplate for the introduction to an \Express{} schema
clause.

\begin{anexample}The command \verb|\schemaintro{\nexp{this\_schema}}| prints:

\schemaintro{\nexp{this\_schema}}
\end{anexample}
  


\ssclause{Short names of entities}

    The command \verb|\shortnamehead|\ixcom{shortnamehead} prints the
heading for the short names annex.

    The command \verb|\shortnames|\ixcom{shortnames} 
produces the boilerplate for the
introduction to the annex listing short names.

\begin{anexample}The command \verb|\shortnames| prints:

\shortnames 
\end{anexample} %end example

\ssclause{Registration commands}

    The command \verb|\objreghead|\ixcom{objreghead} prints the heading
for the information object registration annex.

    The command \verb|\docidhead|\ixcom{docidhead} prints the heading
for the document identification subclause.


    The command 
\verb+\docreg{+\meta{version no}\verb+}+\ixcom{docreg}
produces the boilerplate for document registration. The command takes
one parameter:
\meta{version no} which is the version number.\footnote{The
SD say that the version number should be 1 for a first edition IS.
The version number is incremented by one for each corrigenda,
amendment or new edition.}

\begin{example}The command \verb|\docreg{1}|
         prints:

\docreg{\textit{1}} 
\end{example} % end example

    The command \verb|\schemaidhead|\ixcom{schemaidhead} prints the heading
for the schema identification subclause. 
The command 
\verb|\aschemaidhead{|\meta{schema name}\verb|}|\ixcom{aschemaidhead} 
prints the heading for a particular schema identification subsubclause.

    The command
\verb+\schemareg{+\meta{version no}\verb+}{+\meta{p2}\verb+}{+\meta{p3}\verb+}{+\meta{p4}\verb+}{+\meta{p5}\verb+}{+\meta{p6}\verb+}+\ixcom{schemareg} produces the boilerplate concerning
schema registration. The command takes six parameters.
\begin{enumerate}
\item \meta{version no} The version number;
\item \meta{p2} The name of an \Express{} schema (with underscores);
\item \meta{p3} The number of the schema object (typically 1);
\item \meta{p4} The name of the schema, with hyphens replacing any
                  underscores in the name;
\item \meta{p5} The number identifying the schema;
\item \meta{p6} The clause or annex in which the schema is defined.
\end{enumerate}

\begin{example}The command \\
 \verb|\schemareg{1}{a\_schema}{3}{a-schema}{5}{clause 6}|
prints:

\schemareg{\textit{1}}{a\_schema}{\textit{3}}{\textit{a-schema}}{\textit{5}}{\textit{clause 6}}
\end{example} % end example


\ssclause{Computer interpretable listings} 

    The command \verb|\listingshead|\ixcom{listingshead} prints the
heading for the computer interpretable listings annex.

    The command 
\verb|\expurls{|\meta{short}\verb|}{|\meta{express}\verb|}|\ixcom{expurls}
produces the boilerplate for the introduction to the annex 
listing short names and \Express, where \meta{short} is the URL for the short
names and \meta{express} is the URL for the \Express.

\begin{anexample} The command 
  \verb|\expurls{http:/www.short/}{http://www.express/}| prints:

\expurls{http://www.short/}{http://www.express/}

\end{anexample}

\clearpage
\clause{The \file{ir} package facility} 

    The \file{ir}\ixpack{ir} package provides commands and environments
specifically for the ISO~10303 Integrated Resources series of documents.

    Use of this package requires the use of the \file{step}\ixpack{step} 
package.

\sclause{Boilerplate commands}

    The \file{ir} package modifies the \verb|\fwdshortlist|\ixcom{fwdshortlist}
command to produce extra IR-specific boilerplate.

    The following commands produce boilerplate text as specified by the 
SD\index{SD}.


\ssclause{Integrated resource EXPRESS-G} 

    The command \verb|\expressghead|\ixcom{expressghead} prints the
heading for the \ExpressG{} diagrams annex.

    The command \verb+\irexpressg+\ixcom{irexpressg} 
produces the boilerplate for the introduction to the integrated 
resource \ExpressG{} annex.
\index{expressg@\ExpressG\}

\begin{anexample}The command \verb|\irexpressg| prints:

\irexpressg

 \end{anexample} % end example

%%%%%%%%%%%%%%%%%%%%%%%%%%%%%%
%%%%\end{document}
%%%%%%%%%%%%%%%%%%%%%%%%%%%%%%



\clearpage
\clause{The \file{ap} package facility}

    The \file{ap}\ixpack{ap} package provides commands and environments
specifically for the ISO~10303 Application Protocol series of documents.

    Use of this package requires the use of the \file{step}\ixpack{step} 
package.

\sclause{Preamble commands}

    Certain commands shall be put in the preamble of an AP document.

    The command 
\verb+\aptitle{+\meta{title of AP}\verb+}+\ixcom{aptitle}
shall be put into the preamble. \index{preamble} The parameter shall be of 
such a form that
it will read naturally in a sentence of the form: 
`\ldots for the \meta{title of AP} application protocol.'.

\begin{anexample}
  For the purposes of later examples, the command
\verb|\aptitle{|\texttt{\theap}\verb|}| has been put in the preamble
of this document.
\end{anexample} % end example

    If the AP makes use of one or more
AICs\index{AIC}, then the command \verb|\aicinaptrue|\ixcom{aicinaptrue} 
shall be put in the document preamble.

   If a mapping specification is used instead of a mapping table,
the command \verb|\mapspectrue|\ixcom{mapspectrue} shall be put
in the preamble. If mapping templates are used then 
\verb|\maptemplatetrue|\ixcom{maptemplatetrue} shall also be put in the
preamble.

    If IDEF1X is used instead of \ExpressG{} as the graphical form for the
ARM, then \verb|\idefixtrue|\ixcom{idefixtrue} shall be put in the preamble.

  

\sclause{Heading commands}

    These commands start document clauses with particular titles. The
commands that take no parameters are listed in \tref{tab:apnpheads}.
Some of these commands have predefined labels, which are also listed in 
the table.
\ixcom{inforeqhead}
\ixcom{uofhead}
\ixcom{applobjhead}
\ixcom{applasserthead}
\ixcom{aimhead}
\ixcom{maptablehead}
\ixcom{templateshead}
\ixcom{aimshortexphead}
\ixcom{confreqhead}
\ixcom{aimlongexphead}
\ixcom{aimshortnameshead}
\ixcom{impreqhead}
\ixcom{aamhead}
\ixcom{aamdefhead}
\ixcom{aamfighead}
\ixcom{armhead}
\ixcom{aimexpressghead}
\ixcom{aimexpresshead}
\ixcom{apusagehead}

\settowidth{\prwlen}{\quad Application activity model definitions}
\begin{table}
\centering
\caption{AP package parameterless heading commands}
\label{tab:apnpheads}
\begin{tabular}{|l|c|p{\prwlen}|l|} \hline
\textbf{Command} & \textbf{Clause} & \textbf{Default text} & \textbf{Label} \\ \hline
\verb|\inforeqhead| & C & Information requirements & \verb|;sireq| \\
\verb|\uofhead| & SC & Units of functionality  & \verb|;suof| \\
\verb|\applobjhead| & SC & Application objects  & \verb|;sao| \\
\verb|\applasserthead| & SC & Application assertions  & \verb|;saa| \\
\verb|\aimhead| & C & Application interpreted model & \verb|;saim| \\
\verb|\mappinghead| & SC & Mapping table, or & \verb|;smap| \\
                    &    & Mapping specification  & \verb|;smap| \\
\verb|\templateshead| & SSC & Mapping templates &    \\
\verb|\aimshortexphead| & SC & AIM \Express{} short listing & \verb|;saesl| \\
\verb|\confreqheadhead| & C & Conformance requirements & \verb|;scr| \\
\verb|\aimlongexphead| & NA & AIM \Express{} expanded listing  & \verb|;saeel| \\
\verb|\aimshortnameshead| & NA & AIM short names  & \verb|;sasn| \\
\verb|\impreqhead| & NA & Implementation method specific requirements & \verb|;simreq| \\
\verb|\aamhead| & IA & Application activity model & \verb|;saam| \\
\verb|\aamdefhead| & SC & Application activity model definitions and abbreviations & \verb|| \\
\verb|\aamfighead| & SC & Application activity model diagrams & \verb|| \\
\verb|\armhead| & IA & Application reference model & \verb|;sarm| \\
\verb|\aimexpressghead| & IA & AIM \ExpressG{} & \verb|;saeg| \\
\verb|\aimexpresshead| & IA & AIM \Express{} listing & \verb|| \\
\verb|\apusagehead| & IA & Application protocol usage guide & \verb|;sapug| \\
 \hline
\end{tabular}
\end{table}

    The commands listed in \tref{tab:appheads} take parameters.
\ixcom{auofhead}
\ixcom{mapuofhead}
\ixcom{mapobjecthead}
\ixcom{mapattributehead}

\begin{table}
\centering
\caption{AP package parameterized heading commands}
\label{tab:appheads}
\begin{tabular}{|l|c|l|} \hline
\textbf{Command} & \textbf{Clause} & \textbf{Parameterized title} \\ \hline
\verb|\auofhead| & SSC & \meta{UoF} \\ 
\verb|\mapuofhead| & SSC & \meta{UoF} \\
\verb|\mapobjecthead| & SSSC & \meta{application object} \\
\verb|\mapattribhead| & SSSSC & \meta{attribute} \\
\hline
\end{tabular}
\end{table}

\sclause{Boilerplate commands}

    The following commands produce boilerplate text as specified by the 
SD\index{SD}.

\begin{anote}
 In the examples, the parameters of those commands that
take them have been specified in 
\textit{this font style} so their effects can
be seen in the resulting printed text.
\end{anote}

\ssclause{AP introduction}

    The command \verb|\apextraintro|\ixcom{apextraintro} produces extra
boilerplate for the Introduction to an AP.

\begin{anexample}The command \verb|\apextraintro| prints:

\apextraintro
\end{anexample} %end example

\ssclause{AP scope}

    The command \verb+\apscope{+\meta{application purpose and context}\verb+}+\ixcom{apscope} 
produces the boilerplate for the start of an AP scope\index{scope} clause.

\begin{anexample}The command \verb|\apscope{application purpose and context.}|
         prints:

\apscope{\textit{application purpose and context.}} 
\end{anexample} 

\ssclause{AP information requirements}

  The command \verb|\inforeqhead|\ixcom{inforeqhead} prints the
heading for the information requirements clause.

  The command \verb+\apinforeq{+\meta{AP purpose}\verb+}+\ixcom{apinforeq} 
produces the boilerplate for the clause.

\begin{anexample}The command \verb|\apinforeq{AP purpose.}| prints: 

\apinforeq{\textit{AP purpose.}} 
\end{anexample} % end example

\ssclause{AP UoF}

    The command \verb|\uofhead|\ixcom{uofhead} prints the heading
for the UoF subclause.

    The environment 
\verb+\begin{apuof}+\meta{item list}\verb+\end{apuof}+\ixenv{apuof} 
produces the boilerplate for the introduction to the clause.

\begin{anexample} Remembering that \verb|\aptitle|\ixcom{aptitle}
                  was set to \texttt{\theap} in the preamble,
                  the commands
\begin{verbatim}
\begin{apuof}
\item Name of UoF1;
\item Name of UoF2;
\item Name of UoFn.
\end{apuof}
\end{verbatim}
prints:

\begin{apuof}
\item Name of UoF1;
\item Name of UoF2;
\item Name of UoFn.
\end{apuof}

\end{anexample}

\ssclause{AP application objects}

    The command \verb|\applobjhead|\ixcom{applobjhead} prints the
heading for the application objects subclause.

    The command \verb|\apapplobj|\ixcom{apapplobj} produces the 
boilerplate for the introduction to the clause.

\begin{anexample} Remembering that \verb|\aptitle|\ixcom{aptitle}
                  was set to \texttt{\theap} in the preamble,
                  the command \verb|\apapplobj| prints:

\apapplobj

\end{anexample}

\ssclause{AP assertions}

    The command \verb|\applasserthead|\ixcom{applasserthead} prints the
heading for the application assertions subclause.

    The command \verb|\apassert|\ixcom{apassert}
produces the boilerplate for the clause.

\begin{anexample} Remembering that \verb|\aptitle|\ixcom{aptitle}
                  was set to \texttt{\theap} in the preamble,
                  the command \verb|\apassert| prints:

\apassert

\end{anexample}


\ssclause{AP mapping table/specification}

    The command \verb|\mappinghead|\ixcom{mappinghead} prints
the heading for the mapping table or mapping specification subclause.
The heading text depends on whether or not 
\verb|\mapspectrue|\ixcom{mapspectrue} was put in the preamble.

    The command \verb|\apmapping|\ixcom{apmapping} 
produces the boilerplate for the introduction to the AP mapping table
or specification clause.

\begin{anote}AICs are included in the boilerplate only if the command
\verb|\aicinaptrue|\ixcom{aicinaptrue} is included
in the preamble.
\end{anote}

\begin{example}By default, or when \verb|\mapspecfalse| is in
the preamble, the command \verb|\apmapping|
         prints: \mapspecfalse

\apmapping
\end{example} % end example

\begin{example}When \verb|\mapspectrue| is in the preamble, the command \verb|\apmapping|
         prints: \mapspectrue

\apmapping
\end{example} % end example

\sssclause{AP mapping templates}

    The command \verb|\aptemplatehead|\ixcom{aptemplatehead} prints
the heading for the mapping template subclause (if any).

    The command \verb|\apmaptemplate|\ixcom{apmaptemplate} prints
the boilerplate for the introduction to the clause. This refers to the
UoFs in the AP. The first of the UoFs shall be labelled as 
\verb|\label{;uof1}| and the last of the UoFs shall be
labelled as \verb|\label{;uoflast}|.

\begin{example} If there are three UoFs, then there should be headings
of the form:
\begin{verbatim}
\mapuofhead{First UoF}\label{;uof1}
...
\mapuofhead{Second UoF}...
...
\mapuofhead{Third UoF}\label{;uoflast}
...
\end{verbatim}
\end{example}

\begin{example} Assuming that there are three UoFs as in the previous example, 
the command \verb|\apmaptemplate| prints:

\apmaptemplate
\end{example} % end example

    The command \verb|\sstemplates|\ixcom{sstemplates} prints
the two subclauses for the \xword{subtype} and \xword{SuPeRtype} templates.

\begin{example} \label{ex:sstemplates} In this document, 
and noting that the clause
numbering is not the same as in a real AP document, 
the command \verb|\sstemplates|
         prints:

\sstemplates

\end{example} % end example


\sssclause{Template headings}

    There are three headings used within a mapping template.

    The command \verb|\signature|\ixcom{signature} prints the underlined
Mapping signature header.

    The command \verb|\parameters|\ixcom{parameters} prints the underlined
Parameter definition header.

    The command \verb|\body|\ixcom{body} prints the underlined
Template body header.

\begin{anexample} The results of using the \verb|\signature|\ixcom{signature}
and \verb|\parameters|\ixcom{parameters} commands were illustrated
in \eref{ex:sstemplates} on \pref{ex:sstemplates}.
\end{anexample}



\ssclause{AIM short EXPRESS listing}

    The command \verb|\aimshortexphead|\ixcom{aimshortexphead} prints
the heading for the AIM EXPRESS short listing subclause.

    The command \verb|\apshortexpress|\ixcom{apshortexpress} produces 
the boilerplate for the
first paragraph of the clause.

\begin{anote}AICs are included in the boilerplate only if the command
\verb|\aicinaptrue|\ixcom{aicinaptrue} is included in the preamble.
\end{anote}

\begin{example}
The command \verb|\apshortexpress| without \verb|\aicinaptrue|
in the preamble produces:

\aicinapfalse
\apshortexpress

\end{example} % end example

\begin{example}
With \verb|\aicinaptrue| set in the preamble the command
\verb|\apshortexpress| produces the following:

\aicinaptrue
\apshortexpress
\end{example} % end example


\ssclause{AP conformance}

    The command \verb|\confreqhead|\ixcom{confreqhead} prints the
heading for the conformance requirements clause.

    The command 
\verb+\apconformance{+\meta{implementation methods}\verb+}+\ixcom{apconformance} 
produces the boilerplate for the introduction to the clause.

    The environment 
\verb+\begin{apconformclasses}+\meta{item list}\verb+\end{apconformclasses}+\ixenv{apconformclasses} 
provides some additional boilerplate.

\begin{example}The command \verb|\apconformance{ISO 10303-21, ISO 10303-22}|
         prints:

\apconformance{\textit{ISO 10303-21, ISO 10303-22}} 
\end{example} % end example

\begin{example}The commands
  \begin{verbatim}
\begin{apconformclasses}
\item first class;
\item second class;
\item last class.
\end{apconformclasses}
\end{verbatim}
         print:

\begin{apconformclasses}
\item first class;
\item second class;
\item last class.
\end{apconformclasses}
\end{example}


\ssclause{EXPRESS expanded listing}

    The command \verb|\aimlongexphead|\ixcom{aimlongexphead} prints
the heading for the AIM expanded listing clause.

    The command \verb|\aimlongexp|\ixcom{aimlongexp} 
produces the boilerplate for the introduction to the clause.

\begin{anexample}The command \verb|\aimlongexp|
         prints:

\aimlongexp 
\end{anexample} % end example

\ssclause{AIM short names}

    The command \verb|\aimshortnamehead|\ixcom{aimshortnamehead} prints
the heading for the AIM short names annex.

    The command \verb|\apshortnames|\ixcom{apshortnames} 
produces the boilerplate for the introduction to the AP short name annex.

\begin{anexample}The command \verb|\apshortnames|
         prints:

\apshortnames 
\end{anexample} % end example

\ssclause{Implementation requirements}

    the command \verb|\impreqhead|\ixcom{impreqhead} prints the heading
for implementation method-specific reguirements.

    The command \verb+\apimpreq{+\meta{schema name}\verb+}+\ixcom{apimpreq}
produces the boilerplate for the requirements on exchange structure.

\begin{anexample}The command \verb|\apimpreq{schema\_name}|
         prints:

\apimpreq{\textit{schema\_name}} 
\end{anexample} % end example


\ssclause{AP PICS}

    The command \verb|\picshead|\ixcom{picshead}, 
from the \file{step}\ixpack{step} package,
prints the heading for the PICS annex.

    The command \verb|\picsannex|\ixcom{picsannex}
produces the boilerplate for the start of the AP PICS annex.

\begin{anexample}The command \verb|\picsannex|
         prints:

\picsannex 
\end{anexample} % end example

\ssclause{AAM annex}

    The command \verb|\aamhead|\ixcom{aamhead} prints the heading for
the AAM annex.


    The command \verb|\apaamintro|\ixcom{apaamintro} 
 produces the introductory boilerplate for the introduction of
the AP annex on application activity models.

\begin{anexample}
  The command \verb|\apaamintro| prints:

\apaamintro

\end{anexample} % end example

\ssclause{AP AAM definitions}

    The command \verb|\aamdefhead|\ixcom{aamdefhead} prints the heading
for the AAM definitions subclause.

    The command \verb|\apaamdefs|\ixcom{apaamdefs} produces 
the boilerplate at the start of
the AP subclause on AAM definitions and abbreviations.

\begin{anexample}
  The command \verb|\apaamdefs| prints:

\apaamdefs
\end{anexample} % end example

\ssclause{AAM diagrams annex}

    The command \verb|\aamfighead|\ixcom{aamfighead} prints the heading
for the AAM diagrams subclause.

    The command 
\verb|\aamfigrange{|\meta{figure range}\verb|}|\ixcom{aamfigrange} 
is used to store the activity model diagram figure range for later use.

\begin{example}
    For the purposes of this document we set
\begin{verbatim}
\aamfigrange{figures F.1 through F.n}
\end{verbatim}

\aamfigrange{\textit{figures F.1 through F.n}}

\end{example}

    The command \verb+\aamfigures+\ixcom{aamfigures}
produces the boilerplate for the introduction to an APs AAM figure
subclause.

\begin{example} Noting that we have set 
\verb|\aamfigrange{figures F.1 through F.n}|\ixcom{aamfigrange}, 
the command \verb|\aamfigures| prints:

\aamfigures

\end{example}

\ssclause{ARM annex}

    The command \verb|\armhead|\ixcom{armhead} prints the heading for the
ARM annex.

    The command 
\verb+\armintro+\ixcom{armintro}
produces the boilerplate for the introduction to the ARM figures.

\begin{anexample}The command 
          \verb|\armintro| 
         prints:

\armintro 
\end{anexample} % end example

\ssclause{AIM EXPRESS-G annex}

    The command \verb|\aimexpressghead|\ixcom{aimexpressghead} 
prints the heading for the AIM \ExpressG{} annex.
 

    The command 
\verb+\aimexpressg+\ixcom{aimexpressg}
produces the boilerplate for the introduction to an AP's AIM \ExpressG{}
model. 

\begin{anexample}The command \verb|\aimexpressg|
         prints:

\aimexpressg
\end{anexample} % end example

\ssclause{AIM EXPRESS listing}

    The command \verb|\aimexpresshead|\ixcom{aimexpresshead} prints
the heading for the AIM listing annex.

%    The command \verb|\aimexplisting|\ixcom{aimexplisting}
%produces the boilerplate for the introduction to an AIMs short name and
%\Express{} listing.
%
%
%\begin{example}The command \verb|\aimexplisting|
%         prints:
%
%\aimexplisting 
%\end{example}

    The command 
\verb|\apexpurls{|\meta{short}\verb|}{|\meta{express}\verb|]|\ixcom{apexpurls}
produces the boilerplate for the introduction to the AP annex
listing short names and \Express, where \meta{short} is the URL for the short
names and \meta{express} is the URL for the \Express.

\begin{anexample} The command \verb|\apexpurls{http:/www.short/}{http://www.express/}|
prints:

\apexpurls{http://www.short/}{http://www.express/}

\end{anexample}

%%%%%%%%%%%%%%%%%%%%%%%%%%%%%%
%%%%%%\end{document}
%%%%%%%%%%%%%%%%%%%%%%%%%%%%%%



\clearpage
\clause{The \file{aic} package facility}

    The \file{aic}\ixpack{aic} package
provides commands and environments specifically
for the ISO~10303 Application Interpreted Construct series of
documents.

    The use of this package requires the use of the 
\file{step}\ixpack{step} package.

\sclause{Heading commands}

    The commands described in this subclause start document clauses with
particular titles.

    The commands that take no parameters are listed in \tref{tab:aicnpheads}.
\ixcom{aicshortexphead}

\begin{table}[btp]
\centering
\caption{AIC package parameterless heading commands}
\label{tab:aicnpheads}
\begin{tabular}{|l|c|l|l|} \hline
\textbf{Command} & \textbf{Clause} & \textbf{Default text} & \textbf{Label} \\ \hline
\verb|\aicshortexphead| & C & \Express{} short listing & \verb|;sesl| \\
\hline
\end{tabular}
\end{table}

\sclause{Boilerplate commands}

    The following commands produce boilerplate text as specified by the
Supplementary Directives. 


\ssclause{Introduction text}

    The command \verb|\aicextraintro|\ixcom{aicextraintro}
prints additional boilerplate for the Introduction to an AIC.

\begin{anexample}The command \verb|\aicextraintro|
         prints:

\aicextraintro
\end{anexample}

\ssclause{Definition of AIC}

    The command \verb|\aicdef|\ixcom{aicdef}
prints the definition of `AIC'. It shall only be used within the
\verb|definitions|\ixenv{definitions} environment.

\begin{anexample}The commands:
         \begin{verbatim}
         \begin{definitions}
         \aicdef
         \end{definitions}
         \end{verbatim}
         produce:

\begin{definitions}
\aicdef
\end{definitions}
\end{anexample} % end example

\ssclause{Short EXPRESS listing}

    The command \verb|\aicshortexphead|\ixcom{aicshortexphead} prints
the heading for the AIC short \Express{} annex.

    The command \verb|\aicshortexpintro|\ixcom{aicshortexpintro}
prints boilerplate for the introduction to the short \Express{} listing.

\begin{anexample}The command \verb|\aicshortexpintro|
         prints:

\aicshortexpintro  
\end{anexample} % end example

\ssclause{EXPRESS-G figures}

    The command \verb|\expressghead|\ixcom{expressghead}, 
from the \file{step} package, prints the heading for the \ExpressG{} diagrams
annex.

    The command 
\verb+\aicexpressg+\ixcom{aicexpressg} 
prints boilerplate for the introduction to the \ExpressG\ figures.

\begin{anexample}The command \verb|\aicexpressg|
         prints:

\aicexpressg
\end{anexample}

%%%%%%%%%%%%%%%%%%%%%%%%%
%%%\end{document}
%%%%%%%%%%%%%%%%%%%%%%%%%

\clearpage
\clause{The \file{ats} package facility}

    The \file{ats}\ixpack{ats} package
provides commands and environments specifically
for the ISO~10303 Abstract Test Suite series of
documents.

    The use of this package requires the use of the 
\file{step}\ixpack{step} package.

\sclause{Preamble commands}

    Certain commands shall be put in the preamble\index{preamble} 
of an ATS document.

    The command 
\verb+\APnumber{+\meta{number}\verb+}+\ixcom{APnumber} shall be put 
in the preamble,
where \meta{number} is the ISO 10303 part number of the corresponding AP.

\begin{example}
For the purposes of later examples, the command
\verb+\APnumber{+\texttt{\theAPpartno}\verb+}+ has been put in the preamble.
of this document.
\end{example}

    The command 
\verb+\APtitle{+\meta{title of AP}\verb+}+\ixcom{APtitle} shall be put 
in the preamble,
where \meta{title of AP} is the ISO 10303 part title of the
corresponding AP. This must be given in such a manner that it reads
sensibly in a sentence of the form `\ldots for ISO 10303-299,
application protocol \meta{title of AP}.'

\begin{example}
For the purposes of later examples, the command
\verb+\APtitle{+\texttt{\theAPtitle}\verb+}+ 
has been put in the preamble of this document.
\end{example}

    The command
\verb+\mapspectrue+\ixcom{mapspectrue}
shall be put in the preamble if the AP uses a mapping specification rather
than a mapping table.

\sclause{Heading commands}

    These commands start document clauses with particular titles.
The commands that take no parameters are listed in \tref{tab:atshead}.
\ixcom{purposeshead}
\ixcom{domainpurposehead}
\ixcom{aepurposehead}
\ixcom{apobjhead}
\ixcom{apasserthead}
\ixcom{aimpurposehead}
%%%\ixcom{extrefpurposehead}
\ixcom{implementpurposehead}
%%%\ixcom{rulepurposehead}
\ixcom{otherpurposehead}
\ixcom{gtpvchead}
\ixcom{generalpurposehead}
\ixcom{gvcatchead}
\ixcom{gvcprehead}
\ixcom{gvcposthead}
\ixcom{atchead}
\ixcom{prehead}
\ixcom{posthead}
\ixcom{confclassannexhead}
\ixcom{postipfilehead}
%%%\ixcom{excludepurposehead}
\ixcom{atsusagehead}

\settowidth{\prwlen}{\quad General verdict criteria for all abstract}
\begin{table}
\centering
\caption{ATS package parameterless heading commands} \label{tab:atshead}
\begin{tabular}{|l|c|p{\prwlen}|} \hline
\textbf{Command}             & \textbf{Clause} & \textbf{Default text} \\ \hline
\verb|\purposeshead|         & C   & Test purposes  \\
\verb|\aepurposehead|        & SC  & Application element test purposes \\
\verb|\aimpurposehead|       & SC  & AIM test purposes \\
\verb|\implementpurposehead| & SC  & Implementation method test purposes \\
\verb|\domainpurposehead|    & SC  & Domain test purposes \\
\verb|\otherpurposehead|     & SC  & Other test purposes \\

\verb|\gtpvchead|            & C   & General test purposes and verdict criteria \\
\verb|\generalpurposehead|   & SC  & General test purposes \\
\verb|\gvcatchead|           & SC  & General verdict criteria for all abstract test cases \\
\verb|\gvcprehead|           & SC  & General verdict criteria for preprocessor abstract test cases \\
\verb|\gvcposthead|          & SC  & General verdict criteria for postprocessor abstract test cases \\

\verb|\atchead|              & C   & Abstract test cases \\
\verb|\prehead|              & SSC & Preprocessor \\
\verb|\precoveredhead|       & SSSC & Test purposes covered \\
\verb|\preinputhead|         & SSSC & Input specification \\
\verb|\precriteriahead|      & SSSC & Verdict criteria \\
\verb|\preconstraintshead|   & SSSC & Constraints on values \\
\verb|\preexechead|          & SSSC & Execution sequence \\
\verb|\preextrahead|         & SSSC & Extra details \\


\verb|\posthead|             & SSC & Postprocessor \\
\verb|\postcoveredhead|       & SSSC & Test purposes coverage \\
\verb|\postinputhead|         & SSSC & Input specification \\
\verb|\postcriteriahead|      & SSSC & Verdict criteria \\
\verb|\postexechead|          & SSSC & Execution sequence \\
\verb|\postextrahead|         & SSSC & Extra details \\

\verb|\confclassannexhead|   & NA  & Conformance classes \\
\verb|\postipfilehead|       & NA  & Postprocessor input specification file names \\

\verb|\atsusagehead|         & IA  & Usage scenarios \\

\verb|\apasserthead|         & SSC & Application assertions \\
%%%\verb|\extrefpurposehead|    & SC  & External reference test purposes \\
%%%%\verb|\rulepurposehead|      & SC  & \rulepurposename\  \\
%%%\verb|\excludepurposehead|   & NA  & Excluded test purposes \\ 
\hline
\end{tabular}
\end{table}


    The commands that take a parameter are listed in \tref{tab:atsphead}.
\ixcom{apobjhead}
\ixcom{aimenthead}
\ixcom{atctitlehead}
\ixcom{confclasshead}

\begin{table}
\centering
\caption{ATS package parameterized heading commands} \label{tab:atsphead}
\begin{tabular}{|l|c|l|} \hline
Command               & Clause & Parameterized title \\ \hline
\verb|\apobjhead|     & SSC & \meta{Application object n}  \\
\verb|\aimenthead|    & SSC & \meta{Entity name} \\
\verb|\atctitlehead|  & SC  & \meta{Title} \\
\verb|\confclasshead| & SC  & Conformance class \meta{number}  \\ \hline
\end{tabular}
\end{table}


\sclause{Keyword commands}

    Several keyword (headings) are used in documenting a test case.
\latex{} commands for these keywords are given in \tref{tab:atskey}.
\ixcom{atssummary}
\ixcom{atscovered}
\ixcom{atsinput}
\ixcom{atsconstraints}
\ixcom{atsverdict}
\ixcom{atsexecution}
\ixcom{atsextra}

\begin{table}
\centering
\caption{ATS package keyword commands} \label{tab:atskey}
\begin{tabular}{|l|l|} \hline
Command                & Effect \\ \hline
\verb|\atssummary|     & \atssummary{} \\
\verb|\atscovered|     & \atscovered{} \\
\verb|\atsinput|       & \atsinput{} \\
\verb|\atsconstraints| & \atsconstraints{} \\
\verb|\atsverdict|     & \atsverdict{} \\
\verb|\atsexecution|   & \atsexecution{} \\
\verb|\atsextra|       & \atsextra{} \\ \hline
\end{tabular}
\end{table}

\sclause{Boilerplate commands}

    The following commands produce boilerplate text.

\begin{anote}
 In the examples, the
parameters of those commands that take them have been specified in
\textit{this font style} so that their
effects can be seen in the printed text.
\end{anote}

\ssclause{ATS introduction}

    The command 
\verb|\atsintroendbp|\ixcom{atsintroendbp}
 produces the boilerplate
for the end of the Introduction to an ATS.

\begin{anexample} Remembering that in the preamble 
        \verb|\APnumber|\ixcom{APnumber} was set to \texttt{\theAPpartno} 
        and \verb|\APtitle|\ixcom{APtitle} was set to \texttt{\theAPtitle},
the command \verb|\atsintroendbp| prints:

\atsintroendbp

\end{anexample}


\ssclause{ATS scope}

    The command \verb|\scopeclause|\ixcom{scopeclause}, from the \file{isov2}
class, prints the heading for the Scope clause.

    The command 
\verb|\atsscopebp|\ixcom{atsscopebp}
produces boilerplate for an ATS \textit{Scope}
clause.

\begin{anexample}  Remembering that in the preamble 
        \verb|\APnumber|\ixcom{APnumber} was set to \texttt{\theAPpartno}, 
the command \verb|\atsscopebp| prints:

\atsscopebp

\end{anexample}

\ssclause{Test purpose}

    The command \verb|\purposehead|\ixcom{purposehead} prints the heading
for the test purposes clause.

    The command \verb|\atspurposebp|\ixcom{atspurposebp} 
prints boilerplate for the introduction to the clause.

\begin{anexample}  Remembering that in the preamble 
        \verb|\APnumber|\ixcom{APnumber} was set to \texttt{\theAPpartno}, 
the command \verb|\atspurposebp| prints:

\atspurposebp

\end{anexample}

\ssclause{Application element test purposes}

    The command \verb|\aepurposehead|\ixcom{aepurposehead} prints the
heading for the application element test purposes subclause.

    The command 
\verb|\aetpbp|\ixcom{aetpbp}
prints boilerplate for the clause.

\begin{anexample} Remembering that in the preamble 
        \verb|\APnumber|\ixcom{APnumber} was set to \texttt{\theAPpartno}, 
the command \verb|\aetpbp| prints:

\aetpbp

\end{anexample}

\ssclause{AIM test purposes}

    The command \verb|\aimpurposehead|\ixcom{aimpurposehead} prints the
heading for the AIM test purposes subclause.

    The command 
\verb|\aimtpbp|\ixcom{aimtpbp}
prints boilerplate for the clause.

\begin{anexample} Remembering that in the preamble 
        \verb|\APnumber|\ixcom{APnumber} was set to \texttt{\theAPpartno}, 
the command \verb|\aimtpbp| prints:

\aimtpbp

\end{anexample}

\ssclause{Implementation method test purposes}

    The command \verb|\implementpurposehead|\ixcom{implementpurposehead} prints the
heading for the implementation method test purposes subclause.

    The command 
\verb|\atsimtpbp|\ixcom{atsimtpbp}
prints boilerplate for the clause.

\begin{anexample} Remembering that in the preamble 
        \verb|\APnumber|\ixcom{APnumber} was set to \texttt{\theAPpartno}, 
the command \verb|\atsimtpbp| prints:

\atsimtpbp

\end{anexample}



\ssclause{General test purposes and verdict criteria}

    The command \verb|\gtpvchead|\ixcom{gtpvchead} prints the heading
for the general test purposes and verdict criteria clause.

    The command 
\verb|\atsgtpvcbp|\ixcom{atsgtpvcbp} 
prints boilerplate for the clause


\begin{anexample} The command \verb|\atsgtpvcbp| prints:

\atsgtpvcbp
\end{anexample}

\ssclause{General test purposes}

    The command \verb|\generalpurposehead|\ixcom{generalpurposehead} prints
the heading for the general test purposes subclause.

    The command 
\verb|\gtpbp|\ixcom{gtpbp}
prints boilerplate for the suclause.

\begin{anexample} Remembering that in the preamble 
        \verb|\APnumber|\ixcom{APnumber} was set to \texttt{\theAPpartno}, 
the command \verb|\gtpbp| prints:

\gtpbp

\end{anexample}

\ssclause{General verdict criteria}

    The command \verb|\gvcatchead|\ixcom{gvcatchead} prints the
heading for the general verdict criteria for all cases subclause.

    The command 
\verb|\gvatcbp|\ixcom{gvatcbp}
prints boilerplate for the subclause.

\begin{anexample} Remembering that in the preamble 
        \verb|\APnumber|\ixcom{APnumber} was set to \texttt{\theAPpartno}, 
the command \verb|\gvatcbp| prints:

\gvatcbp

\end{anexample}

\ssclause{General verdict criteria for preprocessor}

    The command \verb|\gvcprehead|\ixcom{gvcprehead} prints the
heading for the general verdict criteria for preprocessor cases subclause.

    The command 
\verb|\gvcprebp|\ixcom{gvcprebp} 
prints boilerplate for the subclause.

\begin{anexample} Remembering that in the preamble 
        \verb|\APnumber|\ixcom{APnumber} was set to \texttt{\theAPpartno}, 
the command \verb|\gvcprebp| prints:

\gvcprebp

\end{anexample}

\ssclause{General verdict criteria for postprocessor}


    The command \verb|\gvcposthead|\ixcom{gvcposthead} prints the
heading for the general verdict criteria for postprocessor cases subclause.

    The command 
\verb|\gvcpostbp|\ixcom{gvcpostbp} 
prints boilerplate for the subclause.

\begin{anexample} Remembering that in the preamble 
        \verb|\APnumber|\ixcom{APnumber} was set to \texttt{\theAPpartno}, 
the command \verb|\gvcpostbp| prints:

\gvcpostbp

\end{anexample}

\ssclause{Abstract test cases}

    The command \verb|\atchead|\ixcom{atchead} prints the heading
for the abstract test cases clause.

    The command 
\verb|\atcbp|\ixcom{atcbp} 
prints the first paragraph of the boilerplate for the clause.

\begin{example} The command \verb|\atcbp| prints:

\atcbp
\end{example}

    The command
\verb|\atcbpii|\ixcom{atcbpii}
prints paragraphs~3 and onwards of the boilerplate.

\begin{example} The command \verb|\atcbpii|
prints:

\atcbpii

\end{example}

\ssclause{Preprocessor}

    The command \verb|\prehead|\ixcom{prehead} prints the title
for the preprocessor subsubclause.

    The command
\verb|\atcpretpc|\ixcom{atcpretpc}
prints boilerplate for the subclause.

\begin{anexample} The command \verb|\atcpretpc| prints:

\atcpretpc
\end{anexample}

\ssclause{Postprocessor}

    The command \verb|\posthead|\ixcom{posthead} prints the title
for the postrocessor subsubclause.

    The command
\verb|\atcposttpc|\ixcom{atcposttpc}
prints boilerplate for the subclause.

\begin{anexample} The command \verb|\atcposttpc| prints:

\atcposttpc
\end{anexample}



\ssclause{Conformance class}

    The command \verb|\confclassannexhead|\ixcom{confclassannexhead}
prints the heading for the conformance classes annex heading.

    The command 
\verb|\atsnoclassesbp|\ixcom{atsnoclassesbp} 
prints the entire boilerplate for the
\textit{Conformance class} annex when the AP has no conformance classes.

\begin{example} Remembering that in the preamble 
        \verb|\APnumber|\ixcom{APnumber} was set to \texttt{\theAPpartno}, 
the command \verb|\atsnoclassesbp| prints:

\atsnoclassesbp

\end{example}

    The command \verb|\confclasshead{|\meta{number}\verb|}|\ixcom{confclasshead}
prints the heading for a conformance class \meta{number} subclause.

    The command 
\verb|\confclassbp{|\meta{number}\verb|}|\ixcom{confclassbp}
prints the
boilerplate for the introduction to a conformance class subclause, where
\meta{number} is the number of the conformance class.

\begin{example} Remembering that in the preamble 
        \verb|\APnumber|\ixcom{APnumber} was set to \texttt{\theAPpartno}, 
the command \verb|\confclassbp{27}| prints:

\confclassbp{\textit{27}}

\end{example}

\ssclause{Postprocessor input specification file names}

    The command \verb|\postipfilehead|\ixcom{postipfilehead} prints
the heading for the postprocessor input file names annex.

    The command 
\verb|\pisfbp{|\meta{12 or 21}\verb|}{|\meta{url}\verb|}{|\meta{ref}\verb|}|\ixcom{pisfbp} 
prints the boilerplate for the annex.

\begin{anexample} The command 
\verb|\pisfbp{12}{http://www.mel.nist.gov/step/parts/parts3456/wd}{\ref{TabB1}}| prints:

\pisfbp{12}{http://www.mel.nist.gov/step/parts/part3456/wd}{\ref{tabB1}}

\end{anexample}

%%%%%%%%%%%%%%%%%%%%%%%%%%%
%%%\end{document}
%%%%%%%%%%%%%%%%%%%%%%%%%%%

\normannex{Additional commands} \label{anx:extraiso}

\sclause{Internal commands}

    The code implementing the various facilities includes many commands
not described in the body of this document. Any command that includes
the commercial at sign (\verb|@|) in its name shall not be used by any author;
the implementer of the package code reserves the right to modify or delete
these at any time without giving any notice.

   Internal commands that have names consisting only of letters may be
used in a document at the author's own risk. These may be changed, but 
if so notification will be given.

\sclause{Boilerplate}

    Much of the boilerplate text is maintained in separate \file{.tex}
files and many of the commands that generate boilerplate merely 
\verb|\input|
the appropriate file.




%%%%%%%%%%%%%%%%%%%%%%%%%%
%%%\end{document}
%%%%%%%%%%%%%%%%%%%%%%%%%%

\normannex{Ordering of LaTeX commands} \label{anx:lord}

    The \latex{} commands to produce an ISO~10303 document are:
\begin{verbatim}
\documentclass[<options>]{isov2}
\usepackage{stepv13}                                % required package
\usepackage{irv12}                                  % for an IR document
\usepackage{apv12}                                  % for an AP document
\usepackage{aicv1}                                  % for an AIC document
\usepackage{atsv11}                                 % for an ATS document
\usepackage[<options>]{<name>}                      % additional packages
\standard{<standard identifier>}
\yearofedition{<year>}
\languageofedition{<parenthesized code letter>}
\partno{<part number>}
\series{<series title>}
\doctitle{<title on cover page>}
\ballotcycle{<number>}
\aptitle{<title of AP>}  % if doc is an AP
\aicinaptrue             % if doc is an AP that uses AICs
\mapspectrue             % if doc is an AP that uses mapping spec.
\APnumber{<number>}      % if doc is an ATS
\APtitle{<title>}        % if doc is an ATS
\mapspectrue             % if doc is an ATS and AP uses mapping spec.
  % other preamble commands
\begin{document}
\STEPcover{< title commands >}
\Foreword                            % start Foreword & ISO boilerplate
  \fwdshortlist                      % STEP boilerplate
\endForeword{<param1>}{<param2>}     % end Foreword & boilerplate
\begin{Introduction}                 % start Introduction & boilerplate
  \aicextraintro             % extra boilerplate for an AIC
  \apextraintro                % extra boilerplate for an AP
  % your text 
\end{Introduction}
\stepparttitle{<Part title>}
\scopeclause                         % Clause 1: Scope clause
  \apscope{<AP purpose>}             % boilerplate if an AP
   % text of scope
\normrefsclause                      % Clause 2: Normative references
  \normrefbp{<document type>}        % boilerplate
  \begin{nreferences}
    % \isref{}{} and/or \disref{}{} list of normative references
  \end{nreferences}
\defclause                           % definitions clause
  \partidefhead                      % defs from Part1 subclause
    % olddefinition list
  \refdefhead{<ISO 10303-NN>}        % defs from Part NN subclause
    % olddefinition list
  \otherdefhead                      % defs in this part
    % definition list
\symabbclause                        % Symbols & abbreviations clause
  % symbol lists
% THE BODY OF THE DOCUMENT
\bibannex                            % optional; the final Bibliography 
  % bibliography listing
% the index
\end{document}
\end{verbatim}


\sclause{Body of a resource document} \index{integrated resource}

    The body of a resource document has the following structure:

\begin{verbatim}
\schemahead{<Schema name>}         % repeat for each schema
  \introsubhead                    % intro subclause
     % text
  \fcandasubhead                   % concepts subclause
     % text
  \typehead{<Schema>}              % if type defs
     \atypehead{<type>}            % type heading     
  \entityhead{<Schema>}{<group>}   % if entity defs
     \anentityhead{<entity>}       % entity heading
  \rulehead{<Schema>}              % if rule defs
     \arulehead{<rule>}            % rule heading
  \functionhead{<Schema>}          % if function defs
     \afunctionhead{<function>}    % function heading
% repeat above for each schema
\shortnamehead                     % Annex A: Short names of entities
  \irshortnames                    % boilerplate
  % list of short names
\objreghead                        % Annex B: Information object registration
  \docidhead                       % Document identification subclause
    \docreg{<param1>}                                   % boilerplate
  \schemaidhead                    % Schema identification subclause
% Either (for single schema)
     \schemareg{<6 parameters>}    % boilerplate
% Or (for multiple schemas) repeat:
     \aschemaidhead{<schema name>} % Schema id subsubclause
       \schemareg{<6 parameters>}
\listingshead                       % Annex C: Computer interpretable listings
  \expurls{<short>}{<express>}      % boilerplate
\expressghead                       % Annex D: EXPRESS-G figures
  \irexpressg                       % boilerplate
  %  EXPRESS-G diagrams
\techdischead                       % optional Technical discussions
  % text
\exampleshead                       % optional Examples
  % text
\end{verbatim}


\sclause{Body of an application protocol} \index{AP}

    The body of an AP document has the following structure:

\begin{verbatim}
\inforeqhead                   % Clause 4: Information requirements
  \apinforeq{<param1>}         % boilerplate
  \uofhead                     % Clause 4.1: Units of functionality
    \begin{apuof}              % boilerplate
      % \item list of UoFs
    \end{apuof}
    \auofhead{<UoF1>}          % repeat for each UoF
      % text
    \applobjhead               % Clause 4.2: Application objects
      \apapplobj               % boilerplate
        % text
    \applasserthead            % Clause 4.3: Application assertions
      \apassert                % boilerplate
        % text
\aimhead                       % Clause 5: Application interpreted model
  \maptablehead                % Clause 5.1: Mapping table/specification
    \apmapping                 % boilerplate
    \maptemplatehead           % if mapping templates used
      \apmaptemplate           % template boilerplate
      \sstemplates             % sup/sub templates
      \templatehead
        % text
     \mapuofhead{<Uof>}        % mapping for <UoF>
       \mapobjecthead{<object>}
         % mapping for <object>
         \mapattributehead{<attr>}
           % mapping for <attr>
  \aimshortexphead             % Clause 5.2: AIM EXPRESS short listing
    \apshortexpress            % boilerplate
      % text
\confreqhead                   % Clause 6: Conformance requirements
  \apconformance{<param1>}     % boilerplate
  \begin{apconformclasses}     % optional boilerplate
    % \item list
  \end{apconformclasses}
     % text
\aimlongexphead                % Annex A: AIM EXPRESS expanded listing
  \aimlongexp                  % boilerplate
     % text
\aimshortnameshead             % Annex B: AIM short names
  \apshortnames                % boilerplate
     % text
\impreqhead                    % Annex C: Impl. specific reqs
  \apimpreq{<schema name>}     % boilerplate
\picshead                      % Annex D: PICS
  \picsannex                   % boilerplate
     % text
\objreghead                    % Annex E: Information object registration
  \docidhead                   % Annex E.1: Document identification
    \docreg{<param1>}          % boilerplate
  \schemaidhead                % Annex E.2: Schema identification
    \apschemareg{<6 params>}   % boilerplate
\aamhead                       % Annex F: Application activity model
  \aamfigrange{<figure range>} % Figure range for AAM diagrams
  \apaamintro                  % boilerplate
      % text
  \aamdefhead                  % Annex F.1: AAM defs and abbreviations
    \apaamdefs                 % boilerplate
       % text
  \aamfighead                  % Annex F.2: AAM diagrams
    \aamfigures                % boilerplate
       % IDEF0 diagrams
\armhead                       % Annex G: Application reference model
   \armintro                   % boilerplate
     % ARM figures
\aimexpressghead               % Annex H: AIM EXPRESS-G
  \aimexpressg                 % boilerplate
     % AIM figures
\listingshead                  % Annex J: Computer interpretable listings
  \apexpurls{<short>}{<express>}   % boilerplate
\apusagehead                   % optional Annex: AP usage
   % text
\techdischead                  % optional Annex: Technical discussions
   % text
\end{verbatim}

\sclause{Body of an AIC} \index{AIC}

    The body of an AIC document has the following structure:

\begin{verbatim}
\aicshortexphead               % Clause 4: EXPRESS short listing
  \aicshortexpintro            % boilerplate
  \fcandasubhead               % Clause 4.1 fundamental concepts
    % text
  \typehead{<Schema>}          % if type definitions
     \atypehead{<type>}        % repeat for each type
  \entityhead{<Schema>}{}      % if entity defs
     \anentityhead{<entity>}   % repeat for each entity
  \functionhead{<Schema>}      % if function defs
     \afunctionhead{<function>} % repeat for each function
\shortnamehead                 % Annex A: Short names of entities
  \shortnames                  % boilerplate
\objreghead                    % Annex B: Information object registration
  \docidhead                   % Annex B.1: Document identification
    \docreg{<version no>}      % boilerplate
  \schemaidhead                % Annex B.2: Schema identification
    \schemareg{<6 parameters>} % boilerplate
\expressghead               % Annex C: EXPRESS-G diagrams
  \aicexpressg               % boilerplate
\listingshead                  % Annex D: Computer interpretable listings
  \expurls                     % boilerplate
\techdischead                  % optional Annex: Technical discussions
\end{verbatim}

\sclause{Body of an ATS document}\index{ATS}

    The body of an Abstract Test Suite
document has the following structure:

\begin{verbatim}
\purposeshead                  % Clause 4: Test purposes
  \atspurposebp    % boilerplate
  \aepurposehead               % 4.1 Application element test purposes
    \aetpbp                    % boilerplate
    \apobjhead{<object>}       % 4.1.n
    ...
  \aimpurposehead              % 4.2 AIM test purposes
    \aimtpbp                   % boilerplate
    \aimenthead{<entity>}      % 4.2.n
    ...
  \implementpurposehead        % (optional) 4.3 Implementation t.p
    \atsimtpbp                 % boilerplate
    % text
  \domainpurposehead           % (optional) 4.2+ Domain test purposes
    % text
  \otherpurposehead            % (optional) 4.2+ Other test purposes
    % text
\gtpvchead                     % Clause 5: General t.p and verdict criteria
  \atsgtpvcbp                  % boilerplate
  \generalpurposehead          % 5.1 General test purposes
    \gtpbp                     % boilerplate
    ...
  \gvcatchead                  % 5.2 General verdict criteria for all ATC
    \gvatcbp                   % boilerplate
    ...
  \gvcprehead                  % 5.3 General verdict criteria for preprocessor
    \gvcprebp                  % boilerplate
     ...
  \gvcposthead                 % 5.4 General verdict criteria for postprocessor
    \gvcpostbp                 % boilerplate
    ...
\atchead                       % Clause 6: Abstract test cases
  \atcbp                       % boilerplate (para 1)
  % your para 2
  \atcbpii                     % boilerplate (paras 3+)
  \atctitlehead{<title>}       % 6.n an abstract test case
    \prehead                   % 6.n.1 Preprocessor
      \precoveredhead..        % Test purposes covered
        \atcpretpc             % boilerplate
      \preinputhead            % Input specification
        % text
      \precriteriahead         % Verdict criteria
        % text
      \preconstrainthead       % Constraints on values
        % text
      \preexechead             % (optional) Execution sequence
        % text
      \preextrahead            % (optional) Extra details
        % text
    \posthead                  % 6.n.2 Postprocessor
      \postcoveredhead         % Test purposes covered
        % text
        \atcposttpc            % boilerplate
      \postinputhead           % Input specification
        % text
      \postcriteriahead        % Verdict criteria
        % text
      \postexechead            % (optional) Execution sequence
        % text
      \postextra               % (optional) Extra details
        % text
\confclassannexhead            % Annex A: Conformance classes
  \atsnoclassesbp              % boilerplate if no conformance classes, else
  \confclasshead{<number>}     % A.n Conformance class <number>
    \confclassbp{<number>}     % boilerplate
    % text
  ...
\postipfilehead                % Annex B: Postprocessor input file names
  \pisfbp{..}{..}{..}                      % boilerplate
   ...
\objreghead                    % Annex C: Information object registration
  \docreg{<partno>}            % registration boilerplate
\atsusagehead                  % Annex D: Usage scenarios
  % text
\end{verbatim}



%%%%%%%%%%%%%%%%%%%%%%%%%%%%%%%%%%%%%%%
% object registration annex
\objreghead

\docreg{-1}

%%%%%%%%%%%%%%%%%%%%%%%%%%%%%%%%%%%%%%

%%%%%%%%%%%%%%%%%%%%%%%%%
%%%\end{document}
%%%%%%%%%%%%%%%%%%%%%%%%%

\infannex{Deprecated, deleted, new and modified commands}

    This release has involved many internal changes to the \latex{}
\file{.sty} files. In particular boilerplate text is, as far as possible,
maintained in external \file{.tex} files in order to save memory
space within the \latex{} processor. 


%%%%%%%%%%%%%%%%%%%%%%%%%
%%%\end{document}
%%%%%%%%%%%%%%%%%%%%%%%%%

\sclause{New commands}

    The commands that are new in this release are:

\begin{itemize}
%%%%%%%%%%%%%%%%%%%%%%%%%% STEP %%%%%%%%%%%%%%%%%%%%%%%%%%%
\item \verb|\bibieeeidefo|\ixcom{bibieeeidefo} STEP: reference to IDEF0 document;
\item \verb|\exampleshead|\ixcom{exampleshead} STEP: clause heading;
\item \verb|\expressgdef|\ixcom{expressgdef} STEP: location of \ExpressG{} definition;

\item \verb|\Theseries|\ixcom{Theseries} STEP: print \verb|\series| argument;
\item \verb|\theseries|\ixcom{theseries} STEP: print \verb|\series| argument in lowercase;
\item \verb|\ifanir|,\ixcom{ifanir} 
      \verb|\anirtrue|,\ixcom{anirtrue}
      \verb|\anirfalse|\ixcom{anirfalse} STEP: flag for an IR document;
\item \verb|\ifhaspatents|,\ixcom{ifhaspatents} 
      \verb|\haspatentstrue|,\ixcom{haspatentstrue} 
      \verb|\haspatentsfalse|\ixcom{haspatentsfalse} STEP: flag for known patents;
\item \verb|\ifmapspec|,\ixcom{ifmapspec} 
      \verb|\mapspectrue|,\ixcom{mapspectrue} 
      \verb|\mapspecfalse|\ixcom{mapspecfalse} STEP: flag for mapping specification;

\item \verb|\ixent|\ixcom{ixent} STEP: index an \Express{} \xword{entity};
\item \verb|\ixenum|\ixcom{ixenum} STEP: index an \Express{}  \xword{enumeration};
\item \verb|\ixfun|\ixcom{ixfun} STEP: index an \Express{}  \xword{function};
\item \verb|\ixproc|\ixcom{ixproc} STEP: index an \Express{}  \xword{procedure};
\item \verb|\ixrule|\ixcom{ixrule} STEP: index an \Express{}  \xword{rule};
\item \verb|\ixsc|\ixcom{ixsc} STEP: index an \Express{}  \xword{subtype\_constraint};
\item \verb|\ixschema|\ixcom{ixschema} STEP: index an \Express{}  \xword{schema};
\item \verb|\ixselect|\ixcom{ixselect} STEP: index an \Express{}  \xword{select};
\item \verb|\ixtype|\ixcom{ixtype} STEP: index an \Express{}  \xword{type};

\item \verb|\maptableorspec|\ixcom{maptableorspec} STEP: prints `table' or `specification';

\item \verb|\xword|\ixcom{xword} STEP: prints an \Express{} keyword;

%%%%%%%%%%%%%%%%%%%%%%%%%%%%%%%% AP %%%%%%%%%%%%%%%%%%%%%%%%%%%%%%%


\item \verb|\apmaptemplate|\ixcom{apmaptemplate} AP: boilerplate;
\item \verb|\apusagehead|\ixcom{apusagehead} AP: clause heading;
\item \verb|\ifidefix|,\ixcom{ifidefix} 
      \verb|\idefixtrue|,\ixcom{idefixtrue} 
      \verb|\idefixfalse|\ixcom{idefixfalse} AP: flag for an IDEF1X ARM;
\item \verb|\ifmaptemplate|,\ixcom{ifmaptemplate} 
      \verb|\maptemplatetrue|,\ixcom{maptemplatetrue} 
      \verb|\maptemplatefalse|\ixcom{maptemplatefalse} AP: flag for 
            using mapping templates;
\item \verb|\mapattributehead|\ixcom{mapattributehead} AP: clause heading;
\item \verb|\mapobjecthead|\ixcom{mapobjecthead} AP: clause heading;
\item \verb|\mapuofhead|\ixcom{mapuofhead} AP: clause heading;
\item \verb|\sstemplates|\ixcom{sstemplates} AP: boilerplate;
\item \verb|\templateshead|\ixcom{templateshead} AP: clause heading;
%%% \item \verb|\apmappingspec|\ixcom{} internal (not used?) 

%%%%%%%%%%%%%%%%%%%%%%%%%%%%%%%% ATS %%%%%%%%%%%%%%%%%%%%%%%%%%%%%

\item \verb|\atcposttpc|\ixcom{atcposttpc} ATS: boilerplate;
\item \verb|\atcpretpc|\ixcom{atcpretpc} ATS: boilerplate;
\item \verb|\atsimtpbp|\ixcom{atsimtpbp} ATS: boilerplate;
\item \verb|\atsusagehead|\ixcom{atsusagehead} ATS: clause heading.


\end{itemize}





\sclause{Modified commands}

    The commands that have been modified in this release are:

\begin{itemize}

%%%%%%%%%%%%%%%%%%%%%%%%%%%% STEP %%%%%%%%%%%%%%%%%%%%%%%%%%%%%

\item STEP: The \verb|\Introduction|\ixcom{Introduction} command is
      now the \verb|Introduction|\ixenv{Introduction} environment,
      with no argument;

%%%%%%%%%%%%%%%%%%%%%%%%%%%%%% IR %%%%%%%%%%%%%%%%%%%%%%%%%%%%%%%

\item \verb|\irexpressg|\ixcom{irexpressg} IR: takes no argument;

%%%%%%%%%%%%%%%%%%%%%%%%%%%%%% AP %%%%%%%%%%%%%%%%%%%%%%%%%%%%%%%

\item \verb|\aimexpressg|\ixcom{aimexpressg} AP: takes no argument;

%%%%%%%%%%%%%%%%%%%%%%%%%%%%%% AIC %%%%%%%%%%%%%%%%%%%%%%%%%%%%%%%

\item \verb|\aicexpressg|\ixcom{aicexpressg} AIC: takes no argument;


%%%%%%%%%%%%%%%%%%%%%%%%%%%%%% ATS %%%%%%%%%%%%%%%%%%%%%%%%%%%%%%%

\item \verb|\atcbpii|\ixcom{atcbpii} ATS:  takes no argument;
\item \verb|\atspurposebp|\ixcom{atspurposebp} ATS: takes no argument;
\item \verb|\pisfbp|\ixcom{pisfbp} ATS: takes 3 arguments.


\end{itemize}





\sclause{Deleted commands}

    The commands that have been deleted in this release are:

\begin{itemize}

%%%%%%%%%%%%%%%%%%%%%%%%%%%%%% STEP %%%%%%%%%%%%%%%%%%%%%%%%%%%%%

\item \verb|\fwddivlist|\ixcom{fwddivlist} STEP: used in Foreword;
\item \verb|\fwdpartslist|\ixcom{fwdpartslist} STEP: used in Foreword;

\item \verb|\introend|\ixcom{introend} STEP: was for use at the end of the
      Introduction;

%%%%%%%%%%%%%%%%%%%%%%%%%%%%%% IR %%%%%%%%%%%%%%%%%%%%%%%%%%%%%

\item \verb|\irschemaintro|\ixcom{irschemaintro} IR:
      has been replaced by
      \verb|\schemaintro|\ixcom{schemaintro};

%%%%%%%%%%%%%%%%%%%%%%%%%%%%%% AP %%%%%%%%%%%%%%%%%%%%%%%%%%%%%

\item \verb|\apintroend|\ixcom{apintroend} AP:
      has been replaced by
      \verb|\apextraintro|\ixcom{apextraintro};

\item \verb|\apschemareg|\ixcom{apschemareg} AP: use 
           \verb|\schemareg|\ixcom{schemareg} instead;

\item \verb|\apmappingtable|\ixcom{apmappingtable} AP:
      has been replaced by
      \verb|\apmapping|\ixcom{apmapping};

\item \verb|\armfigures|\ixcom{armfigures} AP:
      has been replaced by
      \verb|\armintro|\ixcom{armintro};

\item \verb|\maptablehead|\ixcom{maptablehead} AP:
      has been replaced by
      \verb|\mappinghead|\ixcom{mappinghead};

\item \verb|\modelscopehead|\ixcom{modelscopehead} AP: was the heading
      for a `Model scope' annex;

%%%%%%%%%%%%%%%%%%%%%%%%%%%%%% AIC %%%%%%%%%%%%%%%%%%%%%%%%%%%%%

\item \verb|\aicexpressghead|\ixcom{aicexpressghead} AIC: 
      use \verb|\expressghead|\ixcom{expressghead} instead;
\item \verb|\aicshortnames|\ixcom{aicshortnames} AIC: use
      \verb|\expurls|\ixcom{expurls} instead;
\item \verb|\aicshortnameshead|\ixcom{aicshortnameshead} AIC: 
      use \verb|\shortnamehead|\ixcom{shortnamehead} instead;

%%%%%%%%%%%%%%%%%%%%%%%%%%%%%% ATS %%%%%%%%%%%%%%%%%%%%%%%%%%%%%


\item \verb|\excludepurposehead|\ixcom{excludepurposehead} ATS: 
      was the heading for an `Exclude purposes' clause.

\end{itemize}


%%%%%%%%%%%%%%%%%%%%%%%
%%%\end{document}
%%%%%%%%%%%%%%%%%%%%%%%


% sgmlannx.tex    latex and SGML

\infannex{LaTeX, the Web, and *ML} \label{anx:sgml} \index{SGML}

    ISO are becoming more interested in electronic sources for their
standards as well as the traditional camera-ready copy. Acronyms like
PDF, HTML, SGML and XML have been bandied about. Fortunately documents
written using \latex{} are well placed to be provided in a variety of 
electronic formats. A comprehensive treatment of \latex{} with respect
to this topic is provided by Goossens and Rahtz~\bref{lwebcom}.

    SGML (Standard Generalized Markup Language) is a document tagging 
language that is described in ISO~8879~\bref{sgml} and whose usage is described 
in~\bref{bryan}, among others. The principal
mover behind SGML is Charles Goldfarb from IBM, who has authored a detailed 
handbook~\bref{goldfarb} on the SGML standard.

    The concepts lying behind both \latex{} and SGML are similar, but on the face
of it they are distinctly different in both syntax and capabilities. ISO is
migrating towards electronic versions of its standard documents and, naturally, 
would prefer these to be SGML tagged. 
     Like \latex, SGML has a
concept of style files, which are termed DTDs, and both systems support
powerful macro-like capabilities. SGML provides for logical document
markup and not typesetting --- commercial SGML systems often use
\tex{} or \latex{} as their printing engine, as does the NIST SGML
environment for ISO~10303~\bref{pandl}.



NIST have SGML tagged some STEP documents 
using manual methods, which are time consuming and expensive. 
In about 1997 there was a NIST 
effort underway to develop an auto-tagger that would (semi-) automatically 
convert
a \latex{} tagged document to one with SGML tags. This tool assumed a
fixed set of \latex{} macros and a fixed DTD.
 The design of an auto-tagger
essentially boils down to being able to convert from a source document tagged
according to a \latex{} style file to one which is tagged according to an
SGML DTD.
    Fully automatic conversion is really only possible if the authors'
of the documents to be translated avoid using any `non-standard' macros within
their documents. There is a program called \file{ltx2x}\index{ltx2x} available
from SOLIS, which replaces \latex{} commands within a document with
user-defined text strings~\bref{ltx2x}. This can be used as a basis for
a \latex{} to whatever auto-tagger, provided the \latex{} commands are not
too exotic.

    HTML is a simple markup language, based on SGML, and is used for the
publication of many documents on the Web. XML is a subset of SGML and appears
to being taken up by every man and his dog as \emph{the} document markup
language. HTML is being recast in terms of XML instead of SGML. PDF is a page
description language that is a popular format for display of documents 
on the Web.

    \latex{} documents can be output in PDF by using pdfLaTeX. Instead
of a \file{.dvi} file being produced a \file{.pdf} file is output directly.
The best 
results are obtained when PostScript fonts rather than Knuth's cm fonts 
are used. Noting that the \file{isov2} class provides an \verb|\ifpdf| command,
a general form for documents to be processed by either \latex{} or pdfLaTeX
is
\begin{verbatim}
\documentclass{isov2}
\usepackage{times}     % PostScript fonts Times, Courier, Helvetica
\ifpdf
  \pdfoutput=1         % request PDF output
  \usepackage[pdftex]{graphicx}
\else
  \usepackage{graphicx}
\fi
...
\end{verbatim}

    There are several converters available to transform a \latex{} document 
into an HTML document, but like \file{ltx2x} they generally do their own
parsing of the source file, and unlike \file{ltx2x} are typically limited
to only generating HTML. Eitan Gurari's \file{TeX4ht}\index{TeX4ht} 
suite is a notable
exception (see Chapter~4 and Appendix~B of~\bref{lwebcom}). It uses the 
\file{.dvi} file as input, so that all the parsing is done by \tex, and can be
configured to generate a wide variety of output formats.
A set of \file{TeX4ht} configuration files are available for converting
STEP \latex{} documents into HTML\footnote{Later, configuration files for XML
output will be developed.}.

    It is highly recommended that for the purposes of ISO~10303, document editors
refrain from defining their own \latex{} macros. If new generally applicable
\latex{} commands are found to be necessary, these should be sent to the
editor of this document for incorporation into
the \file{isov2}\ixclass{isov2} class, the \file{step}\ixpack{step}
package and/or appropriate other packages.

    Some other points to watch when writing \latex{} documents that will assist
in translations into *ML are given below. Typically, attention to these points
will make it easier to parse the \latex{} source.

\begin{itemize}
\item Avoid using the \verb|\label|\ixcom{label} command within
      clause headings or captions. It can just as easily be placed immediately
      after these constructs.
\item Avoid using the \verb|\index|\ixcom{index} command within
      clause headings or captions. It can just as easily be placed immediately
      after these constructs.
\item Use all the specified tagging constructs when defining an \Express{} 
      model --- this will also assist any program that attempts to extract
      \Express{} source code and descriptive text from a document.
\end{itemize}



\infannex{Obtaining LaTeX and friends} \label{anx:getstuff}

    \latex{} is a freely available document typesetting system. There are many
public domain additions to the basic system, for example the \file{iso.cls}
and \file{step.sty} styles. The information below gives pointers to where
you can obtain \latex{} etc., from the\index{Internet} Internet. 


    \latex{} runs on a wide variety of hardware, from PCs to Crays. Source to build
a \latex{} system is freely available via anonymous ftp\index{ftp} 
from what is called CTAN\index{CTAN}
(Comprehensive \tex\ Archive Network). 
There are three sites; pick the one nearest to you.
\begin{itemize}
\item \url{ftp.dante.de} CTAN in Germany;
\item \url{ftp.tex.ac.uk} CTAN in the UK;
\item \url{ctan.tug.org} CTAN in the USA;
\end{itemize}
The top level CTAN directory 
for \latex{} and friends is \url{/tex-archive}. CTAN contains a wide variety
of (La)TeX sources, style files, and software tools and scripts to assist in
document processing.

\begin{anote}
CTAN is maintained by the \tex{} Users Group (TUG). Their homepage
\isourl{http://www.tug.org} should be consulted for the current list of CTAN sites and mirrors.
\end{anote}

\begin{comment}

\sclause{SOLIS} \index{SOLIS}

    SOLIS is the \textit{SC4 On Line Information Service}. It contains many electronic
sources of STEP related documents. The relevant top level directory is
\url{pub/subject/sc4}.
 In particular, SOLIS contains the source for this document
and the \file{.sty} files, as well as other \latex{} related files. 
The \latex{} root directory is \url{sc4/editing/latex}.
The latest versions of the \latex{}
related files are kept in the sub-directory \url{latex/current}.
Some \latex{} related programs are also available in the 
\url{latex/programs} sub-directory.

    There are several ways of accessing SOLIS; instructions
are detailed by Ressler~\bref{ressler} and Rinaudot~\bref{rinaudot}. 
Copies of these reports may be obtained by telephoning the
IPO Office at \verb|+1 (301) 975-3983|, although they are probably somewhat
dated by now.
The simplest method is to point your browser at the following URL: \\
\isourl{http://www.nist.gov/sc4}

\end{comment}

\bibannex
\label{biblio}

\begin{references}
\reference{LAMPORT, L.,}{LaTeX --- A Document Preparation System,}
            {Addison-Wesley Publishing Co., 2nd edition, 1994} \label{lamport}
\reference{WILSON, P. R.,}{LaTeX for standards: The LaTeX package files
            user manual,}%
           {NISTIR, 
                   National Institute of Standards and Technology,
           Gaithersburg, MD 20899. June 1996.} \label{doc:isorot}
\reference{GOOSENS, M., MITTELBACH, F. and SAMARIN, A.,}{%
           The LaTeX Companion,}
           {Addison-Wesley Publishing Co., 1994} \label{goosens}
\reference{GOOSENS, M. and RAHTZ, S.,}{%
           The LaTeX Web Companion --- Integrating TeX, HTML, and XML,}
           {Addison-Wesley Publishing Co., 1999} \label{lwebcom}
\reference{CHEN, P. and HARRISON, M.A.,}{Index preparation and
           processing,}{Software--Practice and Experience, 19(9):897--915,
           September 1988.} \label{chen}
%\reference{KOPKA, H. and DALY, P.W.,}{A Guide to LaTeX,}
%           {Addison-Wesley Publishing Co., 1993.} \label{kopka}
\reference{ISO 8879:1986,}{Information processing --- 
                                Text and office systems ---
           Standard Generalized Markup Language (SGML)}{} \label{sgml}
\reference{GOLDFARB, C.F.,}{The SGML Handbook,}
           {Oxford University Press, 1990.} \label{goldfarb}
\reference{BRYAN, M.,}{SGML --- An Author's Guide to the Standard Generalized
           Markup Language,}{Addison-Wesley Publishing Co., 1988. }\label{bryan}
\reference{PHILLIPS, L., and LUBELL, J.,}{An SGML Environment for STEP,}%
           {NISTIR 5515, 
                   National Institute of Standards and Technology,
           Gaithersburg, MD 20899. November 1994.} \label{pandl}
\reference{WILSON, P. R.,}{LTX2X: A LaTeX to X Auto-tagger,}%
           {NISTIR, 
                   National Institute of Standards and Technology,
           Gaithersburg, MD 20899. June 1996.} \label{ltx2x}
\bibidefo
\bibieeeidefix
\begin{comment}
\reference{RESSLER, S.,}{The National PDES Testbed Mail Server User's Guide,}
           {NSTIR 4508, National Institute of Standards and Technology,
           Gaithersburg, MD 20899. January 1991.} \label{ressler}
\reference{RINAUDOT, G. R.,}{STEP On Line Information Service (SOLIS),}
          {NISTIR 5511, National Institute of Standards and Technology,
          Gaithersburg, MD 20899. October 1994. } \label{rinaudot}
\end{comment}
\end{references}

    
%  the INDEX
% stepman.tex   Description of option style files for STEP
\documentclass[wd,copyright,letterpaper]{isov2}
\usepackage{stepv13}
\usepackage{irv12}
\usepackage{apv12}
\usepackage{aicv1}
\usepackage{atsv11}
%%\usepackage{isomods}  % must come after the step packages
\usepackage{hyphenat}
\usepackage{comment}

\ifpdf
  \pdfoutput=1
  \usepackage[plainpages=false,
              pdfpagelabels,
              bookmarksnumbered,
              hyperindex=true
             ]{hyperref}
\fi

% general required preamble commands
\standard{ISO/WD 10303-3456}
\yearofedition{2002}
\languageofedition{(E)}
\renewcommand{\extrahead}{N 47b}  % add doc N number to headers
\partno{3456}
\series{documentation methods}
\doctitle{LaTeX package files for ISO 10303: User manual}
\ballotcycle{2}
% required preamble commands for an AP
\aptitle{implicit drawing}
\aicinaptrue % only if the AP uses AICs
\mapspectrue  % only if AP uses mapping specification
% required preamble commands for an ATS
\APtitle{abstract painting}
\APnumber{299}

\changemarkstrue

\makeindex

\setcounter{tocdepth}{3} % add more levels to table of contents
%
% Rest of preamble is some special macro definitions for this document only
%
\makeatletter
%
%   the \file{} command
%
\newcommand{\file}[1]{\textsf{#1}}
%
%   the \meta{} command
%
\begingroup
\obeyspaces%
\catcode`\^^M\active%
\gdef\meta{\begingroup\obeyspaces\catcode`\^^M\active%
\let^^M\do@space\let \do@space%
\def\-{\egroup\discretionary{-}{}{}\hbox\bgroup\it}%
\m@ta}%
\endgroup
\def\m@ta#1{\leavevmode\hbox\bgroup$<$\it#1\/$>$\egroup
    \endgroup}
\def\do@space{\egroup\space
    \hbox\bgroup\it\futurelet\next\sp@ce}
\def\sp@ce{\ifx\next\do@space\expandafter\sp@@ce\fi}
\def\sp@@ce#1{\futurelet\next\sp@ce}
%
% the \setlabel{id}{num} command
% this is based on the kernel \refstepcounter macro (ltxref.dtx)
%
%%\newcounter{lbl}
\ifpdf
  \newcommand{\setlabel}[2]{%
    \protected@write\@auxout{}{%
      \string\newlabel{#1}{{#2}{\thepage}{setlabel\relax}{label.#2}{}}}%
  }
\else
  \newcommand{\setlabel}[2]{%
    \protected@write\@auxout{}{%
      \string\newlabel{#1}{{#2}{\thepage}}}%
  }
\fi
%
% index a command
\newcommand{\bs}{\symbol{'134}}
\newcommand{\ixcom}[1]{\index{#1/ @{\tt \protect\bs #1}}}
% index an environment
\newcommand{\ixenv}[1]{\index{#1 @{\tt #1} (environment)}}
% index an option
\newcommand{\ixopt}[1]{\index{#1 @{\tt #1} (option)}}
% index a package
\newcommand{\ixpack}[1]{\index{#1 @\file{#1} (package)}}
% index a class
\newcommand{\ixclass}[1]{\index{#1 @\file{#1} (class)}}
% index in typewriter font
\newcommand{\ixtt}[1]{\index{#1@{\tt #1}}}
% index LaTeX
\newcommand{\ixltx}{\index{latex@\LaTeX}}
% index LaTeX 2e
\newcommand{\ixltxe}{\index{latex2e@\LaTeX 2e}}
% index LaTeX v2.09
\newcommand{\ixltxv}{\index{latex209@\LaTeX{} v2.09}}
% index a file
\newcommand{\ixfile}[1]{\index{#1@\file{#1}}}
\makeatother
%
%
%   set some labels
% step
\setlabel{;ssne}{A}
%%%\setlabel{;sior}{B}
%%%\setlabel{;scil}{C}
\setlabel{;seg}{D}
% aic
\setlabel{;sesl}{4}
% ap
\setlabel{;sireq}{4}
\setlabel{;suof}{4.1}
\setlabel{;sao}{4.2}
\setlabel{;saa}{4.3}
\setlabel{;saim}{5}
\setlabel{;smap}{5.1}
\setlabel{;saesl}{5.2}
\setlabel{;scr}{6}
\setlabel{;saeel}{A}
\setlabel{;sasn}{B}
\setlabel{;simreq}{C}
\setlabel{;spics}{D}
\setlabel{;saam}{F}
\setlabel{;sarm}{G}
\setlabel{;saeg}{H}
\setlabel{;scil}{J}
\setlabel{tabB1}{B.1}
\setlabel{;uof1}{5.1.2}
\setlabel{;uoflast}{5.1.4}
%
% define a new length
\newlength{\prwlen}
%
% new (La)TeX macros
\newcommand{\latex}{LaTeX}
\newcommand{\tex}{TeX}
%
%%%%%%%%%% END SPECIAL MACROS
%
%   end of preamble
%
\begin{document}


\STEPcover{
%\scivnumber{987}
\wg{EC}
\docnumber{47b}
\oldwg{EC}
\olddocnumber{47a}
\docdate{2002/09/04}
%\partnumber{3456}
%\doctitle{LaTeX package files for ISO 10303: User manual}
%\status{Working draft}
%\primcont
\abstract{This document describes and illustrates the \latex{} macros
for typesetting ISO~10303. The International Organisation for
Standardisation (ISO) has specified editorial directives for all 
international standards published by them. The \latex{} macros
described here were developed to meet additional editorial directives 
for ISO~10303. } % end abstract
\keywords{\latex, document preparation, typesetting ISO standards}
%\dateprojo{May 1996}
\owner{Peter R. Wilson}
\address{Boeing Commercial Airplane\newline
            PO Box 3701 \newline
            MS 2R-97 \newline
            Seattle, WA 98124-2207 \newline
            USA}
\telephone{+1 (206) 544-0589}
\fax{+1 (206) 544-5889}
\email{\url{peter.r.wilson@boeing.com}}
\altowner{Peter R. Wilson}
\altaddress{Boeing Commercial Airplane \newline
            PO Box 3701 \newline
            MS 2R-97 \newline
            Seattle, WA 98124-2207 \newline
            USA}
\alttelephone{+1 (206) 544-0589}
\altfax{+1 (206) 544-5889}
\altemail{\url{peter.r.wilson@boeing.com}}
\comread{\draftctr This document serves two purposes. Firstly, it provides a description
         of the current \latex{} macros for ISO 10303. Secondly, the source
         can be used as an example of using the \latex{} commands.
         Although the document is written as though it were a
         standard, it is not, and is not intended to become, 
         a standard.} %end comread
} % end of STEPcover

\Foreword

\fwdshortlist
\endForeword%
{Annexes A, B and C are}  % normative annexes
{Annexes D, E and F are} % informative annexes

\begin{Introduction}%%%%%%%%%%{documentation methods}

    This part of ISO 10303 specifies the \latex{} facilities specifically 
designed for use in preparing the various parts of this standard.

\begin{majorsublist}
\item the \file{step} package facility;
\item the \file{ir} package facility;
\item the \file{ap} package facility;
%%%\item the \file{am} package facility;
\item the \file{aic} package facility;
\item the \file{ats} package facility.
\end{majorsublist}

    This part of ISO~10303 is intended to be used in conjunction with
\textit{\latex{} for ISO standards: User manual}
which is based in part upon material in the ISO/IEC Directives,
Part 2 (\textit{Rules for the structure and drafting of International 
Standards, Fourth edition}).
The \latex{} facilities described here are based as well
upon the specifications given in ISO TC184/SC4 N1217n 
(\textit{SC4 Supplementary directives --- Rules for the structure
and drafting of SC4 standards for industrial data}).


\sclause*{Overview}


    This document describes a set of \latex{} macro files for use within
ISO~10303, commonly called STEP (STandard for the Exchange of Product
model data). The electronic source of this  document 
also provides an example of the use of these files.

    The current set of macro files have been developed by 
Peter Wilson (\url{peter.r.wilson@boeing.com}) from a macro file developed
by Kent Reed (NIST) for \latex{} v2.09. In turn, this was a revision of
files originally created by Phil Spiby (CADDETC), based on earlier work 
by Phil Kennicott (GE).\footnote{In mid 1994 \latex{} was upgraded 
from version 2.09 to what is called \latex 2e. The files described in 
this document are only applicable to \latex 2e (support for \latex{} v2.09
was dropped in September 1997).}


\begin{anote} 
It is important to remember that these macro files are only compatible with 
\latex 2e.
\end{anote} % end anote

    Documents produced with the \latex{} files have been twice reviewed 
by the ISO Editorial Board in Geneva for conformance to their 
typographical requirements. The first review was of a set of Draft 
International Standard documents. This review resulted in some changes 
to the style files. The second review was of a set of twelve 
International Standard documents (ISO 10303:1994). Likewise, this
review led to changes in the style files to bring the documents into 
conformance.

    With the issuance of the first STEP release, the opportunity was 
taken to provide a new baseline release of the package files. 
In particular, one STEP specific package file is available for all 
STEP parts, while others contain only commands relevant to the 
documentation of particular series of parts. The range of package 
files may be extended in the future to cater for 
documentation specific to all STEP parts.

   The 1997 baseline release was also designed to cater for the 
fact that a major update of \latex{} to \latex 2e took place during 1994.
\latex 2e is the only officially supported version of \latex.

    Because ISO standard documents have a very structured layout, the 
\file{isov2} class and the package files described here have been 
designed to reflect the logical document structure to a much greater 
extent than the `standard' \latex{} files. 

    With ISO's move toward accepting documents in PDF and HTML, 
the advent of second
editions of some of the STEP parts, and a new edition of the STEP
Supplementary Directives, a 2002
baseline release has been developed and is documented here. 



\end{Introduction}

\stepparttitle{Documentation methods: LaTeX package files for ISO 10303:
User manual}


\scopeclause

This part of ISO~10303 describes a set of \ixltx\latex{} facilities for typesetting
documents according to the ISO/IEC Directives Part 2, together with the 
Supplementary Directives for drafting and presentation of ISO~10303.

\begin{inscope}{part of ISO~10303}
\item use of \latex{} for preparing ISO~10303 documents.
\end{inscope}

\begin{outofscope}{part of ISO~10303}
\item use of \latex{} for preparing ISO standard documents in general;
\item use of \latex{} in general;
\item use of other document preparation systems.
\end{outofscope}

\textbf{IMPORTANT:} The preparation of this document has been partly
funded by the US Government and is not subject to copyright.
Any copyright notices within the document are for illustrative purposes only.

\normrefsclause \label{sec:nrefs}

\normrefbp{part of ISO~10303}
\begin{nreferences}

\isref{ISO/IEC Directives, Part 2}{Rules for the structure and drafting 
           of International Standards, Fourth edition.}

\isref{ISO TC 184/SC4 N1217:2001(E)}{SC4 Supplementary directives --- 
       Rules for the structure and drafting of SC4 standards for 
       industrial data.}

%\isref{ISO 10303-1:1994}{Industrial automation systems and integration ---
%        Product data representation and exchange --- 
%        Part 1: Overview and fundamental principles.}
\nrefparti

%\isref{ISO 10303-11:1994}{Industrial automation systems and integration --- 
%        Product data representation and exchange --- 
%        Part 11: Description methods:
%        The EXPRESS language reference manual.}
\nrefpartxi

%\disref{ISO/TR 10303-12:---}{Industrial automation systems and integration ---
%        Product data representation and exchange ---
%        Part 12: Description methods:
%        The EXPRESS-I language reference manual.}
\nrefpartxii

%\disref{ISO/IEC 8824-1:---}{Information technology ---
%       Open systems interconnection ---
%       Abstract syntax notation one (ASN.1) ---
%       Part 1: Specification of basic notation.}
\nrefasni

\disref{P. R. WILSON:---}{LaTeX for ISO standards: User manual.}

\end{nreferences}

\defabbclause
%\clause{Terms, definitions, and abbreviations}

\partidefhead
%\sclause{Terms defined in ISO 10303-1}

    This part of ISO~10303 makes use of the following terms defined in 
ISO~10303-1:

\begin{olddefinitions}
\olddefinition{application protocol (AP)} \index{Application Protocol}
                                            \index{AP}
\olddefinition{integrated resource} \index{Integrated Resource}
\end{olddefinitions}


\otherdefhead
%\sclause{Other definitions}

    For the purposes of this part of ISO~10303, the following definitions
apply.

\begin{definitions}
\definition{boilerplate}{Text whose wording is fixed and which has been
agreed to be present in a specific type of document.} \index{boilerplate}
\definition{style file}{A set of \latex{} macros assembled into a 
single file with an extension \file{.sty}.}
            \index{style file}
\definition{package file}{A style file for use with \latex 2e.}
            \index{package file}
\definition{facility}{A generic term for a set of \latex{} macros
          assembled for a common purpose. The macros may be defined in
          either a style file or a package file.}\index{facility}

\end{definitions}

\abbsubclause
%\sclause{Abbreviations}

    For the purposes of this part of ISO 10303, the following abbreviations
 apply.

\begin{symbols}
\symboldef{AIC}{Application Interpreted Construct} \index{AIC}
\symboldef{AM}{Application Module} \index{AM}
\symboldef{AP}{Application Protocol}  \index{AP}
\symboldef{DIS}{Draft International Standard} \index{DIS}
\symboldef{IS}{International Standard}         \index{IS}
\symboldef{ISOD}{ISO/IEC Directives, Part 2} \index{ISOD} \index{ISO/IEC Directives}
\symboldef{SD}{Supplementary Directives --- 
  \textit{SC4 Supplementary directives --- Rules for the structure and
   drafting of SC4 standards for industrial data}}\index{SD}\index{Supplementary Directives}
\symboldef{IS-REVIEW}{The ISO Editorial Board review (September 1994) of 
            twelve IS documents
            for conformance to ISO typographical and 
            layout requirements.} \index{IS-REVIEW}
\end{symbols}



\clause{Conformance requirements}  \label{sec:iconform}

    The facility files shall not be modified in any manner.

    If there is a need to modify any of the macro definitions then this
shall be done using the \latex{} 
\verb|\renewcommand|\ixcom{renewcommand} and/or the
\verb|\renewenvironment|\ixcom{renewenvironment}
commands. These shall be placed in a new \file{.sty} file (or files) 
which shall be called in within the preamble\index{preamble} of the 
document being typeset.

    There shall be no author specified \verb|\label{...}| commands where
the first two characters of the label are \verb|;s| (semicolon and `s');
the creation of labels starting with these characters is reserved to the 
maintainer of the facility files.

\begin{anote} For conformance to the \file{isov2} class, author specified
labels starting with the characters \verb|;i| (semicolon and `i') are
prohibited.
\end{anote}


\fcandaclause
%\clause{Fundamental concepts and assumptions}

    It is assumed that the reader of this document is familiar with the
\ixltx\latex{} document preparation system and in particular
with the \file{isov2}\ixclass{isov2} class and associated facilities 
described in 
\textit{LaTeX for ISO standards: User manual}.

\begin{note}Reference~\bref{lamport} describes the
      \latex{} system.
\end{note} % end note

    The reader is also assumed to be familiar with the ISO/IEC Directives 
Part~2 (ISOD)\index{ISOD} and
the SC4 Supplementary directives for the structure and drafting of 
SC4 standards (SD).\index{ISOD}\index{SD}

    If there are any discrepancies between the layout and wording of this 
document and the requirements of the ISOD or the SD,
then the requirements in those documents shall be
followed for ISO~10303 standard documents.

    The packages described herein have been designed to be used with
the \file{isov2}\ixclass{isov2} document class. It is highly unlikely that the
packages will perform at all with any other \latex{} document class.

    Because of many revisions over the years to the packages described
herein, a naming convention has been adopted for the package files.
The naming convention is that the
primary name of the file is suffixed by \file{v\#}, where
\file{\#} is the primary version number of the file in question.
All file primary names have been limited to a maximum of eight characters.

\begin{note}Table~\ref{tab:curfiles} shows the versions of the files
that were current at the time of publication.
\ixpack{step}\ixfile{stepv13.sty}
\ixpack{ir}\ixfile{irv12.sty}
\ixpack{ap}\ixfile{apv12.sty}
\ixpack{aic}\ixfile{aicv1.sty}
\ixpack{ats}\ixfile{atsv11.sty}
%%%\ixpack{am}\ixfile{amv1.sty}
\end{note} % end note

\begin{table}
\centering
\caption{File versions current at publication time} \label{tab:curfiles}
\begin{tabular}{|l|l|l|} \hline
\textbf{Facility} & \textbf{File}   & \textbf{Version} \\ \hline\hline
\file{step}    & \file{stepv13.sty} & v1.3.2 \\
\file{ir}      & \file{irv12.sty}   & v1.2   \\
\file{ap}      & \file{apv12.sty}   & v1.2   \\
%%%\file{am}      & \file{amv1.sty}    & v1.0   \\
\file{aic}     & \file{aicv1.sty}   & v1.0   \\
\file{ats}     & \file{atsv11.sty}  & v1.1   \\ 
\hline
\end{tabular}
\end{table}


\begin{note}
This document is not, and is never intended to become,
 a standard, although it has been laid out in a 
similar, but not necessarily identical, manner.
\end{note} % end note


\clearpage
\clause{The \file{step} package facility}

    The \file{step}\ixpack{step} package facility provides commands 
and environments 
applicable to all the ISO~10303 series of documents.

\sclause{Preamble commands}

    Certain commands shall be put in the preamble\index{preamble}
of any document.

    The command 
\verb|\partno{|\meta{number}\verb|}|\ixcom{partno}
is used to specify the Part number of the ISO~10303 standard
(e.g., \verb|\partno{3456}|).

    The command
\verb|\series{|\meta{series title}\verb|}|\ixcom{series}
is used to specify the name of the ISO~10303 series of which the Part 
is a member (e.g., \verb|\series{application modules}|).

    The command
\verb|\doctitle{|\meta{informal title}\verb|}|\ixcom{doctitle}
is used to specify the title to be used on the cover sheet.
For example: \\
\verb|\doctitle{LaTeX package files for ISO 10303: User manual}|

    The command
\verb|\ballotcycle{|\meta{number}\verb|}|\ixcom{ballotcycle}
is used to specify the ballot cycle number for the document
(e.g., \verb|\ballotcycle{2}|).

    The command\ixcom{ifhaspatents}
\verb|\haspatentstrue|\ixcom{haspatentstrue} shall be put in the
preamble when the document includes identified patented material;
otherwise the command \verb|\haspatentsfalse|\ixcom{haspatentsfalse}
may, but need not, be used instead.

    The \verb|\extrahead|\ixcom{extrahead} macro, from the \file{isov2}
class, shall be defined to be the document
number (e.g., \verb|\renewcommand{\extrahead}{47a}|).



\begin{anote}
The commands \verb|\standard|\ixcom{standard}, 
\verb|\yearofedition|\ixcom{yearofedition} and 
\verb|\languageofedition|\ixcom{languageofedition} from the \file{isov2}
class must also be put in the preamble.
\end{anote}


\sclause{Cover page}

    The command \verb+\STEPcover{+\meta{commands}\verb+}+\ixcom{STEPcover}
produces a cover page for a STEP document. 
The complete list of commands is shown below.

\begin{itemize}
\item \verb+\wg{+\meta{working group}\verb+}+\ixcom{wg}
      the working 
      group or other committee producing the document e.g., WG 5
\item \verb+\docnumber{+\meta{number}\verb+}+\ixcom{docnumber}
       the number
       of the document e.g., 156
\item \verb+\docdate{+\meta{date}\verb+}+\ixcom{docdate}
       date of 
       publication e.g., 1993/07/03
\item \verb+\oldwg{+\meta{working group}\verb+}+\ixcom{oldwg}
       superseded 
       working group e.g., WG 1
\item \verb+\olddocnumber{+\meta{number}\verb+}+\ixcom{olddocnumber}
        number of previous document e.g., 107
\item \verb+\abstract{+\meta{text}\verb+}+\ixcom{abstract}
        an abstract 
        of the document
\item \verb+\keywords{+\meta{text}\verb+}+\ixcom{keywords}
        for listing 
        relevant keywords
\item \verb+\owner{+\meta{text}\verb+}+\ixcom{owner}
         name of the project leader
\item \verb+\address{+\meta{text}\verb+}+\ixcom{address}
        address of the project leader
\item \verb+\telephone{+\meta{number}\verb+}+\ixcom{telephone}
         the project leader's telephone number
\item \verb+\fax{+\meta{number}\verb+}+\ixcom{fax}
         the project leader's fax number
\item \verb+\email{+\meta{text}\verb+}+\ixcom{email}
        Email address of the project leader
\item \verb+\altowner{+\meta{text}\verb+}+\ixcom{altowner}
         name of the editor of the document
\item \verb+\altaddress{+\meta{text}\verb+}+\ixcom{altaddress}
         the editor's address 
\item \verb+\alttelephone{+\meta{number}\verb+}+\ixcom{alttelephone}
         the editor's telephone number
\item \verb+\altfax{+\meta{number}\verb+}+\ixcom{altfax}
         the editor's fax number
\item \verb+\altemail{+\meta{text}\verb+}+\ixcom{altemail}
           the editor's Email address 
\item \verb+\comread{+\meta{text}\verb+}+\ixcom{comread}
           comments to 
           the reader
\end{itemize}

    Use only those commands within \verb|\STEPcover| that are relevant 
to the purposes at hand. The order of the commands within 
\verb|\STEPcover| is immaterial.

\begin{example}
The commands used to produce the cover sheet for one version of this 
document were:
\begin{verbatim}
\STEPcover{
\wg{EC}
\docnumber{41}
\oldwg{EC}
\olddocnumber{35}
\docdate{1994/08/19}
\abstract{This document describes the \latex{} style files created for ISO~10303.
          It also describes the program GenIndex which provides some 
          capabilities to assist in the creation of indexes for \latex{}
          documents in general.}
\keywords{\latex, Style file, GenIndex, Index}
\owner{Peter R Wilson}
\address{NIST\newline
         Bldg. 220, Room A127 \newline
         Gaithersburg, MD 20899 \newline
         USA }
\telephone{+1 (301) 975-2976}
\email{\texttt{pwilson@cme.nist.gov}}
\altowner{Tony Day}
\altaddress{Sikorsky Aircraft}
\comread{This document serves two purposes. Firstly, it provides a description
         of the current \latex{} style file for ISO 10303. Secondly, the source
         can be used as an example of using the \latex{} commands.} % end comread
} % end of STEPcover
\end{verbatim}
Note the use of the \verb|\newline| command instead of \verb|\\| in 
the argument of the \verb|\address| command to indicate a new line. The
\verb|\newline| is needed to ensure satisfactory conversion to HTML.
\end{example} % end example

    The macro \verb|\draftctr|\ixcom{draftctr} generates boilerplate that
may be used in the `Comments to Reader' section of a cover page.
\begin{example}
The \latex{} source \verb|\draftctr This document \ldots| prints:

\draftctr This document \ldots
\end{example}

\sclause{Heading commands}

    The commands described in this subclause specify various `standard'
clause headings.

\ssclause{The Foreword commands}

    The \verb+\Foreword+\ixcom{Foreword} command specifies that a 
table of contents, list of figures and a list of tables be produced. 
Page numbering is roman style and the table of contents starts on page iii.
A new unnumbered clause entitled Foreword is started containing both 
ISO required boilerplate and boilerplate\index{boilerplate}
text specific to ISO 10303.


    Any text may be written after the \verb|\Foreword| command. The
Foreword clause is ended by the 
\verb+\endForeword{+\meta{norm annexes}\verb+}{+\meta{inf annexes}\verb+}+ 
command.\ixcom{endForeword} This command takes two parameters.
\begin{enumerate}
\item \meta{norm annexes} A phrase that starts the sentence 
      `\meta{norm annexes} a normative part of this part \ldots'.
     If there are no normative annexes, then use an empty
     argument (i.e., \verb|{}| with no spaces between the braces).
\item \meta{inf annexes} A phrase that starts the sentence 
     `\meta{inf annexes} for information only.'.
     If there are no informative annexes, then use an
     empty argument.
\end{enumerate}

    The \verb|\endForeword| command produces some additional 
boilerplate\index{boilerplate} text specifically for ISO 10303. 

\begin{example}
The \latex{} source for the Foreword for this document is:
\begin{verbatim}
\Foreword
\fwdshortlist
\endForeword
{Annexes A, B and C are}  % normative annexes
{Annexes D, E and F are} % informative annexes
\end{verbatim}
\end{example} % end example


    The \verb|\fwdshortlist|\ixcom{fwdshortlist} command 
produces boilerplate text for inclusion in the foreword referencing
the STEP parts and series. 
\begin{example}
In this document, the command \verb|\fwdshortlist| prints:

\fwdshortlist
\end{example}

    The \verb|\steptrid|\ixcom{steptrid} command
produces boilerplate text for inclusion in the foreword describing 
the creators of a STEP Technical Report.

\begin{example}
The \latex{} command \verb|\steptrid| in this document prints:
  
\steptrid
\end{example}


\ssclause{The Introduction environment}

    The 
\verb+\begin{Introduction}+\ixenv{Introduction}
environment starts a new unnumbered clause 
entitled Introduction and adds some boilerplate\index{boilerplate}
text specifically for ISO~10303.

\begin{example}
    The following \latex{} source was used to specify the Introduction 
to this document. \label{ex:intro}
\begin{verbatim}
\begin{Introduction}

    This part of ISO 10303 specifies the \latex{} facilities 
specifically designed for use in preparing the various parts of 
this standard.

\begin{majorsublist}
\item the \file{step} package facility;
\item the \file{ir} package facility;
\item the \file{ap} package facility;
\item the \file{aic} package facility;
\item the \file{atc} package facility.
\end{majorsublist}

    This part of ISO 10303 is intended to be used ...

\sclause*{Overview}

    This document describes a set of \latex{} files for use
within ISO~10303 ...

\end{Introduction}
\end{verbatim}
\end{example} % end example


\ssclause{The stepparttitle command}

   The \verb+\stepparttitle{+\meta{part title}\verb+}+\ixcom{stepparttitle}
command produces the title for
an ISO~10303 part, where \meta{part title} is the title of the part.

\begin{anexample}The title for this document was produced using:
\begin{verbatim}
\stepparttitle{Documentation methods:
               LaTeX package files for ISO 10303: User manual}
\end{verbatim}
\end{anexample} % end example


\ssclause{Other headings}

    Most of these commands take no parameters. They start document clauses
with particular titles. The commands that take no parameters are listed
in \tref{tab:noparamhead}. Some of these headings commands have predefined
labels, which are also listed in the table.
\ixcom{partidefhead}
\ixcom{otherdefhead}
\ixcom{introsubhead}
\ixcom{fcandasubhead}
\ixcom{shortnamehead}
\ixcom{picshead}
\ixcom{objreghead}
\ixcom{docidhead}
\ixcom{schemaidhead}
\ixcom{expresshead}
\ixcom{listingshead}
\ixcom{expressghead}
%%\ixcom{modelscopehead}
\ixcom{techdischead}
\ixcom{exampleshead}

\begin{anote}
 In the tables, C = clause, SC = subclause, SSC = subsubclause,
NA = normative annex, IA = informative annex.
\end{anote} % end note

\settowidth{\prwlen}{\quad Protocol Implementation Conformance Statement}
\begin{table}
\centering
\caption{STEP package parameterless heading commands}
\label{tab:noparamhead}
\begin{tabular}{|l|c|p{\prwlen}|l|} \hline
\textbf{Command} & \textbf{Clause} & \textbf{Default text} & \textbf{Label} \\ \hline
\verb|\partidefhead| & SC & Terms defined in ISO 10303-1 &  \\
\verb|\otherdefhead| & SC & Other definitions & \\
\verb|\introsubhead| & SC & Introduction &  \\
\verb|\fcandasubhead| & SC & Fundamental concepts and assumptions & \\
\verb|\shortnamehead| & NA & Short names of entities & \verb|;ssne| \\
\verb|\picshead| & NA & Protocol Implementation Conformance Statement (PICS) proforma & \verb|;spics| \\
\verb|\objreghead| & NA & Information object registration  & \verb|;sior| \\
\verb|\docidhead| & SC & Document identification & \\
\verb|\schemaidhead| & SC & Schema identification &  \\
\verb|\expresshead| & IA & \Express{} listing &  \\
\verb|\listingshead| & IA & Computer interpretable listings & \verb|;scil| \\
\verb|\expressghead| & IA & \ExpressG\ diagrams & \verb|;seg| \\
%%%%\verb|\modelscopehead| & IA & Model scope & \verb|;sms| \\
\verb|\techdischead| & IA & Technical discussions & \verb|;std| \\ 
\verb|\exampleshead| & IA & Examples & \verb|;sex| \\
\hline
\end{tabular}
\end{table}

    The commands listed in \tref{tab:paramhead} are equivalent to the
general sectioning commands, but are intended to indicate the start
of a particular documentation element. These commands take either one
or two parameters. The parameters are denoted in the column headed
`Parameterized title'.
\ixcom{refdefhead}
\ixcom{schemahead}
\ixcom{typehead}
\ixcom{entityhead}
\ixcom{rulehead}
\ixcom{functionhead}
\ixcom{atypehead}
\ixcom{anentityhead}
\ixcom{arulehead}
\ixcom{afunctionhead}
\ixcom{aschemaidhead}
\ixcom{singletypehead}
\ixcom{singleentityhead}
\ixcom{singlerulehead}
\ixcom{singlefunctionhead}

\begin{table}
\centering
\caption{STEP package parameterized heading commands}
\label{tab:paramhead}
\begin{tabular}{|l|c|l|} \hline
\textbf{Command} & \textbf{Clause} & \textbf{Parameterized title} \\ \hline
\verb|\refdefhead| & SC & Terms defined in \meta{ISO ref} \\
\verb|\schemahead| & C & \meta{schema name} \\
\verb|\singletypehead| & SC & \meta{schema name} type definition:
\meta{type name} \\
\verb|\typehead| & SC & \meta{schema name} type definitions \\
\verb|\atypehead| & SSC & \meta{type name} \\
\verb|\singleentityhead| & SC & \meta{schema name} entity definition:
\meta{entity name} \\
\verb|\entityhead| & SC & \meta{schema name} entity definitions \meta{group} \\
\verb|\anentityhead| & SSC & \meta{entity name} \\
\verb|\singlerulehead| & SC & \meta{schema name} rule definition:
\meta{rule name} \\
\verb|\rulehead| & SC & \meta{schema name} rule definitions \\
\verb|\arulehead| & SSC & \meta{rule name} \\
\verb|\singlefunctionhead| & SC & \meta{schema name} function definition:
\meta{function name} \\
\verb|\functionhead| & SC & \meta{schema name} function definitions \\
\verb|\afunctionhead| & SSC & \meta{function name} \\
\verb|\aschemaidhead| & SSC & \meta{schema name} identification \\ \hline
\end{tabular}
\end{table}

\sclause{Miscellaneous commands}

    The following commands provide some printing options for commonly 
occurring situations. The \verb|\nexp{}|\ixcom{nexp} command is intended 
to be used for printing \Express{} \index{express@{\Express}} entity names etc.
\begin{itemize}
\item The command \verb|\B{abc}|\ixcom{B} prints \B{abc}
\item The command \verb|\E{abc}|\ixcom{E} prints \E{abc}
\item The command \verb|\Express|\ixcom{Express} prints \Express{}
\item The command \verb|\ExpressG|\ixcom{ExpressG} prints \ExpressG{}
\item The command \verb|\ExpressI|\ixcom{ExpressI} prints \ExpressI{}
\item The command \verb|\ExpressX|\ixcom{ExpressX} prints \ExpressX{}
\item The command \verb|\BG{|\meta{mathsymbol}\verb|}|\ixcom{BG} prints 
      \meta{mathsymbol} in bold font.
\item The command \verb|\HASH|\ixcom{HASH} prints \HASH{}
\item The command \verb|\LT|\ixcom{LT} prints \LT{}
\item The command \verb|\LE|\ixcom{LE} prints \LE{}
\item The command \verb|\NE|\ixcom{NE} prints \NE{}
\item The command \verb|\INE|\ixcom{INE} prints \INE{}
\item The command \verb|\GE|\ixcom{GE} prints \GE{}
\item The command \verb|\GT|\ixcom{GT} prints \GT{}
\item The command \verb|\CAT|\ixcom{CAT} prints \CAT{}
%\item The command \verb|\HAT|\ixcom{HAT} prints \HAT{}
\item The command \verb|\QUES|\ixcom{QUES} prints \QUES{}
%\item The command \verb|\BS|\ixcom{BS} prints \BS{}
\item The command \verb|\IEQ|\ixcom{IEQ} prints \IEQ{}
\item The command \verb|\INEQ|\ixcom{INEQ} prints \INEQ{}
\item The command \verb|\nexp{an\_entity}|\ixcom{nexp} prints \nexp{an\_entity}
\item The command \verb|\xword{ExpResS\_KeyworD}|\ixcom{xword}
      prints \xword{ExpResS\_KeyworD}
\end{itemize}

The command \verb|\ix{|\meta{word or phrase}\verb|}|\ixcom{ix} both prints 
its parameter and also makes an index entry out of it.

The command \verb|\mnote{|\meta{Marginal note text}\verb|}|\ixcom{mnote}
prints its parameter as a 
marginal note. \mnote{Quite a lot of marginal note text.}
Remember, though, that marginal notes are only printed when the 
\file{isov2}\ixclass{isov2} class \file{draft}\ixopt{draft} option
is used. Marginal notes are not allowed by ISO.

\ssclause{Standard reference commands}

    Many parts of STEP use the same normative or informative references.
The most common of these are provided via commands. The currently available 
commands are listed in \tref{tab:nrefc}.
\ixcom{nrefasni}
\ixcom{nrefparti}
\ixcom{nrefpartxi}
\ixcom{nrefpartxii}
\ixcom{nrefpartxxi}
\ixcom{nrefpartxxii}
\ixcom{nrefpartxxxi}
\ixcom{nrefpartxxxii}
\ixcom{nrefpartxli}
\ixcom{nrefpartxlii}
\ixcom{nrefpartxliii}

    The naming convention used for references to parts of ISO~10303 is to
end the command name with the number of the part expressed in lower case 
Roman numerals. Should further references to parts of ISO~10303 be added later, 
the same naming convention will be used.

\begin{table}
\centering
\caption{Commands for common references to standards} \label{tab:nrefc}
\begin{tabular}{|l|l|} \hline
\textbf{Standard} & \textbf{Command} \\ \hline
ISO/IEC 8824-1 & \verb|\nrefasni| \\
ISO 10303-1    & \verb|\nrefparti|  \\
ISO 10303-11   & \verb|\nrefpartxi|  \\
ISO 10303-12   & \verb|\nrefpartxii|  \\
ISO 10303-21   & \verb|\nrefpartxxi|  \\
ISO 10303-22   & \verb|\nrefpartxxii|  \\
ISO 10303-31   & \verb|\nrefpartxxxi|  \\
ISO 10303-32   & \verb|\nrefpartxxxii|  \\
ISO 10303-41   & \verb|\nrefpartxli|  \\
ISO 10303-42   & \verb|\nrefpartxlii|  \\
ISO 10303-43   & \verb|\nrefpartxliii|  \\ \hline
\end{tabular}
\end{table}


\begin{example} The normative references in this document were input as:
\begin{verbatim}
\begin{nreferences}
\isref{ISO/IEC Directives, Part 2}{Rules for the structure and drafting 
       International Standards, Fourth edition.}
\isref{...}
\nrefparti
\nrefpartxi
\nrefpartxii
\nrefasni
\disref{P. R. WILSON:---}{LaTeX for ISO standards: User manual.}
\end{nreferences}
\end{verbatim}
\end{example}

\begin{anote}
For the commands providing references to STEP parts, the part number 
is denoted by lowercase Roman numerals. Should further reference
commands be provided for other STEP parts, then the same naming scheme
will be used.
\end{anote}

    Some informative bibliographic reference commands are also provided.

The command \verb|\bibidefo|\ixcom{bibidefo} produces the reference
entry to the IDEF0 document and \verb|\brefidefo|\ixcom{brefidefo}
can be used for citing the reference in the body of the document.

The commands \verb|\bibidefix|\ixcom{bibidefix} and
\verb|\bibieeedefix|\ixcom{bibieeedefix} produce the reference entry
to the original FIPS version of IDEF1X and the IEEE version of IDEF1X
respectively. 
The command \verb|\brefidefix|\ixcom{brefidefix} can be used for
citing an IDEF1X reference in the body of the document.

   IDEF0 and IDEF1X are references \brefidefo{} and \brefidefix{}
in the bibliography.


\begin{example} Part of the bibliography for this document looks like:
\begin{verbatim}
\begin{references}
...
\reference{BRYAN, M.,}{SGML --- An Author's Guide to the Standard Generalized
           Markup Language,}{Addison-Wesley Publishing Co., 1988. }\label{bryan}
\bibidefo
\bibieeeidefix
\reference{RESSLER, S.,}{The National PDES Testbed Mail Server User's Guide,}
           {NSTIR 4508, National Institute of Standards and Technology,
           Gaithersburg, MD 20899. January 1991.} \label{ressler}
...
\end{references}
\end{verbatim}
\end{example}

\begin{example}The source for one of the sentences above was:
\begin{verbatim}
IDEF0 and IDEF1X are references \brefidefo{} and \brefidefix{} in the bibliography.
\end{verbatim}
\end{example}

    
\sclause{Commands for documenting EXPRESS code} \index{express@\Express\}


    The Supplementary Directives\index{SD} specify the layout of the 
documentation of \Express{} code. The following commands are intended 
to serve two purposes:
\begin{enumerate}
\item To provide environments for the documentation of entity 
      attributes, etc.;
\item To provide begin and end tags around all the \Express{} code 
      documentation.
\end{enumerate}

    This latter purpose is to provide an enabling capability for the 
automatic extraction of portions of the documentation of an 
\Express{} model so that they could be placed into another document. 
For example, tools could be developed that would automatically extract 
pieces of resource model documentation and place them into an AP document.

\begin{anote}
This document uses the \file{hyphenat}\ixpack{hyphenat} 
package which enables automatic hyphenation of `words'
containing the underscore character command 
%(\verb|\_|\index{_/@\verb|\_|}). 
(\verb|\_|\index{_/@\texttt{\bs\_}}). 
Such words would normally have to
be coded as \verb|long\_\-word| to ensure potential hyphenation 
at the position of the underscore. When using the \file{hyphenat} package
it is an error to put the \verb|\-|\ixcom{-} discretionary
hyphen command after the underscore command as this then stops further
hyphenation.
\end{anote}


\ssclause{Environments ecode, eicode and excode}

    The \verb|ecode|\ixenv{ecode} environment is for 
tagging \Express{} code. It prints the appropriate title
and sets up the relevant fonts.

\begin{anexample} The following \latex{} source code:
\begin{verbatim}
\begin{ecode}\ixent{an\_entity}
\begin{verbatm}  % read verbatm as verbatim
*)
ENTITY an_entity;
  attr : REAL;
END_ENTITY;
(*
\end{verbatm}   % read verbatm as verbatim
\end{ecode}
\end{verbatim}

produces:

\begin{ecode}\ixent{an\_entity}
\begin{verbatim}
*)
ENTITY an_entity;
  attr : REAL;
END_ENTITY;
(*
\end{verbatim}
\end{ecode}
\end{anexample} % end example

    Similarly, the \verb|eicode|\ixenv{eicode} and
\verb|excode|\ixenv{excode} environments are for tagging \ExpressI{} 
and \ExpressX{} code and setting up the relevant titles and fonts.


\ssclause{Environment attrlist}

    The \verb|attrlist|\ixenv{attrlist} environment produces 
the heading for attribute definitions and sets up 
a \verb|description|\ixenv{description} list.

\begin{anexample}The following \latex{} source code:
\begin{verbatim}
\begin{attrlist}
\item[attr\_1] The \ldots
\item[attr\_2] This \ldots
\end{attrlist}
\end{verbatim}

produces:

\begin{attrlist}
\item[attr\_1] The \ldots
\item[attr\_2] This \ldots
\end{attrlist}
\end{anexample} % end example

\ssclause{Environment fproplist}

    The \verb|fproplist|\ixenv{fproplist} environment is similar to 
\verb|attrlist|\ixenv{attrlist} except that it is for
formal propositions.

\begin{anexample}The following \latex{} source code:
\begin{verbatim}
\begin{fproplist}
\item[un\_1] The value of \ldots\ shall be unique.
\item[gt\_0] The value of \ldots\ shall be greater than zero.
\end{fproplist}
\end{verbatim}

produces:

\begin{fproplist}
\item[un\_1] The value of \ldots\ shall be unique.
\item[gt\_0] The value of \ldots\ shall be greater than zero.
\end{fproplist}
\end{anexample} % end example

\ssclause{Other listing environments}

    The environments \verb|iproplist|\ixenv{iproplist}, 
\verb|enumlist|\ixenv{enumlist}, and \verb|arglist|\ixenv{arglist} are
similar to \verb|attrlist|\ixenv{attrlist}.
 Respectively they are environments for
informal propositions, enumerated items, and argument definitions.

\ssclause{Indexing}

    The command \verb|\ixent{|\meta{entity}\verb|}|\ixcom{ixent} 
generates an index
entry for the entity \meta{entity}.

    There are similar macros, each of which takes the name of the 
declaration as its argument, for indexing the other \Express{} declarations:
\verb|\ixenum|\ixcom{ixenum} for enumeration,
\verb|\ixfun|\ixcom{ixfun} for function,
\verb|\ixproc|\ixcom{ixproc} for procedure,
\verb|\ixrule|\ixcom{ixrule} for rule,
\verb|\ixsc|\ixcom{ixsc} for subtype\_constraint,
\verb|\ixschema|\ixcom{ixschema} for schema,
\verb|\ixselect|\ixcom{ixselect} for select, and
\verb|\ixtype|\ixcom{ixtype} for type.

\ssclause{Documentation tagging}

    Several environments are defined to tag the general documentation 
of \Express{} code. \index{express@\Express\}

    The environment \verb+\begin{espec}{+\meta{name}\verb+}+\ixenv{espec}
may be used to enclose, and give a name to, a complete specification 
block for an \Express{} entity. There are analogous environments --- 
\verb+fspec+\ixenv{fspec}, 
\verb+rspec+\ixenv{rspec}, 
\verb+sspec+\ixenv{sspec}, and
\verb+tspec+\ixenv{tspec} --- 
for functions, rules, schemas and types respectively.

    The \verb|dtext|\ixenv{dtext} environment may be used to anonymously 
enclose descriptive text.

\begin{example}\label{ex:code} Here is the suggested tagged documentation 
style for part of an \Express{} model.
\begin{verbatim}
%\ssclause{committee\_def}
\begin{espec}{committee_def}
\begin{dtext}
    A committee is composed of an odd number of people. 
Each committee also has a name.
    The ideal size of a committee is less than three.

\begin{anote} Figures and tables may also be placed here. \end{anote} % end note
\end{dtext}
\begin{ecode}\ixent{committee\_def}
\begin{verbatm} % read verbatm as verbatim
*)
ENTITY committee_def;
  title   : name;
  members : SET [1:?] OF person;
DERIVE
  ideal : BOOLEAN := SIZEOF(members) = 1;
UNIQUE
  un1 : title;
WHERE
  odd_members : ODD(SIZEOF(members));
END_ENTITY;
(*
\end{verbatm}   % read verbatm as verbatim
\end{ecode}
\begin{attrlist}
\item[title] The name of the committee.
\item[members] The people who form the committee.
\item[ideal] TRUE if there is only one person 
             on the committee.
             That is, if the committee is the ideal size.
\end{attrlist}
\begin{fproplist}
\item[un1] The \nexp{title} of the committee shall be unique.
\item[odd\_members] There shall be an odd number of people 
                    on the committee.
\end{fproplist}
\begin{iproplist}
\item[chair] The members of a committee shall appoint one of 
             their number as
             chair of the committee.
\end{iproplist}
\end{espec}
\end{verbatim}
\end{example} % end example

\begin{example}
The code in \eref{ex:code} produces the following result:

\begin{espec}{committee_def}
\begin{dtext}
    A committee is composed of an odd number of people. 
Each committee also has a name.
The ideal size of a committee is less than three.

\begin{anote} Figures and tables may also be placed here. \end{anote} % end note
\end{dtext}
\begin{ecode}\ixent{committee\_def}
\begin{verbatim}
*)
ENTITY committee_def;
  title   : name;
  members : SET [1:?] OF person;
DERIVE
  ideal : BOOLEAN := SIZEOF(members) = 1;
UNIQUE
  un1 : title;
WHERE
  odd_members : ODD(SIZEOF(members));
END_ENTITY;
(*
\end{verbatim}   % read verbatm as verbatim
\end{ecode}
\begin{attrlist}
\item[title] The name of the committee.
\item[members] The people who form the committee.
\item[ideal] TRUE if there is only one person on the committee. That is, if
             the committee is the ideal size.
\end{attrlist}
\begin{fproplist}
\item[un1] The \nexp{title} of the committee shall be unique.
\item[odd\_members] There shall be an odd number of people on the committee.
\end{fproplist}
\begin{iproplist}
\item[chair] The members of a committee shall appoint one of their number as
             chair of the committee.
\end{iproplist}
\end{espec}

\end{example} % end example


\sclause{Commands producing boilerplate text} \index{boilerplate}

    The following commands produce boilerplate text as specified by the 
Supplementary Directives\index{SD}.

\begin{anote}
 In the examples, 
the parameters of those commands that
take them have been specified in 
\textit{this font style} so their effects can
be seen in the resulting printed text.
\end{anote}

\ssclause{Definition of \ExpressG}

    The \verb|\expressgdef|\ixcom{expressgdef} prints the boilerplate
for where the definition of \ExpressG{} can be found.

\begin{anexample}
The command \verb|\expressgdef| prints: 

\expressgdef
\end{anexample} 

\ssclause{Major subdivision listing}

    The \verb|majorsublist|\ixenv{majorsublist}
environment prints the boilerplate for the heading of a listing of
major subdivisions of the standard and starts an itemized list.
An illustration of its use is given in \eref{ex:intro} 
on page~\pageref{ex:intro}.

The heading text is produced by the 
\verb|\majorsubname|\ixcom{majorsubname} command.

\begin{anexample} The command \verb|\majorsubname| command prints:

\majorsubname

\end{anexample}

\ssclause{Schema introduction}

    The command \verb|\schemahead{|\meta{schema name}\verb|}|\ixcom{schemahead} prints the heading for a schema clause.

    The command \verb+\schemaintro{+\meta{schema name}\verb+}+\ixcom{schemaintro} 
produces the boilerplate for the introduction to an \Express{} schema
clause.

\begin{anexample}The command \verb|\schemaintro{\nexp{this\_schema}}| prints:

\schemaintro{\nexp{this\_schema}}
\end{anexample}
  


\ssclause{Short names of entities}

    The command \verb|\shortnamehead|\ixcom{shortnamehead} prints the
heading for the short names annex.

    The command \verb|\shortnames|\ixcom{shortnames} 
produces the boilerplate for the
introduction to the annex listing short names.

\begin{anexample}The command \verb|\shortnames| prints:

\shortnames 
\end{anexample} %end example

\ssclause{Registration commands}

    The command \verb|\objreghead|\ixcom{objreghead} prints the heading
for the information object registration annex.

    The command \verb|\docidhead|\ixcom{docidhead} prints the heading
for the document identification subclause.


    The command 
\verb+\docreg{+\meta{version no}\verb+}+\ixcom{docreg}
produces the boilerplate for document registration. The command takes
one parameter:
\meta{version no} which is the version number.\footnote{The
SD say that the version number should be 1 for a first edition IS.
The version number is incremented by one for each corrigenda,
amendment or new edition.}

\begin{example}The command \verb|\docreg{1}|
         prints:

\docreg{\textit{1}} 
\end{example} % end example

    The command \verb|\schemaidhead|\ixcom{schemaidhead} prints the heading
for the schema identification subclause. 
The command 
\verb|\aschemaidhead{|\meta{schema name}\verb|}|\ixcom{aschemaidhead} 
prints the heading for a particular schema identification subsubclause.

    The command
\verb+\schemareg{+\meta{version no}\verb+}{+\meta{p2}\verb+}{+\meta{p3}\verb+}{+\meta{p4}\verb+}{+\meta{p5}\verb+}{+\meta{p6}\verb+}+\ixcom{schemareg} produces the boilerplate concerning
schema registration. The command takes six parameters.
\begin{enumerate}
\item \meta{version no} The version number;
\item \meta{p2} The name of an \Express{} schema (with underscores);
\item \meta{p3} The number of the schema object (typically 1);
\item \meta{p4} The name of the schema, with hyphens replacing any
                  underscores in the name;
\item \meta{p5} The number identifying the schema;
\item \meta{p6} The clause or annex in which the schema is defined.
\end{enumerate}

\begin{example}The command \\
 \verb|\schemareg{1}{a\_schema}{3}{a-schema}{5}{clause 6}|
prints:

\schemareg{\textit{1}}{a\_schema}{\textit{3}}{\textit{a-schema}}{\textit{5}}{\textit{clause 6}}
\end{example} % end example


\ssclause{Computer interpretable listings} 

    The command \verb|\listingshead|\ixcom{listingshead} prints the
heading for the computer interpretable listings annex.

    The command 
\verb|\expurls{|\meta{short}\verb|}{|\meta{express}\verb|}|\ixcom{expurls}
produces the boilerplate for the introduction to the annex 
listing short names and \Express, where \meta{short} is the URL for the short
names and \meta{express} is the URL for the \Express.

\begin{anexample} The command 
  \verb|\expurls{http:/www.short/}{http://www.express/}| prints:

\expurls{http://www.short/}{http://www.express/}

\end{anexample}

\clearpage
\clause{The \file{ir} package facility} 

    The \file{ir}\ixpack{ir} package provides commands and environments
specifically for the ISO~10303 Integrated Resources series of documents.

    Use of this package requires the use of the \file{step}\ixpack{step} 
package.

\sclause{Boilerplate commands}

    The \file{ir} package modifies the \verb|\fwdshortlist|\ixcom{fwdshortlist}
command to produce extra IR-specific boilerplate.

    The following commands produce boilerplate text as specified by the 
SD\index{SD}.


\ssclause{Integrated resource EXPRESS-G} 

    The command \verb|\expressghead|\ixcom{expressghead} prints the
heading for the \ExpressG{} diagrams annex.

    The command \verb+\irexpressg+\ixcom{irexpressg} 
produces the boilerplate for the introduction to the integrated 
resource \ExpressG{} annex.
\index{expressg@\ExpressG\}

\begin{anexample}The command \verb|\irexpressg| prints:

\irexpressg

 \end{anexample} % end example

%%%%%%%%%%%%%%%%%%%%%%%%%%%%%%
%%%%\end{document}
%%%%%%%%%%%%%%%%%%%%%%%%%%%%%%



\clearpage
\clause{The \file{ap} package facility}

    The \file{ap}\ixpack{ap} package provides commands and environments
specifically for the ISO~10303 Application Protocol series of documents.

    Use of this package requires the use of the \file{step}\ixpack{step} 
package.

\sclause{Preamble commands}

    Certain commands shall be put in the preamble of an AP document.

    The command 
\verb+\aptitle{+\meta{title of AP}\verb+}+\ixcom{aptitle}
shall be put into the preamble. \index{preamble} The parameter shall be of 
such a form that
it will read naturally in a sentence of the form: 
`\ldots for the \meta{title of AP} application protocol.'.

\begin{anexample}
  For the purposes of later examples, the command
\verb|\aptitle{|\texttt{\theap}\verb|}| has been put in the preamble
of this document.
\end{anexample} % end example

    If the AP makes use of one or more
AICs\index{AIC}, then the command \verb|\aicinaptrue|\ixcom{aicinaptrue} 
shall be put in the document preamble.

   If a mapping specification is used instead of a mapping table,
the command \verb|\mapspectrue|\ixcom{mapspectrue} shall be put
in the preamble. If mapping templates are used then 
\verb|\maptemplatetrue|\ixcom{maptemplatetrue} shall also be put in the
preamble.

    If IDEF1X is used instead of \ExpressG{} as the graphical form for the
ARM, then \verb|\idefixtrue|\ixcom{idefixtrue} shall be put in the preamble.

  

\sclause{Heading commands}

    These commands start document clauses with particular titles. The
commands that take no parameters are listed in \tref{tab:apnpheads}.
Some of these commands have predefined labels, which are also listed in 
the table.
\ixcom{inforeqhead}
\ixcom{uofhead}
\ixcom{applobjhead}
\ixcom{applasserthead}
\ixcom{aimhead}
\ixcom{maptablehead}
\ixcom{templateshead}
\ixcom{aimshortexphead}
\ixcom{confreqhead}
\ixcom{aimlongexphead}
\ixcom{aimshortnameshead}
\ixcom{impreqhead}
\ixcom{aamhead}
\ixcom{aamdefhead}
\ixcom{aamfighead}
\ixcom{armhead}
\ixcom{aimexpressghead}
\ixcom{aimexpresshead}
\ixcom{apusagehead}

\settowidth{\prwlen}{\quad Application activity model definitions}
\begin{table}
\centering
\caption{AP package parameterless heading commands}
\label{tab:apnpheads}
\begin{tabular}{|l|c|p{\prwlen}|l|} \hline
\textbf{Command} & \textbf{Clause} & \textbf{Default text} & \textbf{Label} \\ \hline
\verb|\inforeqhead| & C & Information requirements & \verb|;sireq| \\
\verb|\uofhead| & SC & Units of functionality  & \verb|;suof| \\
\verb|\applobjhead| & SC & Application objects  & \verb|;sao| \\
\verb|\applasserthead| & SC & Application assertions  & \verb|;saa| \\
\verb|\aimhead| & C & Application interpreted model & \verb|;saim| \\
\verb|\mappinghead| & SC & Mapping table, or & \verb|;smap| \\
                    &    & Mapping specification  & \verb|;smap| \\
\verb|\templateshead| & SSC & Mapping templates &    \\
\verb|\aimshortexphead| & SC & AIM \Express{} short listing & \verb|;saesl| \\
\verb|\confreqheadhead| & C & Conformance requirements & \verb|;scr| \\
\verb|\aimlongexphead| & NA & AIM \Express{} expanded listing  & \verb|;saeel| \\
\verb|\aimshortnameshead| & NA & AIM short names  & \verb|;sasn| \\
\verb|\impreqhead| & NA & Implementation method specific requirements & \verb|;simreq| \\
\verb|\aamhead| & IA & Application activity model & \verb|;saam| \\
\verb|\aamdefhead| & SC & Application activity model definitions and abbreviations & \verb|| \\
\verb|\aamfighead| & SC & Application activity model diagrams & \verb|| \\
\verb|\armhead| & IA & Application reference model & \verb|;sarm| \\
\verb|\aimexpressghead| & IA & AIM \ExpressG{} & \verb|;saeg| \\
\verb|\aimexpresshead| & IA & AIM \Express{} listing & \verb|| \\
\verb|\apusagehead| & IA & Application protocol usage guide & \verb|;sapug| \\
 \hline
\end{tabular}
\end{table}

    The commands listed in \tref{tab:appheads} take parameters.
\ixcom{auofhead}
\ixcom{mapuofhead}
\ixcom{mapobjecthead}
\ixcom{mapattributehead}

\begin{table}
\centering
\caption{AP package parameterized heading commands}
\label{tab:appheads}
\begin{tabular}{|l|c|l|} \hline
\textbf{Command} & \textbf{Clause} & \textbf{Parameterized title} \\ \hline
\verb|\auofhead| & SSC & \meta{UoF} \\ 
\verb|\mapuofhead| & SSC & \meta{UoF} \\
\verb|\mapobjecthead| & SSSC & \meta{application object} \\
\verb|\mapattribhead| & SSSSC & \meta{attribute} \\
\hline
\end{tabular}
\end{table}

\sclause{Boilerplate commands}

    The following commands produce boilerplate text as specified by the 
SD\index{SD}.

\begin{anote}
 In the examples, the parameters of those commands that
take them have been specified in 
\textit{this font style} so their effects can
be seen in the resulting printed text.
\end{anote}

\ssclause{AP introduction}

    The command \verb|\apextraintro|\ixcom{apextraintro} produces extra
boilerplate for the Introduction to an AP.

\begin{anexample}The command \verb|\apextraintro| prints:

\apextraintro
\end{anexample} %end example

\ssclause{AP scope}

    The command \verb+\apscope{+\meta{application purpose and context}\verb+}+\ixcom{apscope} 
produces the boilerplate for the start of an AP scope\index{scope} clause.

\begin{anexample}The command \verb|\apscope{application purpose and context.}|
         prints:

\apscope{\textit{application purpose and context.}} 
\end{anexample} 

\ssclause{AP information requirements}

  The command \verb|\inforeqhead|\ixcom{inforeqhead} prints the
heading for the information requirements clause.

  The command \verb+\apinforeq{+\meta{AP purpose}\verb+}+\ixcom{apinforeq} 
produces the boilerplate for the clause.

\begin{anexample}The command \verb|\apinforeq{AP purpose.}| prints: 

\apinforeq{\textit{AP purpose.}} 
\end{anexample} % end example

\ssclause{AP UoF}

    The command \verb|\uofhead|\ixcom{uofhead} prints the heading
for the UoF subclause.

    The environment 
\verb+\begin{apuof}+\meta{item list}\verb+\end{apuof}+\ixenv{apuof} 
produces the boilerplate for the introduction to the clause.

\begin{anexample} Remembering that \verb|\aptitle|\ixcom{aptitle}
                  was set to \texttt{\theap} in the preamble,
                  the commands
\begin{verbatim}
\begin{apuof}
\item Name of UoF1;
\item Name of UoF2;
\item Name of UoFn.
\end{apuof}
\end{verbatim}
prints:

\begin{apuof}
\item Name of UoF1;
\item Name of UoF2;
\item Name of UoFn.
\end{apuof}

\end{anexample}

\ssclause{AP application objects}

    The command \verb|\applobjhead|\ixcom{applobjhead} prints the
heading for the application objects subclause.

    The command \verb|\apapplobj|\ixcom{apapplobj} produces the 
boilerplate for the introduction to the clause.

\begin{anexample} Remembering that \verb|\aptitle|\ixcom{aptitle}
                  was set to \texttt{\theap} in the preamble,
                  the command \verb|\apapplobj| prints:

\apapplobj

\end{anexample}

\ssclause{AP assertions}

    The command \verb|\applasserthead|\ixcom{applasserthead} prints the
heading for the application assertions subclause.

    The command \verb|\apassert|\ixcom{apassert}
produces the boilerplate for the clause.

\begin{anexample} Remembering that \verb|\aptitle|\ixcom{aptitle}
                  was set to \texttt{\theap} in the preamble,
                  the command \verb|\apassert| prints:

\apassert

\end{anexample}


\ssclause{AP mapping table/specification}

    The command \verb|\mappinghead|\ixcom{mappinghead} prints
the heading for the mapping table or mapping specification subclause.
The heading text depends on whether or not 
\verb|\mapspectrue|\ixcom{mapspectrue} was put in the preamble.

    The command \verb|\apmapping|\ixcom{apmapping} 
produces the boilerplate for the introduction to the AP mapping table
or specification clause.

\begin{anote}AICs are included in the boilerplate only if the command
\verb|\aicinaptrue|\ixcom{aicinaptrue} is included
in the preamble.
\end{anote}

\begin{example}By default, or when \verb|\mapspecfalse| is in
the preamble, the command \verb|\apmapping|
         prints: \mapspecfalse

\apmapping
\end{example} % end example

\begin{example}When \verb|\mapspectrue| is in the preamble, the command \verb|\apmapping|
         prints: \mapspectrue

\apmapping
\end{example} % end example

\sssclause{AP mapping templates}

    The command \verb|\aptemplatehead|\ixcom{aptemplatehead} prints
the heading for the mapping template subclause (if any).

    The command \verb|\apmaptemplate|\ixcom{apmaptemplate} prints
the boilerplate for the introduction to the clause. This refers to the
UoFs in the AP. The first of the UoFs shall be labelled as 
\verb|\label{;uof1}| and the last of the UoFs shall be
labelled as \verb|\label{;uoflast}|.

\begin{example} If there are three UoFs, then there should be headings
of the form:
\begin{verbatim}
\mapuofhead{First UoF}\label{;uof1}
...
\mapuofhead{Second UoF}...
...
\mapuofhead{Third UoF}\label{;uoflast}
...
\end{verbatim}
\end{example}

\begin{example} Assuming that there are three UoFs as in the previous example, 
the command \verb|\apmaptemplate| prints:

\apmaptemplate
\end{example} % end example

    The command \verb|\sstemplates|\ixcom{sstemplates} prints
the two subclauses for the \xword{subtype} and \xword{SuPeRtype} templates.

\begin{example} \label{ex:sstemplates} In this document, 
and noting that the clause
numbering is not the same as in a real AP document, 
the command \verb|\sstemplates|
         prints:

\sstemplates

\end{example} % end example


\sssclause{Template headings}

    There are three headings used within a mapping template.

    The command \verb|\signature|\ixcom{signature} prints the underlined
Mapping signature header.

    The command \verb|\parameters|\ixcom{parameters} prints the underlined
Parameter definition header.

    The command \verb|\body|\ixcom{body} prints the underlined
Template body header.

\begin{anexample} The results of using the \verb|\signature|\ixcom{signature}
and \verb|\parameters|\ixcom{parameters} commands were illustrated
in \eref{ex:sstemplates} on \pref{ex:sstemplates}.
\end{anexample}



\ssclause{AIM short EXPRESS listing}

    The command \verb|\aimshortexphead|\ixcom{aimshortexphead} prints
the heading for the AIM EXPRESS short listing subclause.

    The command \verb|\apshortexpress|\ixcom{apshortexpress} produces 
the boilerplate for the
first paragraph of the clause.

\begin{anote}AICs are included in the boilerplate only if the command
\verb|\aicinaptrue|\ixcom{aicinaptrue} is included in the preamble.
\end{anote}

\begin{example}
The command \verb|\apshortexpress| without \verb|\aicinaptrue|
in the preamble produces:

\aicinapfalse
\apshortexpress

\end{example} % end example

\begin{example}
With \verb|\aicinaptrue| set in the preamble the command
\verb|\apshortexpress| produces the following:

\aicinaptrue
\apshortexpress
\end{example} % end example


\ssclause{AP conformance}

    The command \verb|\confreqhead|\ixcom{confreqhead} prints the
heading for the conformance requirements clause.

    The command 
\verb+\apconformance{+\meta{implementation methods}\verb+}+\ixcom{apconformance} 
produces the boilerplate for the introduction to the clause.

    The environment 
\verb+\begin{apconformclasses}+\meta{item list}\verb+\end{apconformclasses}+\ixenv{apconformclasses} 
provides some additional boilerplate.

\begin{example}The command \verb|\apconformance{ISO 10303-21, ISO 10303-22}|
         prints:

\apconformance{\textit{ISO 10303-21, ISO 10303-22}} 
\end{example} % end example

\begin{example}The commands
  \begin{verbatim}
\begin{apconformclasses}
\item first class;
\item second class;
\item last class.
\end{apconformclasses}
\end{verbatim}
         print:

\begin{apconformclasses}
\item first class;
\item second class;
\item last class.
\end{apconformclasses}
\end{example}


\ssclause{EXPRESS expanded listing}

    The command \verb|\aimlongexphead|\ixcom{aimlongexphead} prints
the heading for the AIM expanded listing clause.

    The command \verb|\aimlongexp|\ixcom{aimlongexp} 
produces the boilerplate for the introduction to the clause.

\begin{anexample}The command \verb|\aimlongexp|
         prints:

\aimlongexp 
\end{anexample} % end example

\ssclause{AIM short names}

    The command \verb|\aimshortnamehead|\ixcom{aimshortnamehead} prints
the heading for the AIM short names annex.

    The command \verb|\apshortnames|\ixcom{apshortnames} 
produces the boilerplate for the introduction to the AP short name annex.

\begin{anexample}The command \verb|\apshortnames|
         prints:

\apshortnames 
\end{anexample} % end example

\ssclause{Implementation requirements}

    the command \verb|\impreqhead|\ixcom{impreqhead} prints the heading
for implementation method-specific reguirements.

    The command \verb+\apimpreq{+\meta{schema name}\verb+}+\ixcom{apimpreq}
produces the boilerplate for the requirements on exchange structure.

\begin{anexample}The command \verb|\apimpreq{schema\_name}|
         prints:

\apimpreq{\textit{schema\_name}} 
\end{anexample} % end example


\ssclause{AP PICS}

    The command \verb|\picshead|\ixcom{picshead}, 
from the \file{step}\ixpack{step} package,
prints the heading for the PICS annex.

    The command \verb|\picsannex|\ixcom{picsannex}
produces the boilerplate for the start of the AP PICS annex.

\begin{anexample}The command \verb|\picsannex|
         prints:

\picsannex 
\end{anexample} % end example

\ssclause{AAM annex}

    The command \verb|\aamhead|\ixcom{aamhead} prints the heading for
the AAM annex.


    The command \verb|\apaamintro|\ixcom{apaamintro} 
 produces the introductory boilerplate for the introduction of
the AP annex on application activity models.

\begin{anexample}
  The command \verb|\apaamintro| prints:

\apaamintro

\end{anexample} % end example

\ssclause{AP AAM definitions}

    The command \verb|\aamdefhead|\ixcom{aamdefhead} prints the heading
for the AAM definitions subclause.

    The command \verb|\apaamdefs|\ixcom{apaamdefs} produces 
the boilerplate at the start of
the AP subclause on AAM definitions and abbreviations.

\begin{anexample}
  The command \verb|\apaamdefs| prints:

\apaamdefs
\end{anexample} % end example

\ssclause{AAM diagrams annex}

    The command \verb|\aamfighead|\ixcom{aamfighead} prints the heading
for the AAM diagrams subclause.

    The command 
\verb|\aamfigrange{|\meta{figure range}\verb|}|\ixcom{aamfigrange} 
is used to store the activity model diagram figure range for later use.

\begin{example}
    For the purposes of this document we set
\begin{verbatim}
\aamfigrange{figures F.1 through F.n}
\end{verbatim}

\aamfigrange{\textit{figures F.1 through F.n}}

\end{example}

    The command \verb+\aamfigures+\ixcom{aamfigures}
produces the boilerplate for the introduction to an APs AAM figure
subclause.

\begin{example} Noting that we have set 
\verb|\aamfigrange{figures F.1 through F.n}|\ixcom{aamfigrange}, 
the command \verb|\aamfigures| prints:

\aamfigures

\end{example}

\ssclause{ARM annex}

    The command \verb|\armhead|\ixcom{armhead} prints the heading for the
ARM annex.

    The command 
\verb+\armintro+\ixcom{armintro}
produces the boilerplate for the introduction to the ARM figures.

\begin{anexample}The command 
          \verb|\armintro| 
         prints:

\armintro 
\end{anexample} % end example

\ssclause{AIM EXPRESS-G annex}

    The command \verb|\aimexpressghead|\ixcom{aimexpressghead} 
prints the heading for the AIM \ExpressG{} annex.
 

    The command 
\verb+\aimexpressg+\ixcom{aimexpressg}
produces the boilerplate for the introduction to an AP's AIM \ExpressG{}
model. 

\begin{anexample}The command \verb|\aimexpressg|
         prints:

\aimexpressg
\end{anexample} % end example

\ssclause{AIM EXPRESS listing}

    The command \verb|\aimexpresshead|\ixcom{aimexpresshead} prints
the heading for the AIM listing annex.

%    The command \verb|\aimexplisting|\ixcom{aimexplisting}
%produces the boilerplate for the introduction to an AIMs short name and
%\Express{} listing.
%
%
%\begin{example}The command \verb|\aimexplisting|
%         prints:
%
%\aimexplisting 
%\end{example}

    The command 
\verb|\apexpurls{|\meta{short}\verb|}{|\meta{express}\verb|]|\ixcom{apexpurls}
produces the boilerplate for the introduction to the AP annex
listing short names and \Express, where \meta{short} is the URL for the short
names and \meta{express} is the URL for the \Express.

\begin{anexample} The command \verb|\apexpurls{http:/www.short/}{http://www.express/}|
prints:

\apexpurls{http://www.short/}{http://www.express/}

\end{anexample}

%%%%%%%%%%%%%%%%%%%%%%%%%%%%%%
%%%%%%\end{document}
%%%%%%%%%%%%%%%%%%%%%%%%%%%%%%



\clearpage
\clause{The \file{aic} package facility}

    The \file{aic}\ixpack{aic} package
provides commands and environments specifically
for the ISO~10303 Application Interpreted Construct series of
documents.

    The use of this package requires the use of the 
\file{step}\ixpack{step} package.

\sclause{Heading commands}

    The commands described in this subclause start document clauses with
particular titles.

    The commands that take no parameters are listed in \tref{tab:aicnpheads}.
\ixcom{aicshortexphead}

\begin{table}[btp]
\centering
\caption{AIC package parameterless heading commands}
\label{tab:aicnpheads}
\begin{tabular}{|l|c|l|l|} \hline
\textbf{Command} & \textbf{Clause} & \textbf{Default text} & \textbf{Label} \\ \hline
\verb|\aicshortexphead| & C & \Express{} short listing & \verb|;sesl| \\
\hline
\end{tabular}
\end{table}

\sclause{Boilerplate commands}

    The following commands produce boilerplate text as specified by the
Supplementary Directives. 


\ssclause{Introduction text}

    The command \verb|\aicextraintro|\ixcom{aicextraintro}
prints additional boilerplate for the Introduction to an AIC.

\begin{anexample}The command \verb|\aicextraintro|
         prints:

\aicextraintro
\end{anexample}

\ssclause{Definition of AIC}

    The command \verb|\aicdef|\ixcom{aicdef}
prints the definition of `AIC'. It shall only be used within the
\verb|definitions|\ixenv{definitions} environment.

\begin{anexample}The commands:
         \begin{verbatim}
         \begin{definitions}
         \aicdef
         \end{definitions}
         \end{verbatim}
         produce:

\begin{definitions}
\aicdef
\end{definitions}
\end{anexample} % end example

\ssclause{Short EXPRESS listing}

    The command \verb|\aicshortexphead|\ixcom{aicshortexphead} prints
the heading for the AIC short \Express{} annex.

    The command \verb|\aicshortexpintro|\ixcom{aicshortexpintro}
prints boilerplate for the introduction to the short \Express{} listing.

\begin{anexample}The command \verb|\aicshortexpintro|
         prints:

\aicshortexpintro  
\end{anexample} % end example

\ssclause{EXPRESS-G figures}

    The command \verb|\expressghead|\ixcom{expressghead}, 
from the \file{step} package, prints the heading for the \ExpressG{} diagrams
annex.

    The command 
\verb+\aicexpressg+\ixcom{aicexpressg} 
prints boilerplate for the introduction to the \ExpressG\ figures.

\begin{anexample}The command \verb|\aicexpressg|
         prints:

\aicexpressg
\end{anexample}

%%%%%%%%%%%%%%%%%%%%%%%%%
%%%\end{document}
%%%%%%%%%%%%%%%%%%%%%%%%%

\clearpage
\clause{The \file{ats} package facility}

    The \file{ats}\ixpack{ats} package
provides commands and environments specifically
for the ISO~10303 Abstract Test Suite series of
documents.

    The use of this package requires the use of the 
\file{step}\ixpack{step} package.

\sclause{Preamble commands}

    Certain commands shall be put in the preamble\index{preamble} 
of an ATS document.

    The command 
\verb+\APnumber{+\meta{number}\verb+}+\ixcom{APnumber} shall be put 
in the preamble,
where \meta{number} is the ISO 10303 part number of the corresponding AP.

\begin{example}
For the purposes of later examples, the command
\verb+\APnumber{+\texttt{\theAPpartno}\verb+}+ has been put in the preamble.
of this document.
\end{example}

    The command 
\verb+\APtitle{+\meta{title of AP}\verb+}+\ixcom{APtitle} shall be put 
in the preamble,
where \meta{title of AP} is the ISO 10303 part title of the
corresponding AP. This must be given in such a manner that it reads
sensibly in a sentence of the form `\ldots for ISO 10303-299,
application protocol \meta{title of AP}.'

\begin{example}
For the purposes of later examples, the command
\verb+\APtitle{+\texttt{\theAPtitle}\verb+}+ 
has been put in the preamble of this document.
\end{example}

    The command
\verb+\mapspectrue+\ixcom{mapspectrue}
shall be put in the preamble if the AP uses a mapping specification rather
than a mapping table.

\sclause{Heading commands}

    These commands start document clauses with particular titles.
The commands that take no parameters are listed in \tref{tab:atshead}.
\ixcom{purposeshead}
\ixcom{domainpurposehead}
\ixcom{aepurposehead}
\ixcom{apobjhead}
\ixcom{apasserthead}
\ixcom{aimpurposehead}
%%%\ixcom{extrefpurposehead}
\ixcom{implementpurposehead}
%%%\ixcom{rulepurposehead}
\ixcom{otherpurposehead}
\ixcom{gtpvchead}
\ixcom{generalpurposehead}
\ixcom{gvcatchead}
\ixcom{gvcprehead}
\ixcom{gvcposthead}
\ixcom{atchead}
\ixcom{prehead}
\ixcom{posthead}
\ixcom{confclassannexhead}
\ixcom{postipfilehead}
%%%\ixcom{excludepurposehead}
\ixcom{atsusagehead}

\settowidth{\prwlen}{\quad General verdict criteria for all abstract}
\begin{table}
\centering
\caption{ATS package parameterless heading commands} \label{tab:atshead}
\begin{tabular}{|l|c|p{\prwlen}|} \hline
\textbf{Command}             & \textbf{Clause} & \textbf{Default text} \\ \hline
\verb|\purposeshead|         & C   & Test purposes  \\
\verb|\aepurposehead|        & SC  & Application element test purposes \\
\verb|\aimpurposehead|       & SC  & AIM test purposes \\
\verb|\implementpurposehead| & SC  & Implementation method test purposes \\
\verb|\domainpurposehead|    & SC  & Domain test purposes \\
\verb|\otherpurposehead|     & SC  & Other test purposes \\

\verb|\gtpvchead|            & C   & General test purposes and verdict criteria \\
\verb|\generalpurposehead|   & SC  & General test purposes \\
\verb|\gvcatchead|           & SC  & General verdict criteria for all abstract test cases \\
\verb|\gvcprehead|           & SC  & General verdict criteria for preprocessor abstract test cases \\
\verb|\gvcposthead|          & SC  & General verdict criteria for postprocessor abstract test cases \\

\verb|\atchead|              & C   & Abstract test cases \\
\verb|\prehead|              & SSC & Preprocessor \\
\verb|\precoveredhead|       & SSSC & Test purposes covered \\
\verb|\preinputhead|         & SSSC & Input specification \\
\verb|\precriteriahead|      & SSSC & Verdict criteria \\
\verb|\preconstraintshead|   & SSSC & Constraints on values \\
\verb|\preexechead|          & SSSC & Execution sequence \\
\verb|\preextrahead|         & SSSC & Extra details \\


\verb|\posthead|             & SSC & Postprocessor \\
\verb|\postcoveredhead|       & SSSC & Test purposes coverage \\
\verb|\postinputhead|         & SSSC & Input specification \\
\verb|\postcriteriahead|      & SSSC & Verdict criteria \\
\verb|\postexechead|          & SSSC & Execution sequence \\
\verb|\postextrahead|         & SSSC & Extra details \\

\verb|\confclassannexhead|   & NA  & Conformance classes \\
\verb|\postipfilehead|       & NA  & Postprocessor input specification file names \\

\verb|\atsusagehead|         & IA  & Usage scenarios \\

\verb|\apasserthead|         & SSC & Application assertions \\
%%%\verb|\extrefpurposehead|    & SC  & External reference test purposes \\
%%%%\verb|\rulepurposehead|      & SC  & \rulepurposename\  \\
%%%\verb|\excludepurposehead|   & NA  & Excluded test purposes \\ 
\hline
\end{tabular}
\end{table}


    The commands that take a parameter are listed in \tref{tab:atsphead}.
\ixcom{apobjhead}
\ixcom{aimenthead}
\ixcom{atctitlehead}
\ixcom{confclasshead}

\begin{table}
\centering
\caption{ATS package parameterized heading commands} \label{tab:atsphead}
\begin{tabular}{|l|c|l|} \hline
Command               & Clause & Parameterized title \\ \hline
\verb|\apobjhead|     & SSC & \meta{Application object n}  \\
\verb|\aimenthead|    & SSC & \meta{Entity name} \\
\verb|\atctitlehead|  & SC  & \meta{Title} \\
\verb|\confclasshead| & SC  & Conformance class \meta{number}  \\ \hline
\end{tabular}
\end{table}


\sclause{Keyword commands}

    Several keyword (headings) are used in documenting a test case.
\latex{} commands for these keywords are given in \tref{tab:atskey}.
\ixcom{atssummary}
\ixcom{atscovered}
\ixcom{atsinput}
\ixcom{atsconstraints}
\ixcom{atsverdict}
\ixcom{atsexecution}
\ixcom{atsextra}

\begin{table}
\centering
\caption{ATS package keyword commands} \label{tab:atskey}
\begin{tabular}{|l|l|} \hline
Command                & Effect \\ \hline
\verb|\atssummary|     & \atssummary{} \\
\verb|\atscovered|     & \atscovered{} \\
\verb|\atsinput|       & \atsinput{} \\
\verb|\atsconstraints| & \atsconstraints{} \\
\verb|\atsverdict|     & \atsverdict{} \\
\verb|\atsexecution|   & \atsexecution{} \\
\verb|\atsextra|       & \atsextra{} \\ \hline
\end{tabular}
\end{table}

\sclause{Boilerplate commands}

    The following commands produce boilerplate text.

\begin{anote}
 In the examples, the
parameters of those commands that take them have been specified in
\textit{this font style} so that their
effects can be seen in the printed text.
\end{anote}

\ssclause{ATS introduction}

    The command 
\verb|\atsintroendbp|\ixcom{atsintroendbp}
 produces the boilerplate
for the end of the Introduction to an ATS.

\begin{anexample} Remembering that in the preamble 
        \verb|\APnumber|\ixcom{APnumber} was set to \texttt{\theAPpartno} 
        and \verb|\APtitle|\ixcom{APtitle} was set to \texttt{\theAPtitle},
the command \verb|\atsintroendbp| prints:

\atsintroendbp

\end{anexample}


\ssclause{ATS scope}

    The command \verb|\scopeclause|\ixcom{scopeclause}, from the \file{isov2}
class, prints the heading for the Scope clause.

    The command 
\verb|\atsscopebp|\ixcom{atsscopebp}
produces boilerplate for an ATS \textit{Scope}
clause.

\begin{anexample}  Remembering that in the preamble 
        \verb|\APnumber|\ixcom{APnumber} was set to \texttt{\theAPpartno}, 
the command \verb|\atsscopebp| prints:

\atsscopebp

\end{anexample}

\ssclause{Test purpose}

    The command \verb|\purposehead|\ixcom{purposehead} prints the heading
for the test purposes clause.

    The command \verb|\atspurposebp|\ixcom{atspurposebp} 
prints boilerplate for the introduction to the clause.

\begin{anexample}  Remembering that in the preamble 
        \verb|\APnumber|\ixcom{APnumber} was set to \texttt{\theAPpartno}, 
the command \verb|\atspurposebp| prints:

\atspurposebp

\end{anexample}

\ssclause{Application element test purposes}

    The command \verb|\aepurposehead|\ixcom{aepurposehead} prints the
heading for the application element test purposes subclause.

    The command 
\verb|\aetpbp|\ixcom{aetpbp}
prints boilerplate for the clause.

\begin{anexample} Remembering that in the preamble 
        \verb|\APnumber|\ixcom{APnumber} was set to \texttt{\theAPpartno}, 
the command \verb|\aetpbp| prints:

\aetpbp

\end{anexample}

\ssclause{AIM test purposes}

    The command \verb|\aimpurposehead|\ixcom{aimpurposehead} prints the
heading for the AIM test purposes subclause.

    The command 
\verb|\aimtpbp|\ixcom{aimtpbp}
prints boilerplate for the clause.

\begin{anexample} Remembering that in the preamble 
        \verb|\APnumber|\ixcom{APnumber} was set to \texttt{\theAPpartno}, 
the command \verb|\aimtpbp| prints:

\aimtpbp

\end{anexample}

\ssclause{Implementation method test purposes}

    The command \verb|\implementpurposehead|\ixcom{implementpurposehead} prints the
heading for the implementation method test purposes subclause.

    The command 
\verb|\atsimtpbp|\ixcom{atsimtpbp}
prints boilerplate for the clause.

\begin{anexample} Remembering that in the preamble 
        \verb|\APnumber|\ixcom{APnumber} was set to \texttt{\theAPpartno}, 
the command \verb|\atsimtpbp| prints:

\atsimtpbp

\end{anexample}



\ssclause{General test purposes and verdict criteria}

    The command \verb|\gtpvchead|\ixcom{gtpvchead} prints the heading
for the general test purposes and verdict criteria clause.

    The command 
\verb|\atsgtpvcbp|\ixcom{atsgtpvcbp} 
prints boilerplate for the clause


\begin{anexample} The command \verb|\atsgtpvcbp| prints:

\atsgtpvcbp
\end{anexample}

\ssclause{General test purposes}

    The command \verb|\generalpurposehead|\ixcom{generalpurposehead} prints
the heading for the general test purposes subclause.

    The command 
\verb|\gtpbp|\ixcom{gtpbp}
prints boilerplate for the suclause.

\begin{anexample} Remembering that in the preamble 
        \verb|\APnumber|\ixcom{APnumber} was set to \texttt{\theAPpartno}, 
the command \verb|\gtpbp| prints:

\gtpbp

\end{anexample}

\ssclause{General verdict criteria}

    The command \verb|\gvcatchead|\ixcom{gvcatchead} prints the
heading for the general verdict criteria for all cases subclause.

    The command 
\verb|\gvatcbp|\ixcom{gvatcbp}
prints boilerplate for the subclause.

\begin{anexample} Remembering that in the preamble 
        \verb|\APnumber|\ixcom{APnumber} was set to \texttt{\theAPpartno}, 
the command \verb|\gvatcbp| prints:

\gvatcbp

\end{anexample}

\ssclause{General verdict criteria for preprocessor}

    The command \verb|\gvcprehead|\ixcom{gvcprehead} prints the
heading for the general verdict criteria for preprocessor cases subclause.

    The command 
\verb|\gvcprebp|\ixcom{gvcprebp} 
prints boilerplate for the subclause.

\begin{anexample} Remembering that in the preamble 
        \verb|\APnumber|\ixcom{APnumber} was set to \texttt{\theAPpartno}, 
the command \verb|\gvcprebp| prints:

\gvcprebp

\end{anexample}

\ssclause{General verdict criteria for postprocessor}


    The command \verb|\gvcposthead|\ixcom{gvcposthead} prints the
heading for the general verdict criteria for postprocessor cases subclause.

    The command 
\verb|\gvcpostbp|\ixcom{gvcpostbp} 
prints boilerplate for the subclause.

\begin{anexample} Remembering that in the preamble 
        \verb|\APnumber|\ixcom{APnumber} was set to \texttt{\theAPpartno}, 
the command \verb|\gvcpostbp| prints:

\gvcpostbp

\end{anexample}

\ssclause{Abstract test cases}

    The command \verb|\atchead|\ixcom{atchead} prints the heading
for the abstract test cases clause.

    The command 
\verb|\atcbp|\ixcom{atcbp} 
prints the first paragraph of the boilerplate for the clause.

\begin{example} The command \verb|\atcbp| prints:

\atcbp
\end{example}

    The command
\verb|\atcbpii|\ixcom{atcbpii}
prints paragraphs~3 and onwards of the boilerplate.

\begin{example} The command \verb|\atcbpii|
prints:

\atcbpii

\end{example}

\ssclause{Preprocessor}

    The command \verb|\prehead|\ixcom{prehead} prints the title
for the preprocessor subsubclause.

    The command
\verb|\atcpretpc|\ixcom{atcpretpc}
prints boilerplate for the subclause.

\begin{anexample} The command \verb|\atcpretpc| prints:

\atcpretpc
\end{anexample}

\ssclause{Postprocessor}

    The command \verb|\posthead|\ixcom{posthead} prints the title
for the postrocessor subsubclause.

    The command
\verb|\atcposttpc|\ixcom{atcposttpc}
prints boilerplate for the subclause.

\begin{anexample} The command \verb|\atcposttpc| prints:

\atcposttpc
\end{anexample}



\ssclause{Conformance class}

    The command \verb|\confclassannexhead|\ixcom{confclassannexhead}
prints the heading for the conformance classes annex heading.

    The command 
\verb|\atsnoclassesbp|\ixcom{atsnoclassesbp} 
prints the entire boilerplate for the
\textit{Conformance class} annex when the AP has no conformance classes.

\begin{example} Remembering that in the preamble 
        \verb|\APnumber|\ixcom{APnumber} was set to \texttt{\theAPpartno}, 
the command \verb|\atsnoclassesbp| prints:

\atsnoclassesbp

\end{example}

    The command \verb|\confclasshead{|\meta{number}\verb|}|\ixcom{confclasshead}
prints the heading for a conformance class \meta{number} subclause.

    The command 
\verb|\confclassbp{|\meta{number}\verb|}|\ixcom{confclassbp}
prints the
boilerplate for the introduction to a conformance class subclause, where
\meta{number} is the number of the conformance class.

\begin{example} Remembering that in the preamble 
        \verb|\APnumber|\ixcom{APnumber} was set to \texttt{\theAPpartno}, 
the command \verb|\confclassbp{27}| prints:

\confclassbp{\textit{27}}

\end{example}

\ssclause{Postprocessor input specification file names}

    The command \verb|\postipfilehead|\ixcom{postipfilehead} prints
the heading for the postprocessor input file names annex.

    The command 
\verb|\pisfbp{|\meta{12 or 21}\verb|}{|\meta{url}\verb|}{|\meta{ref}\verb|}|\ixcom{pisfbp} 
prints the boilerplate for the annex.

\begin{anexample} The command 
\verb|\pisfbp{12}{http://www.mel.nist.gov/step/parts/parts3456/wd}{\ref{TabB1}}| prints:

\pisfbp{12}{http://www.mel.nist.gov/step/parts/part3456/wd}{\ref{tabB1}}

\end{anexample}

%%%%%%%%%%%%%%%%%%%%%%%%%%%
%%%\end{document}
%%%%%%%%%%%%%%%%%%%%%%%%%%%

\normannex{Additional commands} \label{anx:extraiso}

\sclause{Internal commands}

    The code implementing the various facilities includes many commands
not described in the body of this document. Any command that includes
the commercial at sign (\verb|@|) in its name shall not be used by any author;
the implementer of the package code reserves the right to modify or delete
these at any time without giving any notice.

   Internal commands that have names consisting only of letters may be
used in a document at the author's own risk. These may be changed, but 
if so notification will be given.

\sclause{Boilerplate}

    Much of the boilerplate text is maintained in separate \file{.tex}
files and many of the commands that generate boilerplate merely 
\verb|\input|
the appropriate file.




%%%%%%%%%%%%%%%%%%%%%%%%%%
%%%\end{document}
%%%%%%%%%%%%%%%%%%%%%%%%%%

\normannex{Ordering of LaTeX commands} \label{anx:lord}

    The \latex{} commands to produce an ISO~10303 document are:
\begin{verbatim}
\documentclass[<options>]{isov2}
\usepackage{stepv13}                                % required package
\usepackage{irv12}                                  % for an IR document
\usepackage{apv12}                                  % for an AP document
\usepackage{aicv1}                                  % for an AIC document
\usepackage{atsv11}                                 % for an ATS document
\usepackage[<options>]{<name>}                      % additional packages
\standard{<standard identifier>}
\yearofedition{<year>}
\languageofedition{<parenthesized code letter>}
\partno{<part number>}
\series{<series title>}
\doctitle{<title on cover page>}
\ballotcycle{<number>}
\aptitle{<title of AP>}  % if doc is an AP
\aicinaptrue             % if doc is an AP that uses AICs
\mapspectrue             % if doc is an AP that uses mapping spec.
\APnumber{<number>}      % if doc is an ATS
\APtitle{<title>}        % if doc is an ATS
\mapspectrue             % if doc is an ATS and AP uses mapping spec.
  % other preamble commands
\begin{document}
\STEPcover{< title commands >}
\Foreword                            % start Foreword & ISO boilerplate
  \fwdshortlist                      % STEP boilerplate
\endForeword{<param1>}{<param2>}     % end Foreword & boilerplate
\begin{Introduction}                 % start Introduction & boilerplate
  \aicextraintro             % extra boilerplate for an AIC
  \apextraintro                % extra boilerplate for an AP
  % your text 
\end{Introduction}
\stepparttitle{<Part title>}
\scopeclause                         % Clause 1: Scope clause
  \apscope{<AP purpose>}             % boilerplate if an AP
   % text of scope
\normrefsclause                      % Clause 2: Normative references
  \normrefbp{<document type>}        % boilerplate
  \begin{nreferences}
    % \isref{}{} and/or \disref{}{} list of normative references
  \end{nreferences}
\defclause                           % definitions clause
  \partidefhead                      % defs from Part1 subclause
    % olddefinition list
  \refdefhead{<ISO 10303-NN>}        % defs from Part NN subclause
    % olddefinition list
  \otherdefhead                      % defs in this part
    % definition list
\symabbclause                        % Symbols & abbreviations clause
  % symbol lists
% THE BODY OF THE DOCUMENT
\bibannex                            % optional; the final Bibliography 
  % bibliography listing
% the index
\end{document}
\end{verbatim}


\sclause{Body of a resource document} \index{integrated resource}

    The body of a resource document has the following structure:

\begin{verbatim}
\schemahead{<Schema name>}         % repeat for each schema
  \introsubhead                    % intro subclause
     % text
  \fcandasubhead                   % concepts subclause
     % text
  \typehead{<Schema>}              % if type defs
     \atypehead{<type>}            % type heading     
  \entityhead{<Schema>}{<group>}   % if entity defs
     \anentityhead{<entity>}       % entity heading
  \rulehead{<Schema>}              % if rule defs
     \arulehead{<rule>}            % rule heading
  \functionhead{<Schema>}          % if function defs
     \afunctionhead{<function>}    % function heading
% repeat above for each schema
\shortnamehead                     % Annex A: Short names of entities
  \irshortnames                    % boilerplate
  % list of short names
\objreghead                        % Annex B: Information object registration
  \docidhead                       % Document identification subclause
    \docreg{<param1>}                                   % boilerplate
  \schemaidhead                    % Schema identification subclause
% Either (for single schema)
     \schemareg{<6 parameters>}    % boilerplate
% Or (for multiple schemas) repeat:
     \aschemaidhead{<schema name>} % Schema id subsubclause
       \schemareg{<6 parameters>}
\listingshead                       % Annex C: Computer interpretable listings
  \expurls{<short>}{<express>}      % boilerplate
\expressghead                       % Annex D: EXPRESS-G figures
  \irexpressg                       % boilerplate
  %  EXPRESS-G diagrams
\techdischead                       % optional Technical discussions
  % text
\exampleshead                       % optional Examples
  % text
\end{verbatim}


\sclause{Body of an application protocol} \index{AP}

    The body of an AP document has the following structure:

\begin{verbatim}
\inforeqhead                   % Clause 4: Information requirements
  \apinforeq{<param1>}         % boilerplate
  \uofhead                     % Clause 4.1: Units of functionality
    \begin{apuof}              % boilerplate
      % \item list of UoFs
    \end{apuof}
    \auofhead{<UoF1>}          % repeat for each UoF
      % text
    \applobjhead               % Clause 4.2: Application objects
      \apapplobj               % boilerplate
        % text
    \applasserthead            % Clause 4.3: Application assertions
      \apassert                % boilerplate
        % text
\aimhead                       % Clause 5: Application interpreted model
  \maptablehead                % Clause 5.1: Mapping table/specification
    \apmapping                 % boilerplate
    \maptemplatehead           % if mapping templates used
      \apmaptemplate           % template boilerplate
      \sstemplates             % sup/sub templates
      \templatehead
        % text
     \mapuofhead{<Uof>}        % mapping for <UoF>
       \mapobjecthead{<object>}
         % mapping for <object>
         \mapattributehead{<attr>}
           % mapping for <attr>
  \aimshortexphead             % Clause 5.2: AIM EXPRESS short listing
    \apshortexpress            % boilerplate
      % text
\confreqhead                   % Clause 6: Conformance requirements
  \apconformance{<param1>}     % boilerplate
  \begin{apconformclasses}     % optional boilerplate
    % \item list
  \end{apconformclasses}
     % text
\aimlongexphead                % Annex A: AIM EXPRESS expanded listing
  \aimlongexp                  % boilerplate
     % text
\aimshortnameshead             % Annex B: AIM short names
  \apshortnames                % boilerplate
     % text
\impreqhead                    % Annex C: Impl. specific reqs
  \apimpreq{<schema name>}     % boilerplate
\picshead                      % Annex D: PICS
  \picsannex                   % boilerplate
     % text
\objreghead                    % Annex E: Information object registration
  \docidhead                   % Annex E.1: Document identification
    \docreg{<param1>}          % boilerplate
  \schemaidhead                % Annex E.2: Schema identification
    \apschemareg{<6 params>}   % boilerplate
\aamhead                       % Annex F: Application activity model
  \aamfigrange{<figure range>} % Figure range for AAM diagrams
  \apaamintro                  % boilerplate
      % text
  \aamdefhead                  % Annex F.1: AAM defs and abbreviations
    \apaamdefs                 % boilerplate
       % text
  \aamfighead                  % Annex F.2: AAM diagrams
    \aamfigures                % boilerplate
       % IDEF0 diagrams
\armhead                       % Annex G: Application reference model
   \armintro                   % boilerplate
     % ARM figures
\aimexpressghead               % Annex H: AIM EXPRESS-G
  \aimexpressg                 % boilerplate
     % AIM figures
\listingshead                  % Annex J: Computer interpretable listings
  \apexpurls{<short>}{<express>}   % boilerplate
\apusagehead                   % optional Annex: AP usage
   % text
\techdischead                  % optional Annex: Technical discussions
   % text
\end{verbatim}

\sclause{Body of an AIC} \index{AIC}

    The body of an AIC document has the following structure:

\begin{verbatim}
\aicshortexphead               % Clause 4: EXPRESS short listing
  \aicshortexpintro            % boilerplate
  \fcandasubhead               % Clause 4.1 fundamental concepts
    % text
  \typehead{<Schema>}          % if type definitions
     \atypehead{<type>}        % repeat for each type
  \entityhead{<Schema>}{}      % if entity defs
     \anentityhead{<entity>}   % repeat for each entity
  \functionhead{<Schema>}      % if function defs
     \afunctionhead{<function>} % repeat for each function
\shortnamehead                 % Annex A: Short names of entities
  \shortnames                  % boilerplate
\objreghead                    % Annex B: Information object registration
  \docidhead                   % Annex B.1: Document identification
    \docreg{<version no>}      % boilerplate
  \schemaidhead                % Annex B.2: Schema identification
    \schemareg{<6 parameters>} % boilerplate
\expressghead               % Annex C: EXPRESS-G diagrams
  \aicexpressg               % boilerplate
\listingshead                  % Annex D: Computer interpretable listings
  \expurls                     % boilerplate
\techdischead                  % optional Annex: Technical discussions
\end{verbatim}

\sclause{Body of an ATS document}\index{ATS}

    The body of an Abstract Test Suite
document has the following structure:

\begin{verbatim}
\purposeshead                  % Clause 4: Test purposes
  \atspurposebp    % boilerplate
  \aepurposehead               % 4.1 Application element test purposes
    \aetpbp                    % boilerplate
    \apobjhead{<object>}       % 4.1.n
    ...
  \aimpurposehead              % 4.2 AIM test purposes
    \aimtpbp                   % boilerplate
    \aimenthead{<entity>}      % 4.2.n
    ...
  \implementpurposehead        % (optional) 4.3 Implementation t.p
    \atsimtpbp                 % boilerplate
    % text
  \domainpurposehead           % (optional) 4.2+ Domain test purposes
    % text
  \otherpurposehead            % (optional) 4.2+ Other test purposes
    % text
\gtpvchead                     % Clause 5: General t.p and verdict criteria
  \atsgtpvcbp                  % boilerplate
  \generalpurposehead          % 5.1 General test purposes
    \gtpbp                     % boilerplate
    ...
  \gvcatchead                  % 5.2 General verdict criteria for all ATC
    \gvatcbp                   % boilerplate
    ...
  \gvcprehead                  % 5.3 General verdict criteria for preprocessor
    \gvcprebp                  % boilerplate
     ...
  \gvcposthead                 % 5.4 General verdict criteria for postprocessor
    \gvcpostbp                 % boilerplate
    ...
\atchead                       % Clause 6: Abstract test cases
  \atcbp                       % boilerplate (para 1)
  % your para 2
  \atcbpii                     % boilerplate (paras 3+)
  \atctitlehead{<title>}       % 6.n an abstract test case
    \prehead                   % 6.n.1 Preprocessor
      \precoveredhead..        % Test purposes covered
        \atcpretpc             % boilerplate
      \preinputhead            % Input specification
        % text
      \precriteriahead         % Verdict criteria
        % text
      \preconstrainthead       % Constraints on values
        % text
      \preexechead             % (optional) Execution sequence
        % text
      \preextrahead            % (optional) Extra details
        % text
    \posthead                  % 6.n.2 Postprocessor
      \postcoveredhead         % Test purposes covered
        % text
        \atcposttpc            % boilerplate
      \postinputhead           % Input specification
        % text
      \postcriteriahead        % Verdict criteria
        % text
      \postexechead            % (optional) Execution sequence
        % text
      \postextra               % (optional) Extra details
        % text
\confclassannexhead            % Annex A: Conformance classes
  \atsnoclassesbp              % boilerplate if no conformance classes, else
  \confclasshead{<number>}     % A.n Conformance class <number>
    \confclassbp{<number>}     % boilerplate
    % text
  ...
\postipfilehead                % Annex B: Postprocessor input file names
  \pisfbp{..}{..}{..}                      % boilerplate
   ...
\objreghead                    % Annex C: Information object registration
  \docreg{<partno>}            % registration boilerplate
\atsusagehead                  % Annex D: Usage scenarios
  % text
\end{verbatim}



%%%%%%%%%%%%%%%%%%%%%%%%%%%%%%%%%%%%%%%
% object registration annex
\objreghead

\docreg{-1}

%%%%%%%%%%%%%%%%%%%%%%%%%%%%%%%%%%%%%%

%%%%%%%%%%%%%%%%%%%%%%%%%
%%%\end{document}
%%%%%%%%%%%%%%%%%%%%%%%%%

\infannex{Deprecated, deleted, new and modified commands}

    This release has involved many internal changes to the \latex{}
\file{.sty} files. In particular boilerplate text is, as far as possible,
maintained in external \file{.tex} files in order to save memory
space within the \latex{} processor. 


%%%%%%%%%%%%%%%%%%%%%%%%%
%%%\end{document}
%%%%%%%%%%%%%%%%%%%%%%%%%

\sclause{New commands}

    The commands that are new in this release are:

\begin{itemize}
%%%%%%%%%%%%%%%%%%%%%%%%%% STEP %%%%%%%%%%%%%%%%%%%%%%%%%%%
\item \verb|\bibieeeidefo|\ixcom{bibieeeidefo} STEP: reference to IDEF0 document;
\item \verb|\exampleshead|\ixcom{exampleshead} STEP: clause heading;
\item \verb|\expressgdef|\ixcom{expressgdef} STEP: location of \ExpressG{} definition;

\item \verb|\Theseries|\ixcom{Theseries} STEP: print \verb|\series| argument;
\item \verb|\theseries|\ixcom{theseries} STEP: print \verb|\series| argument in lowercase;
\item \verb|\ifanir|,\ixcom{ifanir} 
      \verb|\anirtrue|,\ixcom{anirtrue}
      \verb|\anirfalse|\ixcom{anirfalse} STEP: flag for an IR document;
\item \verb|\ifhaspatents|,\ixcom{ifhaspatents} 
      \verb|\haspatentstrue|,\ixcom{haspatentstrue} 
      \verb|\haspatentsfalse|\ixcom{haspatentsfalse} STEP: flag for known patents;
\item \verb|\ifmapspec|,\ixcom{ifmapspec} 
      \verb|\mapspectrue|,\ixcom{mapspectrue} 
      \verb|\mapspecfalse|\ixcom{mapspecfalse} STEP: flag for mapping specification;

\item \verb|\ixent|\ixcom{ixent} STEP: index an \Express{} \xword{entity};
\item \verb|\ixenum|\ixcom{ixenum} STEP: index an \Express{}  \xword{enumeration};
\item \verb|\ixfun|\ixcom{ixfun} STEP: index an \Express{}  \xword{function};
\item \verb|\ixproc|\ixcom{ixproc} STEP: index an \Express{}  \xword{procedure};
\item \verb|\ixrule|\ixcom{ixrule} STEP: index an \Express{}  \xword{rule};
\item \verb|\ixsc|\ixcom{ixsc} STEP: index an \Express{}  \xword{subtype\_constraint};
\item \verb|\ixschema|\ixcom{ixschema} STEP: index an \Express{}  \xword{schema};
\item \verb|\ixselect|\ixcom{ixselect} STEP: index an \Express{}  \xword{select};
\item \verb|\ixtype|\ixcom{ixtype} STEP: index an \Express{}  \xword{type};

\item \verb|\maptableorspec|\ixcom{maptableorspec} STEP: prints `table' or `specification';

\item \verb|\xword|\ixcom{xword} STEP: prints an \Express{} keyword;

%%%%%%%%%%%%%%%%%%%%%%%%%%%%%%%% AP %%%%%%%%%%%%%%%%%%%%%%%%%%%%%%%


\item \verb|\apmaptemplate|\ixcom{apmaptemplate} AP: boilerplate;
\item \verb|\apusagehead|\ixcom{apusagehead} AP: clause heading;
\item \verb|\ifidefix|,\ixcom{ifidefix} 
      \verb|\idefixtrue|,\ixcom{idefixtrue} 
      \verb|\idefixfalse|\ixcom{idefixfalse} AP: flag for an IDEF1X ARM;
\item \verb|\ifmaptemplate|,\ixcom{ifmaptemplate} 
      \verb|\maptemplatetrue|,\ixcom{maptemplatetrue} 
      \verb|\maptemplatefalse|\ixcom{maptemplatefalse} AP: flag for 
            using mapping templates;
\item \verb|\mapattributehead|\ixcom{mapattributehead} AP: clause heading;
\item \verb|\mapobjecthead|\ixcom{mapobjecthead} AP: clause heading;
\item \verb|\mapuofhead|\ixcom{mapuofhead} AP: clause heading;
\item \verb|\sstemplates|\ixcom{sstemplates} AP: boilerplate;
\item \verb|\templateshead|\ixcom{templateshead} AP: clause heading;
%%% \item \verb|\apmappingspec|\ixcom{} internal (not used?) 

%%%%%%%%%%%%%%%%%%%%%%%%%%%%%%%% ATS %%%%%%%%%%%%%%%%%%%%%%%%%%%%%

\item \verb|\atcposttpc|\ixcom{atcposttpc} ATS: boilerplate;
\item \verb|\atcpretpc|\ixcom{atcpretpc} ATS: boilerplate;
\item \verb|\atsimtpbp|\ixcom{atsimtpbp} ATS: boilerplate;
\item \verb|\atsusagehead|\ixcom{atsusagehead} ATS: clause heading.


\end{itemize}





\sclause{Modified commands}

    The commands that have been modified in this release are:

\begin{itemize}

%%%%%%%%%%%%%%%%%%%%%%%%%%%% STEP %%%%%%%%%%%%%%%%%%%%%%%%%%%%%

\item STEP: The \verb|\Introduction|\ixcom{Introduction} command is
      now the \verb|Introduction|\ixenv{Introduction} environment,
      with no argument;

%%%%%%%%%%%%%%%%%%%%%%%%%%%%%% IR %%%%%%%%%%%%%%%%%%%%%%%%%%%%%%%

\item \verb|\irexpressg|\ixcom{irexpressg} IR: takes no argument;

%%%%%%%%%%%%%%%%%%%%%%%%%%%%%% AP %%%%%%%%%%%%%%%%%%%%%%%%%%%%%%%

\item \verb|\aimexpressg|\ixcom{aimexpressg} AP: takes no argument;

%%%%%%%%%%%%%%%%%%%%%%%%%%%%%% AIC %%%%%%%%%%%%%%%%%%%%%%%%%%%%%%%

\item \verb|\aicexpressg|\ixcom{aicexpressg} AIC: takes no argument;


%%%%%%%%%%%%%%%%%%%%%%%%%%%%%% ATS %%%%%%%%%%%%%%%%%%%%%%%%%%%%%%%

\item \verb|\atcbpii|\ixcom{atcbpii} ATS:  takes no argument;
\item \verb|\atspurposebp|\ixcom{atspurposebp} ATS: takes no argument;
\item \verb|\pisfbp|\ixcom{pisfbp} ATS: takes 3 arguments.


\end{itemize}





\sclause{Deleted commands}

    The commands that have been deleted in this release are:

\begin{itemize}

%%%%%%%%%%%%%%%%%%%%%%%%%%%%%% STEP %%%%%%%%%%%%%%%%%%%%%%%%%%%%%

\item \verb|\fwddivlist|\ixcom{fwddivlist} STEP: used in Foreword;
\item \verb|\fwdpartslist|\ixcom{fwdpartslist} STEP: used in Foreword;

\item \verb|\introend|\ixcom{introend} STEP: was for use at the end of the
      Introduction;

%%%%%%%%%%%%%%%%%%%%%%%%%%%%%% IR %%%%%%%%%%%%%%%%%%%%%%%%%%%%%

\item \verb|\irschemaintro|\ixcom{irschemaintro} IR:
      has been replaced by
      \verb|\schemaintro|\ixcom{schemaintro};

%%%%%%%%%%%%%%%%%%%%%%%%%%%%%% AP %%%%%%%%%%%%%%%%%%%%%%%%%%%%%

\item \verb|\apintroend|\ixcom{apintroend} AP:
      has been replaced by
      \verb|\apextraintro|\ixcom{apextraintro};

\item \verb|\apschemareg|\ixcom{apschemareg} AP: use 
           \verb|\schemareg|\ixcom{schemareg} instead;

\item \verb|\apmappingtable|\ixcom{apmappingtable} AP:
      has been replaced by
      \verb|\apmapping|\ixcom{apmapping};

\item \verb|\armfigures|\ixcom{armfigures} AP:
      has been replaced by
      \verb|\armintro|\ixcom{armintro};

\item \verb|\maptablehead|\ixcom{maptablehead} AP:
      has been replaced by
      \verb|\mappinghead|\ixcom{mappinghead};

\item \verb|\modelscopehead|\ixcom{modelscopehead} AP: was the heading
      for a `Model scope' annex;

%%%%%%%%%%%%%%%%%%%%%%%%%%%%%% AIC %%%%%%%%%%%%%%%%%%%%%%%%%%%%%

\item \verb|\aicexpressghead|\ixcom{aicexpressghead} AIC: 
      use \verb|\expressghead|\ixcom{expressghead} instead;
\item \verb|\aicshortnames|\ixcom{aicshortnames} AIC: use
      \verb|\expurls|\ixcom{expurls} instead;
\item \verb|\aicshortnameshead|\ixcom{aicshortnameshead} AIC: 
      use \verb|\shortnamehead|\ixcom{shortnamehead} instead;

%%%%%%%%%%%%%%%%%%%%%%%%%%%%%% ATS %%%%%%%%%%%%%%%%%%%%%%%%%%%%%


\item \verb|\excludepurposehead|\ixcom{excludepurposehead} ATS: 
      was the heading for an `Exclude purposes' clause.

\end{itemize}


%%%%%%%%%%%%%%%%%%%%%%%
%%%\end{document}
%%%%%%%%%%%%%%%%%%%%%%%


% sgmlannx.tex    latex and SGML

\infannex{LaTeX, the Web, and *ML} \label{anx:sgml} \index{SGML}

    ISO are becoming more interested in electronic sources for their
standards as well as the traditional camera-ready copy. Acronyms like
PDF, HTML, SGML and XML have been bandied about. Fortunately documents
written using \latex{} are well placed to be provided in a variety of 
electronic formats. A comprehensive treatment of \latex{} with respect
to this topic is provided by Goossens and Rahtz~\bref{lwebcom}.

    SGML (Standard Generalized Markup Language) is a document tagging 
language that is described in ISO~8879~\bref{sgml} and whose usage is described 
in~\bref{bryan}, among others. The principal
mover behind SGML is Charles Goldfarb from IBM, who has authored a detailed 
handbook~\bref{goldfarb} on the SGML standard.

    The concepts lying behind both \latex{} and SGML are similar, but on the face
of it they are distinctly different in both syntax and capabilities. ISO is
migrating towards electronic versions of its standard documents and, naturally, 
would prefer these to be SGML tagged. 
     Like \latex, SGML has a
concept of style files, which are termed DTDs, and both systems support
powerful macro-like capabilities. SGML provides for logical document
markup and not typesetting --- commercial SGML systems often use
\tex{} or \latex{} as their printing engine, as does the NIST SGML
environment for ISO~10303~\bref{pandl}.



NIST have SGML tagged some STEP documents 
using manual methods, which are time consuming and expensive. 
In about 1997 there was a NIST 
effort underway to develop an auto-tagger that would (semi-) automatically 
convert
a \latex{} tagged document to one with SGML tags. This tool assumed a
fixed set of \latex{} macros and a fixed DTD.
 The design of an auto-tagger
essentially boils down to being able to convert from a source document tagged
according to a \latex{} style file to one which is tagged according to an
SGML DTD.
    Fully automatic conversion is really only possible if the authors'
of the documents to be translated avoid using any `non-standard' macros within
their documents. There is a program called \file{ltx2x}\index{ltx2x} available
from SOLIS, which replaces \latex{} commands within a document with
user-defined text strings~\bref{ltx2x}. This can be used as a basis for
a \latex{} to whatever auto-tagger, provided the \latex{} commands are not
too exotic.

    HTML is a simple markup language, based on SGML, and is used for the
publication of many documents on the Web. XML is a subset of SGML and appears
to being taken up by every man and his dog as \emph{the} document markup
language. HTML is being recast in terms of XML instead of SGML. PDF is a page
description language that is a popular format for display of documents 
on the Web.

    \latex{} documents can be output in PDF by using pdfLaTeX. Instead
of a \file{.dvi} file being produced a \file{.pdf} file is output directly.
The best 
results are obtained when PostScript fonts rather than Knuth's cm fonts 
are used. Noting that the \file{isov2} class provides an \verb|\ifpdf| command,
a general form for documents to be processed by either \latex{} or pdfLaTeX
is
\begin{verbatim}
\documentclass{isov2}
\usepackage{times}     % PostScript fonts Times, Courier, Helvetica
\ifpdf
  \pdfoutput=1         % request PDF output
  \usepackage[pdftex]{graphicx}
\else
  \usepackage{graphicx}
\fi
...
\end{verbatim}

    There are several converters available to transform a \latex{} document 
into an HTML document, but like \file{ltx2x} they generally do their own
parsing of the source file, and unlike \file{ltx2x} are typically limited
to only generating HTML. Eitan Gurari's \file{TeX4ht}\index{TeX4ht} 
suite is a notable
exception (see Chapter~4 and Appendix~B of~\bref{lwebcom}). It uses the 
\file{.dvi} file as input, so that all the parsing is done by \tex, and can be
configured to generate a wide variety of output formats.
A set of \file{TeX4ht} configuration files are available for converting
STEP \latex{} documents into HTML\footnote{Later, configuration files for XML
output will be developed.}.

    It is highly recommended that for the purposes of ISO~10303, document editors
refrain from defining their own \latex{} macros. If new generally applicable
\latex{} commands are found to be necessary, these should be sent to the
editor of this document for incorporation into
the \file{isov2}\ixclass{isov2} class, the \file{step}\ixpack{step}
package and/or appropriate other packages.

    Some other points to watch when writing \latex{} documents that will assist
in translations into *ML are given below. Typically, attention to these points
will make it easier to parse the \latex{} source.

\begin{itemize}
\item Avoid using the \verb|\label|\ixcom{label} command within
      clause headings or captions. It can just as easily be placed immediately
      after these constructs.
\item Avoid using the \verb|\index|\ixcom{index} command within
      clause headings or captions. It can just as easily be placed immediately
      after these constructs.
\item Use all the specified tagging constructs when defining an \Express{} 
      model --- this will also assist any program that attempts to extract
      \Express{} source code and descriptive text from a document.
\end{itemize}



\infannex{Obtaining LaTeX and friends} \label{anx:getstuff}

    \latex{} is a freely available document typesetting system. There are many
public domain additions to the basic system, for example the \file{iso.cls}
and \file{step.sty} styles. The information below gives pointers to where
you can obtain \latex{} etc., from the\index{Internet} Internet. 


    \latex{} runs on a wide variety of hardware, from PCs to Crays. Source to build
a \latex{} system is freely available via anonymous ftp\index{ftp} 
from what is called CTAN\index{CTAN}
(Comprehensive \tex\ Archive Network). 
There are three sites; pick the one nearest to you.
\begin{itemize}
\item \url{ftp.dante.de} CTAN in Germany;
\item \url{ftp.tex.ac.uk} CTAN in the UK;
\item \url{ctan.tug.org} CTAN in the USA;
\end{itemize}
The top level CTAN directory 
for \latex{} and friends is \url{/tex-archive}. CTAN contains a wide variety
of (La)TeX sources, style files, and software tools and scripts to assist in
document processing.

\begin{anote}
CTAN is maintained by the \tex{} Users Group (TUG). Their homepage
\isourl{http://www.tug.org} should be consulted for the current list of CTAN sites and mirrors.
\end{anote}

\begin{comment}

\sclause{SOLIS} \index{SOLIS}

    SOLIS is the \textit{SC4 On Line Information Service}. It contains many electronic
sources of STEP related documents. The relevant top level directory is
\url{pub/subject/sc4}.
 In particular, SOLIS contains the source for this document
and the \file{.sty} files, as well as other \latex{} related files. 
The \latex{} root directory is \url{sc4/editing/latex}.
The latest versions of the \latex{}
related files are kept in the sub-directory \url{latex/current}.
Some \latex{} related programs are also available in the 
\url{latex/programs} sub-directory.

    There are several ways of accessing SOLIS; instructions
are detailed by Ressler~\bref{ressler} and Rinaudot~\bref{rinaudot}. 
Copies of these reports may be obtained by telephoning the
IPO Office at \verb|+1 (301) 975-3983|, although they are probably somewhat
dated by now.
The simplest method is to point your browser at the following URL: \\
\isourl{http://www.nist.gov/sc4}

\end{comment}

\bibannex
\label{biblio}

\begin{references}
\reference{LAMPORT, L.,}{LaTeX --- A Document Preparation System,}
            {Addison-Wesley Publishing Co., 2nd edition, 1994} \label{lamport}
\reference{WILSON, P. R.,}{LaTeX for standards: The LaTeX package files
            user manual,}%
           {NISTIR, 
                   National Institute of Standards and Technology,
           Gaithersburg, MD 20899. June 1996.} \label{doc:isorot}
\reference{GOOSENS, M., MITTELBACH, F. and SAMARIN, A.,}{%
           The LaTeX Companion,}
           {Addison-Wesley Publishing Co., 1994} \label{goosens}
\reference{GOOSENS, M. and RAHTZ, S.,}{%
           The LaTeX Web Companion --- Integrating TeX, HTML, and XML,}
           {Addison-Wesley Publishing Co., 1999} \label{lwebcom}
\reference{CHEN, P. and HARRISON, M.A.,}{Index preparation and
           processing,}{Software--Practice and Experience, 19(9):897--915,
           September 1988.} \label{chen}
%\reference{KOPKA, H. and DALY, P.W.,}{A Guide to LaTeX,}
%           {Addison-Wesley Publishing Co., 1993.} \label{kopka}
\reference{ISO 8879:1986,}{Information processing --- 
                                Text and office systems ---
           Standard Generalized Markup Language (SGML)}{} \label{sgml}
\reference{GOLDFARB, C.F.,}{The SGML Handbook,}
           {Oxford University Press, 1990.} \label{goldfarb}
\reference{BRYAN, M.,}{SGML --- An Author's Guide to the Standard Generalized
           Markup Language,}{Addison-Wesley Publishing Co., 1988. }\label{bryan}
\reference{PHILLIPS, L., and LUBELL, J.,}{An SGML Environment for STEP,}%
           {NISTIR 5515, 
                   National Institute of Standards and Technology,
           Gaithersburg, MD 20899. November 1994.} \label{pandl}
\reference{WILSON, P. R.,}{LTX2X: A LaTeX to X Auto-tagger,}%
           {NISTIR, 
                   National Institute of Standards and Technology,
           Gaithersburg, MD 20899. June 1996.} \label{ltx2x}
\bibidefo
\bibieeeidefix
\begin{comment}
\reference{RESSLER, S.,}{The National PDES Testbed Mail Server User's Guide,}
           {NSTIR 4508, National Institute of Standards and Technology,
           Gaithersburg, MD 20899. January 1991.} \label{ressler}
\reference{RINAUDOT, G. R.,}{STEP On Line Information Service (SOLIS),}
          {NISTIR 5511, National Institute of Standards and Technology,
          Gaithersburg, MD 20899. October 1994. } \label{rinaudot}
\end{comment}
\end{references}

    
%  the INDEX
\input{stepman.ind}



\end{document}





\end{document}





\end{document}





\end{document}

