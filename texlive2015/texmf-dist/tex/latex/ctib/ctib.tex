%%%%%%%%%%%%%%%%%%%%%%%%%%%%%%%%%%%%%%%%%%%%%%%%%%%%%%%%%%%%%%%%
%        File: ctib.tex
%      Author: Sam Sirlin
% Modified by: Oliver Corff
%        Date: October 10th, 1999
%     Version: \CtibVersionRelease
%
% Description: The ctib.tex file provides access to all
%		commands necessary for writing Tibetan
%		in TeX documents, making use of a modified
%		version of Sirlin's Tibetan fonts.
%
%		Many definitions were actually taken from
%		Sirlin's tibdef.tex (and occasionally modified).
%
%%%%%%%%%%%%%%%%%%%%%%%%%%%%%%%%%%%%%%%%%%%%%%%%%%%%%%%%%%%%%%%%
% -------------------       main code        -------------------
%
% \TibTeX creates the TibTeX logo
%
\def\TibTeX{{cTib\TeX}}
%
%
% s.w. sirlin  11/24/94, modified by o. corff, 1999-10-10
% definitions for tibetan fonts

\font\tibetanzero=ctib scaled\magstep0
\font\tenrmzero=cmr10 scaled\magstep0

\font\tibetanone=ctib scaled\magstep1
\font\tenrmone=cmr10 scaled\magstep1

\def\tsize{%
\def\tibetan{\tibetanzero}%
\def\tenrm{\tenrmzero}}

\def\tsizei{%
\def\tibetan{\tibetanone}%
\def\tenrm{\tenrmone}}

%
%
%%%%%%%%%%%%%%%%%%%%%%%%%%%%%%%%%%%%%%%%%%%%%%%%%%%%%%%%%%%%%%%%
%
% Now Tibetan paragraph properties
%
%\hyphenpenalty=10000\parindent=0pt

\def\tspace{{\hskip\fontdimen7\font}}	% inter-sentence space
\def\notsheg{\noboundary}		% elimininates tsheg
\def\K{\noboundary\tspace}		% Space after k, g

%
% Helps build additional stacks
%
\def\V#1#2{\noboundary%
        \ooalign{\noboundary#1\noboundary\cr
	\lower1.15\fontdimen5\font\hbox{#2\noboundary}}%
        x} % `x' which has no value in ctib generates automatic tsheg

%%%%%%%%%%%%%%%%%%%%%%%%%%%%%%%%%%%%%%%%%%%%%%%%%%%%%%%%%%%%%%%%
%
% Now typical Tibetan identities
%
% Bad: would not cooperate with MonTeX!
\ifx\om\undefined
	\def\om{{\tib\char254}}		% `<'
\fi
\def\visarga{{\tib\char58}}	% `:' 
\def\swasti{{\tib\char64}}	% `@'
%
% special characters
%
\def\dme{\V{de}{ma}x}
%%  
\def\ai{\V{a}{\raise0.85\fontdimen5\font\hbox{\hskip0.25em`}}x}
%%  
\def\hung{\ooalign{{\raise0.7\fontdimen5\font\hbox{\hskip0.15em0}}
		\cr\V{h}{'u}}x}
%%  
\def\hrih{\V{hri}{\raise0.50\fontdimen5\font\hbox{`}}x}
%%  
\def\achung#1{\vtop{%
	\baselineskip0pt\hbox{#1\noboundary}%
	\hbox{\lower0.15\fontdimen5\font\hbox{`}}}
}
%%  
%%  
%%  % operators - new vowels
%%  
%%  % m as o above next letter
%%  %\def \altm{{\tibsp\accent0}}
%%  %\def \altm{ \tibsp\accent0\tibetan}
%%  % expect that tfilt will add junk which we need to junk
%%  \def \altm#1{ \tibsp\accent0\tibetan #1}
%%  
%%  % m as o above, u below
%%  \def\altmu#1{\vtop{\baselineskip0pt\hbox
%%  {\tibsp\accent0\tibetan #1}\hbox{\tibsp\char123}}}
%%  
%%  % m as o below - emphasis
%%  % \def\subm#1{\vtop{\baselineskip0pt\hbox{#1}\hbox{\tibsp\char0}}}
%%  \def\subm#1{
%%  \vtop{\ialign{\hfil##\hfil\cr
%%  \hbox{#1}\cr
%%  \noalign{\kern1pt\nointerlineskip}
%%  \hbox{\tibsp\char0}\cr}}
%%  }
%%  
%%  % below
%%  
%ready to start, default size is 0

\tsize

% end of ctib.tex
