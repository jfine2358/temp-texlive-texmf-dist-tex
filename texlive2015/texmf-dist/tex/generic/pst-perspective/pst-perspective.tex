%% COPYRIGHT 2014 by Thomas Soell
%
% This work may be distributed and/or modified under the
% conditions of the LaTeX Project Public License, either version 1.3
% of this license or (at your option) any later version.
% The latest version of this license is in
%   http://www.latex-project.org/lppl.txt
% and version 1.3 or later is part of all distributions of LaTeX
% version 2003/12/01 or later.
%
% This work consists of the file pstricks.tex, PostScript macros
% for Generic TeX.
% See the PSTricks User's Guide for description.
%
\csname PSTperspectiveLoaded\endcsname
\let\PSTperspectiveLoaded\endinput
\ifx\PSTGradLoaded\endinput \else\input pst-grad \fi
\ifx\PSTricksAddLoaded\endinput\else\input pstricks-add \fi
%
\def\fileversion{1.04}
\def\filedate{2014/04/18}
\message{ v\fileversion, \filedate (ts)}
\message{ v\fileversion, \filedate}
\edef\TheAtCode{\the\catcode`\@}
\catcode`\@=11
\pst@addfams{pst-perspective}
\define@boolkey[psset]{pst-perspective}[Pst@]{translineA}[false]{} % showtransline
\define@boolkey[psset]{pst-perspective}[Pst@]{translineB}[false]{} % showtransline
\define@boolkey[psset]{pst-perspective}[Pst@]{translineK}[true]{} % show transline
\define@boolkey[psset]{pst-perspective}[Pst@]{hideline}[false]{} % show hideline
\define@boolkey[psset]{pst-perspective}[Pst@]{differentcol}[false]{} % unterschiedliche Farben der Fl\"{a}chen
\define@boolkey[psset]{pst-perspective}[Pst@]{LowPoint}[false]{} % Basis durch z.B. tief liegenden Pkt
\define@boolkey[psset]{pst-perspective}[Pst@]{symX}[true]{} % Symmetrieachse bei spiegelverkehrten Bildern
\define@key[psset]{pst-perspective}{topfillstyle}[solid]{\def\psk@transTS@topfillstyle{#1}}
\define@key[psset]{pst-perspective}{topangle}[45]{\def\psk@transTS@topangle{#1}}
\define@key[psset]{pst-perspective}{topmidpoint}[0.5]{\def\psk@transTS@topmidpoint{#1}}
\define@key[psset]{pst-perspective}{toplinecolor}[black]{\def\psk@transTS@toplinecolor{#1}}
\define@key[psset]{pst-perspective}{topfillcolor}[gray]{\def\psk@transTS@topfillcolor{#1}}
\define@key[psset]{pst-perspective}{toplinewidth}[0.8pt]{\def\psk@transTS@toplinewidth{#1}}
\define@key[psset]{pst-perspective}{hidecolor}[black!80]{\def\psk@transTS@hidecolor{#1}}
\define@key[psset]{pst-perspective}{facecolorT}[yellow]{\def\psk@transTS@facecolorT{#1}}
\define@key[psset]{pst-perspective}{facecolorR}[red]{\def\psk@transTS@facecolorR{#1}}
\define@key[psset]{pst-perspective}{hidelinestyle}[solid]{\def\psk@transTS@hidelinestyle{#1}}
\define@key[psset]{pst-perspective}{hidelinewidth}[0.7pt]{\def\psk@transTS@hidelinewidth{#1}}
\define@key[psset]{pst-perspective}{transcolor}[blue]{\def\psk@transTS@transcolor{#1}}
\define@key[psset]{pst-perspective}{transAcolor}[green]{\def\psk@transTS@transAcolor{#1}}
\define@key[psset]{pst-perspective}{transBcolor}[blue]{\def\psk@transTS@transBcolor{#1}}
\define@key[psset]{pst-perspective}{translinestyle}[solid]{\def\psk@transTS@translinestyle{#1}}
\define@key[psset]{pst-perspective}{transAlinestyle}[dashed]{\def\psk@transTS@transAlinestyle{#1}}
\define@key[psset]{pst-perspective}{transBlinestyle}[dashed]{\def\psk@transTS@transBlinestyle{#1}}
\define@key[psset]{pst-perspective}{translinewidth}[0.8pt]{\def\psk@transTS@translinewidth{#1}}
\define@key[psset]{pst-perspective}{transAlinewidth}[0.7pt]{\def\psk@transTS@transAlinewidth{#1}}
\define@key[psset]{pst-perspective}{transBlinewidth}[0.7pt]{\def\psk@transTS@transBlinewidth{#1}}
\define@key[psset]{pst-perspective}{LowP}[{0|0}]{\def\psk@transTS@LowP{#1 }}
\define@key[psset]{pst-perspective}{originT}[{0|0}]{\def\psk@transTS@originT{#1 }}
\define@key[psset]{pst-perspective}{base}[0]{\def\psk@transTS@base{#1 }} % y-Wert der Basis f\"{u}r die Transformation
\define@key[psset]{pst-perspective}{symline}[0]{\def\psk@transTS@symline{#1 }} % Symmetrieachse
\define@key[psset]{pst-perspective}{phi}[45]{\def\psk@transTS@phi{#1 }}% Winkel f\"{u}r das Schr\"{a}gbild
\define@key[psset]{pst-perspective}{deltaphi}[0]{\def\psk@transTS@deltaphi{#1 }}% zus\"{a}tzlicher Winkel f\"{u}r das Schr\"{a}gbild
\define@key[psset]{pst-perspective}{vkf}[0.5]{\def\psk@transTS@vkf{#1 }} % Verk\"{u}rzungsfaktor
\psset[pst-perspective]{phi=45,vkf=0.5,base=0,translineA=false,translineB=false,transAcolor=green,transBcolor=blue,differentcol=false,%
transAlinewidth=0.7pt,transBlinewidth=0.7pt,transAlinestyle=dashed,transBlinestyle=dashed,deltaphi=0,facecolorT=yellow,facecolorR=red,%
symline=0,symX=true,hidecolor=black!80,translineK=true,transcolor=blue,translinewidth=0.8pt,translinestyle=solid,%
LowPoint=false,LowP={0,0},originT={0,0},hideline=false,hidelinewidth=0.7pt,hidelinestyle=dashed,%
topfillstyle=solid,topangle=45,topmidpoint=0.5,toplinecolor=black,toplinewidth=0.8pt,topfillcolor=gray%
}%
\SpecialCoor%
\def\ba{\psk@transTS@base}
\def\pstransTS{\pst@object{pstpstransTS}}%
%
\def\pstpstransTS@i(#1)#2#3{%
\begingroup%
% (#1) Punkt 1
% #2 Node-Name des Startpunktes % #3 Node-Name des transformierten Punktes
\pst@killglue
\use@par
\rput(\psk@transTS@originT){\pnode(#1){#2}}%
\pst@getcoor{#1}\pst@tempA%
\pst@getcoor{\psk@transTS@LowP}\pst@tempB
\rput(\psk@transTS@originT){\pnode(! %
\pst@tempA \tx@UserCoor /y1 ED /x1 ED
\ifPst@LowPoint
\pst@tempB \tx@UserCoor /y2 ED /x2 ED
\else
/y2 \psk@transTS@base def \fi
/lang y2 y1 sub abs \psk@transTS@vkf mul def
/lang2 y2 y1 sub def
lang2 0 ge {/alph2 \psk@transTS@phi \psk@transTS@deltaphi add 180 sub def} {/alph2 \psk@transTS@phi \psk@transTS@deltaphi add def} ifelse
lang alph2 PtoC exch x1 add exch y2 add){#3}
\ifPst@translineA
\ifPst@LowPoint
\pcline[linecolor=\psk@transTS@transAcolor,linestyle=\psk@transTS@transAlinestyle,linewidth=\psk@transTS@transAlinewidth](! \pst@tempA \tx@UserCoor pop \pst@tempB \tx@UserCoor exch pop)(#2)
\else
\pcline[linecolor=\psk@transTS@transAcolor,linestyle=\psk@transTS@transAlinestyle,linewidth=\psk@transTS@transAlinewidth](! \pst@tempA \tx@UserCoor pop \psk@transTS@base)(#2)
\fi
\else \fi%
\ifPst@translineB
\ifPst@LowPoint
\pcline[linecolor=\psk@transTS@transBcolor,linestyle=\psk@transTS@transBlinestyle,linewidth=\psk@transTS@transBlinewidth](! \pst@tempA \tx@UserCoor pop \pst@tempB \tx@UserCoor exch pop)(#3)
\else
\pcline[linecolor=\psk@transTS@transBcolor,linestyle=\psk@transTS@transBlinestyle,linewidth=\psk@transTS@transBlinewidth](! \pst@tempA \tx@UserCoor pop \psk@transTS@base)(#3)
\fi
\else \fi%
}%
\endgroup
\ignorespaces
}
\def\pstransTSX{\pst@object{pstpstransTSX}}%
%
\def\pstpstransTSX@i(#1)#2#3{%
\begingroup%
% (#1) Punkt 1
% #2 Node-Name des Startpunktes % #3 Node-Name des transformierten Punktes
\pst@killglue
\use@par
\rput(\psk@transTS@originT){\pnode(#1){#2}}%
\pst@getcoor{#1}\pst@tempA%
\pst@getcoor{\psk@transTS@LowP}\pst@tempB
\rput(\psk@transTS@originT){\pnode(! %
\pst@tempA \tx@UserCoor /y1 ED /x1 ED
\ifPst@LowPoint
\pst@tempB \tx@UserCoor /y2 ED /x2 ED
\else
/x2 \psk@transTS@base def \fi
/x3 \psk@transTS@symline def
\ifPst@symX
/lang x2 x1 sub x1 x3 sub 2 mul add abs \psk@transTS@vkf mul def
\else
/lang x2 x1 sub abs \psk@transTS@vkf mul def \fi
/lang2 x2 x1 sub def
x2 x1 ge {/alph2 \psk@transTS@phi \psk@transTS@deltaphi add def} {/alph2 \psk@transTS@phi \psk@transTS@deltaphi add 180 sub def} ifelse
lang alph2 PtoC exch x2 add exch y1 add){#3}
\ifPst@translineA
\ifPst@LowPoint
\pcline[linecolor=\psk@transTS@transAcolor,linestyle=\psk@transTS@transAlinestyle,linewidth=\psk@transTS@transAlinewidth](! \pst@tempB \tx@UserCoor pop \pst@tempA \tx@UserCoor exch pop)(#2)
\else
\pcline[linecolor=\psk@transTS@transAcolor,linestyle=\psk@transTS@transAlinestyle,linewidth=\psk@transTS@transAlinewidth](! \psk@transTS@base \pst@tempA \tx@UserCoor exch pop)(#2)
\fi
\else \fi%
\ifPst@translineB
\ifPst@LowPoint
\pcline[linecolor=\psk@transTS@transBcolor,linestyle=\psk@transTS@transBlinestyle,linewidth=\psk@transTS@transBlinewidth](! \pst@tempB \tx@UserCoor pop \pst@tempA \tx@UserCoor exch pop)(#3)
\else
\pcline[linecolor=\psk@transTS@transBcolor,linestyle=\psk@transTS@transBlinestyle,linewidth=\psk@transTS@transBlinewidth](! \psk@transTS@base \pst@tempA \tx@UserCoor exch pop )(#3)
\fi
\else \fi%
}%
\endgroup
\ignorespaces
}
%------------------------------------------------------------------
\def\pstransTSK{\pst@object{pstpstransTSK}}%
\def\pstpstransTSK@i(#1)#2#3#4{%
\begingroup%
% (#1) Koordinaten des Punktes (kartesisch oder Polarkoordinaten)
%  #2  Originall\"{a}nge #3  Node-Name des Startpunktes %  #4   Node-Name des transformierten Punktes
\pst@killglue
\use@par
\rput(#1){\pnode(0,0){#3}}
\rput(#1){\pnode(! %
/l1 #2 def
/l2 l1 \psk@transTS@vkf mul def
l2 \psk@transTS@phi PtoC){#4}}
\ifPst@translineK
\pcline[linecolor=\psk@transTS@transcolor,linestyle=\psk@transTS@translinestyle,linewidth=\psk@transTS@translinewidth](#3)(#4)
\else \fi%
\endgroup
\ignorespaces
}
%--------------------------------------------------------------------------------
%--------------------------------------------------------------------------------
\def\psboxTS{\pst@object{pstpsboxTS}}%
\def\pstpsboxTS@i(#1,#2,#3)#4#5#6#7{%
\begingroup%
% (#1) Koordinaten des Punktes (kartesisch oder Polarkoordinaten)
%  #2  Originall\"{a}nge #3  Node-Name des Startpunktes %  #4   Node-Name des transformierten Punktes
\pst@killglue
\use@par
%-------------------------------------------------
{\ifPst@differentcol
\psset{fillstyle=solid,fillcolor=#7}%
\else
\psset{fillstyle=solid,fillcolor=#7!60}%
\fi
\pspolygon%
(!/l1 #1 #4 add def
/l2 l1 \psk@transTS@vkf mul def
l2 \psk@transTS@phi 180 add PtoC #3 add exch #2 add exch)%
%-------------------------------------------------
(!/l1 #1 #4 add def
/l2 l1 \psk@transTS@vkf mul def
l2 \psk@transTS@phi 180 add PtoC #3 add exch #2 #5 add add exch)%
%-------------------------------------------------
(!/l1 #1 #4 add def
/l2 l1 \psk@transTS@vkf mul def
l2 \psk@transTS@phi 180 add PtoC #3 #6 add add exch #2 #5 add add exch)%
%-------------------------------------------------
(!/l1 #1 #4 add def
/l2 l1 \psk@transTS@vkf mul def
l2 \psk@transTS@phi 180 add PtoC #3 #6 add add exch #2 add exch)%
}%
%-------------------------------------------------
{\ifPst@differentcol
\psset{fillstyle=solid,fillcolor=\psk@transTS@facecolorR}%
\else
\psset{fillstyle=solid,fillcolor=#7!90}%
\fi
\pspolygon%
(!/l1 #1 #4 add def
/l2 l1 \psk@transTS@vkf mul def
l2 \psk@transTS@phi 180 add PtoC #3 add exch #2 #5 add add exch)%
%-------------------------------------------------
(!/l1 #1 def
/l2 l1 \psk@transTS@vkf mul def
l2 \psk@transTS@phi 180 add PtoC #3 add exch #2 #5 add add exch)%
%-------------------------------------------------
(!/l1 #1 def
/l2 l1 \psk@transTS@vkf mul def
l2 \psk@transTS@phi 180 add PtoC #3 #6 add add exch #2 #5 add add exch)%
%-------------------------------------------------
(!/l1 #1 #4 add def
/l2 l1 \psk@transTS@vkf mul def
l2 \psk@transTS@phi 180 add PtoC #3 #6 add add exch #2 #5 add add exch)%
}%
%%-------------------------------------------------
{\ifPst@differentcol
\psset{fillstyle=solid,fillcolor=\psk@transTS@facecolorT}%
\else
\psset{fillstyle=solid,fillcolor=#7!40}%
\fi
\pspolygon%
(!/l1 #1 #4 add def
/l2 l1 \psk@transTS@vkf mul def
l2 \psk@transTS@phi 180 add PtoC #3 #6 add add exch #2 #5 add add exch)%
%-------------------------------------------------
(!/l1 #1 def
/l2 l1 \psk@transTS@vkf mul def
l2 \psk@transTS@phi 180 add PtoC #3 #6 add add exch #2 #5 add add exch)%
%-------------------------------------------------
(!/l1 #1 def
/l2 l1 \psk@transTS@vkf mul def
l2 \psk@transTS@phi 180 add PtoC #3 #6 add add exch #2 add exch)%
%-------------------------------------------------
(!/l1 #1 #4 add def
/l2 l1 \psk@transTS@vkf mul def
l2 \psk@transTS@phi 180 add PtoC #3 #6 add add exch #2 add exch)%
}%
%-------------------------------------------------
\ifPst@hideline
\pcline[nodesep=1.1pt,linecolor=\psk@transTS@hidecolor,linestyle=\psk@transTS@hidelinestyle,linewidth=\psk@transTS@hidelinewidth]%
(!/l1 #1 def
/l2 l1 \psk@transTS@vkf mul def
l2 \psk@transTS@phi 180 add PtoC #3 add exch #2 add exch)%
(!/l1 #1 #4 add def
/l2 l1 \psk@transTS@vkf mul def
l2 \psk@transTS@phi 180 add PtoC #3 add exch #2 add exch)%
%-------------------------------------------------
\pcline[nodesep=1.1pt,linecolor=\psk@transTS@hidecolor,linestyle=\psk@transTS@hidelinestyle,linewidth=\psk@transTS@hidelinewidth]%
(!/l1 #1 def
/l2 l1 \psk@transTS@vkf mul def
l2 \psk@transTS@phi 180 add PtoC #3 add exch #2 add exch)%
(!/l1 #1 def
/l2 l1 \psk@transTS@vkf mul def
l2 \psk@transTS@phi 180 add PtoC #3 #6 add add exch #2 add exch)%
%-------------------------------------------------
\pcline[nodesep=1.1pt,linecolor=\psk@transTS@hidecolor,linestyle=\psk@transTS@hidelinestyle,linewidth=\psk@transTS@hidelinewidth]%
(!/l1 #1 def
/l2 l1 \psk@transTS@vkf mul def
l2 \psk@transTS@phi 180 add PtoC #3 add exch #2 add exch)%
(!/l1 #1 def
/l2 l1 \psk@transTS@vkf mul def
l2 \psk@transTS@phi 180 add PtoC #3 add exch #2 #5 add add exch)%
\else \fi%
\endgroup
\ignorespaces
}
%------------------------------------------------------------------
\def\psCircleTS{\pst@object{pstpsCircleTS}}%
\def\pstpsCircleTS@i#1{%
\begingroup%
\pst@killglue
\use@par
\def\punkte{}
\multido{\i=0+3}{120}{%
\pstransTS(#1;\i){AnfPktC\i}{EndPktC'\i}
\xdef\punkte{\punkte(EndPktC'\i)}}
\expandafter\psccurve\punkte
\endgroup
\ignorespaces
}
%--------------------------------------------------------------------------------
%------------------------------------------------------------------
\def\psCircleTSX{\pst@object{pstpsCircleTSX}}%
\def\pstpsCircleTSX@i#1{%
\begingroup%
\pst@killglue
\use@par
\def\punkte{}
\multido{\i=0+3}{120}{%
\pstransTSX(#1;\i){AnfPktC\i}{EndPktC'\i}
\xdef\punkte{\punkte(EndPktC'\i)}}
\expandafter\psccurve\punkte
\endgroup
\ignorespaces
}
%--------------------------------------------------------------------------------
%------------------------------------------------------------------
\def\psArcTS{\pst@object{pstpsArcTS}}%
\def\pstpsArcTS@i#1#2#3{%
\begingroup%
\pst@killglue
\use@par
\pst@Verb{%
/v \psk@transTS@vkf def%
/ph \psk@transTS@phi def%
/t 1 v 2 exp sub v 4 exp 2 v 2 exp mul 2 ph mul cos mul add 1 add sqrt div def%
/ws 2 sqrt 2 div 1 t sub sqrt mul ASIN 180 mul Pi div def%
}
\def\punkte{}
\multido{\r=0+1}{#3}{%
\pstransTS(! #1 ws #2 add \r\space add PtoC){AnfPkt}{EndPkt\r}
\xdef\punkte{\punkte(EndPkt\r)}}
\expandafter\pscurve\punkte
\endgroup
\ignorespaces
}
%--------------------------------------------------------------------------------
%------------------------------------------------------------------
\def\psArcTSX{\pst@object{pstpsArcTSX}}%
\def\pstpsArcTSX@i#1#2#3{%
\begingroup%
\pst@killglue
\use@par
\pst@Verb{%
/v \psk@transTS@vkf def%
/ph \psk@transTS@phi def%
/t 1 v 2 exp sub v 4 exp 2 v 2 exp mul 2 ph mul cos mul add 1 add sqrt div def%
/ws 2 sqrt 2 div 1 t sub sqrt mul ASIN 180 mul Pi div def%
}
\def\punkte{}
\multido{\r=0+1}{#3}{%
\pstransTSX(! #1 ws 90 add #2 add \r\space add PtoC){AnfPkt}{EndPkt\r}
\xdef\punkte{\punkte(EndPkt\r)}}
\expandafter\pscurve\punkte
\endgroup
\ignorespaces
}
%--------------------------------------------------------------------------------
%------------------------------------------------------------------
\def\psZylinderTS{\pst@object{pstpsZylinderTS}}%
\def\pstpsZylinderTS@i#1#2{%
\begingroup%
\pst@killglue
\use@par
\pst@Verb{%
/v \psk@transTS@vkf def%
/ph \psk@transTS@phi def%
/t 1 v 2 exp sub v 4 exp 2 v 2 exp mul 2 ph mul cos mul add 1 add sqrt div def%
/ws 2 sqrt 2 div 1 t sub sqrt mul ASIN 180 mul Pi div def%
/Ra 0.5 1 t add mul 1 v 2 exp sub mul v 2 exp add v ph cos mul 1 t 2 exp sub sqrt mul add sqrt def%
/Rb 0.5 1 t sub mul 1 v 2 exp sub mul v 2 exp add v ph cos mul 1 t 2 exp sub sqrt mul sub sqrt def%
/we 2 sqrt 2 div 1 t sub sqrt mul v mul ph sin mul Ra div ASIN 180 mul Pi div def%
}
\def\punkteA{}
\def\punkteB{}
\def\punkteC{}
\def\punkteD{}
\multido{\r=0+1}{180}{%
\pstransTS[originT={0,0}](! #1 ph 0 ge {ws 180 add} {ws} ifelse \r\space add PtoC){ZylA\r}{ZylA'\r}%
\xdef\punkteA{\punkteA(ZylA'\r)}}%
\multido{\rB=0+1}{180}{%
\pstransTS[originT={0,#2}](! #1 ph 0 ge {ws} {ws 180 add} ifelse \rB\space add PtoC){ZylB\rB}{ZylB'\rB}%
\xdef\punkteB{\punkteB(ZylB'\rB)}}%
\multido{\rC=0+1}{180}{%
\pstransTS[originT={0,0}](! #1 ws \rC\space add PtoC){ZylC\rC}{ZylC'\rC}%
\xdef\punkteC{\punkteC(ZylC'\rC)}}%
\multido{\rD=0+1}{360}{%
\pstransTS[originT={0,#2}](#1;\rD){ZylD\rD}{ZylD'\rD}%
\xdef\punkteD{\punkteD(ZylD'\rD)}}%
%-----------------------------------------------
\pscustom{%
\expandafter\pscurve\punkteA%
\rlineto(0,#2)
\expandafter\pscurve\punkteB%
\closepath%
}%
%-----------------------------------------------
\pscustom[fillstyle=\psk@transTS@topfillstyle,gradangle=\psk@transTS@topangle,gradmidpoint=\psk@transTS@topmidpoint,linecolor=\psk@transTS@toplinecolor,%
linewidth=\psk@transTS@toplinewidth,fillcolor=\psk@transTS@topfillcolor]{%
\expandafter\psccurve\punkteD%
\closepath
}%
%\psrotate(0,#2){! we}{%
%\psellipse[fillstyle=\psk@transTS@topfillstyle,gradangle=\psk@transTS@topangle,gradmidpoint=\psk@transTS@topmidpoint,linecolor=\psk@transTS@toplinecolor,%
%linewidth=\psk@transTS@toplinewidth,fillcolor=\psk@transTS@topfillcolor](0,#2)(!Ra #1 mul Rb #1 mul)}%
\ifPst@hideline
\psset{linecolor=\psk@transTS@hidecolor,linestyle=\psk@transTS@hidelinestyle,linewidth=\psk@transTS@hidelinewidth}%
\expandafter\pscurve\punkteC%
\else \fi%
\endgroup
\ignorespaces
}
%--------------------------------------------------------------------------------
\catcode`\@=\TheAtCode\relax
\endinput
%%
%% END pst-perspective.tex 